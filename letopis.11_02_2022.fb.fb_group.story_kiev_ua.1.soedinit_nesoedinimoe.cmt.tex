% vim: keymap=russian-jcukenwin
%%beginhead 
 
%%file 11_02_2022.fb.fb_group.story_kiev_ua.1.soedinit_nesoedinimoe.cmt
%%parent 11_02_2022.fb.fb_group.story_kiev_ua.1.soedinit_nesoedinimoe
 
%%url 
 
%%author_id 
%%date 
 
%%tags 
%%title 
 
%%endhead 
\zzSecCmt

\begin{itemize} % {
\iusr{Андрей Надиевец}
Когда то эти пирожки нам раздавали бесплатно...

\begin{itemize} % {
\iusr{Татьяна Соловьева}
\textbf{Андрей Надиевец} Что-то не помню такого коммунизма)

\iusr{Андрей Надиевец}
\textbf{Татьяна Соловьева} 

У меня там работала знакомая, Оксана Соловьева, когда мы с моей тогда ещё
молодой командой проходили мимо, она нам раздавала и не брала деньги Я ее любил
с марта по декабрь 1992 года.


\iusr{Татьяна Соловьева}
\textbf{Андрей Надиевец} О, это другое дело)

\iusr{Оля Лемишко}
\textbf{Андрей Надиевец} За що її і з тріском вигнали.

\iusr{Андрей Надиевец}
\textbf{Оля Лемишко} 

Выгнать с квартиры можно тараканов... Она вышла замуж и уехала в Филадельфию.
Потом о ней сняли клип \enquote{Осень в Филаделфии}, поет один известный певец, звонит
мне по субботам, мы иногда плачем.


\iusr{Оля Лемишко}
\textbf{Андрей Надиевец} як романтично, аж завидки беруть!

\iusr{Андрей Надиевец}
\textbf{Оля Лемишко} Ви мені теж подобаєтесь..

\iusr{Елена Беляева}
\textbf{Андрей Надиевец} Ахаха! Дамский угодник!!!

\iusr{Андрей Надиевец}
\textbf{Елена Беляева} Вы тоже хороши, Елена
\end{itemize} % }

\iusr{Pavlo Kyjeslav Bliznichenko}

От чесно, Київська перепічка фігня рідкісна) пиріжки з Ярослави значно кращі)

\begin{itemize} % {
\iusr{Татьяна Соловьева}
\textbf{Pavlo Kyjeslav Bliznichenko} Ярослава на ЯрВалу - неперевершена!

\iusr{Ольга Кизуб}
\textbf{Pavlo Kyjeslav Bliznichenko} ДА!!!!!

\iusr{Марина Иващенко}
\textbf{Pavlo Kyjeslav Bliznichenko} а тут не про смак, тут про традиції;-).. це як - саме смачне з дитинства, так про Київську перепічку).

\iusr{Pavlo Kyjeslav Bliznichenko}
\textbf{Марина Иващенко} Ярослава ще більші традиції)

\iusr{Ludmila Kilimnik}
\textbf{Марина Иващенко} в дитинстві шлунок ще не страждав від такої їжі!! Справді спогади гарні.... @igg{fbicon.face.zany}{repeat=3} 
\end{itemize} % }

\iusr{Василь Кульбеда}
а где там по левой книжный ??

\begin{itemize} % {
\iusr{Татьяна Соловьева}
\textbf{Василь Кульбеда} Вот сразу за перепичкой, а разве на фотографии не видно?

\iusr{Василь Кульбеда}
\textbf{Татьяна Соловьева} и давно ???

\iusr{Татьяна Соловьева}
\textbf{Василь Кульбеда} Давно, названия только меняет...

\iusr{Василь Кульбеда}

Это следствие того, что центр Киева, почти весь центр, уже не входит в
перечень моих любимых мест....изгажен, сожжен, застроен х-нёй всякой. А
книжный на Крещатике, направо если к Бессарабке уже кто-то проглотил ?? Ещё ж
недавно в нем покпал книги

\begin{itemize} % {
\iusr{Татьяна Соловьева}
\textbf{Василь Кульбеда} Да, того книжного уже нет, к сожалению(

\iusr{Василь Кульбеда}
\textbf{Татьяна Соловьева} печально как -....много книг в моей библиотеке именно оттуда

\iusr{Ludmila Kilimnik}
\textbf{Татьяна Соловьева} краще більше б книжкових крамниць!! А не банків та магазинів з лахміттям (та не дешевим) @igg{fbicon.face.sleepy}{repeat=3} 
\end{itemize} % }

\end{itemize} % }

\iusr{Татьяна Жалнина}

Я сейчас две бумажные книги читаю: 1984 Оруэлл, Псы господние Пикуль, в Ютюбе
нашла аудиокниги, тоже иногда слушаю, жаль, глаза не те

\iusr{Игорь Ивченко}
Воспоминания нахлынули

\iusr{Игорь Ивченко}
и полетели

\iusr{Валентина Бабченко}
Вот прямо почувствовала такой забытый запах новых книг - книжного магазина.

\iusr{Татьяна Соловьева}
\textbf{Валентина Бабченко} @igg{fbicon.heart.red}{repeat=3}!

\iusr{Елена Беляева}
Как СМАЧНО изложено!!!

\iusr{Maxim Kaminsky}
Я прямо напротив я работал после института, Ленина 4

\iusr{Татьяна Соловьева}
\textbf{Maxim Kaminsky} УкрНИИпроект?

\iusr{Irina Popova}

спасибо за вкусную прогулку, а в книжный захожу крайне редко и то только, чтобы
его прочуствовать и посмотреть новинки, вдохнуть любимый запах книги новой, а
вот купить мне уже дороговато, перестаю быть современником

\iusr{Marina Kovalenko}
Не бачу проблеми купити книгу

\iusr{Григорий Московцев}

запах книжного магазина.... это вроде как погружаешься в детство... да и сейчас
я читаю ТОЛЬКО БУМАЖНЫЕ КНИГИ... @igg{fbicon.face.smiling.sunglasses} 

\begin{itemize} % {
\iusr{Татьяна Соловьева}
\textbf{Григорий Московцев} Да, как и запах библиотеки)

\iusr{Григорий Московцев}
\textbf{Татьяна Соловьева} ....а про библиотеки я вообще молчу... ЭТО ХРАМЫ ЗНАНИЙ @igg{fbicon.face.nerd}  @igg{fbicon.face.wink.tongue} 
\end{itemize} % }

\iusr{Феликс Ушеренко}

Дойдешь потом до угла Пушкинской, повернешь налево и через пару минут будешь у
нашего быашего дома на Пушкинской 23б. Кому это понадобилось нас оттуда выжить?


\iusr{Ирина Улюкаева}

Только сначало (где-то 1989 г...) перепичка покупалась не только в окошке, а и
внутри кафе. Это кафе было достаточно большим, но грязноватым. Там можно было
купить ещё какую-то выпечку и жаренных цыплят. В начале - середине 90 - х. в
кафе часто давал о себе знать дефицит продуктов. Перепички очень быстро
заканчивались - не было то сосисок, то ещё чего-то. И территория его быстро
сокращалась.

Но с этим местом много приятных воспоминаний и переживаний. Когда бываю в Киеве
стараюсь на несколько минут вернуться в прошлое. Закрою глаза и ...вкус
перепички возвращает во времена ...

\begin{itemize} % {
\iusr{Татьяна Соловьева}
\textbf{Ирина Улюкаева} Там и сейчас рядом кафе)

\ifcmt
  ig https://scontent-mxp1-1.xx.fbcdn.net/v/t39.30808-6/273288489_1947189985453855_8148992567844188745_n.jpg?_nc_cat=107&ccb=1-5&_nc_sid=dbeb18&_nc_ohc=jGKbIpp0SxYAX872NBG&_nc_ht=scontent-mxp1-1.xx&oh=00_AT8-xeE12d8bMlXyw4cAboUX_r6ei57gkdXN8GbPFrNgmw&oe=620C4BE0
  @width 0.2
\fi

\iusr{Татьяна Соловьева}
А девушка рядом ест перепичку)

\end{itemize} % }

\iusr{Людмила Волошенко}

Щось так і лягли на душу електронні книжки. А яке це хвилюючо-трепетне
передчуття, коли буреш до рук нову, ще не прочитану книгу. Читаємо, друзі!
Скільки зараз книг! За моєї молодості, в Союзі, гарну книгу можна було
«дістати» тільки по блату, або з великою «нагрузкою» з матеріалів Політбюро або
виступів Брежнєва. Але все одно, як би складно і сутужно не жилось, на книги
завжди знаходила кошти.

\iusr{Тома Храповицкая}

Спасибо тебе огромное за Книги! Перелистывание страниц, Запах типографии, есть
ли картинки, как в детстве, Обложка, Супер-обложка, если есть, Форзац,
Коптал... и Начинается Волшебство... с детства, с самого раннего Уважение и
Преклонение...

\begin{itemize} % {
\iusr{Татьяна Соловьева}
\textbf{Тома Храповицкая} Томочка @igg{fbicon.heart.red} ) Как хорошо, когда есть свои! Спасибо тебе!

\iusr{Тома Храповицкая}
\textbf{Татьяна Соловьева} Есть, благодаря Папе! Он Привил Любовь к Книгам!!!

\iusr{Ирина Гутт}
\textbf{Татьяна Соловьева} Вот именно - свои. Со времён...
\end{itemize} % }

\iusr{Valentina Zajfertová}
Это Крещатик 12?

\iusr{Valentina Zajfertová}
Если да, то в 1976 году мой отдел на первом этаже.

\begin{itemize} % {
\iusr{Татьяна Соловьева}
\textbf{Valentina Zajfertová} Если что, да?

\iusr{Valentina Zajfertová}
\textbf{Татьяна Соловьева} если Крещатик 12

\iusr{Татьяна Соловьева}
\textbf{Valentina Zajfertová} А, теперь понятно. Но это нечётная сторона улицы)
\end{itemize} % }

\iusr{Volodymyr Nekrasov}
Київська перепічка з томатним соком - смак знайомий з дитинства  @igg{fbicon.face.savoring.food} 

\iusr{Alex Goldenberg}
Очень интересно. Спасибо.

\iusr{Петро Гарматюк}
Від квартири до \enquote{Перепічки} п'ять хвилин пішки. Зимою в трусах можна бігати.
Все так, все дуже смачно, свіжо, але було це давненько.

\iusr{Светлана Манилова}

Каждый приезд на Крещатик обязательно завершается покупкой перепички. Это уже
многолетний ритуал. Ем её обязательно на ходу, проходя мимо ЦУМа и двигаясь
дальше в сторону конечной 18 и 16 троллейбусов. Завершаю этот процесс всегда в
одном и том же месте, т.е. ем с одинаковой скоростью.  @igg{fbicon.smile}
Спасибо, Татьяна, за приятные и вкусные воспоминания!

\iusr{Мария Маслова}

Прекрасная прогулка, Танюша! Как-то мы с Серёжкой ели перепичку (вот только не
помню, с томатным соком или с кофе) в конце восьмидесятых. Сережа был в новом
пальто и смачно так откусил перепичку, а из нее брызнул жир... Все пальто,
естественно...  @igg{fbicon.face.grinning.squinting}  А из названий книжного помню только \enquote{Читай-город}, покупала
там мужу на подарок ручку в коробочке, на ушке ручки было написано: \enquote{I love
you}... Эх, в книжный захотелось...  @igg{fbicon.heart.sparkling} 

\iusr{Leonid Dukhovny}

Чудесно! Но есть и печальная сторона.... В начале лихих девяностых этот книжный
магазинчик открыл киевский архивист авторской песни Вячеслав Белокриницкий. Он
первым оформил своё владение в Горсовете. Криминалу это бойкое место тоже
сильно приглянулось и \enquote{ребятки} стали \enquote{убеждать} Славу убраться или продать им
по бросовой цене. Но у него были совсем другие планы - он хотел при магазине
открыть музей истории киевской АП, издавать сброники и антологии бардов родного
города... Он был талантливый, честный, упорный и справедливый... Его изощрённо
убили... Светая и долгая память Белокриницкому Вячеславу Наумовичу!

\iusr{Татьяна Соловьева}
\textbf{Leonid Dukhovny} 

Какая трагедия. Хорошо, что вы написали об этом. Многие не знали. Светлая
память Вячеславу Наумовичу Белокриницкому

\iusr{Татьяна Сирота}
Танюша, прямо почувствовала запах свежей Перепічкі. @igg{fbicon.face.upside.down} 
И, конечно, запах новых КНИГ...

\iusr{Татьяна Соловьева}
\textbf{Татьяна Сирота} 

Вот так, Танечка, и соединилось несоединимое  @igg{fbicon.face.blowing.kiss}
@igg{fbicon.heart.red})

\iusr{Валентина Белозуб}

Дякую за цікаву розповідь!!! Улюблена перепічка та чудовий Книжковий магазин!
Люблю заходити в книжковий ! Незважаючи на непосильні для багатьох ціни на
книги, зайти та подивитися теж задоволення!

\iusr{Татьяна Соловьева}
\textbf{Валентина Белозуб} Всегда захожу, даже если и не собираюсь что-то купить!

\end{itemize} % }
