% vim: keymap=russian-jcukenwin
%%beginhead 
 
%%file 16_04_2021.fb.backandalive.1.sever
%%parent 16_04_2021
 
%%url https://www.facebook.com/backandalive/posts/1983943715096389
 
%%author 
%%author_id 
%%author_url 
 
%%tags 
%%title 
 
%%endhead 
\subsection{Снаряд не влучає у одну вирву двічі}
\Purl{https://www.facebook.com/backandalive/posts/1983943715096389}

\ifcmt
	pic https://scontent-mxp1-2.xx.fbcdn.net/v/t1.6435-9/174561110_1983942708429823_6891852919911150770_n.jpg?_nc_cat=109&ccb=1-3&_nc_sid=730e14&_nc_ohc=ov6GZfZY3RwAX-Wure3&_nc_ht=scontent-mxp1-2.xx&oh=7fdd63563b7d71b53e608c087afccc95&oe=60A2874F

	pic https://scontent-mxp1-2.xx.fbcdn.net/v/t1.6435-9/174642931_1983942718429822_6771515320374567024_n.jpg?_nc_cat=106&ccb=1-3&_nc_sid=730e14&_nc_ohc=gOozPnubOTgAX99qVMX&_nc_ht=scontent-mxp1-2.xx&oh=e9def3c31f64d54818b9224390744cb9&oe=60A17BF1

	pic https://scontent-mxp1-2.xx.fbcdn.net/v/t1.6435-9/174427457_1983942721763155_4265865697800776363_n.jpg?_nc_cat=102&ccb=1-3&_nc_sid=730e14&_nc_ohc=zVUeXeEuMBEAX-9LrRw&_nc_ht=scontent-mxp1-2.xx&oh=ad3728bdc6055838b0c09c29a5fa2c61&oe=60A01206
\fi

«Сєвєр» на власні очі бачив спростування міту, нібито снаряд не влучає у одну вирву двічі

До подій 2014 року Костянтин був таким собі звичайним студентом: здобував фах
інженера-біотехнолога в Національному технічному університеті в Харкові, за
можливості підробляв, а у вільний час насолоджувався життям. А ще планував
після завершення навчання вирушити на озеро Байкал у Російську Федерацію, де на
одній з баз відпочинку на нього чекали посада фахівця з обслуговування
апаратури й висока зарплатня. Проте життя змінилося кардинально, коли до юнака
почали долітати перші тривожні звістки про «зелених чоловічків» у Криму.

«Тоді я ніби прокинувся й подивився на світ, де росіяни та українці начебто
брати, іншими очима», — зазначає Костянтин. Тож сьогодні молодший сержант
«Сєвєр» захищає суверенітет і незалежність Батьківщини в 36 окрема бригада
морської піхоти імені контр-адмірала Михайла Білинського .

«Свій позивний отримав ще коли служив у 93-тя ОМБр Холодний Яр

. З одного боку через те, що народився на півночі Росії, у мене мати звідти
родом. А з іншого — на честь загиблого героя з «Правого сектору» Сергія Табали
«Сєвєра». Мене дуже вразила його історія, захотілося в чомусь продовжити
життєвий шлях «кіборга», взявши його позивний», - розповідає «Сєвєр».

До 93-ї МПБр Костянтин потрапив не відразу, адже у 2014-му скептично ставився
до вітчизняного війська. Тому у військкомат не пішов, а почав шукати людей у
рідному Бердянську, які готові були самоорганізуватися та захищати місто.
Хлопець вступив у роту самооборони, у складі якої до 2015-го чергував на
блок-постах на в’їзді-виїзді до Бердянська.

«Коли наш підрозділ підпорядкувався МВС, зрозумів, що мене це не влаштовує. Тож
слідом за побратимом перейшов до роти «Вітра» 93-ї бригади на посаду
розвідника», - пригадує морпіх.

Майже півтора року ротацій з 93-ю бригадою: Авдіївка, шахта «Бутівка», Водяне,
Новотошківське. Як пригадує «Сєвєр», ситуації були різні. Одного разу розвідник
разом з побратимами виявив трьох вербувальників, які переконували українських
вояків іти воювати за так зване «ДНР».

«Це ще було на початку моєї служби у бригаді. Тоді ще з матеріальним
забезпеченням було сутужно, тож ворог намагався зіграти на цьому. Якось у
розмові два чоловіки та жінка з місцевих почали розказувати, які зарплати на
«тому боці». Ми їх взяли та передали СБУ», - ділиться спогадами воїн.

Побачив «Сєвєр» на власні очі і спростування міту про те, що снаряд нібито не
влучає в одну вирву двічі й те, як пів кроку відділяє від смерті.


«Після штурму селища, який проводив підрозділ бригади, проросійські найманці
три дні не хотіли віддавати тіло бійця з позивним «Мольфар». Коли вже
домовилися, то виявилось, що залишилася лише нижня частина нашого побратима: у
ту саму вирву, де чоловік лежав уже мертвий, влучила ще одна міна. А під час
самого штурму наша група з восьми розвідників виносила пораненого. І буквально
за мить — обертаюсь, а двоє хлопців уже позаду, один боєць перемотує іншого,
якому відірвало стопу. Буквально пів кроку вбік на стежці, по якій ми до того
пів дня носили боєприпаси, і — інвалідність», - пригадує захисник України.

Після виходу бригади в ППД Костянтин прийняв рішення залишити військо. Тоді
дружина прямо сказала: або він вертається додому, або вони розлучаються.

«Меншій доньці тоді був рік, старшій — п’ять. Жінці не вистачало допомоги, а
дівчаткам моєї уваги. Тоді задумався: я ж на війну пішов, аби насамперед
захистити рідних. А якщо це все зруйнується… Для мене найголовніше в житті — це
родина. Тож прийняв рішення звільнитися», - розповідає «Сєвєр».

За два роки цивільного життя Костянтин подумки завжди був на війні. Він постійно спілкувався з друзями, які служили, стежив за новинами з передової та з бригади.

«Не зміг пристосуватися до мирного життя. Вирішив прийняти компромісне рішення
— піти служити в частину, яка дислокується неподалік дому. На моє глибоке
переконання, морська піхота — це елітний підрозділ, у якому служать
високомотивовані професійні військові. Суворі вимоги до кожного, незалежно від
звання чи посади, спонукають постійно вдосконалюватись, водночас служба тут
підіймає самооцінку. Окрім того, поза ротацією більшість часу можу проводити з
родиною», - говорить Костянтин.

Сьогодні «Сєвєр» захищає країну на маріупольському напрямку. У ротації отримав
і славнозвісний берет морпіха. До проходження смуги молодший сержант ретельно
готувався, хоча каже відверто, що важче було психологічно, ніж фізично.

«Коли комбриг перед строєм тобі вручає диплом зі статутом морського піхотинця і
надіває берет кольору морської хвилі, ти забуваєш і виснажливі тренування, і
потовчені під час проходження смуги ребра та розбите обличчя. Тебе переповнює
шалене почуття гордості за те, що ти разом з побратимами зміг подолати всі
випробування», - ділиться спогадами морпіх.

А вдома на тата чекають дві чарівні донечки, які неймовірно ним пишаються.

«Сєвєр» та тисячі захисників України щодня стримують російську навалу. Ми разом
з вами вже сім років підтримуємо українське військо й будемо продовжувати
робити це аж до нашої перемоги. Головне — не зупинятися, адже від цього
залежать життя https://savelife.in.ua/donate/

Текст: Оксана Третій, Інформагентство АрміяInform

Фото: 36 окрема бригада морської піхоти імені контр-адмірала Михайла Білинського

Фонд «Повернись живим»
