% vim: keymap=russian-jcukenwin
%%beginhead 
 
%%file 31_01_2022.fb.guzhva_igor.1.mirovaja_fejkovaja_vojna
%%parent 31_01_2022
 
%%url https://www.facebook.com/veprwork/posts/4917235305001974
 
%%author_id guzhva_igor
%%date 
 
%%tags fejk,infvojna,napadenie,rossia,ugroza,ukraina,vojna,vtorzhenie
%%title Мировая фейковая война
 
%%endhead 
 
\subsection{Мировая фейковая война}
\label{sec:31_01_2022.fb.guzhva_igor.1.mirovaja_fejkovaja_vojna}
 
\Purl{https://www.facebook.com/veprwork/posts/4917235305001974}
\ifcmt
 author_begin
   author_id guzhva_igor
 author_end
\fi

Коллеги в последние дни часто меня спрашивают - ну как могут ведущие западные
СМИ так стрелять фейками по поводу \enquote{подготовки российского вторжения в
Украину}? При том, что уже и украинские власти чуть ли не каждый день такую
подготовку опровергают.

\enquote{Ну как можно? Ведь это CNN, Reuters, Bloomberg... Как они могут?}, -
недоумевают коллеги.

А на самом деле - чему удивляться? 

Ведь могли же \enquote{ведущие западные СМИ} четыре года подряд рассказывать со ссылкой
на \enquote{источники}, что Трамп - русский шпион. При том, что ни одного
доказательства так и не появилось. Но четыре года все писали, что \enquote{Трамп -
русский шпион}.

А сейчас все пишут, что \enquote{Путин вот вот нападет на Украину}.

На \enquote{русском шпионе Трампе} была отработана схема создания полностью виртуальной
информационной реальности.

Сейчас просто ее применяют в мировом масштабе.

Историю с Трампом можно назвать \enquote{гражданской фейковой войной}, а то, что
происходит сейчас - \enquote{мировой фейковой войной}.

\enquote{Первой мировой фейковой войной}.

С этого можно было бы просто посмеяться, если б итогом \enquote{гражданской фейковой
войны} не стал вполне реальный и очень глубокий раскол в американском обществе.

Что станет итогом \enquote{мировой фейковой войны} пока можно только догадываться.
