% vim: keymap=russian-jcukenwin
%%beginhead 
 
%%file 09_06_2021.fb.kolomiec_jurij.1.olzhych_oleg
%%parent 09_06_2021
 
%%url https://www.facebook.com/permalink.php?story_fbid=967000874092222&id=100023469540550
 
%%author 
%%author_id kolomiec_jurij
%%author_url 
 
%%tags 1944,istoria,kultura,literatura,oles_oleksandr.ukr.poet,olzhych_oleg.ukr.poet,poezia,ukraina
%%title Олег Ольжич - 77 років - від дня загибелі в 1944 році
 
%%endhead 
 
\subsection{Олег Ольжич - 77 років - від дня загибелі в 1944 році}
\label{sec:09_06_2021.fb.kolomiec_jurij.1.olzhych_oleg}
\Purl{https://www.facebook.com/permalink.php?story_fbid=967000874092222&id=100023469540550}
\ifcmt
 author_begin
   author_id kolomiec_jurij
 author_end
\fi

10 червня минає 77 років від дня загибелі (1944) українського поета,
політичного діяча, археолога, сина відомого українського поета Олександра Олеся
Олега Ольжича (справжнє ім'я – Олег Олександрович Кандиба), (21 липня 1907 р.,
Житомир – концентраційний табір Заксенгаузен).

\ifcmt
  pic https://scontent-lga3-2.xx.fbcdn.net/v/t1.6435-9/198094868_967000584092251_8273869539504580969_n.jpg?_nc_cat=102&ccb=1-3&_nc_sid=730e14&_nc_ohc=Mprto2n8g3sAX8gAAfK&tn=ntrKbsW_7ChXu3v-&_nc_ht=scontent-lga3-2.xx&oh=044de512900e8d71aa3dc0f9f93becaf&oe=60C8E4D5
	caption Олег Ольжич, український поет
\fi

\subsubsection{Життєпис}

Родина в 1912-19 рр. жила заможно та мешкала в дохідному будинку
Файбишенко на Новій будові, вулиці Антоновича, 64/16, точніше на розі вулиць
Антоновича (Горького) та Івана Федорова. Будинок вцілів, однак перебуває в
аварійному стані.

Навчання та наукова діяльність. У 1919-23 рр. здобував середню освіту в
трудовій школі, мешкаючи в Пущі-Водиці поблизу Києва, але закінчити середнє
навчання довелося лише в Празі. Наприкінці 1922 р. він разом із матір'ю виїхав
з України у Берлін, де 3 січня 1923 р. нарешті зустрівся з батьком, який у
лютому 1919 р. був змушений поспіхом емігрувати до Австрії, використавши
придбаний дипломатичний паспорт з повноваженням аташе з культури при
консульстві УНР у Будапешті.

1923 р. родина Кандиби переїхала до Праги. Влітку родина переїхала до Ржевниців
під Прагою, згодом далі – до Горніх Черношинець. Восени 1923 р. Олег разом з
матір'ю повернувся до Праги. Батько залишився в Горніх Черношинцях, де він
організував гурток з науковців, поетів і письменників.

О.Кандиба поступив на платні річні матуральні курси Українського Громадського
Комітету в Празі, які закінчив 11 грудня 1924 р. з відзнакою. Це надавало йому
можливість записатись вільним слухачем на безкоштовній основі у будь-який вищий
навчальний заклад.

Олег записався слухачем на літературно-історичний відділ Українського високого
педагогічного інституту імені Драгоманова у Празі. Цей інститут заснував влітку
1923 р. Український Громадський Комітет і до нього брали всіх охочих. Однак
документи, що підтвердили б його навчання, не збереглися. Одночасно юнак
записався на лекції в Український вільний університет. За непідтвердженими
відомостями, серед засновників був його батько Олександр Олесь (Кандиба).
Документи студента Кандиби не збереглися, однак лишилися спогади його вчителів
і колег, а також Л.Винар віднайшов конспекти О.Ольжича занять з археології, які
проводив професор В.Щербаківський. Тоді ж записався вільним слухачем
філософського факультету Карлового університету в Празі. Влітку 1926 р.,
склавши додаткові іспити з латині, став звичайним студентом університету (роки
навчання – 1924-29). Бувши студентом, 1928 р. отримав грант від Наукового
Товариства імені Шевченка у Львові на експедицію в Більче Золоту.

Член Пласту в Празі. На пластову тематику написав поезію «Пластовий капелюх».
На цей час припали початки наукової діяльності О.Ольжича, яку Л.Винар
періодизує таким чином:

І. 1925-30 рр. – роки навчання та археологічних розкопок у Чехословаччині і Західній Україні.

II. 1931-40 рр. – період інтенсивної наукової і академічно-педагогічної
діяльності О.Кандиби, час появи його основних наукових праць у різномовних
наукових виданнях, його кількарічної співпраці з Гарвардським університетом і
УВУ. Цей період також охоплює заходи з організації Українського Наукового
Інституту в Америці.

III. З 1941 р. до травня 1944 р. – період принагідної наукової роботи. Тоді
О.Кандиба був членом Археологічної Комісії НТШ (1941-44), яку очолював
Я.Пастернак. О.Кандиба перебував у підпіллі, але рівночасно збирав історичні і
фольклорні джерела, співпрацював з українськими вченими в Західній і Східній
Україні.

Восени 1930 р. захистив дисертацію на звання доктора філософії на тему: «Неолітична мальована кераміка Галичини».

З жовтня 1930 р. до травня 1932 р. – асистент кафедри археології УВУ. Працював
у археологічному відділі Національного музею Чехословаччини. У лютому 1931 р.
О.Кандиба одержав від Слов'янського інституту в Празі наукову стипендію в сумі
2 тис. чеських крон на студійну подорож у Німеччину і Австрію для збирання
матеріалів для його дальших студій культури неолітичної кераміки. Проміжні
результати його наукових досліджень надрукували в чеському журналі «Огляд
передісторичний» і в серійному науковому виданні «Пам'ятки археологічні». У
червні 1932 р. провів рятункові розкопки чотирьох кістякових поховань ранньої
бронзової доби в Кобилісах біля Праги. У цей час він активно популяризував
археологію у празьких і львівських масових виданнях.

Навесні 1932 р. в Празі відбувся II Український науковий з'їзд, на якому
працювала підсекція археології та історії мистецтв, де головував
В.Щербаківський. О.Кандиба зробив дві доповіді. Обидві були присвячені
українській мальованій кераміці. Зокрема, археолог подав свою класифікацію
еволюції української мальованої кераміки і її типологію (він визначив шість
основних керамічних форм) і дійшов висновку, що розвиток української мальованої
кераміки є безперервним, органічним і тривалим.

За деякими відомостями, після з'їзду в О.Ольжича виникли серйозні теоретичні
розбіжності з В.Щербаківським та іншими послідовниками школи Любора Нідерле і
він був вимушений звільнитися з кафедри археології УВУ і з-поміж співробітників
археологічного відділу музею. Водночас, у спогадах О.Оглоблин 1956 р. про такий
інцидент не згадав, натомість стверджував, що В.Щербаківський «створив свою
наукову школу, й серед учнів його були визначні археологи, як приміром Др. Олег
Кандиба».

З тих же причин особисту участь у I Міжнародному конгресі доісторичних і
протоісторічних наук, що відбувся того ж року в Лондоні, не брав. Підготував
доповідь «Дністро-Дунайська неолітична область», однак обмежився передачею тез.
Улітку 1932 р. брав участь в американській археологічній експедиції до
Югославії. Восени 1932 р. О.Кандиба узяв участь в археологічних розкопках
печери Доміка на Словаччині, які проводив д-р Й. Бем з чеського Державного
Археологічного Інституту.

Надалі його участь в історичних конференціях звелася до передачі тез своїх
робіт для ознайомлення. Публікував статті, розвиваючи окремі положення своєї
дисертації. Нетривалий час працював у музеях Відня і Берліна. Це була
діяльність самостійного дослідника.

З літа 1932 р., завдяки протекції першого директора Американської школи
доісторичних досліджень (АШДІ) доктора В.Фавкса, який 1932 р. захистив
дисертацію в Празі, О.Кандиба взяв участь у роботі 12-го польового семестру
цієї школи. Там протягом 3-4 тижнів він знайомив американських студентів з
особливостями проведення розкопок і культурою регіонів Європи в епоху неоліту.
Ця робота давала пристойний заробіток (близько 1000 дол.), який давав змогу
скромно прожити до наступного сезону. Відтепер участь у щорічних літніх
експедиціях студентів АШДІ (до 1938 р. включно) стала джерелом його доходів.

У 1938 р. за завданням Проводу ОУН в складі делегації (В.Курманович,
Я.Барановський, Р.Сушко) О.Кандиба виїздив до США. Мета його візиту –
організація Українського наукового інституту в Америці (УНІА). Хоча УНІА так
ніколи створений і не був,  бо засновникам (професор університету в
Міннеаполісі О.Неприцький-Грановський і провідник ОУН О.Кандиба) не вдалося
узгодити необхідні документи, насамперед, статут створюваного університету.
Читання лекцій у Гарвардському університеті є непідтвердженою гіпотезою.
Доказами перебування О.Ольжича у США з метою створення УНІА є його тривале
листування з О.Неприцьким-Грановським та візитна картка О.Кандиби з реквізитами
музею Пібоді. Водночас Л.Винар вважає, що поїздка взагалі не відбулася. У
рамках підготовки збірки наукових праць науковців-українців під егідою
неіснуючого УНІА підготував та опублікував статтю з археології. Збірка вийшла
друком лише у жовтні 1939 р. Стаття до цього збірника стала останньою
опублікованою за життя науковою працею О.Кандиби.

Загалом діапазон наукової діяльності О.Кандиби охоплював, окрім археології, історію, історію мистецтва, етнографію, літературу.

З початку 1930-х рр. О.Кандиба заявив про себе як самобутній і оригінальний
поет. Співпрацював у львівських періодичних виданнях: «Літературно-науковий
вістник», «Вістник», «Обрії», «Напередодні», празьких: «Студентський вісник» і
«Пробоєм».

Діяльність у лавах ОУН. Із молодих літ Олег Кандиба став учасником українського
націоналістичного руху. З 1929 р. – член Організації Українських Націоналістів.
Виконував ряд відповідальних завдань Проводу Українських Націоналістів ОУН,
особисто Є.Коновальця. У 1937 р. очолив культурно-освітню референтуру Проводу
Українських Націоналістів.

Наприкінці 1930-х рр. редагував часопис «Самостійна думка», перетворивши його
на орган ПУН. У 1938-39 рр. брав активну участь у збройній боротьбі проти
угорських окупантів Карпатської України. Його заарештували солдати угорської
армії і він провів три дні у Тячівській в'язниці. Після листа угорських вчених
на захист вченого уряд Угорщини розпорядився випустити О.Кандибу та У.Самчука.
Протягом 1939-41 рр. очолював Революційний Трибунал ОУН, член Проводу
Українських Націоналістів.

На початку радянсько-німецької війни 1941-45 рр. переїхав до Києва разом з
Буковинським куренем, взяв участь у формуванні місцевої адміністрації та
поліції. У 1941-42 рр. О.Кандиба жив у Києві, налагоджував підпільну мережу ОУН
в Україні. В жовтні 1941 р. став одним з організаторів політично-громадського
центру – Української Національної Ради у Києві.

З початком гітлерівських репресій проти українських націоналістів О.Кандиба
переїхав до Львова. У травні 1942 р. Почаївська конференція ОУН обрала його
заступником голови ПУН та головою Проводу на українських землях.

2 серпня 1943 р. у віддаленому селі Яблінка-Вижня біля Турки на Бойківщині
одружився з дочкою літературознавця, професора Л.Білецького Катериною
(Калиною), та їхній шлюб був недовгим. Вже після його смерті народився син
Олег.

У січні 1944 р., після арешту А.Мельника, перебрав посаду Голови ПУН ОУН.

25 травня 1944 р. його заарештувало гестапо у Львові. Є думка, що до цього був
причетний НКВС. На думку сина, Олега Олеговича Кандиби, розгадку арешту та
смерті батька треба шукати насамперед в архівах Москви. 

О.Кандибу ув'язнили у «бараці» Целленбау – окремому блоці для особливо важливих
в'язнів на території концентраційного табору Заксенгаузен. Загинув під час
чергового допиту в ніч проти 10 червня 1944 р., його закатувала гестапівська
трійка (Вольф, Вірзінг, Шульц).

\subsubsection{Вшанування пам'яті}

О.Ольжич справив помітний вплив на літературу української діаспори. О.Стефанович присвятив йому кілька віршів.

Про Олега Ольжича зняли повнометражний двосерійний художньо-документальний
фільм-трилогію за сценарієм Л.Череватенка, режисер А.Микульський «Я камінь з
Божої пращі…», що об'єднує стрічки: «Ольжич» (1996 р.), «Доба жорстока, як
вовчиця», «Незнаний воїн» (2000 р.). Фільму присудили Національну Шевченківську
премію 2002 р.

1994 р. відкрили наукову бібліотеку імені О.Ольжича у м.Києві. 
Тоді ж було встановлено меморіальну дошку О.Ольжичу на будинку №15 по вул. Льва Толстого, де він жив у Києві. 
2007 р. Обласній універсальній науковій бібліотеці в м.Житомирі присвоїли ім'я О.Ольжича. 
На честь О.Ольжича назвали вулиці у Житомирі, Дрогобичі, Золочеві, Львові, Рівному, Заліщиках, Чернівцях, Кропивницькому, Сумах, Миколаєві, Старому Самборі, Новому Бузі та Києві. 
2007 р. НБУ викарбував та ввів до обігу пам'ятну монету на честь О.Ольжича номіналом 2 гривні. 
У 2013 р. відновили анотаційну таблицю на вулиці Ольжича в Житомирі. 
О.Ольжич є Почесним громадянином м.Житомир. 
Пластовий курінь УПЮ в м.Житомир носить ім'я О.Ольжича. 
27 жовтня 2017 р. в Житомирі на розі вулиць Перемоги та Ольжича відкрили пам'ятник О.Ольжичу. 
21 липня 2017 р. на державному рівні в Україні відзначалась пам'ятна дата – 110 років з дня народження О.Ольжича (1907-1944) – письменника, науковця, політичного діяча. 

\subsubsection{Вибрані твори}

Олег Кандиба – автор поетичних збірок «Рінь» (1935), «Вежі»
(1940) та посмертно виданої «Підзамче» (1946), перевиданих у збірках «Поезії»
(1956) та «Величність» (1969), низки праць з археології, серед яких
«Schipeniz-Kunst und Gerete eines neolitisches Dorfes» (1937). У 1941 р. під
псевдонімом Д.Кардаш у Празі видав працю «Євген Коновалець».
