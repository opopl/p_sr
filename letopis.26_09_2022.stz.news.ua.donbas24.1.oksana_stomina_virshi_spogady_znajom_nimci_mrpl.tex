% vim: keymap=russian-jcukenwin
%%beginhead 
 
%%file 26_09_2022.stz.news.ua.donbas24.1.oksana_stomina_virshi_spogady_znajom_nimci_mrpl
%%parent 26_09_2022
 
%%url https://donbas24.news/news/oksana-stomina-cerez-svoyi-virsi-ta-spogadi-znaiomit-nimciv-z-mariupolem
 
%%author_id demidko_olga.mariupol,news.ua.donbas24
%%date 
 
%%tags 
%%title Оксана Стоміна через свої вірші та спогади знайомить німців з Маріуполем
 
%%endhead 
 
\subsection{Оксана Стоміна через свої вірші та спогади знайомить німців з Маріуполем}
\label{sec:26_09_2022.stz.news.ua.donbas24.1.oksana_stomina_virshi_spogady_znajom_nimci_mrpl}
 
\Purl{https://donbas24.news/news/oksana-stomina-cerez-svoyi-virsi-ta-spogadi-znaiomit-nimciv-z-mariupolem}
\ifcmt
 author_begin
   author_id demidko_olga.mariupol,news.ua.donbas24
 author_end
\fi

%\ii{26_09_2022.stz.news.ua.donbas24.1.oksana_stomina_virshi_spogady_znajom_nimci_mrpl.txt}
\ii{26_09_2022.stz.news.ua.donbas24.1.oksana_stomina_virshi_spogady_znajom_nimci_mrpl.pic.front}

\begin{center}
  \em\bfseries\Large
Маріупольчанка через свої вірші розповідає іноземцям про війну
\end{center}

Відома маріупольська поетеса і прозаїк, лауреатка літературної премії ім. Юрія
Каплана, літературної премії \enquote{Слов'янські традиції}, громадська діячка,
засновниця громадської організації \enquote{Паперові сходи} та волонтерка \href{https://www.facebook.com/oksana.stomina}{Оксана
Стоміна}%
\footnote{\url{https://www.facebook.com/oksana.stomina}} %
наразі перебуває в Німеччині. Вона долучилася до декількох проєктів у Берліні,
метою яких є розповісти про війну, про Україну та про її рідний Маріуполь.
Оксана народилася, виросла та створила свою сім'ю саме в Маріуполі.  Вона
прожила з війною 8 років і навіть \enquote{звикла}. Але оголошення війни
Путіним на початку 2022 року змінило життя талановитої і завжди усміхненої
маріупольчанки, як і більшості українців, назавжди.

\begin{leftbar}
\emph{\enquote{Це середньовічна жорстокість, помножена на сучасні можливості та хворі, маніакальні амбіції}}, — наголосила Оксана.
\end{leftbar}

\textbf{Читайте також:} \emph{Маріуполю присвятили поштові листівки з віршами на румейській мові}%
\footnote{Маріуполю присвятили поштові листівки з віршами на румейській мові, Яна Іванова, donbas24.news, 07.07.2022, \par%
\url{https://donbas24.news/news/mariupolyu-prisvyatili-postovi-listivki-z-virsami-na-rumeiskii-movi-foto}%
}

\subsubsection{Про життя Оксани після оголошення повномасштабної війни}

Оксана розповідає за допомогою своїх віршів про те, що відбувається зараз з
кожним українцем, адже вони написані з безпосереднього досвіду та переживань.
Протягом місяця від початку повномасштабного вторгнення рф в Україну Оксана
принципово залишалася в Маріуполі, аби у міру сил допомагати військовим
захищати місто й цивільним — вижити в ньому в оточенні і під щільними
обстрілами. Чоловік поетеси, \textbf{Дмитро Паскалов}, 16 березня наполіг, щоб
вона поїхала з Маріуполя, буквально змусив її сісти в машину. Навіть не
дозволив піднятися до квартири. Тож виїхала Оксана з мінімумом речей, з
малесеньким, на п'ять літрів, наплічником. Наразі Дмитро у російському полоні.
Він один із в'язнів металургійного заводу \enquote{Азовсталь}, був начальником
азовстальського порту та займав дуже активну громадську позицію. Дмитро брав
участь у багатьох загальноміських проєктах у Маріуполі. Маріупольчанка не знає,
що наразі з чоловіком, в яких умовах він знаходиться. Все, що залишається їй та
доньці — це чекати. Про це вона розповідає в одному зі своїх віршів.

\ii{26_09_2022.stz.news.ua.donbas24.1.oksana_stomina_virshi_spogady_znajom_nimci_mrpl.pic.1}

Стоміна, як і багато інших маріупольців, уранці 24 лютого, зібрала невеликий
рюкзак: гроші, папери, мобільний телефон, три ліхтарики, павербанки та кабелі.
На перший погляд вона маленька, тендітна жінка, але її мужність та сила не
можуть не вразити. Одна з її книг називається \textbf{\enquote{Війна приходить без запрошення}.}
Це збірка численних історій простих людей, які розповідають про свій досвід. 

\begin{leftbar}
\emph{\enquote{Мені хочеться розповідати про те, що оці жахливі люди, ці звірі, вони не
просто вбивали: вони наздоганяли та вбивали. Що коли, наприклад,
влучили у театр, або в басейн, або в школу мистецтв, все це були місця
— й всі про це знали, — що там були люди, які вже втратили домівки,
були там і поранені, діти з мамами, були новонароджені немовлята, там
були люди з обмеженими можливостями з районів, що першими потрапили під
обстріли, яких ми рятували, — з Лівого Берега, зі Східного. І потім їх
знову ж наздоганяли і ще раз, і ще раз намагалися вбити. Так само, як
ще раз і ще раз намагалися вбити людей, які намагалися виїхати з
міста}}, — поділилася Оксана.
\end{leftbar}

\textbf{Читайте також:} \emph{Героям маріупольського гарнізону присвятили нову пісню}%
\footnote{Героям маріупольського гарнізону присвятили нову пісню, Яна Іванова, donbas24.news, 14.06.2022, \par%
\url{https://donbas24.news/news/geroyam-mariupolskogo-garnizonu-prisvyatili-novu-pisnyu-video}%
}

\subsubsection{Про проєкт у Німеччині}

У Берліні Оксана розпочала проєкт спільно з режисеркою \href{https://www.facebook.com/profile.php?id=100002359931543}{Анною Жуковець},%
\footnote{\url{https://www.facebook.com/profile.php?id=100002359931543}}
яка теж народилася в Маріуполі. Анна демонструє слайд-шоу про останній міський
фестиваль у Маріуполі. Дівчина розповідає про свою важку подорож, про зустрічі
з друзями та родичами, фестиваль з танцювальними колективами в традиційних
костюмах і мера, якого згодом звинувачують у тому, що він не підготував місто
належним чином до війни. На фото Анни учасники заходу могли побачити не образи
руйнування, а кадри, сповнені меланхолії, які якимось чином передають
передчуття того, що станеться пізніше. Перед очима глядачів постає гарне,
сучасне та яскраве місто, яке за декілька місяців російські військові
перетворили на суцільні руїни. На момент візиту до Маріуполя Анні було 24 роки.
Їй пощастило відвідати Маріуполь за 5 місяців до російсько-загарбницької війни.
До цього дівчина не бачила рідне місто 7 років. Тож вона описує довоєнний
період із дуже суб'єктивної точки зору.

\begin{leftbar}
\emph{\enquote{Я фотографувала місто, в якому народилася, не знаючи, що прощаюся. Це був
останній міський фестиваль Маріуполя. Другого не буде, доки не
відбудеться деокупація}}, — підкреслила Анна.
\end{leftbar}

\ii{26_09_2022.stz.news.ua.donbas24.1.oksana_stomina_virshi_spogady_znajom_nimci_mrpl.pic.2}

Зала була заповнена до останнього місця. На захід завітали як німці, так і
українці, серед яких були і маріупольці. Вони слухали розповідь та вірші
Оксани, які перекладала Rita Grinko, дивилися на фото Анни і не стримували сліз
та своїх емоцій. Оксана планує продовжити співпрацю з Анною та проводити
подібні заходити в інших німецьких містах.

\begin{leftbar}
\emph{\enquote{Ми всі розуміли, що наше місто має стратегічне значення. А ще, мабуть, ми
розлютили. Тому що ми трималися, ми розбудовувалися навіть під час
війни, місто ставало кращим і гарнішим, ми про це казали з гордістю}}, — зауважила Оксана Стоміна.
\end{leftbar}

\textbf{Читайте також:} \emph{\enquote{Маріуполь — це Україна}: як в містах України проходила національна акція}%
\footnote{\enquote{Маріуполь — це Україна}: як в містах України проходила національна акція, Ольга Демідко, donbas24.news, 24.09.2022, \par%
\url{https://donbas24.news/news/mariupol-ce-ukrayina-yak-v-mistax-ukrayini-proxodila-nacionalna-akciya-foto}%
}

\ii{26_09_2022.stz.news.ua.donbas24.1.oksana_stomina_virshi_spogady_znajom_nimci_mrpl.pic.3}

\subsubsection{Проза та вірші Оксани Стоміної про війну}

Оксана є оганізаторкою багатьох соціальних, літературних, правових, благодійних
проєктів. Вона авторка й редакторка кількох збірок про війну в Україні \enquote{АТОмы
судьбы}, \enquote{По живому. Навколовоєнні щоденники}, \enquote{DER KRIEG KOMMT OHNE EINLADUNG}
(\enquote{Війна приходить без запрошення}), \enquote{Per širdį} (\enquote{Через серце}, литовською);
авторка ігрових книжок-путівників для дітей і дорослих \enquote{Прогулянка з Маріком},
\enquote{Дивовижна подорож з Маріком і Марічкою}, а також збірок казок і поезій
\enquote{Таємниці старих стін}, \enquote{Лист дорослому}, \enquote{Нечаянные стихи}, \enquote{Про живе}. Як
свідок злочину росії і героїзму українських захисників, Оксана пише вірші
українською мовою, які перекладаються німецькою. Наразі письмениця працює над
новою книгою \textbf{\enquote{Щоденник того, хто вижив}}.

\begin{quote}
\textbf{КОЛИ Я ЙШЛА}

Коли я ішла, з неба падали не зірки.

Спогад про це пронизливий і гіркий.

Знаю, мій Маріуполю, й за роки

Це не минеться.

Навіть в житті наступному мимовіль

Я відчуватиму невиліковний біль.

Що там мого залишилося в тобі?

Тільки серце.

Що там мого залишилось? Тільки дім.

Очі пече розпачу чорний дим...

Скільки в мені виявилось води

Й жалю!

Скільки в мені пам'яті! Вір не вір,

Кожна з твоїх вулиць веде в мій двір,

І як завжди сонце моє встає

Над \enquote{Азовсталлю}.

Той твій прощальний погляд, останній сум

Я, наче хрест свій відтепер несу.

Кожної ночі плачу твою росу,

Мрію мрії.

Що там мого залишилось? Тільки я.

Ти — мій незгасний вогник, ти мій маяк!

Я повернусь до тебе, душа моя,

Місто Марії!
\end{quote}

\textbf{Читайте також:} \emph{Почуйте голос Маріуполя: історії людей, яким пощастило евакуюватися з блокадного міста}%
\footnote{Почуйте голос Маріуполя: історії людей, яким пощастило евакуюватися з блокадного міста, Алевтина Швецова, donbas24.news, 21.09.2022, \par%
\url{https://donbas24.news/news/pocuite-golos-mariupolya-istoriyi-lyudei-yakim-poshhastilo-evakuyuvatisya-z-blokadnogo-mista}%
}

\ii{26_09_2022.stz.news.ua.donbas24.1.oksana_stomina_virshi_spogady_znajom_nimci_mrpl.pic.4}

\begin{quote}
\textbf{З НЕВІДПРАВЛЕНОГО}

або листи в полон

\emph{Моєму чоловікові присвячується...}

Ми пишемо один одному ці листи. Чисті й прозорі.

Не про війну і зброю, і блокпости, а про вишневі зорі,

Файне гніздечко під соснами, щастя та перемогу.

Пишемо про кохання. Куди ж без нього?

І лише трохи — про те, де ти, і те, як мені без тебе.

Ми пишемо й пишемо ці листи й підкидуємо у небо,

Пишемо й пишемо ці слова й роняємо їх у воду.

Адже не маємо іншого ані виходу, ані входу...

Адже не маємо адресів — вулиць, будинків, міст...

Отже, римовані голоси створять між нами міст.

Адже далі, ніж мій адресат, було б на Марсі або Венері.

Отже... Цілую твоє чоло й лишаю це на папері.
\end{quote}

\textbf{Читайте також:} \emph{Життя після полону: захисник Маріуполя зробив пропозицію коханій}%
\footnote{Життя після полону: захисник Маріуполя зробив пропозицію коханій, Яна Іванова, donbas24.news, 25.09.2022, \par%
\url{https://donbas24.news/news/zittya-pislya-polonu-zaxisnik-mariupolya-zrobiv-propoziciyu-koxanii-video}%
}

\ii{26_09_2022.stz.news.ua.donbas24.1.oksana_stomina_virshi_spogady_znajom_nimci_mrpl.pic.5}

\begin{quote}
\textbf{ДОДОМУ}

\begingroup
\em
Вірність птахів батьківщині має назву \enquote{філопатрія}.

Вимушеним переселенцям присвячується...
\endgroup

Милий мій дім, люба моя фортеця...

Де ти тепер? Як ти тепер без мене?

Хто там тепер дихає, хто сміється,

Хто розкрадає мотлох і рештки серця

Та приміряє щастя моє буденне?

Хто віднесе на смітник всі мої світлини

Й зніме з поличок малюнки моєї малечі?..

Ось і пройшла стадія заперечень.

Біль цей якийсь пронизливий і лелечій,

Дуже важкий й тисне мені на плечі.

Але душа все ще до тебе плине.

Але душа — вірна, нестримна птаха —

Вперто кружляє понад дірявим дахом,

Наче понад розореним вщент гніздом.

Звідки у неї, тендітної, впертість ця та

Здатність не здатися й вірити до кінця?

Пташко моя мала, бережи крильцята

Й вміння вертатись додому!
\end{quote}

Раніше, Донбас24 розповідав \href{https://donbas24.news/news/kiyan-i-gostei-mista-zaprosuyut-na-blagodiinii-aukcion-mariupol-de-ti}{про благодійний аукціон}%
\footnote{\enquote{Маріуполь. Де Ти?}: киян і гостей міста запрошують на благодійний аукціон, Алевтина Швецова, donbas24.news, 26.09.2022, \par\url{https://donbas24.news/news/kiyan-i-gostei-mista-zaprosuyut-na-blagodiinii-aukcion-mariupol-de-ti}}

Ще більше новин та найактуальніша інформація про Донецьку та Луганську області
в нашому телеграм-каналі Донбас24.

ФОТО: з особистого архіву Оксани Стоміної

\ii{insert.author.demidko_olga}
