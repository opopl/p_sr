% vim: keymap=russian-jcukenwin
%%beginhead 
 
%%file 29_03_2022.fb.dykyj_evgen.1.vijna_vijna
%%parent 29_03_2022
 
%%url https://www.facebook.com/evgen.dykyj/posts/10159983890433808
 
%%author_id dykyj_evgen
%%date 
 
%%tags 
%%title ВИБІР МІЖ ВІЙНОЮ ТА ВІЙНОЮ
 
%%endhead 
 
\subsection{ВИБІР МІЖ ВІЙНОЮ ТА ВІЙНОЮ}
\label{sec:29_03_2022.fb.dykyj_evgen.1.vijna_vijna}
 
\Purl{https://www.facebook.com/evgen.dykyj/posts/10159983890433808}
\ifcmt
 author_begin
   author_id dykyj_evgen
 author_end
\fi

ВИБІР МІЖ ВІЙНОЮ ТА ВІЙНОЮ

Всім відома цитата Черчиля про те, що обираючи між ганьбою та війною обрали
ганьбу, але отримали і ганьбу і війну. Здається, Україна збирається побити
рекорд, і вже навіть не маючи вибору, а маючи повноцінну війну, притому війну
далеко не програну, хоче добровільно обрати ганьбу на додаток до цієї війни. 

Умови миру, які сьогодні оголосили Подоляк, Арахамія та Чалий - це умови
де-факто нашої капітуляції, для якої нема жодної підстави з точки зору
військової ситуації. 

Це \enquote{гібридний} мир, який дає Росії все що їй потрібно, окрім програми
\enquote{максимум}, тобто тотального контролю над Україною. 

Натомість РФ отримує де-факто визнання окупації Криму та Донбасу та гарантії що
ми навіть не спробуємо їх визволяти, наш позаблоковий статус (тобто нашу
відмову від щансу у майбутньому отримати захист згідно Атлантичної хартії), і
головне - РФ отримує зняття санкцій.

У свою чергу зняття санкцій означає виживання та безкарність путінського
режиму, який наразі вже тріщить по швах, та внаслідок військової поразки в
Україні та збереження санкцій швидше за все довго не протримається. Натомість
залишений безкарним він повернеться до виконання своєї програми збирання докупи
\enquote{втрачених земель Імперії}.

Тобто \enquote{у сухому залишку} ціною величезних жертв цивільного населення та
героїзму захисників країни ми всього лише отримуємо повернення до ситуації
станом на 23.02.2022 - тільки із тисячами смертей, мільйонами біженців,
зруйнованими містами та розполовиненою економікою. 

А РФ у \enquote{покарання} за агресію також просто повертається у 23.02, і отримує
можливість спокійно провести роботу над помилками та підготоуватись до нової
агресії, в ході якої таки виконати свою програму \enquote{максимум} - \enquote{остаточне
вирішення українського питання} (с). 

Тож \enquote{мир} про який нібито домовились у Стамбулі для нас є одночасно: ганебним
морально, нічим не виправданим з військової точки зору, і самогубчим у
середньостроковій часовій перспективі. 

Допустити такий мир - те саме, що добровільно капітулювати перед путінським
режимом, із усіма відповідними наслідками.

Наш реальний вибір - не між ганебним псевдомиром та війною, а між дуже важкою,
але переможною війною сьогодні, і ще важчою та швидше за все програною війною у
близькому майбутньому. Сподіваюсь, ми у цьому виборі не зхибимо.

Слава Україні.

\ii{29_03_2022.fb.dykyj_evgen.1.vijna_vijna.cmt}
\ii{29_03_2022.fb.dykyj_evgen.1.vijna_vijna.cmtx}
