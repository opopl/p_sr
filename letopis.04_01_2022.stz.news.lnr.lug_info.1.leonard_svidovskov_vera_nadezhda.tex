% vim: keymap=russian-jcukenwin
%%beginhead 
 
%%file 04_01_2022.stz.news.lnr.lug_info.1.leonard_svidovskov_vera_nadezhda
%%parent 04_01_2022
 
%%url https://lug-info.com/news/novogodnij-blic-zhurnalist-leonard-svidovskov-bez-very-i-nadezhdy-ochen-trudno-vyzhit
 
%%author_id 
%%date 
 
%%tags vera,nadezhda,donbass,lnr,zhizn,obschestvo,dnr
%%title Новогодний блиц. Журналист Леонард Свидовсков: "Без веры и надежды очень трудно выжить"
 
%%endhead 
\subsection{Новогодний блиц. Журналист Леонард Свидовсков: \enquote{Без веры и надежды очень трудно выжить}}
\label{sec:04_01_2022.stz.news.lnr.lug_info.1.leonard_svidovskov_vera_nadezhda}

\Purl{https://lug-info.com/news/novogodnij-blic-zhurnalist-leonard-svidovskov-bez-very-i-nadezhdy-ochen-trudno-vyzhit}

\begin{zznagolos}
Заместитель генерального директора Государственной телерадиовещательной
компании, председатель Республиканского союза журналистов Леонард Свидовсков в
традиционном новогоднем блиц-интервью ЛИЦ делится своим видением итогов
ушедшего года и прогнозом на год наступивший.
\end{zznagolos}

- Как вы оцениваете положение и ситуацию в Республиках Донбасса к началу 2022
года?

\ii{04_01_2022.stz.news.lnr.lug_info.1.leonard_svidovskov_vera_nadezhda.pic.1}

- Ситуация в Народных Республиках, выражаясь медицинской терминологией,
\enquote{стабильно тяжелая}. И дело не только и не столько в уровне жизни. Чего греха
таить: хотелось бы лучшего. Чтобы не экономить на всем. Не постоянно \enquote{решать
проблемы}, \enquote{латать дыры}, \enquote{изыскивать средства}. А строить и развиваться. И
глобально, и конкретно! Не только в масштабах всей Республики. Или даже всего
Донбасса. А в каждой семье. И каждому человеку. Чтобы \enquote{семейный бюджет} не
распределялся по статьям: \enquote{на пропитание}, \enquote{на коммуналку} и \enquote{на проезд}. Очень
хочется, чтобы появилась возможность откладывать \enquote{на мебель}. Или, как в
советское время, \enquote{на кооператив}. Шутка ли! Взнос на двухкомнатную
кооперативную квартиру в городе Ворошиловграде еще в середине 1980-х был две с
половиной тысячи рублей. Согласен, сумма немаленькая. Но за полтора-два года
семья рядового инженера, врача, учителя, работника любого из 42 предприятий
могла эту сумму собрать. Сэкономить, накопить.

И самое главное – все вокруг строилось, преображалось. Украшалось и
обновлялось. Фонтаны работали. Дороги и особенно тротуары – постоянно
ремонтировались. Асфальтоукладчики работали по девять месяцев в году. В городе
работали несколько домостроительных комбинатов. А сейчас? Подъемный кран – в
диковинку. А школьники, наверное, его вообще, в глаза не видели. Понятное дело!
Линия фронта в 15 километрах. Но хочется, в конце-то концов помечтать. О том,
что на остановке Коцюбинского (нынешняя остановка \enquote{Театр кукол}) когда-нибудь,
но обязательно сядешь на трамвай \enquote{двойную шестерку}. И прокатишься на ней аж до
Медгородка. И, как на остановке \enquote{Динамо}, подождешь 103-й, рейсовый
автобус-\enquote{гармошку}. И проедешь до кольца - \enquote{Полтинника} (квартал 50 лет
Октября).

Согласен, что сегодня – это лишь мечты. Но нашим землякам необходима надежда на
то, что все обязательно наладится. Профессия строителя - возродится. А работать
в коммунальном предприятии, на благо города станет престижно.

И самое главное. Пока такая надежда слабо ощущается. Послушайте людей. О чем
они говорят? О войне и мире, о санкциях, о \enquote{Северном потоке}. Они сейчас
воспринимают жизнь глобально. В масштабах геополитики. И по большому счету,
наши земляки, хотят они того или нет, стали заложниками мировых кризисов. А
хозяевами своей жизни – перестали быть. И нет уверенности в будущем. В этом,
наверное, главная проблема всех прошлых лет. И особенно 2021-го. Та
\enquote{трудность}, которую необходимо срочно разрешить. Вместе. В ближайшее время.
Без веры и без надежды очень трудно выжить. Даже практически невозможно.

- Что вы считаете наиболее значимыми достижениями Луганской и Донецкой Народных
Республик и каковы были самые серьезные проблемы в ушедшем году?

- Каждый прожитый год для жителей Луганской Народной Республики – само по себе
достижение. Выжили, превозмогли, выстояли. Не сдались. Не пошли на попятную.
Наперекор всему. И не сломились. В этом наша маленькая, но очень значительная
победа. Мы стали полноправными гражданами Российской Федерации. Реализовали
одно из главных гражданских прав – избирательное. Мы вошли в политическую
систему.

На первичном уровне – началось партийное строительство. Это – системообразующий момент.

Экономически - входим в единый промышленный комплекс. Вот, смотрите. Совсем
немного времени прошло с момента опубликования указа президента России
Владимира Владимировича Путина. Но уже заработал Алчевский металлургический
комбинат. Дает продукцию Луганский литейно-механический. Появилась надежда и на
то, что вскоре будут востребованы и наши колесные пары. Это на первом этапе.
Заработают цеха МС-3 и механосборочный-7 на \enquote{Лугансктепловозе}. Хотя мы это
предприятие до сих пор называем \enquote{Завод ОР} (Октябрьской революции). Он у нас в
городе – даже в названиях. \enquote{Кольцо завода ОР}, \enquote{Городок завода ОР}, \enquote{Поселок
завода ОР}. Сейчас главное – запустить производство. А в перспективе – по
необъятным Российским просторам обязательно пройдут наши локомотивы. И даже
чисто экономически. Именно Россия нуждается в наших магистральных тепловозах.
Проблема только в том, что необходимо срочным образом наладить логистику.
Восстановить железнодорожное сообщение. А то ведь как было совсем недавно - в
\enquote{украинское безвременье}. Одной рукой мы подписываем соглашение о создании
\enquote{Еврорегиона Донбасс}, в который должны были войти Луганская, Ростовская и
Воронежская области. Зато другой – издаем распоряжение, чтобы \enquote{железнодорожные
пути разобрали}. На 122-м километре. И в Лантратовке – границу обустроили.
Таможенный пункт. Теперь все это необходимо восстановить.

Авиасообщение – тоже первостепенная задача. Неспроста же глава Республики
Леонид Иванович Пасечник намеревается воссоздать наш аэропорт, разрушенный
украинскими карателями. Ну, и естественно, тогда уже необходимо будет
уничтоженный за четверть века Луганский авиаотряд восстановить. Если кто забыл,
то еще в начале 1990-х в нем было 27 самолетов. Причем один – из новейших
Ту-154. На нем позже президент Азербайджана летал. Сдали в аренду на пять лет.
А потом документы затерялись. Авиаотряд – ликвидировали. Кто-то наверняка
нажился на такой \enquote{сделке}. За наш счет. Придется начинать все с нуля. Но это
крайне необходимо. Чтобы не чувствовать себя оторванными. Потому что на
мировоззренческом – мы давно часть огромной и великой России.

- Ваш прогноз развития ситуации вокруг Донбасса в 2022 году?

- Очень не хочется быть заложниками геополитических процессов. Разного рода
\enquote{оборонные союзы}, \enquote{система безопасности}, \enquote{саммиты} и \enquote{переговоры} пусть как
можно меньше влияют на нашу с вами жизнь. Мы еще восемь лет назад свой выбор
сделали. И пусть все в мире об этом знают. Запомните! Мы никогда не вернемся в
украинское прошлое. А впереди у нас - исключительно Русское будущее. И для того
чтобы его обустроить, у нас дел невпроворот! Промышленность возрождать,
железную дорогу восстанавливать. Аэропорт воссоздать. У нас возможности –
уникальные. А силе духа и твердости характера – все завидуют.

Вот, смотрите. Полвека назад что нам удалось сделать. Первый универсальный
гастроном – \enquote{Ворошиловградский}. Цирк построили. Универмаг \enquote{Россия} открыли.
Это только в Луганске. А по области – всего не перечесть. В конце концов, ровно
50 лет назад ворошиловоградская \enquote{Заря}, команда, представляющая на всесоюзной,
а потом уже и европейской аренах все тот же \enquote{Завод ОР}, стала чемпионом СССР. В
то время это был не просто прорыв. А событие, которое можно было сравнить разве
что с полетом человека в космос. Народный артист Советского Союза, наш земляк
Юрий Богатиков пропел: \enquote{В Москве и Киеве болельщикам не спится. Все изменяется
на свете, говорят. Перемещается футбольная столица в рабочий город
Ворошиловград}. Самое главное, что мы \enquote{утерли нос} амбиционному украинскому
республиканскому руководству. Лично (первому секретарю Центрального комитета
компартии Украины) Владимиру Щербицкому. \enquote{Как это так?! Почему осмелились?
Непорядок!} - возмущались в Киеве. Кстати, нам этого и не простили. Но мы
справились, пережили. Опять-таки, начинали с нуля. Потому что наша \enquote{Заря} — это
явление. Это начало чего-то нового. И всегда – надежда. Вот отпразднуем мы в
новом году юбилей победы легендарной команды. Наградим ветеранов. Соберемся на
\enquote{Авангарде}. И совсем скоро наши мальчишки подрастут. И будут играть в \enquote{Заре} -
на самом высоком уровне. Они-то с пеленок знают о том, что \enquote{\enquote{Заря} – чемпион!}.
И мы все так же будем за них болеть. Переживать. И побеждать. Вместе. А в честь
Победы – пройдемся многотысячной демонстрацией по Оборонной. Так уже было, в
победном 1972 году. Если что – можем повторить! На том стоим!
