% vim: keymap=russian-jcukenwin
%%beginhead 
 
%%file books.bredberi_rej.rasskazy.vse_leto_v_odin_den
%%parent books.bredberi_rej.rasskazy
 
%%url 
 
%%author_id 
%%date 
 
%%tags 
%%title 
 
%%endhead 

\section{Все лето в один день}
\label{sec:books.bredberi_rej.rasskazy.vse_leto_v_odin_den}

— Готовы?

— Да!

— Уже?

— Скоро!

— А ученые верно знают? Это правда будет сегодня?

— Смотри, смотри, сам видишь!

Теснясь, точно цветы и сорные травы в саду, все вперемешку, дети старались
выглянуть наружу — где там запрятано солнце? Лил дождь. Он лил не переставая
семь лет подряд; тысячи и тысячи дней, с утра до ночи, без передышки дождь лил,
шумел, барабанил, звенел хрустальными брызгами, низвергался сплошными потоками,
так что кругом ходили волны, заливая островки суши. Ливнями повалило тысячи
лесов, и тысячи раз они вырастали вновь и снова падали под тяжестью вод. Так
навеки повелось здесь, на Венере, а в классе было полно детей, чьи отцы и
матери прилетели застраивать и обживать эту дикую дождливую планету.

— Перестает! Перестает!

— Да, да!

Марго стояла в стороне от них, от всех этих ребят, которые только и знали, что
вечный дождь, дождь, дождь. Им всем было по девять лет, и если выдался семь лет
назад такой день, когда солнце все-таки выглянуло, показалось на час
изумленному миру, они этого не помнили. Иногда по ночам Марго слышала, как они
ворочаются, вспоминая, и знала: во сне они видят и вспоминают золото, яркий
желтый карандаш, монету — такую большую, что можно купить целый мир. Она знала,
им чудится, будто они помнят тепло, когда вспыхивает лицо и все тело — руки,
ноги, дрожащие пальцы. А потом они просыпаются — и опять барабанит дождь, без
конца сыплются звонкие прозрачные бусы на крышу, на дорожку, на сад и лес, и
сны разлетаются как дым.

Накануне они весь день читали в классе про солнце. Какое оно желтое, совсем как
лимон, и какое жаркое. И писали про него маленькие рассказы и стихи.

Мне кажется, солнце — это цветок, Цветет оно только один часок.

Такие стихи сочинила Марго и негромко прочитала их перед притихшим классом. А
за окнами лил дождь.

— Ну, ты это не сама сочинила! — крикнул один мальчик.

— Нет, сама, — сказала Марго, — Сама.

— Уильям! — остановила мальчика учительница.

Но то было вчера. А сейчас дождь утихал, и дети теснились к большим окнам с
толстыми стеклами.

— Где же учительница?

— Сейчас придет.

— Скорей бы, а то мы все пропустим!

Они вертелись на одном месте, точно пестрая беспокойная карусель. Марго одна
стояла поодаль. Она была слабенькая, и казалось, когда-то давно она заблудилась
и долго-долго бродила под дождем, и дождь смыл с нее все краски: голубые глаза,
розовые губы, рыжие волосы — все вылиняло. Она была точно старая поблекшая
фотография, которую вынули из забытого альбома, и все молчала, а если и
случалось ей заговорить, голос ее шелестел еле слышно. Сейчас она одиноко
стояла в сторонке и смотрела на дождь, на шумный мокрый мир за толстым стеклом.

— Ты-то чего смотришь? — сказал Уильям. Марго молчала.

— Отвечай, когда тебя спрашивают!

Уильям толкнул ее. Но она не пошевелилась; покачнулась — и только. Все ее
сторонятся, даже и не смотрят на нее. Вот и сейчас бросили ее одну. Потому что
она не хочет играть с ними в гулких туннелях того города-подвала. Если
кто-нибудь осалит ее и кинется бежать, она только с недоумением поглядит вслед,
но догонять не станет. И когда они всем классом поют песни о том, как хорошо
жить на свете и как весело играть в разные игры, она еле шевелит губами. Только
когда поют про солнце, про лето, она тоже тихонько подпевает, глядя в
заплаканные окна.

Ну а самое большое ее преступление, конечно, в том, что она прилетела сюда с
Земли всего лишь пять лет назад, и она помнит солнце, помнит, какое оно,
солнце, и какое небо она видела в Огайо, когда ей было четыре года. А они — они
всю жизнь живут на Венере; когда здесь в последний раз светило солнце, им было
только по два года, и они давно уже забыли, какое оно, и какого цвета, и как
жарко греет. А Марго помнит.

— Оно большое, как медяк, — сказала она однажды и зажмурилась.

— Неправда! — закричали ребята.

— Оно — как огонь в очаге, — сказала Марго.

— Врешь, врешь, ты не помнишь! — кричали ей.

Но она помнила и, тихо отойдя в сторону, стала смотреть в окно, по которому
сбегали струи дождя. А один раз, месяц назад, когда всех повели в душевую, она
ни за что не хотела стать под душ и, прикрывая макушку, зажимая уши ладонями,
кричала — пускай вода не льется на голову! И после того у нее появилось
странное, смутное чувство: она не такая, как все. И другие дети тоже это
чувствовали и сторонились ее.

Говорили, что на будущий год отец с матерью отвезут ее назад на Землю — это
обойдется им во много тысяч долларов, но иначе она, видимо, зачахнет. И вот за
все эти грехи, большие и малые, в классе ее невзлюбили. Противная эта Марго,
противно, что она такая бледная немочь, и такая худющая, и вечно молчит и ждет
чего-то, и, наверно, улетит на Землю…

— Убирайся! — Уильям опять ее толкнул. — Чего ты еще ждешь?

Тут она впервые обернулась и посмотрела на него. И по глазам было видно, чего
она ждет. Мальчишка взбеленился.

— Нечего тебе здесь торчать! — закричал он. — Не дождешься, ничего не будет!
Марго беззвучно пошевелила губами.

— Ничего не будет! — кричал Уильям. — Это просто для смеха, мы тебя разыграли.
Он обернулся к остальным. — Ведь сегодня ничего не будет, верно?

Все поглядели на него с недоумением, а потом поняли, и засмеялись, и покачали
головами: верно, ничего не будет!

— Но ведь… — Марго смотрела беспомощно. — Ведь сегодня тот самый день, —
прошептала она. — Ученые предсказывали, они говорят, они ведь знают… Солнце…

— Разыграли, разыграли! — сказал Уильям и вдруг схватил ее.

— Эй, ребята, давайте запрем ее в чулан, пока учительницы нет!

— Не надо, — сказала Марго и попятилась.

Все кинулись к ней, схватили и поволокли, — она отбивалась, потом просила,
потом заплакала, но ее притащили по туннелю в дальнюю комнату, втолкнули в
чулан и заперли дверь на засов. Дверь тряслась: Марго колотила в нее кулаками и
кидалась на нее всем телом. Приглушенно доносились крики. Ребята постояли,
послушали, а потом улыбнулись и пошли прочь — и как раз вовремя: в конце
туннеля показалась учительница.

— Готовы, дети? — она поглядела на часы.

— Да! — отозвались ребята.

— Все здесь?

— Да!

Дождь стихал. Они столпились у огромной массивной двери. Дождь перестал. Как
будто посреди кинофильма про лавины, ураганы, смерчи, извержения вулканов
что-то случилось со звуком, аппарат испортился, — шум стал глуше, а потом и
вовсе оборвался, смолкли удары, грохот, раскаты грома… А потом кто-то выдернул
пленку и на место ее вставил спокойный диапозитив — мирную тропическую
картинку. Все замерло — не вздохнет, не шелохнется. Такая настала огромная,
неправдоподобная тишина, будто вам заткнули уши или вы совсем оглохли. Дети
недоверчиво подносили руки к ушам. Толпа распалась, каждый стоял сам по себе.
Дверь отошла в сторону, и на них пахнуло свежестью мира, замершего в ожидании.

И солнце явилось. Оно пламенело, яркое, как бронза, и оно было очень большое. А
небо вокруг сверкало, точно ярко-голубая черепица. И джунгли так и пылали в
солнечных лучах, и дети, очнувшись, с криком выбежали в весну.

— Только не убегайте далеко! — крикнула вдогонку учительница. — Помните, у вас
всего два часа. Не то вы не успеете укрыться!

Но они уже не слышали, они бегали и запрокидывали голову, и солнце гладило их
по щекам, точно теплым утюгом; они скинули куртки, и солнце жгло их голые руки.

— Это получше наших искусственных солнц, верно?

— Ясно, лучше!

Они уже не бегали, а стояли посреди джунглей, что сплошь покрывали Венеру и
росли, росли бурно, непрестанно, прямо на глазах. Джунгли были точно стая
осьминогов, к небу пучками тянулись гигантские щупальца мясистых ветвей,
раскачивались, мгновенно покрывались цветами — ведь весна здесь такая короткая.
Они были серые, как пепел, как резина, эти заросли, оттого что долгие годы они
не видели солнца. Они были цвета камней, и цвета сыра, и цвета чернил, и были
здесь растения цвета луны.

Ребята со смехом кидались на сплошную поросль, точно на живой упругий матрац,
который вздыхал под ними, и скрипел, и пружинил. Они носились меж деревьев,
скользили и падали, толкались, играли в прятки и в салки, но главное — опять и
опять, жмурясь, глядели на солнце, пока не потекут слезы, и тянули руки к
золотому сиянию и к невиданной синеве, и вдыхали эту удивительную свежесть, и
слушали, слушали тишину, что обнимала их словно море, блаженно спокойное,
беззвучное и недвижное. Они на все смотрели и всем наслаждались. А потом, будто
зверьки, вырвавшиеся из глубоких нор, снова неистово бегали кругом, бегали и
кричали. Целый час бегали и никак не могли угомониться. И вдруг… Посреди
веселой беготни одна девочка громко, жалобно закричала. Все остановились.
Девочка протянула руку ладонью кверху.

— Смотрите, сказала она и вздрогнула. — Ой, смотрите!

Все медленно подошли поближе. На раскрытой ладони, по самой середке, лежала
большая круглая дождевая капля. Девочка посмотрела на нее и заплакала. Дети
молча посмотрели на небо.

— О-о…

Редкие холодные капли упали на нос, на щеки, на губы. Солнце затянула туманная
дымка. Подул холодный ветер. Ребята повернулись и пошли к своему дому-подвалу,
руки их вяло повисли, они больше не улыбались.

Загремел гром, и дети в испуге, толкая друг дружку, бросились бежать, словно
листья, гонимые ураганом. Блеснула молния — за десять миль от них, потом за
пять, в миле, в полумиле. И небо почернело, будто разом настала непроглядная
ночь. Минуту они постояли на пороге глубинного убежища, а потом дождь полил
вовсю. Тогда дверь закрыли, и все стояли и слушали, как с оглушительным шумом
рушатся с неба тонны, потоки воды — без просвета, без конца.

— И так опять будет целых семь лет?

— Да. Семь лет. И вдруг кто-то вскрикнул:

— А Марго?

— Что?

— Мы ведь ее заперли, она так и сидит в чулане.

— Марго…

Они застыли, будто ноги у них примерзли к полу. Переглянулись и отвели взгляды.
Посмотрели за окно — там лил дождь, лил упрямо, неустанно. Они не смели
посмотреть друг другу в глаза. Лица у всех стали серьезные, бледные. Все
потупились, кто разглядывал свои руки, кто уставился в пол.

— Марго…

Наконец одна девочка сказала:

— Ну что же мы?…

Никто не шелохнулся.

— Пойдем… — прошептала девочка.

Под холодный шум дождя они медленно прошли по коридору. Под рев бури и раскаты
грома перешагнули порог и вошли в ту дальнюю комнату, яростные синие молнии
озаряли их лица. Медленно подошли они к чулану и стали у двери.

За дверью было тихо. Медленно, медленно они отодвинули засов и выпустили Марго.
