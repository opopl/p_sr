% vim: keymap=russian-jcukenwin
%%beginhead 
 
%%file slova.volja
%%parent slova
 
%%url 
 
%%author 
%%author_id 
%%author_url 
 
%%tags 
%%title 
 
%%endhead 
\chapter{Воля}
\label{sec:slova.volja}

\emph{Воля}. Это слово состоит из двух понятий: стремления (\enquote{сила воли},
\enquote{\emph{воля} к победе}) и свободы (\enquote{вольный ветер}, \enquote{\emph{воля
вольная}}). В отличие от более социальной \enquote{свободы}, \enquote{воля} это
природная категория (можно сравнить выражения \enquote{свобода слова} и
\enquote{дать волю словам}). Свобода - это право, но оно ограничивается правами
других людей. \emph{Воля} же не имеет никаких ограничений и не связана с законом. О
том, кто покинул тюрьму по окончании срока, мы говорим \enquote{вышел на
свободу}, а о том, кто сбежал - \enquote{вырвался на \emph{волю}}. Русский менталитет
определённо предпочитает волю как ничем не сдерживаемую силу. \emph{Воля} - это
русские бескрайние просторы, «где гуляем лишь ветер... да я!»,
\citTitle{Русский языковой менталитет: 10 основных понятий}, Языковедьма, zen.yandex.ru, 08.06.2021

%%%cit
%%%cit_head
%%%cit_pic
%%%cit_text
Я думаю, Страна, в которой сегодня не удалось еще создать глупого
человека - робота, это Россия. Нам постоянно нужно поделать что-то для души,
потому что хочется. Находясь на работе, в большинстве своем мы думаем только об
одном, когда этот чертов день закончится и побыстрее уже свалить отсюда. Для
нас работа находится далеко не на первом месте. В первую очередь для нас важно
пожить, вечно поспорить, что-то обсудить, в целом чувствовать себя живым.
Россия, как и во все времена, сегодня полна яркими и живыми личностями, каждая
из которых имеет свой уникальный характер. Начиная от простого работяги и
вплоть до нерадивых чиновников. Вот это всегда и напрягало запад - русская
непокорность и \emph{Воля}. Иначе говоря - проявление характера.  Вот друзья, в чем
заключается по мнению брата главное отличие человека русского от человека
запада. И я с ним пожалуй соглашусь
%%%cit_comment
%%%cit_title
\citTitle{После 11 лет жизни в США понял самое важное на мой взгляд отличие между американцами и русскими}, 
За Пределами Онлайна, zen.yandex.ru, 06.06.2021
%%%endcit

