% vim: keymap=russian-jcukenwin
%%beginhead 
 
%%file 10_02_2022.yz.tkachev_andrej.protoierej.1.chem_dokazhesh_chto_zhil_na_svete
%%parent 10_02_2022
 
%%url https://zen.yandex.ru/media/andreytkachev__official/chem-dokajesh-chto-ty-jil-na-svete-6204a94de01a1d6085f8bddc
 
%%author_id tkachev_andrej.protoierej
%%date 
 
%%tags chelovek,vopros,zhizn
%%title Чем докажешь, что ты жил на свете?
 
%%endhead 
 
\subsection{Чем докажешь, что ты жил на свете?}
\label{sec:10_02_2022.yz.tkachev_andrej.protoierej.1.chem_dokazhesh_chto_zhil_na_svete}
 
\Purl{https://zen.yandex.ru/media/andreytkachev__official/chem-dokajesh-chto-ty-jil-na-svete-6204a94de01a1d6085f8bddc}
\ifcmt
 author_begin
   author_id tkachev_andrej.protoierej
 author_end
\fi

То, что мы родились – это понятно. Это было видно, это понятно. Вот – бумажка. 

То, что мы умрем – это тоже понятно. То, что мы умрем, будем видно, к
сожалению. И будет бумажка. И здесь без бумажки не обойдешься.

\ii{10_02_2022.yz.tkachev_andrej.protoierej.1.chem_dokazhesh_chto_zhil_na_svete.pic.1}

А вот докажи, что ты жил! 

– Чем докажешь, что ты жил на свете? – Ну я вот бегал, прыгал, кушал, какал...

– Что еще? Чем еще ты был занят в этом мире? (ты все пела – это дело, так поди
же попляши).

Чем доказать вообще, что ты жил на свете. Помните: Человек должен посадить
дерево, вырастить сына, построить дом!? Почему? – Потому что это – останется. 

Тебя не будет, а кто-то с дерева яблочко сорвет, или грушку, или сливку (в
зависимости от того, что ты посадил). В доме будут жить люди. Может быть – твои
внуки. Может быть - другие люди (всякое бывает). Но – в твоем доме. Сын твой
родит еще сыновей. Твой след продолжит по миру. Останется след твой. 

Я жил! – Как докажешь? – Вот фамилия моя продолжилась в этих детях. – Вот
(скажем) мои картины. – Вот храм, построенный мною! – Вот колодец мною
выкопанный. (Иаков колодцы копал. Детей рожал и колодцы копал)».

Докажи, что ты – жил. Чем докажешь? 

Как говорил Пушкин: «Я не хочу, о други, умирать; Я жить хочу, чтоб мыслить и
страдать!»

Ты мыслил в жизни или не мыслил? Это очень трудное занятие. Думать очень
тяжело. Это только кажется, что мыслить легко. Сяду, мол, и подумаю! Ничего
подобного. Как правило, люди не думают. «Текут» туда, куда «текут» все. 

А вот так, чтобы остановиться, подумать и найти свой путь...

Ну, насчет того, что «страдать» – мы все страдаем. Это – понятно. 

Мыслить и страдать... 

«Я не хочу, о други, умирать; Я жить хочу, чтоб мыслить и страдать!»

Чем ты докажешь, что – жил? Кому помог? Кому слезу вытер? Кого утешил? Кого
спас?

Докажи и ты, что жил. О рождении, о смерти – бумажка есть. А о жизни? Чем
докажешь?

Вдруг нас спросят: чем докажешь, что на земле не просто так, как червь,
прополз, а как человек прожил? 

Страшный вопрос. Страшный... 

Пока не поздно, нужно сделать что-то. Незаметное, но важное. 

Чтобы было ясно, что ты был человеком.
