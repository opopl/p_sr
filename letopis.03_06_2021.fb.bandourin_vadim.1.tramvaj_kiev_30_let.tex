% vim: keymap=russian-jcukenwin
%%beginhead 
 
%%file 03_06_2021.fb.bandourin_vadim.1.tramvaj_kiev_30_let
%%parent 03_06_2021
 
%%url https://www.facebook.com/vadim.bandourin/posts/5843800315660266
 
%%author Бандурин, Вадим
%%author_id bandourin_vadim
%%author_url 
 
%%tags 
%%title ТРАМВАИ «15» и «18» (Киевляне помнят другой номер)
 
%%endhead 
 
\subsection{ТРАМВАИ «15» и «18» (Киевляне помнят другой номер)}
\label{sec:03_06_2021.fb.bandourin_vadim.1.tramvaj_kiev_30_let}
\Purl{https://www.facebook.com/vadim.bandourin/posts/5843800315660266}
\ifcmt
 author_begin
   author_id bandourin_vadim
 author_end
\fi

ТРАМВАИ «15» и «18» (Киевляне помнят другой номер).

Вот, кстати, интересная тема для \enquote{киевлян}.... ПОЧЕМУ у нас до сих пор на этом
маршруте ходят те же самые трамваи, на которых я ещё в начале 90-ых ездил на
\enquote{Контрактову Площу}, чтобы сесть на метро на до \enquote{Минской} к моей первой жене,
когда мы были ещё студентами? ЧТО «ОНИ» СДЕЛАЛИ за 30 ЛЕТ этой НЕЗАВИСИМОСТИ? 

\ifcmt
  pic https://scontent-cdt1-1.xx.fbcdn.net/v/t1.6435-9/195062500_5843800192326945_6646151298271166237_n.jpg?_nc_cat=110&ccb=1-3&_nc_sid=730e14&_nc_ohc=34WQSM_HozAAX_Xqo0v&_nc_ht=scontent-cdt1-1.xx&oh=e74e260daa39f62f7fd3744c57950613&oe=60DE36A6
\fi

Никаких \enquote{Kneeling trams and/or buses} (попытаюсь перевести). «Трамваи и/или автобусы, у которых заезд и выезд ОПУСКАЮЩИЙСЯ».  

Для людей in the wheel chair (в коляске) или пожилых людей с «кравчучкой»
(shopping stroller), то - только такси. А такси – не особо дешёвое средство
передвижения для людей с ограниченными возможностями и для пожилых людей. 

Спросите, что за 30 лет было сделано в Китае. ЧТО за 30 лет было сделано у
наших соседей? В Японии после двух ядерных (и не только) бомбёжек, ЧТО было
сделано в Ниппоне за 30 лет? О Сингапуре промолчим.

Да за 30 лет можно было в ПУСТЫНЕ страну построить. А мы ещё тут катаемся на
Татрах-трамваях выпуска начала 80-ых прошлого века.

Я с большим трудом удерживаюсь от «политики» в данном посте. Просто так.
Подумать «киевлянам», если они остались ещё тут.

Просто подумалось, что события 30-летней давности могут помнить только
киевляне. «Кавычки» вот поэтому я и использовал.

С уважением.
