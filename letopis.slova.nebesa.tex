% vim: keymap=russian-jcukenwin
%%beginhead 
 
%%file slova.nebesa
%%parent slova
 
%%url 
 
%%author 
%%author_id 
%%author_url 
 
%%tags 
%%title 
 
%%endhead 
\chapter{Небеса}
\label{sec:slova.nebesa}

%%%cit
%%%cit_pic
%%%cit_text
Смотрю я на \emph{Небо}. Слова Михаила Петренко. Смотрю я на \emph{Небо} и думу
гадаю, Ну что я не сокол, ну что не летаю, Зачем же мне, боже, ты крыльев не
дал? Я бы землю покинул и в \emph{Небо} слетал. Далёко за тучи, подальше от
света, Искать себе доли, искать там ответа, Да ласки у звёзд и у солнца
просить, И в свете их ярком всё зло утопить. Ведь доли с рожденья сдаюсь не
любимый – Наёмник у доли, мальчишка гонимый... Чужой я у доли, чужой у людей,
Да разве кто любит не родных детей? Живу я в лишеньях и счастья не знаю, И
горько без доли свой век коротаю, Я в горе познал, что лишь только одна,
\emph{Далёкое Небо} – моя сторона
%%%cit_comment
%%%cit_title
\citTitle{Украинские Песни Русскими Словами}, 
БРАТИНА, zen.yandex.ru, 15.12.2020
%%%endcit

%%%cit
%%%cit_head
%%%cit_pic
%%%cit_text
Субъективное восприятие, Гаал. Если вы родились в инкубаторе, стали взрослым в
коридоре, работали в подвале, а отпуск проводили в переполненном солярии, то,
выйдя наружу и не увидев над своей головой ничего, кроме солнца и \emph{неба}, вы
можете серьезно заболеть. Транториане разрешают своим детям выходить сюда раз в
год, после того как им исполнится пять лет. Не знаю, право, дает ли им это
что-нибудь. Во-первых, этого явно мало, а во-вторых, когда детей приводят сюда
первые несколько раз, они закатывают жуткие истерики. Следовало бы их приносить
сюда сразу же после рождения и, по крайней мере, раз в неделю. В общем, это не
имеет такого уж большого значения, — продолжал Джерилл — Что из того, что эти
люди никогда не видели \emph{неба}? Они счастливы там, внизу: они управляют Империей.
По-вашему, на какой высоте мы находимся?
%%%cit_comment
%%%cit_title
\citTitle{Основание}, Айзек Азимов
%%%endcit
