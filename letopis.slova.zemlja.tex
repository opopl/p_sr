% vim: keymap=russian-jcukenwin
%%beginhead 
 
%%file slova.zemlja
%%parent slova
 
%%url 
 
%%author 
%%author_id 
%%author_url 
 
%%tags 
%%title 
 
%%endhead 
\chapter{Земля}
\label{sec:slova.zemlja}

%%%cit
%%%cit_head
%%%cit_pic
%%%cit_text
Фронтмен группы \enquote{\emph{Земляне}}, российский музыкант Сергей Скачков
попал в список лиц, которые представляют угрозу национальной безопасности
Украины из-за выступлений на оккупированных территориях.  Об этом сообщает
пресс-служба Министерства культуры и информационной политики Украины.  Группа
выступала во время так называемого \enquote{референдума} в аннексированном
Крыму в 2014 году, а два году спустя, в 2016, давала концерты на
неподконтрольных Киеву территориях Донбасса.  Министерство культуры вносит
изменения в перечень лиц, которые предоставляют угрозу нацбезопасности, по
представлению СБУ. Сейчас в этом списке находится 200 фамилий.  Группу
\enquote{\emph{Земляне}} создали студентами Ленинградского Радиополитехникума в
1969 году. Коллектив известен своими хитами \enquote{Трава у дома},
\enquote{Каскадеры}, \enquote{\emph{Земля} в иллюминаторе}
%%%cit_comment
%%%cit_title
\citTitle{Сергей Скачков. Солиста группы Земляне признали угрозой нацбезопасности}, 
Эллина Либцис, strana.ua, 18.06.2021
%%%endcit

%%%cit
%%%cit_head
%%%cit_pic
%%%cit_text
В 00 часов 16 минут 30 июня состоялся еще один сеанс связи с кораблем.
Организатор и руководитель подготовки космонавтов генерал Николай Каманин
напомнил о необходимости при спуске вести постоянный репортаж на коротких и
ультракоротких волнах. А после посадки действовать строго по инструкции: резко
не двигаться, люк не открывать и ждать группу поиска.  «Желаю мягкой посадки.
До скорой встречи на \emph{Земле}!» — подытожил Николай Петрович. В ответ
командир экипажа Добровольский сказал: «Вас понял, условия посадки отличные. На
борту все в порядке, самочувствие экипажа отличное.  Благодарим за заботу и
добрые пожелания»
%%%cit_comment
%%%cit_title
\citTitle{«Они были обречены» 50 лет назад погиб экипаж «Союза-11». Кто
виноват в главной трагедии советской космонавтики?: Космос: Наука и техника:
Lenta.ru}, Сергей Варшавчик, lenta.ru, 30.06.2021
%%%endcit

%%%cit
%%%cit_head
%%%cit_pic
%%%cit_text
60 лет назад, 4 октября 1957 года, в Советском Союзе запустили на орбиту первый
искусственный спутник \emph{Земли}. Так в истории человечества началась космическая
эра, в которой наша страна на первых порах была бесспорным лидером. Почему мы
утратили преимущество в освоении космоса и теперь отстаем от США? Догонят ли
Россию в космической гонке Индия и Китай? Когда люди долетят до Марса и
вернутся на Луну? Одиноки ли мы во Вселенной и возможен ли контакт с
инопланетным разумом?
%%%cit_comment
%%%cit_title
\citTitle{«Мы до сих пор летаем на ракетах Королева» Почему Россия превратилась в космического аутсайдера}, 
Андрей Мозжухин, lenta.ru, 04.10.2017
%%%endcit

