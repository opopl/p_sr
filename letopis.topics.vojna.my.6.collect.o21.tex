% vim: keymap=russian-jcukenwin
%%beginhead 
 
%%file topics.vojna.my.6.collect.o21
%%parent topics.vojna.my.6.collect.gen
 
%%url 
 
%%author_id 
%%date 
 
%%tags 
%%title 
 
%%endhead 

Год назад я написал заметку "Запах катастрофы". А сейчас катастрофой в России
уже непросто пахнет. Она на пороге.

Видео дня

Мой прогноз: Россия проиграет войну и режим Путина рухнет максимум в течение
года. Российские власти, кажется, начали об этом догадываться и пытаются срочно
развести украинцев на какие-то переговоры. Именно поэтому Путин во время
переговоров в Сочи с Эрдоганом "намекнул", что может встретиться с президентом
Зеленским. Но украинцы, конечно, обмануть себя не дадут.

Ниже прошлогодний текст "Запах катастрофы".

Знаете, что напоминает мне вся эта пропагандистская свистопляска с
"непобедимыми" ракетами, "не имеющими аналогов", с бесконечными рассказами о
"народе победителе" и угрозами "врагам"?

Узнаете: "И на вражьей земле мы врага разгромим
Малой кровью, могучим ударом!".
А потом была катастрофа 1941 года: "Будь проклят
сорок первый год —
ты, вмерзшая в снега пехота".

В истории России никогда так безудержно не бахвалились, не хвастались своей
мощью, не кичились своими победами, не пытались так унизить соперников, как
перед национальными катастрофами.

"Прошедшее России было удивительно, ее настоящее более чем великолепно, что же
касается до будущего, то оно выше всего, что может нарисовать себе самое смелое
воображение" - так бюрократия оценивала ситуацию в империи Николая 1 накануне
Крымской катастрофы.

Безудержная урапатриотическая, к тому же расистская пропаганда сопровождала
начало катастрофической Русско-японской войны. "Япошки" подавались как какие-то
недочеловеки, с которыми русские богатыри легко и быстро разберутся.

А в 1914 году накануне Первой мировой, за несколько лет до краха империи,
поражения в войне, гибели семьи скорбный умом российский император предавался
завиральным мечтаниям: "Я схожу с ума, когда думаю о перспективах России, мы
станем самым великим народом, самым великим государством, всё в мире будет
делаться с нашего разрешения".

Вот и сейчас со всех российских телеэкранов льется урапатриотическая сивуха,
приводящая потребителей в состояние восторженного отупения, которое заставляет
пьяного совершать безумно самонадеянные поступки: лезть купаться в штормовое
море или задираться в одиночку на большую и сильную компанию. В результате
пьяных дураков бьют. Так уже много раз было в истории России, будет, вероятно,
и в этот раз.
