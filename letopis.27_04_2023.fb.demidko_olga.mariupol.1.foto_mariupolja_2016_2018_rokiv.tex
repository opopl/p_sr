%%beginhead 
 
%%file 27_04_2023.fb.demidko_olga.mariupol.1.foto_mariupolja_2016_2018_rokiv
%%parent 27_04_2023
 
%%url https://www.facebook.com/100009080371413/posts/pfbid037oKARPTHjYYU7FzvwQvmKvJZoX7mtDcfaTU5tpGwFWr8wMbq8E3aU1L9ENSuzppEl
 
%%author_id demidko_olga.mariupol
%%date 27_04_2023
 
%%tags 
%%title Фото Маріуполя 2016-2018 років
 
%%endhead 

\subsection{Фото Маріуполя 2016-2018 років}
\label{sec:27_04_2023.fb.demidko_olga.mariupol.1.foto_mariupolja_2016_2018_rokiv}

\Purl{https://www.facebook.com/100009080371413/posts/pfbid037oKARPTHjYYU7FzvwQvmKvJZoX7mtDcfaTU5tpGwFWr8wMbq8E3aU1L9ENSuzppEl}
\ifcmt
 author_begin
   author_id demidko_olga.mariupol
 author_end
\fi

Фото Маріуполя 2016-2018 років. Можливо, для когось місто здавалось не таким
вже і цікавим, адже яскраві локації з'явилися трохи згодом, але для мене
Маріуполь і тоді, і раніше був містом, що заворожувало своїми краєвидами та
унікальною архітектурою.  Саме у 2016 році долучилася до розпису маріупольських
фасадів. Ніколи не забуду той час. Своїх екскурсантів намагалася водити у
місця, які зберігають найбільше таємниць і цікавинок, де можна було загадати
бажання, чи побачити щось абсолютно незвичне. Сподіваюсь, моя закоханість у
місто передавалась і іншим... Не вистачає мені моїх екскурсій, тому намагаюсь
поступово відновлювати свою діяльність у Києві. Але вірю всім серцем у
найскорішу деокупацію мого  Маріуполя, що дозволить знову провести екскурсії
рідними вулицями...🙏💙💛

