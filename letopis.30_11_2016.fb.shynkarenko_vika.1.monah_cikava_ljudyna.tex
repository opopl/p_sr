% vim: keymap=russian-jcukenwin
%%beginhead 
 
%%file 30_11_2016.fb.shynkarenko_vika.1.monah_cikava_ljudyna
%%parent 30_11_2016
 
%%url https://www.facebook.com/vika.shynkarenko/posts/910043429096669
 
%%author Шинкаренко, Вікторія
%%author_id shynkarenko_vika
%%author_url 
 
%%tags 
%%title Цікава людина...
 
%%endhead 
 
\subsection{Цікава людина...}
\label{sec:30_11_2016.fb.shynkarenko_vika.1.monah_cikava_ljudyna}
\Purl{https://www.facebook.com/vika.shynkarenko/posts/910043429096669}
\ifcmt
	author_begin
   author_id shynkarenko_vika
	author_end
\fi

\index[cities.rus]{Воронеж!Россия!монастырь}
\index[cities.rus]{Пирятин!Украина!скорая помощь}

Цікава людина...

Завжди коли їду, підбираю голосуючих в дорозі. Сьогодні підвозила монаха. Не
запитала ім'я тому далі писатиму Монах. Він вже 7 років онкохворий, у нього
була мрія потрапити на святе джерело десь біля Воронежа.  Монах вважає що саме
воно йому допоможе. Монах назбирав 4600 грн., добрався до Воронежа, а там ще
три години на автобусі потрібно було добиратися. Але, гривні ніхто звичайно
брати не хотів. Він добу сидів на автостанції, поки ним не зацікавились
міліціонери і запитали чому він нікуди не їде. Монах знов дістав наші гроші і
сказав що ніхто за них везти не хоче, вони порадили звернутися до таксистів.

Про себе обурювався, чому у нас можна рублі поміняти а у них гривні ніхто не
хоче брати. Пояснив таксисту свою біду, той дав йому купу \enquote{папірців} в
яких Монах нічого не розумів. Зайшов в автобус, сказав водію візьміть скільки
вам потрібно, водій забрав майже всі гроші. Монах каже, мабуть таксист
обманув... 

\ifcmt
	pic https://scontent.fiev6-1.fna.fbcdn.net/v/t1.0-9/15220224_910043375763341_8239605633974316588_n.jpg?_nc_cat=1&ccb=2&_nc_sid=8bfeb9&_nc_ohc=JAQFnUVPzCwAX91MuVv&_nc_ht=scontent.fiev6-1.fna&oh=d7d21e3c0f3b51e2dba67120746d4320&oe=5FED48C0
	caption Монах, цікава людина\ldots
\fi

Добрався, \enquote{браття} як дізналися що він із Західної України косо на нього
дивилися але келію виділили. Їсти Монах ходив раз в день щоб не мозолити
\enquote{браттям} очі, боявся що поб'ють))) Побув там три дні і вирушив додому.
Сьогодні дев'ятий день!!!! як Монах добирається автостопом Україною додому, я
його довезла Житомир - Рівне. Підвозити когось чи ні то власна справа кожного
водія, я не про це. 

Десь біля Пирятина Монах голодний і холодний втратив свідомість, опритомнів
коли швидка приїхала, високий тиск і кров з носа. Перше що почув

\obeycr
- Гроші маєте?
- Ні
- Пишіть тоді розписку що до нас претензій не маєте!
- Претензій не маю, писати нічого не буду.
\restorecr

Людину не завезли в лікарню бо не було коштів при собі! 

Швидка Пирятина, вам привіт! 

Залишили і поїхали, байдуже що далі... Але Монах не жалівся, просто розповідав.
Я сама поцікавилась як добирався. Монах замерз, зима то і не дивно, зайшов в
магазин попросив окропу, не чаю не кави а просто гарячої води! Продавець
сказала що не дасть бо стаканчики платні, попросив, каже, в долоні
налити...Розповідав як просився на заправках погрітися і переночувати, багато
чого ще розповідав, але я не чула вже, просто уявляла як продавець наливає
кип'яток в долоні, і думала скільки може коштувати той стаканчик... Монах каже,
що мріє потрапити в свою келію і більше ніколи не виходити в люди! І я теж так
захотіла!..

- Ще Монах каже що всі хочуть змінити Державу але самі не змінюються,
черствіють і озлоблюються. І тут він таки правий! Довезла Монаха до виїзду на
Дубно, не знаю скільки він ще днів добиратиметься на Закарпаття, але побачите -
підвезіть! Він вам розповість про те, як 25 років тому в 19 років вирішив стати
монахом, як була проти мама комуністка, про свій монастир і господарство яке
там ведеться. Доречі, він живе в жіночому монастирі, бо жінкам службу правити
не можна. Завжди думала що в жіночий монастир вхід чоловікам заборонено,
виявляється тільки жінкам в чоловічому не місце!
