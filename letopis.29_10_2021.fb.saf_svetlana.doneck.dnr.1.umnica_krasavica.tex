% vim: keymap=russian-jcukenwin
%%beginhead 
 
%%file 29_10_2021.fb.saf_svetlana.doneck.dnr.1.umnica_krasavica
%%parent 29_10_2021
 
%%url https://www.facebook.com/leksiya58/posts/2878364902474224
 
%%author_id saf_svetlana.doneck.dnr
%%date 
 
%%tags deti,dnr,donbass,doneck,semja,skazka,zhizn
%%title Ладно, поясню! Про "Умница, красавица, принцесса"))
 
%%endhead 
 
\subsection{Ладно, поясню! Про \enquote{Умница, красавица, принцесса}))}
\label{sec:29_10_2021.fb.saf_svetlana.doneck.dnr.1.umnica_krasavica}
 
\Purl{https://www.facebook.com/leksiya58/posts/2878364902474224}
\ifcmt
 author_begin
   author_id saf_svetlana.doneck.dnr
 author_end
\fi

Ладно, поясню!

Про "Умница, красавица, принцесса")) 

В давние-давние времена, ну - когда не было мобильных телефонов, а интернет ещё
далеко не у всех, в большой и дружной советской стране, где совсем не страшно
было заболеть (не смертельной, конечно, болезнью), и уж совсем не страшно -  за
неделю до зарплаты сложиться с соседями на килограммовый торт с последней в
семейном бюджете трёшки рублей (мы вообще тогда ничего не боялись - можете
такое себе представить?..), вот тогда, в те бесстрашные времена, Господь Бог и
наградил меня - сыном, а через четыре года ещё и дочерью!

\ifcmt
  tab_begin cols=2

     pic https://scontent-frx5-1.xx.fbcdn.net/v/t39.30808-6/248510837_2878404562470258_7317738615626685148_n.jpg?_nc_cat=100&ccb=1-5&_nc_sid=8bfeb9&_nc_ohc=yUHrRPCPggEAX-eXBr4&_nc_ht=scontent-frx5-1.xx&oh=698eb77e053e2a8aaf6a7a5eea254b06&oe=61A50B29

     pic https://scontent-frx5-2.xx.fbcdn.net/v/t39.30808-6/248373859_2878404619136919_7628176218510030307_n.jpg?_nc_cat=109&ccb=1-5&_nc_sid=8bfeb9&_nc_ohc=DxP6-m1EN_AAX_3tjv_&_nc_ht=scontent-frx5-2.xx&oh=5e75faf646acf6ed043108ea89e348e6&oe=61A56CD2

  tab_end
\fi

И я читала им сказки. Сначала сыну. Но он научился читать в три с небольшим
хвостиком, и стал читать сам!

Потом читала сказки дочке. А когда укладывала спать, то не читала, а сочиняла.

Зачем же сочиняла?, -  спросите вы. Затем, что свет выключен! Как в темноте
читать, а?! 

Дочке, конечно, про гномиков, мышек-хвостиков, и... про принцесс! Ну, не про
машинки же, не про богатырские бои, и не о пиратах девочке рассказывать, вы
что! 

Принцессы у меня всегда получались не очень: то не красавица, то вообще дурочка
с переулочка. Не нарочно, нет. Просто своевольные такие, прёт, например, эдакая
сама себе на уме (или по дури) наперекор сюжету, то Кащея пожалеет, а он её
сожрёт, подлец такой!, то на грабли наступит, в огород соседский наведавшись -
а вот не ходи по огородам, коли принцесса!

Зато все они, все до одной!, получались добрые, не обидчивые:  посмеялась,
шишку от граблей погладив, да пошла себе дальше, покусала Кощея изнутри, он её
и выплюнул, да ещё и задружился с ней, превратился в Иванушку-и̶н̶т̶е̶р̶н̶е̶ш̶е̶н̶е̶л̶, ой!
- в Иванушку-дурачка, и женился на ней! И так любил, так любил эту дурочку (ну
а какая умная принцесса пошла б за Иванушку, который ещё вдруг однажды взял и
испил водицы из копытца?), что считал её самой красивой, самой умной, и вообще
- самой-самой! Да!

И так всегда и говорил ей: умница, красавица, принцесса моя, помоги слезть с
печи после жаркой ночи, да воды наноси, да траву покоси, и на стол накрой,
будем пировать, будем счастье семейное праздновать!

Ну, вот такие принцессы у меня в сказках сами по себе получались.

Потом, как постарше дочь стала, и сказки закончились, то случись обида какая,
или неудача, или двойка в школе вдруг.., так и поговорка у нас появилась.

Про умницу-красавицу-принцессу.

Как? Да вот так!) 

При обиде - "Не, ну ты смотри! Какие же не добрые бывают! Одни мы с тобой -
умницы, красавицы, принцессы! " - и сразу так смешно нам! И всё, и обида
забылась, или не такая большая уже.

Двойка?! "Ах, какая учительница не хорошая, это она сама не выучила урок.
Правда же?! А ты у нас - умница, красавица, принцесса!". Всё! И стыдно, и
хочется двойку исправить, и - смешно!

Дочка выросла. Но, когда возмущается чем-то, рассказывая, то как только мне
стоит сказать: "Да, доця, это кошмар! Одни мы с тобой... Кто?", так тут же
звучит тройным хором (я, дочь, и внучка):

— Умницы! Красавицы! Принцессы!

Иногда сквозь слёзы звучит. Но сразу легче становится, и мир вокруг светлее. 

Вот потому  @igg{fbicon.face.wink.tongue} 
