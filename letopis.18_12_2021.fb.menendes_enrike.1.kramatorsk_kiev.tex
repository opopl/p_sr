% vim: keymap=russian-jcukenwin
%%beginhead 
 
%%file 18_12_2021.fb.menendes_enrike.1.kramatorsk_kiev
%%parent 18_12_2021
 
%%url https://www.facebook.com/e.menendes/posts/6647731435269226
 
%%author_id menendes_enrike
%%date 
 
%%tags donbass,doneck,kiev,kramatorsk,ukraina
%%title Все спрашивают меня, не ошибся ли я, переехав в Краматорск?
 
%%endhead 
 
\subsection{Все спрашивают меня, не ошибся ли я, переехав в Краматорск?}
\label{sec:18_12_2021.fb.menendes_enrike.1.kramatorsk_kiev}
 
\Purl{https://www.facebook.com/e.menendes/posts/6647731435269226}
\ifcmt
 author_begin
   author_id menendes_enrike
 author_end
\fi

После длительного проживания в провинции, Киев выглядит монументально. В
Краматорске у меня всего по одному: я хожу в один супермаркет (ну ладно, в
два), в одно кафе, в одну пиццерию, в одном месте покупаю овощи, в одном мясо и
в одном вино. 

Такое разнообразие и насыщенность всего в Киеве поначалу сбивает с толку, даже
при том, что я житель больших городов со стажем - и в Донецке, и в том же Киеве
мне грех было жаловаться.

Сейчас, во время войны на Донбассе я испытываю зависть к мирным и беспечным
городам, к их чисто бытовой зацикленности, к их богатой интеллектуальной жизни.

Все спрашивают меня, не ошибся ли я, переехав в Краматорск? Не хочу ли
вернуться? Тут жизнь, возможности для заработка, много эфиров.

Нет, я уверен в своем решении. Я не хочу быть голосом Донбасса на расстоянии.
Хочу быть с людьми здесь. И дико хочу вернуть мирный и процветающий Донецк.

\ii{18_12_2021.fb.menendes_enrike.1.kramatorsk_kiev.cmt}
