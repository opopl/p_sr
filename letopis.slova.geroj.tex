% vim: keymap=russian-jcukenwin
%%beginhead 
 
%%file slova.geroj
%%parent slova
 
%%url 
 
%%author 
%%author_id 
%%author_url 
 
%%tags 
%%title 
 
%%endhead 
\chapter{Герой}
\label{sec:slova.geroj}

блин, а есь хотя бы одно предсказание, где мы будем жить вечно, радоватся,
плясать, бухать и балдеть, после чего, по синьке, завоюем галактику, станем
\emph{Героями} и о нас будут как о богах слагать песни и легенды!!?? Почему
всегда какие то страсти!!??)) Я к примеру за такое развитие дальнейших
событий!!))
\citComment{Шура Иванов},
\citTitle{Монах предсказал ужасное будущее для всего мира}, ZOLEF, zen.yandex.ru, 07.06.2021

\enquote{Слава героям! Слава Героям! Честь и достоинство не продаются за газ и
нефть!} - пишет пользователь Jais McKlon. \enquote{Какая ваша цена,
\enquote{паны} из УЕФА? может рубль, может два? Слава Украине!!!! Героям
Слава!!!!} - возмущается Антин Трубчик. \enquote{Слава Украине! Героям слава!
Украина защита для Европы и цивилизации мира! Эти слова - святые слова!} -
считает Дмитрий Конопацкий.  \enquote{Слава Украине! Слава Героям! каждый из
нас, кто от имени Украины борется за ее имя ГЕРОЙ, и не имеет значения или
политик, музыкант, футболист, или военный}, - высказался Barry Doran.
\enquote{Почему бы не требовать изменений в командной форме России и не снять
ее с чемпионата всей командой внутри? сколько бы вы взяли за ЭТО? мы начнем
финансирование толпой. Слава Украине! Слава героям!} - написал Dmitry
Klimovsky,
\citTitle{Украинцы раскритиковали требование УЕФА убрать с формы сборной слова Героям слава!}, Юлия Супрун, strana.ua, 10.06.2021

%%%cit
%%%cit_head
%%%cit_pic
%%%cit_text
Мы -- нация \emph{Героев}, мы -- россияне! Нам управлять этой огромной громадой, нашей
великой страной, и нам в ней жить справедливо и радостно, на зависть остальному
Миру, который не желает разделять наших замечательных жизненных ценностей.
Виват, Россия! Виват, Россияне! Бог с нами! А это самое главное
%%%cit_comment
%%%cit_title
\citTitle{Слава России! Мы на вершине могущества и спускаться более не намерены}, 
Алексей Наст, zen.yandex.ru, 12.06.2021
%%%endcit

%%%cit
%%%cit_head
%%%cit_pic

\ifcmt
  tab_begin cols=2
	width 0.3

     pic https://strana.ua/img/forall/u/0/36/rue4fdd579808.jpg

     pic https://strana.ua/img/forall/u/0/36/2021-06-30_12h47_06.png

  tab_end
\fi

%%%cit_text
В эфире программы \enquote{Великий футбол} \emph{герой матча} Довбик признался, что до
конца не осознает все, что произошло на поле и постепенно приходит в себя. 
\enquote{Наверное, это какая-то судьба. У меня просто нет слов. Такое стечение
обстоятельств, что надо было выходить и помогать партнерам, ждать свой шанс. Я
его все же дождался}, - поделился впечатлениями спортсмен.  Довбик также
вспомнил напутственные слова своего тренера Андрея Шевченко, который сказал,
чтобы Артем всегда находился в штрафной площадке и ждал шанс, старался
цепляться за мячи.  В Киеве у огромных экранов ТРЦ \enquote{Gulliver} фанаты были вне
себя от счастья
%%%cit_comment
%%%cit_title
\citTitle{Артем Довбик - что известно про футболиста, который вывел Украину в четверть финала Евро}, 
Анна Копытько, strana.ua, 30.06.2021
%%%endcit

%%%cit
%%%cit_head
%%%cit_pic
%%%cit_text
Річинський вказував на українські родинні корені Достоєвського по лінії батька.
Говорив і про те, що Достоєвський у «своїх писаннях навіть виявляє недобре
володіння російською мовою». До речі, це твердження, як уже говорилося,
небезпідставне. У творах Достоєвського можемо зустріти слова, які мали
українське походження. Також у цих творах можна побачити й граматичні форми,
які не дуже вписуються в рамки російської літературної мови. Достоєвський,
схоже, розумів, що його російська мова не дуже правильна. Але це по-своєму
виправдовував. Коли одному з \emph{героїв} роману «Біси» закидають те, що він
неправильно говорить російською мовою, той відповідає, що завжди так розмовляв.
Певно, так само міг себе виправдати й Достоєвський
%%%cit_comment
%%%cit_title
\citTitle{Арсен Річинський про українські аспекти творчості Федора Достоєвського (1821–1881)}, 
, www.radiosvoboda.org, 25.07.2021
%%%endcit

