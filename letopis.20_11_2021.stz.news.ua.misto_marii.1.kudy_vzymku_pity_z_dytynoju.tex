% vim: keymap=russian-jcukenwin
%%beginhead 
 
%%file 20_11_2021.stz.news.ua.misto_marii.1.kudy_vzymku_pity_z_dytynoju
%%parent 20_11_2021
 
%%url https://mistomariupol.com.ua/uk/kudy-vzymku-pity-z-dytynoyu-v-mariupoli
 
%%author_id news.ua.misto_marii
%%date 
 
%%tags 
%%title Куди взимку піти з дитиною
 
%%endhead 
 
\subsection{Куди взимку піти з дитиною}
\label{sec:20_11_2021.stz.news.ua.misto_marii.1.kudy_vzymku_pity_z_dytynoju}
 
\Purl{https://mistomariupol.com.ua/uk/kudy-vzymku-pity-z-dytynoyu-v-mariupoli}
\ifcmt
 author_begin
   author_id news.ua.misto_marii
 author_end
\fi

\begin{quote}
\em
20 листопада святкують Всесвітній день дитини. Свято було запропоновано у
1954 році Генеральною Асамблеєю ООН, і вперше його відзначили двома роками
пізніше. А ми підготували добірку місць у Маріуполі, які варто відвідати з
дітьми у зимовий період, бо холод – не привід залишатися удома!
\end{quote}

\subsubsection{Простір для творчості \enquote{Аби Що}}

Створений двома неймовірними дівчатами Машею і Дашею, простір \enquote{Аби Що} уже
полюбився місцевим жителям. Майже щоденно тут проводять майстер класи та творчі
зустрічі для відвідувачів усіх вікових категорій та кола інтересів. Можна
обрати собі до вподоби з величезного списку активностей: розпис одягу і сумок,
комбінування сухоцвітів, стрінг-арт, флюїд арт, флораріум, малювання
акварелями, масляний живопис, створення домашнього декору до свят. Розклад
оновлюється щотижня!

Адреса: Торгова, 24

Телефон: 098 979 56 49

Більше деталей – в інстаграм.

\ii{20_11_2021.stz.news.ua.misto_marii.1.kudy_vzymku_pity_z_dytynoju.pic.1.aby_scho}

\subsubsection{100 мов дитинства}

Ще один центр дитячої творчості, заснований ведучою артисткою студії свят \enquote{31
червня}. У кожній дитині живе митець, тож головна задача викладачів студії –
підтримати будь-які починання та інтереси, без заборон і стандартних обмежень,
підготуватися для школи, навчитися комунікувати з іншими дітками. Це унікальний
центр раннього розвитку, де не тільки дітки, але й батьки зможуть віднайти
безліч цікавого та корисного.

Адреса: вулиця Казанцева, 14а

Більше інформації – на сторінці у фейсбук - \url{https://www.facebook.com/detimrpl}.

\ifcmt
  ig https://i2.paste.pics/PI0F0.png?trs=1142e84a8812893e619f828af22a1d084584f26ffb97dd2bb11c85495ee994c5
  @wrap center
  @width 0.9
\fi

\ii{20_11_2021.stz.news.ua.misto_marii.1.kudy_vzymku_pity_z_dytynoju.pic.2.detimrpl}

\subsubsection{Дитяча зона MKIDS}

По-справжньому весела і захоплююча дитяча ігрова кімната у торгових центрах
\enquote{МЦентр} і \enquote{Браво}. Поки батьки роблять покупки, діти зможуть провести час із
користю та хорошим настроєм. Тут можна пострибати на батуті, гірках, побігати
лабіринтом, а можна приєднатися до цікавих майстеркласів. За дітками ретельно
слідкують няні, за відгуками відвідувачів це те місце, в яке хочеться
повертатися знов і знов.

Більше інформації – на сторінці в інстаграм - \url{https://www.instagram.com/mkidsmariupol}.

\ii{20_11_2021.stz.news.ua.misto_marii.1.kudy_vzymku_pity_z_dytynoju.pic.3}

\subsubsection{FLY Kids}

Абсолютно неймовірний парк розваг – це місце, де свято кожен день і для радощів
не потрібний особливий привід! Десятки атракціонів, майстеркласи для дітей,
анімація, смачна їжа – все, що потрібно для комфорту і крутих вражень. На
території є батутна арена – стрибай досхочу, канатний парк, де можна відчути
себе підкорювачем справжніх джунглях, Х-Box, сухий басейн, тарзанка, лабіринт,
аерохокей і ще безліч усього! Крім того, є можливість незабутньо відсвяткувати
дитячий день народження чи будь-яке тематичне свято – для цього Fly Kids
пропонує кілька варіантів організації під ключ. І можна бути впевненим, що
турбуватися немає про що.

Більше інформації можна знайти на сайті.

Адреса: проспект Металургів, 125 (ТРЦ \enquote{Амстор})

\ii{20_11_2021.stz.news.ua.misto_marii.1.kudy_vzymku_pity_z_dytynoju.pic.4.fly_kids}

\subsubsection{Smile Park}

Ще один крутий парк розваг! Провести час весело і з користю, пограти в ігри,
відсвяткувати День народження – тут можна все! Справжня територія щастя для
дітей. У Smile Park на постійній основі діє приємна акція – для іменинників у
їхній День народження вхід безкоштовний! А ще тут є спеціальні пропозиції для
школярів різного віку – для молодших класів пропонуються квести, ігри на знання
мови і географії, активні рухливі конкурси. Для тих кому 10-14 років – вражаючі
хімічні досліди, командні естафети, вікторини, логічні задачки.

Більше інформації можна знайти на сайті.

Адреса: Запорізьке шосе, 2 (ТРЦ \enquote{Порт Сіті})

Телефон: 095 116 9216

\ii{20_11_2021.stz.news.ua.misto_marii.1.kudy_vzymku_pity_z_dytynoju.pic.5.smilepark}

\subsubsection{Ляльковий театр}

Єдиний у Маріуполі професійний ляльковий театр, який існує уже понад 20 років.
У театрі можна побачити як улюблені класичні вистави на кшталт \enquote{Лисиця та
Ведмідь}, так і більш сучасні – \enquote{Лев, чаклунка і чарівна шафа}.

Більше інформації та бронь квитків на сторінці у фейсбук - \url{https://www.facebook.com/mariupolteatrkukol} - або за телефоном 098
308 45 95.

Адреса: вул. Харлампіївська, 17/25

\ii{20_11_2021.stz.news.ua.misto_marii.1.kudy_vzymku_pity_z_dytynoju.pic.6.teatr_kukol}

\subsubsection{Ice center}

Як щодо активного відпочинку? На ковзанці на Лівому березі систематично
проводяться тренування майбутніх чемпіонів з хокею та фігурного катання. Для
бажаючих гарно провести час усією родиною чи спробувати свої сили на льоду,
ковзанка працює за окремим графіком, який можна перевіряти на сайті. На
території є також смачнющий фудкорт – те що треба для невеличкої перерви між
заїздами.

Тривалість заїзду – 45 хв, вартість – від 80 грн для дітей і від 100 грн для
дорослих.

Адреса: проспект Перемоги, 75/45

Телефон: (098) 792 37 20

\ii{20_11_2021.stz.news.ua.misto_marii.1.kudy_vzymku_pity_z_dytynoju.pic.7.mariupol_ice_center}

\subsubsection{Центр \enquote{Аляска}}

Узимку тут все перетворюється на справжній гірськолижний курорт. Хоча сніжні
зими у південному теплому Маріуполі – рідкість, на \enquote{Алясці} сніг, хоча і
штучний, цього року з'явився у листопаді. Працюватиме гірка буде за мінусової
температури.

Тут можна орендувати обладнання для занять сноубордингом, катання на лижах і
навіть взяти уроки у професійних інструкторів. Для найменших відвідувачів є
окрема траса, де катаються на надувних санчатах – так званий \enquote{сноутюбінг}.
Весело, безпечно і, що важливо в епоху пандемії – на свіжому повітрі.

Година катання на лижах для дітей – 50 грн, прокат дитячого обладнання – 70
грн.

Більше інформації про ціни та розклад на сайті - alaska.in.ua - та за номером 096 901 9191.

Адреса: вулиця Гранітна, 121.

\ii{20_11_2021.stz.news.ua.misto_marii.1.kudy_vzymku_pity_z_dytynoju.pic.8.alaska_center_mariupol}

\subsubsection{Квест-проєкт \enquote{Ізоляція}}

Проходження квест-кімнат – це класна пригода, відпрацювання навичок командної
роботи, розвиток логічного мислення та образної уяви. Задача учасників – за
певний відрізок часу розкрити всі таємниці та загадки, знайти та розгадати
наявні шифри, знайти ключі, максимально дослідивши всі об'єкти. У ТРЦ
\enquote{Порт-сіті} працює квест-проект \enquote{Ізоляція}, де можна обрати квест-кімнату за
рівнем складності загадок і віком. Для дітей рекомендують \enquote{Аліса в Країні
Чудес}. Правила розраховані на команди від двох учасників.

Адреса: Запорізьке шосе, 2 (ТРЦ \enquote{Порт Сіті})

Більше інформації на сайті - \url{https://www.izolyatsiya.com.ua/mariupol}.

\ii{20_11_2021.stz.news.ua.misto_marii.1.kudy_vzymku_pity_z_dytynoju.pic.9.izolyatsiya}

\subsubsection{Водонапірна Вежа}

Гіди КТЦ \enquote{Вежа} із задоволення проведуть захоплюючу екскурсію Водонапірною
Вежею для наймолодших відвідувачів та відкриють усі секрети старого Маріуполя
та найзнаменитішої його споруди, тож завітайте на пізнавальні цікаві екскурсії
усією родиною щодня з 10 до 19 години.

Адреса: вулиця Архітектора Нільсена, 34

Деталі за номером: 097 618 40 85

\ii{20_11_2021.stz.news.ua.misto_marii.1.kudy_vzymku_pity_z_dytynoju.pic.10}

\begin{quote}
\bfseries
*Перелічені заклади працюють згідно карантинних вимог: відвідування можливо за
наявності документу про проходження повного курсу вакцинації (паперовий зразок
або електронний у додатку \enquote{ДІЯ}); негативного результату тестування на COVID-19
методом полімеразної ланцюгової реакції або експрес-тесту на визначення
антигена коронавірусу SARS-CoV-2, яке проведене не більш як за 72 години до
дати відвідування ТРЦ; документу, що підтверджує одужання особи.	
\end{quote}

Фото на головній – mariupol.smilepark.ua
