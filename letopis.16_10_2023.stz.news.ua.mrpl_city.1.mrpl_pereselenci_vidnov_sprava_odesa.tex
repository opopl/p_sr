% vim: keymap=russian-jcukenwin
%%beginhead 
 
%%file 16_10_2023.stz.news.ua.mrpl_city.1.mrpl_pereselenci_vidnov_sprava_odesa
%%parent 16_10_2023
 
%%url https://mrpl.city/news/view/yak-mariupolskim-pidpriemtsyam-vdalosya-vidnoviti-svoyu-spravu-v-odesi-istoriya-nezlamnih-pereselentsiv
 
%%author_id demidko_olga.mariupol,news.ua.mrpl_city
%%date 
 
%%tags 
%%title Як маріупольським підприємцям вдалося відновити свою справу в Одесі – історія незламних переселенців
 
%%endhead 
 
\subsection{Як маріупольським підприємцям вдалося відновити свою справу в Одесі – історія незламних переселенців}
\label{sec:16_10_2023.stz.news.ua.mrpl_city.1.mrpl_pereselenci_vidnov_sprava_odesa}
 
\Purl{https://mrpl.city/news/view/yak-mariupolskim-pidpriemtsyam-vdalosya-vidnoviti-svoyu-spravu-v-odesi-istoriya-nezlamnih-pereselentsiv}
\ifcmt
 author_begin
   author_id demidko_olga.mariupol,news.ua.mrpl_city
 author_end
\fi

\begin{quote}
\bfseries
Андрій та Катерина Єрмоленки – сім'я переселенців з Маріуполя, які стали
амбасадорами \enquote{Спільноти Відновлення} та лауреатами відзнаки \enquote{Незламні зі
сходу}. У рідному місті вони мали власний бізнес із декорування свят, проте
через повномасштабне вторгнення втратили все.
\end{quote}

Подружжя \href{https://v-variant.com.ua/article/pidpryiemtsi-z-mariupolia-v-odesi}{скористалося}%
\footnote{\enquote{Спільнота Відновлення}: як підприємці з Маріуполя відкрили сімейну майстерню неонових виробів у Одесі, %
Інга Павлій, Східний Варіант, 06.10.2023, \par%
\url{https://v-variant.com.ua/article/pidpryiemtsi-z-mariupolia-v-odesi}
}
унікальним шансом відновити підприємницьку справу, і вже
понад рік їхня сімейна майстерня неонових виробів \href{https://www.instagram.com/ua_wonder}{WONDER}%
\footnote{\url{https://www.instagram.com/ua_wonder}}
працює в Одесі.

\textbf{Читайте також:} \emph{ВПО з Донбасу пропонують безкоштовне житло та роботу на Київщині - деталі}%
\footnote{ВПО з Донбасу пропонують безкоштовне житло та роботу на Київщині - деталі, Ольга Демідко, mrpl.city, 19.09.2023, \par%
\url{https://mrpl.city/news/view/go-ta-visim-dij-ta-proponue-vpo-z-donbasu-zhitlo-i-pratsevlashtuvannya-minus-detali}%
}

\enquote{Спільнота Відновлення} — засноване Східним Варіантом ком'ю\hyp{}ніті незламних
жителів сходу, які заново поставили на рейки свою діяльність у сфері бізнесу чи
в громадському секторі.

\ii{16_10_2023.stz.news.ua.mrpl_city.1.mrpl_pereselenci_vidnov_sprava_odesa.pic.1}

\begin{quote}
\em\enquote{Коли приходили гості, коли забирали кульки, вони були щасливими. Це дуже
позитивна професія. Там ніколи не буває негативу. І ти — людина свята! Тому що,
коли ти в цій професії, ти повинен бути завжди позитивним і з усмішкою. Адже
люди замовляють у тебе свято}, 
\end{quote}
— розповідає Катерина.

Незабаром Андрій та Катерина зрозуміли, що потрібно шукати окреме приміщення та
відкривати майстерню повітряних кульок. Водночас родина планувала розпочати ще
одну справу — виробництво неонових вивісок. Ця ідея завжди цікавила Катерину та
здавалася їй дуже привабливою. Єдине — вона не знала, як це працює. Та коли
Андрій пояснив дружині, що насправді всьому можна навчитися самостійно, сумніви
остаточно зникли.

\ii{16_10_2023.stz.news.ua.mrpl_city.1.mrpl_pereselenci_vidnov_sprava_odesa.pic.2}

\begin{quote}
\em\enquote{Ми планували так, що новим виробництвом займатиметься Андрій, а я продовжу
свій бізнес декорування. Для нової ідеї ми заповнили грантову заявку. 20 лютого
мені прийшло на пошту повідомлення, що її погодили. А через 4 дні розпочалася
повномасштабна війна},
\end{quote}

- поділилася дівчина.

Першого березня в будинок, який був поруч із будинком батьків Катерини,
прилетів снаряд. Родина дуже злякалася, не знала, що робити. Вони вирішили
відправитися у найближче бомбосховище. Згодом росіяни зайняли позиції біля
бомбосховища. Через це виходити назовні майже не було можливості — поруч
постійно йшли важкі вуличні бої. Вийти на поверхню змогли лише після того, як
окупанти змінили позицію. Та виникла нова проблема: Андрій та Катерина не
знали, де їхні батьки та що з ними. Тож перед тим, як покинути майже окуповане
місто, розпочали їхні пошуки. На щастя, і батьки Андрія, і батьки Катерини були
живі та неушкоджені. Тож родина почала виїжджати в бік підконтрольних територій
України.

\textbf{Читайте також:} Чи позбавлять переселенця виплат, якщо він відкриє бізнес - роз'яснення
\footnote{Чи позбавлять переселенця виплат, якщо він відкриє бізнес - роз'яснення, Еліна Прокопчук, mrpl.city, 05.10.2023, \par%
\url{https://mrpl.city/news/view/chi-pozbavlyat-pereselentsya-viplat-yakshho-vin-vidkrie-biznes-rozyasnennya}
}

У селищі Нікополь, що поруч із Маріуполем, сімʼю змушували пройти фільтраційні
заходи. Та за допомогою одного перевізника вони змогли виїхати одразу до
Бердянська, оминувши принизливі \enquote{перевірки} росіян. Із Запоріжжя родина рушила
до західних областей. А потім отримала повідомлення від грантодавця, який питав
у підприємців, чи збираються вони використати свій грант. Це був шанс для
Андрія та Катерини.

\begin{quote}
\em\enquote{Це був такий дзвінок, завдяки якому ми залишилися підприємцями. Тож ми поїхали
до Одеси, де проживали наші родичі. Ми вирішили, що треба триматися разом,
великою сім'єю. Трапилося так, що довелося переписати бізнес-план, адже ціни та
ситуація на ринку змінилася за кілька місяців. Після всіх підготовчих робіт ми
змогли запустити виготовлення неонових виробів}, 
\end{quote}

— розповіла Катерина.

Потроху родина навчалася новій справі, запускала рекламу, приймала перші
замовлення. Андрій відповідає безпосередньо за виготовлення продукції, Катерина
— за замовлення та соціальні мережі. Двоюрідний брат дівчини працює над
контекстною рекламою. Батьки подружжя теж іноді беруть участь у розвантаженні
чи завантаженні обладнання. Тож виробництво WONDER — це дійсно сімейна
майстерня, де робота знайдеться кожному й кожній.

Наразі серед асортименту бренда: неонові вироби, дерев'яні аксесуари,
світильники-нічники, магніти, підвіски в машину, елементи домашнього декору.

\begin{quote}
\em\enquote{Зараз замовлення йдуть переважно на неонові вивіски, що і є нашим пріоритетом.
Це вивіски для кав'ярень, салонів краси, крамниць. Уже почалися в нас великі
замовлення. І за цей час роботи ми змогли вийти на достойний рівень роботи}, 
\end{quote}
— каже Катерина.

\ii{16_10_2023.stz.news.ua.mrpl_city.1.mrpl_pereselenci_vidnov_sprava_odesa.pic.3}

Для маріупольців було надважливо відновити свою діяльність навіть в умовах
повномасштабної війни, адже тільки так можна напрацьовувати втрачене та
освоювати нові ніші. У Маріуполі Катерина і Андрій втратили власний бізнес,
домівку та все минуле життя. Тепер треба починати спочатку.

\textbf{Читайте також:} \emph{\enquote{Існуть перевізники, які заробляють на переселенцях} - як люди будують бізнес на війні}%
\footnote{\enquote{Існуть перевізники, які заробляють на переселенцях} - як люди будують бізнес на війні, Ольга Демідко, %
mrpl.city, 08.10.2023, \par%
\url{https://mrpl.city/news/view/isnut-perevizniki-yaki-zaroblyayut-na-pereselentsyah-yak-lyudi-buduyut-biznes-na-vijni}
}

Катерина каже, що раніше не було міста кращого, ніж Маріуполь. Одеса хоч і
відрізняється від їхньої домівки, та тут, за словами підприємиці, гарні та
доброзичливі люди. А про минуле життя нагадує лише море. Тільки не Азовське, а
Чорне.

Молоде подружжя буде готове розглянути повернення до Маріуполя після його
деокупації. Та для цього, кажуть підприємці, потрібно бачити перспективу міста
як для життя, так і для бізнесу. Повинні бути умови для підприємців, а також
можливості логістики та передусім безпеки.

Маріупольчанка також бажає повернутися до бізнесу з декорування свят. Та це
буде вже після війни. Дівчина вважає, що зараз для українців важкі часи, і на
свято не завжди є час:

\begin{quote}
\em\enquote{Я не можу створювати радість і свято, розуміючи, що хтось зараз на фронті,
хтось зараз страждає. Це не дуже справедливо, мені здається. Краще зараз жити
тихенько, без помпезних свят. Свято може влаштовувати лише той, у кого серце
наповнене радістю. А в мене воно наповнене трагедією}.
\end{quote}

\ii{16_10_2023.stz.news.ua.mrpl_city.1.mrpl_pereselenci_vidnov_sprava_odesa.pic.4}

Раніше розповідали, що \href{https://mrpl.city/news/view/pereselentsyam-u-troh-oblastyah-viplatyat-koshti-na-orendu-zhitla-podrobitsi}{%
переселенцям з трьох областей}%
\footnote{Переселенцям у трьох областях виплатять кошти на оренду житла - подробиці, Еліна Прокопчук, mrpl.city, 01.10.2023, \par%
\url{https://mrpl.city/news/view/pereselentsyam-u-troh-oblastyah-viplatyat-koshti-na-orendu-zhitla-podrobitsi}}
виплатять кошти на оренду житла.  

Фото: з відкритих джерел
