% vim: keymap=russian-jcukenwin
%%beginhead 
 
%%file 11_11_2021.fb.fb_group.story_kiev_ua.3.stiljagi
%%parent 11_11_2021
 
%%url https://www.facebook.com/groups/story.kiev.ua/posts/1795432683986858
 
%%author_id fb_group.story_kiev_ua,bubnov_jurij
%%date 
 
%%tags kiev,komsomol,molodezh,sssr,stiljagi
%%title Стиляги
 
%%endhead 
 
\subsection{Стиляги}
\label{sec:11_11_2021.fb.fb_group.story_kiev_ua.3.stiljagi}
 
\Purl{https://www.facebook.com/groups/story.kiev.ua/posts/1795432683986858}
\ifcmt
 author_begin
   author_id fb_group.story_kiev_ua,bubnov_jurij
 author_end
\fi

Совсем маленькая история, даже не история, а так... Речь пойдет о «стилягах». С
середины 50-х и до начала 60-х, в Советском Союзе передовой отряд
Коммунистической партии – Комсомол – начал вести непримиримую борьбу с так
называемыми «стилягами». В очень большой степени коснулось это и Киева. 

\begin{multicols}{2} % {
\setlength{\parindent}{0pt}

Со слов
очевидцев, а в их число вхожу и я, на улицах города, чаще всего на Крещатке,
можно было видеть комсомольских дружинников – группу молодых людей в необычайно
широких брюках, этак сантиметров 40-45, внимательно всматривающихся во что
одеты прогуливающиеся юноши, особенно обращая внимание на ширину их брюк.

\ii{11_11_2021.fb.fb_group.story_kiev_ua.3.stiljagi.pic.1}
\ii{11_11_2021.fb.fb_group.story_kiev_ua.3.stiljagi.pic.2}
\ii{11_11_2021.fb.fb_group.story_kiev_ua.3.stiljagi.pic.2.cmt}

\end{multicols} % }

Следует сказать, в те времена модными были брюки «дудочки», шириной 16-18 см,
которые действительно с трудом можно было натянуть на ноги. Ходить в них еще
можно было, но сидеть... Поверьте, сам испытал. Тоже самое касалось и длины
пальто. Длина пальто должна была быть не выше щиколоток, в противном случае, ты
уже был почти стиляга. Нельзя было скрыться от зоркого взора дружинников и тем
ребятам, у которых на голове красовался т. н. «кок» - жалкое подобие прически
Элвиса Пресли. В этом случае, на работе, в техникуме или институте ты уже
попадал в стенгазету под названием «Окно сатиры». По некоторым свидетельствам,
были случаи, кода дружинники хватали такого «стилягу» и ножницами разрезали ему
брюки от щиколотки до колена. В других случаях, они же, выстригали механической
машинкой продольную полосу на голове парня, носившего «кок».

Пройдет немного времени, брюки «дудочки» уйдут в прошлое, на смену им придет
«клеш» или, как их еще называли «колокола». И что же вы думаете? Непримиримые
борцы со «стилягами», быстро сориентировавшись, оденут «дудочки» и начнут
бороться с теми, кто носит «клеш». Вся жизнь – борьба.

К чему все это я. Как-то мне попался на просторах интернета (замечательное
выражение, всегда выручает) вот этот рисунок-карикатура на стилягу тех времен
Пятигорского В., который учился в нашей, 91-й школе на Чеховском переулке,
потом работал в «Гравючас» на Евбазе, в доме быта на углу Дмитриевской и
Чкалова (см. фото 1957 г.) и с которым я, как и многие мои соученики, был
«шапочно» знаком. Вообще-то, парня на рисунке изуродовали – на самом деле, он
был симпатичным и нравился девушкам. Отличался от многих тем, что носил - ужас!
- темные очки (первый признак шпиона), одевал узкие брюки и яркие цветные, как
говорили тогда «гавайские» рубашки. Таким образом, борцы с «чуждыми нам
элементами», сами исчезнув в глубине лет, прославили стилягу В. Пятигорского и
пронесли его образ через многие десятилетия, благодаря интернету, о чем сам
Пятигорский мог только мечтать, если бы знал тогда, что такое интернет.

\ii{11_11_2021.fb.fb_group.story_kiev_ua.3.stiljagi.cmt}
