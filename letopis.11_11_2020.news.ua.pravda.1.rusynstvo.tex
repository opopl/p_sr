% vim: keymap=russian-jcukenwin
%%beginhead 
 
%%file 11_11_2020.news.ua.pravda.1.rusynstvo
%%parent 11_11_2020
 
%%url https://www.pravda.com.ua/articles/2020/11/11/7273111/
%%author 
%%tags 
%%title 
 
%%endhead 

\subsection{Русинська карта. Як ми шукали руку Москви і сепаратистів на Закарпатті ─ репортаж із Ужгорода}

\Purl{https://www.pravda.com.ua/articles/2020/11/11/7273111/}
\Pauthor{Тараненко, Андрій}
\index{Русь!Закарпаття}

\ifcmt
pic https://img.pravda.com/images/doc/e/0/e0a0912-rusyny.jpg
\fi

"Українство на Закарпатті ─ це вірус, що потрапив сюди наприкінці 19 століття з
Галичини і заразив русинів", ─ каже мій випадковий попутник у потязі Львів ─
Ужгород.

Усі свої хитромудрі алегорії він чомусь конструює на термінах з вірусології.

Це не дивує ─ за останній рік пандемія багатьох перетворила на стихійних
вірусологів.

Дивує хіба що категоричність "адепту русинського радикалізму", як він себе
презентує.

За всеукраїнським переписом 2001 року на Закарпатті проживало близько 10 тисяч
русинів. Якщо розглядати сучасне "підкарпатське русинство" не в етнографічному,
а в політичному сенсі, воно нагадує казан с булькотючою юшкою. Зі смаком то
українського борщу, то російських щів, то угорського гуляша, то чеської
часникової полівки.

Хто і навіщо підкладає дрова у багаття під тим казаном?

Чи насправді є ризик, що на українському західному кордоні може виникнути ще
одна так звана "народна республіка" під парасолькою Москви?

Чому зводити русинське питання до недолугого сепаратизму – це ще одна гілка
хмизу у вогнище?

На ці питання багато відповідей, і усі вони суперечать одна одній.

"Українська правда" вирушила на Закарпаття, щоб з’ясувати, що таке "русинська
карта", яку в прикордонному регіоні України періодично розігрують Чехія,
Угорщина і Росія.

\subsubsection{Етнічна вірусологія}

Потяг робить петлю навколо села Опорець, щоб згодом увійти в Бескидський
тунель, що розмежовує Львівщину та Закарпаття.

─ Яка мета будь-якого вірусу? Створювати свої копії, ─ продовжує свої
вірусологічно-геополітичні розвідки мій співрозмовник. ─ Вірус починає
передаватися у спадок, як звичайна мутація ДНК. Тому, називаючи себе
підкарпатським русином, чітко усвідомлюю, що насправді я етнічний мутант.

Я з цікавістю вивчаю "етнічного мутанта". Йому років 25, худий, з дещо нудним
обличчям.

Правильна мова одразу видає в ньому філолога. Здається, закарпатські "фуй!"
("погано") та "фоссо!" ("дуже добре") він додає виключно, щоб потішити чужинця.

Хлопець, що сидить навпроти мене, називає себе сучасним "радикальним ідейним
русином". На цьому місці я насилу заспокоюю себе, щоб не запанікувати ─ ось
він, живий сепаратист, заколот, зрада. Намагаюсь стриматися, щоб терміново не
телефонувати на "гарячу лінію" СБУ. 

Правда, його "ідейність" зводиться до того, що він дуже не любить
праворадикальних українських націоналістів з організації "Карпатська Січ". Та
ратоборствує у соцмережах за визнання підкарпатських русинів окремим етносом, а
так звану "русинську бесіду" ─ окремою мовою.

А ще він мріє перекласти русинською "Ілюзії" Річарда Баха.

\ifcmt
img_begin 
    url https://img.pravda.com/images/doc/7/5/750d57a-dscf1068.jpg
    caption На полицях книгарні "Файні книгы", заснованої 30 років тому, можна знайти як сучасні русинськомовні видання, так і перевидані книги, зокрема, "Грамматику руського языка" Івана Гарайди або "Книгу варенія для сельских карпаторусских женщин" Анни Микити. всі Фото: Євген Кудрявцев.
    width 0.7
img_end
\fi

Згодом, коли потяг вже починає гальмувати перед Мукачевом, філолог-вірусолог
застібає куртку і раптом каже: "А, може, все навпаки, і саме русинство є
вірусом?".

Я витріщився на нього. Він ловить погляд і пояснює.

─ Якщо я прохолов або перенервував, в мене на губі може виникнути герпес. Це не
означає, що я недавно десь підчепив герпес. Ні ─ весь цей час вірус жив у моєму
організмі і заявив про себе, коли імунна система дала тимчасовий збій.

Я до чого це кажу? Імунна система колективного організму під назвою Україна
збоїть вже багато років. А тому не дивно, що вірус русинства раз у раз проявляє
себе на Закарпатті.

Мій попутник сходить, а я, дещо спантеличений, їду далі, до Ужгорода.

\subsubsection{Мельхіседек і русинська ядерна зброя}

Про те, що закарпатські русини вважають себе окремим народом і навіть
претендують на автономну республіку, газета The Washington Post писала ще в
1995 році.

У статті йшлося про такий собі Домініон Мельхіседека, віртуальну державу, яку
згодом звинуватили у причетності до банківських махінацій.

Ця ніким не визнана держава нібито оголошувала своєю територією ділянки землі
по всьому світу. Починаючи від французького атола Кліппертон і закінчуючи
Закарпаттям, яке в документах Домініона називалося "Підкарпатська Русь" або
"Рутенія".

Марк Педлі, творець Домініона Мельхіседека, навіть погрожував Франції ядерною
зброєю, яка нібито залишилася на Закарпатті після Радянського Союзу і
знаходилася на той момент у розпорядженні русинів.

Все це нагадувало гротескну комедію з ухилом у повну нісенітницю. А проблеми
закарпатських русинів були представлені у настільки безглуздому контексті, що
примушували згадати голлівудські бойовики на кшталт "Червоної спеки" з
Арнольдом Шварценеггером у ролі радянського капітану міліції Івана Данко. 

Лише згодом стало зрозумілим, що ці проблеми справді існують. Більш того, час
від часу вони загострюються настільки, що навіть таке поняття як "русинський
сепаратизм" вже не сприймається оксюмороном.

"Ти занадто драматизуєш русинське питання", ─ каже мій друг Тарас, кинувши
погляд на монітор ноутбука і прочитавши останню фразу.

Авжеж, дуже хотілося би зосередитись на "культурному русинстві", згадати
відомих русинів як-от суперзірку поп-арту Енді Воргола, поринути у тонкощі
русинської кухні.

\ifcmt
img_begin 
    url https://img.pravda.com/images/doc/e/a/eaacac6-dscf1200.jpg
    caption Енді Воргол (Андрій Варгола) ─ найвідоміший русин у світі. В Ужгороді його також шанують, хоча батьки митця родом не з Закарпаття, а з села, що зараз знаходиться на території Словаччини
    width 0.7
img_end
\fi

Але саме "політичне русинство" привертає найбільшу увагу суспільства. Останнім
часом у цьому казані знов забулькотіло.

Засуджений ще при Януковичі за сепаратизм священник УМЦ МП Димитрій Сидор на
останніх виборах балотувався від ОПЗЖ в Ужгородську міську раду. Цього разу
пролетів, як фанера над Говерлою, але нагадав про себе.

\ifcmt
img_begin 
        url https://img.pravda.com/images/doc/7/3/735c943-dscf1069.jpg
        caption Кафедральний Хресто-Воздвиженський собор в Ужгороді належить до Української православної церкви Московського патріархату. Настоятелем храму є сумнозвісний священик Димитрій Сидор. Саме Московський патріархат на Закарпатті є головним натхненником сепаратистських настроїв у русинській громаді.
        width 0.7
img_end
\fi

12 років тому проти нього відкрили кримінальне провадження за статтею
"Зазіхання на територіальну цілісність України". Голова так званого "Cойму
підкарпатських русинів" Сидор 1 грудня 2008 року проголосив незалежність так
званої "Підкарпатської Русі". В 2012 році Ужгородський міськрайонний суд виніс
вирок: три роки позбавлення волі – умовно, та сплату 1840 гривень штрафу.

Найкращі друзі отця Сидора, бізнесмени Василь Джуган та Михайло Тяско, які
представляються керівниками "Всесвітньої ради підкарпатських русинів" та мають
тісні зв'язки з Росією і проросійськи налаштованим президентом Чехії Мілошем
Земаном, нещодавно були заарештовані чеською поліцією за звинуваченням у
несплаті податків.

Про це писали медіа, і русинська громада у Закарпатті починає "миригувати",
тобто "нервувати".

"Поспілкуйся з місцевими русинами і зрозумій, що насправді ніякого екшна,
жодної драми у цій темі не існує. Лише мляві віртуальні бійки у соцмережах і
нагнітання політичної істерики у тих самих медіа", – заспокоює Тарас.

Щоб переконатися в цьому, прямую на зустріч з істориком Володимиром Феничем,
автором монографії "Угорська Русь і "Ганнібалова присяга" Михайла Драгоманова:
початки інтелектуального завоювання Закарпаття Україною".

Ця книга 5 років тому викликала справжній скандал і коштувала Феничу місця
декана історичного факультету Ужгородського університету. 

\subsubsection{"З національністю не народжуються – з національністю лише
помирають"}

Ми п’ємо каву неподалік колишньої синагоги, побудованою у мавританському стилі
– у 1947 році її перетворили на філармонію.

У сквері поруч – погруддя Євгена Фенцика, русина-москвофіла, засновника
"Товариства імені Духновича".

\ifcmt
img_begin 
        url https://img.pravda.com/images/doc/b/7/b73f2b2-dscf1063.jpg
        caption Погруддя Євгена Фенцика, засновника "Товариства імені Духновича". Саме завдяки Фенцику сьогоднішні закарпатські русини так люблять російський триколор.
        width 0.7
img_end
\fi

Це погруддя, встановлене тут ще у 1926 році, за останні роки неодноразово було
облито чорною або червоною фарбами. Зараз його в черговий раз відчистили.

Біля Фенцика голосно розмовляють літні чоловіки: "Так просрати гарне місто
Ужгород можуть тільки галичани та наші місцеві бандюки, що називають себе
бізнесменами, прокурорами, суддями та взагалі міською владою".

Згадувати "незлим тихим словом" галичан – то давня закарпатська традиція.

Мармуровий Фенцик робить вигляд, що нічого не чує. Саме ця здібність вдавати з
себе нечуючого допомогла погруддю пережити всі режими, які пройшли Ужгородом.

На завзятого українофоба, а саме в цьому звинувачували Фенича одразу після
виходу його книги, пан Володимир, не схожий – занадто елегантний та привітний.

Але хто знає, як має виглядати справжній українофоб?

До 1993 року Володимир Фенич, як й усі народжені у СРСР, не надто переймався
власною етнічною ідентичністю. Навіть його мати-угорка була записана у
радянському паспорті як українка. 

Але у 1993-му Фенич знайомиться з книгою Пола Магочія "Формування національної
самосвідомості: Підкарпатська Русь". Саме з цієї книги він дізнається, що на
Закарпатті до 1946 року жили русини.

Після цього Володимир Фенич повернувся до ідентичності своїх предків по
батьківській лінії. Що не заважає йому бути патріотом України. Але після виходу
книги "Угорська Русь" його наукове бачення історії зненацька виявилось
"сепаратистським".

Про що була ця книга? Про те, що Михайло Драгоманов був першим, хто вважав, що
русини Угорщини – то братерський народ, і цей народ обов’язково треба залучати
до українського руху.

Начебто нічого крамольного. Але деякі українські історики були настільки не
згодні з окремими тезами цієї книги, що замість того, щоб відкрити наукову
дискусію, писали доноси в СБУ з проханням перевірити книгу на прояви
екстремізму та сепаратизму.

Звісно, ніхто нічого не знайшов, але після цього інциденту Володимир Фенич
змушений був покинути університет.

\ifcmt
img_begin 
	url https://img.pravda.com/images/doc/4/6/46af240-dscf1163.jpg
	caption Книга Володимира Фенича "Угорська Русь" пройшла перевірку щодо проявів екстремізму та сепаратизму. А скоро в нього вийде нова книга – біографія Пола Магочія.
    width 0.7
img_end
\fi

Я питаю його, як взагалі сталося, що русинський дискурс у нашій країні почав
сприйматись як геополітична спекуляція Кремля?

І чи можна інакше сприймати етнічну громаду, чиї лідери їздять до окупованого
Криму та супроводжують Земана під час його візиту до Москви? Або закликають
Путіна "активно освоювати Закарпаття" та проголошують віртуальну республіку
"Підкарпатська Русь"? Або хваляться тим, що формують у так званій "ЛНР"
русинські бойові загони, які воюють проти української армії (Петро Гецко)?

Адже з такими лідерами русинам не потрібні вороги – вони самі собі найгірший
ворог.

– Навколо русинського питання завжди було багато спекуляцій та маніпуляцій, –
каже Володимир Фенич. – Чому, наприклад, ви відразу прирівнюєте русинськи
товариства, що намагаються зберегти мовне різноманіття та традиції, і всіляких
Джуганів та Тяско? В кожного народу є такі Джугани.

В нас ще є й Світова Рада Русинів, почесним головою якого залишається Павло
Роберт Магочій, завідувач кафедри україністики університету Торонто. В 2008
році він зробив заяву про відмежування цієї інституції від будь-яких форм
екстремізму в русинському русі.

Але ви, говорячи про русинів, згадуєте не Магочія, а Сидора та Гецка, які
ніякими лідерами тутешніх русинів ніколи не були! Це афільовані з Росієї особи,
не треба забувати, що саме комуністична Москва у 1946 році в адміністративному
порядку замінила підкарпатським русинам національність у паспортах на
українську.

В словах пана Фенича є сенс. Коли мова йдеться про русинів, останнім часом
згадують саме тих, "афільованих з Росією".

– Чомусь українські націоналісти або ангажовані "експерти з історії" вважають,
що від русинів може надходити якась геополітична загроза або національна
небезпека, – знизує плечима Володимир Фенич.

Згідно з останніми переписами населення, русинів зараз 106 тисяч у 12 країнах
світу. Як вони можуть бути загрозою для когось? Але коли в українських медіа
заходить мова про русинів, чомусь завжди у контексті політичних фанаберій
Росії, Угорщини або Чехії.

Я не кажу, що ці держави не намагаються використати русинську карту у своїх
інтересах – це безумовно так. Але чому тоді Україні не зіграти на випередження,
вирішити русинське питання самим і у такий спосіб вибити цей "козир" з рук
усіх, хто грає на цьому питанні?

Визнавши русинів, Україна одразу нівелює усі дії Москви на Закарпатті, забуває
про усіх Тяско та Джуганів та спостерігає, як ОПЗЖ поступово втрачає будь-який
інтерес до отця Сидора. Логічно?

До кав’ярні, де ми спілкуємося, заходять хлопець з дівчиною у яскраво-рожевому
пуховику. Напружено прислухаюся до їхньої розмови. Чи не прибічники отця
Сидора?

"Як ти йому сказав?" – питає вона хлопця. "Фрас боо тя узяв!" – відповідає він.
"А що це означає?". "Те ж саме, що й у вас у Львові, коли ви кажете "шляк би
тебе трафив" – каже він і замовляє два тістечка.

Здається, на сепаратистів не схожі. 

– Людина з національністю не народжується – продовжує Володимир Фенич після
паузи. – Вона народжується з громадянством. Національність вона може тільки
набути. З національністю можна лише померти. Тому що це – свідомі, набуті у
зрілому віці, речі.

Українське громадянство не означає, що я українець за етнічним походженням. На
Закарпатті окрім українців та русинів проживає багато угорців, румунів,
словаків, ромів, поляків... І у демократичній державі усі ці національності
повинні мати такі самі права, що і представники більшого етносу.

А коли держава не опікується своїми етносами, чому вона потім дивується, що
ними зацікавились сусіди і використовують їх заради політичних спекуляцій?

\subsubsection{Традиція любити московитів}

Я мешкаю на вулиці Корзо, у самому центрі Ужгорода, на перехресті біля
католицького собору, яке місцеві називають "Хрест".

Вдень на "Хресті" працюють вуличні музиканти. Спочатку літній чоловік у
капелюсі грає на акордеоні "Лібертанго" та "Гуцулку Ксеню".

Потім приходить черга довговолосого молодика, який спеціалізується на
репертуарі гурту "Кіно".

Потім юна скрипалька почала грати "Бєлой акациї грозді душистиє". Я одразу
нагострив вуха, але "білогвардійський романс" швидко перетік у "Ніч яка
місячна", а потім й зовсім поринув у якусь "вівальдівщину" з присмаком "сім
сорок".

Судячи з усього, єдиним "радикальним ідейним русином" був той хлопець, що їхав
зі мною до Мукачева. Авжеж, є ще ангажований Москвою та ОПЗЖ отець Сидор. Але,
по-перше, він вже кілька років відмовляється спілкуватись з журналістами. А
по-друге, здається, він нікому не цікавий навіть тут, в Ужгороді.

Переходжу через міст та йду до пам’ятника Олександру Духновичу, що стоїть перед
Закарпатським музично-драматичним театром. Можливо, хоч там хтось з русинів
марширує під триколором?

\ifcmt
img_begin 
	url https://img.pravda.com/images/doc/8/7/8775861-dscf1065.jpg
	caption Олександр Духнович для закарпатських русинів – це як Шевченко для українців або Міцкевич для поляків
	width 0.7
img_end
\fi

У пошуках екшну для репортажу починаю трохи сумувати за часами, коли деякі
русинські лідери наймали для охорони своїх збіговиськ бойовиків зі Сходу – до
2014 року з Одеси сюди приїздили навіть активісти проросійської партії
"Родина".

У ті роки багато було розмов, що Росія створює на Закарпатті "інтерфронт".
Інколи сценарії таких "інтерфронтів" справді вистрілювали у мультикультурних
регіонах – спочатку Придністров’я, згодом Крим.

На Закарпатті також були спроби реалізувати щось подібне. Але русинська тема не
спрацювала, і Москва почала розробляти набагато успішніший "угорський
сценарій". Достатньо згадати скандал, що виник через видачу угорських паспортів
мешканцям Закарпаття, та заяви угорських ультраправих з партії "Йоббік" щодо
надання автономії угорській етнічній меншині у Берегово…

Територія навколо музично-драматичного театру й справді нагадує місце бойових
дій. Але не тому, що тут відбуваються якісь драматичні події – просто жахливий
занепад, від якого хочеться триматися подалі.

Бронзовий Олександр Духнович зосереджено спостерігає, як на протилежному березі
річки Уж діти годують лебедів та качок.

Цей русинський Будитель ненавидів мадяр, але дуже любив московитів, писав оди
на честь царя Миколи Першого та російської армії з нагоди перемоги над турками.
Вважав російську мову справжньою спадкоємницею церковнослов’янської та
слав’яноруської мов. І мріяв, що у майбутньому літературна мова закарпатських
русинів буде засновуватися не на русиньскому підгрунті, а саме на російській
мові.

До речі, проект створення самостійної русинської нації був для нього чужим –
тобто він був не сепаратистом у сучасному розумінні, а навпаки – соборником. Бо
мріяв, що одного разу русини Закарпаття та Галичини стануть єдиним народом з
московітами.

Така собі парадигма "русского міра" 19-го століття, але у нинішньому контексті
шестирічної війни з Росією, це сприймається якось зовсім не актуально.

В кафе на вулиці Льва Толстого зустрічаюся з Владиславом Товтином, популярним
ужгородським екскурсоводом.

\ifcmt
img_begin 
    url https://img.pravda.com/images/doc/c/b/cb432ed-dscf1079.jpg
    caption Владислав Товтин: "Галичани не дуже люблять згадувати, що ще 100 років тому вони також були русинами. Я їм кажу: "У вас по хатам портрет найвідомішого русина – Івана Франко. Навіть Степан Бандера по документам – русин"
        width 0.7
img_end
\fi

Про те, що він русин, Владислав дізнався від діда, коли був ще зовсім малим.
Але він вважає, що русини – це давня назва всіх українців, а у неорусинському
руху, що був ініційований Полом Магочієм, занадто багато постмодернізму та
"неадекватів".

– Взагалі, чому люди починають себе ідентифікувати з русинами, а не з
українцями? – питає Владислав і сам собі відповідає. – У їхніх очах Україна як
держава – це невдаха у політичному та економічному сенсі. З цих причин на Сході
стають росіянами, а тут стають русинами. Наприклад, русини у Словаччині не
бажають себе ідентифікувати з українцями, бо українець – то бідний сусід зі
Сходу.

Питаю про нову русинську генерацію на Закарпатті – де вона? Владислав сміється
і відповідає – у соцмережах. До того ж ця молодь наполягає, що займається
виключно русинською культурою, а не політикою.

– Вони не розуміють, якщо ти використовуєш русинську символіку, то ти вже у
політиці. Взагалі, у цієї нової генерації є декілька цікавих проєктів, але у
більшості випадків… починаєш копати – і раптом виповзає якесь "дрімуче
православіє" або москвофілія.

Якщо розглядати русинське питання крізь призму історії, воно так або інакше
завжди впирається в москвофільство.

– В Ужгороді багато хто з русинів спілкувався російською мовою ще при
Чехословаччині, – каже Владіслав. – Зараз, наприклад, ми п’ємо каву на вулиці
Льва Толстого, яка так називалась ще у 20-х роках ХХ століття. Можемо, до речі,
прогулятися до "Снека" – там є багато цікавих експонатів того періоду.

"Снек" – маленьке кафе, що має офіційну назву "Під замком", але так його
називають лише туристи. Власник "Снеку" на ім’я Дьюри – румун, який вважає себе
угорцем, але крім румунської, угорської, української, російської та русинської
добре спілкується ще й італійською. Як справжній добрий газда він миттєво
приносить нам з Владиславом по чарці домашньої сливовиці.

Цей колоритний заклад дещо нагадує кафе "У Ріка" з фільму "Касабланка". Тут
також за сусідніми столиками можна інколи побачити людей з протилежних
політичних таборів, хлопців з "Карпатської Січі", угорських націоналістів.

Але сьогодні тут пусто. Відсутність людей мене дивує – Закарпаття оголошено
"червоною зоною", і вже завтра всі кафе, бари і ресторани мають бути зачинені
на два тижні. Але, судячи з усього, в Ужгороді немає звички "нагулятися
заздалегідь".

У "Снеку" зібрано багато речей австро-угорського та чехословацького періодів, у
тому числі й таблички з державних та громадських установ або зі старими назвами
ужгородських вулиць. І роздивляючись цей антикваріат, усвідомлюєш, що російська
мова на Закарпатті у 1920-х роках зустрічалась доволі часто. Та й взагалі до
1946 року всередині русинського руху були течії "укрофілов", "русофілов",
"мадярофілов", "чехофілов".

\ifcmt
img_begin 
        url https://img.pravda.com/images/doc/1/4/14ec600-kollazh.jpg
        caption Вивчати історію Закарпаття краще за все, розглядаючи інтер'єр ужгородського кафе "Снек". Саме тут розумієш, чому так зване русинське "москвофільство" – не сучасна вигадка, а історична традиція
        width 0.7
img_end
\fi

Це природно – бездержавний народ завжди буде шукати собі "парасольку".

Я питаю Владислава, у чому полягає проблема визнання Україною підкарпатських
русинів окремим народом, а їхню "бесіду" – окремою мовою. Чи не логічніше
вибити цей козир з рук Москви, Будапешта і Праги?

– Ти вибиваєш цей козир з рук Росії, але тим самим даєш іншій, – вважає
Владислав. – Чому Закарпаття українське? Тому що радянська влада його приєднала
до України. На яких підставах? Тому що на Закарпатті проживають етнічні
українці. І якщо зараз Україна погоджується, що русини – то не українці, таким
чином Закарпаття перетворюється на територію, що безпідставно та несправедливо
була колись приєднана до України.

Якщо русини не українці, то й договір про передачу Закарпаття до складу СРСР
недійсний. Бо офіційне формулювання – возз'єднання українського народу з
частиною українського народу, яка просто називаються інакше – русини. 

А тоді ми назавжди втрачаємо Крим. Тоді приєднання Криму до України в 1954 році
– також незаконне. Бо на той час українців в Криму була меншість від загального
населення. За цією логікою, приєднання Закарпаття до України – це обман, а
приєднання Криму – обман ще більший. Ви розумієте? Це політика, а не
етнографія, тут треба усе враховувати на пару кроків вперед.

Владислав запевняє мене, що русинське питання і проблеми, пов’язані з ним –
явище не унікальне. Це звичайна проблема розмитості ідентифікації на
прикордонні.

У Чехії, наприклад, ті самі проблеми з мораванами – Словаччина їх визнає
окремою нацією, а чехи – ні. Тому коли хтось з закарпатських русинів їздить до
Земана, а потім розповідає, що президент Чехії обіцяв допомогти "натиснути на
українську владу", той або бреше, або нічого не розуміє – ніколи Земан не стане
на захист русинів. Тому що йому у ту ж мить прилетить з Моравії.

\ifcmt
img_begin 
        url https://img.pravda.com/images/doc/5/c/5ce1d41-dscf1108.jpg
        caption Владислав Товтин: "У чому специфіка закарпатської мови? От кажуть – у вас тут багато чого намішано. Так, запозичення всюди є. Але в нас тут ще є і дуже багато архаїзмів. На Закарпатті як периферії збереглася та мова, якою писали літописи. Навіть не церковнослов’янська, а саме літописна".
        width 0.7
img_end
\fi

– Ще з часів Духновича та Фенцика існує русинська традиція любити московитів, –
пояснює Владислав. – Навіть русини, які проживають у інших державах – завзяті
москвофіли. У них всюди портрети Путіна, світлини Красної площі, російські
прапори.

Коли до них приїжджає Пол Магочій, він кожен раз питає – ви що, здуріли, навіщо
ця маячня? Але навіть його слова не діють. Тому що традиція. І Москва працює з
цією традицією. А Україна не працює навіть з закарпатськими русинами. Зараз
наша держава намагається вибудувати таку струнку красиву українську
ідентичність, але історія набагато складніша за ідеологічні канони.

\subsubsection{Зручної історії не буває}

Є на Закарпатті дотепний русинський жарт "Людина людині – вовк, а руснак
руснаку – українець".

Я згадую про цей жарт, коли бачу, як поліціянти, що поміж собою спілкуються
"закарпатською", під час сварок з місцевими власниками кав’ярень та ресторанів,
що відкриті попри заборону – "червона зона", як-ніяк – одразу переходять на
державну мову.

Шукати заклад, який працює підпільно, ані у мене, ані у Володимира Фенича немає
бажання, тому наша друга зустріч відбувається у затишному сквері. Я розповідаю
йому про нашу розмову з Владиславом. Фенич не погоджується з висновками
Товтина.  

– Так званий "іншій козир" Росії дають не русини, а українські
"декомунізатори", – каже пан Володимир. – Саме вони здатні з часом
"декомунізувати" договір по Криму 1954 року або піддати сумніву
радянсько-словацький договір 1945 року про входження Закарпаття до складу СРСР.

Адже декомунізація – це не просто повалити пам’ятники Леніну і змінити назви
вулиць. Якщо до влади в Україні прийдуть люди, яким декомунізація таких
договорів буде на руку, то вони будуть мати всі підстави для цього.

Щодо русинського питання Фенич взагалі не бачить офіційної позиції України. У
сусідніх країнах це питання було вирішено у демократичний спосіб, а Україна
лише декларує наміри бути європейською країною. Тим паче, що "Русь" та
"руськість" – це набагато ширше явище, ніж винятково етнографічний сегмент.

– Так, русини не вважають свою рідну мову діалектом української, – каже Фенич.
– Але зараз ми з вами спілкуємося державною мовою, чи не так? І державна мова у
нас всіх одна – і в українців, і у підкарпатських русинів. І територіальні
кордони нашої держави є непорушними і суверенними для всіх.

Але чому в мене, громадянина України, у якого мати угорка, а тато русин,
ставлення до своєї держави цілком патріотичне, а ось ставлення держави до мене
– шовіністично-дискримінаційне?

Це імперський підхід. Коли Путін каже, що росіяни та українці – єдиний народ, а
українську націю та мову придумав австрійський генштаб, мене це обурює і як
громадянина України і як професійного історика. Але чому Україна так само
ставиться до русинів и русинської мови?

\ifcmt
img_begin 
        url https://img.pravda.com/images/doc/f/c/fcc6797-dscf1192.jpg
        caption Володимир Фенич: "Чому русини з ностальгією згадують часи, коли Закарпаття увійшло до складу Чехословаччини? Тому що Чехословаччина – це демократичне дитя Версаля. Її демократія дозволяла спокійно співіснувати і москвофілам, і українофілам, і угрофілам. Аби тільки не порушували законів".
        width 0.7
img_end
\fi

Точку зору Володимира Фенича я розумію. Але не розумію, чому на мітингах
закарпатські русини й досі використовують триколор, що дуже нагадує російський
прапор.

Коли йдеться про автономію, русини люблять згадувати, що в них вона вже колись
була – Підкарпатська Русь. Але прапором Підкарпатської Русі було жовто-блакитне
полотнище.

Чому коли йдеться про автономію, Підкарпатська Русь – це зразок, а коли йдеться
про прапор, то Підкарпатська Русь вже ніякий не зразок? Адже саме через той
триколор постійно виникають непорозуміння.  

Коли в Україні вже шість років йде війна з Росією на Донбасі, коли Крим був
анексований саме під триколором, хіба доцільна впертість щодо традицій, які
були започатковані при Чехословаччині москвофільськими товариствами Фенцика?

– Чесно кажучи, мені це самому незрозуміло і неприйнятно, – розводить руками
Фенич. – Так, на прапорі Підкарпатської Русі були жовтий і блакитний кольори,
тому що це кольори династії Габсбургів. Саме тому русинськими природними
кольорами є ті ж самі кольори, які присутні на державному прапорі України.

Але… Ви розумієте, коли Фенцик у 1923 році почав використовувати триколор, він
мав на увазі не більшовицьку Росію, а ту Росію, що боролася з більшовиками,
тобто, була жертвою. Тоді на той триколор взагалі ніхто не зважав, тому що
партія Фенцика була лише однією з 30 політичних партій, які брали участь у
виборах до чехословацького парламенту. Але зараз… так, я згоден з вами – це
неправильно.

\ifcmt
img_begin 
        url https://img.pravda.com/images/doc/3/1/3109269-dscf1097.jpg
        caption Герб Підкарпатської Русі на стіні Будинку національних спільнот в Ужгороді. Закарпаття отримало власний герб після включення його до складу Чехословаччини в 1919 році. Саме цей герб закарпатські (або підкарпатські) русини позиціонують як символ всіх русинів у світі
        width 0.7
img_end
\fi

Ми з Володимиром Феничем заходимо до внутрішнього двору Ужгородського
художнього музею, що розташований у приміщеннях колишнього будинку колишнього
Ужанського комітату – адміністративно-територіальної одиниці в Угорському
королівстві.

– Розумієте, – каже Фенич, – тутешнє москвофільство – невід’ємна історія
закарпатського краю. Ми не повинні переписувати історію – ми повинні її
переосмислювати. Нам треба вчитись жити з незручною історією, бо "зручної"
історії немає у жодної нації.

От скажіть мені, хто є більшим соборником? Августин Волошин, який у березні
1939 року проголосив незалежність Карпатської України, тобто реалізував
сепаратистський проект як громадянин Чехословаччини? Чи комуніст Іван Туряниця
(теж чехословацький громадянин), який у листопаді 1944 проголосив про
возз’єднання Закарпаття з Україною?

Для мене як історика неприпустимо "канонізувати" Волошина та Карпатську Україну
і забути назавжди про Туряницю і про Закарпатську Україну. Тому що усе це
історія. А історія завжди має в собі велику кількість незручностей для
ідеології.

Коли у 2002 році Августина Волошина посмертно нагородили званням Героя України,
на Закарпатті серед русинів виникла напруга, тому що тут ще пам’ятають і про
концтабір Думен поблизу Рахова, і про встановлення у краї авторитарного
волошинського режиму.

Але що можна зробити, щоб ідеологічні концепти не йшли у розріз з історичними
фактами?

– "Pacto del olvido – договір про забуття. А також заборона державним
чиновникам всіх рівнів коментувати історичні події. Тому що стаючи на один бік
історичних подій чи діячів, вони від імені держави скривджують пам'ять іншої
частини суспільства, предки яких не знали, що їхні нащадки колись житимуть в
Україні, – вважає Володимир Фенич.

А ще він впевнений, що Україні навіть вигідно з ідеологічної точки зору визнати
русинів за окремий етнос, тому що русинська мова – це коріння, з якого потім
проросли українська, російська та білоруська мови. Тобто, русинська мова як
релікт збереглася крізь час, попри усі монархічні, імперські та тоталітарні
режими. Українцям лише треба переступити через власний етноегоїзм, породжений
вигаданою національною титульністю, винятковістю та обраністю.

Можливо, саме на цьому грунті й треба шукати компроміс?

Держава може сказати русинам – так, хлопці, ми не визнаємо вас окремим етносом.
Але первородство у царині української мови та деяких інших слов’янських мов
належить саме вам. Ви – релікт нашої давньої руськості, наш зв'язок з предками.
Закарпаття може стати колискою слов’янських мов. Хіба це не вигідно Україні з
точки зору ідеології?

Але відповідь на це питання можна отримати лише у Києві.

\subsubsection{Фінальна алегорія}

\begin{itemize}
  \item – Ну що, знайшов русинських радикалів? – дещо в’їдливо питає мене Тарас, коли я сідаю до нього у авто.
  \item – Ні, – відповідаю.
  \item – З чого почнеш свій репортаж?
  \item – З фрази "Українство в Закарпатті – це вірус".
  \item – Ого, – дивується Тарас. – Провокаційний початок. А назву вже придумав?
  \item – Так, можливо, "Русинська карта".
\end{itemize}

У політичному сенсі на Закарпатті є дві карти, котрі час від часу розігруються
Росією – русинська та угорська.

І дехто з політиків сприймає ці карти як "туза" і "короля".

Але, чесно кажучи, ця пара нагадує комбінацію у покері, яку називають "Анна
Курнікова" – не лише тому, що ініціали російської тенісистки "АК" збігаються з
позначеннями туза і короля в колоді.

А ще й тому, що ця комбінація, як і дружина Енріке Іглесіаса, виглядала
привабливо, але дуже рідко вигравала.
