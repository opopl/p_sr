% vim: keymap=russian-jcukenwin
%%beginhead 
 
%%file slova.pensia
%%parent slova
 
%%url 
 
%%author 
%%author_id 
%%author_url 
 
%%tags 
%%title 
 
%%endhead 
\chapter{Пенсия}
\label{sec:slova.pensia}

%%%cit
%%%cit_head
%%%cit_pic
%%%cit_text
50-я статья Конституции гласит, что каждый имеет право на безопасную для жизни
и здоровья окружающую среду и на возмещение причиненного нарушением этого права
вреда.  Однако пока не слышно о компенсациях из-за загрязнений окружающей
среды.  Статья 56 говорит, что каждый имеет право на возмещение за счет
государства материального и морального вреда, причиненного незаконными
решениями, действиями или бездействием представителей государства.  Однако,
например, \emph{пенсионерам} Донбасса задолжали 11 млрд гривен. И никто
никакого ущерба им не возмещает
%%%cit_comment
%%%cit_title
\citTitle{День Конституции Украины 28 июня - какие статьи нарушаются сильнее всего}, 
Оксана Малахова; Максим Минин, strana.ua, 28.06.2021
%%%endcit

%%%cit
%%%cit_head
%%%cit_pic
%%%cit_text
Ситуацию делает еще сложнее то, что даже из тех работников, которые якобы
остались на ХГАПП, львиную долю составляют \emph{пенсионеры} (средний возраст
работников 60 лет), которым просто нет смысла искать другую работу. А более
молодые, или профессиональные специалисты уже давно нашли себе другие места.
Таким образом, можно говорить, что начать в таких условиях, даже при получении
заказов, серийное строительство новых самолетов предприятие не в состоянии.
Поскольку нет необходимой для выполнения подобных заказов количества
профессиональных специалистов
%%%cit_comment
%%%cit_title
\citTitle{ХГАПП больше не может производить самолеты}, Антон Щукин, strana.ua, 27.06.2021
%%%endcit

%%%cit
%%%cit_head
%%%cit_pic
\ifcmt
  pic https://strana.ua/img/forall/u/0/36/2021-07-03_14h48_06.png
\fi
%%%cit_text
\enquote{Если нам не нравится Россия и ее президент, то разве это говорит о том, что мы
не можем говорить на этом языке?}, - отвечает нам \emph{пенсионерка}
%%%cit_comment
%%%cit_title
\citTitle{Что говорят украинцы о пресс-конференциях футболистов на русском языке. Опрос Страны}, 
Антонина Белоглазова, strana.ua, 03.07.2021
%%%endcit

%%%cit
%%%cit_head
%%%cit_pic
%%%cit_text
Ну... прошёл годик после воцарения надёжи-государя. Народец заволновался.
Потому как рожи новые разглядел, речи их послушал. Один боярин по фамилии
Брагар предложил \emph{пенсионеру} собаку продать, чтобы за ЖКХ платить
вовремя. Столбовая боярыня из комитета по вопросам социальной политики
глубокомысленно посетовала: куды идём, к чему стремимся, когда людишки
малообеспеченные рожают детей «очень низкого качества, сажая их на шею
государству».  Буйство интеллекта и правды-матки из боярского «Слуги народа»
так и прёт. Забыв выключить микрофон: государевы единомышленники Корниенко и
Арахамия обозвали свою гламурную коллегу... «рабочей бабой», сравнили с
«корабельной сосной».  Интернет опешил, потом долго лечил заворот кишок,
случившийся от великого смеха. «Корабельная сосна» … гы. Теперь это уже
классика, не хуже гоголевской
%%%cit_comment
%%%cit_title
\citTitle{Такие смешные: украинская политика... через призму творчества Гоголя}, 
Исторические напёрстки, zen.yandex.ru, 28.10.2021
%%%endcit
