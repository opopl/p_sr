% vim: keymap=russian-jcukenwin
%%beginhead 
 
%%file slova.poproshajka
%%parent slova
 
%%url 
 
%%author 
%%author_id 
%%author_url 
 
%%tags 
%%title 
 
%%endhead 
\chapter{Попрошайка}

%%%cit
%%%cit_head
%%%cit_pic
%%%cit_text
В ожидании \emph{подаяния}.  Недоверие к Украине... Привычная и непреодолимая
банальность. Мы – \emph{попрошайка} на всемирной городской площади, с угрозами
и стенаниями требующая \emph{подаяния}.  Бесстыдство политиков, уверовавших в
искренность собственной лжи. И в свои оптимистические прогнозы.  Примитивный
поиск врага, виновного во всем. От Путина и социологов, до Меркель и Байдена.
Каждый открыто поясняет: я гораздо лучше и популярнее, нежели моя
характеристика у социологов. Иосиф Виссарионович также не верил в точные науки,
включая демографию и социологию. Не только в генетику.  Да что там Сталин. Тот
знал, понимал, именно поэтому запрещал демографические и социологические
исследования. Он не исследовал свой рейтинг, диктатор использует другие
источники информации. Его рейтинг был абсолютным. Эти, наши – рейтинги
заказывают. У своих социологов, личных. А затем – комментарии к ним, у своих же
политологов.  Профессиональные \emph{попрошайки} рейтинги доверия не
заказывают. Они пользуются иным критерием доверия – количеством полученных
денег. Мы, Украина доверие потеряли, напрасно ожидаем \emph{подаяния}. \emph{Не
подадут}.  Грустно и гнусно одновременно
%%%cit_comment
%%%cit_title
\citTitle{Украина потеряла доверие и напрасно ожидает подаяния}, 
Семен Глузман, strana.ua, 17.06.2021
%%%endcit

