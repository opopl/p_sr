% vim: keymap=russian-jcukenwin
%%beginhead 
 
%%file books.tim_taler
%%parent books
 
%%url 
 
%%author 
%%author_id 
%%author_url 
 
%%tags 
%%title 
 
%%endhead 

\ii{books.tim_taler.rus}

\url{https://zarlit.com/lib/kryus/1.html}

Тім Талер, або Проданий сміх - Джеймс Крюс

Переклад Ю. Лісняка

ВІД АВТОРА

Описану тут історію розповів мені один чоловік років п'ятдесяти, що приїхав до
Лейпціга, як і я, наглядати, як друкується одна книжка. (Коли не помиляюся, то
була книжка про ляльок та ляльковий театр.) Найцікавіше в тому чоловікові було
те, що він, хоч уже й немолодий, умів сміятися так приємно та весело, мов
десятирічний хлопчак.

Хто він був, я можу тільки здогадуватись. Невідомо також, коли діялась ця
історія, та й ще дещо в ній неясне. (Правда, де з чого видно, що головні події
відбувались десь близько 1930 року.)

Треба ще згадати, що писано цю оповідь у перервах між роботою, на зворотному
боці великих відбракованих друкарських аркушів. Ось чому книжка й поділена на
аркуші, хоч по суті це звичайнісінькі розділи.

Іще хочеться попередити, що вам не багато доведеться сміятися над цією книжкою,
хоч мовиться в ній про сміх. Однак зважте й на те, що цей шлях крізь темряву
все ж веде. хоч і манівцями, до світла.

КНИГА ПЕРША

УТРАЧЕНИЙ СМІХ

Ну що ж, забились! Але хто ж не знає,

Що сміх людей від звірів відрізняє!

Адже з того людина й пізнається,

Що в слушну мить вона сміється.

З прологу до лялькової

вистави «Липкохвіст»

Перший аркуш

БІДНИЙ ХЛОЧИК

У великих містах із широкими вулицями ще й досі є на околицях завулочки, такі вузенькі, що з вікна у вікно навпроти можна поздоровкатись за руку. Коли чужоземні туристи, грошовиті та чулі, попадають ненароком у такий завулочок, вони вигукують: «Ах, як тут мальовниче!»

А жінки зітхають: «Як ідилічно! Як романтично!»

Але вся та ідилія й романтика - облуда, бо на околицях живуть люди, що не мають
грошей. А хто в такому великому, багатому місті не має грошей, той [11] стає
понурий, заздрісний, а часто й сварливий. І винні в тому не так самі люди, як
завулки.

Малий Тім попав у такий-от вузенький завулочок, як було йому тільки три роки.
Тімова весела, повненька матуся померла, а татові довелось піти працювати на
будівництво, бо тоді саме було безробіття. Отож і мусили батько з сином
залишити своє світле, гарне помешкання з заскленим балкончиком край міського
парку, і перебратися до вузького завулка, забрукованого кругляками. В тому
завулку весь час пахло перцем, кмином та ганусом, бо там стояла єдина на все
місто фабрика прянощів. А незабаром у Тіма з'явилася мачуха - худюща, з
гострим, ніби мишачим, писком - і на додачу зведений брат, білий як сметана,
розпещений та вредний хлопець.

Як на свої три роки, Тім був хлопчик міцненький; він умів навдивовижу дзвінко
сміятись і цілком самостійно кермував пароплавом, зробленим із двох стільців,
або автомашиною з диванних подушок. Небіжчиця матуся, було, сміялася до сліз,
коли Тім вибирався в подорож по світу на тих стільцях чи подушках та вигукував:
«Чах-чах-чах, Америка!» А от мачуха його за таке лупцювала. І він ніяк не міг
того зрозуміти.

Та й свого нерідного брата Ервіна Тім не розумів, бо той виявляв йому свою
братню любов, шпурляючи на малого поліняками або вимащуючи його сажею, чорнилом
чи сливовим повидлом. Однак найнезрозуміліше було те, що карали тоді не Ервіна,
а Тіма ж таки. Одне слово, стільки було в тій новій оселі незрозумілого й так
воно гнітило хлопчика, що він майже відучився сміятися. Лиш тоді, коли був
удома тато, лунав іноді в хаті Тімів тоненький заливистий сміх із кумедним
«ік!» на кінці.

Та, на жаль, тепер тато рідко бував удома, бо працював на далекій будові. (Адже
й оженився він удруге тільки задля того, щоб не кидати малого сина самого
вдома.) Лиш у неділю міг тато побути зі своїм синочком. [12] Щоранку він брав
малого Тіма за руку й казав мачусі: Ми йдемо гуляти». Насправді ж вони йшли на
іподром, на кінські перегони, де тато, потай від мачухи відклавши собі за
тиждень якусь копійчину, ставив на котрого-небудь коня. Він сподівався колись
виграти на перегонах стільки грошей, щоб перебратися з цього вузького завулка
кудись у краще, просторіше помешкання. Але сподівання ті, як і в більшості
гравців були, звичайно, марні. Йому не щастило, він майже завжди програвав, а
коли й вигравав, то лише на цукерки синові на кухоль пива собі та на трамвай.

Малого Тіма перегони тішили не дуже. Мелькає щось там далеко - годі й
розглядіти. До того ж перед ним завжди стояло дуже багато людей, і хлопчикові
навіть із таткового плеча мало що було видно.

Та хоч Тіма й не цікавили коні й жокеї, він дуже швидко збагнув усю механіку
гри: коли вони з татом вернися додому трамваєм, а Тім діставав трубочку
льодяників, - це означало, що тато виграв. Коли ж тато сидовив його собі на
плечі й вони вертались додому пішки без цукерок, значить, вони програли.

Та чи пригравали вони, чи вигравали, малому було однаковісінько. Бо їхати верхи
на таткові йому було так само весело, як і в трамваї, а то й ще веселіше.
Головне - що вони вдвох, що неділя, що Ервін із мачухою десь далеко-далеко,
мовби їх і зовсім нема.

Але, на жаль, решту шість днів тижня вони були... Тімові тоді жилося достоту як
тим дітям у казці, що мали лиху мачуху. Тільки Тімові було навіть гірше, бо ж
казка - то казка та й годі, вона починається на першій сторінці й закінчується
щонайпізніш на дванадцятій. А терпіти отаку халепу день у день, рік у рік - це
вам неабищиця. І якби не оті неділі. Тім, певно, просто навсупереч усім
зробився б справжнім зухвалим хамидником. А так він усе ж лишався життєрадісним
хлопчиком і не забував свого сміху, дзвінкого, заливистого сміху, що починався
десь глибоко в животі й кінчався кумедним «ік!». [13]

Та, на жаль, дуже рідко траплялося Тімові сміятись. Він став мовчазний і
гордий, просто неймовірно гордий. Так він боронився від мачухи, що гризла йому
голову за всяку дрібничку, хоч, може, часом і не зі злості, а. просто так.

Коли Тіма віддали до школи, він дуже зрадів. У школі він з рана й до обіду
почував себе далеко від свого завулка - далеко-далеко, хоч насправді від дому
до школи не було й півкілометра. В першому класі Тім знову став часто й весело
сміятися, і за той сміх учитель часто вибачав хлопчикові різні дрібні провини.
Тім тепер намагався догоджати навіть мачусі. Коли вона вряди-годи хвалила його
- за те, скажімо, що сам доніс додому цілих п'ять кілограмів картоплі,- він
ставав щасливий, ласкавий і послужливий. Та згодом вона знову лаяла його ні за
що, і він знову замикавсь у собі й удавав гордого. Тоді, було, до нього ані
приступу.

Така ото незлагода між ним і мачухою погано позначилась на його шкільній науці.
Хоч Тім мав куди бистріший розум, ніж багато його однокласників, проте він
діставав гірші оцінки, ніж вони, через те, що часто бував неуважний на уроках.
Та ще через домашні завдання.

Бо йому було дуже важко їх робити. Тільки-но сяде зі своєю дощечкою за стіл у
кухні, як надходить мачуха й проганяє його до дитячої кімнати. А там хазяйнував
його нерідний брат Ервін, що не давав малому ні на хвилину спокою. То йому
забагнеться гратися з Тімом, а як той відмовиться - Ервін злоститься; то
розкладеться зі своїм «конструктором» на весь стіл, так що Тімові ніде й
примоститися з зошитами. Одного разу Тім у праведному гніві вкусив Ервіна за
руку. Та не було йому з того добра. Мачуха, побачивши кров на синочковій руці,
зарепетувала на всю хату й назвала Тіма гадюченям. І навіть татко за вечерею не
озвався до нього й словом. Відтоді Тім більше не заводився з розбещеним
мазунчиком, а почав нишком учити уроки [14] в татковій та мачушиній спальні.
Але Ервін, помітивши те, виказав його: адже дітям не дозволялось заходити в
спальню дорослих, такий був один із найперших мачушиних законів.

І мусив Тім знову якось примудрятися вчити уроки в не дуже приємному товаристві
свого зведеного брата. Коли той не пускав його до невеличкого столика, єдиного
в їхній кімнатці. Тім сідав на ліжко й клав зошит на тумбочку. Однак і там
Ервін весь час заважав йому. Тільки в середу, коли Ервін був у школі по обіді,
Тімові щастило вивчити уроки так ретельно, як вимагав учитель. А Тім дуже хотів
йому догодити - цей малий хлоп'як, що вмів так гарно сміятися, радий був з усім
світом жити в злагоді.

Та, на жаль, Тімові домашні роботи дедалі менше подобались учителеві. «Здібний
хлопець, тільки ледачий і неуважний»,- казав він. Звідки ж мав учитель знати,
що хлопцеві доводиться день у день відвойовувати собі місце, щоб вивчити уроки.
А Тім йому того не розповідав, бо певен був, що вчитель і сам усе знає. Отак і
школа знов довела хлопця до сумного висновку, що життя незрозуміле й що дорослі
всі несправедливі. Всі, крім його тата.

Та й ця єдина в світі справедлива людина покинула його. Бо коли Тім на
превелику силу добравсь у школі до п'ятого класу, сталося нещастя: тата вбило
на будівництві дошкою, що впала звідкись згори.

Оце вже була найнезбагненніша подія в Тімовому житті. Він нізащо не міг
збагнути, яке право мала та дошка отак із доброго дива впасти й накоїти такого
страшного лиха. Спершу він просто не хотів повірити, що це правда. Лиш у день
похорону, коли розхвильована, заплакана мачуха надавала йому ляпасів за те, що
забув почистити їй черевики, Тім зрозумів, який він тепер самотній. Бо ховали
тата якраз у неділю.

Лиш тоді Тім заплакав. Він плакав і за батьком, і над самим собою, й над усім
світом. Навіть мачусі стало [15] соромно, і Тім уперше зроду почув від неї:
«Ну, вибач, Тіме, не плач, я ж нехотячи...»

Година на кладовищі була для Тіма наче лихий сон, що його хочеться чимшвидше
забути й що від нього лишається тільки гнітючий плутаний спогад. Тім ненавидів
усіх людей, що стояли біля ями, говорили промови, співали, бубоніли «отченаш».
То один, то другий підходив до мачухи, висловлював їй «своє глибоке співчуття»,
вона, хлипаючи, щось лопотіла у відповідь, і Тіма страшенно дратувало те
лопотіння. Він хотів журитись за своїм татком сам. І коли люди почали
розходитись, хлопець скористався з тієї нагоди й простісінько втік.

Він довго тинявся вулицями, і, коли опинився біля того будинку край парку, де
вони колись жили з татом і мамою, де він ще зовсім маленьким хлопчиком сміявся
та кричав: «Чах-чах-чах, Америка!», йому стало так гірко, що аж світ в очах
потемнів. Із його колишньої кімнатки виглядала у вікно чужа дівчинка. На руках
вона держала дорогу, пишно вбрану ляльку. Покмітивши, що Тім на неї дивиться,
дівчинка показала йому язика, і Тім квапливо пішов далі.

«Якби я мав багато-багато грошей,- думав він, ідучи,- то найняв би велику
квартиру, щоб у мене була своя кімната, а Ервінові давав би щодня гроші на кіно
та на морозиво, а мачуха нехай би купувала на обід що хоче». Але то була тільки
мрія, і Тім це знав.

Він ішов, куди очі світять; ноги самі несли його до іподрому, куди вони ходили
щонеділі з татом, як той був іще живий.

Другий аркуш

КАРТАТИЙ ЧОЛОВІК

Перші перегони вже наближались до кінця, коли Тім прийшов на іподром. Глядачі
кричали, свистіли, й дедалі гучніш лунало назвисько «Вітер». [16]

Тім стояв і важко дихав. Не тільки тому, що стомився: йому раптом уявилося,
ніби десь в оцій галасливій юрмі стоїть його тато. Хлопець немовби опинився в
рідній домівці. Адже це тут, тільки тут йому траплялось побути вдвох із татом.
Без мачухи. І без Ервіна. Всі ті неділі на іподромі з татом раптом ніби
втілились в оцю юрму і оцей галас. Наче й не було ні кладовища, ні сліз. Тім
став навдивовижу спокійний, майже веселий. А коли по натовпу прокотився
радісний крик і немов з одних уст пролунало: «Вітер!» - хлопець аж засміявся.
Він згадав, як батько одного разу сказав: «Вітер іще молодий коник, Тіме, може,
навіть замолодий для перегонів, але колись про нього заговорять, ось побачиш!»

І ось про Вітра справді заговорили, та батько того вже не чує... Тім і сам не
знав, чого йому стало смішно. Але він про це й не думав. Він ще не доріс до
того віку, коли люди замислюються над собою.

Якийсь чоловік, що стояв поблизу Тіма, зачувши той дзвінкий сміх із кумедним
«ік!» на кінці, швидко повернув голову й пильно подивився на хлопця, а тоді
погладив замислено своє підборіддя й не довго думаючи рушив до Тіма. Але не
спинивсь перед хлопцем, а проминув його, наступивши йому на ногу, й кинув через
плече:

- Вибач, малий, я ненавмисне.

- Та нічого,- знову засміявся Тім.- Однаково в мене черевики запорошені.-
Зиркнувши на свої ноги, він раптом побачив на траві перед собою новеньку
блискучу монету - п'ять марок. Чоловік, що наступив йому на ногу, пішов собі
далі, й біля Тіма не було нікого. Хлопець швиденько став на монету ногою,
озирнувся сторожко довкола, нахилився - ніби зав'язати шнурок, швиденько,
крадькома підняв гроші, сховав у кишеню й навмисне неквапливо почовгав геть.

Та зненацька його перепинив високий худий чоловік у картатому костюмі й спитав:

- То що, Тіме, хочеш спробувати щастя? [17]

Хлопець спантеличено звів очі на незнайомця. Він навіть не впізнав у ньому того
самого чоловіка, який щойно наступив йому на ногу. У незнайомця був беззубий,
немов бритвою прорізаний рот, тонкий карлючкуватий ніс, а під ним рідесенькі
чорні вусики. З-під низько насунутого великого кашкета дивились на Тіма
водяво-блакитні очі. Кашкет був картатий, як і костюм.

Коли чоловік раптово звернувся до Тіма, хлопцеві немов клубок у горлі став. Нарешті він видушив із себе:

- У ме... у мене немає грошей на заклад.

- Чого ж нема? А п'ять марок? -відказав чоловік. Потім додав спокійнісінько: - Я бачив, як ти знайшов гроші. Коли хочеш закластися на них, ось тобі квиток. Я його вже заповнив. Певний виграш.

Тім, слухаючи його, то білів, то червонів на лиці. Та помалу обличчя хлопцеве набуло природної барви - світло-смаглявої, як у небіжчиці мами,- і він відказав, іще трохи затинаючись:

- Адже ж... дітям, мабуть, не дозволяється грати на гроші...

Та незнайомець не відступався.

- Воно-то так,- погодився він,- але якраз на цьому іподромі не боронять грати й дітям. Я, звісно, не кажу, що тут це дозволяють, але принаймні дивляться крізь пальці. Ну то як, Тіме, береш цього квитка?

- Так я ж вас зовсім не знаю,- тихо відповів Тім. (Аж тепер він помітив, що незнайомець називає його на ім'я.)

- Зате я тебе знаю дуже добре,- сказав чоловік.- Я був знайомий із твоїм татом.

І Тім наважився. Хоч йому й не вірилося, що батько міг знатися з таким вичепуреним паном, але ж цей пан знає, як його, Тіма, звуть, то, певно, якось знав і татка.

Повагавшися ще трохи, хлопець узяв заповнений квиток, вийняв із кишені п'ять марок і рушив до каси. [18]

В ту мить із гучномовців оголосили, що починаються другі перегони, і картатий пан гукнув Тімові навздогін:

- Швидше, хлопче, а то касу зачинять! Ось побачиш, мій квиток щасливий!

Хлопець віддав дівчині-касирці гроші та заповнений квиток і дістав назад відрізаний купон. А коли обернувся до незнайомця, того вже й знаку не було.

Почалися другі перегони, і той кінь, що на нього поставив Тім, виграв, набагато випередивши всіх інших. Хлопець одержав таку купу грошей, якої ще зроду не бачив. І знову Тім то білів, то червонів, та цього разу з радощів і гордості. Сяючи, показував він усім свій виграш.

Та диво дивне - як близько одне від одного живуть радість і смуток! Враз Тімові знову згадався похований сьогодні тато. Адже татові ні разу не пощастило стільки виграти... Хлопцеві набігли на очі сльози, і він мимохіть привселюдно заплакав.

- Гей, малий, ну хто ж це плаче, маючи таке щастя! - пролунав раптом поруч нього чийсь голос - густий рипучий бас.

Тім обернувсь і крізь сльози побачив перед собою чоловіка з пом'ятим обличчям і в пом'ятому костюмі. Ліворуч того чоловіка дивився на Тіма згори вниз височенний рудий паруб'яга, а праворуч стояв гарно вбраний панок із великою лисиною і співчутливо розглядав хлопця.

Видно, всі троє були одна компанія, бо майже в один голос спитали Тіма, чи не хоче той відсвяткувати свій виграш, випивши з ними лимонаду.

Тім, трохи сторопілий від такої приязності й від багатства, що так несподівано звалилося на нього саме в цю неділю, кивнув головою, схлипнув іще раз і сказав:

- Тільки отам позаду, в садочку.

Бо там він часто пив лимонад зі своїм татом.

Троє чоловіків сказали: «Добре, малий, ходімо в садочок!» [19]

У садочку вони посідали за стіл під великим, старезним каштаном. Незнайомець, що йому Тім завдячував свій виграш, більше не показувався. І Тім скоро забув про нього, бо троє чоловіків за столом, що замовили собі пива, а Тімові лимонаду, розважали щасливчика-новобагатька дивовижними штуками. Рудий довгань поставив собі на ніс повний кухоль пива й балансував ним, не розливши й краплини; чоловік у пом'ятому костюмі й з пом'ятим обличчям видобув із кишені колоду карт і щоразу безпомилково витягав із неї кожну карту, що її навмання називав Тім. А лисий панок робив фокуси з Тімовими грішми. Він загорнув їх усі в хусточку, туго скрутив її, тоді розгорнув - і грошей там уже не було.

Лисий захихотів і сказав Тімові:

- Ану, сягни-но в свою ліву кишеню!

Тім застромив туди руку й здивувався: всі гроші справді лежали там.

Неділя таки була дивовижна. Ще о другій годині Тім блукав містом безмірно нещасливий, а тепер, о п'ятій, він сміявся так щиро, так весело, як не сміявся вже давно-давно. Він аж похлинувся кілька разів зо сміху. Троє його нових приятелів подобались йому надзвичайно. Він дуже пишався тим, що знайшов собі трьох дорослих знайомих, та ще людей такого рідкісного фаху. Бо пом'ятий чоловік був друкар і друкував гроші, рудий - майстер-гаманник, а лисий назвався якимсь букмекером. Тім не знав, що воно й таке.

Коли підійшов офіціант і Тім гордо простяг йому гроші за всіх, троє чоловіків усміхаючись, але рішуче відхилили його руку. За всіх заплатив лисий панок, і за Тімів лимонад теж. Отже, коли Тім попрощався зі своїми новими приятелями, виграш його лежав у кишені ще цілісінький.

А на трамвайній зупинці перед Тімом раптом знову де не взявся картатий чоловік. Він сказав Тімові навпростець: [20]

- Тіме, Тіме, який же ти недотепа! І знов у тебе нема ні шеляга!

- Помиляєтесь, добродію! - засміявся Тім.- Ось де мій виграш! - Він вийняв з кишені паку грошей, показав незнайомцеві й, повагавшись трохи, додав: - Це все ваше...

- Ті гроші, що ото у тебе в руці, на варті й дірки з бублика,- презирливо мовив незнайомець.

- Але ж я їх у касі одержав! - вигукнув Тім.- Не вірите?

- У касі, хлопче, ти одержав добрі гроші. Але оті троє в садку, я певен, підмінили їх тобі на фальшиві. Я тих людців знаю. Шкода, що я запізно побачив тебе в їхньому товаристві, Тіме. Перше ніж я встиг утрутитися, вони повшивалися. То шахраї.

- Не може бути, пане! Один з них майстер-гаманник...

- Авжеж, майстер тягати гаманці з чужих кишень. Кишеньковий злодій.

- Злодій? - спантеличено перепитав Тім.- А той другий, що гроші друкує?

- Фальшиві гроші він друкує!

- А третій? Той, як його... букмек?..

- Букмекер - це той, хто влаштовує заклади на перегонах. Але твій букмекер намовляє людей на незаконні, недозволені заклади.

Тім ніяк не хотів у те повірити, і врешті картатий пан вийняв зі свого гаманця банкноту й порівняв її з однією Тімовою. І справді, на Тімових грошах, коли дивитися проти світла, не видно було водяних знаків.

- Тепер ти бачиш, що я правду кажу, Тіме?

Хлопець пригнічено кивнув головою. Тоді несподівано шпурнув свої гроші додолу й заходився люто топтати їх ногами. Якийсь поважний дідусь, що саме проходив повз них, здивовано витріщив очі на Тіма, на гроші, на картатого пана й раптом кинувся навтіки, мовби за ним сам чорт гнався. Картатий якусь хвилину [21] мовчав. Потім вийняв із кишені п'ять марок, дав їх сторопілому хлопцеві й сказав, щоб той другої неділі, коли хоче, знову прийшов на іподром. А тоді швидко зник Тімові з очей.

«А чому ж він сам не грає?» - подумав був Тім, та враз і забув про ту думку. Сховав п'ять марок у кишеню й пішки подався додому. А фальшиві гроші так і зостались на тротуарі.

На превелике диво, мачуха не набила його, хоч він вернувся дуже пізно, та ще й утік із батькового похорону. Вона просто не дала йому вечеряти й тільки буркнула, щоб лягав спати. А Ервінові дозволено було ще посидіти з поминальниками, що дивилися на Тіма мовчки і якось чудно.

За тією химерною неділею потягся довгий, сумний тиждень. Мачуха лупцювала Тіма, як звичайно, а вчитель картав його ще частіше, ніж звичайно, бо Тім був украй неуважний на уроках. Хлопець весь час думав, іти йому в неділю на іподром чи не йти. П'ять марок, щоб не відняв Ервін, Тім сховав у шпарці між двома цеглинами в стіні сусіднього будинку. І щоразу, проходячи повз те місце, мимоволі сміявся. Його тішила думка, що йому, можливо, ще раз пощастить виграти на перегонах.

Третій аркуш

ВИГРАШ І ВТРАТА

Коли настала довгождана неділя, Тім уже з самого ранку знав, що по обіді таки побіжить на іподром. Тільки-но годинник у світлиці вибив третю годину, хлопець викрався надвір, виколупав із шпарки свої п'ять марок і як шалений помчав до іподрому.

Біля брами Тім налетів на якогось чоловіка. То був не хто інший, як картатий незнайомець.

- Гопля! - вигукнув картатий.- Що, дочекатися не можеш?

- Вибачте! - насилу вимовив захеканий Тім. [22]

- Нічого, нічого! Я чекав на тебе. Ось тобі квиток. П'ять марок іще цілі?

Тім кивнув головою й видобув монету з кишені.

- Добре, добре! Йди до каси, плати й одержуй купон. Якщо виграєш, підожди мене тут, біля брами. Я хочу про щось із тобою поговорити.

- Гаразд, пане!

Отож Тім поставив п'ять марок на того коня, що його вписав у квиток картатий незнайомець, і коли скінчилися перегони, виграв купу грошей, як і минулої неділі. Але цього разу відійшов від каси швиденько, не показуючи нікому свого виграшу. Запхав гроші у внутрішню кишеню курточки, силкуючись якомога удавати байдужого, й вибрався з іподрому крізь дірку в паркані. З картатим паном йому не хотілося зустрічатись. Цей чоловік став йому якийсь страшнуватий. А крім того, він же подарував Тімові й п'ять марок і квиток!

За іподромом прослалася зелена лука, а по ній росли поодинокі дуби. Тім ліг на траву за грубезним дубом і замріявся про те, що ж він зробить із своїм багатством. Він хотів на ті гроші привернути до себе всіх, усіх - і мачуху, і зведеного брата, й учителя, й своїх однокласників. А батькові Тім надумав поставити мармуровий надгробок. І хай на ньому буде написано золотими літерами: «Від твого сина Тіма, що ніколи тебе не забуде».

А як зостануться ще гроші, Тім купить собі самокат - такий, як у пекаревого сина, на гумових шинах, із сиреною. Хлопець отак собі снив із розплющеними очима, аж поки стомився від тих мрій і заснув насправді. Про картатого пана він уже й не згадував. Та коли б Тім побачив його в ту хвилину, то, напевне, дуже здивувався б, бо химерний незнайомець саме розмовляв із тими трьома чоловіками, що минулої неділі запросили хлопця відсвяткувати виграш і обікрали його. Та, на своє щастя - чи то пак нещастя,- Тім того не бачив. Він спав. [23]

Розбудив його чийсь різкий голос. То був якраз голос картатого пана. Він стояв у ногах у Тіма, дивився на хлопця й питав не дуже приязно:

- Ну, виспався?

Тім кивнув головою, ще трохи очманілий після сну, підвівся, сів на траві й про всяк випадок лапнув зверху за кишеню. Вона видалась йому дивно порожньою. Хлопець квапливо застромив туди руку й ураз пробуркався зовсім: кишеня справді була порожня. Гроші зникли.

Картатий пан зловтішне вишкірив зуби.

- Г... г... гроші у вас? - затинаючись, спитав Тім.

- Ех ти, сонько! Гроші в одного з тих трьох шахраїв, що ти з ними пив-гуляв тієї неділі! Він таки підстеріг тебе. А моя доля, мабуть, уже така, що я приходжу запізно. Вгледівши мене, він чкурнув навтіки. От тому я тебе й помітив.

- А куди він побіг? Треба покликати поліцію!

- Я не люблю синіх мундирів,- відказав картатий.- Вони мені надто вульгарні. Та й однаково злодій уже хтозна-де, його не спіймаєш. Та встань, нарешті, хлопче! І гайда додому. А другої неділі приходь знову!

- Та ні, мабуть, я більше не прийду,- відповів Тім.- Не може людині стільки щастити. Мені це ще тато казав.

- Таж кажуть, ніби щастя й нещастя завжди буває тричі вряд. А ти, напевне, хотів би щось собі купити, правда?

Тім кивнув головою.

- От і купиш, якщо прийдеш у неділю сюди та облагодимо з тобою одну комерційну справу! - Незнайомець глянув на годинника й раптом заквапився: - Ну, бувай, до неділі,- й побіг.

А Тім поплентався додому, і в голові йому мішалося. А вдома його чекала прочуханка та Ервінова зловтіха.

І знову потягся довгий-довгий тиждень. Але весь цей тиждень Тім був на диво бадьорий. Хоч картатий незнайомець і здавався йому трошечки страшнуватий, [24] але хлопець уже твердо вирішив погодитись на ту комерційну справу. Адже комерційна справа, міркував собі хлопець, це щось порядне, законне. Не те, що виграти цілу купу грошей за п'ятимаркову монету, та ще й знайдену. Ні, в комерційній справі кожен щось віддає й щось одержує навзамін.

Може, це й дивно, що хлопчик-п'ятикласник міркував так серйозно. Але в убогих вузьких завулках, де людям, щоб прожити, доводиться бути ощадливими, вони змалку привчаються серйозно дивитись на гроші та на комерційні справи.

Думка про наступну неділю цілий тиждень допомагала Тімові терпіти всі прикрощі. Часом він міркував собі: а може, то батько попросив картатого пана наглядати за Тімом, коли з ним самим щось скоїться? Але за хвилину Тімові вже думалося, що батько, певне, обрав би для того якусь приємнішу, ласкавішу людину.

Та однаково Тім ладен був погодитись на ту комерційну справу, і думка про неї тішила його. Він раптом ніби знову навчився сміятись так, як колись. І той сміх подобався всім людям. У Тіма враз стало так багато друзів, як не було ще ніколи.

Дивна річ: цей хлопець, що не міг здобути собі друзів ніякими зусиллями, ніякою послужливістю, запобігливістю, лестощами,- цей самий хлопець майже кожного схиляв до дружби самим своїм сміхом. Чи, принаймні, подобався всім. Йому тепер вибачали навіть такі пустощі, за які раніш карали. Раз якось Тім посеред уроку раптом згадав, як спрожогу налетів на картатого пана біля іподромної брами, і від того спогаду зненацька засміявся, як дзвіночок. Після свого кумедного «ік!» він зразу ж похопився, що сидить на уроці, й злякано затулив рота рукою. Але вчителеві й на думку ре спало сваритись на нього. Тімів сміх був такий несподіваний та веселий, що за Тімом мимоволі зареготав увесь клас. І навіть учитель. Він тільки підняв угору палець і сказав: [25]

- Я знаю, Тіме, що залпи сміху - це краще, ніж залпи з гармат, але ще краще на уроках не давати ніяких залпів!

На перервах хлопці один поперед одного пхалися гратись із Тімом; навіть мачуха й Ервін тепер іноді заражались Тімовим сміхом.

Щось незбагненне зробив із Тімом картатий незнайомець, але Тіма та нова незбагненність не бентежила, бо він її і не відчував.

Хоч життя в завулку багато чого навчило Тіма, все ж він був іще дитина, простосерда й довірлива дитина. Тім не помічав, що сміх його подобається людям і що від того дня, як помер батько, він ховав той сміх, мов скнара своє золото. Він думав собі по-дитячому, ніби пригоди на іподромі зробили його розумнішим і тому він тепер уміє краще ладнати з усіма людьми. Та, на жаль, такі думки тільки зашкодили Тімові. Коли б він був уже тоді знав, який коштовний скарб його сміх, то багато чого лихого не звідав би в житті. Але ж він був іще дитина...

Одного разу, вертаючись додому зі школи. Тім зустрів картатого незнайомця на вулиці. Хлопець якраз спостерігав джмеля, що пробував сісти на вухо сонному котові. Було то дуже кумедне видовище, і Тім знову весело засміявся. Та коли він упізнав незнайомця з іподрому, всю веселість із нього мов вітром звіяло. Тім чемненько вклонився і сказав: «Добридень».

Та незнайомець удав, ніби й не бачить хлопця. Тільки буркнув, минаючи його: «У місті ми незнайомі»,- і пішов далі, навіть голови не повернувши.

«Це, мабуть, так і треба в комерційних справах»,- подумав Тім і в ту ж мить знову засміявся, бо джміль був сів котові на вухо, а кіт тіпнув ним і злякано підскочив. Товстун-джміль, сердито гудучи, полетів геть, а Тім, висвистуючи, почимчикував додому. [26]

Четвертий аркуш

ПРОДАНИЙ СМІХ

Настала довго сподівана неділя. Тім хотів утекти на іподром раніше, ніж звичайно. Але, на його біду, о пів на третю мачушин погляд випадково впав на календар, і вона раптом згадала, що сьогодні роковини її весілля: цього числа й місяця вона одружилася з Тімовим татом. Мачуха пустила кілька сльозин (плакати їй було як з гори котитись), і враз стало треба зробити тисячу різних справ: віднести квіти на могилку, купити печива, намолоти кави, запросити сусідку, випрасувати нову сукню, перевдягтися; Тімові наказано було почистити всі черевики, а Ервінові - піти купити квіти.

Тім радніший був би збігати по квіти та віднести їх на кладовище: якби поспішити, то можна було б іще й на перегони встигнути вчасно. Але схвильованій мачусі (а їй хвилюватись було як мед їсти) перечити не випадало, бо вона розхвильовувалась іще дужче й нарешті падала, ридаючи, в крісло, а тоді вже хоч-не-хоч доводилось її слухатись. Отож Тім і не пробував відмагатись, а слухняно пішов до пекаря по тістечка. («У задні двері! Тричі постукай! Скажи, що дуже треба!»)

Тіма нітрохи не спантеличило сердите пекаришине обличчя. («Не зважай, що вона бурчатиме! І не вертайся без тістечок! Не відступайся, поки стара відьма не дасть, що треба!») Він до слова переказав пекарисі мачушине замовлення («Шість марципанів! Та гарних! Свіженьких! Так і скажи!») Та, на жаль, від пекарихи почув таку відповідь, до якої мачуха його не підготувала. Бо пані Бебер - так звали пекариху - сказала:

- Поки старого боргу не заплатите, не дам нічого! Так і скажи вдома! Коли купила немає, то нічого й купувати марципани! Так і скажи! На двадцять шість марок печива! Цікава я знати, хто його у вас жере стільки! Директорова жінка з гідростанції й то стільки не купує! А в директора люблять ласо їсти, ого-го, хлопче! [27]

Тім хвилинку стояв німий із подиву. Йому, звісно, вряди-годи перепадав удома кренделик чи півтістечка. Але на двадцять шість марок печива... Це ж цілі гори! Невже мачуха потай від нього й Ервіна напихається солодким печивом, як покличе сусідку на каву? Тім знав, що вона часто сидить пащекує з сусідкою на кухні, поки вони з Ервіном у школі. Чи, може, то Ервін стільки набравсь у пекарихи?

- То мій брат набрав стільки? - спитав хлопець.

- І він теж,- буркнула пані Бебер.- Та найбільше брала твоя мати, чи то пак мачуха, собі на сніданочок. А ти, мабуть, і не знав про те нічого, еге?

- Та знав, знав,- похопився Тім.- Чого там не знав!

Однак насправді нічого він не знав. Він навіть не обурився й не розсердився, йому тільки сумно стало, що мачуха ласувала потай від нього, та ще й у борги залізла.

- Ось як,- додала пані Бебер.- Нічого я тобі не дам. Іди додому й перекажи те, що я сказала. Чуєш?

Але Тім не зрушив з місця. («Не зважай, що вона бурчатиме! І не вертайся без тістечок! Не відступайся, поки стара відьма не дасть, що треба!») Він сказав:

- Адже ж сьогодні роковини татового з мамою весілля. Чи то пак із мачухою. Та й...- Тімові раптово пригадався іподром, перегони, загадкова комерційна справа з картатим паном, і він квапливо додав:

- Та й однаково ввечері я вам, пані Бебер, принесу гроші. Й за ті марципани, що візьму зараз, теж заплачу. Ось побачите!

- Ти принесеш гроші?

Пані Бебер завагалась була, але щось у хлопцевому голосі неначе впевнило її, що він таки справді принесе гроші, хоч, може, й не всі зразу.

Про всяк випадок вона спитала;

- А де ж ти їх візьмеш?

Тім скорчив страшне обличчя, немов у розбійника [28] в ляльковому театрі, й сказав якомога товстішим голосом:

- Украду, пані Бебер! У директора гідростанції!

Хлопець так добре вдав розбійника, що пані Бебер засміялася, полагіднішала, і, одне слово, Тім одержав свої шість марципанів, ще й на додачу сьомий задарма.

Мачуха саме стояла на дверях, коли Тім вернувся з тістечками. Ще й досі (чи вже знову) розхвильована, вона заторохтіла без ком і крапок:

- Требабуломенісамійпіти! Казалащосьтавідьмапроборг? Приністістечкачині? Чогомовчиш?

Тім краще відкусив би собі язика, ніж переказав свою розмову з пані Бебер. Крім того, йому вже час було поспішати на іподром, а з мачухою як заведешся, то не скоро відкараскаєшся. Тому він тільки сказав:

- Ще й задарма один марципан дала. Можна мені йти гуляти, ма? (Він так ніколи ні разу й не спромігся сказати на мачуху «мамо».)

Мачуха навдивовижу охоче відпустила його і навіть дала один марципан із собою:

- Навіщо тобі слухати жіночі балачки, воно тобі зовсім не цікаве. Йди собі гуляй, тільки не дуже допізна.

І Тім щодуху подався до іподрому. Та хоч як він поспішав, а проте встиг дорогою вм'яти марципан, тільки тричі ляпнувши начинкою,- щоправда, один раз на сині святкові штанці.

Картатий незнайомець уже стояв біля брами. Хоч перші перегони давно почались, він не виявляв ані найменшої нетерплячки чи досади. Навпаки, сьогодні він був сама приязнь та ласкавість. Відразу потяг Тіма до садочка при кав'ярні й замовив йому лимонаду та ще один марципан. Самі марципани та й марципани! Цілу неділю. А картатий пан із якнайсерйознішим обличчям витинав такі жарти, що Тім аж качався зо сміху.

«А він таки непоганий дядько,- думав хлопець.- Тепер я розумію, чому тато його вподобав».

До того ж незнайомець тепер дививсь на нього лагідними, привітними карими очима. Якби Тім був спостережливіший, [29] він мусив би помітити, що досі цей пан мав холодні водяво-блакитні очі, наче риб'ячі. Але Тім був не дуже спостережливий. То лиш пізніше життя навчило його все помічати.

Нарешті картатий пан заговорив про свою справу.

- Любий мій Тіме,- сказав він,- я дам тобі стільки грошей, скільки ти захочеш. Я тільки не можу викласти їх готівкою на стіл. Але я можу наділити тебе здатністю вигравати будь-який заклад. Будь-який, розумієш?

Тім кивнув головою трохи розгублено. Однак слухав дуже уважно.

- Звісно, цю здатність я дам тобі не задарма. Така здатність, сам розумієш, має велику ціну!

Тім знову кивнув головою. Тоді збуджено спитав:

- А що ж ви за неї хочете?

Хвилинку незнайомець замислено дивився на Тіма, певне, вагаючись.

- Що... я... за... неї... хо-чу, пи-та-єш? - промовив [30] він нарешті, розтягуючи слова, мов гумку-жуйку. Та потім вони посипались йому з рота так швидко, що годі було й добрати: - Яхочущобтивіддавменізанеїсвійсміх!

Він, мабуть, і сам помітив, що говорить занадто швидко й незрозуміле, бо проказав іще раз повільніше:

- Я хочу, щоб ти віддав мені за неї свій сміх!

- Оце й усе? - спитав Тім сміючись. Але в ту мить він помітив, як чудно, майже сумно дивляться на нього карі незнайомцеві очі, й Тімів сміх урвався навіть без звичайного «ік!» на кінці.

- Ну, то як? - спитав картатий.-Згода?

Тім опустив погляд і побачив перед собою марципани на блюдечку. Йому згадалася пані Бебер, борг, що він обіцяв заплатити, тоді спало на думку, скільки всякої всячини можна купити, маючи багато грошей. І він сказав:

- Коли це насправді, то я згоден.

- От і добре. В такому разі треба підписати угоду.- Картатий пан витяг із кишені якийсь папір, розгорнув його й поклав на столі перед Тімом.- Прочитай уважно!

Тім почав читати:

«І. Цю угоду укладено між паном Т. Трочем з одного боку й паном Тімом Талером з другого боку « »...19.. року в місті... й підписано в двох однакових примірниках обома сторонами».

- Якими це сторонами? - спитав Тім.

- Це так в угодах називаються особи, що ті угоди між собою укладають.

- А...- І Тім почав читати далі:

«2. Пан Тім Талер згідно з цією угодою передає панові Т. Трочеві свій сміх у цілковите й необмежене розпорядження».

Прочитавши вдруге слова «пан Тім Талер», хлопець здався собі майже дорослим. Він ладен був підписати угоду задля самих цих трьох слів. Він і гадки не мав, [31] як змінить усе його життя оцей коротенький другий пункт угоди.

Далі в ній стояло:

«З. На оплату вищезазначеного сміху пан Т. Троч зобов'язується дбати про те, щоб пан Тім Талер вигравав будь-який заклад чи то заставу. Цей пункт чинний без жодного обмеження».

Тімове серце забилося ще дужче. Він читав далі:

«4. Обидві сторони зобов'язуються тримати цю угоду в цілковитій таємниці».

Тім кивнув головою, не підводячи очей.

«5. У тому разі, коли одна зі сторін повідомить про цю угоду якусь третю особу чи осіб і таким чином порушить установлене пунктом 4 зобов'язання, друга сторона залишає за собою здатність відповідно: а) сміятися або ж б) вигравати заклади, тим часом як винна сторона цю здатність а) сміятися або ж б) вигравати заклади втрачає цілковито».

- Оцього я не втямлю,- промовив Тім, наморщивши лоба.

Пан Т. Троч - тепер ми нарешті знаємо, як його звуть,- пояснив йому:

- Розумієш, Тіме, коли ти порушиш обіцянку мовчати й розкажеш кому-небудь про цю угоду, то вже не могтимеш вигравати заклади, але й сміху свого назад не одержиш. Коли ж про неї пробалакаюсь я, то ти одержиш свій сміх назад, проте виграватимеш заклади й далі.

- Тепер зрозумів,- відказав Тім.- Це означає: мовчи, то будеш багатий, тільки не сміятимешся. А пробалакаєшся - будеш бідний і однаково не сміятимешся.

- Атож. Ну, читай далі.

І Тім прочитав далі: [32]

«6. У тому випадку, коли пан Тім Талер програє який-небудь заклад, пан Т. Троч зобов'язується повернути панові Тімові Талерові його сміх. Однак у такому разі пан Тім Талер надалі втратить здатність вигравати заклади».

- Це означає...-хотів був пояснити пан Троч. Але Тім уже зрозумів усе сам:

- Знаю, знаю! Коли я програю який-небудь заклад, то дістану свій сміх назад, але закладів більше не виграватиму.- І швидко перебіг очима останній пункт:

«7. Ця угода набуває чинності з того моменту, коли обидві сторони поставлять свої підписи під обома її примірниками.

Місто... « »... 19.. року».

Панів Трочів підпис уже стояв ліворуч. Подумавши трохи. Тім вирішив, що угода цілком законна. Він вийняв із кишені недогризок олівця й хотів був підписати, та пан Троч зупинив його.

- Це треба підписувати чорнилом,- і він подав Ті-мові свою авторучку.

На вигляд та ручка була з чистого золота, а на дотик якась химерно теплувата, наче наповнена теплою водою. Та хлопець не помітив ні золота, ні тієї теплоти. Він уже думав тільки про своє багатство, а тому не вагаючись підписав обидва папери. Чорнило в ручці виявилося червоне.

Тільки-но Тім підписав угоду, як пан Троч вельми приємно засміявся й сказав: «Щиро дякую». Тім відказав: «Будь ласка!» - і теж спробував засміятись, але не спромігся навіть усміхнутися. Губи його мимохіть стислися, й рот зробився тоненькою рискою, наче бритвою прорізаний.

Пан Троч узяв один примірник угоди, згорнув його й сховав у внутрішню кишеню піджака. А другий простяг Тімові зі словами: [34]

- Сховай як слід! Коли через твоє недбальство ця угода попадеться комусь на очі, це означатиме, що ти не дотримав таємниці. І з того може вийти велика біда для тебе!

Тім кивнув головою, також згорнув свій примірник і заховав у кашкет, під надпороту з одного боку підшивку.

Потім картатий пан поклав перед ним на столі дві монети по п'ять марок і сказав:

- Оце буде підвалина твого багатства!

І знову засміявся Тімовим сміхом. Але враз чогось дуже заквапився, покликав офіціантку, заплатив, підвівся, кинув: «Ну, щасти тобі, хлопче»,- і пішов.

Тімові довелось поспішати на іподром, бо от-от мали вже початися останні перегони. Він підбіг до каси, взяв квитка й не довго думаючи поставив гроші на кобилу Мавріцію-другу. Коли угода, що лежить у нього в кашкеті, дійсна, то ця кобила має виграти перегони.

І Мавріція-друга виграла.

Тім, що цього разу поставив десять марок, одержав кілька сотень. Озираючись, чи ніхто не дивиться, він сховав гроші в ліву внутрішню кишеню курточки й швиденько подався з іподрому.

П'ятий аркуш

ДОПИТ УВЕЧЕРІ

Аж вийшовши за браму. Тім знов обережно помацав кишеню, де лежали виграні гроші. Цупкий папір залопотів, і в Тіма шалено закалатало серце. Він, Тім Талер,- багата людина! Він може поставити надгробок татові. Він може заплатити борг пані Бебер. Він може щось купити мачусі й Ервінові, а собі, коли схоче, придбає самокат. На гумових шинах, із сиреною!

Щоб досхочу натішитися своїм щастям. Тім пішов додому пішки. Він радо купив би дорогою що-небудь для мачухи, але всі крамниці стояли зачинені - неділя. [35]

Свій виграш у кишені Тім міцно стискав рукою.

Дорогою йому стрілося троє його однокласників. Постояли, поговорили, тоді один спитав:

- Що там у тебе в кишені, Тіме? Жаба?

- Ні, паровозі - відказав Тім і хотів був усміхнутися, та знову губи його стислися в тоненьку рисочку.

Але хлопці того не помітили. Вони засміялися на ту відповідь і один вигукнув:

- Ану, покажи свого паровоза!

- Може, поїдемо на ньому в Гонолулу! - озвався другий.

Але Тім ще дужче стис гроші в жмені й відповів:

- Мені треба додому. Бувайте!

Та не так легко було від хлопців відкараскатись. Вони почекали, поки Тім трохи відійде, а тоді тихенько наздогнали його, підкралися ззаду, несподівано висмикнули йому руку з кишені - і самі сторопіли: в повітрі замелькали й посипались додолу гроші. Великі банкноти - по двадцять, по п'ятдесят, навіть по сто марок!

Хлопці здивувались - вони ж бо знали, що Тім живе в так званому бідняцькому передмісті.

- Звідки в тебе стільки грошей? - спитав один.

- Украв у директора гідростанції,- відповів Тім і хотів засміятись, хоч і був сердитий на хлопців. Але сміху в нього не вийшло, натомість він так зухвало ошкірився, що хлопці аж полякались. Вони справді повірили в те, що сказав Тім, і враз кинулись навтікача. Відбігши вже далеченько, хлопці загукали: «Тім Талер украв гроші! Тім Талер злодій!»

Тім чув те. Він зажурено позбирав гроші й запхав у кишеню. Тоді пішов до річечки, що текла через їхнє місто, сів там на лавочку й задививсь на качачу родину, що гуляла попід берегом.

Каченята чапали по травиці перевальцем, іще зовсім незграбно, і вчора Тім був би неодмінно посміявся з них. А сьогодні вони навіть не здались йому кумедними. І від того він ще дужче засмутився. Він дививсь [36] на каченят так, як дивляться на голу стіну, зовсім байдуже. І хлопець відчув, що цієї неділі він став зовсім не той.

Лиш як почало смеркати. Тім почвалав додому.

Звернувши до свого завулка, він побачив, що біля дверей їхнього дому стоїть мачуха з кількома сусідками. Вони про щось збуджено гомоніли. Та ледве вгледівши хлопця, сусідки сипнули врозтіч, мов кури, й позникали в своїх помешканнях. Однак двері скрізь лишились прочинені, і як Тім ішов завулком, у кожному вікні відхилялась завіска.

А мачуха стояла біля напіввідчинених дверей із такою міною, немов чекала кінця світу. З її обличчя, білого як крейда, стримів просто на Тіма гострий червоний ніс. Тільки-но хлопець підійшов, вона мовчки ляснула його по щоці, по другій і потягла в дім.

- Де гроші? - вереснула вона в сінях.

- Гроші? - перепитав сторопілий Тім.

І знову ляснули два ляпаси: Тімові аж у голові загуло й на очах виступили сльози.

- Давай сюди гроші, поганцю, злодюго! Гайда в кухню!

І мачуха поволокла хлопця за собою. А йому й досі невтямки було, що ж сталося. Проте він витяг гроші з кишені й поклав на кухонний стіл.

- Боже, та тут же тисячі! - жахнулась мачуха й витріщилась на Тіма, мов на яку потвору.

На щастя, в ту мить двері до кухні відчинились, і ввіпхалася засапана пані Бебер. За нею ввійшов і Ервін, що зразу вп'явся круглими з подиву очима в купку грошей на столі.

- Директора ніхто не обікрав! - важко хекаючи, оголосила пекариха.- Всі гроші в них цілісінькі.

Аж тепер Тім зрозумів, чому це його так зустріли вдома: він же вдень жартома сказав пані Бебер, що обікраде директора гідростанції. Та й хлопцям на вулиці бовкнув те саме. А вони ж бачили, що в нього ціла купа грошей. Отож, певно, й виказали його. Он воно що! [37]

Він хотів був пояснити все, та мачуха вже знов за-торохтіла без ком і крапок:

- Тозначитьневдиректора? Виходитьдесьінде! Детивкравгроші? Кажиправду! Покиполіціянеприбігла! Бовжевсявулицязнає! Кажиправду!

Тім сказав правду:

- Ніде я їх не вкрав!

Тоді на нього посипався цілий град ляпасів і запотиличників; урешті пані Бебер спинила мачушину ру-ку й тихо спитала хлопця:

- Тіме, хіба ти не казав мені сьогодні, що ввечері заплатиш борг за печиво?

- Боргзапечиво? Дочоготут боргзапечиво? - знову заверещала мачуха.

- Будь ласка, пані Талер, дайте мені спокійно поговорити з хлопцем,- відказала їй пекариха.

Мачуха, ридаючи, впала на стілець і вхопилась за Ервінову руку. Син скрививсь, але руки не відняв. А пані Бебер допитувалась далі:

- Тіме, скажи мені правду! Звідки ти знав, що ввечері матимеш стільки грошей?

Думки хурчали в Тімовій голові, як сполохані горобці: «Тільки не бовкнути нічого про пана Троча! Ні слова про угоду! А то вона зробиться недійсною!»

Нарешті він сказав, затинаючись:

- Я... колись давно... знайшов.. п'ятнадцять марок. І надумав піти з ними на іподром, де закладаються на перегонах,- він говорив уже знов упевнено й рів-но.- Я думав - може, що виграю... Поставив на кобилу Мавріцію-другу - й справді, бачте, виграв,- він показав на купку грошей на столі, тоді знайшов у кишені купон від квитка й поклав його на стіл до грошей.

Пані Бебер хотіла була подивитись на той купон, та мачуха вже вхопила вузеньку паперову смужечку. Вона розглядала її добрих хвилин п'ять, і всі в малесенькій кухоньці мовчали. Тім стояв випроставшись, Ервін несміливо позирав на нього збоку, а пані Бебер [38] згорнула руки на грудях і всміхалась. Аж ось мачуха кинула купон на стіл і підвелася.

- Виграні гроші - то теж нечесний заробіток! - промовила вона й вийшла з кухні.

Аж тоді й пані Бебер подивилась на папірець, покивала головою й сказала:

- Пощастило ж тобі, Тіме!

Мачуха крикнула з-за дверей: «Ервіне!» - і її мазунчик покірливо вийшов, не обізвавшись до Тіма й словом.

Хлопець, що продав свій сміх, здавався сам собі якимсь бридким шолудивцем. Насилу стримуючи сльози, спитав він у пані Бебер:

- І справді виграш - це нечесні гроші?

Пекариха не відповіла прямо, тільки сказала:

- А он Нойбауер, що на різниці працює, теж виграв гроші. В лотерею. Й купив за них собі дім. А хіба Нойбауери погані люди?

Тоді взяла з грошей на столі три банкноти по десять марок, вийняла з кишені на фартуху чотири марки решти, поклала на стіл і промовила:

- Борг заплачено, Тіме. Ну, не журися!

І пішла собі. Тім почув, як грюкнули за нею двері.

Він лишився в кухні сам. Образа, розпач і тяжкий смуток сповнювали йому серце. Подумавши трохи, Тім зібрав зі столу гроші, запхав їх у кишеню й рушив до дверей. Він вирішив піти геть із цього дому. Світ за очі.

Та в сінях його спинив мачушин голос:

- Зараз мені лягай спати!

Потім вона додала вже нерішучіше:

- Гроші поклади в буфет на кухні.

Тім помітив, що настрій у неї перемінився. Він послухавсь, відніс гроші назад у кухню й ліг у ліжко, голодний, схвильований і знесилений. Ліжко поряд було порожнє, Ервіна мачуха поклала біля себе.

Швидше, ніж можна було сподіватися, Тім поринув у важкий сон. [39]

Шостий аркуш

МАЛИЙ МІЛЬЙОНЕР

Пекарисі, пані Бебер, кілька днів по тому таки добре торгувалося, її крамничка майже цілий день була повна цікавих, що їм пекариха мусила розповідати про Тімів Талерів виграш. Оповідання те пані Бебер дуже спритно приправляла рекламою свого печива:

- ...А хлопець мені й каже, що хоче вкрасти гроші в директора гідростанції. Між іншим, у директора вдома та-ак люблять наші кренделі! Еге ж, так ото я й подумала, що зараз мене грець поб'є, як почула, буцім у хлопця тисячі в кишені. Я зразу святкову сукню на себе та до директора. Неділя ж була, та й однаково пані директрриха торт замовляла з написом: «Вітаємо з днем народження!» Знаєте, як мій чоловік торти робить! Еге ж... коли мені там кажуть, що ніщо в них не вкрадено. Люба моя пані Бебер, каже мені директор, я, каже, знаю, що ви розумна жінка, й ваші булочки з родзинками дуже добрі, але тут, певно, сталась якась помилка. В нас, каже, нічого не вкрадено, каже...-Отак вона торохтіла без угаву.

Тім став героєм дня. Серед сусідів, у школі, а почасти навіть і вдома. Мачуха - в неї раптом де не взявся біля пальта хутряний комір - стала поводитись із ним делікатніше, а зведений брат при кожній нагоді просто засипав його запитаннями про іподром та перегони. Сусіди називали його напівжартома, напівзаздрісно «малим мільйонером», а в школі хлопці аж билися за те, щоб бути ближче до нього.

Хлопця тішила така загальна увага. Він давно про бачив трьом своїм товаришам те, що вони поспішили виказати його, і мачусі - що вона його набила. Він радий був би тепер жартувати з усіма на світі. Але вже не міг... Коли Тім пробував усміхнутись, нічого в нього не виходило, він тільки зухвало вишкіряв зуби.

Скоро він облишив навіть спроби засміятись чи пожартувати. [40] Він звик напускати на себе поважність. А це, мабуть, найгірша річ, що може статися з дитиною.

Сусіди почали казати на нього: «Чванько!» Хлопці в школі, вдовольнивши першу цікавість, уже уникали його, і навіть мачуха, що стала тепер трохи не така нервова, називала його мурмилом.

А втім, вона вже не казала, що виграш - то нечесний заробіток. Вона тепер вважала кінські перегони цілком законною й почесною справою. Вона навіть спитала Тіма, чи не візьме він із тих грошей двадцять марок, щоб у неділю сходити ще раз на іподром та поставити гроші на якого-небудь коня.

Тім ще досі не одержав зі свого виграшу ані пфень га й мусив поки що поховати свої мрії про мармуровий надгробок батькові та про самокат, затявся й не взяв грошей. Після тієї історії з боргом за печиво, він почав дивитись на мачуху іншими очима.. Він їй більше не вірив. А це теж погана річ для дитини.

Того тижня Тімові вперше в житті хотілося, щоб неділя не настала зовсім. Він боявся, що мачуха таки вмовить його піти на іподром. І боявся недарма. Вже в суботу за вечерею почалося: «Може, тобі ще масла, Тіме?.. А знаєш, кажуть, ніби як щастить виграти, то тричі підряд... Ну, нічого, до завтра ще є час. Ти ще подумаєш, чи йти, чи не йти...»

І Тім, звичайно, пішов! Не тільки тому, що Ервін із мачухою ще за сніданком почали закидати про перегони, а й тому, що хотів випробувати угоду, оту чудернацьку угоду, сховану в кашкеті під підшивкою. Він уже не був певен, чи то справжня, чесна угода, чи підле ошуканство.

На іподром поїхали трамваєм усі троє. У Ервіна від хвилювання повиступали на блідих щоках червоні плями, а мачуха цілу дорогу лопотіла без ком і крапок про ризик, про іподромні махінації, про зависокі ставки. Вона дала Тімові двадцять марок, сто разів сказавши: «Гляди ж не загуби!» - і ще додала: [41]

- Та не став на Фортуну, Тіме! Я чула в трамваї, що Фортуна не має ніяких шансів. У неї якась там коняча хвороба, чи що. Чуєш, не на Фортуну!

Звичайно, Тім тепер вирішив поставити якраз на Фортуну. Однаково з угодою в кашкеті програти він не може. А крім того, треба показати мачусі, що він краще за неї знається на таких справах.

Та, прийшовши на іподром, мачуха з Ервіном майже забули про Тіма. Надто їх захопило все, що вони бачили довкола: розкішне вбрані дами та елегантні пани, породисті коні на доріжках, невеличкі жокеї в червоних кашкетиках, галас і штовханина біля кас та біля бар'єрів...

- А ти не хочеш дивитися на перегони? - спитала мачуха, коли Тім здав до каси заповненого квитка й одержав купон.

Хлопець похитав головою.

- А на якого коня ти поставив? - спитав Ервін.

- На Фортуну! - голосніше, ніж треба, відповів Тім.

Мачуха аж скинулась:

- На Фортуну? Алеж ятобіказала, щоцяконяка... я чула втрамва-аї...

Стартовий постріл урвав її лопотіння. Затупотіли підкови, загаласували, загукали глядачі, й мачуха з Ер-віном кинулись уперед, щоб хоч через голови, через циліндри, капелюхи та вуалі побачити перегони. Вони поставали недалеко від Тіма, що сів собі на траву, й Ервін раз по раз збуджено гукав до нього:

- Фортуна йде на третьому місці! - кричав він. Потім: - Фортуна наздоганяє! - І нарешті радісно заверещав: - Фортуна попереду!

Та скоро Фортуна знову відстала, видимо, знесилившись, і Ервін закричав:

- Пропали наші гроші! Сплохувала Фортуна!

Тоді й мачуха повернула до Тіма голову, і погляд її ніби промовляв: «Яжтакізнала! Непослухався-мене!» [42]

Та перед самим фінішем Фортуна знову припустила так, що аж не вірилося. Ервін зарепетував як несамовитий:

- Так, так, так, Фортуно! Добре! Ну, Фортуночко, ну, ну, ну!

І весь натовп загукав щодуху:

- Фортуна! Фортуна! Фортуна!

Потім по юрбі перебіг голосний крик, і Тім зрозумів: Фортуна перемогла! І пан Троч також переміг!

Треба сказати. Тім сів осторонь іще й того, що сподівався побачити пана Троча. Але з-під тих небагатьох картатих кашкетів, що він угледів, дивились на нього зовсім незнайомі обличчя. Троча не було видно ніде. (І все ж він, хоч уже й не в картатому вбранні, був на іподромі й нишком, ховаючись у натовпі, спостерігав Тіма примруженими очима.)

Ось надбіг задиханий Ервін.

_ Ми виграли! - загорлав він.- Давай сюди купон, Тіме!

Але Тім не віддав йому купона. Аж дочекавшись, поки люди біля кас розійшлися, він підійшов туди й одержав свій виграш - цілих дві тисячі марок!

- Багатенько ми виграли,- сказав Тім і простяг гроші мачусі.- Тут має бути дві тисячі.

- Атиперелічив, Тіме? Тамрівно двітисячі?

- Та рівно, рівно,- відказав хлопець.

- Те-те-те! Дайсюдияперелічу! - Вона майже вирвала гроші з пасинкових рук, почала перелічувати їх, збилася, почала спочатку і, перелічивши, сказала: - Справді, рівно дві тисячі!

Усі повмовкали. Мачуха не спускала ока з жмутка грошей у руці, Ервін стояв роззявивши рота, а Тім на-пустив на себе звичайний свій поважний вигляд.

Нарешті мачуха порушила мовчанку:

- Щожмитепер робитимемо зтакоюгрошвою!

- Не знаю,- знизав плечима Тім.- Робіть, що хочете, то все ваше. [43]

Тоді мачуха раптом заплакала - чи то з радості, чи з несподіванки, чи від зворушення, чи з усього того заразом. Схлипуючи, вона поцілувала обох хлопців, витерла хусточкою очі й сказала:

- Ну, ходімо, діти! Треба ж відсвяткувати таке щастя!

І знову Тімові довелося сісти під тим самим крислатим каштаном у садочку біля кав'ярні, під яким він сидів іще з татом, потім із шахраями, а востаннє - з картатим паном Трочем.

Мачуха жваво цокотіла:

- Яжтакізнала, щотімнедарма постав ивнафортуну! Охтихитрунчику! - І вона вщипнула Тіма за вухо. Потім замовила лимонаду й тістечок. Щоправда, не марципанів.

Ервін весь час говорив про іграшкову залізницю та про жовті черевики на каучуковій підошві. Лиш Тім сидів німий як риба - хлопець, що більше не вмів сміятися.

Сьомий аркуш

БІДНИЙ БАГАТІЙ

Відтоді Тім мусив щонеділі йти з мачухою та Ервіном на іподром. Йому дуже не хотілося того робити, і він інколи навіть прикидався хворим. Або втікав рано з дому й повертавсь аж пізно ввечері. Тоді мачуха з Ервіном ішли на іподром самі. Але їм ніколи не щастило. В найкращому разі вони вигравали якихось кілька марок.

Тож Тімові й доводилося знову й знову йти з ними на іподром і закладатись на дедалі більші суми. На іподромі вже пальцями на нього показували, і навіть прислів'я таке пішло поміж грачами: «Щастить, як Тімові!»

Правда, в хлопця вистачило розуму комбінувати так, що раз він вигравав більше, а раз менше. Коли Тім ставив на того коня, що на нього поставило вже багато [44] людей, то виграш був невеликий. А коли на якого-небудь невідомого, непримітного коника, що на нього майже ніхто не ставив,- тоді Тім вигравав дуже багато.

Мачуха спершу була оголосила, що всі виграні гроші належать Тімові й що вона тільки порядкує ними як опікунка, але дуже скоро почала казати «наш виграш», «наші гроші», «наш рахунок». А Тімові завжди давала тільки трохи кишенькових грошей. І все ж навіть із тих грошей хлопець зумів наскладати на мармуровий надгробок батькові. Обмінявши дрібні гроші на паперові, він ховав їх у великому годиннику, що стояв у вітальні: Тім випадково відкрив, що футляр годинника має подвійне дно й верхню дощечку можна піднімати.

Несподіване багатство зовсім закрутило мачусі голову. Дуже скоро весь завулок зненавидів її. Своїй давній приятельці-сусідці мачуха сказала просто в вічі, що соромиться показуватись із нею на вулиці - ти, мовляв, не по моді вбрана. (До того, щоб купити та подарувати біднішій приятельці сукню, мачуха, видно, не додумалась.) Печиво пані Бебер вона стала привселюдно гудити й почала купувати куди дорожчі ласощі в центрі міста. (А що пані Бебер, бувало, давала їй цілі гори печива наборг, вона, видно, зовсім забула.)

Ервін, що йому мати потихеньку тицяла більше кишенькових грошей, ніж Тімові, став корчити з себе панича. Він носив черевики на грубезній підошві, довгі штани, строкаті краватки, почав нишком курити й удавав знавця породистих коней.

І лише Тім потай проклинав усе те багатство. Він часто годинами блукав по всьому місту, сподіваючись зустріти де-небудь пана Троча, бо мав надію, що картатий пан віддасть йому його сміх, коли він зречеться свого багатства. Але Троч не показувався.

А проте картатий пан не спускав хлопця з очей. Іноді поблизу Тімової оселі проїздила велика розкішна машина. Ззаду в цій машині сидів чоловік у картатому кашкеті. Загледівши де-небудь Тіма, він наказував шоферові зупинитись і стежив за хлопцем - може, й не [45] з боязким, але зі стурбованим виразом на обличчі. Цей чоловік подбав і про те, щоб до помешкання в завулку попав примірник рекламного календаря, де поміж віршиками, що вихваляли каву, какао та масло, були надруковані мудрі вислови уславлених людей. І не випадково на першій сторінці стояло:

«Комерційну угоду треба розглядати як шлюбний контракт: довго й ретельно зважувати, перше ніж підписати Ті, але потім дотримуватись угоди твердо й вірно.

Т. Троч».

На Тімове щастя, мачуха зразу вирізала той аркуш, бо на звороті писалося про астрологію. (Мачуха народилася під сузір'ям Скорпіона.)

Згодом хлопцеві стала тяжко дошкуляти ворожість усього завулка. Постійну Тімову поважність люди вважали за пиху, чваньковитість і дивились на нього так само, як на мачуху та Ервіна: чваньки-новобагатьки, мовляв.

І тому Тім зрадів дужче за всіх (наскільки він взагалі ще міг радіти), коли мачуха вирішила вибратись із завулка й найняла цілий поверх у розкішному будинку на гарній вулиці.

Старі меблі, що їх. мачуха не схотіла забрати з собою, вона роздарувала тим небагатьом людям у завулку, з якими ще розмовляла. Вона хотіла віддати комусь і великий годинник, де Тім сховав свої заощадження. На щастя, Тім почув про те вчасно й попросив мачуху поставити годинник у його кімнаті в новому помешканні. Він просив так настирливо, що мачуха погодилася - скоріше здивовано, ніж сердито. І таким чином той сейф, що вибивав години, помандрував разом із Тімом до його власної кімнати, де хлопець уперше зроду дістав змогу спокійно, на самоті вчити уроки.

У новому помешканні мачуха зразу найняла собі служницю. Але довго витримати в неї не могла жодна дівчина. За Марією прийшла й пішла Берта, за Бертою Клара, за Кларою Йоганна, поки врешті найняли одну [46] літню жінку на ймення Гріт. Та лишилась надовго, бо не мовчала хазяйці, а огризалася щоразу, коли пані нападала на неї.

Отак вони то гризлись, то знов мирились рік за роком, аж поки Тімові сповнилось чотирнадцять літ і настав час віддавати його вчитись якогось фаху.

Мачуха наполягала, щоб Тім пішов учитись до контори, що приймає заклади на іподромі. Вона недарма добивалась того: якраз на свій тринадцятий день народження Тім поставив дуже багато грошей на одного старого коня, що його іподромне начальство востаннє, ніби з ласки випустило на перегони. На цього коня не ставив ніхто - крім Тіма! Та через те, що на нього поставив Тім, кінь прийшов до фінішу перший - на превеликий подив усіх знавців.

Хлопець одержав цілих тридцять тисяч марок. Але після того виграшу він заявив своїй мачусі, що вони тепер мають уже досить грошей і що він більше не гратиме на перегонах. І ні сльози, ні штурхани не могли [47] вплинути на нього. Відтоді він ні разу не пішов на іподром.

Мачуха з Ервіном іще кілька разів самі спробували щастя. Але, програвши загалом три тисячі марок, а вигравши ледве триста, вони теж покинули грати на перегонах.

Отож мачуха й сподівалась, що Тім, навчаючись у закладовій конторі, знову захопиться перегонами. Вона вже навіть ходила домовлятися з найбагатшим у їхньому місті власником такої контори. Але Тім затявся й сказав, що буде моряком і ніяких контор та іподромів більше й знати не хоче.

Одного разу, кілька днів по тому, як Тім закінчив школу, мачуха знову завела свою стару пісню про Ті-мову майбутню роботу.

- Тижтаки вженедитина, Тіме! Требаждочогосьбратися! Приконторі тизтвоїм хистом мігбибагатієм стати, Тіме! Яжнепросебедбаю! Яжпротебедумаю!

- А я не хочу в контору! Я хочу стати моряком! - відрубав Тім.

Тоді мачуха почала сердито вичитувати йому, далі розкричалась, а врешті пустила сльозу. Стала, як завжди, скиглити, що він хоче її залишити на старість без шматка хліба, щоб вона з торбами ходила, що він, певно, надумав стати багатієм сам, а її та свого брата Ервіна покинути в біді й що взагалі він завжди був у родині як чужий. Він же навіть сміятись уже не вміє!

Мачуха й гадки не мала, як боляче вразили Тіма ті останні слова. Йому аж кров ударила в голову. Він ладен був схопитись і кудись забігти.

Та відтоді, як Тім утратив свій сміх, він так навчився володіти собою, що аж страх, як на його літа. І цього разу він теж зумів так стриматися, що мачуха помітила тільки, як він почервонів.

- Дасте мені в неділю стільки грошей, як тоді, коли я закладався востаннє,- сказав він.- Я, напевне, виграю дуже багато.

Не встигла мачуха відповісти, як Тім вибіг із хати, [48] помчав щодуху до річки й сів там на самотній лавочці, силкуючись опанувати своє хвилювання. Але цього разу не спромігся. Його душили сльози; він щосили стримував плач і тому аж тіпався, хлипаючи. Лише як він кинув стримуватися, сльози й хлипання помалу минулися, і чотирнадцятирічний хлопець почав спокійно й холодно обмірковувати своє майбутнє.

Він вирішив у неділю знову поставити на найгіршого коня й виграти багато грошей. Ті гроші він віддасть мачусі, а потім просто втече з дому. Може, найметься юнгою на пароплав, а може, щось інше знайде. За гроші він не турбувався: іподроми є всюди. А крім того, бути багатієм йому зовсім не хотілося. Тім це вже зрозумів. Він продав свій сміх за те, що зовсім йому не потрібне.

І ось хлопець, що сидів на лавочці над річкою, постановив собі куди важливішу річ: він вирішив повернути свій сміх! Наздогнати, знайти його. Розшукати пана Троча, хоч би де той був.

Добре було б, якби Тім міг комусь розповісти про ту свою ухвалу - хоч кому-небудь, хоч якому п'яному візникові чи напівбожевільному дивакові-волоцюзі. Найскладніші речі можуть стати найпростішими, коли порадитися про них з ким-небудь. Але Тім не мав права розповідати про свою біду. Він повинен був замкнутися в собі, як устриця замикається в своїй скойці. Аркуш паперу, що лежав тепер у подвійному дні годинника, зробив із Тіма найсамотнішого й найсмутнішого хлопця в цілому світі.

У тій самотності й смутку несподівано пригадався Тімові небіжчик батько і гроші, відкладені на мармуровий надгробок. І Тім вирішив іще одне: перше ніж утекти з дому, треба таки поставити батькові той надгробок. Він знав, що зробити це буде нелегко. Та дарма. Якось упорається!

Вже спокійніший підвівся хлопець із лави. В нього тепер були плани, що їх треба здійснити. І ті плани додали йому сили. [49]

Восьмий аркуш

ОСТАННЯ НЕДІЛЯ

Настала неділя - остання Тімова неділя в рідному місті. Мачуха хвилювалася вже зранку. Вона зварила дуже міцної кави, жадібно пила її за сніданком і майже нічого не їла. Тімові вона дала навіть більше грошей, ніж той просив. А сама вбралася в свою найдорожчу сукню - шовкову вишивану - і вийняла з шафи лисяче хутро.

- Ой, якбижзнати читожмивиграємо, - торохтіла вона. - Ативжезнаєш наякогоконяпоставиш, Тіме?

- Ні,- відказав хлопець щиру правду.

- Невжетищененадумав?

- Тім знає сам, що йому робити! - встряв зведений брат. Тімові успіхи на іподромі сповнили Ервіна не лише заздрістю, а й повагою.

Поснідавши, всі троє поїхали в таксі на іподром. Мачуха зразу попрямувала до кас. Але Тім сказав, що йому треба ще трохи походити між людьми, послухати, що балакають про перегони та коней. Мачуха з Ер-віном мусили погодитись і відпустили його самого поблукати в натовпі.

На іподромі Тіма вже майже забули, бо ж він цілий рік не був там. Та знайшлися й люди, що впізнали хлопця й стали, перешіптуючись, показувати на нього іншим. Найдужче зацікавився Тімом один чоловік із кучерявим каштановим чубом та дивно колючими водяво-блакитними очима. Він ходив кругом хлопця, мов собака кругом свого хазяїна, й стежив за ним невідчепно, хоч і непомітно. А коли Тім підійшов до списку коней і почав його вивчати, чоловік став поруч нього й сказав нібито зовсім байдуже, навіть не дивлячись на хлопця:

- На Вихора, здається, ніхто не ставить... - Потім спитав: - А ти що, теж хочеш поставити?

- Аякже! - відповів Тім.- І якраз на Вихора.

Аж тоді незнайомець обернувся до нього: [50]

- Сміливий же ти, хлопче! Адже у Вихора майже ніяких шансів на виграш.

- Побачимо,- відказав Тім.

Йому раптом стало смішно. Але сміятися він не міг. Поважно, трохи аж сумно дивився він на незнайомця. А той почав кепкувати з Тімових сміливих намірів та з Вихора.

Тім рушив до каси, і незнайомець пішов поруч, без угаву жартуючи, дуже дотепно підсміюючись із малих жокеїв і водночас пильно стежачи за Тімовим обличчям. Але Тім і бровою не моргнув на ті жарти.

Не дійшовши трохи до каси, Тімів супутник зупинився. Мимоволі пристав і Тім.

- Мене звуть Крешимир,- сказав незнайомець.- Я тобі добра зичу, хлопче. Я знаю, ти на цьому іподромі ще ні разу не програв закладу. Не часто таке трапляється... Просто диво бере. Можна щось у тебе спитати?

Тім поглянув у водяво-блакитні очі, що нагадували йому когось,-він тільки не знав кого. Потім відказав:

- Будь ласка, питайте...

Не спускаючи Тіма з очей, пан Крешимир тихо - запитав:

- А чого ти, хлопче, ніколи не смієшся? Ти не любиш сміятися? Чи... не можеш?

Тімові аж кров шибнула в голову. Що це за чоловік? Хто він такий? І що він знає? Хлопцеві здалося враз, що в цього Крешимира Трочеві очі. Може, це сам Троч? Може, він навмисне так перемінився, щоб випробувати Тіма?

Хлопець, певне, трошки задовго вагався, бо пан Крешимир раптом сказав:

- Можеш не відповідати, я вже й так бачу. Колись, може, я зумію тобі допомогти. Пам'ятай же, мене звуть Крешимир. Бувай здоров!

І чоловік зник у тисняві. Згубивши його з очей, розтривожений Тім підійшов до каси й поставив усі гроші на Вихора. Після зустрічі з цим паном Крешимиром [51] він іще твердіше вирішив не пізніш як завтра податись геть із міста.

Біля каси до нього підбігли мачуха та Ервін. Видно, вони чекали його там. Цього разу Тім так і не сказав їм, на якого коня поставив. Але вперше за весь час він разом із ними стежив за перегонами.

Вихор був надзвичайно баский молодий жеребчик, що біг на перегонах тільки втретє. Люди всі гадали, що його випущено на перегони трохи зарано. Досі він займав тільки середні місця. Одного разу на початку перегонів Вихор був стрілою вихопився вперед, але скоро збився з ходи і, як звичайно, добіг до фінішу поміж «середнячків».

Про все те Тім дізнався з розмови двох чоловіків, що стояли поруч нього. Хлопця вперше зацікавили перегони. Він боявся, що після розмови з паном Крешимиром його угода з картатим паном Трочем стала недійсна, і на цих перегонах хотів пересвідчитися, слушний його страх чи ні.

Пролунав стартовий постріл. Вихор із самого старту біг четвертим, не наздоганяючи трьох передніх, але й не відстаючи від них. Двоє чоловіків поруч Тіма вже розмовляли про коня, що біг попереду. Але потім розмова знову зійшла на Вихора. В дедалі гучнішому галасі Тім чув лиш уривки: «... добре виїжджений...», «... ощаджує силу...», «... колись іще буде добрий кінь...»

Вихор, здавалось, цього разу не мав шансів на перемогу. Він тримався на четвертому місці, але вже потроху відставав від трьох передніх. Ервін та мачуха напосідали на Тіма, щоб сказав їм, на якого коня він поставив. Але хлопця пойняла непевність. Очі його тривожно стежили за перегонами. Вихор трохи піддав ходи, і відстань між ним та передніми трьома помалу-помалу скорочувалась. Але й до фінішу було вже зовсім близько.

І зненацька передній кінь спіткнувся. Двоє, що бігли зразу за ним, схарапудились і сахнулись убік. А Вихор [52] в ту мить стрілою вихопивсь уперед і домчав до фінішу перший.

Натовп загукав скоріше розчаровано, ніж радісно. Тім почув біля себе:

- Таких дурних перегонів я ще зроду не бачив!

На великому табло нагорі з'явився напис «Вихор». Тімові аж на серці полегшало. Йому так хотілося засміятись! Але натомість він мовчки вийняв із кишені купон, віддав мачусі й сказав:

- Підіть одержіть гроші, будь ласка. Ми виграли!

Пані Талер з Ервіном бігом кинулись до каси. Тім не став дожидати їх, а мерщій поїхав трамваєм додому, вийняв із годинника свій скарб, заховав угоду під підшивку кашкета, а гроші в кишеню пальта, перекинув пальто через руку й уже хотів був вийти з хати, коли враз почув, що вертаються мачуха з Ервіном. Він швиденько сховався за завіскою в комірчині. Мачуха покликала: «Тіме!» Але Тім не озвався.

- Іделишеньтойхлопець? - почув він. - Якийвінчуднийстав останнімчасом!

Голоси помалу затихли в дальших кімнатах. Тім іще почув, як Ервін спитав: «То ми тепер дуже багаті?» - й мачушин різкий голос відповів щось наче: «... рок п'ять тисяч!»

«Ну й добре,- спокійно й холодно подумав Тім.- Тепер я їм більше не потрібен».

Він виступив із комірчини, відімкнув тихенько двері, так само тихо замкнув їх, прокрався попід самими вікнами до парку, а тоді чимдуж помчав на кладовище, що було в східній частині міста.

Лиш як гладкий вусатий сторож при цвинтарній брамі спитав його, який номер батькової могилки, хлопець зрозумів, що звертатися йому слід було не сюди. І все ж він вирішив спробувати:

- А можна у вас замовити мармуровий надгробок?

- Мармур у нас не дозволяється. Тільки пісковик,- буркнув вусань.- Та й не в мене слід про це питати. А каменяр у неділю не працює. [53]

І раптом Тімові блиснула зухвала думка:

- А давайте заб'ємося об заклад, що в мого тата буде мармуровий надгробок. Ще й із золотим написом: «Від твого сина Тіма, що ніколи тебе не забуде».

- Програв ти свій заклад наперед, хлопче!

- А однаково, давайте заб'ємося! На плитку шоколаду. (Тім побачив ту плитку шоколаду на підвіконні в сторожці.)

- А ти маєш за що купити плитку шоколаду, як програєш?

Тім вийняв із кишені пальта всі свої гроші й показав сторожеві.

- Ну то як, забились?

- Дурнуватий якийсь заклад, я ще зроду так по-дурному не забивався,- промимрив той.- Ну, та дарма! - вони потисли один одному руки на знак того, що забились, і рушили величезним, схожим на парк кладовищем до могилки Тімового батька.

Ще здалеку обидва побачили не могилці трьох робітників у комбінезонах. Гладкий сторож піддав ходи.

- Що ж це таке!..- він засопів, наче морж, і пустився бігцем.

На могилку якраз поставили новенький надгробок. Мармуровий! На ньому золотими літерами було викарбувано батькове ім'я й дати народження та смерті. А нижче: «Від твого сина Тіма, що ніколи тебе не забуде».

Сторожів крик нітрохи не збентежив робітників. Вони показали йому папери, які свідчили, що надгробок поставлено цілком законно. Був там навіть окремий дозвіл на мармур. Вони, мовляв, просто не схотіли будити сторожа, як приїхали, бо той саме задрімав.

- А втім, - додав один із робітників, - за надгробок має ще заплатити якийсь Тім Талер.

- Правда, - озвався Тім. - Ось вам гроші. - Він знову видобув їх із пальта й відрахував тому робітникові скільки треба. [54]

Тімові лишилося всього півмарки. Сторож буркнув щось і почвалав до сторожки. Робітники зібрали своє начиння, козирнули Тімові на прощання й теж пішли.

А Тім із п'ятдесятьма пфенігами в жмені та зі своєю чудернацькою угодою в кашкеті лишивсь на батьковій могилці й почав розповідати мертвому все те, що радий був би розповісти комусь живому.

Висловивши все, він іще раз подививсь на надгробок, що дуже йому подобався, тоді сказав:

- Я прийду, як могтиму знову сміятися. Скоро вернуся! - і, збентежившись раптом, додав: - Сподіваюся, що скоро...

У сторожці Тім забрав у насупленого вусаня виграний шоколад. А останні свої гроші заплатив за квиток у трамваї. Куди він поїде, хлопець сам ще не знав. Він тільки знав, що хоче розшукати пана Троча й повернути собі свій проданий сміх. (Тім же й гадки не мав що за рогом цвинтарного муру троє робітників іще раз одержали гроші від одного добродія в карта тому костюмі, бо вони, люди побожні, могли взяти на себе такий гріх, як робота в неділю, тільки за подвійну платню.)

Дев'ятий аркуш

ПАН РІКЕРТ

Трамвай був майже порожній. Опріч Тіма, в ньому їхав іще якийсь літній панок, повненький, із веселим кирпатим обличчям. Він зразу спитав хлопця, куди той їде.

- На вокзал,- відповів Тім.

- Тоді доведеться тобі пересісти! Цей трамвай на вокзал не їде. Я добре знаю, бо мені самому треба туди.

Тім сидів, держачи кашкета на колінах. Під пальцями в нього зашурхотів папір угоди, і йому раптом сяйнула блискуча думка: вигадувати якнайбезглуздіші, [55] якнайнеймовірніші заклади. Може, він програє який-небудь і тоді дістане назад свій сміх!

Тому хлопець відповів:

- А давайте заб'ємося об заклад, пане, що цей трамвай поїде на вокзал.

Панок засміявсь і сказав точнісінько те саме, що сторож на кладовищі:

- Програв ти свій заклад наперед, хлопче! - А потім додав: - Бо ми їдемо в дев'ятому номері, а цей номер ніколи ще не ходив до вокзалу.

- Однаково, забиймося! - промовив Тім так упевнено, що панок аж сторопів.

- О, ти, здається, певен свого виграшу, хлопче! А на що ти хочеш закластися?

- На квиток до Гамбурга,- швиденько відповів Тім - і сам зчудувався: чого це раптом йому набігло таке на думку! (А втім, нічого дивного в тому не було: адже він давно вже збирався стати моряком.)

- То ти хочеш їхати до Гамбурга?

Тім кивнув головою.

Ласкаве кирпате обличчя зморщилося в усмішці.

- Тоді не треба й закладатися! Бо я сам їду до Гамбурга й маю квиток на ціле купе. Той пан, що збирався їхати зі мною, затримується тут. От ти й будеш мені за товариство.

- І все ж давайте заб'ємося,- поважно сказав Тім.

- Ну, гаразд. Забилися. Але попереджую: ти програєш! Як тебе звати?

- Тім Талер.

- Гарне прізвище. Грішми пахне*. А мене звуть Рікерт.

Вони потисли один одному руки, воднораз познайомившись і забившись об заклад.

Коли до вагона ввійшов кондуктор перевіряти квитки, пан Рікерт спитав його:

- Ми їдемо на вокзал?

[* Талер - давня німецька монета.] [56]

Тільки-но кондуктор розкрив рота відповісти, як трамвай різко зупинився: Тім мало не впав на пана Рікерта. Кондуктор побіг до передніх дверей. Там якраз заходив до вагона ревізор у мундирі з товстим срібним аксельбантом. Він щось гукнув кондукторові, а той йому; потім кондуктор повернувся до вагона й відповів на Рікертове запитання:

- Зараз ми, пане, справді поїдемо на вокзал, бо на нашій лінії порвався провід. Але звичайно дев'ятка на вокзал не ходить.

Торкнувши пальцем кашкета, він знову пішов наперед.

- Грім побий, швиденько ж ти виграєш заклади, Тіме! - засміявся пан Рікерт.- Ти, напевне, знав, що той провід порветься, еге?

Тім сумно похитав головою. Він радніший був би програти цей заклад. Та все ж йому хоч ясно стало, що пан Троч має такі здібності, які можна назвати принаймні незвичайними.

На вокзалі пан Рікерт спитав, де ж Тімові речі.

_ Все, що треба, при мені,- відповів Тім якось непевно, трошечки по-дитячому.-А паспорт у кишені.

У хлопця справді був паспорт. Як йому минуло чотирнадцять років, він добивсь від мачухи, щоб вона взяла йому паспорт. Мовляв, на іподромі можуть часом спитати документи. Одного цього натяку вистачило: адже якраз тоді Тім був відмовився вигравати гроші.

А тепер от виявилося, що паспорт йому справді потрібен, бо ж він їхав до чужого міста, до Гамбурга.

Пан Рікерт мав у поїзді ціле купе першого класу. На дверях була пришпилена картка з написом: «Крістіан Рікерт, директор пароплавства». Але нижче стояло ще одне прізвище, і, прочитавши його, Тім сполотнів. Бо там було написано: «Барон Томас Троч». Коли вони посідали, пан Рікерт запитав?

- Тобі недобре, Тіме? Чого ти раптом так поблід? [57]

- Та ні, нічого, зі мною часом таке буває,- відказав Тім, і то, власне, була правда, бо з ким на світі такого не буває?

Поїзд довго їхав понад берегом Ельби. Пан Рікерт, видимо, зацікавлено розглядав річку й береги, а Тім нічого того не бачив.

Час від часу ласкаві Рікертові очі крадькома поглядали на Тіма, тоді знову повертались до краєвиду в вікні. Пана Рікерта чимсь непокоїв цей поважний, аж сумний хлопець, і він вирішив розважити його кумедними історіями з моряцького життя. Але скоро він помітив, що Тім задумався про щось своє і не слухає його.

Аж як пан Рікерт заговорив про барона Троча, на чиєму місці їхав тепер Тім, хлопець видимо зацікавився й навіть сам розговорився.

- Барон, мабуть, дуже багатий? - спитав він.

- Без міри багатий! У нього є підприємства по всьому світі. Те гамбурзьке пароплавство, де я директором, також належить йому.

- То барон живе в Гамбурзі?

Пан Рікерт якось непевно розвів руками, немов хотів сказати: «Що я знаю?» Тоді пояснив:

- Барон живе скрізь і ніде. Сьогодні він у Гамбурзі, завтра в Ріо-де-Жанейро, а післязавтра, може, в Гонконгу абощо. А головна його резиденція, скільки мені відомо,- один замок десь у Месопотамії.

- Ви, певне, дуже добре знаєте його?

- Його, Тіме, ніхто добре не знає. Він переміняється, як хамелеон. Приміром, він багато років мав тонкі стиснуті губи й колючі очі - я ладен заприсягтися, що вони були водяво-блакитні. І ось я побачив його вчора,- аж він має лагідні карі очі. І темних окулярів на вулиці вже не носить. Але найдивовижніше ось що: я ніколи не чув, щоб він сміявся, а от учора він реготав, як малий хлопчак. І ні разу навіть губів не стис, як бувало звичайно.

Тім квапливо відвернувся до вікна, бо сам в ту мить мимоволі стис губи. [58]

Пан Рікерт відчув, що його оповідання чимсь і дуже цікавить, і водночас бентежить хлопця. Тому він змінив тему:

- А чого ти, власне, їдеш до Гамбурга?

- Хочу найнятись на який-небудь пароплав учнем офіціанта! - і знову Тім сам здивувався зі своєї несподіваної ухвали, хоч дивного в ній нічого й не було: адже треба з чогось починати, коли хочеш стати моряком.

Пан Рікерт аж засяяв, гордий від того, що може допомогти хлопцеві.

- То ти щасливець, Тіме! - сказав він майже урочисто.- Коли тобі треба на вокзал, трамвай наче тільки задля тебе повертає туди, а коли тобі потрібна робота, мов з неба падає людина, що може її тобі дати!

- А ви можете влаштувати мене учнем офіціанта?

- Офіціант на пароплаві називається стюард,- поправив його директор пароплавства.- А тобі доведеться попервах бути юнгою чи боєм. Я тільки хотів би знати одне: чи твої тато й мама згодні?

Тім задумався лиш на мить, а тоді відповів:

- У мене немає тата й мами.

Про мачуху він не сказав, бо знав, що вона ніколи не дала б йому дозволу найнятись на пароплав. І взагалі він не дуже задумувався над тим, що залишає позаду.

Куди дужче непокоїло його інше: чи ця зустріч із паном Рікертом справді щасливий випадок, а чи, може, й тут доклав рук картатий пан, як у тій історії з мармуровим надгробком та з трамваєм?

Разом зі своїм сміхом Тім утратив іще одне: простосердість і довіру до людей. І це було дуже погано.

Пан Рікерт спитав його ще щось, але хлопець закліпав очима, не зрозумівши: так вирували думки у нього в голові.

- Я питаю, чи не треба тобі допомогти взагалі? - ще раз спитав пан Рікерт.- Чи, може, я тобі не подобаюся? [60]

- Ну що ви, що ви! - швидко відповів Тім.- Навіть дуже!

Він сказав це цілком щиро. В ньому враз зародила-ся впевненість, що цей чоловік хоч і служить у картатого пана (чи то пак у багатого барона Троча), але не може бути Трочевим поплічником. Тім знову став простосердною дитиною, звичайнісіньким чотирнадцятирічним хлопчаком.

А пан Рікерт спитав уже навпростець:

- Скажи, що з тобою таке? Адже ти сьогодні ще ні разу не засміявся, хоч посміятись таки було з чого. Може, в тебе якась біда?

Хлопець радий був би кинутись на шию панові Рікертові, як роблять часом люди у виставах. Але в Тіма то була не гра, не комедія, а справді нестерпуче поривання до людини, якій можна все розповісти.

І так важко було хлопцеві стримати те поривання, що від розпачу й безпорадності йому з очей покотилися, мов горох, сльози.

Пан Рікерт сів поруч нього й сказав якомога спокійніше, аж холодно:

- Ну, не плач! Розкажи, що з тобою таке.

- Я не можу! - вигукнув Тім, тоді просто тицьнувся обличчям у плече панові Рікертові, й сльози полилися ще дужче.

Він так хлипав, що аж тіпався весь.

Невеличкий повненький директор пароплавства взяв хлопця за руку й тримав, поки Тім, утомившись плачем, заснув.

Десятий аркуш

ЛЯЛЬКОВИЙ ТЕАТР

Вантажно-пасажирське судно, що на ньому Тім мав служити як стюардів помічник, називалося «Дельфін» і ходило на лінії Гамбург-Генуя.

До відплиття пароплава лишалося ще три дні, й пан [61] Рікерт узяв хлопця на той час до себе додому, до гарненької вілли, що стояла на розкішній набережній над Ельбою.

Біла, мов хмаринка на літньому небі, вілла мала на чільній стіні, над дверима, півкруглий балкон, підпертий трьома колонами. Ґанок з обох боків стерегли два лагідні леви, витесані з каменю.

Коли Тім побачив той світлий, веселий будиночок, йому аж серце стислося. Колись, як він був іще веселим хлопчаком, той будиночок, певне, здався б йому чарівним сном, житлом щасливого принца з якоїсь казки. Та хто продав свій сміх, той хіба може бути щасливий? Поважний і смутний пройшов Тім між двома лагідними левами до дверей білої вілли.

Пан Рікерт жив удвох із своєю матір'ю, добросердою старенькою жінкою з сивими кучериками й таким голоском, наче в маленької дівчинки. Вона сміялася з усього, як дитина.

- Чого ти весь такий сумний, хлопче? - спитала вона Тіма.- Недобре так у твоїх літах! Іще встигнеш скуштувати серйозного в житті. Правда, ж Крішане?

Її син, директор пароплавства, кивнув головою, відвів матір убік і пояснив їй, що з хлопцем, видно, сталося щось жахливе, отож нехай вона буде з ним обережніша.

Старенькій нелегко було зрозуміти, що має на увазі її син. Вона виросла в багатому, веселому домі, вийшла заміж за багатого й веселого чоловіка й тепер доживала віку в достатку й веселощах. Вона знала вбогі завулки лише із зворушливих оповідань, що, читавши їх, завжди плакала, а гризні, заздрощів, лукавства старенька просто не бачила в житті, бо не хотіла бачити такої погані. Вона ціле своє життя лишалась дитиною, небесно-блакитною квіткою, що не могла відцвісти.

- А знаєш що, Крішане,- сказала старенька, вислухавши сина.- Я поведу хлопця погуляти, покажу [62] йому місто. Ось побачиш, я його розважу, він у мене таки засміється!

- Тільки обережно з ним, мамо,- попросив пан Рікерт. І старенька пообіцяла, що буде обережна.

Прогулянки з нею стали для Тіма справжньою мукою, бо він дуже полюбив це вісімдесятирічне дитятко й радий був би усміхнутись до неї, підморгнути їй, коли вона брала його за руку своєю маленькою м'якенькою ручкою. Йому хотілось навіть подражнити її, немов старшу сестричку. Але сміх його був далеко. З тим сміхом ходив десь по світу один химерний багатий барон. Тім зрозумів тепер: він продав найкраще з того, що мав.

У вівторок старенькій пані Рікерт набігла цікава думка. Вона прочитала в газеті, що один ляльковий те-атр виставляє казку «Липкохвіст». То була казка про королівну, що не вміла сміятися. Пані Рікерт добре пам'ятала всю ту казку. І вона вирішила сходити на виставу й повести на неї хлопця, що не вмів сміятися.

Та думка дуже сподобалась самій пані Рікерт, але вона зразу нікому нічого не сказала. Цілий ранок тихенько собі посміювалась і аж по обіді запросила до театру свого сина й Тіма. Обидва вони не могли відмовити старенькій і пішли з нею.

Ляльковий театр був від них не дуже далеко - в невеличкому гамбурзькому передмісті Евельгене, що лежить попід кручами на березі Ельби й складається, власне, з одного рядка маленьких чепурненьких будиночків у садочках. Ото там, у задній кімнаті одного ресторанчика, й містився ляльковий театр.

Невеличка зала була повна дітей. Поміж них сиділо лиш декілька батьків і матерів.

Пані Рікерт зразу нагледіла троє вільних місць у другому ряду і, сміючись та вимахуючи руками, пропхалася туди. її син із Тімом пішли за нею. Тільки-но вони повсідались, як у залі погасили світло й невеличка червона завіса перед сценою піднялася вгору. [63]

Вистава почалася з віршованої розмови між королем і волоцюгою. Вони стрілися глупої ночі, в чистім полі, при місяці. Королеве обличчя було бліде й поважне. А волоцюга мав свіжі, червоні навіть при місячному світлі щоки й усміхнені уста. Ось яка була їхня розмова, що з неї почалась вистава:

Король

Дійшли мені, добродію, до слуху

Чутки про королівну Несміюху.

Я сам статечний, не люблю сміятись,

Отож би добре з нею нам побратись.

Та хтозна, де живе вона, одначе.

Не скажете? Я щедро вам віддячу.

Волоцюга

Чом ні? Я вас охоче проведу,

Бо я і сам оце туди іду.

Та вам на шлюб не раджу сподіватись,

Бо я навчу ту панночку сміятись.

Король

Не навчите, земляче, далебі,

Бо хто хоч раз помислить сам собі,

Що все живе на світі має вмерти,

Тому гіркої думки не відперти,

Що світ - то лиш блискуча банька з мила:

Гра, поки лусне; а тоді - могила.

По думці ж тій годиться не сміятись,

А гідному й поважному лишатись.

Волоцюга

Ви, пане, бачу, мудрий чоловік,

Та трошечки не в той загнули бік.

Пошився в дурні, хієш смерті жде:

Життя тим часом мимо нього йде.

Бо ж келих роблять не щоби розбити,

А щоби мед-вино із нього пити. [64]

Він знає, що розіб'ється колись,

Та поки цілий - дзень собі та блись!

Король

Як може він радіти, що блищить,

Коли він знає: час його строщить?

Волоцюга

Отож він так і квапиться радіти,

Бо знає, що не вік йому ряхтіти!

Король

Вам річ моя, мов по стіні горох.

Ходім лишень до королівни вдвох.

Як ви насмішите її, то вам

Я королівство все своє віддам!

Волоцюга

Ну що ж, забились! Але ж хто не знає,

Що сміх людей від звірів відрізняє!

Адже з того людина й пізнається,

Що в слушну мить вона сміється.

Завісу опустили, і в залі стало майже темно. Тільки крізь нещільно позавішувані вікна просочувалося трохи світла. Діти, що з них більшість не зрозуміли коротенького прологу, перешіптувались та чекали нетерпляче, коли ж нарешті Почнеться справжня вистава.

Спереду, в другому ряду, наші троє глядачів тихо сиділи на своїх місцях і думали про зовсім різні речі, кожне про своє. Старенька пані Рікерт сердилася, що мусить погодитися саме з волоцюгою. Бо вона не дуже шанувала волоцюг (хоч роздавала жебракам гроші без ліку). Вона б охочіше погодилася з королем, таким поважним і вродливим!

Пан Рікерт, що сидів праворуч неї, силкувався в півтемряві розглядіти Тімове обличчя. Та лиш один тоненький промінчик падав на хлопцеве чоло, бліде, як [65] у короля з вистави. Пан Рікерт побоювався, що вигадка його матусі - повести хлопця до лялькового театру - не дуже вдала. Бо напередодні він бачив, як Тім плакав.

А в Тімовій голові билася тільки одна думка: «Хоч би ніхто зараз не заговорив до мене!» Йому зовсім здушило горло, аж віддих перехопило. І весь час крутились у мозку останні рядки з прологу: «... Сміх людей від звірів відрізняє! Адже з того людина й пізнається, що в слушну мить вона сміється... сміється... сміється...»

Ось завіса знов піднялась, і Тімів погляд, а тоді й усю його увагу прикувала до себе бліда-бліда, поваж-на-поважна королівна, що виглядала з віконця у високій замковій вежі.

В садочок під віконцем вийшов король-батько. Побачивши його, дочка сахнулась назад і потихеньку зачинила вікно.

Його величність король сів скраєчку на цямриння фонтана й почав виливати квітам та воді свій жаль: зібрав він до замку всіх жартунів та витівників, улаштовує всілякі можливі розваги, щоб насмішити свою дочку, та все марнісінько.

Зітхаючи, король підвівся. Діти в залі принишкли, як мишенятка.

Його величність почав ходити по садочку туди й сюди, нарікаючи на свою та на доччину долю. Та зненацька спинився й вигукнув:

- Якби хто її насмішив, я б йому зразу віддав її за дружину й півкоролівства дав би в посаг!

В ту мить якраз увійшли до садочка волоцюга з чужим смутним королем.

Волоцюга зачув розпачливу мову короля-батька й гукнув, не довго думаючи:

- Дотримайте ж слова, ваша величність! Коли я насмішу королівну, я візьму її собі за дружину. А півкоролівства можете лишити собі, бо оцей ось добродій, що йде зі мною, пообіцяв мені своє ціле. [66]

Король зчудовано витріщився на обох чужинців, що мимоволі підслухали його. Блідий чужий король сподобався йому дужче, ніж здоровий червонощокий волоцюга. (В королів у таких речах свій власний смак.) Та все ж він не став зрікатися слова й сказав:

- Коли тобі, чужинцю, пощастить насмішити королівну, ти будеш принцом і її чоловіком!

Така обіцянка задовольнила волоцюгу. Він повернувсь і кудись прожогом побіг, лишивши двох королів самих.

Завіса знов опустилась, і настала коротенька перерва. Малим глядачам було страшенно цікаво: чи ж то засміється королівна?

А Тім Талер потай надіявся, що вона не засміється. Королівна вже здавалась йому немовби рідною сестрою; вдвох із нею вони не сміятимуться наперекір усьому світові. Але Тім добре знав, як звичайно кінчаються казки. Тому він пригнічено чекав тої миті, коли королівна засміється.

І на жаль, довго чекати йому не довелося. Коли завіса піднялась, королівна вже знов виглядала з віконця, а обидва королі сиділи на цямрині фонтана. З-за сцени почувся спів і сміх, і зненацька до садочка вбіг волоцюга. Він вів на золотому ланцюжку білого лебедя.

Якийсь гладкий чоловік тримався правою рукою за лебедячий хвіст, неначе був до нього приклеєний. А лівою рукою гладун тяг за собою миршавенького худого чоловічка, а той за собою - стареньку бабусю, а бабуся - хлопчика, а хлопчик - дівчинку, а дівчинка - собаку. Всіх їх немовби скувала в один ланцюг якась чарівна сила. Вони підскакували, пританцьовували, ніби на невидимих пружинах. І реготали, аж лящало в садочку.

Королівна вихилилась далеко з віконця, щоб краще все побачити. Вона здивовано витріщила очі, однак навіть не всміхнулась. [67]

«Не смійся, сестричко!» - подумки благав її Тім.- Нехай хоч весь світ регоче, а ми не сміймося!»

Та надарма благав. Сумний чужий король не встерігся й доторкнувся до собачки на кінці тієї чудернацької низ очки. І зразу ж ніби прилип до собачого хвостика. А з переляку вхопився вільною рукою за правицю старого короля-батька. Тепер уже обидва королі міцно приклеїлись до дивовижної процесії. Вони смикалися, тіпались, і видно було, що обидва раді б звільнитися від тих незбагненних чарів. Але ніяк не могли. Довелось обом змиритися зі своїм химерним становищем, і здавалось навіть, наче воно вже починає тішити їх.

Ноги їхні мимохіть подригувалися до танцю, кутики вуст сіпались; і раптом обидва королі незграбно, кумедно застрибали, а тоді пирснули реготом.

В ту саму мить згори, з віконця, пролунав і королівнин сміх. Заграла музика. Всі пустилися в танок, усі стрибали, сміялись, і діти в залі теж сміялись і тупотіли ногами з радощів.

Сердешний Тім сидів, наче камінь, посеред того моря сміху. Старенька пані Рікерт поруч нього так сміялась, що аж затулила обличчя руками й перехилилась наперед, бо від сміху з очей котилися сльози.

В ту хвилину Тім уперше помітив, як схоже виявляються в людини сміх і плач. І він зробив страшну річ: закрив обличчя долонями, нахиливсь уперед і вдав, ніби сміється.

А насправді Тім плакав. Крізь сльози він мурмотів: «Сестричко-королівно, ну навіщо ти засміялася? Навіщо?!»

Коли завіса впала й у залі засвітилося світло, пані Рікерт дістала з кишені мереживну хусточку, витерла собі очі, а тоді простягла хусточку Тімові:

- На, Тіме, витрись. Бач, і в тебе зо сміху аж сльози потекли. Я ж знала, що з такої комедії й ти засмієшся! - І старенька переможно глянула на свого сина. [68]

- Авжеж, мамо,- чемно відказав той.- Це ти справді непогано придумала.- Але ті слова пан Рікерт вимовив невесело. Він бачив, що хлопець із доброго серця й розпачу обдурив стареньку. І Тім знав, що пан Рікерт це бачить.

Уперше від того нещасливого дня на іподромі в хлопцевій душі знялася безсила лють на барона Троча. Та лють просто полонила його всього, і він іще твердіше вирішив за будь-яку ціну відвоювати свій сміх.

КНИГА ДРУГА

ПРИГОДИ

Навчи мене сміятись, порятуй

мою душу!

Англійське прислів'я

Одинадцятий аркуш

СТРАШНИЙ БАРОН

На Тімову полегкість другого дня «Дельфін» відплив до Генуї. Старенька пані Рікерт на прощання довго махала Тімові рукою з ґанку білої вілли, й Тім махав у відповідь, поки бачив її.

Директор пароплавства сам відвів хлопця на судно. На дорогу він купив йому костюмчик, черевики, ручний годинник і новісіньку матроську торбу.

Прощаючись із Тімом на пірсі, він сказав:

- Хвоста вгору, хлопче! Як повернешся за три тижні сюди, то дивитимешся на світ зовсім іншими очима. Тоді ти, напевне, вже сміятимешся. Тож хвоста вгору, хлопче! Домовились? [70]

Тім одну мить повагався, тоді швидко відповів:

- Домовилися! Коли я повернуся до вас, пане Рікерт, я вже сміятимусь.

Він іще раз подякував директорові за все, затинаючись, бо йому раптом здушило горло, і мерщій побіг трапом на пароплав.

Капітан «Дельфіна» був похмурий чоловік, що любив перехилити чарчину й геть на все дивився крізь пальці.

Він ледве глянув на Тіма, коли той увійшов до нього, і буркнув:

- Іди до стюарда. Він у нас теж новенький. Спатимете вдвох у одній каюті.

Тім, що вперше в житті потрапив на пароплав, довго блукав безпорадно по залізних трапах, по вузьких переходах, по бакові та по юті, шукаючи стюарда. Команда на пароплаві не була вбрана в матроську форму й відрізнялась від пасажирів тільки своєю робочою одежею. Тому хлопець не знав, до кого звернутися. Поблукавши довгенько, він урешті зайшов у якісь відчинені двері на шкафуті й побачив перед собою щось подібне до чекальні чи вітальні, а посеред неї вистелені килимом сходи з гарно вигнутим лакованим поруччям, що вели вниз, у нутро пароплава. Звідти здіймався смачний дух смаженої риби, і Тім здогадався, що там, унизу, йому, напевне, й доведеться працювати.

Хлопець спустився вниз. Там, праворуч, зразу біля сходів, був камбуз, звідти смачно пахло вареним і смаженим. " А просто проти сходів - розчинені двійчаті двері до великої зали-їдальні. Столи й стільці в ній були попригвинчувані до підлоги.

Якийсь чоловік у білій куртці саме накривав столи. Вся його постать, а надто кругла голова з кучерявим каштановим чубом, видалась Тімові знайомою, хоч хлопець іще не збагнув, хто ж це такий.

Коли Тім увійшов до їдальні, чоловік у білій куртці обернувся й сказав зовсім не здивовано: [71]

- А ось і ти прийшов!

Але Тім не здивувався. Цього чоловіка він знав. Навіть ім'я його пам'ятав. Це був Крешимир, той самий чоловік, що недавно на іподромі так налякав Тіма своїм запитанням, але потім сказав: «Колись, може, я зумію тобі допомогти».Той самий, що його водяво-блакитні очі так нагадували Троча, барона, якого шукав Тім.

Пан Крешимир не дав Тімові довго думати. Він відвів хлопця до їхньої спільної каюти, закинув Тімову матроську торбу на верхню койку, а тоді дав йому картаті штани та білу куртку - такі самі, як мав на собі.

Нова одіж прийшлася на Тіма якраз і дуже личила йому.

- Та ти ж наче вродився, щоб стати стюардом! - засміявся Крешимир. Але, глянувши на поважне Тімове обличчя, зразу змовк. Замислено подивившись на хлопця, він промурмотів скоріше сам до себе, ніж до нього: - Цікавий я знати, яка в тебе з ним справа була!

Але потім, немовби .відганяючи неприємну думку, випростався, поправив на собі білу куртку й мовив владно:

- Ну, до роботи! Йди в камбуз до Енріко, допоможи йому чистити картоплю. Як мені треба буде, я тебе покличу. Ну, гайда!

І Тім до самого вечора чистив у камбузі картоплю. Кок Енріко був кумедний стариган-генуезець. Він, як і капітан, ні про що не дбав. У тісному корабельному світі капітан - не тільки володар, а й взірець для всіх. Коли капітан суворий і дбайливий, така в нього буває й команда. Коли ж він недбайло, як на «Дельфіні», то й уся команда в нього недбала, аж до кока Енріко.

Той Енріко майже без передиху розповідав хлопцеві всілякі анекдоти на химерній суміші з німецької та італійської мов. Бачачи, що Тім не сміється, він думав, що хлопець його не розуміє, проте розповідав далі й далі, для власної розваги, і навіть не бачив, що Тім чистить картоплю затовсто. [72]

Коли пароплав аж надвечір вийшов нарешті з гамбурзького порту, пан Крешимир покликав Тіма в їдальню собі на поміч. Хлопець весь час дуже бентежився, бо водяво-блакитні стюардові очі раз по раз допитливо позирали на нього. Від того збентеження Тім переплутав кілька замовлень: одній американці замість віскі приніс лимонного соку, а шотландському лордові, що замовляв яєчню з шинкою, подав дві порції горіхового торта.

Пан Крешимир виправив його помилку, не сказавши лихого слова. Він весь час ніби мимохідь навчав Тіма:

- Страву подавай з лівого боку! Ліву руку, подаючи, закладай за спину. Виделку клади ліворуч, а ніж праворуч, вістрям до тарілки.

Після вечері пан Крешимир знову послав Тіма до камбуза - помагати кокові мити посуд. Працював хлопець неуважно й недбало, бо в голові у нього роїлося з сотню всіляких «чому»: чому барон не поїхав у тому купе, що в ньому пан Рікерт привіз Тіма до Гамбурга? Чому пан Крешимир раптом опинився стюардом на тому самому пароплаві, де влаштували Тіма? І чому пан Рікерт улаштував його саме на цей пароплав? Чому? Чому? Чому?

Враз у Тімових вухах пролунало справжнє кимсь вимовлене «чому». Чийсь різкий чоловічий голос спитав:

- Чому це ви опинилися на цьому судні?

А Крешимирів голос відповів:

- А чом би й ні?

- Вийдімо на палубу! - звелів перший голос.

По вузенькому залізному трапі, що вів на ют, загрюкотіли ноги. Потім тупіт і голоси затихли. Але в Тімовій голові вони ще лунали. Йому здалось, ніби він уже колись чув той перший голос, що звертався до Крешимира. І раптом - Тім якраз витирав рушником велику супницю - раптом Тім збагнув, чий то голос. [73]

То був голос людини, що їй Тім продав свій сміх,- голос барона Троча.

Супниця вислизнула з Тімових рук і розлетілася долі на дрібні скалки.

Кок Енріко відскочив убік, злякано вигукнувши: «Мамма міа!» («Матінко моя!»), а Тім прожогом кинувся на ют, слідом за голосами.

Нагорі не видно було нікого. Два корабельні ліхтарі тьмяно освітлювали палубні надбудови та затягнену брезентом шлюпку. Аж ураз Тім почув тихі голоси і, озирнувшись ліворуч, звідки вони долітали, угледів невиразно, як під шлюпкою щось ворухнулося. Навшпиньки підкрався хлопець ближче й побачив під шлюпкою дві пари чоловічих ніг. Розглядіти щось більше він не зміг, але й так ясно було, що голоси ті долітали з-за шлюпки. Тім потихеньку, тамуючи віддих, підійшов іще ближче. Один раз у нього під ногою рипнула палубна мостина, але ті двоє за шлюпкою, видно, нічого не помітили.

Нарешті хлопець підкрався так близько, що зміг підслухувати приглушену розмову.

- ...Смішно слухати! - просичав баронів голос.- Невже ви хочете, щоб я вам повірив, ніби ви вже витратили всі гроші, які мали з акцій?

- Незабаром після того, як ви дали мені ті акції, їхній курс упав зовсім низько,- спокійно відказав Крешимир.

- Нехай так! - барон засміявся купленим сміхом.- Курс акцій упав тому, що я маю деякий вплив на біржі. Але ж чверть мільйона вам однаково ще лишилося!

- А ті чверть мільйона я поклав до одного банку, що невдовзі збанкрутував, бароне.

- Таке ваше щастя,- знову засміявся барон, і Тім аж здригнувся від того сміху. Його поривало кинутись туди, але він зметикував, що розумніше буде зачекати й послухати, що ж далі.

- Ну що ж, навіть коли вам доводиться знову працювати, [74] це ще не підстава працювати саме на цьому судні й саме з цим хлопцем.

Цього разу засміявся Крешимир.

- А хто мені заборонить? - вигукнув він.

- Тихше! - просичав барон.

Півголосом Крешимир мовив далі:

- Я продав вам свої очі, взявши за них ваші риб'ячі баньки. Заплатили ви мені акціями на мільйон марок, але з того мільйона ні одна марка не попала до моєї кишені. Ви мене перехитрували. Але цього разу я буду хитріший за вас, бароне. Я двічі бачив вас на іподромі з цим хлопцем. І спостеріг, що цей хлопець потім став щоразу вигравати. А ще я спостеріг, що хлоп'я стало сумне та понуре, мов старий, недужий самотній пенсіонер.

Тімове серце закалатало, коли він почув ті слова. Але він стояв тихо, не ворухнувся.

А Крешимир провадив далі:

- Я дізнаюся, що ви виманили в цього хлопця! Я за ним стежу вже чотири роки, і мені нелегко було влаштуватися стюардом на цьому пароплаві. Але тепер...

Баранів голос перебив Крешимира:

- Тепер я вам, коли хочете, дам іще мільйон! Цього разу готівкою, з. рук у руки!

- Ні, бароне, цього разу сила моя! - Крешимир говорив дуже помірковано.- Я можу використати те, що знаю, в три способи: або забрати від вас свої очі, або взяти мільйон, або - і це, мабуть, не найгірше - примусити вас розірвати вашу угоду з цим хлопцем, хоч я й не знаю, яка та угода.

Тім притис у темряві кулак до рота, щоб не виказати себе стогоном. Якусь хвилину було тихо. Потім знову почувся баронів голос:

- Мої справи з хлопцем вас не обходять. Та коли вам такі потрібні ваші очі, я згоден на певних умовах...

Крешимир перебив його, аж задихаючись:

- Авжеж, бароне, мені потрібні мої очі! Мої давні лагідні, дурні коров'ячі очі! Вони мені миліші за [75] всі скарби світу, хоч ви того, може, ніколи не зрозумієте!

- Так, я ніколи цього не зрозумію,- підтвердив баронів голос.- І все ж я на певних умовах згоден розірвати нашу угоду. Гляньте, будь-ласка, на себе в оце дзеркальце.

Настала тиша, така що чутно було, як б'ється серце. Тім аж упрів від хвилювання, в яке вкинула його та розмова, й від зусиль триматися тихо.

Нарешті він почув Крешимирів шепіт:

- Мої очі! В мене! Знов!

- А тепер мої умови,- сказав барон.

Але Тім тих умов уже не почув. Крешимир повернув собі свої очі, а його, Тімів, давній утрачений дитячий сміх був так близько, ось, рукою досягти! Той сміх, що його Тім прагнув понад усе на світі!

Хлопець не міг більше стримуватися. Він кинувся вперед і закричав:

- Віддайте мені мій...

Але перечепився об бухту линви, вдарився головою об ніс шлюпки й гримнувся на палубу непритомний.

Дванадцятий аркуш

КРЕШИМИР

Тім розплющив очі. В круглому ілюмінаторі перед ним танцювала вгору і вниз червоняста зірка - лихе вогнисте око Марса. Хлопець лежав на верхній койці в їхній із стюардом каюті. Над Атлантичним океаном сірів світанок.

Унизу в каюті хтось товкся. Тім повернув голову й побачив Крешимира. Стюард теж якраз повернув голову до нього й погляди їхні зустрілись. У тьмяному світлі Крешимирового нічника Тім побачив, що в стюарда вже інші очі - карі, лагідні.

- Ну, синку, як ти себе почуваєш? - приязно спитав Крешимир. [76]

Тім іще не очутився як слід. Він не пам'ятав, як опинився тут.

А цей Крешимир, що заговорив до нього, був зовсім не такий, як той, що йому Тім допомагав у їдальні, а куди спокійніший і приязніший.

Стюард підійшов до койки.

- Тобі вже краще, синку?

Аж тоді в Тімовій голові помалу зринули всі події минулого вечора: голоси перед дверима до камбуза, розмова за шлюпкою, його власний крик і падіння.

- А де барон? - спитав він.

- Не знаю, Тіме. На судні його вже нема. Але скажи мені, ти нас учора ввечері підслухав?

Хлопець на койці кивнув головою.

- Я такий радий, що ви знову маєте свої очі, пане Крешимир!

- А ти, Тіме? Що ти хотів забрати назад у барона?

- Мій...- Тім затнувся. Йому пригадалась угода, й він стис губи.

Крешимир враз ляснув себе долонею по лобі й вигукнув:

- І як я досі не здогадався! Адже цей вельмишановний шахрай сміявсь, наче мале хлоп'я. Тож-то я й відчув у ньому щось чуже, таке, що не пасує до нього. Тепер вже я знаю! То був його сміх! Чи то пак,- Крешимир глянув Тімові просто в вічі,- твій сміх!

- Я вам цього не казав! - скрикнув Тім.- Чи, може... вчора ввечері й про те кричав?..

- Ні, Тіме, ти не встиг. І, мабуть, це твоє щастя. Я ж знаю ті пункти про мовчанку в баронових угодах. Заспокойся, ти нічого вчора не сказав.

Однак Тім не дуже заспокоївся. Йому треба було негайно ж перевірити, чи дійсна ще його угода. «Треба забитись із Крешимиром об заклад про щось зовсім неймовірне»,- надумав він і хотів був устати з койки, але тільки-но звівся, як у очах йому потемніло, заболіла [77] голова, кров загупала в скроні, і він знову впав на подушку.

Крешимир подав йому склянку води, й таблетку, наготовану заздалегідь.

- Випий оце, Тіме. І полеж сьогодні. До завтра все минеться. Наш стерничий - а він колись був санігаром - каже, що ти тільки ґулю собі набив.

Тім слухняно ковтнув таблетку, думаючи, про що б його забитись із Крешимиром. А надумавши, знову завів розмову про барона, бо заклад той стосувався саме його.

- А яку умову поставив вам барон, пане Крешимире? Цебто вчора ввечері, як віддав вам ваші очі?

- А ніякої! - засміявся Крешимир.- Коли ти закричав і гепнувся на палубу, надбігли матроси, і барон сховався в затінок від шлюпки. Я тоді шепнув йому: «Або я одержу свої очі без жодної умови, або ж я все розкажу оцим хлопцям!»

- Ну і що?

Крешимир знову засміявся:

- Барон аж заїкатись почав від хвилювання: «Не-нехай бу-буде бе-без жодної умови!»

Тім швиденько відвернувся до стіни, бо йому стало смішно, а що сміятися він не міг, то обличчя його тільки жахливо перекривилося.

- Цікавий я знати, де той барон тепер! - промурмотів Крешимир.

Оцього саме й чекав Тім. Уже опанувавши себе, він сказав:

- Закладаюся, що за п'ять хвилин ми дізнаємось, де тепер барон!

- А на що ти хочеш закластися?

- На порцію горіхового торта!

- Ну, на таке в мене грошей вистачить. Бо, коли не помиляюсь, ти маєш цей заклад виграти, як і всі заклади. Ну, гаразд, забилися! - стюард подав Тімові руку, і хлопець міцно стис її.

Ту ж мить у суміжній каюті хтось увімкнув радіо. [78]

Диктор оголосив прогноз погоди, а тоді стали передавати останні новини.

Тім із Крешимиром спершу розсердилися, що їх перебито, але потім почали прислухатись. Голос із гучномовця сповістив:

«Відомий комерсант барон Троч, що його багатство оцінюють у кілька мільярдів доларів, цієї ночі давав у Ріо-де-Жанейро банкет для ділових кіл бразільської столиці. Зразу ж на початку банкету він вийшов і повернувся лиш за дві години, видимо чимось збентежений. Усім упало в вічі, що повернувся він у темних окулярах. Напевне, в нього знову загострилася давня хвороба очей, що була нібито вилікувана й кілька років не поновлювалась. Нас повідомлено телефоном, що банкет триває і що барон, очевидно, знову...»

Радіо вимкнули і за стіною задзюрчала вода.

Тімове обличчя поблідло, як вранішнє світло. Він виграв заклад, а отже, угода була ще дійсна. Але його налякало те дивовижне повідомлення.

- Як же це можна дістатися так швидко до Ріо-де-Жанейро? - розгублено спитав він.

- Маючи такі гроші, можна що завгодно,- відповів Крешимир.

- Але ж так швидко навіть літаки не літають! - вигукнув хлопець.

На те стюард спершу не сказав нічого. Потім буркнув:

- А я гадав, ти знаєш, із ким зв'язався...

А тоді раптом заквапився до роботи. Але в дверях іще раз обернувся й сказав:

- Спробуй заснути, Тіме! В ліжку не годиться сушити собі голову.

На щастя, хлопець, маючи здорову натуру, справді заснув. А коли опівдні він прокинувся й Крешимир приніс йому юшки в каструльці та виграну порцію торта, Тімові стало якось аж легко на серці. Адже він тепер не сам ніс тягар своєї страшної таємниці, він поділяв її з іншою людиною. І ця людина, до того ж, перемогла барона! Це навівало Тімові таку надію й упевненість, [79] що він на якийсь час просто забув дивовижну звістку з Ріо-де-Жанейро.

По обіді до нього на хвилинку зайшов стерничий, височенний паруб'яга з Гамбурга, Джонні на ім'я. Він спитав Тіма, як той почуває себе, помацав йому ґулю на диво обережними пальцями, дав іще одну таблетку й сказав:

- Завтра ти вже бігатимеш, малий! Та бережись надалі, не перечіпайся на мотузках,- і пішов собі.

Тім подумав: «Коли б ти тільки знав, на якій падлючій мотузці я перечепився!» - і знову заснув: стерничий дав йому снодійну таблетку.

Пізно вночі, коли Крешимир повернувся до каюти, Тім прокинувся знову. Стюард, зіпершися ліктями на край Тімової койки, заговорив до хлопця:

- Ну й паскудство ж він з тобою зробив!

- Про що це ви?

- Та чого ти на мене викаєш?

- Ну... про що це ти?

- А про те саме! Я знаю, ти повинен мовчати. Добре, мовчи! Але я тепер і сам знаю, в чому річ: він сміється твоїм сміхом, а ти виграєш заклад. А що буде, коли ти програєш який-небудь?

- Оцього б я й хотів,- тихо відповів Тім. І більше не сказав нічого.

- Добре, поміркуємо,- мовив Крешимир. Тоді роздягся й ліг.

Обидва погасили нічники, й стюард почав розповідати Тімові про свою батьківщину, рідне село в Хорватії, на узбережжі Адріатичного моря. Сім днів на тиждень малий Крешимир був голодний, сім днів на тиждень мріяв про щастя й багатство. І ось одного дня селом проїхав автомобіль. За кермом сидів якийсь пан у картатому костюмі. Той пан дав Крешимирові пакуночок спілих червоних гранатів. Аж сім штук там було, і кожен гранат коштував тоді один динар. Крешимир поніс їх до моря, на пляж, за цілих десять кілометрів, і там продав. [80]

- Отак у мене, Тіме, вперше зроду з'явилися власні гроші, великі гроші, як мені тоді здпвалось - додав він - Цілих сім динарів! І знаєш, що я на них купив. Не хліба й хоч як мені хотілося хліба! Купив шматок торта. Великий такий шмат, із кремом, із вишнями зверху а посередині - половинка волоського горіха. Я про такі торти чув у селі від дівчат, що їм доводилось бувати на морі.

- Усі свої гроші віддав я за той шмат торта. А тоді сів на стосі дощок над причалом і зразу з’їв усе, думаючи собі: це ж ангели на небі щодня таке їдять!

А потім мене занудило, і я все, пробач на слові, виблював.

Мій бідняцький шлунок не міг такого стравити. Мені всі кишки повивертало. А як зійшов я з причалу й рушив додому - бачу, стоїть та машина з паном у картатому.

Крешимир замовк.

І Тімові раптом пригадався маленький хлопчик у вбогому [81] вузенькому завулку, де пахло перцем, кмином та ганусом.

Стюард почав розповідати далі: як картатий пан став частенько приїздити в село з гранатами, як він одного разу завів розмову з хлопцевими батьком та матір'ю, як узяв хлопця на один із своїх пароплавів стюардом, як потім не раз брав його з собою в подорожі, а найчастіше - на іподром, як Крешимир, закладаючись легковажно на перегонах, заборгував йому багато грошей і врешті змушений був продати найкраще з того, що мав,- свої очі.

- А тепер я вернув їх собі! - закінчив Крешимир.- І ти вернеш собі свій сміх. Я не я буду, коли не вернеш! Ну, добраніч!

Тімові здушило горло. Тоненьким голоском він відповів:

- Добраніч, Крешимире! Спасибі тобі!

Тринадцятий аркуш

БУРІ Й СТРАХИ

Крешимирове оповідання схвилювало Тіма. Та й море тої ночі дуже розбурхалося. Тому хлопець спав неспокійно, кидаючись та перевертаючись із боку на бік. Серед того ріденького сну раптом ударив грім, і за хвилинку крізь заплющені повіки сяйнула сонному хлопцеві сліпуча блискавка й у вухах загримів іще страшніший грім.

Тім, скрикнувши, підхопився. В тому громі йому вчувся власний сміх. Він розплющив очі, й погляд його впав на ілюмінатор. Крізь грубе скло дивились до каюти, просто в обличчя хлопцеві, двоє водяво-блакитних очей.

З переляку Тім знову заплющився, і його обілляв піт. Він не міг навіть ворухнутись - так і сидів, скорчившись, цілу вічність, поки нарешті зважився знову розплющити очі й тихенько покликати Крешимира. [82]

Стюард не обізвався. Надворі, за тоненькою залізною стінкою, пінилось море, і хвилі розмірене, важко били в борт. Тім боявся знову глянути в ілюмінатор. Він покликав Крешимира гучніше, але відповіді й тепер не почув.

Тоді він закричав не своїм голосом, аж сам злякався того крику:

- Крешимире!

Але й на той нелюдський крик відповіді не було.

Тім знову заплющив очі, щоб не глянути ненароком в ілюмінатор, і помацки знайшов над головою шнурок від вимикача. З хвилювання хлопець так сіпнув за нього, що аж обірвав, але лампа засвітилася. І Тім розплющив очі.

Разом із темрявою й страхи поховались по кутках. Тім перехилився через край койки й глянув униз. Крешимира не було.

Тоді з усіх кутків порожньої каюти знову посунув на нього страх. Хлопець затремтів усім тілом, побачив себе в дзеркалі над умивальником і сам злякався спотвореного обличчя, що дивилось на нього з дзеркала,- свого власного обличчя.

Але той ляк, на диво, ніби розбуркав, підштовхнув Тіма. Він сплигнув з койки й хапцем одягнувся. Здавалось, неначе всі страхи перейшли в дзеркало, до Тімового відображення, а сам він звільнився від них і міг робити, що хоче. Отож він набрався відваги навіть вийти з каюти в коридор. Пароплав дуже хитало на хвилі, але Тім зумів помацки дістатися до залізного трапа й видерся ним угору. На шкафуті його зразу облило й до рубця промочило хвилею, але він чіплявся за натягнені линви, за підпірки й уперто пробирався далі. Зціпивши зуби, видерся на човнову палубу й добувся нарешті до теплої, задимленої тютюном, освітленої тьмяною лампочкою стернової рубки.

Там спокійнісінько стояв стерничий Джонні, здоровило з Гамбурга. Він здивовано глянув на Тіма:

- Чого тобі тут треба в таку негоду? [83]

- Джонні, де Крешимир? - голосно гукнув Тім, силкуючись перекричати гримотіння хвиль.

- Крешимир захворів, Тіме. Але не хвилюйся. Звичайний апендицит. Від цього нині не помирають.

- А де ж він? - не вгавав Тім.- Де він тепер?

- Ми зустріли випадково патрульний катер, і Крешимира забрали, щоб відвезти на берег. Ти хіба не чув, як зупиняли машину?

- Ні,- пригнічено відказав Тім і додав уже спокійніше: - Крешимир не хворий. То все барон підстроїв. Я бачив його очі в ілюмінаторі.

- Ти бачив в ілюмінаторі баронові очі? - Джонні засміявся.- Тобі щось верзеться, хлопче! Роздягайся та лягай он на тапчанчику. Ось тобі ковдра. Тут, у мене, тобі ніщо лихе не присниться.

В теплій рубці, поруч цього спокійного, добродушного велетня Тімові й самому вже здалося, ніби то все йому просто примарилось. Але враз він знову згадав повідомлення по радіо про баронове зникнення з банкету в Ріо-де-Жанейро, згадав своє власне перекривлене обличчя в дзеркалі над умивальником і подумав, що від барона можна сподіватися чого завгодно. І Тім вирішив надалі не боятись його, бо, на своє щастя, бачив барона й переможеного.

Тім мовчки ліг на м'який тапчанчик, що розгойдувався туди й сюди: адже тут, нагорі, хитавиця давалася взнаки ще дужче, ніж унизу, в каюті.

Думки клубком спліталися в голові, під грудьми млоїло від хитавиці, й Тім так і не зміг заснути. Годину за годиною лежав він, дивлячись на Джонні, що спокійно стояв біля стерна, час від часу запалюючи нову сигарету. Буря помалу втихала.

Усі ті години Тім вигадував собі якийсь незвичайний заклад. Такий незвичайний, щоб його не можна було виграти. Барон лякав Тіма - хай же злякається й сам! Та хоч скільки сушив собі голову хлопець, він не міг придумати нічого такого, що було б не до снаги диявольському баронові. Ну, скажімо, забився б він, що ліщиновий горіх [84] буває більший за кокосовий. Та хто ж це погодиться на такий безглуздий заклад? А крім того хто знає, чи Троч не зуміє розшукати десь таке чудне місце, де ліщинові горіхи справді ростуть більші за кокосові? І Тім відкинув цю вигадку, як і безліч інших. Бо йому раз у раз спадала на думку його пригода з паном Рікертом у трамваю.

А що подумав він раптом, коли, скажімо, важкий залізний трамвай муситиме знятись із рейок і полетіти? Трамвай же не горобець, та й Троч, попри всі свої неймовірні й страшні здібності, не чаклун.

Тім вирішив, що знайшов у барона ахіллесову п'яту. Він звівсь на лікті й гукнув:

- Джонні, а ви знаєте, що в Генуї трамваї літають?

- Лежи й спи, - не дуже й здивовано відказав стерничий - Знову тобі щось приверзлося!

- Вибачте, Джонні, але ж я не сплю, то як мені може щось верзтися? Я цілком напевне знаю, що в Генуї можна побачити летючі трамваї. Ось забиймося об заклад. На пляшку рому?

- Тринди-ринди,- відказав Джонні.- А крім того, цікавий я знати, де ти візьмеш грошей на пляшку рому?

- А в мене в торбі є готова пляшка! - збрехав Тім.- Ну то як, заб'ємося?

Джонні обернувся до нього й сказав:

- Хоч би й на мільйон закладався, однаково не повірю. Бо я надто добре знаю й Геную, й трамваї.

- То чого ж ви боїтеся? Виграєте пляшку рому!

- А даси мені слово честі, що ляжеш і заплющиш очі, коли я з тобою заб'юся?

- Слово честі!

Тоді стерничий подав йому руку, промовив: - Коли в Генуї - і затнувся, бо в вікно неначе вдарило щось тверде. Але нічого більше не сталося, й Джонні проказав знову - Коли я побачу в Генуї летючий трамвай, то я програв заклад і ставлю тобі пляшку рому. А як не побачу, то пляшка, що в тебе в торбі, буде моя. Добре? А тепер лягай, будь ласка, і спи. За три години тобі ставати до роботи. [85]

Цього разу Тім справді заснув. І вві сні чув власний сміх, але його перекривав залізний брязкіт трамвайного вагона, що летів по небі в Тіма над головою. Коли стерничий на світанку розбудив хлопця, в того ще бряжчало в вухах, і Тіма те лякало.

Він боявся Генуї.

Чотирнадцятий аркуш

НЕМОЖЛИВИЙ ЗАКЛАД

Тім боявся Генуї, але водночас йому й нетерпілося прибути туди.

І те боязке нетерпіння мучило його довго, бо минуло ще багато днів, поки пароплав «Дельфін» одного ясного сонячного ранку, десь перед дванадцятою годиною, підплив до Генуезького порту.

Тім знайшов якусь приключку, щоб забігти в рубку до стерничого. Там він став поруч Джонні й звів очі на горішню частину міста. На хлопцеві були штани в чорно-білу клітинку й фартух із сирового полотна, що давав йому кок Енріко, як треба було чистити картоплю.

Генуезькі будинки видно було вже виразно. Навіть автобуси та легкові автомобілі можна було розглядіти на горішніх вулицях міста. І щохвилини картина та ставала ще виразніша.

Раптом Джонні здушено скрикнув, ніби з подиву. Тім сторопіло глянув на нього: стерничий заплющив очі. Потім розплющив, знову заплющив і ще раз розплющив. А тоді сказав повільно, майже урочисто:

- Здурів я, чи що?

Тімові шибнув у голову страшний здогад. Йому аж у горлі пересохло. Але він не важився подивитися знову на Геную. Він прикипів очима до Джонні.

Той перевів погляд на нього й сказав, хитаючи головою:

- Твоя правда, Тіме: таки є в Генуї летючі трамваї! Ти виграв. [86]

Тім ковтнув клубок у горлі. Шкода відвертати очі від неминучого. Він повернув голову й глянув на горішнє місто. Там на одній вулиці, поміж будинками, летів у повітрі трамвай, справжній трамвай. Його було видно дуже виразно. Та ось під трамваєм з'явився брук, твердий брук із рейками на ньому. Трамвай уже не летів, а котився по рейках вулицею.

- Так то ж було тільки марево! - зраділо вигукнув Тім.- Я програв!

- Ти неначе радієш із того, що програв! - здивовано сказав Джонні, й Тім зрозумів, що сплохував. Та перше ніж устиг поправитися, Джонні додав: --Однаково ти виграв. Ти ж закладався» що в Генуї можна побачити летючі трамваї, а не що вони там справді є. А я таки їх побачив, нікуди не дінешся.

- Тоді, виходить, я й справді виграв. От і добре,- промовив Тім, силкуючись говорити весело. Та голос його був хрипкий і зовсім не радісний.

Добре, що Джонні саме відвернувся до стерна.

- А чого це тобі раптом спав на думку такий чудернацький заклад? - спитав він через плече.- Тобі часто щастить у закладах?

- Я ще ніколи ні одного не програв,- байдуже відказав Тім.- Я виграю завжди.

Стерничий скоса зиркнув на нього:

- Не дуже вихваляйся, хлопче. Є такі заклади, що їх неможливо виграти.

- Ану, які ж це? - зацікавлено спитав Тім.- Назвіть мені хоч один.

І знову Джонні допитливо зиркнув на нього. Щось у цьому хлопцеві бентежило стерничого. Але він звик завжди відповідати, коли його питали. Отож він насунув свого білого кашкета на чоло й почухав потилицю. І знов у вікно вдарилось щось тверде. Джонні повернув голову, але нічого не побачив. І враз йому спала на думку відповідь.

- Я придумав один заклад, такий, що ти його нізащо не виграєш, Тіме! [87]

- Ану, давайте заб'ємося. Я наперед згоден. Як програю, залишите ваш ром собі.

- Хочеш купити кота в мішку? Ну що ж, про мене. Від рому я ніколи не відмовлявсь, і якщо ти неодмінно хочеш програти, то будь ласка. Отже, закладаємося...- Джонні урвав мову, подивився на хлопця й спитав: - Ти по-справжньому хочеш забитись? Бо ж пляшка рому...

- По-справжньому, аякже! - відказав Тім так упевнено, що Джонні заспокоївся.

- Отже, закладаюся, що сьогодні до вечора ти не станеш багатший за найбагатшого чоловіка на світі!

- Цебто за Троча? - майже нечутно видихнув із себе Тім.

- Атож!

І хлопець простяг йому руку швидше, ніж Джонні сподівався. Це ж справді неможливий заклад! І Тім його програє! Він голосно, дзвінко сказав:

- Закладаюся, що іще до вечора я буду багатший за барона Троча!

- Ти схибнувся, хлопче! - відказав Джонні й випустив Тімову руку.- Зате я принаймні відіграю назад свій ром.

У ту хвилину до рубки зайшов капітан.

- А що тут робить бой? - похмуро спитав він.

- Та це я його покликав, щоб він приніс мені кави, капітане,- відповів Джонні.

- То чого ж він витрішки продає?

І Тім мерщій побіг униз до камбуза. Йому аж співати хотілося. Але хто не вміє сміятися, той не вміє й співати.

Коли він приніс у рубку каву, тільки двічі схлюпнувши її дорогою, капітан іще був там. Джонні, осміхаючись, підморгнув хлопцеві за спиною в старого. Тім підморгнув йому у відповідь, однак із поважною міною. Тоді збіг униз на палубу. Його поривало голосно засміятись, але уста тільки скривилися в неприродну гримасу сміху, і регіт не вирвався з них. [88]

Одна літня голландка, що зустрілася хлопцеві на палубі, аж перелякалась того дикого виразу на його обличчі. Згодом вона сказала своїй сусідці по каюті:

- Замикайте вночі свою каюту. В цьому хлопчиськові сам чорт сидить.

Схвильований Тім забіг аж на ніс пароплава, сховався там за брашпилем, сівши на купу змотаних линв, і вирішив не виходити зі схованки, аж поки пароплав стане на якір у Генуї. Він чув колись, що в цьому місті є славнозвісний ляльковий театр. Туди він хотів піти, щоб насміятися досхочу серед загального сміху. А ще приємніше було уявляти собі, як він гулятиме вулицями та всміхатиметься кому-небудь незнайомому, що вподобається йому: то маленькій дівчинці, то старенькій бабусі. Тім весь поринув у той уявний сонячний, радісний світ. А сонце, що світило йому в обличчя з синього неба, робило ту мрію ще живішою й правдоподібнішою.

Деренчливий голос із корабельного гучномовця щось оголосив. Замріяний Тім не почув нічого.

За хвилину оголошення пролунало ще раз. Уловивши в ньому своє ім'я, Тім прислухався пильніше і встиг почути кінець:

«... Талера негайно до капітана в стернову рубку!»

Мрія луснула, мов мильна банька. І щедре південне сонце враз потемніло в Тімових очах.

Понурий, байдужий до всього капітан доти ні разу не звертав уваги на Тіма. Отже, видно, трапилося щось надзвичайне, коли він тепер викликає хлопця.

Тім підвівся, вийшов із-за брашпиля, перейшов палубу й утретє за той ранок піднявся залізним трапом на човнову палубу, тримаючись за залізне поруччя раптом спітнілими, аж мокрими руками.

У рубці капітан глянув на нього якось чудно, зовсім не так байдуже, як доти. Зате стерничий дивився просто перед себе й навіть не повернув до Тіма голови.

- Тебе звуть...- капітан затнувся, прокашлявся й почав знову: - Вас звуть Тім Талер? [90]

- Так, пане капітане!

- Ви народились...- з аркуша паперу, що його тримав у руці, капітан вичитав, де й коли народився хлопець, і Тім щоразу підтверджував:

- Так, пане капітане!

Від нетерпіння й цікавості йому аж сльози на очах виступили.

Коли той коротенький допит скінчився, капітан опустив руку з папірцем, і в рубці запала якась дивна тиша. На підлозі тремтіли сонячні зайчики; Тім стояв, утупивши очі в широку потилицю стерничого, що все дивився просто перед себе.

- Тоді дозвольте мені першому поздоровити вас,- порушив тишу капітан.

- Із чим, пане капітане? - тоненьким, писклявим голосочком спитав Тім.

- А ось із оцим! - капітан кивнув головою на аркуш у своїй руці. Тоді спитав: - Ви родич баронові і Трочеві?

- Ні, пане капітане...

- Але ви з ним знайомі?

- Та знайомий...

- Тоді слухайте радіограму:

«барон троч помер крапка повідомте тіма талера що за заповітом він єдиний спадкоємець крапка брат небіжчика новий барон троч призначений опікуном до повноліття спадкоємця крапка пароплавство фенікс акційного товариства троч і компанія генуя підписав директор грандіцці».

Тім скам'яніло дивився на потилицю стерничого. Отже, найнеймовірніший у світі заклад виграно. За одну пляшку рому. Він, чотирнадцятирічний хлопчисько, став у ту хвилину найбагатшою людиною в світі. Але сміх його помер разом із бароном і разом з ним буде похований. Найбагатша людина в світі була найбідніша з усіх людей: вона навіки втратила свій сміх. [91]

Потилиця стерничого ворухнулась. Повільно-повільно Джонні повернув голову, й на Тіма глянули чужі, здивовані очі. Але хлопець бачив їх лише одну мить.

Джонні ледве встиг підхопити зомлілого Тіма.

П'ятнадцятий аркуш

ПЕРЕПОЛОХ У ГЕНУЇ

Опритомнівши, Тім побачив перед собою неголене вилицювате обличчя, з якого дивились на нього двоє приязних блакитних очей.

- Ти мене чуєш? - спитав тихий голос.

- Чую, Джонні,- шепнув Тім.

Велика рука трохи підвела йому голову, обережно піднесла до рота склянку з водою. Потім голос біля вуха прошепотів:

- Чому це я бачив у Генуї летючий трамвай? І чому барон помер так вчасно? І чому ти не зрадів, що виграв, а зомлів?

У Тімовій голові помалу яснішало, і в ній аж гули ті «чому», розворушуючи безліч власних, давніх Тімових «чому». Від безмежної розгубленості він трохи знову не зомлів.

У ту хвилину почулися чиїсь кроки й голоси, й до рубки ввійшов капітан із якимсь незнайомим паном.

Тім, лежачи на канапці, спершу вгледів із того пана тільки здоровенну сніжно-білу мереживну хусточку, що стриміла з нагрудної кишеньки на темному піджаці. Та ще враз запахло гвоздиками. Тімові аж у голові запаморочилось від того запаху, коли незнайомий пан підійшов і відрекомендувався:

- Діретторе Грандіцці. Я маю себе за дуже часливи перши привітати вас ім'я вся товариство, синьйоре! Я дуже шкода, що ви не зовсім здорови, але розумію невелички удар...- Він розвів руки й схилив голову набік: - Ах, таки багати за таки коротки мить, це вераменте, цебто справді нелегко, але... [92]

Що казав директор Грандіцці далі, Тім не зрозумів. Ще слабий, він стомився, слухаючи, і розчув лише останні слова, бо директор низько нахилився над ним:

- Зараз я віднесу вас до катер, синьйоре!

Але тут утрутився Джонні:

- Я сам віднесу,- буркнув він.- Не чіпайте хлопця. Пане капітане, потримайте стерно!

Хоч пароплав уже об'якорився, капітан, як і всі, був такий сторопілий, що мовчки, покірно став до стерна.

При борті «Дельфіна» гойдавсь на хвилі катер пароплавства, що мав відвезти спадкоємця на берег. Джонні спритно балансував по трапах,, несучи на руках Тіма, немов клуночок із білизною. А директор Грандіцці махаючи напахченою хусточкою, скакав довкола нього, мов пудель довкола свого хазяїна. Аж тепер Тім побачив, що голова в директора майже зовсім лиса Тільки два чорних пасемця з боків були начесані гострим кутом на його чоло. Через них кругле обличчя здавалося ледь-ледь загрозливим і подібним до машкари.

В катері від носа до корми тяглись попід бортами довгі лави. Джонні відніс Тіма на корму й посадив там на подушки, намощені в кутку лави. Несучи хлопця, він потай шепнув йому: «Я ще маю віддати тобі дві пляшки рому. Ті, що програв. Приходь о восьмій годині до пам'ятника Христофорові Колумбові. Тільки сам. А надто як тобі потрібна буде поміч, приходь неодмінно. Зрозумів?»

Тім не кивнув головою. Він тільки тихо гмукнув: «Угу», бо навчився бути обережним.

- Ну, щасти тобі, Хлопче,-товстим голосом прогул Джонні навмисне для директорових вух. Тоді подав Тімові руку й подерся трапом на пароплав.

Коли катер відчалив від борту, знову запахло гвоздиками: біля Тіма сів директор Грандіцці. Двом по-святковому вбраним добродіям, що сиділи на носі катера, він подав рукою знак поводитися тихіше. Вони згідливо [93] закивали головами й зашепотілися, з неприхова-ною цікавістю в очах дивлячись на Тіма.

- Синьйоре, я вас відвезу до ваша готель,- півголосом мовив директор.- Там ви будете спочинути одна година, а потім пароплавство чекає вас на один невеличка прийом. Гаразд?

Тім, який щойно був боєм на не дуже й великому вантажно-пасажирському пароплаві, раптом опинивсь у ролі багатого спадкоємця, коло якого всі на задніх лапках ходять. Проте, в гонитві за своїм сміхом уже трохи звикнувши прикидатись, він не спантеличився. Турбувало його зовсім інше: що тепер йому й гнатись більше ні за чим, бо сміх його помер.

Він ствердно кивав головою на все, що пропонував йому директор Грандіцці. Тільки один раз, коли директор заговорив про зустріч із журналістами о восьмій годині ввечері, Тім похитав головою.

- А, ви не любите преса, синьйоре? Але ж газети дуже корисні, синьйоре, дуже багато корисні!

- Я знаю,- відповів Тім. У катері, що злегенька погойдувався на хвилі, він уже відчув себе краще.

- Коли ви знаєте, чо газети багато корисні, то чому ви не хочете зустрітися з журналісти? - не вгавав директор.

- Бо...- Тім аж лоба наморщив, гарячкове придумуючи відповідь.- Бо все це ще надто нове для мене. Чи не можна відкласти ту зустріч на завтра?

- Можна, аякже, синьйоре. Але сьогодні ввечері...

- Сьогодні ввечері я хочу сам сходити оглянути місто,- перебив його Тім. Директорова підлесливість просто спокушала гримати на нього. Але Грандіцці нітрохи не образився.

- Ні, ні, синьйоре, не сам,- відказав він рішуче.- Одна детектіве тепер буде вас завжди проводити, бо ви ж таки багати, розумієте?

- А я хочу погуляти по місту! - вигукнув Тім.

Святково вбрані пани на носі катера спантеличено [94] поглянули на хлопця. Один протанцював по хисткому катері до нього й спитав:

- Можу я чимось прислужитися? З вашої ласки, я Пампіні, головний перекладач пароплавства.- Видно, він просто скористався з нагоди, щоб відрекомендуватися багатому спадкоємцеві. Та тільки-но він простяг хлопцеві руку, як катер круто звернув праворуч. Перекладач упав прямо Тімові в обійми, підхопився, ревно перепрошуючи хлопця, але катер крутонувся знову, й Пампіні впав уже в обійми директорові Грандіцці.

Директор, весь почервонівши, мов буряк, нагримав спершу на перекладача, потім на стерничого. Одного він назвав бовдуром, другого йолопом. Потім йому спало на думку, що стерничий однаково не розуміє по-німецькому, і він налаяв його ще раз, по-італійському, від чого та лайка стала принаймні вп'ятеро довша. Перекладач, знітившись, вернувсь на своє місце. І зразу ж катер пристав до причалу. [95]

Там уже стояв шофер у синій формі, шанобливо тримаючи кашкета в руці. Він простяг Тімові руку, щоб допомогти йому вийти з катера, а директор скоріше про око підтримав хлопця під лікоть, Тім перший ступив на причал. Із ним панькались так, наче з дуже старим, немічним чоловіком.

Нагорі, на пірсі, генуезький краєвид заступила від нього ціла шеренга панів у темних костюмах. Директор Грандіцці по черзі відрекомендував їх Тімові. У всіх у них прізвища кінчалися на «іцці», або «оцці», й Тім зразу ж їх усі позабував.

Химерне вийшло видовище: все те врочисте мурмотіння було звернене до чотирнадцятирічного хлопця в дрібнокартатих кухарських штанях, випнутих пухирями на колінах, та в завеликому для нього светрі стерничого Джонні. По правді сказати, сміховинна була картина. Але ніхто навіть не всміхнувся, всі лишалися поважні-поважні. І так, певно, було краще для сердешного Тіма.

І далі все відбувалося теж поважно й церемоніальне: під'їхала довга чорна легкова машина на шестеро дверцят, шофер відчинив дверцята спершу Тімові, тоді директорові Грандіцці, обидва посідали на червоні шкіряні сидіння, машина рушила, і вся шеренга панів у темних костюмах попіднімала праві руки й з гідністю помахала їм услід.

Аж як виїхали з порту в місто, Тім згадав про свою матроську торбу, панів Рікертів подарунок, що лишилася на пароплаві. Коли він сказав про неї Грандіцці, директор усміхнувся:

- Звичайно, ми можемо привезти ваші речі з пароплава, синьйоре. Але пан барон уже подбав про елегантна гардероба для вам.

- Барон? - сторопіло перепитав Тім.

- Нови пан барон, синьйоре.

- А!..

Тім відкинувсь на сидіння й уперше, глянувши крізь вікно машини, побачив трошки Генуї: мармурові сходи [96] з колонами й жовту мідну табличку на дверях із написом: «ГОТЕЛЬ «ПАЛЬМАРО». Потім промайнули віяла невисокої пальми, далі кругла клумба з кущем лаванди посередині. А потім машина дуже плавно зупинилася. Шофер відчинив перед хлопцем дверцята, і швейцар у лівреї, обшитій золотими шнурами, взяв Тіма за руку та вивів з машини - знову так обережно, мовби старенького дідуся.

Тім опинився перед мармуровими сходами. З верхньої приступки хтось гукнув йому:

- Просимо, просимо!

Тім звів очі й побачив високого худого чоловіка в картатому костюмі й великих темних окулярах.

- Нови пан барон, рідни брат-близнюк небіжчик,- шепнув хлопцеві на вухо Грандіцці. Але Тім чогось не дуже повірив у того брата-близнюка. А коли новий барон, спускаючись сходами, вигукнув: «Що за чарівний голодранець?» - і засміявся. Тім за мить знав уже більше за директора. Він упізнав цього чоловіка по своєму власному сміхові. Ніякого брата-близнюка не існувало.

Барон був живий. А з ним і Тімів сміх.

Шістнадцятий аркуш

КІНЕЦЬ однієї люстри

У готелі, в своїй розкішній кімнаті, чи то, краще сказати, в своїх покоях із трьох кімнат (такі покої називаються ще апартаменти) Тім уперше за той день після всіх несподіванок і тривог лишився сам. Барон поїхав на якусь нараду, сказавши, що іще заїде по Тіма пізніше.

Хлопець, що й досі ще мав на собі штани в клітинку та бахматий светр, напівлежав у шезлонгу. Спина його й голова спочивали на купі смугастих шовкових подушок, а ноги звисали вниз. Тім не спускав ока з великої люстри, зробленої неначе зі скляних сліз. [97]

Уперше за довгий-довгий час у хлопця було майже легко на серці. Не раптове багатство та шана тішили його; про те багатство Тім навіть не мав іще справжнього уявлення. Хлопця тішило, що його сміх не вмер. І ще, попри все своє замішання, Тім збагнув: барон тепер його опікун, а отже, якось прив'язаний до нього. І в своїй гонитві за власним сміхом. Тім принаймні весь час матиме ту дичину перед носом. Тепер треба тільки знайти в барона вразливе місце.

Тім іще не знав, що здалеку іноді буває видніше, ніж зблизька.

У двері постукали, і, не дожидаючи Тімового запрошення, ввійшов барон.

- Лежи, лежи,- сказав він хлопцеві. Потім, переломившись, мов складаний ножик, упав у розкішне крісло, викладене візерунками із слонової кості. Закинувши ногу за ногу, ,він весело поглянув на Тіма:

- А оцей останній заклад придумано непогано, Тіме Талере! Вітаю, вітаю!

Тім дививсь на барона знизу й мовчав.

Троча, видно, бавило й те. Він спитав:

- А ти, власне, хотів виграти чи програти? Мені дуже цікаво це знати.

Тім відповів ухильно:

- Об заклад звичайно забиваються, щоб виграти...

- Ну, то це було просто чудово придумано! - захоплено вигукнув барон. Тоді схопивсь на ноги, схрестив руки на грудях і почав ходити туди й сюди по Тімових покоях.

Тім, лежачи на шезлонгу, спитав його:

- А наша угода, власне, ще дійсна? Я ж уклав її з тим, першим бароном Трочем, а не з його гаданим братом.

Троч повернувся з вітальні до спальні й сказав, не спиняючись:

- Угоду укладено з паном Т. Трочем. Мене звуть Тобіас Троч. Раніше я називав себе Томас Троч. І так, і так Т., хлопчику мій. [98]

- А кого ж поховано замість вас, коли ніякого брата не існує? - допитувався Тім.

- Одного вбогого, безрідного пастуха, мій юний друже,- Задоволене плямкнувши губами, Троч додав: - На Месопотамському нагір'ї, поблизу гори Джебел-Сінджар, розташована моя головна резиденція, невеличкий замок. Ото там його зараз і ховають замість мене.

Він знову вийшов у вітальню й уже з-за дверей Тім почув:

- Мій замочок стоїть у єзидському краю. Ти знаєш, хто такі єзиди?

- Ні, - відказав Тім, здивований бароновою балакучістю.

Голос наблизився знову:

- Єзиди - це плем'я, що поклоняється чортові, дияволові. Вони вірять, ніби бог простив диявола й віддав під його владу всі земні справи. Тому вони й поклоняються чортові як володареві землі.

Барон знову вернувся до спальні. Тім сказав досить байдуже:

- А, он як!

- А он як! - передражнив хлопця барон видимо роздратовано, і веселий вираз уперше зійшов з його обличчя.- Тебе, видно, чорт зовсім не цікавить, еге?

Тім не зрозумів, чого це раптом барон так схвилювався, і простодушно спитав:

- А хіба чорт і справді є на світі?

Троч знову впав у оздоблене слоновою кістю крісло.

- Ти справді такий наївний чи тільки прикидаєшся? Невже ти ніколи не чув про людей, що укладали угоди з чортом і підписували ті угоди своєю кров'ю?

Зачувши слово «угода», Тім зразу нашорошився. Він думав, що Троч зараз заговорить про їхню угоду. Але барон теревенив далі про чортів та демонів. Розповідав про Вельзевула, володаря пекла, про демонів Форкаса, Астарота й Бегемота, про відьом, про чорну магію, [99] про славнозвісного чаклуна доктора Фауста, що мав за слугу молодшого чорта Мефістофеля.

Помітивши, що хлопець нудиться, Троч підвівся й промимрив сам до себе:

- Доведеться діяти відвертіше...

Тім знову ліг у подушки і втупив очі в люстру. Звісивши додолу праву руку, він несвідомо крутив у ній одну з шовкових пантофель, що поклали йому біля шезлонга. В скляних бурульках люстри, химерно спотворена, безліч разів відбивалася кощава баронова постать.

Троч спитав навпростець:

- Хочеш, я навчу тебе тієї замови, що нею доктор Фауст приворожив собі свого чорта?

- Навіщо вона мені...- не повертаючи голови, відказав Тім і побачив, як тіпнулося з досади розмножене в скляних бурульках баронове обличчя. Тоді знову почув його голос.

- То, може, хоч переказати її тобі? - спитав Троч, видимо насилу стримуючи злість.

- Як хочете, бароне...- 3 хлопцевого голосу добре чути було, що все те йому зовсім не цікаве. А втім, цікавість ворухнулась у ньому, коли він побачив, як малюсінькі Трочі в шліфованих скельцях заклинально підносять тонюсінькі ручки.

Троч дуже повільно, якимсь дивно глухим голосом почав вимовляти чудернацькі слова:

Багабі лака бахабе

Ламак каї ахабабе

Карелійос

Ламак ламек Бахаліяс

Кабаагі забалійос...

По цих словах люстра затремтіла - може, від того, що барон вимахував руками,- і з її середини почав спускатись на павутинці здоровенний сполоханий павук.

Тім дуже гидував павуками. І взагалі ота таємнича замова вже дратувала його. Тому він підняв шовкову [100] пантофлю, що її крутив у руці, й щосили жбурнув на павука.

Барон саме вимовляв дальші слова:

Барій олас

Лагоц ата кабійолас...

І враз під стелею щось тонко заскреготіло, а тоді в ногах у Тіма задзвеніла, забряжчала, загримотіла, впавши додолу, вся величезна люстра.

Тім перелякано підібгав ноги. Барон стояв за спинкою свого крісла, роззявивши рота, з іще піднесеними догори руками, а на лобі в нього вже набігла ґуля. Видно, його зачепило люстрою.

В покоях стало тихо-тихо. Але оглушливий брязкіт, певно, почули в готелі, бо за хвилинку хтось міцно постукав у двері.

Аж тоді барон опустив руки. Згорбившись, наче вкрай знесилений, підійшов він до дверей, прочинив їх трошечки й сказав кілька слів по-італійському. Тім нічого не зрозумів.

Потім Троч знову причинив двері, сперся на них плечима й промовив:

- Марна річ... На чисте серце нема ради.

Хлопець у шезлонгу, що зрозумів ті слова не краще, ніж таємниче закляття, підвівся й спитав:

- Що марна річ?

- Середньовіччя! - відповів барон начебто зовсім не до ладу, й Тім однаково нічого не зрозумів. Тому він не став далі допитуватись, а сказав:

- Вибачте... Я ненавмисне люстру звалив. Я просто хотів павука вбити.

_ Ет, пусте. Запишуть у рахунок за номер, заплатимо і квит,- відмахнувся барон.

- Як то «заплатимо»? - обурився хлопець: йому нараз пригадалося його незмірне багатство. І він додав: - За люстру заплачу я сам, бароне!

- Нічого не вийде,-відказав Троч, і в кутиках його вуст ізнову заграли веселі зморщечки.- Ти, любий [101] мій, іще неповнолітній, а тому не можеш витратити жодної марки без дозволу свого опікуна, барона Тобіаса Троча.- І, осміхаючись, уклонився: - Але кишенькові гроші ти, звичайно, одержуватимеш.

Тім теж уклонився й мовив:

- Вам, я бачу, також спадають на думку вельми розумні речі, бароне. Але дозвольте мені нарешті пере вдягтися. Я хотів би лишитися сам.

Троч спершу безмовно витріщився на хлопця, тоді лунко зареготав.

- А в вас, я бачу, таїться більше, ніж я сподівався, Тіме Талере! Низенько кланяюсь! - сміючись, вигукнув він. І аж тоді помітив, як зблід Тім від того сміху. Веселий регіт, що ним Троч, мов арканом, притягував до себе людей, цього хлопця не брав, бо то був його власний, Тімів, регіт.

Троч швидко рушив до дверей. Але, перше ніж вийти, тернув рукою по блискучій стільниці невеличкого письмового столика, що стояв біля самих дверей, і, зиркнувши скоса на Тіма, посунув на середину столика шкіряний бювар.

Аж тоді він відчинив двері, кинувши через плече:

- Я завжди до ваших послуг, пане Талер. Слугу я покличу. Це мій вірний чоловік із Месопотамії.

- Спасибі,- відказав Тім.- Я звик перевдягатися сам.

- Іще краще,- осміхнувся Троч.- Дешевше буде.-

І нарешті вийшов, тихенько зачинивши за собою двері.

В коридорі барон хвилинку замислено постояв.

- Хлопець хоче повернути собі свій сміх,- промурмотів він сам до себе.- Він зневажає ту силу, що її дає темрява. Чи, може, просто байдужий до неї. Він хоче світла, а світло...- барон повільно рушив до свого номера.- Світло ламається в дзеркалі. Треба спробувати ще так...

Вернувшись до своїх покоїв, Троч знову впав у крісло. Над ним висіла така сама люстра, як у вітальні Тімового номера. Баронів погляд упав на скляні бурульки, що злегенька погойдувалися. Йому пригадалось, як Тім пожбурив у люстру пантофлю; й раптом [102] Троч засміявся. Він реготав так, що крісло під ним аж двигтіло, аж рипіло.

Барон сміявсь, як малий хлопчак. Сміх здіймавсь йому з нутра, як бульбашки в келиху з шампанським. Починаючись із найнижчої ноти, перебігав по всій гамі - вище й вище - і кінчався кумедним «ік!». Барон переводив дух - і знову ціла гама й кумедне «ік!» на кінці.

Однак барон ніколи не міг весело й бездумно віддатись хоч би якому почуттю. Йому бракувало хисту бути просто щасливим. Він мусив усе пояснювати собі, розбирати на частки, аналізувати - навіть свої почуття.

Так і того разу - тільки-но ікнувши востаннє, барон замисливсь: а чого ж він сміється? І здивовано відзначив, що сміявся з самого себе, зі своєї невдалої спроби справити на Тіма Талера враження фокусами чорної магії.

Спроба не вдалася, Троч зазнав поразки, і все ж він сміявся. Досі з бароном такого не бувало ще зроду.

Він підвівся з крісла, й, ходячи туди й сюди по "вітальні, заговорив сам із собою.

- Дивовижна річ,- бурмотів він.- Я купив собі сміх, щоб здобути владу над серцями. А тепер...- барон уражено зупинився...- тепер я здобув владу над самим собою, над своєю норовливою вдачею. Я більше вже не підкоряюся своїм норовам, я їх проганяю сміхом! і

Троч знову почав ходити по вітальні.

- Раніш я нетямився з люті, коли не моє бувало ] зверху. Ладен був на стіну дертися. А тепер, навіть спіймавши облизня, засміюся - і моє однаково зверху!

Барон із майже щасливим виразом на обличчі схопився за голову й вигукнув:

- Це ж неймовірної Всю свою зверхність мені доводилось весь час підтримувати хитрістю, підступами, каверзами, перемогами над іншими людьми. А тепер вона сама спливає на мене, бо я вмію пускати з рота якісь кумедні хихи та хахи. Виходить, сміх вартий більш, ніж я гадав! За нього справно можна віддавати цілі царства! [104]

Вдоволений кощій знову впав у крісло, і обличчя його на мить стало таке лукаве, як колись, на іподромі.

- Поганяйся тепер за своїм сміхом, Тіме Талер! Ти його не наздоженеш! Бо я за нього руками й зубами держатимусь!

Сімнадцятий аркуш

БАГАТИЙ СПАДКОЄМЕЦЬ

Форма молодих багатих спадкоємців за тих часів була така: сірі штани з тонкого сукна, піджак у чорно-червону пересмужку, сніжно-біла шовкова сорочка, червона краватка в чорну клітинку, такі самі шкарпетки й жовті замшеві черевики.

Тім стояв у тому вбранні перед високим, від самої підлоги, дзеркалом і причісувався, вперше в житті примочивши задля того чуба. На килимі біля дзеркала лежала газета зі знімком відомого тенісиста. Поморочившись, Тім зачесав чуба майже так, як на знімку.

Якусь хвилину Тім роздивлявся сам себе в дзеркалі: спробував навіть скласти уста в якусь подобу усмішки, але нічого не вийшло. Тоді він сумно відвернувся і почав знічев'я ходити по кімнатах свого номера. Сів у крісло-гойдалку, гойднувся раз-другий і знуджено встав. Пороздивлявся картини на стінах - усе кораблі на бурхливих хвилях,- підняв трубку білого телефону, та зразу ж поклав її, а врешті розгорнув пишно оздоблений золотим тисненням шкіряний бювар, що його барон залишив посередині полірованого письмового столу. В бюварі був поштовий папір. На кожному аркуші в лівому.. .верхньому кутику стояло надруковане сірими літерами: [105]

Тім Талер,

власник акційного товариства «Барон Троч».

А в другому кутику:

Генуя, « » ... 19.. р.

В шовковій кишеньці збоку лежали конверти. Тім вийняв один і прочитав унизу зворотну адресу:

Тім Талер, Генуя, Італія, готель «Пальмаро».

Хлопець сів на стілець перед письмовим столом, І зняв ковпачок з авторучки, що лежала біля бювара, й вирішив написати листа.

Підсунувши до себе бювар і вийнявши з нього один аркуш, він побачив на полірованій поверхні стільниці дзеркальний відбиток напису. І Тімові впало в око одне слово:

ТРОЧ

«Читається неначе «чорт»,- подумав Тім.- Видно, як зайде мова про чорта, то він починає ввижатися скрізь. Чи хоча б його ім'я».

Хлопець поклав папір перед собою і почав писати:

«Любий пане Рікерт!

Я діставсь до Генуї не зовсім щасливо. Барон по-мер, а я тепер його спадкоємець. Але я цього зовсім не-хотів. Скоріш, навпаки, але цього я на жаль неможу вам пояснити. Може, колись потім. Будь ласка, спробуйте розшукати стюарда, його звати Крешимир і в нього апендицит. Крешимир може вам усе розказати, а я на жаль неможу. Поговоріть іще зі стерничим із Дельфіна, його звати Джонні й він із Гамбурга. Він знає як усе вийшло.

Тепер я найбагатша людина в світі, а так званий новий барон мій опікун. Гарного в тому нема нічого але може буде корисно. Баронові я і взнаки не дам що я всього цього нехочу. [106]

Ви, ваша матуся, стюард і Джонні були дуже добрі до мене.

Може ви найдете яку-небудь раду на мою біду. Але мабудь доведеться Ті мені самому шукати. Та воно мабудь і добре, бо я хоч матиму якийсь план і якусь мету й забуватиму за тим що я не такий як люди.

Вітайте будь ласка вашу матусю і вам щиро дякує ваш сумний Тім Талер.

5. Р. Але мені листів не пишіть. Може згодом я найду таємну адресу. Тім».

Хлопець перечитав листа, згорнув його, засунув у конверта й заклеїв. Та щойно хотів його заадресувати, як у коридорі почулася чиясь хода.

Тім швиденько сховав листа в нагрудну кишеню піджака. Ту ж мить у двері постукали, і, так само не дожидаючи запрошення, ввійшов барон.

Побачивши відкриту авторучку й розгорнений бювар, він спитав:

- Листи пишемо, пане Талер? З такими речами будьте обережні. А втім, до ваших послуг є секретар.

Тім згорнув бювар, накрутив ковпачок на авторучку й аж тоді відповів:

- Як мені потрібен буде секретар, я його покличу.

- О, голосок прорізується,- засміявся Троч.- Ви, я бачу, разом з одежею й манери змінили. Хвалю, хвалю!

У двері знов постукали. Барон досадливе гукнув:

- Че коза воле? (Чого вам треба?)

- Ля гардероба пер іль сіньйоре Талер! (Панові Талерові речі!) - відповіло з-за дверей.

- Аванті! (Заходьте!) - буркнув Троч.

Слуга в довгому зеленому фартусі вніс, улесливо зігнувшись, Тімову матроську торбу, поклав її на стояк для валіз, а сам став біля дверей.

Тім підійшов до нього, подав йому руку й сказав:

- Щиро вам дякую! [107]

Спантеличено, незграбно й видимо незадоволено слуга схопив Тімову руку й промимрив:

- Нон капіско...

- Не розумію, каже,- сміючись, переклав барон.- Зате оце ось він напевно зрозуміє,- і, витягши з кишені паку грошей, дав один папірець слузі.

Той засяяв, загукав: «Граціє! Міллє граціє, синьйоре бароне!» (Спасибі! Тисячу разів спасибі, пане бароне!») - і, вклоняючись, позадкував із номера.

Троч замкнув за ним двері й сказав:

- Колись, у давнину, як раб заходив до покою свого пана, то перше роззувався, тоді підповзав до пана навколішки і цілував пайового чобота. Як шкода, що ці благословенні часи минулися...

Тім не звернув ніякої уваги на баронові слова. Його раптом ніби окропом ошпарила думка: адже ж у торбі лежить його кашкет, а в кашкеті під підшивкою - угода з Трочем! Він ніби знічев'я підійшов до торби, розв'язав її, побачив кашкет із самого верху і взяв його в руку. Під пальцями зашурхотів папір, і хлопець полегшено зітхнув. Слухаючи, що каже далі барон, він по змозі непомітно витяг угоду з кашкета й сховав у кишеню.

- В такому готелі, як цей,- просторікав Троч, - досить подавати руку трьом: по-перше, старшому швейцарові, бо його треба часом просити, щоб він казав усім, ніби вас немає вдома; по-друге, директорові, щоб він ніде не розповідав про нас зайвого; і по-третє, головному кухареві, щоб він смачніше годував тих, із ким ми маємо справу.

- Запам'ятаю,- мовив Тім, а сам подумав: «Як я знову вмітиму сміятися, мені дуже приємно буде подавати руку слугам та покоївкам».

Задзвонив телефон. Хлопець узяв трубку й сказав у неї:

- Тім Талер слухає.

- Вашу машину подано, синьйоре! - промовило в телефоні. [108]

- Дуже дякую,- відповів Тім і поклав трубку.

Барон, що весь час стежив за хлопцем, сказав:

- Ніколи не озивайтесь у телефон своїм повним ім'ям, любий мій! Досить кинути: «Алло!» І то таким тоном, щоб зразу ясно стало: вас потурбували і ви невдоволені! І не кажіть «дуже дякую», коли вам доповідають, що машину подано. Досить буркнути: «Гаразд». Багатство зобов'язує до деякої неввічливості, пане Тале-ре. Людей треба тримати на відстані.

І знову Тім промовив: «Запам'ятаю»,- а собі знову подумав: «Стривай-но, поверну я собі свій сміх!»

Потім обидва спустились до прихожої, що в таких розкішних готелях називається вестибюль. Загледівши їх, із крісел попідводились якісь пани й низько вклонилися. Один підійшов і звернувся до Троча:

- Пане бароне, дозвольте...

Троч, не глянувши на нього, відрубав:

- Потім. Ми поспішаємо.

І вони з Трочем зійшли мармуровими сходами до великого автомобіля на шестеро дверцят.

Шофер повідчиняв перед ними дверцята, й вони сіли на м'які червоні сидіння.

Тім не завважив, що ззаду й спереду їдуть машини з охороною. Не зрозумів він і вигуків газетярів, що продавали на вулицях газети:

«Іль бароне Троч е морто! Адессо ун рагаццо ді кваттордічі анні е іль піу річчо уомо дель мондо!»

Барон, стримуючи усмішку, переклав йому ті слова:

- «Барон Троч помер! Тепер чотирнадцятирічний хлопчак - найбагатша людина в світі!»

Автомобіль зупинився перед світлофором. Троч саме навчав хлопця, як йому слід поводитись на тому банкеті, куди вони їхали. Але Тімову увагу цілком поглинула маленька смаглява дівчинка з чорними очицями-тернинками, що стояла біля ятки з овочами й, широко роззявивши рота, силкувалась угризнути здоровецьке яблуко. Помітивши Тімів погляд, вона відняла яблуко від рота й усміхнулась до хлопця. [109]

Тім кивнув їй і знову забув, що кожна спроба засміятись погано кінчається для нього. Дівчатко, побачивши, як страшно перекривилося обличчя за склом, перелякалось, заплакало й сховалось за спину продавця овочів.

Тім хапливо затулив долонями обличчя й відхилився на м'яку спинку. А барон, що спостерігав ту сценку в дзеркальце, опустив шибку біля себе й, сміючись, щось гукнув малій по-італійському.

Дівчинка з іще мокрими від сліз щоками знову визирнула з-за спини в продавця, приступила до авто й простягла баронові крізь вікно своє яблуко. Коли ба-рон дав їй за нього блискучу монетку, вона вся засяяла, писнула: «Граціє, синьйоре»,- й знову засміялася.

Ту хвилину автомобіль рушив з місця, й барон подав яблуко Тімові. Хлопець простяг був руку, та в останню мить відсмикнув її; велике яблуко, лиснюче, мов полаковане, скотилося з Тімових колін додолу й покотилось до шофера.

- Вам, пане Талер, треба навчитись замінювати усмішку грішми,- сказав Троч.- Здебільшого чайові діють краще, ніж приязнь.

«Навіщо ж ти тоді вимантачив у мене сміх?» - подумав собі Тім, а вголос промовив:

- Запам'ятаю, бароне!

Вісімнадцятий аркуш

У «ПАЛАЦЦО КАНДІДО»

«Палаццо Кандідо», як свідчить італійська назва, це білий палац: знадвору білий мармур, усередині білий тиньк.

Коли барон із Тімом зійшли на другий поверх сходами, що теж були з білого мармуру, їх знову обаранили директори -іцці та -оцці, що їх Тім невиразно пам'ятав іще з порту. Вони шанобливо мовчали, бо барон розмовляв із Тімом. [110]

Цей палац - музей,- півголосом сказав Троч,- і за користування Ним нам доведеться заплатити дуже дорого. У покоях палацу висять картини давніх італійських та голландських майстрів. Нам треба їх оглянути. Так годиться в нашому становищі. А оскільки ви, пане Талер, треба гадати, нічого не тямите в мистецтві, то я ; раджу вам розглядати картини мовчки, з поважною міною. Кожну картину, біля якої я кахикну, розглядайте трохи довше, і вдавайте, що вона вас дуже зацікавила. Тім поважно, безмовно кивнув головою.

Та коли вони, оточені роєм директорів, рушили оглядати картинну галерею. Тім не став дотримуватися Трочевих порад. Картини, що біля них Троч кахикав, він минав швидко, і навпаки, біля інших, де Троч не кахикав, він спинявся куди надовше.

У музеї були виставлені переважно портрети, зображення людських облич. На картинах голландських майстрів обличчя мали ледь прозору шкіру (крізь неї часом аж світилися сині жилки) і суворо стиснуті тонкі губи. Натомість на портретах роботи італійських малярів обличчя були смагляві, гладенькі, з ямочками в кутиках уст, що оживляли ті обличчя усміхом. Видно, голландські портрети були славніші, бо барон кахикав здебільшого біля них; але Тіма дужче надили інші, з відвертішими мінами й з усмішечкою в кутиках уст. Баронові доводилось часом аж підштовхувати хлопця далі, коли той прикипав до такої картини. Але директорам -іцці та -оцці Тімів смак видався непоганим. Коли Троч те помітив, він урвав огляд картинної галереї, сказавши:

- А зараз перейдімо до головної частини нашої зустрічі, панове!

Усі зайшли до великої зали, де стояли підковою розкішно понакривані столи. В чільному кінці було одне місце, прикрашене лавровим гіллям. Там мав сидіти Тім.

Та перше ніж повсідалися за столи, з'явився фото граф - миршавенький жвавий чоловічок із занадто довгим чорним чубом, що весь час ліз йому в очі, і фотограф [111] раз по раз владним рухом голови відкидав його назад. Він попросив усіх присутніх стати півколом по обидва боки Тіма. (До директорів тим часом прилучилося ще багато інших людей, однак тим людям спадкоємець уже не повинен був подавати руку.)

Фотограф пригвинтив апарат на штатив, подивився в видошукач і розставив усе товариство, як йому було треба, несамовито вимахуючи руками й весь час вигукуючи:

- Рідере! Соррідере! Соррідере, прего!

Тім, що стояв перед Грандіцці, спитав директора ; через плече:

- Що він каже?

- Він каже, чоб їй... пробачте, чоб ви... Він каже, чоб ми сміхалися!

- Спасибі,- подякував хлопець і дуже зблід. А фотограф звернувся вже просто до Тіма:

- Соррідере, синьйоре! Усміхніться, будь ласка!

Усі втупили очі в хлопця, що стояв, міцно стуливши губи.

Фотограф знову розпачливо гукнув:

- Усміхніться, прошу вас!

Барон, що стояв аж за Грандіцці, не озвався й словом, щоб вирятувати хлопця.

Тоді Тім сказав:

- Моя спадщина - це нелегка ноша, пане фотографе. Я ще не знаю, чи мені з усього цього сміятися, чи плакати. А тому дозвольте мені поки що не всміхатися.

По півколу, що оточувало його, перебіг шепіт. Хто перекладав сусідам його слова, хто говорив про хлопця захоплено чи зчудовано. Тільки баронові було, видно, д трошечки смішно.

Потім їх таки сфотографовано, хоч і без спадкоємцевої усмішки, і всі повсідалися за стіл. По боках у Тіма сиділи барон і Грандіцці. Із мереживної хусточки в директора й досі лилися пахощі гвоздики, що нагадували дух солодкого перцю. [112]

Перед їжею виголошено кілька промов, то по-італій ському, то каліченою німецькою мовою. І щоразу, коли ; лунав сміх та оплески, всі позирали на хлопця в чільному кінці столу. Одного разу барон шепнув йому:

- Нелегке життя наворожили ви собі своїм необміркованим закладом, пане Талер!

- Я знав, що на мене чекає,- шепнув Тім у відповідь. (Насправді ж йому ще ніколи й ніде не було так тяжко, як на чолі того святкового столу, де всі на нього витріщались, мов на якого заморського звіра. Але твердий намір не пасувати перед бароном додав йому духу.)

На одну коротку хвильку хлопцеві пригадався стерничий Джонні, й Тім ураз знову став малим хлопчиком, ладним от-от заплакати. На щастя, в ту мить підвівся й почав промовляти Троч, і Тім опанував себе.

Барон спершу довго вихваляв видатні здібності свого буцім-то померлого брата, далі заговорив про високі завдання людей, що мають віднині порядкувати таким великим багатством, і на закінчення в кількох словах побажав молодому спадкоємцеві сили й мудрості, щоб користатись такою незліченною спадщиною так, як треба. Потім додав кілька слів по-італійському. Видно, то був жарт, бо сам барон засміявсь, як мале хлоп'я. Всіх дам і панів за столом так зачарував той сміх, що й вони засміялись і заплескали в долоні.

А Тім лишився незворушний. Він якраз поглянув на годинника, що подарував йому в Гамбурзі пан Рікерт і що його він тепер весь час не знімав з руки. Було вісімнадцять годин тридцять хвилин, пів на сьому. О восьмій він хотів зустрітися з Джонні. А банкет, як судити з кількості страв та напоїв на столі, мав закінчитися ще не скоро. Отже, треба якось утекти раніше. А як те зробити, коли він сидить тут на чільному місці? І справді, бенкетували дуже довго. Коли після юшки та закусок подали головну страву - нирки в білому вині,- було вже двадцять хвилин на восьму. [113]

Тім, що весь час думав про Джонні, навіть не помітив, які складні застільні звичаї вищого світу. Він їв так, як їли пасажири у салоні пароплава «Дельфін», і барон раз у раз чудувався з невимушеного й гарного хлопцевого поводження. Дивлячись, як той делікатно настромлює виделкою шматочок нирки, він промурмотів сам до себе:

- А я цього хлопця недооцінив...

За двадцять хвилин до восьмої Тім нахиливсь до барона й шепнув:

- Мені треба до...

Не давши йому вимовити незручного слова, Троч відказав:

- Туалети в коридорі за дверима праворуч.

- Спасибі,- подякував Тім, підвівся й поміж столами пішов до дверей праворуч. Його проводжало щонайменше сто пар очей, але він силкувався йти так, як ходить звичайний чотирнадцятирічний підліток.

За дверима йому раптом закортіло вигукнути одне дуже непристойне слово. Але там стояв слуга в гаптованій золотом лівреї, й Тім спокійно, незворушно зайшов до туалету, де вимовив те слово перед дзеркалом аж тричі, повільно й виразно.

Коли він вийшов у коридор, слуга саме відвернувся. І Тім навшпиньки (бо мармурова підлога дуже лунка) подибав до сходів, а тоді чкурнув ними вниз.

Перед палацовими дверима стояв швейцар у мундирі з золотими галунами. Але він, видно, не знав хлопця, бо тільки байдуже подививсь на нього спідлоба. Тім набрався зухвальства й спитав його, як знайти пам'ятник Христофорові Колумбові. Але швейцар нічого не зрозумів. Він тільки безпорадно розвів руками й показав трамвайну зупинку. Туди й подався хлопець. [114]

Дев'ятнадцятий аркуш

ДЖОННІ

Та хвилина, що Тім простояв на зупинці, чекаючи трамвая, видалась йому нескінченною. Тім кілька разів озирався через плече на сходи палацу, але, опріч швейцара, там не показувався ніхто. Певно, його довга відсутність іще нікого не стривожила.

Хлопець почав нетерпляче розглядати карту міського транспорту на зупинці. Посередині тієї карти було вправлено довгасту смужечку дзеркала. І раптом - уже втретє за той день - відзеркалення розкрило йому баронову натуру. Бо в дзеркалі він побачив, що за рогом палацу стоїть та машина, якою вони з Трочем приїхали туди. А за тим автомобілем стояло ще дві легкові машини: біля передньої розмовляло двоє чоловіків, і один із них саме показував пальцем на Тіма.

Хлопцеві враз пригадалися слова директора Грандіцці на катері про детективів, що мають весь час його охороняти. Певне, ото й є ті таємні охоронці. Кепська справа, бо ж барон не повинен дізнатися, що Тім зустрічався з Джонні.

Аж ось над'їхав трамвай. До попереднього вагона було причеплено дві так звані платформи, відкриті на обидва боки.

Ті відкриті платформи прийшлися дуже до речі Тімові. Відколи хлопець утратив свій сміх, він навчився спокійно й холоднокровне обмірковувати будь-яке складне становище. І він зразу зметикував, що треба зробити. Він сів на середню платформу, пропхався поміж людьми, що стояли там, і перше ніж трамвай рушив, зійшов з другого боку. Просто перед колесами в спортивної машини, що мчала вулицею, Тім вискочив на протилежний тротуар.

Звертаючи у вузенький завулочок, він іще раз квапливо озирнувсь і побачив, як один із детективів теж наміряється перебігти вулицю. Тім зразу зміркував, що йому потрібна не так швидкість, як хитрість, щоб [115] утекти від своїх наглядачів. На щастя, він саме попав у район старих заплутаних завулків, де мало не кожен дім мав виходи на обидва боки. Хлопець спокійно зайшов до якоїсь невеличкої харчівеньки, де пахло смаженою рибою та маслиновою олією, вийшов із неї другими дверима, потрапив у інший завулок, де просто під вікнами продавали смажених каракатиць, шмигнув у якісь двері під вивіскою «Тратторія», перейшов через ту тратторію, цебто трактир, вискочив у ювелірську вуличку, де в вітринах блищали цілі гори оздоб, трохи пробіг попід вікнами, звернув у вузесенький провулочок на другому боці, опинився на малесенькому риночку серед галасливих, балакучих жінок, перебіг іще через одну тратторію, де кисло пахло вином, і раптом тицьнувся просто у відчинені двері автобуса на зупинці. Швиденько вскочив у автобус, двері за ним зачинились, і машина рушила.

Кондуктор, усміхаючись, посваривсь на нього пальцем і простяг руку по гроші за квиток. Тім, що досі про гроші й не подумав, мимохіть хапнувся за кишеню й відчув з полегкістю, що там є й папірці, й монети. ' Він дав кондукторові одного папірця й сказав:

- Христофор Колумб.

- Га? - перепитав кондуктор.

- Христофор Колумб! Пам'ятник! - іще раз проказав хлопець, намагаючись вимовляти слова якнайвиразніше.

Тепер кондуктор зрозумів.

- Іль монументе ді Крістофоро Коломбо! - повчальним тоном поправив він Тіма. І хлопець слухняно проказав за ним:

- Іль монументо ді Крістофоро Коломбо!

- Бене! Бене! - засміявся кондуктор.- Доббрре, доббрре! - Потім віддав Тімові вісімдесят п'ять лір решти, відірвав йому квитка й на мигах пояснив, що скаже, де зійти.

Тім із поважним обличчям кивнув головою й подумав: [116]

«От пощастило!» Радіти з того він не міг, але принаймні на серці йому полегшало.

За десять хвилин - автобус тим часом проїхав понад портом, а тоді звернув у завулок, що круто піднімався вгору,- кондуктор плеснув Тіма по плечі й показав на обсаджений пальмами великий білий пам'ятник, що стояв перед величезним будинком із багатьма скляними дверима. Хлопець вимовив єдине італійське слово, що знав: «Граціє! Дякую!» - вийшов з автобуса й розгублено спинився серед широкого майдану. Він уже розглядів, що велика будівля - то вокзал. Годинник над головним входом показував п'ять хвилин до восьмої.

Серед людей на майдані Тім не побачив жодного з двох охоронців. Але й стерничого Джонні він теж, на жаль, не побачив. Тому хлопець навмисне повільно рушив до пам'ятника, обійшов кругом нього - і наткнувся на стерничого, що стояв під пальмою. Тім просто не міг не помітити такого велетня. Він підбіг до Джонні й радий був би його обняти, якби тільки той не був такий височенний.

- Я втік, Джонні! - зовсім засапаний, сказав Тім.-

Барон приставив до мене детективів. Але...

- Барон? - гостро перебив його стерничий.- А я

думав, що він помер!

- Ні, він тільки перемінився в свого буцімто брата-близнюка.

Джонні стиснув зуби. Потім узяв хлопця за руку.

- Ходімо сядьмо тут в одному шиночку. Там він нас так швидко не знайде.- І повів хлопця завулками.

Те, що Джонні назвав шиночком, заслуговувало, по правді, на кращу назву. Довге, вузьке приміщення на кінці розширювалось у напівтемну, майже квадратну залу. Підлога там була зі струганих дощок, а всі стіни аж до стелі вкриті полицями з темного дерева, де стояла сила пляшок усякої форми й кольору. Вигляд у тієї зали був майже урочистий: якийсь немовби собор із пляшок. [117]

Джонні підвів Тіма до одного вільного столика в кутку зали. Там їх не видно було від дверей. Коли підійшов офіціант, Джонні замовив дві склянки червоного вина. Потім витяг із обох внутрішніх кишень своєї куртки по пляшці рому, поставив їх під Тімів стілець і пояснив:

- Це твій виграш. Я їх сховав, щоб офіціант не подумав, ніби ми тут жлуктимо принесене з собою.

А Тім вийняв із кишені свого листа до пана Рікерта.

- Відвези цього листа до Гамбурга, Джонні. А то я боюся посилати його поштою.

- Давай сюди! - Джонні сховав листа в кишеню. ' Потім сказав: - А ти став справжній панич, Тіме. Що, приємно багатому?

- Та трохи морочливе,- відповів Тім.- Зате можна поводитись, як хочеш. Не треба сміятися, коли не хочеться,- хіба що перед фотографом. А це вже немала перевага.

-А ти що, не любиш, коли сміються! - спантеличено запитав Джонні.

Тім помітив, що ляпнув зайве. Адже він нікому не мав права казати, що продав свій сміх. Та перше ніж він придумав якусь відмовку, Джонні вже загомонів знову. Видно, коли мовилося про сміх, тоді стерничий почував себе в своїй стихії, бо й заговорив він раптом жвавіше й навіть красномовніше.

- Воно-то правда, що від сміху нещирого, тільки задля годиться, часом із душі верне. Я не знаю нічого гидшого, ніж матроський нічліжний дім, де з ранку до вечора усміхаються до тебе підстаркуваті тітоньки. Усміхаються, вмовляючи не пити горілки; усміхаються, накладаючи тобі на тарілку кислої капусти; усміхаються, нагадуючи, щоб ти помолився; усміхаються навіть тоді, як тобі вирізають апендицит. Все усміхаються, усміхаються, усміхаються, і вранці, й удень, і ввечері, й уночі... Такого справді не можна витерпіти. Але...

Надійшов офіціант з вином і теж усміхнувся до них завченою офіціантською усмішкою. Тім понурив очі [118] в стіл, губи його сіпалися. Джонні зачудовано помітив, що хлопець ось-ось заплаче. Тому він не став говорити далі, коли офіціант пішов, а тільки підняв свою склянку й промовив:

- Ну, будьмо здорові, Тіме! За твоє щастя!

- Будьмо, Джонні!

Тім тільки пригубив кисленького вина.

Ставлячи склянку на стіл, Джонні пробурмотів сам до себе:

- Якби лиш мені докопатися, в чому річ!

Тім добре розчув ті слова. Раптом пожвавішавши, він пошепки сказав стерничому:

- Розшукай Крешимира й поговори з ним. Він усе знає й може тобі все розповісти. А я не можу. Не маю права.

Стерничий хвильку подививсь на нього замислено, тоді сказав:

- Я, здається, вже знаю, кому ти попав у лабети.- Потім нахиливсь до хлопця й спитав настійної-Він тебе заморочив якимись фокусами?

- Та ні - відповів Тім. - Нічим він мене не заморочив, тільки проказав одне якесь стародавнє закляття - І хлопець розповів стерничому про розмову в готельних покоях, про химерне заклинання та про розби-^І^орТз люстрою страшенно насмішила стерничого.

Він реготав на все горло, ляскав долонею по столу аж склянки підскакували й схлюпувалося вино, та вигукував, захлинаючись:

- Ой, умерти можна! Оце-то так! Та ти знаєш що влучив того блазня в найдошкульніше місце, Тіме, їй же богу, правда! - Джонні відхиливсь на спинку стільця - Таж ти нічого кращого й придумати не міг, ніж розтрощити ту люстру. Це ж йому як віжки під хвіст' Та ще в таку хвилину!

Стерничий, весело ошкірившись, підняв угору обидві руки - точнісінько як Троч, коли той вимовляв закляття - й виголосив удавано поважно: [119]

Володарю щурів, мишей,

Жаб, мух, ґедзів, блощиць, вошей!

Для Тіма то була велика втіха - чути, як сміються й глузують із барона. Уперше за довгий-довгий час йому подобався чужий сміх.

Слухаючи глузливе заклинання. Тім опустив погляд. І раптом побачив на дощаній підлозі здоровезного, гладкого щура. Пронизливо повискуючи, щур безстрашно біг до ноги Джонні, наче наміряючись укусити стерничого.

Тім, що страшенно гидував щурами, скрикнув:

- Джонні, щур!

Та Джонні й сам уже побачив потвору. Неймовірно спокійно й швидко він підібгав ту ногу, до якої біг і щур, а другу блискавично підняв і, тупнувши нею щосили, роздушив тварюці голову. На підлозі лишився труп, такий огидний, що Тіма аж занудило, і він швиденько відвернувся.

Та Джонні, нездоланний Джонні, мовив осміхаючись:

- Хазяїн присилає гінців... Ковтни вина, Тіме, й не дивись туди.

Цього разу Тім хлиснув добрий ковток, і нудота майже зразу минулася. Зате голова пішла обер том, наче в ній повільно-повільно закрутилось якесь колесо.

Джонні провадив далі:

- Ну, часу в нас небагато, Тіме, Скоро й самого хазяїна принесе. Ти тільки одне знай: у що ти не віриш, і того нема на світі! Розумієш мене?

Тім безтямно похитав головою: колесо в ній крути лося дедалі швидше.

- Я хотів сказати ось що: розбивай люстру щоразу, як барон почне в'язнути до тебе зі своїми фокусами. Второпав? - пояснив Джонні.

Тім кивнув головою, хоча лише наполовину зрозумів слова стерничого.

Повіки хлопця обважніли, бо йому ще в «Палаццо [120] Кандідр» довелося випити вина, а він не був до цього звичний.

- Висміюй того блазня при кожній нагоді,- вів далі Джонні.- Зовнішню волю ти можеш купити собі за гроші, їх ти тепер маєш досить. Але внутрішню волю, хлопче, можна придбати тільки за один капітал: за сміх. Є таке старовинне англійське прислів'я...

Стерничий наморщив лоба й пробурчав:

- От чудасія! Щойно я те прислів'я знав, а тепер вилетіло з голови. Але на язиці крутиться... Це, мабуть, вино мене запаморочило.

- А мене й вино не бере,- неповоротким язиком вимовив Тім. Та Джонні мовби й не чув його. Він усе згадував те прислів'я. Нарешті вигукнув:

- Знаю, знаю! Згадав! Тич мі лафте, сейв май соул!

Як це воно мені зразу не згадалося! - Він сам засміявся зі своєї забутливості, ляснув себе по лобі й несподівано, ще сміючися, сповз із стільця додолу й простягся там, непорушний та блідий, поряд із роздушеним щуром. Тім, ураз протверезівши, підхопивсь і злякано озирнувся, кого б покликати на поміч. Він побачив офіціанта, що байдуже дивився в їхній бік. А коло офіціанта стояв якийсь пан і давав йому гроші. Хоч той пан стояв до Тіма спиною, проте хлопець відразу впізнав його. То був барон. До Тіма вмить повернулося те напруження, що в ньому він чинив і говорив зовсім не так, як підказувала йому його власна вдача. Зовні спокійний. Тім підкликав офіціанта, кивнувши йому рукою, а сам став на коліна біля Джонні. Той крізь свою непритомність іще раз вимовив повільно, важко, але дуже виразно англійське прислів'я:

- Тич, мі лафте, сейв май соул.

І в ту саму хвилину Тім побачив над собою офіціанта, а за спиною в нього - барона.

- Ви тут, пане Талер! Що за несподіванка! - вигукнув Троч нібито дуже здивовано.- А ми вас уже цілу годину шукаємо! [121]

Пустивши баронові слова повз вуха, Тім відказав:

- Коли зі стерничим сталося щось лихе, я на вас заявлю в поліцію, бароне! І на офіціанта теж!

Троч ошкірився:

- Не хвилюйтеся! Він буде живий і здоровий. Правда, з роботи в нашій фірмі нам доведеться його звільнити. Але такий дужий чолов'яга легко знайде собі роботу в порту. Відвідувачі шинку тим часом зацікавлено стовпились біля столу й один з-перед одного давали всілякі поради, видно, маючи Джонні за п'яного.

Троч, що не любив привертати до себе уваги, взяв Тіма за рукав і потяг геть.

- Ваше фото, пане Талер, сьогодні надрукували в усіх газетах. Було б дуже неприємно, якби вас тут упізнали. За стерничого можете не турбуватися, слово честі! Ходімо!

Хоч Тімові й не хотілося кидати Джонні непритомного, та все ж він дав баронові вивести себе з шинку. Нехай Троч не знає, що він насправді думає. А крім того, хлопцеві чомусь здавалося, що в цій заплутаній історії з убитим щуром, непритомним стерничим та англійським прислів'ям переможець не барон, а Джонні.

Не тільки зовні, а й усередині спокійніший, ніж можна було гадати, Тім вийшов із зали, обставленої по всіх стінах пляшками.

Автомобіль на шестеро дверцят, що стояв надворі, майже перегородив собою завулочок. За ним стояли ще дві машини, а в них Тім розгледів двох уже знайомих йому добродіїв.

Хлопця раптом потягло на пустощі, й він чемно кивнув їм головою. Вони теж кивнули у відповідь - трохи спантеличено.

У великій машині сидів іззаду директор Грандіцці. Коли Тім із бароном сіли поруч нього, він вдоволено захихотів: [122]

- А, маленьки втікач! Ви нас дуже обкрутили круг палець, сіньйоре! Але мій друг Астарот...

- Припни язика, Бегемоте! На цей гачок він не клює,- раптом голосно, брутально гримнув на директора барон. Але зразу ж обернувся до Тіма й приязно пояснив хлопцеві, що вони з Грандіцці - члени так званого Пекельного клубу й часом жартома величають один одного клубними прізвиськами.

Тімові здавалося, що він уже колись чув від барона ті назвиська - Астарот та Бегемот,- але він не міг пригадати, де й коли.

Та й не дуже силкувався пригадати, бо весь час подумки твердив те англійське прислів'я, що сказав йому Джонні.

Коли автомобіль їхав повз пам'ятник Христофорові Колумбові, Троч звернувся до Тіма:

- Завтра ми відлітаємо до Афін, пане Талер. Своїм власним літаком. О восьмій годині ранку нам його подадуть.

Тім мовчки кивнув головою. Він, мабуть, уже вдесяте проказував подумки англійське прислів'я. Кінець кінцем він спитав Грандіцці:

- Що це означає: тичмілафта сейфмайсол?

- А яка це мова? - перепитав Грандіцці.

- Англійська,- спокійно відповів барон.- Це старовинне прислів'я, безглузде, як і більшість прислів'їв.- Він промовив те прислів'я чистою англійською мовою: - «Тич мі лафте, сейв май соул».- Тоді півголосом переклав: - «Навчи мене сміятися, порятуй мою душу».

Тім протяг якомога байдужіше: «А-а-а...» - і більше нічого.

Але в думці він міцно затвердив ті слова: «Навчи мене сміятися, порятуй мою душу...» Ще й додав від себе: «...стерничий!» [123]

Двадцятий аркуш

ВІДКРИТТЯ В АФІНАХ

У Афінах, стародавній грецькій столиці, містилась найбільша філія акційного товариства барона Троча. Тому, певно, барон був там такий незвичайно жвавий та люб'язний. У Афінах він по змозі не надокучав Тімові з банкетами та з директорами, а гуляв із хлопцем вулицями пішки. Правда, на пристойній відстані за ними їхала машина, що могла на Трочів знак будь-коли під'їхати до тротуару й забрати їх.

Барон не водив Тіма в ті місця, задля яких звичайно приїздять до Афін чужоземці. Він не сходив із ним на Акрополь, звідки поміж колонами храмів видніє весела, осяйна синь Егейського моря; не підводив його до мармурових статуй, від кісточок на ногах до ямочок у кутиках уст, сповнених небесного сміху; не показував йому, як ясно сяє небо над білими храмами. Натомість він повів Тіма на афінський ринок.

- Із тих грошей, що заробляються тут,- сказав Троч,- понад половину переходить через мої руки. Як моєму спадкоємцеві, вам, пане Талер, треба знати, де робиться наше багатство. Хіба ж не втіха для очей - усі оці барви?..

Насамперед барон повів Тіма в рибні ряди. Окаті, з яскраво-червоними зябрами, тисячами лежали рибини в великих відкритих льодовнях. Морське багатство, викладене напоказ, аж очі вбирало. Блищало сріблом, мінилося синьою крицею, жаріло червоними цятками, темніло матово-чорними смугами. Та барон дививсь на все те очима крамаря.

- Тунців нам привозять турки,-- пояснив він Тімові.- Майже задарма купуємо. Тріска - з Ісландії. З неї ми маємо найкращий зиск. Барабульки, каракатиць та сардини постачають нам італійці й тутешні грецькі рибалки. На них багато не заробиш. Але ходімо далі. пане Талер, ходімо далі!

Троч ніби сп'янів на тому ринку. [124]

Вони підійшли до тинькованої стіни, де були розвішені оббіловані вівці.

- Це вівці з Венесуели,- сказав барон.- А отих-о свиней куплено в Югославії. Дуже вигідно.

- А в Греції, крім риби, є й ще щось? - спитав Тім.

- Аякже,- засміявся Троч.- Дещо є. Родзинки, вино, банани, печиво, маслинова олія, гранати, вовна, сукна, інжир, горіхи, баклажани, боксит.

Троч виголосив той перелік урочисто, немовби родовід царя Давида з біблії. Тим часом вони з Тімом уже опинились у сирному ряду, де на столах лежали гори білого сиру. Весь час Тім із Трочем мусили проштовхуватись поміж натовпом галасливих продавців та покупців.

У рибних рядах вони чалапали по калюжах, де плавали кільця покришеної цибулі; біля овець обходили розлиту кров, а там, де продавали овочі, земля була суспіль усіяна лушпинням.

Перед Тімом пробігло троє хлопчаків. Ось вони привселюдно поцупили з діжки по жмені маринованих маслин.

Нікого те не вразило, навіть продавців, що тільки гиркнули на хлопців злісно й зразу ж знов пооберталися до своїх клієнтів. Малі злодюжки реготали.

Проблукавши добрих дві години по цьому ринкові-кошмарі, цьому величезному череву міста-ненажери, серед цього закликання, галасу, штовханини, що так тішила барона, Тім вийшов звідти зморений, очманілий.

На баронів знак під'їхала машина - цього разу менша, всього на четверо дверцят, і з чорними сидіннями.

Троч звелів шоферові їхати до візантійського музею. Тімові він сказав:

- Вам там сподобається, пане Талер. Але я не скажу вам наперед чому.

Тіма те «чому» не дуже й цікавило. Він був страшенно стомлений і голодний. Але ні слова про те не сказав. [125]

Він не хотів виявляти своєї слабкості перед цим химерним гендлярем, що відкупив у нього сміх. Тому він набравсь терпіння й дав відвезти себе до візантійського музею.

Картини, що до них барон підвів Тіма, були так звані ікони.

Малювали їх, як пояснив йому Троч, здебільшого ченці, що сотні років додержувались у своєму мистецтві все тих самих суворих правил.

Тім дуже скоро здогадався, чого це барон привіз його сюди.

Намальовані на іконах обличчя з великими застиглими очима й довгими носами, що ділили овали тих облич на дві рівні половинки, не всміхалися. Цим вони були подібні до блідих голландських облич у «Палаццо Кандідо» в Генуї. Тімові вони здавалися жахливими. Коли Троч на цілу хвилину затримав його перед іконою святого Георгія, де був намальований вершник у криваво-червоному плащі на тлі похмурого оливково-зеленого гірського краєвиду, Тім пробурмотів собі під ніс прислів'я Джонні: «Навчи мене сміятись, порятуй мою душу!»

І диво дивне: згадавши Джонні, Тім раптом став дивитись на ікони зовсім іншими очима. Враз він помітив, що малярі-ченці, не дозволяючи людям на своїх іконах сміятися, жити, цвісти, все те дозволяли тваринам і рослинам. Поки Троч просторікав про священні правила богомазів, Тім познаходив на іконах сміхотливих собачок, грифонів, що підморгували очима глядачеві, веселих пташок та усміхнені лілеї. І знов йому пригадався один вислів - той, що він чув у ляльковому театрі в Гамбурзі: «Сміх людей від звірів відрізняє». Тільки тут сміялися якраз звірі, а люди суворо, безжаль но вдивлялися у світ, що не знав щастя.

На другому поверсі Троч зупинився поговорити з директором музею: того вже повідомили про відвідини багатія-барона. І Тім відчиненими дверима вийшов на невеличкий балкончик. [126]

Звідти він побачив на подвір'ячку внизу маленьку ' дівчинку, що креслила паличкою якісь лінії на втоптаній землі, а потім викладала їх різнобарвними камінчиками.

Видно, вона щойно була в музеї, в тому залі, де виставлено мозаїки, і тепер теж по-своєму робила мозаїку. Дівчинка викладала з камінчиків обличчя, схоже на ікону, але на місці рота було півколо, вигнуте догори: обличчя сміялося.

Та якраз коли дівчинка неквапливо вставляла в те обличчя зелене око, надійшов якийсь хлопець, глянув на майже готову картину, кисло скривився й човгнув по ній ногою.

Обличчя було знівечене.

Дівчинка злякано глянула на руйнівника, і з очей їй враз покотилися рясні сльози. Потім вона, хлипаючи, покірливо заходилася збирати свої камінці.

А хлопець стояв поруч, сховавши руки в кишені, з чоловічою зневагою в погляді. Тім розлютився й хотів був уже збігти вниз, заступитися за дівчинку, та, рвучко обернувшись, побачив перед собою барона, що, видно, теж спостерігав усю ту сценку.

- Не втручайтеся, пане Талер,- усміхнувшись, сказав Троч.-Звісно, цей хлопець учинив дуже негарно. Але так воно ведеться в світі. Так само по-варварському, як отой хлопець, важкі солдатські чоботи розтоптують довершені витвори мистецького генія; та коли кінчається війна, ті самі варвари, скривившись кисло, підписують кредити на відбудову зруйнованого. А ми на тому заробляємо. Наша фірма, наприклад, після війни реставрувала в Македонії понад тридцять церков, і чистий наш прибуток із того становить понад мільйон драхм.

Тім промимрив завчену фразу:

- Я запам'ятаю, бароне. Але зараз,- додав він,- мені хотілось би пообідати.

- Знаменита думка! - засміявся Троч.-Я знаю тут один чудовий літній ресторанчик. [127]

Навіть не глянувши більше ні на картини по стінах, ні на дітей на подвір'ячку, барон попрямував до своєї машини, що стояла перед брамою музею. Тім мовчки тюпав поруч нього.

В літньому ресторані - на Тімове здивування, він виявився зовсім не такий розкішний, як любив барон звичайно,- їх, низько кланяючись, зустріли власник, директор і старший офіціант. Барон говорив із ними по-грецькому, проте до Тіма він звертався по-німецько-му. їх провели до стола в куточку, постелили задля них сніжно-білу скатерку, поставили квіти й принесли з будинку ще один невеличкий столик для закусок. Усі відвідувачі ресторану уважно, напружено стежили за тими готуваннями.

Деякі перешіптувалися, крадькома позираючи на Тіма.

- Що, моє фото й тут надруковане в газетах? - пошепки спитав хлопець.

- Аякже! - голосно відказав барон.- У Греції, пане Талер, нічого так не шанують, як багатство, бо це бідна країна. Для таких, як ми, Греція - рай. Навіть в оцьому абиякому ресторанчику нас нагодують таким обідом, що хоч би й королю такий подати, то не сором. Тут багатству віддають царську шану. Ось чому я й люблю так Грецію.

Троч, мабуть, іще довго просторікував би, дратуючи Тіма, якби не прийшов офіціант і не шепнув йому щось на вухо.

- Мене кличуть до телефону. Вже прознали мій улюблений ресторан,- сказав барон Тімові.- Вибачте.- Він підвівся й пішов за офіціантом до будинку.

Хлопець, лишившись сам, задивився на один стіл навпроти їхнього - єдиний стіл, що за ним ніхто не витріщався на Тіма.

Його цікавили там дві родини.

Перша сиділа за столом - огрядненька чорнява мама з мушкою на щоці й дві донечки: одна років п'яти, друга років двох. [128]

Друга родина - велика сіра кицька й троє кошенят. двоє чорних і одне сіре - вовтузилась під олеандровим кущем біля столу.

Обидві мами - і мама-грекиня, і мама-кицька - були дуже сердиті. Коли менша донечка-грекиня залізла на клумбу, вимазалась у землю й почала рвати та пхати собі в ротик листя, мама з мушкою схопилась і наклепала її долонею по щоках, по губах, по носі. Мале ревло, аж заходилось, а пухка мамина долоня все ляскала по мокрому від сліз личку.

І в мами-кицьки вдача була така самісінька. Щоразу, як котре-небудь із кошенят підскакувало до неї чи стрибало їй на хвіст, вона сердито сичала. Найдужче в'їлась вона на одне чорненьке кошенятко. Коли воно жалібно нявкнуло, кицька щосили вдарила його лапою, хоча й не випустивши пазурів - так би мовити, долонею. Маля все ж знову підскочило до неї - вона вдарила його ще раз, і котяче нявчання злилося з дитячим верещанням.

Тім відвернув погляд - не міг довше на таке дивитися. І якраз ту хвилину повернувся барон. Він, видно, й цього разу бачив усю ту сценку й розгадав хлопцеві думки, бо, сідаючи, сказав:

- От бачите, пане Талер, між людьми й звірами зовсім невелика різниця. Майже невідчутна, можна сказати.

- Це я вже третю думку чую про ту різницю,- трохи спантеличено мовив Тім: - У одному гамбурзькому театрі я чув такі слова: «Сміх людей від звірів відрізняє»,- цебто тільки людина вміє сміятися, а звірі - ні. На картинах у музеї було навпаки, там сміялися тільки звірі, але не люди. А ви, бароне, кажете мені, що між людьми й звірами взагалі нема ніякої різниці.

- Ніщо в світі не буває таке просте, щоб його можна було з'ясувати одною фразою,- відповів Троч.- [129]

А що означає для людини сміх, цього, любий мій пане Талер, ніхто достоту не знає.

Тім згадав нараз одне зауваження стерничого Джонні й вимовив ті слова скоріш сам до себе. але досить голосно, так що почув і барон:

- Сміх - це внутрішня воля.

Ті слова дивно вплинули на барона; він тупнув ногою й закричав:

- Це тобі стерничий сказав!

Тім зачудовано закліпав очима. І раптом цей чотирнадцятирічний підліток, іще напівдитина, збагнув, навіщо барон відкупив у нього сміх і чому нинішній барон Троч так відрізняється від похмурого картатого пана з іподрому. Він став вільною людиною. А розлютився І тому, що Тім це відкрив.

А втім, барон, як і завжди, зразу ж опанував себе. Вже знову спокійний і чемний, як завжди, він змінив ; тему:

- Ви знаєте, пане Талер, наше становище на масляному ринку стало загрозливе. Мені треба не пізніше як завтра зустрітися з іншими керівниками нашої фірми й порадитися, що робити. На такі наради ми звичайно збираємося до мого месопотамського замку. Може, й ви поїдете зі мною туди? Те, що вам треба ще побачити в Афінах, я покажу вам коли-небудь згодом.

- Як хочете.- відказав Тім немовбито байдуже. Насправді ж він нічого так не хотів, як побачити те таємниче місце, де здебільшого, мов павук усередині своїх тенет, сидів барон.

Але Трочеві не хотілось залишати Афіни. Коли їм подали їжу, він зітхнув:

- Останній цього разу обід у цій благословенній країні... Ну, смачного! [130]

КНИГА ТРЕТЯ

МАНІВЦІ

Сміх - то не крам на продаж, як,

скажімо, маргарин.

Хто його продає, чинить необачно.

Салех-бей

Двадцять перший аркуш

ЗАМОК У МЕСОПОТАМІЇ

Уже вдруге Тім летів у невеликому двомоторному літаку, що належав фірмі барона Троча. Вилетіли вони вдосвіта і хлопець насилу розрізняв зі свого вікна море й небо. Та враз він побачив за невеличким темним горбком якогось острівця вогненне коло. Сонце зійшло так швидко, немов випірнуло з моря.

- Ми летимо на схід. назустріч сонцю,-пояснив Троч.- У Афінах воно ще не зійшло. Слуги в моєму замку моляться на сонце. Вони називають його Еш-Шемс.

- А я гадав, ваші слуги моляться чортові,- озвався Тім.[131]

- Авжеж, вони шанують шайтана, як володаря землі, але не як володаря неба.

Хлопець хотів був промовити: «А...» - та вчасно згадав, що вже раз розсердив барона тим байдужим слівцем. Тому він не сказав нічого. Мовчки дивився він униз, на море, що його олив'яно-сіра барва навдивовижу швидко яснішала, поки стала зелена, мов скло.

Тім не боявся, летячи в літаку, але й не радів із польоту. Навіть не дивувався, бо хто не вміє сміятись. той не вміє й дивуватися.

Барон почав пояснювати йому «становище на масляному ринку». Тімові про те становище було байдужісінько: та все ж він зрозумів, що їхня фірма пересварилася з кількома великими молочарськими компаніями і що якась інша фірма продає в Норвегії, Швеції, Данії, Німеччині та Голландії краще й дешевше масло, ніж їхня. От через те й летіли вони до месопотамського замку. Там барон хотів «з'ясувати ситуацію» та «обміркувати заходи». Двоє Трочевих компаньйонів також мали прилетіти до замку: один, містер Пенні,- з Лондона, а другий, сеньйор ван дер Толен.- з Лісабона.

Літак уже давно летів над одноманітними голими нагір'ями Малої Азії, а барон усе патякав про гатунки масла та ціни на нього. У мові його рясніли такі вислови, як «фронт збуту», «армія споживачів», «наступальна рекламна кампанія», неначе він був генерал, що готується до великої битви.

Задля годиться, аби щось сказати. Тім мовив, коли барон на хвилинку замовк:

- А в нас удома ніколи не бувало масла, самий маргарин.

- На маргарині капіталів не наживеш,- буркнув барон.- Бутерброди з маргарином - гидота.

- Але ж маргарин іде не тільки на бутерброд! - зауважив Тім.- У нас його і в печиво клали, й у страву, й смажили на ньому. [132]

Аж тоді барон зацікавився:

- То в вас, виходить, маргарин правив і за смалець, і за олію, й за масло, еге?

Тім кивнув головою:

- По-моєму, в самому нашому завулку за день споживалось не менш як півцентнера маргарину.

- Цікаво,- промимрив Троч.- Дуже цікаво, пане Талер! Обхідний маневр із маргарином і виграш простору на масляному ринку. Майже геніально! Але як це зробити?

Барон поринув у задуму, аж згорбившись на своєму сидінні. Тімові це було до речі, бо він саме задивився вниз. Там у зморшках-долинах гірської країни тяглись із різних кінців каравани віслюків, що прямували до одного місця - певне, до селища, де був базарний день. Пілот вів літак навмисне, задля хлопця, зовсім низько, так що Тімові досить виразно видно було погоничів тих віслюків. Звісно, обличчя видавались йому просто світлими цятками, з вусами чи без, отож він міг судити про людей тільки з їхнього одягу, а той одяг був, як на його погляд, такий чудернацький, що люди нагадували йому скоріше дивовижних заморських звірів із звіринця. Звичайно, це була нісенітниця: якби ті люди були поголені, підстрижені та вбрані, скажімо. так. як убиралися люди в Тімовому рідному місті, то хлопець і не вбачив би в них нічого незвичайного, крім хіба трохи темнішого кольору облич. Та чи можна вимагати від чотирнадцятирічного підлітка, несподівано завезеного в чужий край, щоб він зразу ж склав собі правдиву думку про ніколи не бачений народ? А втім. на прикладі Салех-бея Тім мав дуже швидко переконатися, що не слід надто поквапливо судити про нових знайомих та інші народи.

Той Салех-бей якраз виїхав верхи на коні з маслинового тйга, коли літак приземлився на рівній, як стіл, гірській долині й Тім перший вийшов із нього. Троч привітав вершника по-арабському, надзвичайно чемно. Вклоняючись, барон шепнув хлопцеві: [133]

- Це великий комерсант і ватаг єзидів. Він учився в вашому рідному місті. Зараз він заговорить до вас по-німецькому. Поводьтесь із ним шанобливо, вклоняйтеся низько.

А Салех-бей уже обернувся до хлопця, і той добре-таки зніяковів. На цьому бородатому дідові було таке вбрання, що Тім лиш помалу розрізнив його окремі частини: сорочку, куцинку, каптан, іще один каптан - верхній строкатий широкий пояс, обмотаний круг живота. а поверх усього - халат, такий, як носять жінки. З-під халата виглядали шаровари.. Все те було вельми барвисте; в барвах переважала іржаво-червона. Темне Салех-беєве обличчя було худорляве, однак не зморшкувате. З-під кошлатих чорних брів дивилися блакитні очі.

- Ви, юначе, напевне, й є той славнозвісний спадкоємець, що про нього пишуть усі газети,- промовив він на диво чистою німецькою мовою.- Вітаю вас і зичу вам благословення божого. [134]

Старий уклонився, й Тім відповів поклоном. Хлопець спантеличився ще дужче, бо ж цей чоловік, що зичив йому благословення божого, був вождь легендарних поклонників диявола. До того ж, у цій химерній постаті, що нагадувала Тімові манекен із паноптикуму, як видно, таївся вельми освічений чоловік, Між подобою й дійсністю була така велика різниця, як між восковою квіткою й живою трояндою. Оце й збентежило хлопця. Але він давно вже навчився приховувати свої почуття. І Тім чемно відповів старому:

- Дуже радий познайомитися з вами. Барон багато розповідав мені про вас. (Це, звичайно, була неправда, але Тім тепер часто чув такі ґречні брехеньки й наслідував їх, коли треба.)

До замку вони поїхали у великій чотириколісній кареті, запряженій парою коней. Салех-бей їхав поряд верхи й розмовляв із бароном по-арабському.

Коли карета об'їхала маслиновий гайок, перед ними показався замок, що стояв на вершку положистого пагорба.

Замок той був справжня почвара, цегляна химерія з зубчастими вежами на даху та драконячими пащами на кінцях ринв.

- Не подумайте, будь ласка, що це я побудував таку гидь,- звернувсь до Тіма барон.- Я просто купив цей замок готовий у однієї витребенькуватої англійської леді, бо мені подобається цей куточок землі. А парк насадив уже я сам.

Той парк, що спускався по схилу терасами, був опоряджений на французький лад.

Дерева, підстрижені в формі конусів, кубів та куль, саджались, певно, під циркуль та лінійку - такі рівні були алеї й такі круглі клумби. На кожній терасі вони утворювали інший рисунок. Доріжки були посипані червоною рінню.

- Як вам подобається парк, пане Талер?

Тім, що йому отака підстрижена природа видавалася просто безглуздою, відповів: [135]

- Добре розв'язана задачка з геометрії, бароне!

Троч засміявся.

- Ви висловлюєте своє несхвалення в дуже чемній формі, пане Талер. Мушу визнати, що ви чудово розвиваєтесь.

- Коли така молода людина не каже навпростець того, що думає, вона розвивається погано,- озвався зі свого коня Салех-бей. Він вигукнув те дуже голосно, щоб перекричати торохтіння коліс.

Троч відповів йому по-арабському - досить різко, як здалося Тімові.

Вершник змовчав. Він тільки подивився на хлопця довгим, замисленим поглядом. А невдовзі попрощався й швидкою риссю поїхав попід пагорбом у напрямку далеких гір.

Дивлячись йому вслід, барон сказав:

- Розумна голова, але страшенний мораліст. Він вичитав із закордонних газет, що я видав могилу пастуха Алі за свою, а сам обернувся на свого нібито брата. Він про все мовчатиме, але вимагає, щоб я спокутував той гріх, побудувавши його єзидам новий храм. Видно, не минеться послухатись.

- Якби я вмів, то зараз засміявся б,- поважно відповів Тім.

Замість нього засміявся Троч. Та цього разу той розгонистий сміх із кумедним «ік!» на кінці зовсім не пригнітив Тіма. Хлопець був навіть задоволений, що тепер весь час матиме свій сміх поблизу. Він гадав, що так легше буде його вхопити при слушній нагоді. І не знав, що прикро помиляється. Одне слово. Тім вирішив поки що не розлучатися з бароном.

Їхня карета спинилася перед сходами, що вели з тераси аж нагору, до замку. Знизу ті сходи здавалися нескінченні. Та найхимернішими там, на сходах, були собаки. Статуї собак, що стояли обабіч кожної сходинки, непорушно, ніби дивлячись у долину. Там були їх сотні: пінчери, такси, сетери, фокстер'єри, афганські хорти, чау-чау, спанієлі, бульдоги, шпіци, мопси. Всі вони були [136] череп'яні, полив'яні, барвисте помальовані й двома строкатими вервечками тяглися обабіч сходів до самого замку.

- Стара леді дуже любила собак,- пояснив барон.

І Тім відповів:

- Воно й видно.

Троч хотів був звеліти кучерові підвезти їх до самого замку звивистою дорогою ліворуч від сходів, коли це височенько вгорі з-за одного череп'яного бульдога виступив якийсь чоловік і помахав їм рукою.

- То сеньйор ван дер Толен,- сказав Троч.- Висядьмо та піднімімось до нього пішки. Мені кортить розповісти про наші маргаринові плани. Ото здивується!

Вони вийшли з карети, і барон майже побіг сходами нагору. Тім повільно йшов за ним, розглядаючи полив'яних собак. Розмови про маргарин його не цікавили. Він поки що й гадки не мав, яку важливу роль відіграє ще маргарин у його житті.

Двадцять другий аркуш

СЕНЬЙОР ВАН ДЕР ТОЛЕН

Усередині замок було опоряджено так, що зразу впадало в вічі: барон, такий охочий ходити по художніх виставках, справді має добрий смак. Усе, аж до клямок на дверях, попільничок та підстилочок біля ванн, було просте, гарне і, видимо, дуже дороге. Тіма помістили в затишній напівкруглій кімнаті в одній із веж. Із вікна тої кімнати видно було парк та долину з маслиновим гаєм. І невеличкий аеродром також можна було розглядіти. Як і годиться, аеродром мав бетоновану доріжку з двома рядами ліхтарів, кілька ангарів для літаків і довгий одноповерховий будинок із плескатим дахом - для радистів, метеорологів та іншої обслуги.

Виглянувши у вікно, хлопець побачив на аеродромі два літаки. А третій якраз приземлився. Перед білою [138] стіною аеродромного барака нерухомо стояв барвисте вбраний вершник - напевне, Салех-бей.

Враз хлопця хтось півголосом покликав:

- Пане Талер!

Тім відійшов від вікна й розчинив двері. Перед ним стояв сеньйор ван дер Толен, що з ним Тім напередодні на собачих сходах перемовився лиш кількома словами, бо барон майже без передиху торохтів про маргарин.

- Можу я з вами поговорити так, щоб про це не дізнався барон, пане Талер?

- Коли хочете, я не скажу йому нічого. Але де він тепер?

- Поїхав на аеродром зустрічати містера Пенні.

Сеньйор ван дер Толен уже зайшов до кімнати й упав у плетене крісло-гойдалку. Тім замкнув двері й сів на канапці: звідти він міг дивитись і на свого гостя, й у вікно. Ван дер Толен, як помітив Тім іще напередодні, був чоловік не надто балакучий. Це видно було навіть по його роті - тоненькій рисочці з ледь загнутими вгору кінцями, схожій на стулену акулячу пащу.

- Я прийшов до вас тому, що вашу спадщину ще не оформлено офіційно,- сказав португалець із голландським прізвищем.- Мені йдеться про баронові контрольні акції. Ви знаєтесь хоч трохи на акціях?

- Ні,- відповів хлопець біля вікна. (Він саме дивився, як баронова коляска під'їздить до аеродрому.)

Сеньйор ван дер Толен повільно гойдався в своєму кріслі. Його водяво-блакитні очі неспокійно дивились на Тіма. Погляд їхній був холодний, однак не колючий.

- Акції - це, розумієте...

(Барон у колясці обернувся й помахав Тімові рукою. Тім теж помахав йому.)

- Акції - це паї в капіталі, що...

(Вершник коло білої стіни зрушив з місця: Салех-бей поїхав назустріч Трочеві.)

- Ні, краще я поясню вам на прикладі. Ви хоч слухаєте мене? [139]

- Авжеж,- Тім відвернувся від вікна.

- Так-от, уявіть собі, пане Талер, що хтось надумав посадити сад.

Хлопець кивнув головою.

- Але йому не вистачає грошей, щоб закупити саджанців на всю площу саду. Тому він сам засаджує тільки частину площі, а решту саджанців пропонує купити й посадити іншим людям. А коли яблуні виростуть і почнуть родити, кожен, хто їх посадив, одержує щороку таку частку врожаю, яку частину яблунь він посадив.

Тім почав рахувати вголос:

- Отже, як із сотні яблунь я посадив двадцять та зібрано сто центнерів яблук, то я одержу двадцять центнерів. Так?

- Не зовсім,- сеньйор ван дер Толен ледь помітно осміхнувся.- Адже треба ще заплатити садівникам, що доглядали садок, та робітникам, що збирали яблука. Але приблизно ви, мабуть, зрозуміли, що таке акції.

Тім кивнув головою.

- Мої акції - це ті дерева в садку, що їх посадив я. Це моя частка саду й урожаю.

- Цілком слушно, пане Талер.

Сеньйор ван дер Толен замовк, погойдуючись у кріслі, а Тім знову виглянув у вікно. Коляска вже поверталася з аеродрому до замку. Салех-бей, як і вчора, їхав поряд верхи. Біля барона сидів високий, огрядний лисий чоловік.

- Барон повертається, сеньйоре ван дер Толен.

- Тоді я коротко викладу вам своє прохання, пане Талер. Заповіт небіжчика барона складено так, що новий барон...

- Який це новий барон? - перебив його Тім, але зразу побачив із виразу сеньйорового обличчя, що той про баронову таємницю нічого не знає. Тому Тім похопився: - Вибачте, що я вас перебив.

Хоч ван дер Толен дивився на нього, запитливо звівши брови, наче сподівався, що Тім пояснить своє [140] дивне запитання, хлопець не сказав більше нічого. Тому сеньйор почав спочатку:

- Заповіт складено так хитро, що новий барон, коли схоче, може відсудити в вас усе майно. Ну, це справа його й ваша. Мене цікавлять контрольні акції. У вікні Тім побачив, що коляска зупинилася біля підніжжя сходів і Троч, лисий пан та Салех-бей про щось жваво розмовляють.

- А що таке контрольні акції? - спитав він.

- У нашому акційному товаристві, пане Талер, є кілька паїв загальною вартістю близько двадцяти мільйонів португальських ескудо. Хто володіє цими акціями, той має право голосу в правлінні товариства. Тільки він вирішує, як будуть вестися справи, і більше ніхто.

- І ті контрольні акції успадкував я, сеньйоре ван дер Толен?

- Частину, мій юний друже. Решта належить Салех-беєві, містерові Пенні й мені. (Містер Пенні, очевидно, був той огрядний лисий пан, що вже помалу піднімався сходами до замку разом із Трочем та Салех-беєм.)

- І ви хочете відкупити в мене контрольні акції?

- Я не можу цього зробити, бо, поки вам не вийде двадцять один рік, усім порядкує барон. Та коли ви дійдете повноліття, вступите за всією формою у володіння спадщиною, тоді я радо відкуплю в вас акції. За них я вже сьогодні даю вам будь-яке підприємство нашої фірми. Це підприємство належатиме вам і тоді, коли небіжчиків баронів заповіт із якоїсь причини оголосять недійсним.

Португалець підвівся з крісла-гойдалки. Рот його вже знову нагадував стулену акулячу пащу. Він не звик говорити так багато; Тепер мав щось відповісти Тім.

- Я поміркую про вашу пропозицію, сеньйоре ван дер Толен,- сказав він.

- Поміркуйте, поміркуйте, юначе! Даю вам на це три дні часу. [141]

При цих словах комерсант пішов.

Коли Тім виглянув у вікно, на сходах уже не було нікого.

В півкруглій кімнатці, у вежі замку, що стояв на Месопотамському нагір'ї, сидів самотній підліток на ймення Тім Талер, чотирнадцяти років, зрослий у вбогому завулку великого міста, хлопець без усмішки, однак за своїм багатством і могутністю - майбутній король, коли б його тільки вабило те королювання.

Хоч Тім іще не уявляв собі як слід усього безміру свого багатства, він усе ж знав, що під бароновим ім'ям цілий велетенський флот кораблів плаває по всіх морях. Він здогадувався, що по всьому світі великі ринки, як отой афінський, день у день додають до його багатства нові скарби; він бачив цілу армію директорів, заступників, службовців, робітників, сотні, тисячі, може, десятки тисяч, що виконуватимуть його накази. І те уявлення приємно лоскотало його душу.

Тім згадав, яку йому колись доводилося провадити сміховинну боротьбу за місце, де він міг учити уроки; подумав, який дрібний, незначний тепер проти нього директор гідростанції. І, дивлячись згори на розкішний, хоч і чудний парк, хлопець уявив себе тим самотнім баварським королем із казки, що про нього колись розповідала на уроці історії їхня пристаркувата вчителька. Тім почав мріяти, як би він у золотій кареті, І в супроводі Салех-бея верхи на коні, під'їхав до крамнички пані Бебер та як би витріщили на нього очі й роззявили роти всі сусіди.

Хлопець у вежі забув на хвилину утрачений сміх і снив про королювання.

Та дійсність була зовсім не така. Дійсність звалася «маргарин» і мала ще дуже відчутно нагадати йому про втрачений сміх. [142]

Двадцять третій аркуш

ЗАСІДАННЯ

В замку була обшита дерев'яною панеллю зала для нарад. Там стояв довгий стіл, а навколо нього - масивні крісла. Коли ввійти до зали, погляд зразу падав на картину в широкій золотій рамі, що висіла на стіні навпроти дверей. То був той самий славнозвісний Рембрандтів автопортрет, що про нього в усьому світі гадали, ніби він пропав під час якоїсь війни.

Під тим портретом, на чільному місці, сидів за столом барон. Ліворуч нього сиділи Салех-бей і Тім Талер, а праворуч - містер Пенні та сеньйор ван дер Толен. Розмовляли - цього разу цілком офіційно - про «становище на масляному ринку». Задля Тіма говорилося по-німецькому, хоч містер Пенні й не дуже добре знав німецьку мову.

На початку засідання (бо такі розмови називаються засідання, неначе головне в них - сидіння) містер Пенні спокійно, по-діловому спитав, чи Тім Талер і надалі братиме участь у всіх таємних нарадах. Салех-бей висловився за, але решта компаньйонів була проти. Хлопець мав узяти участь тільки в цьому одному засіданні - по-перше, щоб трохи познайомитися зі справами фірми, а по-друге - тому, що він мав доповісти про споживання маргарину в завулку, де він виріс.

Але спершу, хоч це може здатися дивним, мова зайшла про афганських гострив. Із розмови компаньйонів Тім узнав таке: акційне товариство барона Троча роздало в Афганістані задарма мільйонів зо два дешевих ножів і ножиць, але не з великої щедрості, а щоб на цьому заробити. Бо ті ножі й ножиці коштували товариству не більш як по п'ятнадцять пулів (пуль - найдрібніша афганська монета); а щоб погострити ножі й ножиці, треба заплатити двадцять пулів. А що ножі й ножиці були не дуже добрі, то гострити їх доводилося не менш як двічі на рік. Отож товариство понаймало до себе на роботу всіх гостріїв у Афганістані, й один [143] чоловік на ймення Рамадулла, раніш відомий на всю країну жорстокий розбійник, був у них за старшину й тримав їх усіх у суворому послуху. Він постачав їм точила й забезпечував їх роботою, але вимагав за це таку велику частку їхнього заробітку, що мав змогу з кожного погостреного ножа десять пулів, цебто половину плати, здавати товариству. Як подумати, скільки ще прилипало до пальців йому, то неважко собі уявити, скільки лишалось самим гостріям.

Незабаром потрібно стало ще й рекламувати тих гостріїв. А в такій бідній країні, як Афганістан, цього не можна було зробити ні по радіо, ні через газети, ні плакатами. Бо мало хто з афганців уміє читати, а радіо там і взагалі нема.

Тому понаймали вуличних співаків, щоб вони співали пісню про гостріїв.

У тій пісні, що про неї довго розмовляли компаньйони, не вихвалялися вміння та вправність гостріїв, а оспівувалася їхня бідність, щоб люди їх жаліли та віддавали їм гострити ножі й ножиці.

Пісня та була така:

Точило крутить, крутить

Гострій отой убогий,

Все крутить, крутить, крутить

За мідний гріш гіркий.

З села в село він ходить,

Гострій отой убогий,

Несіть ножі, дівчата,

Щоб він їх погострив.

Остання строфа мала показати, який щасливий бу ває гострій, коли йому приносять гострити ножі та ножиці:

І гострить, гострить, гострить

Гострій отой веселий. [144]

Спасибі, добрі люди.

Тепер він має хліб.

Про те, що вбогі гострії мало не весь свій заробіток віддають Рамадуллі, а той знов же більшу частину тих грошей приносить у замок, пісня мовчала.

Тімові пригадався дідусь, що приходив із гладкою глухонімою дочкою до їхнього завулка гострити ножі, і хлопець спитав себе подумки: чи й той старий мусив ділитися своїм заробітком із якимсь акційним товариством? Тіма гнітила думка про брудне королівство, що його він успадкував, і Салех-бей, видно, здогадався, про що він думає, бо сказав:

- Наш юний друг, здається, не схвалює методів товариства. Він, мабуть, думає, що розбійник Рамадулла зовсім не перемінив свого ремесла, а тільки став грабувати культурніше, ніж перше. Що ж, панове, моя думка така сама.

- А ми дауно знаємо уашу думку,- коротко й сухо відповів містер Пенні.

А барон жваво додав:

- Коли в замученій розбійниками країні зробити розбійників культурними, це, Салех-бею, вже великий прогрес. Згодом, коли в країні завдяки нашій допомозі запанують закон і лад, тоді, звісно, й наші комерційні методи стануть законні.

- Те саме казали ви мені й тоді, як ми говорили про злочинне низьку платню на плантаціях цукрової тростини в одній південноамериканській країні,- відповів Салех-бей.- А тепер у тій країні за допомогою наших грошей став президентом злодій і вбивця, й жит-тя там іще погіршало.

- Але ж той президент шануй релігія! - докинув містер Пенні.

- В такому разі я волію порядного президента-безвірника,- буркнув Салех-бей.

Аж тоді вперше озвався сеньйор ван дер Толен:

- Панове, ми ж прості комерсанти, яке нам діло до [146] політики? Треба сподіватися, що світ покращає, щоб ми всі могли торгувати мирно, як добрі друзі. І перейдімо до головного: до масла.

- Краще сказати - до маргарину.- сміючись, поправив його барон і зразу ж розпочав предовгу доповідь про все те. що вже розповідав Тімові в літаку. Але говорив він не як мирний торговець, а як полководець. що намірився стерти на порох своїх ворогів - інших торговців маслом.

Тім слухав усе те лише краєчком вуха. В нього аж голова обертом ішла. Він питав себе: а навіщо взагалі провадити комерцію, коли це можливо тільки в такий огидний спосіб? Баронове королівство вже не надило його. Хлопця пройняв страх перед торгівлею. Навіть крамничка пані Бебер уже здавалась ЙОМУ чимось страшним.

На щастя, хлопець вчасно пригадав, що не має ще права зрікатись торгівлі, бо йому ще треба виторгувати назад свій сміх. Отож доведеться ще якийсь час між вовками жити й по-вовчому вити.

Аж ось барон попросив Тіма ще раз переказати все те. що він розповідав у літаку про споживання маргарину в їхньому завулку.

Коли Тім закінчив, у залі засідань на хвилину запала мовчанка.

.Тоді містер Пенні промимрив:

- Ми спрауді легкоуажили маргарином...

- А тим часом наше товариство й розбагатіло якраз на всяких дрібницях, потрібних бідним людям,- додав сеньйор ван дер Толен.- Ми злочинно занедбали маргариновий ринок. Треба його перебудувати якось цілком по-новому.

Тім, що зовсім заспокоївся, озвавсь:

- Мене завжди сердило те. що багатим продають Їхнє масло гарно запаковане в срібний папір, а нам наш Маргарин вишкрібають лопаткою з бочки та ляпають Йа шматок рудого паперу. Хіба й ми б не могли продавати [147] бідним людям наш маргарин запакований? Грошей же в нас на це вистачить.

Четверо дорослих отетеріло витріщилися на нього. тоді. мов по команді, зареготали.

- Та ви ж неосіненна людина, пане Талер! - вигукнув містер Пенні.

- Розв'язок проблеми був у нас перед очима, а ми його не бачили,- реготав барон. Навіть сеньйор ван дер Толен схопився з місця й вирячивсь на Тіма. мов на яке заморське диво.

Старий Салех-бей був іще порівняно спокійний, і тому Тім спитав його. що їм видалось таким дивовижним у його пропозиції.

- Любий мій юначе,- врочисто промовив старий.- Ви ж оце щойно винайшли марковий маргарин.

Двадцять четвертий аркуш

ЗАБУТИЙ ДЕНЬ НАРОДЖЕННЯ

Лише на другий та третій день Тім помалу збагнув. що такого незвичайного в тому марковому маргарині. Бо в замку ті два дні тільки й мови було, що про маргарин. Навіть слуги, здавалося, шепотілись про нього по-арабському та по-курдському.

Річ була ось у чому: масло вже віддавна продавалось гарно запаковане й під певною назвою. В Німеччині. наприклад, вироблялося «Німецьке сільське масло» і просто «Німецьке масло», в Голландії - «Голландське масло».

Комерсантові, що хотів торгувати маслом, треба було ладнати з молочарськими товариствами. А компанія барона Троча, на жаль, пересварилася з трьома найбільшими товариствами, і тепер тисячі малих молочарень постачали своє масло іншій компанії, що, до того ж, продавала масло дешевше, ніж барон.

З маргарином же справа стояла інакше. Він продавався незапакований і без назви. То був не «марковий виріб», його завозили до крамниць у бочках, а вже [148] продавці набирали його з бочки дерев'яною лопаткою й відважували покупцям. А через те. що маргарин ніколи не мав своєї марки, та й фабрики, де він вироблявся, лишались невідомі покупцям, у продаж часто потрапляв дешевий, але поганий маргарин із невеличких фабрик, і великим торговцям важко було «прибрати до рук маргариновий ринок», як висловився сеньйор ван дер Толен.

Але цьому мав настати край. Акційне товариство барона Троча вирішило «викинути на ринок» новий ґатунок маргарину, гарно упакований і з назвою. І випуск того маргарину був спланований, як бойова операція під час війни. Треба було таємно поскуповувати всі великі маргаринові фабрики, дослідити в лабораторії всі ґатунки маргарину, вибрати з-поміж них найкращий і налагодити його виробництво якнайдешевшим коштом на всіх тих фабриках. Іще важливо було забезпечити гарну рекламу, щоб господині купували замість дорогого масла «майже такий самий», але набагато дешевший за нього марковий маргарин. (Безіменний поганий маргарин витісниться з ринку, так би мовити, сам собою.) Звичайно, всі ті операції мали провадитись якомога швидше й у цілковитій таємниці, щоб якась інша фірма не випередила компанії барона Троча. За ті дні в замку відбувались телефонні розмови з майже всіма великими містами Європи, надходили й відсилалися телеграми, а часом прилітав на літаку який-небудь пан, що замикавсь на годину-другу з бароном та його компаньйонами в залі засідань, а тоді зразу ж відлітав.

Тім мав багато вільного часу ті дні. Півдня він просидів у своїй кімнаті над тією нещасливою угодою, що її підписав, іще бувши малим дурним хлопцем, під грубезним крислатим каштаном. Але так і не надумав ніякого способу повернути собі свій сміх. Та ще всі оті розмови про великі справи геть закрутили йому голову, і він не бачив, що до втраченого сміху є дуже короткий шлях. [149]

Але троє людей у Гамбурзі вже відкрили той шлях. і дивний випадок з'єднав хлопця з тими людьми. Випадок той скористався з телефону. Невеличкий телефонний апарат у Тімовій кімнаті раптом задзеленчав, а коли хлопець підняв трубку й приклав до вуха. в ній пролунав далекий голос:

- Говорить Гамбург. Це пан барон?

Тімові зразу аж дух перехопило. Та за мить він закричав:

- Це ви, пане Рікерт? Я Тім!

Далекий голос у трубці став трохи гучніший і виразніший:

- Так, так, це я! Господи, Тіме. як же нам пощастило! В мене були Крешимир і Джонні. Крешимир знає...

На лихо, Тім не дав панові Рікертові договорити. Згарячу він перебив його:

- Вітайте Джонні, пане Рікерт! І Крешимира теж! І вашу матусю! І подумайте, будь ласка...

Через Тімове плече на телефон лягла чиясь рука... Розмову було перервано. Тім, зляканий, аж блідий. рвучко обернувся. За спиною в нього стояв барон. У своєму щасливому хвилюванні хлопець не почув, як той увійшов до кімнати.

- Ви повинні забути своїх колишніх знайомих, пане Талер,- спокійно мовив Троч.- Скоро ви станете королем, успадкуєте королівство рахунків. А там правлять цифри, а не почуття.

Тім хотів був відказати: «Я запам'ятаю, бароне»,- як він казав уже не раз. Та цього разу хлопець не зумів опанувати себе. Він поклав руки й голову на телефонний столик і заплакав. Звідкись наче здалеку-здалеку пролунав голос: «Залиште мене з хлопцем на самоті, бароне». Тоді почулися кроки, грюкнули двері, і все стихло. Чутно було тільки, як хлипав Тім.

Старий Салех-бей сів на канапці в кутку біля вікна й дав хлопцеві виплакатися.

За якусь часину він промовив: [150]

- Здається мені, юначе, ця спадщина затяжка для ваших пліч.

Тім хлипнув іще разів зо два, тоді втерся мереживною хусточкою й відповів:

- Я її зовсім не хочу, цієї спадщини, Салех-бею.

- Чого ж ти тоді хочеш, хлопче?

Тімові так приємно стало, що до нього звертались на «ти». Йому нестерпно захотілось розповісти Салех-беєві про свою біду. Але ж тоді його сміх пропаде назавжди... І Тім мовчав.

- Ну гаразд,- буркнув старий.- У барона багато таємниць. І одна з тих таємниць - ти. Здається, це якась дуже негарна таємниця.

Тім кивнув головою, але не сказав нічого. Салех-бей облишив ту тему й розповів хлопцеві, яким побитом він став одним із головних компаньйонів у цій багатющій фірмі.

- їм треба було для своїх азіатських справ якої-небудь дуже шанованої людини. Коли б вони обрали магометанина, уразились би буддійські країни, а обравши буддійця, розгнівали б магометан. Тому й було обрано проводиря невеличкої секти, що її всі вважали за чудну, але гідну пошани,- цебто мене. Задля мене барон і замок цей купив. Крім того, він цікавиться нашою вірою.

- Але ж вам багато що в цьому товаристві не подобається,- сказав Тім.- Нащо ж ви погодились увійти до нього?

- Я погодився тільки з умовою, що мені дадуть частину контрольних акцій. І вони мусили пристати на мою умову. Тепер я маю право голосу в правлінні й можу де в чому їм перешкоджати, хоч і не багато, в чому. А крім того...- Салех-бей захихотів і повів далі вже пошепки: - Крім того, всі ті мільйони, що я в них наживаю, я обертаю проти них-таки. В Південній Америці я найняв армію, що повалить того злодія й убивцю, якому наше товариство допомогло посісти президентське крісло. А в Афганістані... [151]

У двері постукали, й Салех-бей зразу змовк.

- Відчинити? - потихеньку спитав Тім.

Старий кивнув головою, і хлопець пішов до дверей. У кімнату спрожогу влетів звичайно такий незворушний та надутий містер Пенні й здушено забелькотів:

- Що означає зіс демд... м-м... сей прокляти... м-м...

- Говоріть по-англійському.- сказав йому Салех-бей.- Я перекладу хлопцеві.

І містер Пенні залопотів по-англійському. Потім раптово урвав свою мову. показав на Тіма й промовив до Салех-бея:

- Перекладіть йому, будь ласка!

Старий спокійно запросив англійця сісти, а коли містер Пенні знесилено впав у крісло-гойдалку, сказав і Тімові:

- Барон щойно звільнив з посади директора нашого гамбурзького пароплавства пана Рікерта. А що містер Пенні має найбільшу частину пароплавних акцій, то він відмовляється погодитись на це. Він твердить, що Рікерта дуже поважають у Гамбурзі, і коли його звільнити, вийде великий скандал, який погано відіб'ється на справах пароплавства. І це нібито ваша провина, каже містер Пенні, що пана Рікерта звільняють.

- Моя провина? - зчудовано перепитав поблідлий Тім.

- Аужеж, так, уаша проуина! - містер Пенні знову схопився з крісла.- Так сказала... м-м.. сказало... м-м... так сказау барон.

Тім, певна річ. зрозумів, що Рікертове звільнення пов'язане з їхньою телефонною розмовою; але звертати на нього провину - це вже диявольське паскудство: хіба ж міг Тім хотіти, щоб пан Рікерт позбувся своєї посади!

Салех-бей раптом вийшов із кімнати, у дверях мовивши до містера Пенні:

- Побалакайте собі з паном Талером по-німець-кому. це примусить вас говорити повільно й спокійно. [152]

Огрядний лондонець гепнувся на канапу в кутку, де щойно сидів старий, і простогнав:

- Я се не можу розуміти!

Тім спочатку хотів був сказати, що барон набрехав. Але йому спала на думку розмова з сеньйором ван дер Толеном, що про неї він дуже багато міркував. І та думка напровадила його на іншу.

- Містер Пенні,- почав він.- Ви ж. певне, знаєте. що я, дійшовши повноліття, успадкую цілу купу контрольних акцій.

- Знаю,- буркнув гладун у кутку.

- Якби я вам пообіцяв письмово, що відступлю вам ті акції, коли мені вийде двадцять один рік, ви дали б мені зараз ваші акції гамбурзького пароплавства?

Містер Пенні притих у своєму кутку й ледь прищулив очі. Чути було тільки, як він сопе. Аж ось англієць важко видихнув із себе:

- А уи не джартуєте, містере Талер?

- Ні, містере Пенні, кажу цілком щиро.

- То замкніть двері!

Тім послухався його. А тоді в замкненій кімнаті уклав з містером Пенні угоду, що її він повинен був тримати в такій самій таємниці, як і угоду з Трочем, бо ні в якому разі не можна було допустити до того, щоб барон дізнався про неї. Була тільки одна прикрість: на передачу акцій пароплавства до інших рук існував обмежувальний термін, і Тім міг одержати їх лише через рік. Але, може, так воно було й краще для тих планів. що їх наукладав Тім за безсонну ніч після того дня.

Як на чотирнадцятирічного хлопця, то були грандіозні плани. Бо Тім надумав не більш не менш, як із Салех-беєвою допомогою так розхитати баронове акційне товариство, цю найбагатшу й наймогутнішу в світі фірму, щоб баронові лишилось тільки віддати Тімові його сміх або ж зразу втратити все своє багатство й могутність.

Замір той був зовсім божевільний і навіть із Салех-беєвою допомогою нездійсненний. Тім іще тільки-тільки [153] почав принюхуватись до світу великої комерції, а тому ще не вмів оцінювати витривалості такої всесвітньої фірми, застрахованої на тисячу ладів. Недооцінював він і тих людей, що з ними мав справу, та їх одностайність у хвилину небезпеки. Кожен з них коли завгодно кинув би напризволяще, на поталу злидням свою жінку, дітей, батька, матір, якби це могло відвернути падіння фірми. А Троч віддав би навіть сміх. Та й сам Тім був іще замолодий і не досить хитрий для таких планів. Сміх його можна було відвоювати куди простішим способом - усього лиш кількома словами. Але при бароні Тім розучився дивитися на життя просто. Замість іти навпростець, він блукав манівцями. Удосвіта, о четвертій годині, так і не заснувши, він іще раз перечитав укладену з містером Пенні угоду, і йому впала в вічі дата: тридцяте вересня. То був Тімів день народження.

Тімові вийшло повних п'ятнадцять років. День, що його Тімові ровесники відзначали тортами, какао та веселим сміхом, для Тіма був днем таємних угод і чорних планів. Гіркі сльози знову обернули захопленого таємними планами змовника на звичайного нещасливого підлітка без усмішки й принесли йому, коли він нарешті заплющив очі. майже легкий сон.

Двадцять п'ятий аркуш

У ЧЕРВОНОМУ ПАВІЛЬЙОНІ

День у замку минав суворо за розкладом. Уранці о восьмій годині в Тімові двері стукали, і входив, не чекаючи дозволу, молодий приязний слуга, що з ним Тім, на жаль, не міг поговорити, бо не знав арабської мови. Слуга відгортав завіси на вікнах, приносив глек гарячої води й виливав його в умивальник.

Умившись і вдягнувшись. Тім смикав за гаптований пас від дзвоника. Тоді слуга приносив сніданок на таці, пересував до вікна столик, розставляв на ньому посуд, наливав у чашку какао, насипав цукру, клав вершки, [154] присовував стілець і, поклавши руки на бильце, чекав, поки Тім сідатиме, щоб підсунути йому стільця ближче.

Першого дня слуга широко всміхався Тімові. Але вже з другого дня він не всміхнувся ні разу. Обличчя в нього було весь час сумне, немовби він знав Тімове горе.

Тім приймав усі ті послуги мовчки. Хоч він і відчував юнакове співчуття і хоч слуга йому подобався, все ж Тім бував щоразу радий, коли церемонія сніданку закінчувалась і він лишався біля свого вікна сам.

Уранці після тієї безсонної ночі Тімові дуже не хотілося вставати. До того ж іще не зовсім і розвиднілось, бо церемоніал зі слугою відбувся як і щоранку, їздячи з бароном. Тім навчився самовладання, дисципліни.

Снідаючи. Тім дивився у вікно на замкові сходи. Барвисті полив'яні собаки блищали під дощем. Та однаково мали вони жалюгідний вигляд, сидячи так застигло й безпорадно під струменями зливи, в безглуздому суворому послуху.

Тімові здавалось, наче й сам він стане одним із тих собак, коли йому не пощастить невдовзі повернути собі свій сміх.

Задзеленчав телефон. Дзвонив Троч. Він попросив Тіма прийти до нього на п'яту годину пити чай. До Червоного павільйону.

Тім відповів: «Гаразд, бароне!» - й сів доїдати сніданок, думаючи, що це Трочеві треба. Досі барон просто приходив до хлопцевої кімнати, коли хотів щось сказати йому. Видно, тут справа якась надзвичайна, що він хоче зустрітись із Тімом у павільйоні.

Рівно о першій годині удар гонга покликав Тіма на обід. Широкими сходами з гарно вигнутим різьбленим поруччям хлопець спустивсь на широкий поверх, до їдальні. Хоч за обідом він сидів поруч барона, той ані словом не згадав про своє запрошення.

Салех-бей, що завжди приходив до замку тільки по обіді, цього разу обідав з ними, і в Тіма склалось таке [155] враження, ніби того ранку відбулась якась важлива розмова. Але компаньйони про неї не згадували. Взагалі ; той обід був такий мовчазний, як іще ні разу.

Після обіду звичайно всі розходились по своїх кімнатах. Тім здебільшого брався за книжку, бо в його кімнаті стояла шафа з німецькими книжками. Найбільше він любив червоно-брунатні томики з нижньої полиці, твори Чарлза Діккенса. Він поглинав романи про бідних нещасливих дітей, мов марципани пані Бебер. Але щоразу боявся щасливого кінця роману. Три книжки він узагалі не дочитав, помітивши, що дія в них прямує до щасливого кінця.

Дощ лив і лив цілий день. У таку негоду тільки й читати сумні романи. Але Тімові чогось не читалося. Він сидів на канапці біля вікна й видивлявся надвір, на сіру долину, слухав одноманітне шелестіння дощу й силкувався пригадати свої нічні плани. Але в голові у нього було порожньо, як виметено. Він не міг думати. Сiв і тупо дивився на дощ, на смутних собак понад сходами та на фургон, що в ньому кожного дня по обіді привозили з Мосула свіжі припаси на кухню.

За кілька хвилин перед п'ятою годиною прийшов молодий слуга з парасолькою в руках. Він, видно, збирався відвести хлопця до Червоного павільйону, несучи парасольку над ним. Але Тім забрав її в нього з рук і на мигах показав, що піде сам. Потім надів легкого плаща, купленого на ринку в Афінах, і вийшов з кімнати.

На сходах стояв Салех-бей. Потискуючи Тімові руку, він непомітно - хоч поблизу не видно було нікого - вклав йому в долоню авторучку й шепнув: «Підпишеш нею».

Не встиг Тім нічого спитати, як старий уже пішов. Хлопець поклав ручку до кишені, спустився сходами вниз і перейшов вестибюль. Старий швейцар відчинив перед ним великі двері.

Тім хотів був уже вийти під дощ, але ззаду гукнули:

- Одну хвилиночку, пане Талер! [156]

Із-за колони виступив сеньйор ван дер Толен. Махнувши рукою старому, щоб той відійшов, він півголосом спитав Тіма:

- Ну, ви вже надумались, пане Талер? Пообіцяйте, що віддасте мені свої контрольні акції, а я за те подарую вам велике підприємство.

Тім трохи-трохи не сказав: «Я вже пообіцяв їх містерові Пенні». Але похмурий сльотавий день допоміг хлопцеві хоч в одному: поки млявий, аж сонний Тім зібравсь розкрити рота. він устиг похопитись та відповісти обережніше:

- Я не можу укласти з вами такої угоди, сеньйоре ван дер Толен.

- Шкода,- з незворушним обличчям сказав португалець. А вже повернувшись іти, додав: - Ну хоч у маргариновій справі пристаньте на наше прохання. пане Талер.

Аж тоді він пішов.

Тім не знав, що йому й думати про обидві ті зустрічі. Спершу Салех-беєва таємнича авторучка, тепер сеньйорові ван дер Толенові незрозумілі слова про маргаринову справу. «Ще не вистачало, щоб мене й містер Пенні перестрів»,- подумав хлопець.

Так і вийшло. Спускаючись під парасолькою замковими сходами, він побачив біля одного мокрого полив'яного хорта містера Пенні - також під парасолькою.

- Прошу уас сілкоуита моучанка про наш неуелички фчорашні умова,- сказав англієць.

- Добре, добре,- заспокоїв його Тім.

Містер Пенні, здавалося, мав на думці ще щось, але, видно, не міг наважитись його висловити. Кивнувши хлопцеві головою, він рушив сходами вниз.

Тім розгубився. Певно, його розмова з бароном у павільйоні матиме для компаньйонів велику вагу, а то б вони не стали один по одному перепиняти його. Тяжко замислившись, попрямував він до павільйону. Той так званий Червоний павільйон стояв на середній терасі парку. Називався він так, мабуть, по вогненно-червоному [157] півникові на шпилі конусовидного даху, бо сам і павільйон був білий.

Підстрижені дерева й кущі нагадували поважних панів, що попали ненароком під дощ, вимокли, змерзли й чекають, поки їх хтось вирятує. Тім швиденько пройшов алеєю, що вела до павільйону. Барон уже стояв у напіввідчинених скляних дверях і чекав на хлопця.

- Ви спізнились на три хвилини,- сказав він Тімові.- Вас хтось затримав?

- Так,- відповів Тім. однак допитуватися барон не став.

У круглій кімнаті павільйону стояли легенькі меблі, оббиті шовком у жовту та ясно-брунатну пересмужку. Служниця поналивала з російського самовара чаю в чашки й хотіла була вийти. Тім побачив, що парасольки в неї нема, й спинив її:

- Стривайте!

Жінка обернулась, і хлопець простяг їй свою парасольку.

Служниця неначе аж злякалася. Напівспантеличено, напівзапитливо глянула вона на барона. Але той тільки засміявся й помахом руки звелів їй щезнути разом із парасолькою. Жінку мов вітром звіяло.

- А знаєте, отакі ваші невеличкі люб'язності справляють враження на людей, пане Талер. Тільки не треба надуживати їх,- сказав барон. Тоді поміг Тімові скинути плащ, і вони посідали.

- Бачте, пане Талер, річ у тому, що люди поділені на дві половини: на панів і на слуг. Наш час хоче стерти той поділ, однак це небезпечно. Треба, щоб були люди, які думають і наказують, і люди, що не думаючи виконують накази.

Тім спокійно надпив із чашки чаю і аж тоді відповів:

- Як я був іще малий, бароне, мій тато якось сказав мені: «Не вір у панів і слуг, синку! Вір тільки. що є люди розумні й дурні, і бійся дурості, коли вона не добродушна!» Я тоді ці слова навіть у зошит записав, тому й пам'ятаю їх досі. [158]

- Ваш батько сказав по суті те саме, що й я, пане Талер. Бо розумні - то й є пани, а дурні - то слуги.

- А Салех-бей,- не поступався Тім,- розповідав мені, що в Афганістані й у Південній Америці панують тільки ті, хто випадково народився паном.

- Народитись випадково не можна,- невдоволено буркнув Троч.- І взагалі ваш Салех-бей, пане Талер, комуніст. Навіть попри свою віру. Він тільки сам цього не знає. Зате я знаю, що він найняв у Південній Америці армію, щоб скинути нашого президента. Знаю Я й те, що в Афганістані він підбурює гостріїв проти нашого уповноваженого Рамадулли.

- Ви це знаєте? - в Тіма на обличчі відбився такий жах, що барон зареготав.

- Я знаю більше, ніж ви гадаєте,- сміючись, вигукнув він.- Навіть про вашу умову з містером Пенні. І здогадуюсь, яку пропозицію зробив вам ван дер Толен.

Тім аж чаєм похлинувся. Невже Троч умів читати думки? Але баронове всевидіння пояснювалося куди простіше. Він сам розтлумачив хлопцеві:

- В цьому замку всі слуги - мої шпиги. Ви не помітили, пане Талер, що на вашому письмовому столі лежить нова вимочка?

- Ні...

- А на такі дрібнички треба звертати увагу! Бо як піднести стару вимочку до дзеркала, на ній можна досить виразно прочитати вашу умову з містером Пенні.

Ту хвилину Тім зрозумів, що ніколи не спроможеться перехитрити барона в комерційних справах. І всі нічні плани розтанули, як пара над чашкою чаю. В боротьбі за свій сміх хлопець програв один тур.

- А ви тепер щось зробите Салех-беєві та містерові Пенні, бароне?

Троч ізнову засміявся:

- Ні, любий мій! Із мене досить того, що я про все знаю. Звісно, я спершу був розсердився, коли довідався. що вони зробили чи надумали. Але, на щастя, ваш сміх допомагає мені не брати таких речей близько до серця. [159] Він розвіює прикрість і дає мені полегкість. От бачте. пане Талер, для якої корисної мети вживаю я ваш сміх.

- Ви, здається, в усьому шукаєте тільки користі, бароне.

- За двома винятками, пане Талере: моя зацікавленість мистецтвом цілком некорислива. А також моя зацікавленість релі... Ні,- перебив Троч сам себе: - Релігією я цікавлюся теж із певною метою.

Тім поквапився звести розмову на інше, бо напохваті не було підходящої люстри. Він спитав:

- А як же моя умова з містером Пенні?

- Що ж, пане Талер... Містер Пенні одержить контрольні акції в тому разі, коли ви в двадцять один рік справді ще будете спадкоємцем усього мого майна разом із контрольними акціями. А Друга частина вашої умови чинна безумовно. Рівно за рік більшість акцій нашого гамбурзького пароплавства стане ваша. Ви, певне, хочете знову настановити пана Рікерта директором?

- Так,- не вагаючись відповів Тім.

- Ну що ж, сподіваймося, що за рік він буде ще живий і здоровий...

Ці останні слова, хоч барон вимовив їх цілком побіжно, налякали хлопця. Бо Троч, безперечно, здатен на все: він може навіть яким-небудь способом укоротити віку панові Рікертові. Отже, Тімові треба вдати, ніби пан Рікерт не дуже й цікавить його. Тому хлопець сказав:

- Мені просто ніяково було, що пан Рікерт позбув ся посади через мене. Ось тому я й уклав умову з місте ром Пенні.

Троч налив собі в чай рому з невеличкої кришталевої карафки й спитав:

- І вам?

Тім кивнув головою і барон долив рому й йому. А тоді промовив: [160]

- Я хочу зробити вам одну пропозицію, пане Талер. Протягом одного року не шукайте ніякого зв'язку з паном Рікертом та іншими вашими гамбурзькими друзями. і я подбаю, щоб за рік акції гамбурзького пароплавства справді належали вам. Згода?

- Добре.- замислившись на хвилинку, відповів Тім.- Згода.

А собі подумав: «Цілий рік без сміху - це дуже важко, але ж ціле життя без сміху - це нестерпно. Треба якось перетерпіти цей рік. А за такий довгий час я, може. придумаю, як пошити барона в дурні. Як комерсанта я його перехитрити не зумію, але, може. знайду в нього якесь вразливе місце просто як у людини».

Немов прочитавши Тімові думки, Троч додав:

- Давайте, пане Талер, весь цей рік подорожувати. Довкола світу, вдвох, як приватні особи. Я дарую вам цю подорож на день народження. І щиро вітаю вас, хоч і з запізненням.- Він дзвінко засміявся й подав Тімові холодну-холодну руку.

- Щиро дякую,- відповів Тім і сьорбнув гарячого чаю.

_ А ви знаєте, що п'єте ром стерничого Джонні. пане Талер?

- Цебто як, пробачте?

- Ви ж забули в шинку в Генуї дві пляшки рому, що виграли в стерничого. Вас тоді відвезли до готелю, а ром я переслав сюди, щоб вигране справді дісталося вам. Я в таких дрібницях пунктуальний.

Тім замовчав, тільки проказав собі подумки прислів'я Джонні: «Навчи мене сміятись, порятуй мою душу... стерничий;»

Троч перебий хлопцеві думки:

- Ну, а тепер до діла, пане Талер. Поговорімо про маргарин.

- Гаразд, бароне, гаразд, до діла, то й до діла. [161]

Двадцять шостий аркуш

МАРГАРИН

Барон підвівся й почав походжати павільйоном, заклавши руки за спину.

Ходячи, він говорив до Тіма:

- Ви ж знаєте, пане Талер, що нашому майбутньому маргаринові треба дати назву. Назва має бути приваблива й легко запам'ятовуватись;, найкраще, коли вона буде пов'язана з чимсь відомим. Ми дуже довго радились про ту назву, бо це надзвичайно важлива справа. Добра назва виробу - це однаково що готові гроші.

Тім кивнув головою, ще не розуміючи, до чого ж тут він. Але скоро він мав про це дізнатися.

- Після всіляких можливих пропозицій,- не зупиняючись, провадив барон,- Салех-бей висловив одну думку, і ми її схвалили відразу й одноголосне, як найкращу з усіх. Салех-бей, зауважу до речі, дуже корисна для нас людина попри всі свої химерні ідеї. Але це я так, між іншим... Він запропонував для маргарину назву «Тім Талер».

Троч зупинився й поглянув на хлопця крізь темні окуляри, що їх тепер майже ніколи не скидав. Але на Тімовому обличчі не відбилось нічого. Хлопець сприйняв ту пропозицію зовсім байдуже, чи, може, й не зрозумів її. Тому барон почав змальовувати йому наслідки.

- Ви повинні зрозуміти, пане Талер, що маркового маргарину не було й нема ще ніде в світі. Коли ми несподівано, навально, в великій кількості викинемо його в продаж, то нам, можливо, пощастить завоювати весь світовий маргариновий ринок. У кількох південноамериканських країнах ми зможемо купити навіть монополію на продаж маргарину. А все це означає, пане Талер, що ваше ім'я буде на устах у всіх людей від Нью-Йорка до Токіо, від Стокгольма до Кейптауна. Навіть найменша крамничка в найглухішому перському селищі продаватиме маргарин із вашим прізвищем. І по всьому світі розійдеться надрукована синьою фарбою [162] а жовтому папері фотографія усміхненого хлопця - ваша фотографія, пане Талер!

Аж тоді Тім весь нашорошився й тихо-тихо спитав:

- Як же я всміхатимуся, коли я не вмію сміятись?

- Це питання другорядне, пане Талер. До нього ми зараз перейдемо. А спершу я хочу знати ось що: чи згодні ви на таку назву маргарину?

Тім відповів не зразу.

Він зрозумів нарешті, чому така марка на маргарині дуже вигідна товариству. Адже він. Тім Талер,- славнозвісний багатий спадкоємець, і портрети його та прізвище й досі з'являються в газетах по всьому світі. Отже, не нова марка маргарину вславить його ім'я, а навпаки, його ім'я, вже широко відоме, уславить нову марку маргарину.

- Це треба вирішити швидко, бароне?

- Сьогодні, пане Талер! Отут на місці, не виходячи з павільйону. Хоч маргарин буде випущено тільки за рік, однак усе найважливіше треба ухвалити вже цими днями. В справу треба вкласти силу-силенну грошей. Ставка в цій грі така висока, що все наше товариство може піти димом догори в разі невдачі.

Застромивши руку в кишеню. Тім намацав там Салехбеєву авторучку. В вухах йому ще лунали Трочеві слова, що в разі невдачі з маргарином може піти димом догори все товариство. Чи не хоче Салех-бей за допомогою цієї ручки «пустити все товариство димом догори»? Отже, Салех-бей - таємний Тімів спільник?

Нібито в задумі, хлопець вийняв ручку з кишені й почав крутити її в руках, щоб мати напохваті, коли буде треба.

Що він програє, коли маргарин називатиметься його ім'ям? А виграти він може багато, маючи Салех-бея на своєму боці. І хлопець вирішив довіритися старому.

- Скажіть вашим компаньйонам, бароне, що я згоден!

Троч зітхнув із видимою полегкістю, однак більш нічим не виказав хвилювання. [163]

- В такому разі,- сказав він,- треба підписати ще одну угоду.- Тоді відсунув убік Тімову чашку й поклав перед Тімом два однакові примірники угоди.

Барон чекав, що Тім спершу прочитає її. Але хлопець, боячись, що Троч дасть йому іншу ручку, підписав не читаючи Салех-беєвою ручкою.

Потім барон розписався на кожному примірнику аж і двічі: один раз від імені своєї фірми, а другий - як Ті- ! мів опікун. На жаль. Тім не звернув на це уваги.

- Ну, випиймо за маргарин «Тім Талер», Тіме Талере! - барон дістав із маленького буфетика дві кришталеві чарочки й налив у них рому. Вони цокнулися. Хлопець сам не знав, чи п'є він за своє щастя, чи за нещастя. А втім, то ж був ром стерничого Джонні, й Тімові це здалося доброю прикметою.

Барон сів і почав тлумачити, як вони збираються рекламувати маргарин «Тім Талер».

- Ми розповідатимемо людям, як маленький хлопчик із убогого завулка зворушив серце багатого барона, як той барон відписав йому все своє добро і як хлопець тоді подбав, щоб усі люди в бідних завулках могли мастити хліб дешевим і добрим маргарином.

- Але ж це все брехня! - обурився Тім.

- Ви говорите точнісінько як Салех-бей.- зітхнув Троч.- Реклама - це не брехня, а висвітлення.

- Висвітлення?!

Барон кивнув головою.

- Адже ж усі факти правдиві, пане Талер: ви справді виросли у вбогому завулку: барон відписав вам усе своє майно; і навіть марковий маргарин - це ваша ідея. Лишається тільки висвітлити ці факти як треба, і наша зворушлива казочка готова. Дуже добра реклама, конкуренти просто показяться. Але все це ви полиште на нас, пане Талер. Поговорімо про ваше фото.

- Про фото усміхненого хлопця?

- Атож. пане Талер. Я сам аматор мистецтва фотографії, хоч і досить скромний. І зроблю те фото власними руками. Усе вже підготовлено. [164]

Троч відгорнув завіску, що закривала, як гадав Тім. невеличку кухоньку. Але натомість він побачив там фотоапарат на штативі, а біля нього стілець. На бильці висів старенький хлоп'ячий светр. Та найдужче вразила Тіма велетенська, на всю стіну ніші, фотографія. То був знімок його колишнього завулка. Якраз посередині - двері до їхнього будинку. Все збігалося до найменших дрібничок. Тім навіть розгледів у стіні сусіднього будинку шпарку між цеглинами, де він колись сховав був п'ять марок. Йому аж запахло перцем, кмином та ганусом.

- Надягніть, будь ласка, оцього светра та станьте перед фотографією, пане Талер! - Троч тим часом обережно переніс триніг із фотоапаратом на середину павільйону.

Тім, немов уві сні, зробив усе, що просив Троч. У голові в нього товпилися картинки минулого: батько... мачуха... блідий Ервін... сусідка з будинку ліворуч, що ходила до мачухи пити чай із тістечками... праворуч крамничка пані Бебер... неділі... перегони... допит увечері... картатий пан... угода... Тім так схвилювався, що мусив на хвилинку сісти на стілець.

Троч довго морочився біля апарата. Нарешті влаштував усе; тоді навмисне недбало обсмикнув на Тімові светра, трохи розпатлав хлопцеві чуба й поставив його перед фотографією завулка, а сам відійшов до апарата.

- Дуже добре, пане Талер! Отак і стійте. А тепер прокажіть слідом за мною: «Позичаю свій сміх тільки на півгодини. Життям присягаюся, що поверну його».

- Позичаю свій сміх...- Голос Тімові урвався. Та барон допоміг йому:

- Кажіть частинами. Так буде легше. Отже: «Позичаю свій сміх...»

- Позичаю свій сміх...

- «Тільки на півгодини».

- ...Тільки на півгодини.

- «Життям присягаюся...»

- Життям присягаюся... [165]

- «Що поверну його».

- ...Що поверну його.

Тільки-но Тім вимовив останнє слово, як Троч моторно сховав голову під чорну хустку. Наче лялька в ляльковому театрі. Тімові нестерпно захотілося сміятись, і він... справді засміявся. Сміх той здіймався в ньому звідкись із живота, лоскотав у горлі, й Тім реготав, аж заходився, аж у боки йому кололо, аж сльози на очах повиступали. В павільйоні аж лящало від Тімового реготу, і стілець біля нього дрижав, ніби теж сміявся разом із хлопцем. Світ неначе знову став на своє місце. Тім Талер сміявся. Барон, схований під чорною хусткою, перечікував той сміх. Пальці його, що тримали спусковий тросик, тремтіли.

Насміявшись, Тім весело вишкірився й спитав:

- Оце та маргаринова усмішка, що ви хотіли, бароне?

Йому було легко, радісно на серці, хотілося щось витівати; барон і досі нагадував йому ляльку на ниточках. [166] Тім не вірив, що це лиш на півгодини, він був певен. що сміх повернувся назавжди. А отого Троча під чорною хусткою, барона, що не мав сміху, йому було майже шкода. Навіть той здушений голос, що ним Троч наказував Тімові, як повернутися та як стати, будив у ньому скоріше співчуття, ніж насмішку. Хлопець слухняно виставив уперед праву ногу, нахилив голову трохи набік, усміхнувся, на Трочеве прохання вимовив слово «кишмиш», тоді знову приставив праву ногу до лівої - і аж засміявся з полегкості, коли на фотоапараті сяйнула лампа-спалахівка.

- Сподіваюся, знімок вийшов гарно, бароне! - Тім потягся солодко, втомившися стояти нерухомо, і оскірився весело в об'єктив фотоапарата. Але Троч не скидав із себе чорної хустки. Він пояснив з-під неї, що на один знімок покладатися не можна, треба зробити ще принаймні три.

- І все це задля дрібочки маргарину? - засміявся Тім. Але комизитись не став, дав себе поставити як треба й сфотографувати свою усмішку ще тричі.

Після четвертого, останнього, знімка Тім був уже такий стомлений і замлілий від позування, що йому здавалося, ніби він простояв не менш як годину. Він і гадки не мав, що з тієї півгодини, на яку він позичив сміх, минуло всього дві хвилини. І ще невтямки було йому, чого це барон і далі ховається під своєю хусткою. Тому він підійшов ближче, відкинув ту хустку й сміючись спитав:

- Може. ви там уже потихеньку маргарин робите. бароне?

Та сміх завмер на його устах, коли на нього знизу вгору глянуло зле тонкогубе обличчя в темних окулярах - обличчя картатого пана з іподрому!

Тім зрозумів, що то його отуманив власний сміх. Цей чоловік не віддасть йому того сміху по-доброму! Цей чоловік страшний.

Та сміх іще раз підманив хлопця, бо аж підпирав йому під горло, і Тім глузливо вигукнув: [167]

- Годі вам гратися в чорта, бароне! Дограли ви свою гру! Більше ви мене не побачите!

І метнувся до скляних дверей. Розчинив їх і, як був у старому светрі, вибіг під рясний дощ на паркову терасу.

Хоч барон і не гнався за ним. Тім помчав мов шалений вузенькою алейкою поміж високими живоплотами з підстриженого тису. Ця алейка вивела його в другу, та - в третю, в четверту, і так без кінця.

Тім звертав то праворуч, то ліворуч, упирався враз у густу непроглядну зелену стіну, вертавсь назад, знову опинявся у глухому кутку, ще раз вертався, витирав руками дощову воду з очей - і врешті зовсім заблудився в цих чудернацьких алейках, що мали вхід, та. здавалось, не мали виходу.

Раптом хлопець якось обважнів, немов руки й ноги йому налилися оловом. Він просто фізично відчував, як сміх залишає його. Стояв, мокрий, між мокрими зеленими стінами своєї в'язниці, мов скутий. А дощ ляпотів по калюжах біля його ніг. Сама вода, патьоки, краплі, плюскіт - великий нескінченний плач довкола, а посередині - зовсім маленький Тім із своїм поважним, сумним обличчям. Та враз сміх його вернувся знову - знайомий дзвінкий сміх із кумедним «ік!» на кінці. Тім і сам не знав: чи то засміявся він, чи сміх ховається в тисових живоплотах?

Та все було куди простіше: за Тімовою спиною стояв Троч.

- Ви попали в так званий лабіринт, пане Талер. Ходімо, я вас виведу.

Тім покірливо дав баронові руку, в павільйоні покірливо дозволив витерти себе й перевдягти, а тоді покірливо пішов до замку зі слугою, що ніс над ним парасольку.

Тільки в своїй кімнаті він помалу опам'ятався. Та цього разу сльози не дали йому полегкості. Цього разу його пойняла холодна лють. Ухопивши червоний келих на тонкій ніжці, що стояв на полиці, він роздушив його з такою злістю, що порізав до крові руки. Кинувши [168] скалки просто додолу, він смикнув за гаптований пас від дзвоника, а коли прийшов слуга - мовчки показав скривавленою рукою на червоні скалки. Слуга прибрав їх, обмив і перев'язав Тімові руку. а тоді вперше за всі дні промовив чотири слова:

- Моя не шпиг, пане!

- А я звідки знаю? Може, не шпиг, а може, й шпиг,- відказав Тім.- Та спасибі вам і за вашу приязнь.

Увійшов Салех-бей і вислав слугу з кімнати. Тоді втупив очі в Тімову руку:

- Ти не підписав угоди? Щось трапилось?

- Нічого страшного, Салех-бею. Угоду я підписав.

- А де ручка?

- Отут у кишені. Візьміть її, будь ласка, самі.

Старий дістав ручку з кишені, й Тім спитав:

- А що це за ручка?

- В ній таке чорнило, що помалу вицвітає, аж поки зникне зовсім. І коли наше товариство за рік оголосить випуск на ринок маргарину «Тім Талер», на угоді, що лежатиме в сейфі, твого підпису не буде. І ти тоді можеш не дозволити продавати маргарин. Але зроби це аж тоді, як про маргарин стане відомо на весь світ.

- І товариство тоді полетить димом догори?

Старий засміявся.

- Ні, синку, не полетить, для цього воно все ж надто тривке. Але воно зазнає величезних збитків. Поки підготують нову марку, конкуренти ж не спатимуть. Згодом наше товариство однаково матиме величезний зиск із маркового маргарину, але цілком завоювати ринок уже не зможе ніколи.

Салех-бей сів на канапі біля вікна й задививсь надвір, у дощ. Не повертаючи голови, він промовив:

- Я не знаю, чи пощастить коли нам із тобою перехитрити барона. Він хитріший за нас обох разом. І все ж я спробую допомогти тобі. Барон, здається, відучив тебе сміятись; а мені б хотілося, щоб ти навчився знову. [159]

Зляканий Тім хотів був щось відказати, та Салех-бей відмахнувся:

- Краще мовчи, синку. Але й не покладай великих надій на мою поміч. Сміх, Тіме, це не крам на продаж, як, скажімо, маргарин. Хто ним торгує, той чинить необачно. Сміху не можна виторгувати. Його можна тільки заслужити.

Задзеленчав телефон. Глянувши на перев'язану Тімову руку, Салех-бей підійшов до столика, взяв трубку, відповів, послухав, тоді прикрив мікрофон долонею й сказав:

- Якийсь чоловік із Гамбурга хоче поговорити з тобою. Каже, дуже важлива справа.

Тім зміркував умить: він же пообіцяв Трочеві цілий рік не шукати зв'язку зі своїми гамбурзькими друзями, і коли не дотримає обіцянки, з паном Рікертом може статися щось лихе. Отже, Тім повинен поки що зректися свого давнього друга задля його ж добра. Тому хлопець приклав пальця до вуст, і Салех-бей мовив у телефон:

- Пан Талер уже поїхав.

І якось нерішуче поклав трубку.

Скоро він пішов, а хлопець став біля вікна й довго дивився надвір. За вікном невпинно, рівно лив дощ.

Тім думав про те, що за рік стане власником гамбурзького пароплавства і подарує його панові Рікертові: за рік Тімів зниклий підпис на угоді наробить розруху в маргариновому царстві; за рік він побачить Крешимира й Джонні, пана Рікерта і його матусю; за рік...

Хлопець боявся навіть мріяти про своє можливе щастя. Але він мав надію на нього. Він мав також надію, що відбуде ту цілорічну подорож довколо світу вдвох із Трочем спокійно й пристойно.

А надія підносить прапори волі. [170]

КНИГА ЧЕТВЕРТА

ПОВЕРНЕНИЙ СМІХ

Де людина сміється,

Там кінчається чортова влада

Пані Бебер

Двадцять сьомий аркуш

РІК У ПОЛЬОТІ

Рік подорожі був для Тіма роком у польоті. Той рік почався у невеличкому двомоторному літаку їхньої фірми, що на ньому хлопець із бароном відлетіли до Стамбула. Летячи над горами, Тім знову бачив унизу чоловіків і жінок, що гнали гірськими стежками своїх в'ючаків. Але ті люди не були вже такі чудні й чужі Тімові, як того першого разу, коли він їх побачив. Одіж їхня нагадувала вбрання Салех-бея та декого зі слуг у замку. І хоч Тім зовсім не знав тих людей, а проте вони йому подобалися. Він любив їх і співчував їм, так як і афганським гостріям. [171]

У Стамбулі Тім із бароном перебули тиждень. Барон водив Тіма по мечетях та музеях, і подорож уже майже тішила хлопця. Все, що скоїлося з ним у замку, на якийсь час відбило йому охоту ганятися за своїм сміхом. Та водночас хлопець плекав надію, що за рік усе стане інакше. Думка про Салех-бея та гамбурзьких друзів так заспокоювала його, що він по-справжньому вірив, ніби після тієї подорожі його сміх сам упаде йому до рук, немов стигле яблуко з яблуні.

Та ця надія таїла в собі й небезпеку: адже Тім міг облишити всякі спроби змінити свою долю й змиритися з таким сумним становищем.

Однак Тімова зовнішня байдужість мала й перевагу: вона заспокоювала барона. Троч справді почав думати, що Тім змирився зі своєю долею, й уже не так пильно наглядав за хлопцем.

Із тижня на тиждень, із місяця на місяць він дужче й дужче впевнювався, що Тімові Талерові дедалі більше подобається роль багатого спадкоємця і що хлопець уже не схоче поміняти життя нероби-мільярдера ні на що, навіть на свій веселий сміх.

І справді, під час тієї подорожі Тім згадував свій утрачений сміх рідше, ніж доти. У розкішних готелях біля дуже багатих людей просто на задніх лапках ходять. Коли директор такого готелю помітить, що його багатий пожилець не любить сміятись, вмить уже вся прислуга від старшого швейцара й до покоївки знатиме, що поблизу цього пана ніхто ніколи не сміється.

Але ж світ - навіть для дуже багатих людей - складається не з самих розкішних готелів. Навіть мільярдерам іноді потрібне буває свіже повітря. І завжди, виходячи погуляти, чи то сам, чи то вдвох із бароном, Тім помічав, що весь світ повний сміху.

Після Стамбула він іще раз побачив Афіни. З Афін поїхали до Рима, з Рима до Ріо-де-Жанейро, далі до Гонолулу, Сан-Франціско, Лос-Анджелеса, Чікаго, Нью-Йорка. Звідти - до Парижа, потім до Амстердама, [172] Копенгагена, Стокгольма, на Капрі, до Неаполя, на Тенеріф, у Каїр, у Кейптаун. Літали до Токіо, Гонконга, Сінгапура, Бомбея. Тім побачив московський Кремль, ленінградські мости, Варшаву й Прагу, Белград і Будапешт. І всюди, де приземлявся їхній літак, Тім чув на вулицях всесвітній сміх. Сміялися чистильники черевиків у Белграді й хлопці-газетярі в Ріо. Сміялися продавці орхідей у Гонолулу й продавщиці тюльпанів у Амстердамі. Усміхався стамбульський казаняр і багдадський водонос. На празьких мостах жартували й хихотіли так само, як і на ленінградських. А в токійському театрі плескали в долоні й реготали так самісінько, як у театрі на Бродвеї в Нью-Йорку. Сміх потрібен людині, як квітці сонце. Коли б сміх раптом вимер у всьому світі, людство обернулось би на велетенський зоопарк або на суспільство ангелів - поважне, велично-байдуже й нудне.

І Тімові, хоч який він був поважний на вигляд, дуже хотілось знову навчитися сміятись. Хоч зовні він здавався цілком задоволеним, але радий був би стати жебраком у Нью-Йорку, якби це дало йому змогу прилучити й свій голос до всесвітнього реготу.

Але Тімів сміх Тімові не належав. Поруч нього, часом за кілька кроків, інша людина сміялася його заливистим дитячим сміхом. І Тім - от яке лихо!- за той рік майже звик до цього. Притерпівся й дбав зовсім про інше. Він учився. Учився всього, що повинна вміти й знати так звана «світська людина». Він уже вмів вельми зграбно їсти омари, фазанів, кав'яр; умів ковтати устриці й відкорковувати пляшки з шампанським; найпоширеніші вислови чемності від «дуже радий познайомитись» до «це для мене велика честь» умів сказати тринадцятьма мовами; знав, кому й скільки треба давати «на чай» у всіх найрозкішніших готелях світу; вмів не готуючись виголошувати невеличкі промови й звертатися за всіма правилами до графів, герцогів та князів; знав, до яких шкарпеток яка краватка [173] пасує; знав, що по шостій годині вечора не годиться носити жовті черевики, а тільки чорні (афте сікс но браун, як сказав би англієць); дізнався, що, піднімаючи чашку з чаєм, не слід відставляти мізинець; мимохідь, без словників, навчився трохи говорити по-французькому, по-англійському й по-італійському; навчився вдавати зацікавленого, коли насправді тобі нудно; навчився грати в теніс, кермувати яхтою та автомобілем і навіть лагодити мотор; і взагалі навчився прикидатись так добре, що барон був просто захоплений.

Хоч барон майже в кожному місті вів якісь таємні переговори про маргарин, але Тіма на них не водив, дозволяв хлопцеві робити, що йому до вподоби. Лиш кілька разів довелось Тімові сходити з Трочем на банкети - або дуже визначні (тоді репортери фотографували їх), або ж потрібні для фірми. На таких банкетах Тім познайомився з одним англійським герцогом, що захищав афганських гостріїв, і з одним аргентінським фабрикантом м'ясних консервів, що боронив привілеї англійського дворянства.

Щодо питання про панів і слуг - а Тіма в його новому становищі воно дуже цікавило - в усьому світі, видимо, панувала велика плутанина. Найкращу відповідь на нього почув Тім від однієї перекладачки в Москві.

Катерина Павлівна - так звали ту дівчину - обідала з бароном і Тімом у готелі «Москва». Нахилившись над своїм смаженим курчам, барон глузливо зауважив:

- Ви, комуністи, вірите у вселюдську рівність, Катерино Павлівно. Але ж це безглуздя.

Катерина Павлівна всміхнулась і показала на смажене курча:

- Якби оцей півник був іще живий, я б ніколи не стала вимагати від нього, щоб він ніс яйця. Нехай би він собі правив курником, бо на це тільки він і здатен!

Відповідь була й розумна, й дотепна. Барон гучно зареготав і вигукнув: [174]

- Отже, ви вірите в панів від природи, Катерино Павлівно?

- Трошечки вірю, бароне. По-моєму, буває в людей від природи своєрідний хист до керування, командування, організаторства чи як хочете називайте. Однак я не вірю, що цим хистом наділені тільки королі, князі та багаті спадкоємці. Такі таланти можуть вирости і в моєму передмісті, й навпаки, в багатьох пишних палацах вони не виростають. Ось цей юнак, наприклад,- Катерина Павлівна показала на Тіма,- колись має правити вашим королівством пароплавів, родзинок та масла. Але я боюся, що він занадто великодушний для такої справи!

- Може, й так,- буркнув Троч і урвав розмову. Але Тімові вона ще довго не йшла з думки. Тім зовсім не образився на Катерину Павлівну, бо знав, що вона каже правду: адже він був хлопець розумний і не чванько. (А до того ж уже доходив віку, коли людина починає пізнавати сама себе.)

За календарем Тім під час своєї подорожі постаршав на рік. Йому скоро мало сповнитися шістнадцять. Але подорослішав він років на п'ять чи шість. І підріс дуже, та й обличчям виглядав уже на двадцятирічного.

За той рік Тім майже три рази облетів у літаку довкола світу. Але не тільки в літаку, а й у житті він весь час був високо вгорі, ніби на вершинах, де й дихати легше, й видно далі, ніж у долині.

Сказати, що Тім почував себе кепсько на тій захмарній височині,- це була б неправда. Адже жилося там легко. (Надто людині, що не вміє сміятися.) А крім того, кмітливий хлопець міг там багато чого дізнатися й навчитись.

Але крамничка пані Бебер, де пахло хлібом та свіжими млинцями, цей невеликий копійчаний світ балакучих сусідок, ця коробочка з плітками, вимощена по стінах присмаглими хлібинами, була йому незрівнянно миліша за готель «Пальмаро» чи месопотамський замок. [175]

Треба зазначити, що барон чомусь обминав Тімове рідне місто десятою дорогою.

Кілька разів Тім просив його заїхати туди, але Троч ні разу не відмовивши Тімові навпростець, удавав, ніби не почув, або ж вигадував якісь невідкладні наради в інших містах.

Наприкінці того року подорожування Тімові стало страшенно важко удавати з себе спокійного та грати перед бароном роль задоволеного багатого спадкоємця. І що ближче надходив його день народження, то неспокійніший він ставав. Коли Троч тепер сміявся при Тімові, хлопця аж трусило. Якось уночі в одному брюссельскому готелі йому приснилась та його нескінчена телефонна розмова з паном Рікертом у Трочевому замку. Як він прокинувся, вона ще лунала йому в голові, й він виразно пригадав панові Рікертові слова: «Крешимир знає...»

Що знає Крсшимир? Як відібрати в барона Тімів сміх? Хлопець чесно дотримувався обіцянки не шукати ніякого зв'язку зі своїми гамбурзькими друзями. Але йому несила було діждатись кінця того року, коли термін обіцянки мине.

За кілька днів перед Тімовими іменинами вони полетіли до Лондона, де Тім при баронові одержав із рук містера Пенні пакет акцій. То була більша частина гамбурзького пароплавства.

Містер Пенні тим часом уже довідався, що Троч прочитав на вимочці його таємну умову з Тімом Талером.

Спершу англійцеві було дуже ніяково, проте скоро він заспокоївся. Адже ж однаково барон про все дізнається, коли надійде час передавати акції.

В літаку, що нарешті мчав Тіма до Гамбурга, хлопець спитав барона:

-А ви розмовляли з містером Пенні так само приязно й чемно, як і перше. Хіба ви не сердитесь на нього за те, що він поза вашою спиною відкупив у мене контрольні акції? [176]

Троч лунко зареготав:

- Любий мій пане Талер, таж я на місці Пенні зробив би так самісінько! Чого ж я маю на нього сердитись? Боротьба за контрольні акції, що їх я тепер маю найбільше, провадиться потай весь час. Але через те ми ж не видряпуємо один одному очей! Ми як та левина сім'я: коли трапиться здобич, леви погризуться трошки, ділячи її, й найстаршому левові - цебто мені - дістанеться найбільша пайка. Та як тільки здобич поділено - ми знову одна сім'я, і ніхто нас не розцурає.

- І Салех-бей такий самий?- тихо спитав хлопець.

- Салех-бей - це, можливо, виняток, пане Талер,- спокійно відповів Троч.- Він вважає себе за неймовірно хитрого, а насправді він дуже наївний. Часом нас це дратує, але здебільшого смішить. І ми, можна сказати, навіть любимо його.

- Але ж армія в Південній Америці...- мимохіть прохопився Тім.

- Та так звана армія, пане Талер, складається наполовину з наших людей. А зброя, що Салех-бей купує для них за свої власні гроші, походить із наших складів. Таким чином, Салех-беєві гроші повертаються до нашої фірми. Виходить наче кругообіг води в природі. І ті гроші, що їх Салех-бей витрачає проти нас у Афганістані, теж майже всі повертаються в нашу касу.

- А навіщо тоді ви прийняли Салех-бея до вашої фірми? Тільки тому, що він однаково добре вміє ладнати з буддійцями й магометанами?

-Не тільки тому, пане Талер. Його дуже шанують у всьому світі. Одні - за те, що він захищає бідних і пригноблених, другі - за те, що він старшина релігійної секти й сам побожний чоловік. А я, наприклад, шаную його за надзвичайно мудрі погляди щодо чорта,- усміхаючись пояснив Троч.

- Ну, а як там марковий маргарин?- нібито зовсім не до речі спитав Тім. Але барон дуже добре [178] зрозумів його.

- Салех-беєва спроба поламати наші маргаринові плани - така сама наївна вигадка.

Тімове серце забилося швидше. Невже Троч знає, що Тім підписав угоду Салех-беєвим невидимим чорнилом? Спитати про те хлопець не наважився. Та барон однаково відповів на його думку:

- В тій ручці, пане Талер, що нею ви підписалися, було, певна річ, звичайнісіньке чорнило. Один із слуг у Салех-беєвому домі - моя людина. Він вчасно замінив чорнило. Та коли б ваш підпис навіть зник з угоди, однаково лишився б підпис вашого опікуна. Бо я ж підписав кожного примірника двічі: раз за товариство, а другий - як ваш опікун.

Тім не сказав нічого. Він дививсь у віконечко вниз, на землю. Там удалині вже видно було високі будинки - певне, гамбурзькі. Йому так хотілось опинитись там, у місті, десь на вулиці звичайним, нікому не відомим хлопцем. Велика комерція була понад його силу.

Тім знав, що зможе повернути собі сміх, тільки спустившись із захмарної височини додолу. Він думав про Джонні, Крешимира, пана Рікерта. Післязавтра він матиме право зустрітися з ними...

Коли вони ще в Гамбурзі. І коли вони ще живі.

Двадцять восьмий аркуш

ЗУСТРІЧ БЕЗ ОБІЙМІВ

Коли вони з Трочем де-небудь виходили зі свого літака, барон звичайно пропускав хлопця вперед, бо на аеродромі їх здебільшого чекали репортери. Але тут, на гамбурзькому аеродромі Фюльсбютель, барон вийшов із літака перший. І ніхто на них не чекав: ні газетярі, ні фотографи. Навіть директор якого-небудь їхнього підприємства не зустрів їх. Однак фірма привітала їх величезним плакатом на стіні митниці: [179]

ПАЛЬМАРО

Перший у світі марковий маргарин

Дешевий, як маргарин, і смачний,

як масло

Для смаження, в печиво, в страву

й на бутерброди

Тім подивився спершу на плакат, потім на барона.

- Вас дивує назва маргарину, пане Талер? Розумієте, за цей час ми з'ясували, що назва «Тім Талер» не скрізь буде зручна. Багатьма мовами ваше ім'я важко написати. Крім того, в Африці краще сприйняли б плакати з усміхненим обличчям чорного хлопця, а не білого. Та й зворушлива казочка про бідного хлопчика не завжди доречна, бо ж наш маргарин купуватимуть не самі бідняки.

Тим часом вони пройшли митницю, де на Тімові й Трочеві валізи без довгої мороки поставили крейдою хрестики.

Надворі барон підкликав таксі. Тім здивувався: їх не чекала машина їхньої фірми. А коли автомобіль рушив, хлопець побачив у дзеркальце одного з генуезьких детективів, що марно оглядався довкола, шукаючи таксі й для себе.

У машині Троч повів далі:

- Ми назвали наш маргарин «Пальмаро», бо ця назва легко вимовляється майже всіма мовами світу. Та й пальму знають усі. На півночі про неї мріють, на півдні вона росте в кожного біля порога.

- То виходить, Салех-беєва ручка в кожному разі була марна, бароне?

Троч кивнув головою. Потім нахилився до шофера й звелів:

- Через центр не їдьте.

Шофер теж мовчки кивнув головою. Барон знову відхилився назад і спитав:

- А що ви хочете зробити з акціями гамбурзького пароплавства, пане Талер? [180]

- Я подарую пароплавство панові Рікертові, бароне.- І Тім, силкуючись говорити спокійно та холодно, пояснив:- Тоді мене більше не гризтиме сумління, що через мене він утратив посаду.

Шофер, видно, черкнув колесом узбіччя тротуару, бо машину труснуло.

- Правуйте уважніше, стонадцять чортів!- вигукнув обурено Троч.

- Вибачте,- буркнув шофер. І Тімові раптом здалося, що він уже десь чув той голос. Він скоса поглянув у дзеркальце на шоферове обличчя, але його майже зовсім закривали великі вуса, темні окуляри та насунутий низько на лоба кашкет.

Поруч хлопця раптом пролунав заливистий сміх і урвався на кумедному «ік!»

- Ви й досі не маєте правдивого уявлення про наше товариство,- сказав барон.- Ви не можете просто так ні сіло ні впало подарувати наше гамбурзьке пароплавство панові Рікертові, пане Талер.

- А чому?

- Бо ви з вашими акціями тільки так званий пайовик без голосу. Правда, більша частина чистого прибутку від пароплавства йде до вашої кишені: але всі справи й надалі вирішуватиме керівна рада товариства, власники контрольних акцій, цебто я, містер Пенні, сеньйор ван дер Толен і Салех-бей.

- Отже, як пан Рікерт повернеться на директорську посаду, ви могтимете будь-коли звільнити його знову?

- Авжеж.

Водій зменшив швидкість, закашлявся: він, видно, був застуджений.

Тім замислено-замислено відвернувся до вікна. Машина їхала тихою, гарною вулицею понад Альстером. Але Тім нічого не бачив...

- Слухайте, бароне...

- Що, пане Талер?

- Вам дуже потрібні ті акції пароплавства?

Троч допитливо глянув на Тіма. Та хлопець і бровою [181] не моргнув. Спереду вже долинало гудіння моторів на пожвавленішій вулиці.

Нарешті барон відповів начебто недбало - а недбалістю, як Тім добре знав, Троч приховував своє хвилювання:

- Це пароплавство - невеличка коштовна перлина, що її ще бракує в короні мого морського королівства. Сама собою вона не дуже й цінна, але, як то кажуть, було б гарне заокруглення.

Коли барон отак пересипав свою мову непотрібними прикрасами, це свідчило, що він дуже зацікавлений справою. Тімові це було відомо, й тому він не сказав нічого, а став чекати дальшого запитання. І дочекався:

- А що ви хочете за акції, пане Талер?

Відповідь у Тіма була готова давно. Проте він удав, ніби ще обмірковує її. Аж за хвилину відказав:

-• Дайте мені за них невелике, але солідне пароплавство в Гамбурзі, що не належить вашому товариству.

- Невже ви хочете конкурувати зі мною, пане Талер? Ви ж самі собі шкодитимете.

- Ні, я маю на увазі таке пароплавство, що до нього нашій фірмі нема ніякого діла. Прибережне абощо.

Барон нахилився до шофера:

- Яке, по-вашому, пароплавство каботажної плавби в Гамбурзі найприбутковіше?

Шофер замислився на хвильку, потім відповів:

- ПГГ, «Пасажирська лінія Гамбург - Гельголанд». Шість пароплавів ходять весь рік. Власники - сім'я Денкерів.

- А де можна знайти пана Денкера?

Шофер поглянув на годинника.

- Зараз пан Денкер у своїй конторі. В порту, на шостому причалі.

- Везіть нас до шостого причалу й почекайте нас там. Заплатити вам зараз?

- Не треба,- буркнув шофер, і Тімові голос його знову здався якимсь знайомим. [182]

Недалеко від порту таксі затрималося перед світлофором. Перед Тімовими очима на сизо-блакитному вересневому небі чітким рисунком із прямовисних ліній чорніли щогли та крани. Хоч шибки в машині були всі попіднімані, однак Тімові виразно почулися портові запахи солі, смоли, трухлого дерева.

Ті зроджені уявою запахи розбудили цілий натовп спогадів. У цьому порту він напав на баронів слід, у цьому порту почалася його гонитва за своїм сміхом, гонитва непрохідними хащами, марне полювання.

І ось Тім повернувся до вихідної точки. Те, чого не спромігся вполювати сам, він сподівався тепер добути з допомогою своїх друзів.

Один кран підняв високо в повітря великий ящик із намальованою на ньому пальмою. Хлопець тільки ковзнув по тій пальмі поглядом: він пас очима людей на тротуарах, сподіваючись побачити серед них Джонні, Крешимира або пана Рікерта. Адже в його пам'яті їхні образи зливалися з цією картиною - щоглами, кранами, з оцим лісом, над яким цвіли вимпели. Але нікого зі своїх друзів Тім не побачив. Він не знав навіть, чи знайде їх взагалі. На серці в нього було важко. І коли таксі рушило, самий лиш рух дав хлопцеві якусь полегкість.

Барон теж мовчки розглядав порт, поки машина стояла перед світлофором. Але він ні про що не мріяв, а пильно дивився на великий ящик із намальованою пальмою. Він знав, що то вивантажують маргарин «Пальмаро».

Далі думки обох пасажирів таксі перескочили на пароплавство, що його вони збиралися купити, на «Пасажирську лінію Гамбург - Гельголанд». Трочеву думку можна було передати двома словами: вигідна справа.

Тімові думки й почуття були складніші. Пригніченість пом'якшувала йому надія, а надію тіснив таємний страх. Самому йому те пароплавство було не потрібне: його цікавило в усьому світі лиш одне: його сміх, його воля. Але він повинен був якось вистояти в тому паперовому [183] бою за багатство, що його називають комерційною справою. Коли сам він з усього свого багатства не зможе взяти нічого в нове життя, то нехай хоч його друзі матимуть якусь користь. Це пароплавство буде знаком Тімової вдячності - коли він справді одержить те, за що хоче дякувати їм.

Таксі зупинилось біля шостого причалу. Троч із Тімом вийшли з нього й пішли до контори ПГГ, де старий пан Денкер, власник пароплавства, на превелике їхнє здивування, трохи не кинувся їх обнімати.

- Слово честі, дивний збіг, панове,- сказав він.- Я оце саме сушу собі голову, як продати моє пароплавство, а тут як із неба падаєте ви й хочете купити його. Диво дивне, справді!

Пан Денкер, напевне, так не дивувався б, якби він упізнав шофера таксі, що чекав біля причалу на Тіма й барона. Та, на щастя, пан Денкер того шофера не бачив. А хоч би й побачив, то навряд чи впізнав би його - так само, як і Тім.

А той шофер, сидячи в машині, обережними пальцями обмацував свої вуса й час від часу поглядав у дзеркальце назад. Там, метрів за сто від його машини, зупинилося ще одне таксі, але пасажир так і лишився сидіти в машині.

Десь за годину барон із Тімом вийшли з панової Денкерової контори, випивши там по три чарочки й уклавши так звану попередню угоду. Остаточна угода мала бути укладена другого дня.

Шофер у таксі вдавав, ніби спить. Троч, що був у доброму гуморі, сам відчинив собі дверцята. Тім сів у машину з другого боку. Аж тоді шофер буцімто прокинувся. Він дуже добре вдав несподівано розбудженого. Коли барон сказав йому їхати до готелю «Зима й літо», він навіть заникався цілком природно, відповідаючи.

- Ви, мабуть, знали, що це пароплавство продається?- спитав його Троч, як машина виїхала з порту. [184]

- Ні, не знав,- відказав шофер.- Але диво не велике. Пан Денкер уже не молоденький, підупав на силі, а дочки його, певне, вимагають своєї частки грішми. Не хочуть морочитись з пароплавством, це для них надто брудне діло. А вас, дозвольте спитати, зацікавило ПГГ?

Барон, іще в чудовому гуморі, відповів:

- Воно вже моє.

- Грім побий, швиденько ж ви впорались! Це як у казці «Липкохвіст», коли знаєте: тільки доторкнись,- і вже прилип.

Швидкий шоферів погляд у дзеркальці ковзнув по Тімовому обличчі. По тих шоферових словах воно зразу ж пересмикнулось, а тоді застигло, мов кам'яне. Як часто хлопець ховав за тією кам'яною міною страшенну схвильованість!

Хвилювання те було зрозуміле: шофер нарешті розкрив себе перед Тімом. Розкрив натяком, зовсім невинним для баронових вух,- на казку «Липкохвіст», що в ній королівна навчилася сміятись. Того знаку Тім потай дожидав весь час - знаку, що його друзі пильнують.

«Липкохвіст»! Перший сигнал до початку ловів.

Тепер Тім точно знав, хто такий цей шофер. Щось підкотилося хлопцеві з живота до горла - не сміх, а те, що не дає вимовити слова. Те, що звичайно називають клубком у горлі.

Таксі тим часом звернуло до Альстеру й зупинилося перед готелем. Водій вийшов і відчинив дверцята. Він уперше показався на весь свій неабиякий зріст. Тепер у Тіма вже не лишилось ніякого сумніву, хто це такий.

Коли барон розплатився й рушив до готельних дверей, Тім насилу втримався, щоб не кинутись тому здоровилові на шию. Захриплий від хвилювання, шепнув він: «Джонні!»

Шофер скинув темні окуляри, глянув хлопцеві в вічі й голосно сказав:

- До побачення, паничу!- І подав Тімові руку. Потім [185] знову начепив окуляри, сів у свою машину й поїхав.

А Тім відчув у своїй жмені невеличкий папірець, манюсіньку записочку. Та хоч який папірець був нікчемний, а з ним хлопець ураз відчув себе багатшим, ніж з усіма акціями товариства барона Троча - навіть із контрольними.

Майже щасливий, рушив він за Трочем до готелю. Там, у вестибюлі вже йшов їм назустріч директор, широко розгорнувши руки.

«Пане бароне, яка честь!»- ніби промовляли ті руки. Та перше ніж директор устиг виповісти свою радість і словами, барон приклав пальця до вуст.

- Будь ласка, без галасу. Ми приїхали інкогніто. Містер Браун із сином, лондонські комерсанти.

Директорові руки зразу опали. Він чемно й стримано вклонився.

- Ез ю лайк іт, місте Браун! Й о бегідж із олреді гіе!- цебто: «Як вам завгодно, містере Браун! Ваші речі вже прибули».

Уся та сценка видалась Тімові надзвичайно кумедною.

Він ладен був навіть директора обняти - так ураз змінила для нього світ ота невеличка записочка.

Але Тім нікого не обняв і не засміявся. Та й як би він міг засміятись? Він тільки промовив спокійно й чемно, як уже звик за довгі невеселі роки:

- Сенк'ю вері мач! («Дуже дякую!»)

Двадцять дев'ятий аркуш

Забуті обличчя

Подорожуючи довкола світу, Тім помалу звик, що їх весь час охороняють детективи. Ті детективи робили своє діло скромно й непомітно. Кілька разів хлопцеві впадали в очі двоє знайомих добродіїв із Генуї, але не потривожили вони його більше ні разу, бо ж він під час подорожі удавав слухняного Трочевого супутника. Але тепер, коли в Тіма в кишені лежала коштовна записка, йому за кожною складкою портьєри ввижався [186] детектив. Він не міг наважитись вийняти з кишені ту записочку й прочитати. Та й маскування Джонні свідчило, що за Тімовими друзями стежать так само, як і за ним.

Нарешті, коли барон ліг на часинку спочити, хлопець зайшов до ванної кімнати при своєму номері, засунув двері на засув, сів на край обличкованої блакитними кахлями ванни й вийняв записочку з кишені.

Папірець був не більший, ніж складені докупи чотири поштові марки. Один бік його списано дрібнесенькими літерами. Такого дрібного письма хлопець не міг прочитати простим оком. Треба було десь узяти збільшувальне скло.

Але де його взяти? Ховаючи записочку в кишеню, Тім міркував: коли він попросить кого-небудь із слуг у готелі добути лупу, детективи про це дізнаються. Коли він сам купить її в крамниці, детектив потім може спитати в продавця, що хлопець купив. То як же непомітно дістати лупу? -.-,,

Раптом Тім почув, що до номера хтось постукав, тоді відчинив двері і ввійшов. Подумавши, що то барон, хлопець задля обережності спустив воду, потім якомога тихіше відсунув засув і повернувся до вітальні.

Посеред неї стояв круглий стіл, а навколо столу четверо крісел. У кріслі якраз навпроти дверей, що ними ввійшов Тім, сиділа, нахилившись уперед, літня, густо нафарбована жінка в крикливо-барвистому, зовсім дівочому вбранні. Коси її, схожі на солому, були накручені кучериками. А поряд неї сидів блідий довготелесий парубійко, що мав замість нормальної краватки яскравого, надміру великого «метелика». Тімові зненацька запахло перцем, кмином та ганусом.

На мачуху й Ервіна ті гості були схожі зовсім мало. Але то були вони.

Тім безмовно застиг у дверях. Цих облич він не сподівався побачити. Упізнав він їх усього за одну мить, але давні, знайомі риси розгледів у тих обличчях [187] не зразу. І вперше в житті він побачив, які вони дурні, ті обличчя.

У вухах йому залунали батькові слова: «Бійся дурості, коли вона не добродушна...» І він уперше збагнув те, про що малим тільки здогадувався невиразно. Він зрозумів, що тільки завдяки таткові він не розучився сміятись іще зовсім малим.

Тімові набігли на очі сльози - не від зворушення, а ьід пильного погляду. Мачушине обличчя розпливлось, і натомість випливло інше - обличчя тієї, що дала йому сміх, обличчя його рідної матусі. Чорні кучері, блискучі чорні очі, світло-смаглява барва шкіри, ямочки в кутиках уст.

І ще одне збагнув Тім у ту хвилину: він так уподобав картини в генуезькому палаці через те, що несвідомо впізнав у них знайоме. З кожного італійського портрета дивилось на нього матусине обличчя. Кожен той портрет показував Тімові його минуле і - треба сподіватися - його майбутнє.

Мачуха, вгледівши Тіма, схопилась, подибала до нього й почепилася йому на шию. Тім, збентежений спогадами про матір, зопалу трохи не обняв мачуху. Але він був уже не той бідний малий хлопчик, що колись. Він навчився опановувати збентеження й не боятись незрозумілого. Тому він мовчки, лагідно, однак рішуче відпихнув мачуху від себе. І вона скорилася. Схлипнула, подріботіла назад до столу, де лежала її сумочка, видобула хусточку й почала вимочувати нею штучні вії.

Аж тоді підвівся й Ервін. Недбало підійшовши до зведеного брата, подав йому м'яку безсилу руку й промовив:

- Здоров, Тіме!

- Здоров, Ервіне!

Сказати більше вони не встигли, бо двері розчинились - вбіг задиханип барон.

- Що це за люди?

Звичайно, Троч здогадався зразу, що це за люди, [188] Тім це бачив. Однак він ґречно відрекомендував баронові своїх непроханих гостей:

- Дозвольте, бароне, познайомити вас із моєю мачухою, пані Талер. А це мій зведений брат Ервін.

Тоді підкреслено поважно, завченим красивим рухом показав мачусі на свого опікуна:

- Барон Троч.

Мачуха піднесла праву руку майже під носа Трочеві, видно, сподіваючись, що він поцілує їй ту руку, й прощебетала:

- Дуже приємно, пане бароне!

Однак Троч не звернув на руку ніякої уваги.

- Облишмо комедію, пані Талер. Вам же, здається, взагалі не щастить із театром.

Мачуха була вже роззявила рота, щоб ображено огризнутися, та похопилась і змінила тактику. Вона обернулась до Тіма, подивилась на нього з солодким виразом на своєму кислому виду, ступила крок назад і сказала:

- Аз тебе став справжній джентльмен, синочку! Я дуже пишаюся тобою. Ми ж про тебе все читали в газетах, правда, Ервіне?

Її син із видимою досадою гмукнув собі під носа: «Угу». Він, видно, ставився до матері так само, як і давніше. Мазаний, розпещений і водночас прикутий до неї,- бо сам він не вмів добувати собі в житті те, чого хотів,- Ервін при людях соромився своєї матері. Він визискував її тваринну любов, але не терпів її самої.

Тепер Тім був радий, що мачуха не любила і його так, як свого мазунчика. Бо така любов зламала б його силу, і він не зміг би витримати того пекла, що в ньому жив стільки років.

Тімові та зустріч була дуже доречна й корисна. Із свого вбогого завулка він піднявся звивистими шляхами на високу гору, в захмарну височінь, і тепер бачив початок свого шляху далеко-далеко внизу. І ще він побачив, що мачуха з Ервіном лишаються там, де й були, і не піднялись ні на крок вище. Хоч вони стояли [189] в нього перед очима, в його номері в готелі «Зима й літо», та були від нього такі далекі, що він насилу чув їхні голоси.

Мачуха якраз говорила:

- Ми тепер лишимось назавжди з тобою, будемо дбати про тебе. Ти ж законний спадкоємець усієї фірми, а завтра тобі вийде шістнадцять років, і ти будеш...

- Іще ні в якому разі не повнолітній!- перебив її барон.

Пані Талер рвучко повернула голову до Троча. В очах її спалахнув неприродний вогонь, що ото називають «гарячковим блиском». Тім добре пам'ятав той вогонь. Але тепер він нагадував хлопцеві вологий блиск великих коров'ячих очей, що їх ми колись у дитинстві так боялись і що тепер здаються нам трошечки дурні й зовсім-зовсім не страшні. «Нема гіршої біди, як дурість»,- подумав Тім.

А Троч, насилу стримуючи усмішку, тлумачив Тімовій мачусі, чому в шістнадцять років Тім іще не буде повнолітній.

- У вашій країні, пані Талер, хлопці досягають повноліття тільки в двадцять один рік, а отже, тільки тоді спадщина переходить у їхнє цілковите володіння. Ви, напевне, довідалися, що я сам - громадянин країни, де чоловік стає повнолітнім у шістнадцять років. Але вашого пасинка Тіма це зовсім не стосується. Він підлягає законам вашої країни. І одержить спадщину тільки в двадцять один рік.

Мачуха не перебила барона ані словом. Тільки повіки її ледь тремтіли та рука нервово м'яла хусточку. Аж ось вона обернулась до пасинка й спитала, насилу опанувавши своє хвилювання:

- А хіба ти не перейшов у баронове громадянство?

Тім дивився на неї зовсім байдуже і навіть не розчув того запитання, бо задумався про своє. Він тільки помітив, що вона щось сказала, і, аби не здатись нечемою, показав їй на стілець: [190]

- Сядьмо, чого ж говорити стоячи,

Всі мовчки посідали за стіл.

Тім заклав ногу за ногу й сказав:

_ Я ніколи не замислювався про те, хто ж тепер, власне, мій опікун. Коли барон...- він помовчав хвильку,- помер, було оголошено, що мій опікун - новий барон. Аж тепер мені спадає на думку, що на це потрібна була згода моєї мачухи. Ви давали ту згоду чи...

Пані Талер знітилася й промимрила:

- Ти знаєш, Тіме, ми в таку скруту попали, як ти від нас пішов... Нам дуже не щастило, і я...

- ...І пані Талер офіційно передала мені своє опікунство,- докінчив за неї Троч, звертаючись до Тіма.- За значну суму грошей, що її вона витратила на купівлю естрадного театру. А тепер той театр, видимо, збанкрутував.

- Але ж я не винна, просто часи тяжкі настали,- пані Талер схлипнула, а тоді заторохтіла без передиху, як колись: [191]

- Язнающонемаюніякихправ, алежвінмоядитина, амитеперопинилисьізсиномнавулиці, ія...

Цього разу її перебив Тім:

- Якщо ви продали своє опікунство, то боюся, що нічого не можна вдіяти.

- Продала! Продала! Небудьтакийжорстокийтіме! Менежскрутазаставила!

- А скільки грошей треба вам тепер?

- Хтожговоритьпрогро-ші? Мижтеперлишимосьізтобою, Тіме!

-Ні,- відповів хлопець.- Ви зі мною не лишитесь. Я сподіваюся, що ми оце бачимось востаннє. Але якщо я можу допомогти вам грішми, то радо це зроблю. Скільки вам потрібно?

- Тільки за моєю згодою,- докинув барон. Але Тім ніби й не почув його.

- Ой, Тіме!- мачуха знову роблено захлипала.- Ти ж тепер такий незмірне багатий, то ж не годиться твоїм родичам жити в злиднях!

Барон трохи не зареготав, але схаменувся й затулив долонею рот: йому вчасно спало на думку, що цим людям - матері й синові - знайомий його сміх. Отож треба подбати, щоб вони ніколи не зустрічалися з ним. А тому доведеться труснути капшуком. І він звернувся до жінки:

- У мене, пані Талер, є на Ямайці один пляжний заклад, досить прибутковий. Переважно для американських туристів. Річний обіг шістдесят тисяч доларів. Ви ж знаєте, що Ямайка - це острів вічної весни. Ваше бунгало стоїть у пальмовому гаю над морем.

«Ти диви, барон наче з путівника читає!- зчудовано подумав Тім.- Виходить, він і так уміє!»- Проте хлопець добре помітив, як Троч урвав свій сміх, і зрозумів, чому барон хоче здихатись цих людей. Тож Тім не здивувався навіть, коли Троч пообіцяв їм квиток першого класу на пароплав.

Уже знову (чи ще й досі), хлипаючи, мачуха промовила: [192]

- Ви такий добрий, пане бароне!

Ервінові аж очі загорілись на думку про Ямайку. Він закліпав ними точнісінько, як мати.

- Зайдіть, будь ласка, до мого номера, там ми зразу владнаємо всю справу,- запропонував барон, підвівся і з глузливою чемністю відчинив жінці двері.

Пані Талер подріботіла за ним, та вчасно згадала про пасинка, обернулась до нього й сказала:

- Ти ж нас не забудеш, Тіме?

- По-моєму, я вас уже забув,- відказав Тім півголосом. Тоді подав їй руку й сказав поважно:- Щасти вам на Ямайці!

_ Спасибі, спасибі, дитино моя!- вона кисло всміхнулась і побігла до дверей.

Ервін також подав Тімові руку й хотів був іти за матір'ю. Але Тім притримав його й шепнув:

- Добудь мені лупу й поклади під червоною лавкою над Альстером, навпроти готелю. Ось, на!- Він вийняв із кишені жмут грошей і тицьнув у руку зведеному братові.

Ервін подививсь на гроші й спитав:

- А це що за папірець?

- Ой, це мені ще потрібне!- трохи не закричав Тім. Та, на щастя, все ж не закричав, а тільки шепнув.

Записку він сховав назад у кишеню, й Ервін пішов, шепнувши на відповідь:

- Я ні мур-мур!

Тім кивнув головою й зачинив двері за зведеним братом і за своїм далеким минулим.

Тридцятий аркуш

СПРАВИ

Просто диво, як швидко багаті, впливові люди можуть відбувати всілякі формальності, що на них простій, як то кажуть, людині часом буває мало цілих місяців. [193] Видно, з захмарних висот легко приборкувати й бюрократизм.

Одна-єдина контора товариства барона Троча, філія так званого юридичного відділу, оформила для Тіма й барона такі справи:

Пляжний заклад на Ямайці було переписано на пані Талер та її сина Ервіна як рівноправних власників. (Отже, Тімові довелося побачити їх іще раз, але тільки на хвилинку, і Ервін шепнув йому, що лупа лежить під лавкою.) Пароплавство «Пасажирська лінія Гамбург - Гельголанд», чи скорочено ПГГ, з того самого дня перейшло у власність Тіма Талера. (Дотеперішній власник, старий пан Денкер, підписавши угоду, щиро стис Тімові руку, промовив: «Тьху, тьху, тьху»,- і тричі сплюнув йому через ліве плече.) А пакет акцій гамбурзького пароплавства, який Тім Талер недавно отримав у Лондоні з рук містера Пенні, перейшов до барона - також із того самого дня. (Річний термін відпадав, бо Троч був власником контрольних акцій.)

І нарешті малося оформити контракт про спадщину: Троч аж доти зволікав із тим контрактом, а Тім йому про нього ні разу не нагадував.

Чому барон раптом вирішив оформити той контракт, Тім не знав, та його те мало й цікавило. Світ великих справ став йому так само байдужий, як і велике багатство. Він знав єдину важливу справу - боротьбу за свій сміх. І здогадувався, що манюсінька записочка

в кишені у нього (на ніч він сховав її під подушку) - то ключик від замка, що ним замкнено його сміх. Тому хлопцеві не терпілося забрати з-під лави лупу. І він став раз у раз прикладати руку до лоба, вдаючи, що його зовсім зморили всі нескінченні формальності укладання трьох угод. Вони таки справді були стомливі.

- Коли у вас болить голова, відкладімо контракт про спадщину на завтра,- сказав, помітивши те, барон.- Добре, пане Талер? [194]

Тім погодився не зразу, він уже був надто хитрий для того. Навпаки, він відказав, що краще було б покінчити заразом і з цією справою, але, на жаль, у нього страшенно болить голова, а контракти годиться підписувати на свіжу голову: тому й справді, мабуть, краще відкласти те на завтра.

Тією хитрістю він добився, чого хотів. Читання й підписання контракту відклали на другий день, а Тімові, заставивши його проковтнути дві таблетки, ще й порадили погуляти над Альстером біля готелю. На свіжому повітрі все як рукою зніме, сказав один із адвокатів.

Знаючи, що десь поблизу пильнує його детектив, Тім не зразу й не відкрито забрав із-під червоної лави лупу. Він спершу купив собі газету й сів із нею на лаві. Де лежить лупа, він уже нагледів.

Читаючи газету, він ніби ненароком упустив із неї внутрішній аркуш. Той аркуш упав під лаву, а хлопець нахиливсь і разом із газетою підняв лупу. І, так само закрившись газетою, він укинув ту лупу в нагрудну кишеньку піджака (тепер Тім здебільш носив костюм із тонкого сірого сукна або в дрібнесеньку клітинку) .

За чверть години хлопець згорнув газету, поклав її на лавці - хай іще хтось почитає - й повернувся до готелю. Коли він брав унизу ключ від свого номера, швейцар дав йому згорненого папірця. То була коротенька записочка від барона.

«Якщо вам полегшало, зайдіть, будь ласка, до мого номера.

Троч».

Тім піднявся нагору до барона. Та спершу він зайшов до свого номера, поклав лупу в скриньці з аптечкою, що висіла на стіні над ванною, а записку Джонні сховав у напівпорожню скляну трубочку з таблетками від головного болю. Аж тоді він пішов до барона. [195]

Троч мав звичку під час важливих розмов тримати в руці папірця з нотатками. І цього разу Тім побачив у нього в руці такий папірець. На ньому стояло стовпчиком троє слів. Усіх їх хлопець не розібрав, але верхнє, безперечно, було прізвище Рікерт.

- Завтра, пане Талер,- почав барон,- кінчається термін нашої невеличкої умови щодо пана Рікерта. Коли ви до завтра не зустрінетесь із вашими гамбурзькими друзями, я знову прийму його на посаду директора пароплавства. А за своїм віком він зможе зразу ж піти на почесну пенсію. І пенсія буде досить щедра. Завтра нам, на жаль, треба летіти до Каїра, бо одна єгипетська фірма заявила претензію на назву «Пальмаро». Отже, коли ви хочете зустрітися з вашими гамбурзькими приятелями, то повинні зробити це сьогодні. Однак тоді нашої умови не буде дотримано, і панові Рікертові доведеться лишитись вантажником у порту.

- Вантажником?

- Так, пане Талер, вантажником. А йому скрутненько доводиться на цій роботі: літа не ті... Тому я сподіваюся, що ви визволите його з цього невеселого становища, відмовившись від зустрічі з ним, паном Крешиміром та паном Джонні. Чи як?..

Троч дивився на хлопця з майже боязкою увагою. І Тім знав чому: напевно, один із його друзів мав у руках ключик до його сміху, і барон, видно, про це здогадувався. (Того ж він тепер навіть не всміхнувся.)

- Нехай пан Рікерт буде знову директором пароплавства!- твердо сказав хлопець.

- Отже, ви дотримаєте нашої умови, пане Талер?

Хлопець кивнув головою. Але він дурив барона, бо й гадки не мав відмовитись від зустрічі з друзями. Навпаки, треба було побачити їх сьогодні ж, бо завтра буде запізно. Та однаково пан Рікерт мав стати директором пароплавства, тільки не баронового, а Тімового власного, що сьогодні вранці було переписане на Тіма,- пароплавства ПГГ. [196]

Троч видимо полегшено зиркнув у свої нотатки й сказав:

- Другий пункт, пане Талер, стосується...- він завагався, проте все ж вимовив тяжке слово:- Пункт другий стосується вашого сміху.

І знову допитливо зиркнув на Тіма. Але хлопець уже давно навчився ховати свої почуття за маскою холоднокровності. Навіть голос не зрадив його, коли він спитав:

- А що з моїм сміхом, бароне?

- Рік тому, пане Талер, я випробував у Червоному павільйоні мого замку, чи вас іще цікавить ваш сміх. Я вам позичив його на півгодини і з тієї невеличкої спроби дізнався, що ви тоді ще дуже сумували за своїм сміхом. А оце тепер я непомітно для вас знову влаштував вам невеличке випробування, ї цього разу наслідки його втішили мене. Ви добровільно відмовилися від зустрічі з єдиними трьома людьми, що знають про нашу угоду й могли б у разі потреби щось вам порадити.

Барон задоволене відхилився на спинку крісла.

- Видно, за останній рік ви навчилися цінувати владу, багатство та розкішне життя більше, ніж якийсь там нікчемний сміх.

Тім лиш головою кивнув. Цього разу він був тільки наполовину нещирий: йому справді подобалося бути завжди гарно вбраному, мати зручне помешкання, ванну, досхочу кишенькових грошей. Але не так дуже все те йому подобалося, щоб він погодився, задля нього довіку лишатись людиною без сміху.

- І ось тепер я пропоную 'вам,- Троч нахилився вперед,- одну додаткову угоду.

- Яку, бароне?

- А ось яку, пане Талер: я зобов'язуюсь виклопотати для вас підданство країни, де ви вже від сьогоднішнього дня вважатиметесь за повнолітнього й зможете негайно вступити у володіння спадщиною.

- А чого ж ви за те жадаєте від мене, бароне? [197]

- Двох речей, пане Талер: по-перше, ніколи не вимагати свого сміх:у назад, а по-друге - відс'і'упити мені половину всієї спадщини з контрольними акціями включно.

- Пропозиція варта уваги,- мовив Тім повільно, аби виграти час. Звичайно, він і не думав про те, щоб офіційно, за підписом і печаткою, зректися свого сміху. Але Трочеві цього не слід знати. Якраз тепер треба було замилити баронові очі, щоб легше одурити детективів та вислизнули на побачення з друзями. Записочка й лупа мали вказати хлопцеві дорогу до них.

Хлопцеві набігла добра думка: треба поторгуватися з бароном. Тоді Троч іще певніше повірить, ніби Тім остаточно зрікся свого сміху і визнав, що могутність та багатство важливіші за якесь там хихотіння.

І Тім почав торгуватись.

Він, мовляв, зна<Е;, що барон до Тімового двадцять першого року може зробити багато дечого, щоб не дати Тімові посісти спадщину. Сеньйор ван дер Толен уже перестеріг його. І тому він охоче підпише цю додаткову угоду, але вимагає за те три чверті всієї спадщини й три чверті контрольних акцій.

Хлопець добре завважив усмішку, що майнула при тих словах на Трочевому обличчі. Барон, очевидно, був уже зовсім певен, що Тім зречеться свого сміху. А хлопець ТОГО ЛИШ І ХОТІВ!.

Торгувались вони добрих півгодини, й Тім трохи знизив свої вимоги: три чверті спадщини й половину контрольних акцій.

- Приставайте на цю умову, бароне, і завтра в Каїрі ми підпишемо додаткову угоду.

- Ну, це діло треба', переспати. Хай ніч-мати дасть пораду, пане Талер. Заівтра в Каїрі я скажу вам свою остаточну ухвалу. А тепер...- барон усміхнувся.- Тепер мій останній пункт.- Він підвівся, подав Тімові руку й промовив:- Щиро вітаю вас із шістнадцятиріччям. Коли маєте яке-небудь бажання, пане Талер...

Бажання? Тім замислився. Коли цей день подарує [198] йому найкращий дарунок, його сміх, він, мабуть, не буде вже багатієм, бо ж своє пароплавство він збирається віддати друзям.

То що ж попросити собі в Троча?

Нарешті йому сяйнула добра думка:

- Купіть мені ляльковий театр, бароне!

- Ляльковий театр?

- Еге ж, бароне, ляльковий театр, щоб у ньому дітей смішити.

Ось коли Тім зрадив себе! Та барон зрозумів його по-своєму.

- А, розумію!- вигукнув він.- Ви хочете купити й собі невеличкий сміх, і театр вам потрібен, щоб було де вибрати! Непогана думка! Мені така ще ніколи й не набігала!

Тіма немов хто по голові вдарив. Виходить, барон щиро вірить, ніби він, Тім Талер, після всіх страхіть, що зазнав сам, іще здатен украсти сміх у малої дитини!

«Цей чоловік,- подумав хлопець,- напевно, сам чорт!»

Цього разу Троч не міг би не побачити хлопцевого збентеження. Та, на щастя, він саме відвернувся. Він уже телефонував комусь про ляльковий театр. Усього за півгодини їм пощастило: хазяїн одного театрика, що вже кілька років насилу зводив кінці з кінцями, погодився за пристойну суму продати його. Той театрик містився поблизу головного вокзалу.

- Їдьмо туди зразу, пане Талер,- запропонував барон.- Я візьму з собою нотаріуса й чек. Іменинні подарунки треба оплачувати зразу.

В невеличкій брудній кімнатці, що, мабуть, правила в театрику за контору, було підписано ще одну угоду. Тім Талер став власником лялькового театру. Все це здавалося ще казковіше, ніж вистава в такому театрі.

До готелю барон із хлопцем поверталися пішки. Дорогою Тім уперше за весь час спитав:

- А чому ви так уподобали мій сміх, бароне, що ладні віддати за нього півкоролівства? [199]

- Мені дуже дивно, що ви досі ні разу про це не спитали, пане Талер,- відказав Троч.- Відповісти на ваше запитання не так просто. В кількох словах я можу сказати ось що: коли ви ще були маленьким хлопчиком із убогого завулка, пане Талер, ви зуміли пронести свій сміх крізь стільки всіляких незбагненних прикрощів, що він загартувавсь і став міцний, як діамант. Вашого сміху не можна розбити, пане Талер!

- Але ж мене можна, бароне,- відповів Тім дуже поважно.

- Отож-бо,- сказав Троч. Та перше ніж юнак зрозумів огидний зміст того «отож-бо», вони ввійшли до готелю.

Директор привітав їх:

- Добривечір, містере Браун.

Барон неуважно кивнув йому головою.

Нагорі, перед дверима до Тімового номера, Троч спитав:

- А навіщо вам, власне, контрольні акції, пане Талер? Ви ж однаково за вашою умовою з містером Пенні муситимете віддати їх йому.

Тім подумав розпачливо: «Знову він про умови! Цілий день народження - самі справи та папери!» Хлопцеві не терпілося швидше взяти в руки невеличкий папірець у скляній трубочці, і йому нелегко було придумати, що відповісти. Але він усе ж здобувсь на відповідь:

- А може, мені якраз і треба, щоб у містера Пенні було більше контрольних акцій, ніж у вас, бароне!

- Гм...- замислено гмукнув барон.- У мене сьогодні ще кілька важливих нарад. А що робитимете ви?

Тім схопився за лоба.

- У мене знову голова розболілася, бароне. Я піду ляжу.

- Ідіть,- засміявся Троч.- Сон - найкращі ліки.- І пішов.

А Тім нетерпляче відімкнув двері, зайшов до вітальні, замкнув двері за собою й кинувся до ванної кімнати. [200]

Тридцять перший аркуш

ТАЄМНИЧА ЗАПИСКА

У ванній кімнаті Тім увімкнув лампочку над умивальником та дзеркалом. Потім добув із аптечки лупу й трубочку з таблетками від головного болю, де лежала записка.

Серце його так калатало, наче хотіло з грудей вискочити.

Перш ніж прочитати записку, панич у елегантному сірому костюмі з тонкого сукна ще раз пересвідчився, чи засунуті на засув двері. Тоді став біля умивальника, глянув крізь лупу на записочку і з панича зразу став - попри всі костюми та сукна - просто безмірно схвильованим підлітком.

Лупа в руці в нього тремтіла; і все ж Тім спромігся розібрати дрібнесеньке письмо. Дивуючись дедалі дужче, він читав:

«Туди, де липкохвіст, прийди. Те, що знайшла королівна, знайди. Шлях простіший, ніж ти гадаєш. Його тобі покаже той чоловік, що відраджував від вихора. Візьми таксі, де знайомий шофер. Він стереже дім радників. Приходь у годину трамваїв (темну!). Бійся щура, одури його. Шлях простий. Тільки веде на нього чорний хід. Вір нам і приходь!»

Тім опустив руку з запискою й сів на край ванни. Він іще тремтів від хвилювання, але в голові йому вже проясніло. Він розумів: записочку зашифровано на той випадок, якби вона попала в руки Трочеві. Тепер треба розшифрувати її.

Він знову став під лампочкою біля умивальника й повільно перечитав записку:

«Туди, де липкохвіст, прийди».

Ну, це зрозуміти неважко. Треба піти до лялькового театру, де він колись бачив виставу. До шиночка в Евельгене. [201]

«Те, що знайшла королівна, знайди».

Ці слова були ще зрозуміліші - найважливіші, найприємніші слова в усій записочці. Вони означали: поверни собі свій сміх. А що це можливе, свідчило дальше речення:

«Шлях простіший, ніж ти гадаєш».

Але що означає ось це?

«Його тобі покаже чоловік, що відраджував від вихора».

Тімові довелось покопатися в своїй пам'яті. Та все ж він пригадав: Вихор - то назвисько коня! Останнього коня, що виграв йому заклад на іподромі! А один чоловік, що його Тім тоді ще не знав, відраджував Тіма ставити на Вихора. Той чоловік був Крешимир!

Отже, Крешимир знає, як Тім може повернути собі свій сміх! Хлопець, правда, вже давно про це здогадувався. І однаково ця звістка приголомшила його. Йому довелося знову сісти на краєчок ванни.

Світла було досить і там, і Тім прочитав далі:

«Візьми таксі, де знайомий шофер. Він стереже дім радників».

Шофер - то Джонні, Тім збагнув зразу. Але «дім радників»?

Над цією досить немудрою загадкою Тім міркував довгенько, поки здогадався, що то просто міська рада, чи ратуша. Вона ж стоїть зовсім близько від готелю. Значить, там Джонні чекатиме на Тіма з машиною й відвезе його до Евельгене.

«Приходь у годину трамваїв (темну!)».

Його друзі знали про дві пригоди з трамваями: поперше, той трамвай, що повернув на вокзал, коли Тім забивсь об заклад із паном Рікертом, і, по-друге, летючий [202] рамвай у Генуї, що його Тім бачив разом із Джонні. Певне, в записці мались на увазі обидві пригоди, бо там було написано «трамваїв», а не «трамвая». «Година трамваїв»... Що ж це за година? Летючий трамвай вони бачили опівдні, цебто о дванадцятій годині. А як Тім уперше побачив пана Рікерта в трамваї, теж була дванадцята година.

Отже, о дванадцятій годині, опівдні. А тепер...- Тім глянув на годинника,- уже п'ята година. Виходить, треба прийти завтра? Чи, може, треба було сьогодні опівдні?..

Але ж тут написано ще слово «темну!» В дужках і зі знаком оклику. Що це за темна година опівдні?

І знову Тім розгадав ту досить просту загадку не зразу. Та врешті він збагнув: ідеться про темну дванадцяту годину, цебто про північ! А до півночі ще довгих сім годин.

Решту записки знов було дуже легко зрозуміти:

«Бійся щура, одури його. Шлях простий. Тільки веде на нього чорний хід. Вір нам і приходь!»

Отже, Тім повинен стерегтися Троча й вийти з готелю потай - може, навіть перевдягшись. Бо саме слово «чорний хід» нагадувало про пригоди з детективних романів, карколомні втечі, перевдягання, гримування.

Коли хлопець дочитав таємничу записочку, йому стало так легко, наче в нього виросли крила. Закортіло сміятись, і навдивовижу уста його не стислись, як звичайно. Навпаки, йому здалося, ніби він усміхається.

Зрадівши й злякавшись воднораз. Тім підхопився й глянув на себе в дзеркало. І справді, в кутиках вуст його назначились ямочки, як на італійських портретах у «Палаццо Кандідо» в Генуї. То був іще не сміх, по суті, навіть не усмішка; але ямочки в кутиках уст видно було виразно. А вони ж не з'являлися там від того самого дня, коли Тім підписав угоду під каштаном у садку. [203]

Виходить, щось змінилось уже сьогодні! Надія,'мов пензель маля.ра, виворожила йому на обличчі тінь усмішки.

Тім сховав записочку в кишеню, вимкнув світло, вийшов із ванної кімнати й, заклавши ногу за ногу, сів на крісло у вітальні - подумати.

Барон тим часом сидів недалеко від Тіма, в павільйоні над Альстером. Він розмовляв із представником тієї єгипетської фірми, що заявила претензію на марку «Пальмаро». Фірма вимагала, щоб Троч назвав свій маргарин якось інакше.

На баронові чогось не знати було того спокою й зверхності, що стали його другою натурою відтоді, як він заволодів Тімовим сміхом. Звичайно, треба було, щоб марковий маргарин під уже розрекламованою назвою якомога швидше завоював легіони покупців, від цього залежало дуже багато. Але ж баронові якраз не слід було показувати, як це його турбує. Треба вдавати спокійного, упевненого, треба всміхатися! Адже задля того він і купив собі сміх!

При слушній нагоді Троч засміявся, весело, дзвінко, не забувши й кумедного «ік!» на кінці, як належить. Але йому здалося, наче чогось тому сміхові бракує. Та й співрозмовника той сміх не звеселив, а скоріше якось збентежив.

Барон підвівся, вибачився й вийшов до вмивальні. Там він став перед дзеркалом і, усміхнувшись Тімовою усмішкою, почав пильно розглядати своє обличчя.

На перший погляд, усе лишалося, як і було. Та як придивитись пильніше - барон іще раз усміхнувся ї дзеркало,- то в кутиках уст бракувало гарненьких ямочок. І усмішка від того здавалася вимушена, штучна, ніби чужа, позичена.

Троча охопило почуття, вже кілька років невідоме йому: ляк! І вперше за ці роки він відчув щось схоже на докори сумління. Не тому, що він зробив щось погане - Троч не вмів розрізняти доброго й поганого. Просто він зрозумів, що впоров дурницю. [204]

Цей коштовний сміх вуличного хлопчака, іскристий, загартований, як діамант, цей дзвіночок із кумедним «ік!» на кінці треба було привласнити інакшим, простішим способом: не за допомогою крутійської угоди, по окремих пунктах; не за фокусницьку здатність вигравати заклади (все ота клята скнарість!), а...

До умивальні раптом зайшов Трочів співрозмовник і побачив бліде, аж перекривлене баронове обличчя. Він, певно, подумав, що то Троч так схвилювався через марку «Пальмаро», і Троч теж подумав, що єгипетський представник так подумав. Становище було збіса незручне. Барон навіть не зважився вдатися знову до усмішки, бо раптом його взяв сумнів, чи він зуміє усміхнутися. Тому він сказав, дуже незграбно вдаючи хворого:

- Обговорімо все завтра в Каїрі. Мені чогось недобре. Мабуть, салат з омарами завадив...

Він вийшов з умивальні, випровадив свого гостя, тоді помчав щодуху, схожий на довгоногого коника, до готелю. Люди, що гуляли на пишному Дівочому бульварі - скромно напудрені дами та статечні пани - обурено зводили брови, оглядаючись на нього:

- Хто ж так бігає по бульвару! Яка некультурність!

Троч нічого того не чув і не бачив. Він відчував, що сміх вислизає від нього, й здогадувався як. А тому хотів урятувати, що можна, утримати руками й зубами. Того він і мчав так стрімголов по Дівочому бульвару, не зважаючи ні на людей, ні на машини, мчав наосліп уперед, як навіжений. Перебігаючи вулицю, він спіткнувся, впав, зачув, як скреготнули гальма, закричали люди; по нозі йому потекло щось гаряче, й він вигукнув, уже непритомніючи:

- Тім Талер!

Той нещасливий випадок був і несподіваний, і неминучий. Страх призводить до невпевненості, невпевненість - до розгубленості, а розгубленість - до нещасливих випадків. Отож барон цілком неминуче попав під машину, почавши боятися за свій сміх. [205]

Однак Троч був міцніший, ніж здавалося на перший погляд. Та й шофер у останню мить устиг загальмувати. Під колеса барон не попав, а знепритомнів тільки від того, що, падаючи, вдарився головою об брук.

Скоро підбігли два захекані детективи: вони підняли його й поклали в машину швидкої допомоги, що примчала за п'ять чи шість хвилин. Детективи поїхали з бароном і в лікарню, де він досить швидко опритомнів.

Перших його слів не зрозумів ніхто в палаті. Бо він сказав:

- Найкраще сміється той, хто сміється останній.

Потім надійшов лікар, і Троч утомленим голосом сказав охоронцям, що вони йому більше не потрібні.

- У монастирській лікарні найкраща охорона - благочестя,- жартома сказав він і тихенько засміявся. Той сміх, здавалось, оживив його.

Охоронці вийшли, і лікар ретельно оглянув Троча. Виявилося, що барон відбувся кількома синцями та невеличким струсом мозку. Йому приписали легку дієту й постільний режим на кілька днів. А крім того, порадили по змозі не приймати відвідувачів.

А гооте Троч іще того ж самого вечора прийняв одного досить дивного відвідувача. То був невеличкий миршавенький чоловічок у нікельованих окулярах, із пом'ятим обличчям і в пом'ятому костюмі. Медсестра аж здивувалася, що цей пацієнт - такий начебто значний пан - водиться з такими людьми.

Троч про дещо спитав того чоловіка й дещо йому доручив.

- Ви бачили хлопця після тієї історії на іподромі?

- Ні, пане бароне!

- Тихіше! Мене звуть містер Браун.

- Слухаюсь, пане... містере Браун. Я ще хотів вам сказати, що знаю хлопця з фотографій у газетах.

- Ну, й те добре. Та все ж постарайтесь побачити його й навіч, коли зумієте. Тільки непомітно, бо він може [206] вас упізнати. Нікельовані окуляри дуже -мало змінили вас.

- Слухаюсь, пане... Браун.

- Та пам'ятайте: обережність і ще раз обережність. Щоб він ні в якому разі не помітив, що крім наших постійних детективів за ним стежить іще хтось. Зрозуміли?

- Авжеж.

- Іще одне запитання...

- Слухаю вас, пане Браун.

- Та це вже приватне запитання. Ви знаєте казку «Липкохвіст»?

- Іще б пак! Мені ж довелося дивитись на ту виставу, коли я два роки тому стежив за хлопцем тут, у Гамбурзі, пане ба... раун. Він же тоді ходив із тими Рікертами до театру. І вистава називалася «Липкохвіст».

- Ага! Тоді мені ясно...- барон на мить заплющив очі й пригадав, як вони їхали в таксі з вокзалу. Тім сидів поруч нього, а шофер сказав: «Грім побий, швиденько ж ви впорались! Це як у казці «Липкохвіст», коли знаєте...» Потім барон побачив перед собою Тімове обличчя. Спершу воно здригнулося, тоді наче скам'яніло. Немовби перед хвилюванням опустили завісу. Троч той вираз хлопцевого обличчя давно знав. І тепер він зрозумів, чому шофер згадав казку «Липкохвіст». А коли чоловічок у нікельованих окулярах переповів йому зміст тієї казки, він зрозумів іще більше.

До найдивовижніших Трочевих талантів належала пам'ять на числа, що зраджувала його дуже рідко. І цього разу та пам'ять послужливо підказала йому номер таксі. Троч записав його на відірваному клаптику газети й віддав той клаптик чоловічкові.

- Коли хлопець сяде в таксі з таким номером, негайно сповістіть мене! Спитайте в медсестри номер мого телефону й запишіть. Ясно?

- Ясно, містере Браун!

- Моєму шоферові перекажіть, хай чекає з машиною напоготові біля лікарні. Машину хай найме де-небудь, [207] тільки аби не наша. А ви подзвоніть мені негайно, коли хлопець сяде в те таксі. Не-гайно! Тоді буде дорога кожна хвилина.

- Слухаюсь.

- Ну, тоді йдіть.

Чоловічок рушив був до дверей, але Троч затримав його ще раз.

- Ті люди, здається, перевдяглися. Може, хлопець теж перевдягнеться. Це я вам про всяк випадок нагадую.

- Гаразд, бароне... Браун.

Чоловічок пішов. Троч підвівся, пошкандибав до дверей, тихенько замкнув їх, одягнувся - тільки черевиків не взув,- тоді знову відімкнув двері й ліг одягнений на ліжко. Ту ж хвилину задзвонив телефон.

- Як ви себе почуваєте, бароне?- спитав у трубці Тімів голос.

Троч простогнав:

- Кепсько, пане Талер! Перелами, тяжкий струс мозку. Ледве здужаю поворухнутися.- Затамувавши подих, вслухався він у трубку. Але почув тільки спокійний юнаків голос:

- Тоді я не втомлюватиму вас, бароне. Видужуйте. Та будьте обережні!

- Щодо цього не бійтеся, пане Талер!

Троч тихо поклав трубку на апарат. Тоді опустив голову на подушку й задививсь у вікно. Там крутились у повітрі дві ластівки.

«Сміх,- подумав Троч,- теж птиця, але така, що ні в які сильця не йде. Птиця, що її не можна спіймати».

Потім додав вголос:

- І мене теж ніхто не спіймає!

Тридцять другий аркуш

ЧОРНИЙ ХІД

Тімів номер дивився вікнами в двір, тому хлопець не почув Трочевого крику, коли з бароном сталося нещастя. В загальному сум'ятті ніхто не здогадався [208] сповістити його про це зразу. А після недовгої телефонної розмови з Трочем Тімові чогось подумалося, що й цей нещасливий випадок наче вичитаний із пригодницької книжки - як і все, що діялось того дня. .Коли він згадував про буцімто тяжко покаліченого барона, йому ставало соромно за ту думку, і все ж вона майже заглушила співчуття.

І те, що зробив Тім далі, було ніби вичитане з пригодницької книжки. Бо він учинив так, як радила таємнича записка («Чорний хід»!) і як здогадувався Троч («Може, хлопець теж перевдягнеться»). Отут хлопцеві придалося те, що барон останній рік щедріше давав йому кишенькові гроші.

Тім викликав дзвінком покоївку й пообіцяв дати їй триста марок, коли вона чимшвидше потай принесе йому ношену матроську одежу: сині парусинові штани, темно-синій светр і кашкет.

Дівчину страшенно захопило таємниче доручення: вона, певне, читала копійчані романи. Вона сказала, що має кавалера з «Матроської спілки» й зустрінеться з ним сьогодні о восьмій годині. В нього вона може випросити й одежу.

- Гаразд,- сказав Тім.- Тоді вгорнете її в свіже простирадло й занесете мені до дев'ятої години.

- Ну що ви, містере Браун!- відказала дівчина (адже Троч і Тім оселились у готелі як батько і син Брауни).- О дев'ятій годині простирадел у нас не міняють. Хіба купальне!

- Ну то загорніть у купальне простирадло, мені однаково, аби ви принесли одежу.

- А що мені сказати тому панові внизу, містере Браун?

'- Якому панові?- перепитав Тім.

- А тому, що дав мені сто марок, щоб я нишком стежила, що ви робитимете!

- А, панові детективові! Скажіть йому, ніби я попросив у вас таблеток від головного болю, а ви мені сказали, що вони є в аптечці у ванній кімнаті. [209]

- Гаразд, містере Браун!

- І ще одне: о дев'ятій годині, мабуть, уже ж не ваша зміна буде? То чи зможете ви на той час надягти свою робочу сукню?

- Авжеж, містере Браун! Інакше мене б сюди й не пустили. В мене вдома є робоча сукня, я її надягну під пальто. А поверх чепчика запнуся хусткою. Це щоб тут не марудитись, перевдягаючись.

- Чудово,- сказав Тім, і в кутиках уст у нього з'явилися виразні ямочки.- Отже, я можу бути спокійний, одежу ви принесете?

- Принесу, містере Браун. А... гроші ж ви мені віддасте?

- Та візьміть хоч зараз! - юнак вийняв з гаманця три папірці по сто марок і простяг покоївці.

- Який же ви легковажний! - засміялася дівчина.- Хіба за таке платять наперед? Ну та дарма, я вас не одурю. А поки що спасибі! І до побачення!

- До дев'ятої,- відказав Тім. Тоді замкнувся й ліг на диван. Заснути він, звичайно, не засне, але. хоч спочине.

Зразу по дев'ятій годині прийшла, як умовлялися, покоївка. В чорній сукні з штучного шовку й у білому чепчику. Купальне простирадло вона несла згорнене перед собою.

- Той чоловічок спитав мене, чого я до вас іду,- прошепотіла вона.- Та я йому сказала, що ви вдень замовили на дев'яту годину свіже простирадло для купання.

- Спасибі,- голосно промовив Тім, а пошепки додав: - Вітайте вашого кавалера з «Матроської спілки»!

Тоді й дівчина відповіла голосно:

- Дякую, містере Браун! Щиро дякую! - І вийшла з номера, ще раз моргнувши Тімові в дверях. Юнак теж моргнув їй.

Покоївчин кавалер був, видно, трохи вищий за Тіма; aле ж штани можна підтягти вище на шлейках, а светри носять і просторі. [210]

У дзеркалі Тім сам себе не пізнав, надто як насунув на лоба кашкет. Тільки пещений колір обличчя зраджував його. Однак і на те знайшлася рада: він натер собі щоки пемзою, що знайшлась на умивальнику, й намазав їх землею з квіткового горщика. Потім умився й зробив те саме ще, й ще, й ще раз. Вийшло так, як треба: Тімове обличчя стало таке червоне, наче він тільки-но видужав від кору. Хлопець із голови до п'ят просто тхнув «Матроською спілкою».

Тепер треба було подумати, що взяти з собою, бо ж до цього номера він, мабуть, уже не вернеться. Тім знав, що життя багатого спадкоємця минеться для нього, коли він поверне собі свій сміх, але зовсім не журився тим. То що ж узяти з собою?

Тім вирішив забрати тільки деякі папери: свій паспорт, угоду про сміх, угоду про купівлю пароплавства ПГГ, третю угоду про ляльковий театр і манюсіньку таємничу записочку.

Ті п'ять паперів Тім, рівненько позгортавши, сховав у велику задню кишеню матроських штанів, тоді дбайливо застебнув її.

Хлопець був готовий до найважливішої справи в своєму житті. (Тим часом надходила вже одинадцята година.) Про всяк випадок він іще викурив підряд три сигарети, щоб тхнути тютюном і мати хрипкий голос. (Взагалі він не курив, але для гостей мав на столі скриньку з червоного дерева, повну сигарет.)

Тепер лишилось якось вислизнути з готелю, щоб не помітили детективи. (Поки Тім курив, стало вже п'ятнадцять хвилин на дванадцяту.) Вилізти у вікно? Помітять. Отже, тільки коридором. А для того треба якось відвести очі детективові у вестибюлі. Тім уже надумав як: він написав баронові коротенького листа, де побажав йому швидше видужувати, й викликав дзвінком хлопця-коридорного. (Було вже пів на дванадцяту.)

Хлопець, що прибіг на дзвінок, був приблизно одного віку з Тімом, але видавався набагато молодшим. [211] Він мав руду чуприну й зухвале кирпате обличчя, що сподобалося Тімові.

- Ви можете зіграти для мене невеличку комедію, коли я дам вам двісті марок?

То були всі Тімові гроші.

Хлопець вишкірив зуби:

- А в чому річ?

- У мене біля дверей стоїть детектив...

- Знаю,- все осміхаючись, мовив хлопець.

- То відведіть йому очі. Візьміть оцього листа й покладіть за вилогу рукава, щоб трохи було видно. А коли той чоловік спитає вас, що то за лист,- а він напевно спитає, я знаю,- удайте, ніби збентежились, бо вам звеліли нікому його не показувати. Втікайте від нього, він побіжить за вами, буде давати вам гроші, щоб ви показали йому листа.

- О, будьте певні, містере Браун!

- Авжеж. Я знаю. То ви сперечайтеся з ним якомога довше, щоб я встиг вийти з номера й утекти з готелю чорним ходом. А листа, звісно, можна йому показати.

Кирпатий ніс під рудою чуприною весело наморщився.

- Виходить, чотири-п'ять хвилин. Це можна. Правда, за такий час я встигну виторгувати в нього більше, тож із вас я візьму тільки сто марок.

Тім хотів був заперечити, але хлопець відмахнувся:

- Ні, ні, не треба! Ста марок вистачить. Ви ж перебралися так, що, певне, йдете не між багаті люди. Там вам гроші придадуться.

- Може, й ваша правда,- погодився Тім.- Ну, спасибі вам. Нате лист і сто марок. А як заведете детектива за ріг, удайте, ніби закашлялись.

- Зроблю як треба, містере!

Хлопець сховав гроші в кишеню, а листа за вилогу рукава. Потім - усупереч усім готельним правилам - подав Тімові руку й приязно сказав:

- Ну, хай щастить, Брауне!

- Та щастя б мені не завадило,- відповів Тім і стис хлопцеві руку. [212]

Коли коридорний вийшов, Тім приклав вухо до дверей. Серце його знову билось, аж із грудей вискакувало.

Ось почувся гавкучий кашель. (До дванадцятої години лишалось п'ятнадцять хвилин.) Тім обережно прочинив двері. У вестибюлі не було нікого.

Тім тихенько причинив за собою двері й, не замикаючи їх (шкода часу!), метнувся до чорного ходу.

Ніхто його не зупинив. Тільки зустрілась одна покоївка, він буркнув їй «добривечір», але вона його, видно, не впізнала.

На вулиці під яскравими ліхтарями блищала мокра бруківка. На Гамбург уже сипалася мряка. На тротуарі навпроти стояв якийсь чоловік під парасолькою, але не дивився в Тімів бік. У світлі від ліхтаря блищала нікельована оправа його окулярів.

Тільки не бігти! Треба човгати неквапливо, посвистуючи, мовби справді матрос. Тім озирнувся довкола, ніби не знаючи, куди податись, тоді засвистів і рушив у бік ратуші. Нічиїх кроків за собою він не чув, а оглянутися боявся. Зовні безтурботний, а насправді згоряючи з хвилювання, нога за ногою дійшов він до рогу, звернув у завулок - і побіг щодуху. Лиш вибігши на майдан перед ратушею - на вежі якраз задзвонили дзиґарі - він спинився. На майдані стояв рядочок таксі, але тільки в одному мотор був заведений. Підійшовши до тієї машини, перебраний на матроса Тім упізнав за кермом Джонні - також перебраного й загримованого.

Дзиґарі дзвонили, настала північ, «темна година трамваїв».

Тім відчинив дверцята машини й сів поруч шофера.

- Вибачте, мою машину замовлено. Візьміть, будь ласка, іншу - не глянувши на свого пасажира, промовив Джонні. Він водив поглядом по майдані, ніби когось шукаючи.

- «Туди, де «Липкохвіст», прийди. Те, що знайшла королівна, знайди»,- півголосом відказав Тім.

Аж тоді Джоні рвучко повернув голову:

- Тіме, ти? На кого ти схожий! [213]

- У моєї покоївки кавалер із «Матроської спілки», Джонні!

- За тобою є «хвіст»?

- По-моєму, нема.

Повз кілька освітлених вітрин таксі проїхало вулицею вгору, до Редінзького ринку, а там круто звернуло праворуч і помчало в напрямі порту.

- А за тобою стежили, Джонні?

- Здається. В мене вже з годину таке відчуття. Правда, якоїсь певної людини чи машини я не помітив, просто відчуття таке. Ми поїдемо глухими вулицями.

Поруч стерничого Тім уже не так хвилювався. Ця опівнічна подорож у таксі доти уявлялась йому куди страшнішою й цікавішою. Хоч вони весь час їхали темними, таємничими глухими вулицями, то були найспокійніші хвилини за весь переповнений пригодами день. [214]

Джонні їхав швидко, але впевнено. Час від часу він кидав погляд на дзеркальце. Спершу ніхто за ними начебто не гнався. Однак незабаром за ними вчепилась якась машина, що їхала з вимкненими фарами.

Тім почав був допитуватися про Крешимира, але Джонні урвав його:

- Почекай, скоро сам його побачиш. Не питай, прошу тебе.

- А можна щось спитати не про Крешимира?

- Ну, що?

Таксі вже їхало по Альтоні.

- Звідки ти дізнався, що ми з бароном прилітаємо?

Стерничий засміявся.

- А ти пам'ятаєш одного пана на ім'я Салех-бей?

- Іще б пак!

- Отож він знайшов нас та повідомив. Коли прилетів ваш літак, ми понаймали всі таксі в аеропорту. Вам лишалось сісти тільки в оцю машину. Це мого родича машина.

- А звідки ж ви довідались, що ми поїдемо в таксі? Ми ж звичайно їздимо своєю.

- Салех-беєві було відомо, що ви маєте прибути інкогніто. Навіть фірма не повинна була знати про ваше прибуття. І мачуху твою нацькувати на барона придумав також Салех-бей. Помогло це тобі хоч трохи?

- Ні, Джонні, не помогло. Коли й Крешимир не поможе, тоді...

- Тоді плюнь мені в вічі. Але годі вже про це. Потерпи!

Вони вже були недалеко від Евельгене. Раптом Джонні засміявся.

- Чого ти?

- Та я згадав, як ти мінявся з бароном. Акції міняв на пароплавство. Я, звісно, зразу ж зметикував і назвав таке пароплавство, про яке добре знав, що воно продається. Ти справді його виміняв?

- А ось у мене в кишені купча! [215]

- Молодець, Тіме! ПГГ - добряча фірма. Як тобі потрібен буде стерничий...

Вони виїхали на набережну. Зиркнувши в дзеркальце, Джонні знову побачив машину з вимкненими фарами, що їхала за ними.

Він нічого не сказав Тімові, тільки піддав швидкості, все позираючи скоса на дзеркальце.

Тім щось сказав, але Джонні того не чув. Він бачив, що й машина позаду теж піддала ходу й помалу наздоганяє їх.

Враз, мов свиня під різницьким ножем, вереснули гальма.

Правою рукою Джонні притримав Тіма, щоб той не вдарився лобом об вітрове скло. Таксі спинилось, і темна машина пролетіла повз них.

- Вилазь!- ревнув Джонні. А спереду вже завищали ще одні гальма.

Стерничий потяг Тіма за собою. Через вулицю, вниз по вузеньких, крутих сходах, у кущі праворуч; далі, перелізши через цегляну огорожу, в якусь пивничку, другими дверима з пивнички, ще через одну огорожу, тоді другими, ще крутішими сходами знову вниз.

- Що сталося, Джонні? Хіба за нами хто женеться?

- Мовчи, бо захекаєшся! Ми їх перехитрували, відірвались від них, тож не даймо себе догнати! Там, унизу, жде Крешимир!

Тім спіткнувся, Джонні підхопив його й просто на руках зніс униз.

Тімів погляд ковзнув по ледь освітленій табличці: «Чортові сходи».

Поставивши хлопця на ноги, Джонні свиснув, і десь близенько почувся свист-відповідь.

- Роби зразу, що тобі Крешимир скаже!- шепнув Джонні Тімові.

З пітьми виринули дві постаті: Крешимир і пан Рікерт. Тімові знову підкотивсь до горла клубок - цього разу з добре яблуко завбільшки. [216]

- Дай руку. Тіме. і забиймося об заклад, що ти одержиш свій сміх назад. Швидше! -почув Тім знайомий Крешимирів голос.

Тім поборов своє остовпіння й подав Крешимирові руку.

- Закладаюся...

- Стій! - гукнуло щось угорі на сходах.- Стій! - Однак нікого там не було видно.

Крешимир промовив спокійно й твердо:

- Закладаюся, що ти не одержиш назад свого сміху, Тіме. На один пфеніг.

- Стій! - знову крикнуло нагорі.

- Не слухай! - шепнув Тімові Джонні.

- А я закладаюся, що одержу свій сміх назад, Крешимире. На один пфеніг.

Джонні перебив їм руки. як заведено. І запала моторошна тиша.

Тім забивсь об заклад, як йому сказано, але ще не розумів, що ж сталось, і стояв безмовний, розгублений.

На нього дивилося троє знайомих облич, ледь освітлених променями далекого ліхтаря. Власне його обличчя було в затінку, і тільки смужечка лоба біліла в пітьмі.

Панів Рікертів погляд ніби прикипів до того блідого чола. Таким він уже раз бачив Тіма, отак само ледь освітленого. Всього за кілька кроків звідси, на ляльковій виставі. «Адже з того людина й пізнається, що в слушну мить вона сміється». Чи це справді слушна мить?» - питав себе пан Рікерт.

Джонні й Крешимир теж не зводили очей із того блідого чола - єдиного, що видно було з Тіма в темряві.

А хлопець стояв, понуривши очі. І все ж він відчував на собі ті допитливі погляди. Йому було чогось тяжко, аж млосно, хоч із живота вже піднімалось щось легеньке. тихеньке, крилате - мов жайворонкова пісня. [217] що проситься на волю. Але Тім був іще надто смутний та безпорадний. І сміх-дзвіночок із кумедним «ік!» на кінці вирвався з нього зовсім мимохіть. Ніби не Тім заволодів знову своїм давнім сміхом, а сміх заволодів Тімом.

Ця мить, така жадана, така довгождана, ця радість виявилась понад силу Тімові. Він розгубився перед своїм щастям.

Колись давно, в ляльковому театрі, йому впало в очі. що сміх і плач бувають зовні дуже схожі. А тепер він дізнався, що сміх і плач часом дуже подібні не тільки зовні - їх іноді й розрізнити важко. Тім сміявся й плакав воднораз. Він здригався, ніби хлипаючи. і з очей йому текли сльози, а руки безвладно обвисли. Він забув геть про все довкола, бо неначе вдруге народжувався на світ.

Потім сталася дивовижна річ: крізь сльози Тім побачив три світлі плями облич, що наближались до нього. і раптом ніби проваливсь кудись у минуле. Йому здалося, наче він. малий хлопчик, стоїть біля віконечка каси на іподромі. Він виграв гроші, цілу купу грошей. і тепер плаче зі щастя, що виграв, і з журби, що вже не може поділити те щастя з татком. Тоді V вухах ЙОМУ пролунав рипучий густий голос: «Гей. малий, ну хто ж це плаче, маючи таке щастя!»

Тім підвів очі.

Тепер із якогось куточка його пам'яті мав би виступити чоловічок із пом'ятим обличчям і в пом'ятому костюмі. Але натомість Тім побачив іншу постать, велику, справжню. живу: стерничого Джонні. І разом зі стерничим до Тіма раптом повернулася дійсність - ніч біля шинку в Евельгене. ліхтар над сходами, що вели кудись угору, в темряву, і обличчя всіх трьох Тімових друзів, що дивились на нього, не знаючи, плакати їм чи сміятися.'

Сміх, що повернувся до Тіма Талера, налетів на нього, мов шквал. Та по тому шквалі вже настав штиль. [218] Тім уже цілком володів своїм сміхом. Витерши рукою сльози, він спитав:

_ Пане Рікерт, ви ще пам'ятаєте, що,я пообіцяв вам. від'їжджаючи з Гамбурга?

- Ні. Тіме.

_ Я тоді сказав: «Коли я повернусь, я вже сміятимусь». І ось я сміюсь. Я вмію сміятись, пане Рікерт! Умію, Крешимире! Джонні. я вмію сміятись! Тільки... Тім аж похлинувся сміхом.- Тільки я не збагну, як воно так вийшло.

Тімові друзі, що вже злякались були, чи не збожеволів Тім від щастя, зраділи, почувши, що він говорить цілком розумно.

- Ти давно вже міг вернути собі сміх,- сказав Крешимир.

- Як же це?

- Про що ми з тобою щойно забилися?

- Що я одержу назад свій сміх.

_ Так. Ну. а що мало статися, коли б ти виграв цей заклад?

- Я б одержав свій сміх назад. Так воно й вийшло!

_ Але ти б його однаково одержав, хоч би й програв заклад, Тіме!

Аж тоді хлопець уторопав, у чому річ. Сміючись. ляснув він себе по лобі:

- Авжеж! Програвши заклад, я також вернув би собі сміх! Та й дурний же я був! Я б міг закластися з ким завгодно, що одержу свій сміх назад. І так чи так одержав би!

- Ну. не так воно просто, хлопче,- озвався Джонні.- 3 ким завгодно закластися ти б не зміг, бо ж тоді ти б мусив перше сказати, що не маєш сміху, а цього ти не мав права робити. Ти міг забитися тільки з тим. хто здогадався про вашу з Трочем угоду, цебто з Крешимиром.

- Але зі мною,- докінчив Крешимир.- твій заклад був безпрограшний. [219]

Тепер, позбувшися своєї біди, Тім ураз побачив, яка проста була вся справа. Виходило, що він, розгублений і зневірений, цілі роки тицькався з одного чорного ходу в інший, замість іти навпростець. Він вигадував хитромудрі плани, рахував на мільйони. А сміх собі повернув за багато дешевшу ціну, дешевше, ніж коштує мала пачечка маргарину,- за один пфеніг.

Ось який дешевий сміх, коли платити за нього грішми, але справжню його ціну не вбереш і в мільйони. Сміх, сказав Салех-бей, це не крам на продаж. Його треба заслужити.

Тридцять третій аркуш

ПОВЕРНЕНИЙ СМІХ

Мряка тим часом перейшла в дрібний дощ, але ніхто з чотирьох того не помітив. Не чули вони й чиїхось непевних кроків на сходах. Лиш як усі четверо разом замовкли, то почули, що хтось до них спускається, і обернулись.

На вузеньких кам'яних сходах, похитуючись, виринула з темряви худа, згорблена постать. У промінні ліхтаря з'явилися довгі ноги в чорних штанях, тоді білі руки з довгими пальцями, біла манишка, над нею довгобразе обличчя. Нарешті постать, уже освітлена з ніг до голови, спинилась під табличкою «Чортові сходи» й замислено сперлась на мур із обтесаного каменю.

То був барон.

Тімові знову залоскотав у горлі сміх. «Чорний хід!»- глузливо пролунало в нього в голові. Таж це чорт із лялькового театру, рухлива лялька на ниточках, така кумедна та недотепна, що аж жаль було її. І Тім Талер, хлопець, що знову навчився сміятись, стримав себе: не засміявся.

Барон сів на кам'яний східець і повернув своє біле [220] як крейда обличчя з міцно стуленими губами до Тіма та його товаришів.

Тім піднявся до нього.

- Верніться в лікарню, бароне.

Троч підвів на хлопця очі й ще міцніше стис губи.

- Бароне, вам не можна тут сидіти.

Тоді Троч нарешті розтулив рот і хрипким голосом спитав:

- А на що ви забивались, пане Талер?

- На один пфеніг, бароне.

- На один пфеніг? - Троч аж підхопивсь, але зразу ж обперся знову об стіну. Майже жіночим голосом він вереснув:

- Дурні! Ви б могли забитись на мою спадщину!

Згори надбіг, перестрибуючи через три східці зразу, знайомий Тімові шофер.

- Пане бароне, вам же не можна так!

- Дайте мені дві хвилини поговорити з цим юнаком. а тоді відвезете назад у лікарню.

- Як хочете, але в такому разі я ні за що не відповідаю, пане бароне!

Шофер піднявсь на кілька східців вище й став там.

А внизу при підніжжі сходів стояли Джонні. Крешимир та пан Рікерт - мовчазна Тімова охорона.

- Можна на хвилину спертись на вас. пане Талер?

- Будь ласка, бароне! Я тепер при силі! - Тім тихенько засміявся.

Троч зіперся на юнакове плече.

- Ваша спадщина, пане Талер...

- Я зрікаюся її, бароне!

Троч отетерів, але тільки на одну мить.

- Розумно робите.- сказав він.- Це спрощує все діло. А маючи пароплавство, ви зможете жити більшменш у достатку.

- Пароплавство, бароне, я подарую своїм друзям.

Троч з подиву так вирячив очі. що те видно було навіть крізь темні окуляри. [221]

- Але ж тоді наша угода не дала вам ніякої користі, пане Талере. Ви тепер такий самий бідний, як були спочатку! Не маєте нічого, крім прогорілого лялькового театру!

Тім засміявся вже голосніше.

- Ваша правда, бароне. Я мушу все починати спочатку. Але те, що я маю тепер, за останні роки стало для мене цінніше, ніж будь-які акції в світі.

- Що ж це таке?

Замість відповіді Тім просто засміявся на весь голос. Барон відчув, як під його рукою трясеться від сміху хлопцеве плече. І почув у тому сміхові зовсім нові нотки, густіші, мужніші, ніби акомпанемент до давнього сміху-дзвіночка.

Троч відвернувся й махнув рукою своєму шоферові. Той квапливо збіг униз, підхопив барона під руку й повів нагору.

Тім гукнув:

- Видужуйте, бароне! Щоб були знову цілий-здоровий! І спасибі вам за всю науку!

Троч не оглянувся. Шофер почув, як він буркнув:

- Еге, цілий! Як же я буду цілий без цього!

Тім дивився вслід баронові, аж поки того поглинула темрява.

Тімові друзі теж піднялися до нього й проводжали Троча поглядами. Джонні прогув:

- Такий багатющий, а щастя чортма!

За хвилинку й вони пробрались нагору. Попереду, на набережній, загурчав, заводячись, мотор, рушила з місця машина, і гудіння мотора стихло вдалині.

Скоро вони вийшли на набережну. Навпроти стояло темною тінню таксі родича Джонні.

- їдьте до мого гаража, Джонні,- сказав пан Рікерт.- Ми пройдемося пішки.

- А хіба ви й досі живете в тій білій віллі, пане Рікерт? Барон же казав мені, що ви тепер працюєте вантажником. [222]

- Так воно й є, Тіме. Завтра я тобі все розкажу. Ти ж не відмовишся погостювати в мене?

- Ну, що ви, пане Рікерт! Я ж іще повинен вашій матусі показати, що вже вмію сміятися. Чи, може...- Тім відвернув обличчя.- Може, вона вже...

- Ні, ні, Тіме! Жива й здорова! Ходімо!

Над дверима до вілли світилася лампочка. Білі двері з круглим балконом над ними та з двома світлими кам'яними левами обабіч них нагадували острів у океані темряви, привітний, гостинний берег після довгих блукань у бурхливому морі.

Тімові здушило горло, коли він підійшов до лагідних левів. А коли відчинилися двері й вийшла, спираючись на палицю, старенька пані Рікерт, повненька, круглявенька, з білими як сніг кучериками, хлопець насилу стримався, щоб не розревтись та не кинутись їй на шию. Спинившись перед нею, він пробелькотів, сміючись і плачучи воднораз: «Ну, що ви тепер скажете, пані Рікерт?» Але ніхто тих слів не зрозумів, та й не намагався зрозуміти, бо командування вже перебрала до своїх рук старенька господиня. Вона спитала:

- Все гаразд, Крішане?

А коли син її кивнув головою, полегшено перевела дух і сказала:

- Так це ж треба відсвяткувати, дітки! Але хлопець хай лягає спати. Бо він зовсім не при собі.

Як звеліла пані Рікерт, так і сталося: Тімові довелось хоч-не-хоч лягти в ліжко, і виявилося, що так воно й краще, бо він майже зразу заснув, як каменюка.

Другого дня старенька влаштувала так, щоб бути вдома самій, коли прокинеться Тім.

Зробити це було неважко, бо хлопець прокинувся аж пополудні.

Вони вдвох поснідали - хоч було таки пізненько для сніданку, але пані Рікерт так любила снідати, що рада була зробити це й удруге. А тоді Тімові Талерові довелось розповісти про все до крихти, що він [223] зазнав відтоді, як виїхав із Гамбурга. Та він і сам радий був про все розповісти.

Розмахуючи газетою, він вигукував: «Сенсаціоне! Сенсаціоне! Іль бароне Троч е морто! Ун рагаццо ді кваттордіні анні адессо іль піу річчо уомо дель мондо! Сенсаціоне!»

«Який він став гарний! - подумала пані Рікерт, слухаючи хлопця.- І як уміє балакати по-чужоземному!»

Тім розповідав старенькій про свої пригоди, наче про якусь комедію, виставу в театрі. Тепер, коли до нього повернувся сміх, багато такого, що доти було жахливе, здавалося просто смішне. Він розповідав про свої чудернацькі заклади з Джонні, про розбиту люстру в готелі «Пальмарр», про картини в генуезькому та афінському музеях, про інтриги в замку, про Салех-бея, про маргарин, про подорож довкола світу, про повернення до Гамбурга, про мачуху та Ервіна, про темну годину трамваїв.

Тоді настала черга господині розповідати, і вона почала видимо задоволене:

- Знаєш, Тіме, як ти не повернувся з Генуї та як до нас прийшов спершу пан Крешимир, а потім отой здоровило Джонні, я зразу здогадалася: тут щось не так. Мені не хотіли сказати, в чому річ. Бо в мене, щоб ти знав, вада серця. Але я вже більш як вісімдесят років із нею живу, з тією вадою, і ми вже звикли одна до одної. Отож я трошечки понишпорила й знайшла того листа, що ти писав із Генуї. Ну, і дізналася вже трошки більше.

Старенька, що тепер бачила в Тімові панича й намагалась говорити «по-правильному», знову збилася на свою гамбурзьку говірку.

- Стала я тоді, як прийде до Крішана пан Крешимир або Джонні та про щось там забалакають, нищечком їх підслухувати. Приходили-то вони не дуже часто, бо ж працювали в порту вантажниками. Іншої роботи хоч [224] плач не могли знайти, мов зурочив який чортяка. А воно, бач, і справді без нього не обійшлось, еге? Ну, хай там як, а я все знала, про що вони там шепотілися. Знала я й те, що сина мого з посади прогнали, хоч він усе від мене приховував.

- А він справді став вантажником? - перебив її Тім.

- Авжеж, синку, був і вантажником. Ти, може, не знаєш, як воно в нас у Гамбургу ведеться. Усі такі чесні, так своєї слави пильнують! Як кого звільнили з доброї посади, та як підуть чутки всякі, хоч і брехні, то вже ніхто його до себе в контору не візьме. Розумієш?

Тім кивнув головою.

- Ну, мені то пусте, в мене ж іще й свій маєток є. В паперах здебільшого.

- В акціях? - спитав Тім.

- Еге ж, в акціях, синку. Ти ж тепер і сам трошки на цьому знаєшся, еге? Отож, кажу, синові моєму й не треба було зовсім іти в ті вантажники, ми ж таки не бідняки. Але такий ото він удався, що не може сидіти без діла. А надто без порту він - як риба без води. Ну, ото й пішов у вантажники. Але з дому, було, вийде вбраний, як лялечка, й такий самий вернеться з роботи, а перевдягався там, у порту. Думав, я й не доберу, що в нього вже не та робота, бо я ж здебільшого вдома сиджу. Але ж є нащось телефон, еге?

Тім мимохіть засміявся зі старенької, а пані Рікерт засміялася теж.

- Я ж дурна стара гуска... ні, ні, мовчи, я сама краще знаю, хто я... але ж не така я вже й дурна! Спершу я погомоніла з паном Салех-беєм, як він сюди до нас зателефонував. А тоді панове змовники й самі про все признались, розказали мені. Я, звісно, вдала, ніби й не знаю нічого. Весь час очі отак-о витріщаю й вищу: «Та не може бу-у-ути!» Ну, хай там як, а розказали вони мені все. Так ото я й записочку тобі писала. Крізь лупу. То я ще як у школу ходила, то ми з дівчатами [225] було, цілими днями вправляємось писати крізь лупу. В мене найкраще з усіх виходило. Я якось цілий роман на аркуш поштового паперу переписала, їй-бо, правда!

- Та не може бу-у-ути! - передражнив її Тім.

- Е, ти все смієшся з мене, шибенику!

Задзвонив дзвінок біля дверей, і пані Рікерт попросила Тіма піти подивитись, хто там такий.

То виявився рудий хлопець-коридорний із готелю. Весь упрілий, він стояв між сімома валізами.

- Я приніс ваші речі, пане Талер! - весело оскірившись, сказав він.

- Учора ви ще називали мене містером Брауном. Звідки ви раптом дізналися, хто я такий?

Хлопець знов осміхнувся.

- А ви хіба газет не читаєте?

- А, он що! - Тім трохи спантеличено сягнув рукою в кишеню.

Та рудий відмахнувся:

- Нічого я з вас не візьму, пане Талер! Я можу з газети цілу купу грошей вицідити, коли розкажу, як я вчора ввечері пошив у дурні детектива. Можна?

- Та розказуй, що з тобою вдієш!

- Щиро дякую! Занести вам валізи?

- Спасибі, я сам. Тільки не розповідай там у газеті небилиць.

- А навіщо? І так буде цікаво.

Хлопець подав Тімові руку:

- Ну, щасти тобі, Тіме!

- Спасибі. Хай щастить тобі в газеті!

Два хлопці, стоячи між лагідними кам'яними левами, сміючись, потисли один одному руку. Потім рудий сів у готельну машину, що привезла його, й поїхав, а Тім позаносив валізи до вілли.

Того ж таки дня Тім пішов із Джонні, Крешимиром та паном Рікертом до одного нотаріуса, доброго знайомого старенької пані Рікерт.

Там пароплавство «Пасажирська лінія Гамбург - [226] Гельголанд», чи скорочено ПГГ, було переписане рівними паями на Джонні, Крешимира та пана Рікерта. Щоправда, пароплавство не зразу перейшло в їхню власність, потрібна була ще ціла купа всіляких формальностей (адже Тім не був уже спадкоємцем-мільярдером), але щонайпізніш за два тижні, сказав нотаріус, усе буде владнано.

Тімові друзі спершу нізащо не хотіли прийняти той дарунок, та коли Тім сказав, що віддасть пароплавство кому попало, вони поступилися. І навіть радо. Пан Рікерт мав іще досить сил, щоб керувати конторою старого пана Денкера на шостому причалі, а Джонні й Крешимир радніші були працювати стерничим та стюардом на своїх власних пароплавах, ніж на чужих.

Коли всі четверо вийшли від нотаріуса (його бюро містилось поблизу головного вокзалу), Джонні спитав:

- А що ж ти тепер робитимеш, Тіме?

Хлопець показав праворуч:

- Он у тому старому будинкові є ляльковий театр. Він мій. Я зроблю з нього мандрівний.

- То тобі потрібен автобус! - сказав Крешимир.

-І переносна сцена,-докинув Джонні.

- І все це купимо тобі ми, синку! - закінчив пан Рікерт.- І мовчи, не відмовляйся, бо тоді ми не візьмемо від тебе пароплавства.

-Згода! - засміявся Тім. Тоді додав уже серйозно: - Яке щастя, що є на світі ви!

- І Салех-бей,- сказав пан Рікерт.

- Авжеж,- потвердив Тім.- І Салех-бей. Треба послати йому телеграму.

І вони подались на пошту.

Старий у Месопотамії всміхнувся, прочитавши:

«чорт бери маргарин крапка сміх одержують за-

дарма крапка я його одержав крапка щиро дякую

за поміч ваш тім талер». [228]

Того дня й скінчилась наша історія, хоча в газетах вона тільки почалася (так, як зрозуміли її журналісти; а вони здебільшого її не зрозуміли).

Пан Рікерт знову став директором пароплавства, Джонні стерничим, а Крешимир стюардом.

Про барона Троча тепер можна почути дуже рідко. Кажуть, він майже весь час живе
самотній і сумний у своєму месопотамському замку, уникаючи людей. Проте фірма
його процвітає.

А Тімів слід загубився. Відомо тільки, що він удвох із старенькою пані Рікерт
вигадав лялькову виставу, що називається «Проданий сміх». А потім щез із
Гамбурга, й жоден газетяр так і не дізнавсь, куди він подався.

Лиш двічі ще доходили звістки про нього. На кладовищі в одному великому
німецькому місті до одного мармурового надгробка було покладено вінок із
написом на стрічці: «Я прийшов, як навчився сміятись. Тім».

А остання звістка про Тіма надійшла з однієї хлібної крамнички. Кілька років тому туди зайшов якийсь чемний юнак. Господиня, не впізнавши його, спитала, чого він бажає; тоді юнак скорчив похмуре обличчя й пробурмотів:

- Пограбувати директора гідростанції, пані Бебер.

- Тім Талер! - вереснула з несподіванки пекариха.

Та юнак приклав палець до вуст і сказав:

- Ц-с-с... Не зрадьте мене, пані Бебер! Я тепер звусь Енріко Грандіцці, власник найвеселішого в світі театру - лялькового театру «Ящик маргарину».

- Кумедія! - вигукнула пані Бебер.- А я ж учора там була. Хтось незнайомий прислав мені квитка. Тобто...- вона скоса зиркнула на Тіма.- Тобто, може, й знайомий.

- Може,- відказав юнак із ямочками в кутиках уст.

- Там показували виставу про проданий сміх,- провадила [229] пані Бебер.- Гарна п'єса. На всілякі думки наводить.

- А на які думки вона вас навела, пані Бебер?- поцікавився юнак.

- Та, правду сказати, спершу мені аж моторошно було. А під кінець сміялась, аж
за боки хапалася. Отоді й подумала: де людина сміється, там кінчається чортова
влада.

- Гарно сказано, пані Бебер,- відповів юнак.- Отак і треба поводитися з чортом, щоб він поламав собі роги.
