% vim: keymap=russian-jcukenwin
%%beginhead 
 
%%file 11_02_2023.fb.fb_group.mariupol.pre_war.7.a_vi_zamechali_skolk.cmt
%%parent 11_02_2023.fb.fb_group.mariupol.pre_war.7.a_vi_zamechali_skolk
 
%%url 
 
%%author_id 
%%date 
 
%%tags 
%%title 
 
%%endhead 

\qqSecCmt

\iusr{Maryna Holovnova}

Улюблені!!!!

\iusr{Maryna Holovnova}

Чудові фото!!

\iusr{Maryna Holovnova}

А я колись рожевий бачила!! Десь на Приморському бульварі наче.

\ifcmt
  igc https://scontent-fra3-1.xx.fbcdn.net/v/t39.30808-6/330599679_709479604006037_4801512611268118934_n.jpg?_nc_cat=101&ccb=1-7&_nc_sid=dbeb18&_nc_ohc=_IgEGGp99tsAX-3FuR2&_nc_ht=scontent-fra3-1.xx&oh=00_AfDJ4KrZCMFw6zSeZgpGFW_ki3pXL5wbOitqAe-EVNP3lA&oe=63FBF9DE
	@width 0.4
\fi

\begin{itemize} % {
\iusr{Viktoria Zaytseva}
\textbf{Maryna Holovnova} Это итальянский. Он не болеет этой болячкой, что поразила все наши каштаны. И он с"едобный.
\end{itemize} % }

\iusr{Юля Рудченко}

На левом целая каштановая аллея🥰🥰🥰🥺🥺🥺🥺

\iusr{Galina Gerich}

Я дуже люблю весну, коли все цвете та такий чудовий запах квітів по усему
місту.

\iusr{Татьяна Тищенко}

И вдоль всего нашего Морского бульвара росли огромные раскидистые каштаны. Как
они красиво цвели вместе с сиренью в начале, середине мая.

Да, было очень много каштанов по всему городу.

Пожалуй, тополя, каштаны и клёны самые распространённые деревья у нас. Были...
😞💔

\iusr{Eleonora Ehlkina}

а в мене є таке фото, хто впізнає де це?

\ifcmt
  igc https://scontent-fra3-1.xx.fbcdn.net/v/t39.30808-6/330754938_488353043299700_4812220150048428616_n.jpg?_nc_cat=104&ccb=1-7&_nc_sid=dbeb18&_nc_ohc=SgKuyarkQ64AX9Uv1ID&_nc_ht=scontent-fra3-1.xx&oh=00_AfB0LEkMQc7BXYNHGTFF8ECptzDW1eXUWt65dLD3TXAOMA&oe=63FB7FF1
	@width 0.4
\fi

\begin{itemize} % {
\iusr{Клавдия Макарова}
\textbf{Элька Элькина} Пешеходная дорожка возле башни.
\end{itemize} % }

\iusr{Yana Berezhna}

Да.. из своего окна любовалась каждый год((

\iusr{Елена Аврамова}

Мої улюблені каштаны 🌰 🌰 🌰

Не уявляю без них Маріуполь

\ifcmt
  igc https://scontent-fra3-1.xx.fbcdn.net/v/t39.30808-6/330808789_893924691754847_3318487909186922191_n.jpg?_nc_cat=104&ccb=1-7&_nc_sid=dbeb18&_nc_ohc=wIcSn8TcNqgAX8_JN2F&_nc_ht=scontent-fra3-1.xx&oh=00_AfB91EwAiYPyfE5U-XP7-eurPaep8ze_C_PoQGEl9kVUSA&oe=63FD090B
	@width 0.4
\fi

\iusr{Оля Томина}

Да, замечали👍💔

А ещё любили с дочкой орехи по осени собирать по частному сектору😔

\begin{itemize} % {
\iusr{Олена Сугак}
\textbf{Оля Томина} 

орехами была засажена вся ул. Зелинского. Наш двор на Куприна, тоже. Вдоль нашего 8
подъездного дома, через 1 - орех, каштан. Каштаны - красиво, орехи -
вкусно... но местные автомобилисты их недолюбливали... ибо в ветреную погоду,
это все \enquote{счастье} било машинки....

\end{itemize} % }

\iusr{Witalii Lybarskii}

Нет мы же дебилы что за глупые вопросы?

\begin{itemize} % {
\iusr{Олена Сугак}
\textbf{Witalii Lybarskii} что за настроение? Все отвечают воспоминаниями, шлют фото ... Не нравится, прошли мимо.

\iusr{Натали Михайловская}
\textbf{Witalii Lybarskii} не злитесь, это просто воспоминания...

\iusr{Witalii Lybarskii}
\textbf{Натали Михайловская} да я не злюсь просто есть люди у которых бак подсвтсует вот и всё

\iusr{Олена Сугак}
\textbf{Witalii Lybarskii} у вас бак вместо ума. При чем пустой! Раз такое можете написать...
\end{itemize} % }

\iusr{Натали Михайловская}

Да, в центре города, на Приморском булвпре, у нас на Черемушках... были...

\iusr{Viktoria Zaytseva}

Замечали, что все каштаны больные и рано или поздно городу придётся от них
избавиться. Деревья просто медленно умирали. Сейчас, когда город будут
восстанавливать после войны, садить надо другие деревья, а больные каштаны
вырубить. Кстати, каштаны больные везде, и здесь, в Европе.

\begin{itemize} % {
\iusr{Олена Сугак}
\textbf{Виктория Зайцева} 

при чем весной распускались листики, цветы ...а когда кашташ отцветал,
начиналось ... Лист желтел и опадал. До настоящего листопада, уже и опадать
было нечему. Жалко. Красивое дерево. Тень от него хорошая.

\end{itemize} % }

\iusr{Люба Копейкина}

Я обожала наши каштаны весной - эти незабываемые свечечки в виде цветов)

\iusr{Eleonora Ehlkina}

ещё в городе много тополей и тополиный пух в мае это было нечто...

\iusr{Sergey Drovorub}

Да

\iusr{Марина Бутенко}

Заметила только сейчас сколько их было, когда больше не вижу их

\iusr{Alexandr Pliner}

Вулиця Московська - тунель з каштанів. Мабуть вже був...
