% vim: keymap=russian-jcukenwin
%%beginhead 
 
%%file 11_12_2022.fb.rojz_svitlana.kyiv.1.metafory
%%parent 11_12_2022
 
%%url https://www.facebook.com/svetlanaroyz/posts/pfbid08zmj5Tr7UqQopUda7mhhs6xmWNLhW9W6fUvjBmKgJ311y3Ajmr1KsPs3UYLWS3Rpl
 
%%author_id rojz_svitlana.kyiv
%%date 
 
%%tags 
%%title Три важливих сьогоднішніх метафори
 
%%endhead 
 
\subsection{Три важливих сьогоднішніх метафори}
\label{sec:11_12_2022.fb.rojz_svitlana.kyiv.1.metafory}
 
\Purl{https://www.facebook.com/svetlanaroyz/posts/pfbid08zmj5Tr7UqQopUda7mhhs6xmWNLhW9W6fUvjBmKgJ311y3Ajmr1KsPs3UYLWS3Rpl}
\ifcmt
 author_begin
   author_id rojz_svitlana.kyiv
 author_end
\fi

Три важливих сьогоднішніх метафори: Ми їхали сьогодні по трасі, повністю
вкритою туманом. В полі видимості - 7-10 метрів дороги. Я напружено вдивлялась
в цей туман, як в можливо небезпечну невизначеність. 

Я подумала, що все, як в житті - перед тобою туман, майбутнє передбачити зовсім
неможливо. Хочеться зупинитись - перечекати. Але ти робиш крок за кроком, їдеш
метр за метром і дорога стає яснішою. З кожним метром вона відкривається. Не
небезпекою, а більшою визначеністю. Потім думаєш, як же я це пройшла?

Буду собі нагадувати знов і знов: В будь-якому тумані, якщо ти вже в ньому -
просто йди далі в нього і роби маленькі кроки. 

🌲 Ми їхали по трасі, туман відійшов, а з двох сторін дороги ліс. Я дивилась на
дерева. Голі мокрі гілки сосен та ялинок, як в казці, тяглись з двох сторін
своїми руками, наче хотіли вхопити. Було моторошно. Сказала собі - а раптом в
цій казці вони хочуть благословити. І здалося, що дерева трохи нахилились, на
дорозі в цей момент не було машин - і дерева, як Рід, благословляють саме наш
Шлях. Одразу стало світліше і затишніше. 

Звісно, соснам і ялинкам все одно. Вони просто дерева, що ростуть у дороги.

Але буду собі нагадувати: Ти пишеш свою історію. Твій вибір, якою буде її жанр. 

🌲 У нас в домі тепло дає пічечка. Стоїть вона на плитці, а плитка на паркеті.
Попіл з пічки викидаємо в відро. Сьогодні раптом в темряві побачили, як з відра
сиплеться вогняний попіл. Не знаю, як це сталося, як затупили і обрали
пластикове відро. Зазвичай, з пічки діставали вже холодний попіл. Але сьогодні,
мабуть, залишився ще не остужений, він розгорівся в відрі, пластик поплавився,
горячий попіпіл посипався. Якщо б це сталося вночі і попіл з вогнем попав на
паркет - було б зовсім небезпечно. 

Щастя, що ми були поруч, що він просипався на плитку. Все загасили, вичистили,
одразу взяли металеве відро. 

А я подумала - яка шикарна метафора: Якщо ти (як всі ми зараз) намагаєшся
вмістити в собі "горячий", палаючий, болючий досвід, щоб тебе "не поплавило" -
ти маєш постійно дбати про свій "контейнер". В першу чергу, тіло. Звільняти
його час від часу, вивільняти тими методами, які вважаєш екологічними.  І мати
"кахель" - заземлення - знання, практики, лікарів - ліки, психологів, друзів...
Щоб, якщо і поплавить - могли терміново допомогти. 

Будь- який туман розсіюється, в наших історіях, в яких ми автори- є і буде
багато різних частин різних жанрів, хочеться, щоб радості і благословіння було
більше. І ми точно витримаємо і зможемо вмістити в себе цей різний досвід. 

Обіймаю, Родино ❤️ як хочу Перемоги.

\ii{11_12_2022.fb.rojz_svitlana.kyiv.1.metafory.orig}
\ii{11_12_2022.fb.rojz_svitlana.kyiv.1.metafory.cmtx}
