% vim: keymap=russian-jcukenwin
%%beginhead 
 
%%file slova.vershina
%%parent slova
 
%%url 
 
%%author 
%%author_id 
%%author_url 
 
%%tags 
%%title 
 
%%endhead 
\chapter{Вершина}
\label{sec:slova.vershina}

%%%cit
%%%cit_head
%%%cit_pic
\ifcmt
  tab_begin cols=3
     pic https://static2.gazeta.ua/img2/cache/gallery/1044/1044544_2_w_1000.jpg?v=0

     pic https://static2.gazeta.ua/img2/cache/gallery/1044/1044544_3_w_1000.jpg?v=0
		 
		 pic https://static2.gazeta.ua/img2/cache/gallery/1044/1044544_4_w_1000.jpg?v=0
  tab_end
\fi
%%%cit_text
Перша українка на Евересті 33-річна Ірина Галай підкорила К2 Чогорі - другу
\emph{вершину} у світі, яку вважають набагато складнішою за Джомолунгму.
Галай стала першою українкою за час незалежності, кому вдалося зійти на
\emph{вершину}, передає "Сьогодні". Через надзвичайну складність сходження К2
називають Дикою горою, а у світі альпіністів - "горою-вбивцею". Рівень
смертності під час сходження на К2 становить рекордні 23\%. Тобто сходження
кожного четвертого альпініста закінчуються трагедією.
"Я багато років мрію піднятися на К2. Для альпіністів це вважається серйознішим
досягненням, ніж зійти на Еверест. Це як у боксі: можна виграти Олімпіаду, а
можна стати абсолютним чемпіоном світу серед професіоналів", – заявила Галай
перед штурмом \emph{вершини}
%%%cit_comment
%%%cit_title
\citTitle{Ірина Галай підкорила вершину К2, ставши першою українкою на Чогорі}, 
gazeta.ua, 27.07.2021
%%%endcit
