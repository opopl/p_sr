% vim: keymap=russian-jcukenwin
%%beginhead 
 
%%file 16_02_2022.stz.news.ua.strana.3.putin_ne_napal_infvojna.3.pochemu_zapad_nagnetaet
%%parent 16_02_2022.stz.news.ua.strana.3.putin_ne_napal_infvojna
 
%%url 
 
%%author_id 
%%date 
 
%%tags 
%%title 
 
%%endhead 

\subsubsection{Почему Запад нагнетает со \enquote{вторжением}?}

После того, как сегодня Путин не напал, да и вообще ничего страшного не
произошло, вся эта история все больше стала напоминать сугубо виртуальную. 

\enquote{Страна} уже описывала разные сценарии того, что может произойти,
включая военные. Но \enquote{казус 16 февраля} показал, что пока все движется
по пути чисто информационного воздействия. 

Понятно, что военного сценария исключить полностью нельзя. Но, похоже, сейчас
ни одной стороне он не выгоден.

Россия рискует нарваться на крупные санкции. Украина - потерять новые
территории, а затем погрузиться в политический хаос. Запад по той же причине
может потерять контроль над Украиной - что в США будет воспринято как поражение
Байдена и перечеркнет его политическую карьеру. 

Резко против любых силовых решений и Европа, которая усиленно пытается
разрулить конфликт путем челночной дипломатии между Россией и Украиной. 

То есть можно на время отложить военные варианты и прикинуть, чего помогает
добиться информационная истерия, которую Запад закручивает вокруг
\enquote{вторжения России}. 

1. Фоновый шум для российско-американских переговоров по стратегической стабильности.

Шум этот нужен потому, что Москва хочет вести этот диалог максимально публично
и в сжатые сроки. Конечная цель РФ - зафиксировать юридические гарантии Запада
по вопросам натовских вооружений у российских границ. И отдельно - отказа брать
Украину в Альянс. 

Причем, судя по сигналам, США готовы такие обещания давать (не размещать ракеты
в Украине, Польше и Румынии, не брать Украину в НАТО как минимум в ближайшее
время). Но поскольку уровень доверия между сторонами околонулевой, Москва
требует максимально публичных договоренностей и прозрачных формулировок.

Но такая транспарентность может обернуться огромной \enquote{зрадой} для Байдена в
Америке. Поэтому всю эту историю и обставляют \enquote{борьбой с вторжением}.

И если по итогу будут подписаны какие-то договоренности, то их подадут как
спасение от войны в Европе (или вообще ядерного апокалипсиса, смотря куда
дальше заведет западную информационную машину поднятие ставок). 

Подыгрывает ли Москва этой стратегии, чтобы \enquote{прикрыть} Байдена - не совсем
понятно.

С одной стороны, там открыто троллят США и Британию с их \enquote{датами вторжения}. Но
на общую медийную повестку Запада это не влияет. Поэтому не исключено, что в
Кремле как минимум понимают, что происходят и готовы это терпеть ради
возможности договориться по существу. 

2. Интересно, что та же логика может сработать для Киева, с которого и должна
вообще-то начаться деэскалация. Но для того, чтобы ее провести, страна должна
понять, что стоит на пороге войны. И такое впечатление у жителей Украины на
Западе усиленно создают. 

Параллельно началась серьезная работа по лоббированию политической части
\enquote{Минска-2}, которую, судя по заявлениям канцлера Олафа Шольца, возглавила
Германия.

По его словам, Украина обязалась внести на рассмотрение ближайшего заседания в
Трехсторонней контактной группе законы по особому статусу и выборам. То есть
то, чего с момента подписания \enquote{Минска-2} так и не сделала ни разу. 

Посмотрим, внесет ли, но тема очевидно пришла в движение. И не исключено, что в
той же плоскости находится вчерашнее признание Госдумой \enquote{ДНР} и \enquote{ЛНР}.

Похоже, это намек, что если Киев не начнет работу по политической части
\enquote{Минска}, Путин утвердит решение Думы, и Украина утратит даже нынешние шансы
реинтегрировать эти территории. 

Причем там есть четкий маркер - ближайшее заседание Трехсторонней контактной
группы, куда Ермак обязался вынести нужные законопроекты. Очередная встреча
планировалась на 2 марта. Видимо, после этой даты и будет понятно, идет ли Киев
на прогресс по Минским соглашениям (и работает ли угроза \enquote{вторжения} или надо
еще поднять градус). 

3. Война за европейский и мировой рынок газа.

Этот вопрос в контексте \enquote{вторжения} анализирует эксперт Игорь Димитриев. 

Он пишет, что в основе нынешнего противостояния является борьба за новые
правила игры на мировом газовом рынке, в которых активно участвует как Россия,
так и с США. Причем у обоих противоположные подходы. Россия хочет закрепить
свое положение на рынке Европы за счет долгосрочных контрактов на газ. В тоже
время, американцы пытаются превратить мировой газовый рынок в объект глобальных
финансовых спекуляций (подобно рынку нефти), а потому против долгосрочных
контрактов и принуждают всех перейти на краткосрочные договора (\enquote{спотовый}
рынок).

"У Штатов свои виды на газ: они хотят создать над этим рынком огромную
финансовую спекулятивную надстройку, которая поглотит часть долларового пузыря,
который угрожает американской экономике. Для этого расширяют долю сжиженного
газа, а поставщиков трубопроводного газа вынуждают переходить с долгосрочных
контрактов на рыночные механизмы продажи.

Россия занимает на рынке значительное место, но во время его обсуждения стороны
решили использовать украинский фактор. А ещё есть Британия, которая с
сателлитами занимает не последнее место на рынке газа, и которая, видимо, в
свою очередь решила просто построить военно-санкционный барьер между Россией и
Европой и использовать его в своих целях. Для этого они готовы устроить
какую-то гуманитарную катастрофу типа как делали в Сирии, и сначала на это
активно намекали. Но потом, не вписались в общий расклад", - пишет Димитриев. 

Ключевой момент в данной ситуации - это торги вокруг запуска Северного потока-2
и увеличение поставок российского газа в Европу по долгосрочным контрактам.
Россия ставит именно на них предлагая Европе следующие аргументы: цены на газ
по долгосрочным контрактам ниже, чем на спотовом рынке. И будут ниже еще долго.
Потому что в мире (и особенно в Европе) нарастает дефицит газа из-за отказа от
угольной электрогенерации. При том, что газ уже признан экологически чистым
топливом и в ближайшие десятилетия на него "зеленые" ограничения
распространяться не будут.

Поэтому Россия и предлагает зафиксировать Европе увеличение закупок газа в
долгосрочной перспективе. И, к слову, именно с этим связывает вопрос продления
транзита газа через Украину (через нее будут прокачиваться дополнительные
объемы и после запуска Северного потока-2). 

Американцы (а также англичане, связанные с добытчиками газа на Ближнем Востоке)
заинтересованы сами заполнить дефицит газа, но, по описанным выше причинам, по
краткосрочным контрактам. 

И потому, чтоб перебить экономические аргументы России (выгода для европейцев),
продвигают тему \enquote{вторжения России}, пытаясь таким образом отговорить европейцев
от заключения долгосрочных контрактов с \enquote{Газпромом} под лозунгом \enquote{избавления от
зависимости от российского газа}.

Именно поэтому они и накручивают ситуацию с \enquote{датами вторжения}, чтоб не дать
спасть напряжению в отношениях Европы и России (а в идеале и вообще
спровоцировать реальную войну). В общем торги идут очень большие и пока они не
закончатся тема с \enquote{вторжением} будет педалироваться и далее.

Самое интересное, что для Украины с чисто экономической точки зрения, было бы
выгодно, чтоб сработал именно российский сценарий поставок газа в Европу
(увеличение по долгосрочным контрактам). Потому что именно в таком случае
украинская труба не останется пустой даже после запуска Северного потока-2.
Поэтому довольно удивительно радость некоторых украинских экспертов по поводу
увеличения поставок американского сжиженного газа в Европу. Чем больше его там
будет, тем меньше шансов на то, что украинская труба останется заполненной. 
