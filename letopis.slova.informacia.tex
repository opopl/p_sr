% vim: keymap=russian-jcukenwin
%%beginhead 
 
%%file slova.informacia
%%parent slova
 
%%url 
 
%%author 
%%author_id 
%%author_url 
 
%%tags 
%%title 
 
%%endhead 
\chapter{Информация}
\label{sec:slova.informacia}

%%%cit
%%%cit_head
%%%cit_pic
%%%cit_text
А  шефиня Центра \emph{информационной} безопасности (ЦИБ) Людмила Цибульская
сообщила, что будет заниматься тремя направлениями. А именно: стратегическими
коммуникациями, противодействием \emph{дезинформации} путем постоянного
\emph{информирования} и адвокацией (чем-чем? – Г.Б.) усилий государства и
общественных организаций на международном уровне. И поведала, что орган успел
запустить несколько «важных проектов».  Интересно следующее. «ЦПД может
получать в установленном порядке от органов исполнительной власти,
правоохранительных и разведывательных органов, прокуратуры, органов местного
самоуправления, предприятий и организаций, независимо от форм собственности и
подчинения, статистические данные, \emph{информацию}, справочные материалы и
т.п.» То бишь любая \emph{информация}, представляющая интерес для СНБО и его
\emph{«дезинформационного»} отростка, будет использоваться, даже несмотря на
то, что персональные данные граждан Украины вроде бы защищены законодательством
%%%cit_comment
%%%cit_title
\citTitle{Как борьба с «дезой» на Украине множит сущности}, Георгий Бородин, odnarodyna.org, 07.01.2021
%%%endcit


%%%cit
%%%cit_head
%%%cit_pic
%%%cit_text
Прежде, чем мы попадём в эпоху \emph{Информационную}, которая вот-вот наступит,
или уже наступила… нам придется пожить в нынешней эпохе,
\emph{Дезинформационной}.  Этот этап неприятный, но вынужденный, – как
прелюдия, увертюра, очищение пространства и разгон (от слова разгоняться).
Зато потом!!!  Представляю, как классно будет жить в \emph{Информационной
эпохе}, которая станет гармоничной, умной и излучающей озарения!  Там все
учатся наукам.., все друг с другом обмениваются йотобайтами.., атмосфера и
стратосфера превращены в ноосферу... Красота!
%%%cit_comment
%%%cit_title
\citTitle{Нынешняя Дезинформационная эпоха когда-то превратится в Информационную}, 
Владимир Спиваковский, strana.ua, 06.08.2021
%%%endcit

%%%cit
%%%cit_head
%%%cit_pic
%%%cit_text
США проиграли войну во Вьетнаме.  Только они проиграли не потому, что были
разгромлены. Не потому, что были ослаблены и не могли дальше продолжать.  Они
проиграли по другой причине. В \emph{информационной} войне.  Знаете как это произошло?
В 1968 коммунистические лидеры Северного Вьетнама были на грани поражения. Они
понимали, что против мощи США им не выстоять и если текущая тенденция
сохранится, то вскоре войну они проиграют. Их просто уничтожат постоянными
бомбардировками. В лагере даже звучали призывы о переговорах. И ситуацию нужно
было срочно менять.  Как?
%%%cit_comment
%%%cit_title
\citTitle{В современной войне можно выиграть благодаря правильной картинке}, 
Павел Вернивский, strana.ua, 09.08.2021
%%%endcit

%%%cit
%%%cit_head
%%%cit_pic
\ifcmt
  pic https://strana.ua/img/forall/u/0/36/QIP_Shot_-_Screen_365.png
  width 0.4
\fi
%%%cit_text
Но подключилась более тяжелая артиллерия, чем блогеры и экс-нардепы.
Минобороны Украины обещало провести беседу с Магучих. Об этом в воскресенье, 8
августа, сообщила замминистра главы ведомства Анна Маляр.  Замминистра
напомнила, что Магучих является младшим лейтенантом вооруженных сил Украины.
Она призвала украинцев прекратить травлю спортсменки, дождавшись ее официальных
объяснений.  При этом Маляр утверждала, что неосторожное поведение спортсменов
может становиться объектом "\emph{информационных спецопераций врага}".
"Спортсмены, представляющие Украину на международных соревнованиях, должны
понимать, что в Украине продолжается российско-украинская война и это
накладывает определенные ограничения и ответственность.  Верное следование
спортивным традициям, к примеру – фото объятий со спортсменами-гражданами РФ,
когда они становятся призерами, сразу использует враг как
\emph{информационно-психологическую пушку}, которая демонстрирует миру, что "их
тут нет", - написала Маляр на своей странице в соцсети
%%%cit_comment
%%%cit_title
\citTitle{\enquote{Поддерживают дискурс вражды}. Как власти кошмарят олимпийскую призерку Магучих за фото с россиянкой}, 
Анна Копытько, strana.ua, 09.08.2021
%%%endcit

%%%cit
%%%cit_head
%%%cit_pic
%%%cit_text
Представителей Донбасса, по мнению, сочинявших эту ересь неадекватов, в
принципе быть не может, а потому их следует называть «приглашенными к работе в
ТКГ лицами в составе российской делегации».  Вместо «Белоруссия» — «Беларусь».
Вместо «Приднестровская Молдавская Республика» — «Приднестровский регион
Молдовы».  Заметьте, сочинителей это \emph{информационно} «пурги» граждане
содержат за счет своих налогов!
%%%cit_comment
%%%cit_title
\citTitle{Краткий словарь грантоедов под редакцией СНБО / Лента соцсетей / Страна}, 
Александр Карпец, strana.news, 26.10.2021
%%%endcit

%%%cit
%%%cit_head
%%%cit_pic
%%%cit_text
Компания также таргетировала рекламу на самых наивных людей, которые верят в
различные конспирологические теории и фейковые новости. Как мы вообще очутились
в потоке \emph{информации}, которой даже не можем доверять? Связано ли это с
бизнес-моделями платформ?  Как мы здесь очутились? Я думаю, этому
способствовало несколько факторов.  Во-первых, социальные сети, в отличие от
газет или телевизора, открыты. В них можно просто зарегистрироваться и начать
производить контент. Это лишает нас возможности верифицировать
\emph{информацию}. Алгоритмы соцсетей, напомню, нацелены на максимальную
вовлеченность пользователей. Это делает бизнес-модель соцсетей эффективной,
поскольку они зарабатывают на рекламе
%%%cit_comment
%%%cit_title
\citTitle{«У людей должен быть выбор» Как соцсети взламывают мозг человека и кто на самом деле их контролирует?: Coцсети: Интернет и СМИ: Lenta.ru}, 
Федор Тимофеев, lenta.ru, 28.10.2021
%%%endcit

%%%cit
%%%cit_head
%%%cit_pic
%%%cit_text
Руйнування налагодженої державної системи ніякими іншими \emph{інформаційними} силами
не замінити. Наш ворог озброєний неймовірним впливом на території України, в
тому числі і п’ятої колони. Він має програму, яку передає через наші
\emph{інформаційні} канали. А наш воїн йде на цю війну з голими руками. До цих пір не
створена програма для участі України у цій \emph{інформаційній} війні. А щоб
перемогти, треба діяти і на території ворога.  Усунення тієї проблеми, яка
існує для України на Донбасі, нам пропонується у двох напрямках. Або Україна
повністю відмовиться від цих територій, або терпітиме до тих пір, поки Москва
не "навчить" нас, яке рішення прийняти. Насправді на цій території тепер
необхідно діяти за іншою програмою з розумінням того, що так є частина людей,
яких треба рятувати, з якими треба працювати. А є частина, яких треба усунути
від можливості впливати на суспільну свідомість України.  Проблема
\emph{інформаційної війни} – це аналогія до практики збройних військових операцій, які
здійснюються під час звичайної війни. Коли ворог в тебе стріляє, треба давати
адекватну відповідь. Але ми не маємо з чого стріляти. Наш український
\emph{інформаційний простір} знекровлений
%%%cit_comment
%%%cit_title
\citTitle{Як виграти інформаційну війну? – Слово Просвіти}, , slovoprosvity.org, 01.11.2021
%%%endcit

%%%cit
%%%cit_head
%%%cit_pic
%%%cit_text
Это не запрет отрицания, а ртов затыкание.  Зелебобики продолжают отвлекать
внимание от кризиса в Украине идиотскими законопроектами. Социальная ситуация в
стране обостряется с каждым днем. Из-за нехватки топлива и его дороговизны
надвигается энергетический кризис, а зиму обещают холодную. Но «куриные мозги»
Зе-власти традиционно заняты не этими «мелочами», а «борьбой с Путиным».
Причем речь, как и водится, идет о своего рода «потешной» борьбе с соседним
супостатом и его проклятыми агентами в \emph{информационной} сфере, поскольку ехать на
Донбасс, чтобы повоевать по-настоящему под пулями и снарядами с риском остаться
там навсегда, представители «элитняка» явно не торопятся. Зато желающих
\emph{«информационно бороться»} с московской агрессией, сидя в уютных столичных офисах
и элитной недвижимости, — хоть отбавляй, о чем уже приходилось писать ( см.
«Словарный запас грантоедских зелебобиков и пляски на памяти о Великой войне».
В нем речь шла, напомним, о «глоссарии», в который включены слова, запрещенные
к употреблению (причем, одним одним из таких «кремлевских фейков» объявлена
Великая Отечественная война)
%%%cit_comment
%%%cit_title
\citTitle{Украину хотят окончательно превратить в патриотический зоопарк / Лента соцсетей / Страна}, 
Александр Карпец, strana.news, 11.11.2021
%%%endcit
