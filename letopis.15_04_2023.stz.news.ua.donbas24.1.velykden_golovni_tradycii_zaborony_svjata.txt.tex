% vim: keymap=russian-jcukenwin
%%beginhead 
 
%%file 15_04_2023.stz.news.ua.donbas24.1.velykden_golovni_tradycii_zaborony_svjata.txt
%%parent 15_04_2023.stz.news.ua.donbas24.1.velykden_golovni_tradycii_zaborony_svjata
 
%%url 
 
%%author_id 
%%date 
 
%%tags 
%%title 
 
%%endhead 

Ольга Демідко (Маріуполь)
15_04_2023.olga_demidko.donbas24.velykden_golovni_tradycii_zaborony_svjata
Україна,Великдень,date.15_04_2023

Великдень — головні традиції і заборони свята

На Великдень не можна сумувати та скупитися

Великдень вважається найдавнішим і найважливішим християнським святом, що
встановлено на честь Воскресіння Ісуса Христа. Дата Великодня кожного року
обчислюється за місячно-сонячним календарем, що робить це свято переходящим.
Цьогоріч православні віряни і греко-католики святкуватимуть Великдень 16
квітня. Багато хто вірить, що на Великдень відкриваються небеса. З огляду на це
народні цілителі вважають, що період великодніх свят — найкращий момент для
зцілення від хвороб.

Читайте також: Що не можна класти до великоднього кошика — заборонені продукти

Історія і традиції свята

Після того, як Ісус прийшов до Єрусалима, був зраджений і страчений на хресті,
а потім похований у печері відданими учнями — Він воскрес. За Євангеліями,
недільного дня, прийшовши до Його гробу, жінки-мироносиці виявили, що місце
поховання порожнє. Натомість побачили ангела, який сповістив їм, що Ісус
воскрес. Невдовзі одній із них явився Христос і повідомив про Своє воскресіння.
Радісна звістка надзвичайно швидко поширилася Єрусалимом, а потім і всією
Іудеєю. Саме тому в ніч проти неділі в усіх церквах правиться святкова
всенощна, кульмінацією якої є урочиста хода вірян навколо храму з запаленими
свічками, великодній подзвін і сповіщення священиком двохтисячолітньої звістки:
«Христос воскрес!», на що паства радісно відповідає: «Воістину воскрес!».

Читайте також: Великодній кошик-2023: скільки доведеться заплатити за святковий
набір продуктів

За традиціями на Великдень віряни відвідують церкву, освячують паски, воду,
продукти та крашанки. Освятити паски і крашанки потрібно до сходу сонця. У
церквах освячується і артос — квасний хліб. Його роздають вірянам для
зберігання вдома протягом року. Артос приймають натщесерце при хворобах. Також
є традиція цього дня стукатися крашанками. У деяких сім'ях на Великдень
обмінюються подарунками. У свято до батьків часто приїжджають діти, сім'ї
збираються великими компаніями за столом, спілкуються, відпочивають. Ще один з
атрибутів свята — квіти. Люди святять на службах живі або штучні букети. Потім
приносять їх додому, ставлять перед іконами, прикрашають ними святковий стіл.
Традиційними стравами на Великдень є холодець, домашні ковбаси, сало, запечене
молоде порося, гусак, фарширований яблуками, пироги з м’ясною і сирною
начинками. На святковій трапезі центральне місце займають паски, сирні паски та
фарбовані яйця. Починати трапезу потрібно саме з яєць. У деяких українських
сім'ях існує давня традиція вимірювати силу шляхом розбивання пасхальних яєць.
Вважається, що той член сім'ї, який розіб'є всі яйця і залишиться з цілим,
вважається найсильнішим.

Читайте також: Красиві привітання з Великоднем — у листівках, прозі та віршах
(ФОТО)

Головні заборони свята

— Забороняється займатися важкою фізичною працею. Не можна будувати, лагодити,
прибирати, прати. Саме тому господині намагаються напередодні завершити всі
приготування на кухні, щоб в день свята бути вільними від домашніх турбот.

— Не рекомендується сумувати, ходити похмурим, грубити, лаятися з близькими людьми, скупитися.

— Краще не відмовляти в милостині або допомозі нужденним.

— Крім того, на Великдень не можна вінчатися та відвідувати кладовище.

— Не слід викидати освячену їжу — її радять віддати тваринам чи спалити.

— На Великдень не можна святити:

алкоголь;
страви з крові тварин;
фрукти та овочі;
матеріальні цінності.

Читайте також: Що покласти у великодній кошик — список продуктів

Основні народні прикмети

На Великдень мороз — варто чекати на щедрий врожай.
Якщо вдень на Пасху тепло, літо буде сонячним.
Після освячення великодніх страв, потрібно роздати милостиню нужденним. За віруваннями, це допомагає залучити в дім достаток.
Дощ на Пасху — до дощової весни.
Якщо вдасться побачити великодній світанок, то слід очікувати удачу в справах.
Нагодувати вуличних птахів хлібними крихтами — весь рік буде супроводжувати удача і багатство.
Якщо на Великдень прохолодно, буде холодне літо.
До Великодня розтане весь сніг — буде гарний урожай.

Текст підготовлено за наступними джерелами: glavcom.ua, chas. news

Раніше Донбас24 розповідав, які сімейні ігри можна провести на Великдень.

Ще більше новин та найактуальніша інформація про Донецьку та Луганську області в нашому телеграм-каналі Донбас24

ФОТО: з відкритих джерел
