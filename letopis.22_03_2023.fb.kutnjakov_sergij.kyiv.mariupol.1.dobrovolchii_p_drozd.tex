%%beginhead 
 
%%file 22_03_2023.fb.kutnjakov_sergij.kyiv.mariupol.1.dobrovolchii_p_drozd
%%parent 22_03_2023
 
%%url https://www.facebook.com/permalink.php?story_fbid=pfbid02VTHahrxkZSNd3MLHNhd35WHbL4v3gm8wkFiWLZievG5CShQdCcToYo886DiZcPDal&id=100000943996304
 
%%author_id kutnjakov_sergij.kyiv.mariupol
%%date 22_03_2023
 
%%tags 
%%title Добровольчий підрозділ К-8 ТРО Києва, березень 2022-го
 
%%endhead 

\subsection{Добровольчий підрозділ К-8 ТРО Києва, березень 2022-го}
\label{sec:22_03_2023.fb.kutnjakov_sergij.kyiv.mariupol.1.dobrovolchii_p_drozd}

\Purl{https://www.facebook.com/permalink.php?story_fbid=pfbid02VTHahrxkZSNd3MLHNhd35WHbL4v3gm8wkFiWLZievG5CShQdCcToYo886DiZcPDal&id=100000943996304}
\ifcmt
 author_begin
   author_id kutnjakov_sergij.kyiv.mariupol
 author_end
\fi

На блокпосту в Святошинському районі столиці, до Ірпеня 11 км, добровольчий
підрозділ К-8 ТРО Києва, березень 2022-го. Коли записався, зброї ще не було,
тож на перше чергування прийшов з ножем із викидним лезом, захопивши по путі
дві каменюки. Ще та зброя в часи Майдану і первісних людей була*)

Думка, якщо з автівки при перевірці вискочить озброєне ДРГ, кинути влучно з
криком \enquote{Граната!} каменюку і під прикриттям захисних споруд ховатись за
будівлею. Але мені пощастило: на блокпост в той же день привезли чотири Калаша,
які ми з патронами передавали під розпис в журналі по вахті. Надія, що тепер
мали власну зброю або могли підібрати її у вбитих чи поранених, надихала.

%\ii{22_03_2023.fb.kutnjakov_sergij.kyiv.mariupol.1.dobrovolchii_p_drozd.pic.1}

Рух був двобічний жвавий, на кожному по два автоматника за укриттями + кілька
хлопців у резерві. Поки один перевіряє документи і спілкується з водієм, інший
з ліхтариком оглядає салон і багажник автівки чи мікроавтобуса. Третій за
командою відсовував ланцюг з колючками і після проїзду машини повертав його на
місце – ніхто прорватися без проколотих шин не зміг би.

Ще по одному хлопцю з пляшками бандерівського смузі і запальничками в руках
стояли за мішками з піском на кожному напрямку. Готовність була максимальна:
кількох диверсантів та їх транспорт знешкодили б. При більшому ворожому
підрозділі наші шанси були б мінімальні.

Заступав я завжди на нічне чергування. Спочатку з 20:00 до 4:00, потім зміна
відбувалась о 2-й ночі. Час зручний, бо всерівно не спав, бодрий, їсти не
хотілось. Народ не шастає, транспорт, окрім військових, поліції, СБУ і колон з
біженцями, не їздить – комендантська ж година. Зимно тільки було і канонади
арти і Градів + роботи нашої ППО в небі поруч або іноді над нами не стихали. Не
скучали")

\ii{22_03_2023.fb.kutnjakov_sergij.kyiv.mariupol.1.dobrovolchii_p_drozd.pic.1.large}

З хлопцями мінялись ролями, де з першої ночі попав на обхід з напарником
блокпосту на 50–100 метрів по периметру, аби вороженьки з темряви не
підібралися. Вікна майже всі потушені, ми під вуличними ліхтарями як на
долонці, підходи не проглядається, тож одразу з командирами домовився і вранці
видзвонив \enquote{Київміськсвітло}, аби приїхала вишка і викрутила лампи над
блокпостом. 

На диво черговий одразу дав мобільний директора, той пообіцяв бригаду в обід –
і ось наступної ночі ми в затемненні. Сусідній блокпост взнав, теж попросив –
зробив і йому. Коли розмовляв мобілою з командиром батальйону ТРО нашого району
– запропонував для підвищення маскування цю ініціативу і йому – той подякував,
підхопив. 

На пропозицію відзначити хоча б квітами дівчат, які готували нам каву–чай–їжу
(розмова була напередодні 8 березня) комбат приємно здивував: – Це комуняцьке
свято Клари Цеткін і Рози Люксембург, святкувати не будемо, крапка. Тож дівчат
нашої їдальні–кав'ярні в підземному переході та двох медсестричок відзначили
букетами, шоколадками і тортиком самі ~ заслужили. Кава-чай чи щось смачненьке
з їх рук о півночі біля бочок з багаттям – незабутнє для серця і шлунка, щирий
респект!*)

Подобалися нічні обходи прилеглого мікрорайону, територію якого поділили між
сусідніми блокпостами. Один автоматник залишався на цьому напрямку руху вулиці
у блокпосту, мене ж та інший АК-74 брав командир підрозділу. На маршруті кожен
з нас освітлював і уважно оглядав всі місцини між будинками і кіосками, де міг
би сховатися порушник комендантської години чи диверсант. 

Не сховалися б. Вікна, за поодинокими випадками, не горіли, ми їх оглядали на
предмет можливих світових сигналів ворогу, але даремно. Іноді хтось з нижніх
поверхів висовував голову у вікно і доповідав свої підозри щодо кольорового
блимання світла у віддаленому хмарочосі, що теж після колегіальних консультацій
знімалося з повістці дня. Новорічні гірлянди чи лампи з ритмічним міганням –
явище поширене, хоча для нічного бомбардування ще той орієнтир.

Відзначу, що наші обходи + інформатори, які нам про підозрілих осіб чи
порушників вночі дзвонили, а вдень прибігали на блокпост – все це
унеможливлювало мародерство і пересування потенційних ДРГ. Окрім кіоску з
алкоголем і табачними виробами, в який хтось проник ще до створення блокпосту,
інших фактів мародерства виявлено на наших і суміжних територіях не було. 

Це на противагу рідного Маріуполя, кинутого мером і більшістю його команди ще у
лютому 2022-го на розграбування. Сил ЗСУ, Нацполіції і скромних підрозділів
Тероборони там на внутрішню охорону і патрулювання критично не вистачало,
добровольців не залучали, їх було мало. Окрім роздачі продуктів, що без
електрики швидко псувались, організація системного розподілу залишків харчів за
однаковою нормою в руки майже не відбувалася. Хтось набивав у свою автівку і
накопичував тонни награбованих продуктів і побутових товарів, інші порядні
голодували, як моя сім'я. Маріуполь завдяки дезертирству і безконтрольності
Бойченка і оппо-влади перетворився в місто мародерів.

З моїми дівчатами півмісяця із оточеного міста зв'язку не було. Це вкрай
напружувало, тож у вільний від чергування час займався на блокпосту рубкою
дров, підтримкою вогню в бочках, прибиранням, переглядом новин. Також напружило
під час патрулювання, що один з автоматів був зі сточеним бойком, небоєздатний.
Тобто з кабінету НВП, навчальний, без штик-ножа.

– То що, мій ніж – це єдина в нас при обході зброя, хоч і холодна? - обережно
уточнюю я. Аргументи, що порушники комендантської години (про яких мова окрема)
цього не знають, а в рукопашному бою прикладом можна звалити з ніг парирую, що
буду ефективнішим зі своїм ножем, ломіком і дистанційно – з цеглиною нападнику
в лоб. Надалі, коли сам брав автомат, уточнював, чи не навчальний? Імітувати
дієву зброю на війні апріорі не треба")

Взагалі був один з небагатьох, хто мав армійський досвід, бував на
артилерійських, піхотних, авіаційних і військово-морських навчаннях та
полігонах. Стріляв, і таки влучно з АК, пістолета, кидав бойову гранату,
проходив вишкіл у різних родах військ, комунікував з бійцями і командирами в
зоні АТО-ООС. Ділився досвідом з молоддю і сам всмоктував нове з життя, новин,
ютуба.

15 березня мої дівчата вийшли на зв'язок і я впрягся у консультації щодо
маршруту пересувань азовським узбережжям і пошуку водіїв з Маріуполя, а потім з
Бердянська. Один виявився шахраєм: просив 100\% передоплати, ледь домовився за
100 грн/чол авансу. В час виїзду з окупованого міста чмо на зв'язок не вийшло,
мої і дівчат дзвінки скидувало. Хлопці на посту сказали, що за таке саперною
лопаткою забили б. Я ж пожалкував, що заяву на ушльопка з телефоном і рахунком
в СБУ/НПУ не скинув. Час і нерви тоді були дорожче.

Коли в середині березня з Гостомеля, Бучі, Ірпеня відкрили евакуаційний
коридор, нам додали наряд поліцейських. Хоча в них були АКСУ з короткими
стволами, та все ж надійніше, ніж нічого. Одного разу познайомився з кремезним
поліцейським, Олексієм звали. Тоді ж почув його розповідь, як в Ірпені, до якої
від нас хвилин 15 їзди, він з колегами взяв у полон бурята. Який, аби
згвалтувати 13-річну дівчинку, застрелив її батька.

На цьому моменті Олексій підніс руки до моєї шиї і, не торкаючись, сказав: \enquote{–
Власноруч його задушив}. Напружений той був, я ж спокійно це, дивлячись йому в
очі, пережив, про інші події розпитував. До звільнення Ірпеня і Бучі та
відкриття міжнародній спільноті злочинів російських воєнних залишалося днів
десять...

Якщо цікаво – продовжу розповіді. Окрема вдячність всім, хто тієї весни вистояв
зі мною біля кільцевої дороги столиці свої вахти – за українську мову! Її
намагалися відтворювати при перевірках і в розмовах всі російськомовні. Та
відточували в кодових словах типу \enquote{паляниця–залізниця} кожний з нас. Як отой
фільтр \enquote{свій–чужий}. Паролі й відгуки – то окрема сакральна пісня*)

Спочатку було Слово!..

%\ii{22_03_2023.fb.kutnjakov_sergij.kyiv.mariupol.1.dobrovolchii_p_drozd.cmt}
