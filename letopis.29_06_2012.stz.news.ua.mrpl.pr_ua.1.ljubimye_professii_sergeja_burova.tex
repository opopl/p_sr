% vim: keymap=russian-jcukenwin
%%beginhead 
 
%%file 29_06_2012.stz.news.ua.mrpl.pr_ua.1.ljubimye_professii_sergeja_burova
%%parent 29_06_2012
 
%%url https://pr.ua/news.php?new=19615
 
%%author_id news.ua.mrpl.pr_ua,harakoz_natalia.mariupol
%%date 
 
%%tags 
%%title Любимые профессии Сергея Бурова
 
%%endhead 
 
\subsection{Любимые профессии Сергея Бурова}
\label{sec:29_06_2012.stz.news.ua.mrpl.pr_ua.1.ljubimye_professii_sergeja_burova}
 
\Purl{https://pr.ua/news.php?new=19615}
\ifcmt
 author_begin
   author_id news.ua.mrpl.pr_ua,harakoz_natalia.mariupol
 author_end
\fi

\begin{qqquote}
Наверное, правомерно в знаменательные для каждого человека даты вспоминать
все, что им сделано значительного для людей, для общества, для времени, в
котором он живет... И счастлив тот, кто сумел реализовать свои способности,
свой жизненный потенциал в разных сферах...
\end{qqquote}

\textbf{ИМЯ} Сергея Давидовича Бурова хорошо знакомо мариупольцам и постоянным
читателям \enquote{Приазовского рабочего}. Его очерки с интересными и
познавательными фактами о старом Мариуполе и его замечательных людях, которые
оставили заметный след в истории города и страны, много раз публиковались на
страницах \enquote{При-азовского рабочего} и дочерней газеты
\enquote{Мариупольская неделя}.

Кроме того, как вспоминает Сергей Давидович, начиная с лета 1994 года волею
обстоятельств ему пришлось сотрудничать с телерадио-компа\hyp{}нией \enquote{Сигма},
где была задумана еженедельная программа \enquote{Мариуполь. Былое}. С того
часа и по настоящее время Сергей Давидович остается верен этой полюбившейся
многим передаче.

\ii{29_06_2012.stz.news.ua.mrpl.pr_ua.1.ljubimye_professii_sergeja_burova.pic.1}

А вот в 2003 году увидела свет первая книга автора \enquote{Мариуполь. Былое},
созданная по мотивам телефильмов и статей в периодике. Но за прошедшее время у
автора появилось много дополнительного материала, связанного с Мариуполем, с
новыми именами и фактами. И в 2011 году в ЧАО \enquote{Газета
\enquote{Приазовский рабочий}} выходит новая книга Сергея Бурова
\enquote{Мариуполь и мариупольцы}. Она стала столь же необходимой для
библиотек, учебных заведений и любителей истории, как и книга
\enquote{Мариуполь. Былое}.

Об этом свидетельствует хотя бы такая запись в Интернете от друзей и
почитателей, сделанная после выхода \enquote{Мариуполя и мариупольцев}:
\enquote{\enquote{Старый Мариуполь} поздравляет Сергея Давидовича с выходом
новой книги,  желает дальнейших творческих успехов! И надеется на дальнейшее
плодотворное сотрудничество! Ждем новых книг! Надеемся, что к этим пожеланиям
присоединятся все наши читатели!}

Я же особо радуюсь этому приятному событию в жизни давнего друга
\enquote{Приазовского рабочего} и литературного объединения \enquote{Азовье},
ведь мы с Сергеем Давидовичем начинали заниматься журналистским и литературным
творчеством в стенах \enquote{Приазовского} и поддерживаем дружеские и деловые
связи все эти годы... И сразу же после выхода второй книги Бурова \enquote{ПР}
откликнулся статьей \enquote{Минувшее проходит предо мною...} А сам Сергей
Давидович отмечает, что без поддержки главного редактора \enquote{Приазовского
рабочего} заслуженного журналиста Украины Николая Токарского и генерального
директора телекомпании \enquote{Сигма} Вадима Шкурупия вряд ли бы вышла вторая
его книга о старом Мариуполе...

\textbf{А ОТКУДА} пришло увлечение историей, краеведением? В предисловии к книге
\enquote{Мариуполь. Былое} Сергей Давидович вспоминает, что интерес и любовь к
родному городу исподволь привила ему мама Александра Петровна. Семилетним
мальчиком вместе с ней он впервые переступил порог Мариупольского
краеведческого музея, а несколько позже она подарила ему купленную по случаю
известную многим книгу \enquote{Мариуполь и его окрестности}. Многое о быте и
нравах жителей нашего города начала ХХ века почерпнуто им из устных рассказов
бабушки Татьяны Кирилловны Литвиненко. О событиях 30-50-х годов он узнал от
папы Давида Вениаминовича. С благодарностью отзывается Сергей Давидович и о
своей жене Валентине Ивановне, которая оказывает ему большую помощь в
подготовке материалов, с пониманием относится к его работе.

На его долю, как и на долю его многих сверстников, выпали полуголодное детство,
немецкая оккупация. Подробности этого периода жизни можно найти во второй книге
в очерках \enquote{Дом на Торговой}, \enquote{Бабушкины сказки}, \enquote{Звуки
прошлого}, в которых автор детально описывает свое военное и послевоенное
детство, которое прошло в дедушкином доме: \enquote{Окна в дедушкином доме были
большие... Увиденное в детстве из окна осело в памяти, как кадры кинофильма,
никогда никем не снятого... Немцы в городе, непрерывным потоком катятся тяжелые
грузовики, кузова закрыты тентами.  Этот грязно-зеленый поток непрерывен и
нескончаем... Идут люди, занимая всю ширину улицы, идут молча, лишь изредка
кто-нибудь из них перебрасывается короткими фразами с идущим рядом, лица у всех
отрешенные... Это потом станет известно, что мариупольские евреи будут
расстреляны у противотанкового рва близ поселка Агробаза}...

\textbf{КСТАТИ}, история краеведения в Мариуполе имеет давние корни. \enquote{А в настоящее
время краеведы объединены в городскую организацию  Национального Союза
краеведов Украины, - рассказала председатель организации, зав. отделом истории
досоветского периода краеведческого музея \textbf{Раиса БОЖКО}. – И Сергей Давидович
Буров является одним из наших активистов. Выступает с краеведческими докладами,
а в первом выпуске \enquote{Мариупольского краеведческого сборника}, выпущенного нашим
обществом краеведов, есть и его работа об А. И. Куинджи, во втором готовящемся
сборнике будет продолжение этой темы. Что касается его книг, то они ценны тем,
что являются первоисточниками, то есть основаны на личных знаниях и
воспоминаниях автора о событиях прошлого. А это немаловажно!}

К портрету юбиляра – почетного гражданина Мариуполя, члена Национального Союза
краеведов Украины и Национального Союза кинематографистов Украины Сергея
Бурова, которому 28 июня 2012 года исполнилось 75 лет, следует добавить, что
эта его эффективная деятельность по увековечению истории города Мариуполя
является как бы второй любимой профессией. Кандидат технических наук Сергей
Давидович Буров более 40 лет своей жизни посвятил промышленному производству. В
свободное от работы время занимался кинолюбительством, печатался в газетах и
литературно-публицистических и художественных сборниках, в том числе \enquote{Прибой},
\enquote{Мариуполь в созвездии Лиры} и др. В соавторстве с народной артисткой Украины
Светланой Отченашенко им написана книга \enquote{Из истории Мариупольского театра}.

Так и идут у него по жизни рядом металлургия, кинолюбительство, телевидение,
краеведение, журналистика... Скромный, интеллигентный человек, он не любит
говорить о своих заслугах, а всегда заботится о том, чтобы рассказать людям
что-то новое, интересное – будь то городская улица или дом, или личность,
привлекшая внимание общественности.

Наталья ХАРАКОЗ.
