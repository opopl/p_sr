% vim: keymap=russian-jcukenwin
%%beginhead 
 
%%file 29_04_2022.fb.fb_group.story_kiev_ua.1.hronika_harkov.cmt
%%parent 29_04_2022.fb.fb_group.story_kiev_ua.1.hronika_harkov
 
%%url 
 
%%author_id 
%%date 
 
%%tags 
%%title 
 
%%endhead 
\zzSecCmt

\begin{itemize} % {
\iusr{Татьяна Зубко Маркина}
Чудово написано про героїчний Харків. Дякую

\iusr{Inna Bermant}

\obeycr
Спасибо
Харьков болит очень
Больше других
Там же каждый переулок о чём-то напоминает
Выросла « на Госпроме» ,как у нас говорили .Университет ( училась там)
Дворец Труда,часто там бывала,в том самом разбомбленном дворе,папа работал в экскурсбюро.наверно от него и передалась эта любовь к городу
Не могу смотреть на эти фото без слез
Ещё раз спасибо .Фарид.
\restorecr

\ifcmt
  ig https://scontent-mxp1-1.xx.fbcdn.net/v/t39.30808-6/279174851_128422923119193_6207426979073182804_n.jpg?_nc_cat=110&ccb=1-5&_nc_sid=dbeb18&_nc_ohc=l8L5tCvQxKMAX8ocXp9&_nc_ht=scontent-mxp1-1.xx&oh=00_AT_llC2Ua1WZU1LK0feeu6cpIpuX2c8Onpvhb1ueW4-uhw&oe=62729E24
  @width 0.3
\fi

\iusr{Iryna Tsvelikh}
Місто-мрія... Завжди хотілось туди вернутись. Вірю, що ще не один раз побачу місто моєї любові

\iusr{Maria Aronova}
\textbf{Iryna Tsvelikh}

\ifcmt
  ig https://scontent-mxp1-1.xx.fbcdn.net/v/t39.1997-6/47270791_937342239796388_4222599360510164992_n.png?stp=cp0_dst-png_s110x80&_nc_cat=1&ccb=1-5&_nc_sid=ac3552&_nc_ohc=GTabmFLxV7YAX_ydnTP&_nc_ht=scontent-mxp1-1.xx&oh=00_AT_Ljiwh5dqsS9AHqjBzO6D3WME316azNsIg1TJZ8awYmg&oe=6272AEF2
  @width 0.1
\fi

\iusr{Alla Zakon}
Варвары!!! Ничего святого, ничего человеческого у них нет!!!

\iusr{Ludmila Ketsko}
Дякую. Такі щирі та теплі слова про місто, де я народилася.

\iusr{Anna Zagorulko}

Яке чудове місто! А зараз місто-мученик, місто-герой. Сльози на очах, коли
показують, як там висаджують квіти - це у місті, яке обстрілюють кожного дня!

\iusr{Alla Kenya}
Харькiв -це наша спiльна рана, це наш бiль та молитви @igg{fbicon.hands.pray}
@igg{fbicon.heart.blue}  @igg{fbicon.heart.yellow}  @igg{fbicon.biceps.flexed} 

\iusr{Katrin Odinets}
Поясніть мені, будь ласка, з якого дива Харків є першою столицею. В мене по датах не складається

\begin{itemize} % {
\iusr{Фарид Степанов}
\textbf{Katrin Odinets}
З 19 грудня 1919 по 24 червня 1934 рр. Харків був першою столицею Радянської України, звідси назва «перша столиця».

\iusr{Katrin Odinets}
\textbf{Фарид Степанов} 

ключове тут \enquote{радянської}. А першою з 1918 року був таки Київ. Сподіваюся, що
декомунізація зрештою і до першої столиці добереться  @igg{fbicon.smile} 

\iusr{Alexey Novozhylov}
\textbf{Фарид Степанов} 

існують такі сторінки в історії, якими не варто пишатися, принаймні притомним
людям, Харків і без цієї ганьби а-ля пєрвая сталіца залишається українським
містом-героєм.

\iusr{Фарид Степанов}
\textbf{Alexey Novozhylov}
Не зовсім розумію, в чому Ви мене намагаєтися переконати?

\iusr{Alexey Novozhylov}
\textbf{Фарид Степанов} не намагався переконати, просто сказав, забийте на це.
\end{itemize} % }

\iusr{Світлана Куликова}

\ifcmt
  ig https://i2.paste.pics/2ed4c2fe824f828ab2d4d24e1059a07d.png
  @width 0.2
\fi

\iusr{Mariya Butkovska}

Дякую, за такі теплі слова про Харьків і чудові фото ! Дуже красиве місто, не
хочеться вірить, що з ним зробили рашисти. Не має їм прощення.

\iusr{Princess Ofmyland}
\textbf{Alina Kirievsky} \textbf{Eva Brylynska} \textbf{Ally Kuperman} \textbf{Igor Likhosherstov}

\iusr{Elena Sapunova}

Были в этом прекрасном городе в чуть больше года назад. Прекрасный город, в
котором захотелось побывать ещё. Разделяю вашу боль...

\ifcmt
  ig https://scontent-mxp1-1.xx.fbcdn.net/v/t39.30808-6/279253829_1937906563076956_7312472402139941870_n.jpg?_nc_cat=111&ccb=1-5&_nc_sid=dbeb18&_nc_ohc=G1_YAIf-krgAX-EEYu5&_nc_ht=scontent-mxp1-1.xx&oh=00_AT8iSkE8ixCcdqdApUD8e26teBZDEV9jQYEMbCoJc25Ppg&oe=6273D910
  @width 0.3
\fi

\iusr{Валентин Багинский}

Так... Разделяю, сопереживаю, бывал неоднократно, сын - дирижёр в ХНАТОБе, с
трудом выехавший 3-5 марта под обстрелами -с двумя детьми...., и всё же вопрос
не совсем уместный: Зачем повторять советскую кальку \enquote{Первая столица}?)
Удач!

\begin{itemize} % {
\iusr{Фарид Степанов}
\textbf{Валентин Багинский}

Самі харків'яни так називають своє місто. Думаю, що ця назва вже і не
асоціюється з радянською дійсністю.

\iusr{Валентин Багинский}

) В курсе, что \enquote{называют}! Не правильно называют. Не истинно. Не надо
дублировать не-истину!)) Пожалуйста!

\begin{itemize} % {
\iusr{Фарид Степанов}
\textbf{Валентин Багинский}
Це, скоріш до самих харків'ян.
Вважаю, що не коректно мені киянину вказувати їм, як їм своє рідне місто називати.

\iusr{Валентин Багинский}

Понял! Так этот пост - продукт \enquote{коллективного разума} харьковчан! К ним
у меня нет вопросов!)

\iusr{Фарид Степанов}
\textbf{Валентин Багинский}
В каком месте должно быть смешно?

\iusr{Валентин Багинский}
\textbf{Фарид Степанов} 

Спасибо, что в будущем на клише \enquote{Первая столица} будете смотреть с должной
мерой критической рефлексии, и тем самым - не предлагать тратить время на
пустые разговоры.)) Здесь = Смех!))) Спасибо!

\iusr{Фарид Степанов}
\textbf{Валентин Багинский}

Так у нас вся речь состоит из клише.

Начиная от критического мышления, до пустых разговоров.

Текст был о Харькове. О моем к нему отношении и о боли, которую испытываю. Вы
же хотите со мной обсуждать экскурсы в историю. Не расположен сейчас это
делать. Хоть и историк по одному из образований. Слишком много сейчас истории в
жизни каждого украинца, чтобы ещё и устраивать священные войны между собой.

Хорошего дня. И мира нам всем.

\end{itemize} % }

\end{itemize} % }

\iusr{Олена Шелест}

Дякую. Підтримую Ваші думки. Також не можу повірити, що зруйнували всю ту
красу, яку з таким смаком створювали в Харкові. Його парки - це взірець
сучасного ландшафтного дизайну, високий європейський рівень.

\iusr{Таня Бер}

Пусть читают рус-е и удивляются:талантами простого народа, его смелостью,
дерзостью и свободой.... И да, обзавидуются...

\iusr{Тома Храповицкая}

Дякую Фарід... ніколи не була, але болить, дуже...і болить за коментарі про
столицю, перша чи... Людоньки схаменіться! Саме Це на часі? От саме зараз?
Дякую Фарід...

\iusr{Anton Koval}
Перепрошую, перша столиця чого?)

\iusr{Кузя Музя}
Тю! Щойно народився???

\iusr{Раиса Карчевская}

Фарид!

Большое спасибо за Вашу статью о Харькове. Я была там дважды и он мне очень
понравился. У меня болит сердце из'-за того, что орки сделали с ним. Настанет мир
и его обязательно отстроят

\iusr{Inna Bermant}
Символ моего Харькова - железобетон

\ifcmt
  ig https://scontent-mxp1-1.xx.fbcdn.net/v/t39.30808-6/279223168_128541559773996_3225905909464887787_n.jpg?_nc_cat=110&ccb=1-5&_nc_sid=dbeb18&_nc_ohc=nHNP6BNZ9mYAX-KBkCT&_nc_ht=scontent-mxp1-1.xx&oh=00_AT-LZc6mvB_BuhUz7Y_QBG9MwKtJ87ox-0nNJLNIKO7IqA&oe=6273CC0B
  @width 0.3
\fi


\end{itemize} % }
