% vim: keymap=russian-jcukenwin
%%beginhead 
 
%%file 13_08_2021.fb.bilchenko_evgenia.1.maguchih
%%parent 13_08_2021
 
%%url https://www.facebook.com/yevzhik/posts/4149364981765247
 
%%author Бильченко, Евгения
%%author_id bilchenko_evgenia
%%author_url 
 
%%tags __aug_2021.maguchih.foto.olimpiada.lasickene,bilchenko_evgenia,maguchih_jaroslava,razocharovanie
%%title БЖ. "Я буду вместо, вместо, вместо неё..."
 
%%endhead 
 
\subsection{БЖ. \enquote{Я буду вместо, вместо, вместо неё...}}
\label{sec:13_08_2021.fb.bilchenko_evgenia.1.maguchih}
 
\Purl{https://www.facebook.com/yevzhik/posts/4149364981765247}
\ifcmt
 author_begin
   author_id bilchenko_evgenia
 author_end
\fi

БЖ. "Я буду вместо, вместо, вместо неё..."

Очень больно было узнать об унизительном "покаянии" нашей спортсменки. Как
будто исчез ещё один образец мужества, за который можно было схватиться, чтобы
не умереть. В античной Греции спорт считался средством воспитания моральной
силы. И ещё древние считали, что "в здоровом теле здоровый дух". Наверное,
христианство показала, что языческая истина гармонии психики и физики - не
вполне верна: сильное тело не предполагает сильного духа и наоборот. 

\ifcmt
  pic https://scontent-cdg2-1.xx.fbcdn.net/v/t1.6435-9/237057289_4149364828431929_4634901171939102064_n.jpg?_nc_cat=104&ccb=1-5&_nc_sid=8bfeb9&_nc_ohc=7WsZB82OHk8AX_ShEJQ&_nc_ht=scontent-cdg2-1.xx&oh=e9f4555507d33a2612cf04b815d656b5&oe=613AED36
  width 0.4
\fi

Победители Олимпийских игр былых времен, которых наши бабушки и дедушки считали
настоящими героями, умели вкладывать в спорт нечто большее, чем политику и
бизнес: солидарность, честь, любовь, Родину. Сейчас - это просто политика на
службе рынка. Политика, рождающего конформистов. И, честно говоря,
экзистенциальную тошноту вызывают толерантные полуправды, направленные на то,
чтобы оправдать всё это. Анатолий Шарий очень резко высказался насчёт извинений
нашей победительницы, следует отметить. Очень. Но где его сермяжная правда,
если его журналист назначил мне встречу и не пришел? Где?

Мне особо ничего не надо, кроме двух вещей: не отмечать меня в тэгах об этой
спортсменке и позволить мне сказать, что, если ее уважаемая коллега из России
моему хилому тельцу позволит, безработный профессор из Украины обнимает ее
вместо нее.

\ii{13_08_2021.fb.bilchenko_evgenia.1.maguchih.cmt}
