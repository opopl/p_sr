% vim: keymap=russian-jcukenwin
%%beginhead 
 
%%file 02_03_2023.stz.news.ua.donbas24.1.arhiv_vijny_hto_zbyraje_svidchennja_mrplciv_kyiv
%%parent 02_03_2023
 
%%url https://donbas24.news/news/arxiv-viini-xto-zbiraje-svidcennya-mariupolciv-u-kijevi
 
%%author_id demidko_olga.mariupol,news.ua.donbas24
%%date 
 
%%tags 
%%title "Архів війни" — хто збирає свідчення маріупольців у Києві
 
%%endhead 
 
\subsection{\enquote{Архів війни} — хто збирає свідчення маріупольців у Києві}
\label{sec:02_03_2023.stz.news.ua.donbas24.1.arhiv_vijny_hto_zbyraje_svidchennja_mrplciv_kyiv}
 
\Purl{https://donbas24.news/news/arxiv-viini-xto-zbiraje-svidcennya-mariupolciv-u-kijevi}
\ifcmt
 author_begin
   author_id demidko_olga.mariupol,news.ua.donbas24
 author_end
\fi

\ii{02_03_2023.stz.news.ua.donbas24.1.arhiv_vijny_hto_zbyraje_svidchennja_mrplciv_kyiv.pic.front}
\begin{center}
  \em\bfseries\Large
Всі, хто став жертвою та свідком воєнних злочинів, можуть розповісти про це світові
\end{center}

Влітку 2022 року в Україні було започатковано проєкт \enquote{Архів війни}, створений
ГО \enquote{DOCU DAYS}. Це об'єднана база відео та аудіоматеріалів про \href{https://archive.org/details/24_02_2023.olga_demidko.donbas24.rik_vijny}{\emph{війну в Україні}}.%
\footnote{Рік повномасштабної війни: події, які увійшли в історію України, Ольга Демідко, donbas24.news, 24.02.2023, \par%
\url{https://donbas24.news/news/rik-povnomasstabnoyi-viini-podiyi-yaki-uviisli-v-istoriyu-ukrayini}, \par%
Internet Archive: \url{https://archive.org/details/24_02_2023.olga_demidko.donbas24.rik_vijny}%
} Представники проєкту зберігають та упорядковують воєнну хроніку. За допомогою
найпростіших засобів документування вони намагаються зробити воєнні злочини рф
видимими та покарати винних. У Києві в рамках \enquote{Архіву війни} маріупольці
діляться пережитим досвідом. Їхні свідчення збирають представники проєкту,
зокрема і актор театру та кіно, педагог і тренер Ігор Аронов.

\textbf{Читайте також:} \href{https://donbas24.news/news/ne-bacat-svitla-ta-misyacyami-ne-miyutsya-yak-zivut-diti-u-baxmuti-ta-kudi-yix-evakuyuyut}{\emph{Не бачать світла та місяцями не миються — як живуть діти у Бахмуті та куди їх евакуюють}}%
\footnote{Не бачать світла та місяцями не миються — як живуть діти у Бахмуті та куди їх евакуюють, Еліна Прокопчук, donbas24.news, 01.03.2023, \par%
\url{https://donbas24.news/news/ne-bacat-svitla-ta-misyacyami-ne-miyutsya-yak-zivut-diti-u-baxmuti-ta-kudi-yix-evakuyuyut}%
}

\subsubsection{Про координатора проєкту}

Про \href{https://ukrainewararchive.org}{проєкт \enquote{Архів війни}}%
\footnote{\url{https://ukrainewararchive.org}}
Ігорю Аронову розповіла знайома і він вирішив
долучитися. До речі, актор колись хотів стати футбольним журналістом. Ще у
студентські роки Ігор працював журналістом у \enquote{Спорт-Експрес в Україні}. З
огляду на це він мав досвід спілкування з людьми. Збирати свідчення почав
наприкінці червня. Вже опитав приблизно 50 — 60 осіб. Зустрічався з херсонцями,
чернігівцями, мешканцями Київської області. Проте найбільше спілкувався саме
маріупольців. Ці історії пов'язані з граничними для людського організму
психологічними, емоційними чи фізичними навантаженнями.

\ii{02_03_2023.stz.news.ua.donbas24.1.arhiv_vijny_hto_zbyraje_svidchennja_mrplciv_kyiv.pic.1}

\begin{leftbar}
\emph{\enquote{Насправді ці свідчення з перших вуст — є унікальною базою. Це важливо і для
нашої історії, і для подальших трибуналів. Маріупольці пережили великий стрес,
тому їхні розповіді відрізняються від інших}}, — наголосив Ігор.
\end{leftbar}

У актора є своя історія, адже повномасштабне вторгнення змусило змінити всі
плани, відмовитися від багатьох мистецьких проєктів. Зокрема, він є автором
освітньо-дослідницького проєкту \textbf{\enquote{Актор без прикриття}}, в рамках якого Ігор
проводить психофізичні воркшопи для акторів та всіх зацікавлених. Водночас
Аронов працював тренером у художніх фільмах і помічником режисера в театрі. У
березні 2022 року йому запропонували бути тренером акторської майстерності у
художньому фільмі. У квітні і травні планував закінчити зйомки в художньому
повнометражному українському фільмі \enquote{В зеніті}, де у нього була одна з головних
ролей.

\textbf{Читайте також:}\par\noindent \href{https://donbas24.news/news/minus-visim-odinic-yak-desantnik-spaliv-rosiisku-bronetexniku-pid-maryinkoyu-video}{\emph{Мінус вісім одиниць: як десантник спалив російську бронетехніку під Мар'їнкою}}%
\footnote{Мінус вісім одиниць: як десантник спалив російську бронетехніку під Мар'їнкою, Наталія Сорокіна, donbas24.news, 28.02.2023, \par%
\url{https://donbas24.news/news/minus-visim-odinic-yak-desantnik-spaliv-rosiisku-bronetexniku-pid-maryinkoyu-video}%
}

\ii{02_03_2023.stz.news.ua.donbas24.1.arhiv_vijny_hto_zbyraje_svidchennja_mrplciv_kyiv.pic.2}

Коли почалося повномасштабне вторгнення, Ігор брав участь у воркшопі з
традиційного співу, що проходив у Карпатах. Він не одразу міг потрапити до
рідного Києва і намагався робити все можливе, щоб бути корисним. Оскільки актор
не мав жодного військового досвіду, вирішив займатися волонтерською діяльністю.
Стала у нагоді багаторічна робота в Польщі, адже більше 10 років він
реалізовував там мистецькі проєкти. Так, протягом першого місяця війни Аронов
виконував багато волонтерської та координаторської роботи, спрямованої на
допомогу українцям в Польщі. Це була і цілеспрямована допомога біженцям, і
організація притулків та постачання військових речей з Польщі.

\begin{leftbar}
\emph{\enquote{Я викладав польську мову для українців та українську мову для поляків для
всіх охочих. Також я перекладав польські театральні п'єси для субтитрів, які
ставили в польських театрах для українців. Намагався спрямувати свою енергію в
правильному напрямі}}, — розповів актор.
\end{leftbar}

\textbf{Читайте також:} \href{https://donbas24.news/news/zitelka-krasnogorivki-cudom-vizila-pislya-togo-yak-vorozii-snaryad-pociliv-u-yiyi-budinok-video}{\emph{Жителька Красногорівки чудом вижила після того, як ворожий снаряд поцілив у її будинок}}%
\footnote{Жителька Красногорівки чудом вижила після того, як ворожий снаряд поцілив у її будинок, Тетяна Веремєєва, donbas24.news, 28.02.2023, \par%
\url{https://donbas24.news/news/zitelka-krasnogorivki-cudom-vizila-pislya-togo-yak-vorozii-snaryad-pociliv-u-yiyi-budinok-video}%
}

\ii{02_03_2023.stz.news.ua.donbas24.1.arhiv_vijny_hto_zbyraje_svidchennja_mrplciv_kyiv.pic.3}

\subsubsection{Про важливість проєкту \enquote{Архів війни}}

Загалом, за словами Ігоря Аронова, важливо наразі знайти способи, які
допоможуть прийняти нову реальність та намагатися бути корисними і щодня робити
все можливе для нашої спільної перемоги. Після отримання свідчень актор
продовжує спілкуватися з багатьма маріупольцями. Намагається бути їм чимось
корисним.

\begin{leftbar}
\emph{\enquote{Хочу зрозуміти, чим я можу допомогти, чим буду корисним конкретно цій людині}}, — поділився Ігор.
\end{leftbar}

\textbf{Читайте також:} \href{https://donbas24.news/news/20-ricna-divcina-iz-zakarpattya-cekaje-na-pervistka-vid-60-ricnogo-pereselencya-z-mariupolya}{\emph{20-річна дівчина із Закарпаття чекає на первістка від 60-річного переселенця з Маріуполя}}%
\footnote{20-річна дівчина із Закарпаття чекає на первістка від 60-річного переселенця з Маріуполя, Яна Квітка, donbas24.news, 23.02.2023, \par%
\url{https://donbas24.news/news/20-ricna-divcina-iz-zakarpattya-cekaje-na-pervistka-vid-60-ricnogo-pereselencya-z-mariupolya}%
}

\ii{02_03_2023.stz.news.ua.donbas24.1.arhiv_vijny_hto_zbyraje_svidchennja_mrplciv_kyiv.pic.4}

Найбільше актора вразила історія відомого маріупольця\par\noindent В'ячеслава Долженка, який
самотужки створив у Маріуполі унікальний музей, а у березні 2022 року дивом
врятував і свою 91-річну маму, і себе. Чоловік бачив, як російські окупанти
знищили і рідне місто, і справу всього його життя.

\begin{leftbar}
\emph{\enquote{Зараз В'ячеслав живе єдиною мрією — повернутись до Маріуполя і зробити
кращий музей, ніж він мав до цього. І багато хто вже пообіцяв свої предмети до
майбутнього музею. Вважаю, що про цю дивовижну людину мають знати}}, —
підкреслив Ігор. 
\end{leftbar}

Загалом цей проєкт дуже вплинув на актора, змінилися життєві цінності. Маючи
цей безпосередній контакт з таким горем і болем, вирішив реалізовувати нові
проєкти. Зокрема, Ігор планує створити соціально-мистецький проєкт, спрямований
на реабілітацію маріупольських студентів. Така ідея з'явилася у Ігоря давно,
але саме після спілкування з маріупольцями, які ділилися трагічними спогадами
виживання під час облоги міста чи виїзду з нього, він вирішив, що такий проєкт
набуває особливої актуальності саме зараз.

\begin{leftbar}
\emph{\enquote{У моєму житті часто з'являються певні обставини, які мене до чогось
підштовхують. Загалом зараз я відчуваю, що я повинен жити своїм життям і робити
все, щоб побудувати таке суспільство, державу, країну, які хотіли б побачити
загиблі герої}}, — підсумував Аронов. 
\end{leftbar}

\textbf{Читайте також:} \href{https://donbas24.news/news/dvoje-xlopcikiv-z-rubiznogo-znaisli-mati-yaka-bezvisti-propala-foto}{\emph{Двоє хлопчиків з Рубіжного знайшли мати, яка безвісти зникла}}%
\footnote{Двоє хлопчиків з Рубіжного знайшли мати, яка безвісти зникла, Яна Іванова, donbas24.news, 27.02.2023, \par%
\url{https://donbas24.news/news/dvoje-xlopcikiv-z-rubiznogo-znaisli-mati-yaka-bezvisti-propala-foto}%
}

\ii{02_03_2023.stz.news.ua.donbas24.1.arhiv_vijny_hto_zbyraje_svidchennja_mrplciv_kyiv.pic.5}

Для того, щоб долучитися до проєкту \enquote{Архів війни} та розповісти свою історію,
можна написати \href{https://www.facebook.com/aronov.san}{Ігорю Аронову}%
\footnote{\url{https://www.facebook.com/aronov.san}}
особисто в Facebook аба заповнити Google форму.%

\ifcmt
  ig https://i2.paste.pics/PSHLW.png?trs=1142e84a8812893e619f828af22a1d084584f26ffb97dd2bb11c85495ee994c5
  @wrap center
  @width 0.9
\fi

Адже сьогодні кожен відзнятий кадр, кожне голосове повідомлення — це частина
нової історії, яку ми творимо разом. А надані свідчення зможуть бути
використані проти агресора в кримінальних судах, журналістських розслідуваннях
та подальших наукових і мистецьких роботах.

Нагадаємо, раніше Донбас24 розповідав, що в Україні \href{https://donbas24.news/news/v-ukrayini-stvorili-sait-yakii-dopomoze-oznaiomitisya-z-naslidkami-viini-video}{\emph{створили сайт}},%
\footnote{В Україні створили сайт, який допоможе ознайомитися з наслідками війни, Тетяна Веремєєва, donbas24.news, 01.03.2023, \par\url{https://donbas24.news/news/v-ukrayini-stvorili-sait-yakii-dopomoze-oznaiomitisya-z-naslidkami-viini-video}}
який допоможе ознайомитися з наслідками війни.

Ще більше новин та найактуальніша інформація про Донецьку та Луганську області
в нашому телеграм-каналі Донбас24.

Фото: Степана Рудика, Афіни Хайї, Анастасії Телікової та з відкритих джерел

\ii{insert.author.demidko_olga}
%\ii{02_03_2023.stz.news.ua.donbas24.1.arhiv_vijny_hto_zbyraje_svidchennja_mrplciv_kyiv.txt}
