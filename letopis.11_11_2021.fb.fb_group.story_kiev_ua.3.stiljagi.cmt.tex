% vim: keymap=russian-jcukenwin
%%beginhead 
 
%%file 11_11_2021.fb.fb_group.story_kiev_ua.3.stiljagi.cmt
%%parent 11_11_2021.fb.fb_group.story_kiev_ua.3.stiljagi
 
%%url 
 
%%author_id 
%%date 
 
%%tags 
%%title 
 
%%endhead 
\zzSecCmt

\begin{itemize} % {
\iusr{Тома Храповицкая}

\ifcmt
  ig https://scontent-frt3-1.xx.fbcdn.net/v/t39.30808-6/255490103_1780228092187002_5959204929586824861_n.jpg?_nc_cat=104&ccb=1-5&_nc_sid=dbeb18&_nc_ohc=PlcRkPgXgykAX_4bkAP&_nc_ht=scontent-frt3-1.xx&oh=00_AT-h3ykbRDKrEyqt4yJ34VyEsETQh8AsZ3VetBkiVEkqZw&oe=61CB8D83
  @width 0.3
\fi

\begin{itemize} % {
\iusr{Тома Храповицкая}
Папа, 1939 года рождения.

\iusr{Людмила Гаценко}
\textbf{Тома Храповицкая}
Гарний, елегантний.

\iusr{Тома Храповицкая}
\textbf{Людмила Гаценко} Дякую Людмила!!!

\iusr{Анна Шустерман}
\textbf{Тома Храповицкая} , стиляга. @igg{fbicon.flame} 

\iusr{Тома Храповицкая}
\textbf{Анна Шустерман} Ага! Было такое, и брюки-дудочки, туфли на \enquote{манной каше}...

\iusr{Анна Шустерман}
\textbf{Тома Храповицкая} , асоциальный элемент. @igg{fbicon.wink} 

\iusr{Тома Храповицкая}
\textbf{Анна Шустерман} Было... песочили...

\iusr{Людмила Гаценко}
\textbf{Анна Шустерман}
У кращому розумінні цього слова.
Денді.

\iusr{Людмила Гаценко}
\textbf{Анна Шустерман}
І який же гарний і приємний, з розумом у погляді, елемент.

\iusr{Людмила Гаценко}
\textbf{Тома Храповицкая}
І шийна хусточка замість галстука. Як гарно.

\iusr{Тома Храповицкая}
\textbf{Людмила Гаценко} Дякую Вам Людмила, шийну хусточку татові подарувала полячка на фестивалі молоді і студентів у 1957 році.

\iusr{Людмила Гаценко}
\textbf{Тома Храповицкая}
А ще носили шовкові хусточки. Мій друг носив такі.

\end{itemize} % }

\iusr{Николай Бурчин}
40-45 см брюки это от чего до чего замер?  @igg{fbicon.smile} 

\begin{itemize} % {
\iusr{Yuriy Bubnov}
\textbf{Николай Бурчин} От сих до сих. @igg{fbicon.face.wink.tongue} 

\iusr{Оксана Гриневич}
\textbf{Николай Бурчин} нижняя ширина брючины

\iusr{Николай Бурчин}
\textbf{Оксана Гриневич} дружинники ходили в клешах полуметровых?  @igg{fbicon.face.nerd} 

\iusr{Оксана Гриневич}
\textbf{Николай Бурчин} да. Сначала самые продвинутые модники)))

\iusr{Людмила Гаценко}
\textbf{Mykola Burchin}

У 90-і були такі брюки модні, а у жінок широкі (довгі й короткі) називалися
брюки- спідниця. Зараз мода повернулася, і тепер жіночі кльошні брюки (і знову
ж таки довгі й короткі) називаються шоти.

\end{itemize} % }

\iusr{Тамара Ар}
Надо же, как всех сейчас интересует прошлое время!

\begin{itemize} % {
\iusr{Наталя Кудря}
\textbf{Toma Ar} 

думаю про цей час і згадати нічого буде- крім масок і сертифікатів!!
@igg{fbicon.wink}  тому минуле й цікаве- просто жили!!

\iusr{Тамара Ар}
\textbf{Наталя Кудря} Это прекрасно, на самом деле! Но, мое мнение, о прошлом или хорошо, или ничего.

\iusr{Наталя Кудря}
\textbf{Toma Ar} минуле- це те єдине що не можуть забрати в людини!!! ну крім Айцгеймера- але тоді вже байдуже...
\end{itemize} % }

\iusr{Игорь Печерный}
Вот вспомнил пару строк из песенки тех лет:

\obeycr
Мы идем по Уругваю
Ночь, хоть выколи глаза
Слышны крики попугаев
И мартышек голоса
Я иду с своей красоткой
Брюки дудочки на мне
Обезьянья походка
Папироса на губе
\restorecr

\iusr{Oksana Garnets}

Ещё туфли на «манной каше» - толстой белой подошве из каучука и очень пестрый
галстук, если рубаха и пиджак. У меня старший двоюродный брат был стилягой. Как
мне нравилось!

\iusr{Всеволод Цымбал}

Папа моего друга дядя Виталик в молодости на танцах в клубе в Дарнице сплясал
рок-н-ролл, за что был задержан дружинниками и записан в стиляги

\iusr{Ludmila Debourdeau}

Расскажу вам одну историю, которая приключилась с моими родителями в 1957 году.
Маме только исполнилось 17 лет, папе было 18. Они были настоящими стилягами.
Мама на свой день рождения попросила брюки. И пошли мама с папой на Крещатик,
мама очень довольная в новых брюках, а папа в клетчатом пиджаке, брюки дудочки
и кок на голове. На встречу им комсомольский патруль. Маме порезали вдоль её
подарок, а папе отрезали кок. Мама была безутешна. А папа потом ходил в
институт с перебинтованной головой пока волосы не отросли. Так что всё так и
было. сейчас они об этом вспоминают с юмором.

\begin{itemize} % {
\iusr{Всеволод Цымбал}
\textbf{Ludmila Debourdeau} Наши люди

\iusr{Max Gopencko}
Сейчас придёт какой-нибудь ярый фанат советчины и расскажет вам, что вы черните СССР.

\begin{itemize} % {
\iusr{Ludmila Debourdeau}
\textbf{Max Gopencko} 

я сама выросла в Советской союзе и первая вам скажу, что было и плохое, но и
очень много хорошего. Поэтому не стоит всё смешивать в один котёл. Мои родители
вспоминают это событие с юмором. Это была их молодость, впереди была оттепель,
лучшее время советской истории нашей страны. Вся жизнь впереди, в которой было
очень много хорошего и счастливого.


\iusr{Max Gopencko}
\textbf{Ludmila Debourdeau} 

здесь речь конкретно идёт о насилии над личностью, о травле непохожих. Считаете
это хорошим? А вчера одна мадам в комментариях к моей публикации горделиво
сказала, что в США все корни из СССР, на что я просто заметил, что в СССР не
было американских корней лишь потому, что все бежали только в одном
направлении.... и услышал в своё адрес, что я черню СССР и что какая же
всё-таки плохая Америка. Я об этом говорю, а в о чём?

Главное не путать тоску по собственной молодости, с тоской по государственному
строю, ибо многие путают.

\iusr{Наталя Кудря}
\textbf{Max Gopencko} 

як у вас будуть перевіряти сертифікат в транспорті-згадайте що советчини типу
немає @igg{fbicon.wink}  чи є?? при андропові теж перевіряли


\iusr{Max Gopencko}
\textbf{Наталя Кудря} 

може мені ще й прадіда репресованого згадати (ім'я якого троюрідний брат
реабілітував) і заморену голодом родину прабаці, яка все це пережила й встигла
ще й сопогадами поділитись зі мною?


\iusr{Наталя Кудря}
\textbf{Max Gopencko} 

так бачите різницю?? хіба не нащадки тих зараз при владі?? все по
колу.. особливо з Голодомором -на пенсію 1854 знаєте як зараз старенькі
виживають?? а все дорожчає й дорожчає.. а ще комуналка...-ті самі при владі

\iusr{Max Gopencko}
\textbf{Наталя Кудря} 

навряд я знайду спільну мову з людиною, яка порівнює фізичне знищення мільйонів
у 1920-х та 1930-х людей з нинішніми часами та намагається виправдати соціальну
травлю \enquote{інакших} якимись \enquote{позитивними моментами}.

\ifcmt
  ig https://i2.paste.pics/c43acef2294e6942b5f186e37fa6b8e3.png
  @width 0.2
\fi

\iusr{Наталя Кудря}
\textbf{Max Gopencko} розумію- сучасниї жаліти не так пафосно.. шкода що ви не знайомі з такими старенькими які виживають на 1854... і проблема купівлі палива на зиму для них просто космічна!! про те скільки зараз померло буде відомо потім- як і тоді до речі! спільної мови точно не знайти- та й навіщо??

\iusr{Наталя Кудря}
дурні гіфки якраз підходять до вашого коменту
\iusr{Max Gopencko}
\textbf{Наталя Кудря} а я вважаю, що дурні гіфки дуже підходять до вашого порівняння сучастності зі сталінськими репресіями.

\iusr{Max Gopencko}
Виправдовувати таким чином речі, про яки йдеться - це просто дно.

\iusr{Наталя Кудря}
\textbf{Max Gopencko} 

ви не в адекваті- які речі я виправдовую? 7 де ви це прочитали?? що курите то??
нема про що з вами спілкуватись!! я написала-що все як і було!!!! при владі всі
ті самі- вас це зачепило?? ви при владі?? таки дно


\iusr{Max Gopencko}
\textbf{Наталя Кудря} це ви в абсолютному неадекваті, судячи про те, що ви стали розповідати у відповідь на мій коментар.

\iusr{Max Gopencko}
зла морда насмішила  @igg{fbicon.face.tears.of.joy} 

\end{itemize} % }

\end{itemize} % }

\iusr{Dima Shkreba}
А борцофф-то и следа не осталось. Или?..

\iusr{Игорь Горовой}
\textbf{Dima Shkreba} Многие просто уже там. Но поискать можно.

\iusr{Надежда Лабик}
А, девушки в пышных юбках, танцующие буги-вуги, класс! Это этапы нашей жизни.

\ifcmt
  ig https://scontent-frx5-2.xx.fbcdn.net/v/t39.1997-6/s168x128/198679738_1581651558693954_1103589833244933008_n.png?_nc_cat=1&ccb=1-5&_nc_sid=ac3552&_nc_ohc=0deytPwwai0AX8OloGL&_nc_ht=scontent-frx5-2.xx&oh=00_AT8hMHqLkZMe6Qc7XGL9Y5krCJkAKTtFAkP0jdYMdSESsQ&oe=61CB0CB7
  @width 0.1
\fi

\iusr{Татьяна Сирота}

Да...были люди в наше время... @igg{fbicon.face.upside.down}{repeat=3} 

\obeycr
"...Бедные дружинники
глядят,
дрожа,
как синенькие джинсики
дают
дрозда.
Лысый с телехроники,
с ног чуть не валясь,
умоляет:
«Родненькие,
родненькие,
вальс!»
Но на просьбы робкие -
свист,
свист,
и танцуют
родненькие
твист,
твист..."
\restorecr

(Е.Евтушенко)

\iusr{Георгий Майоренко}

А есть ли более подробные сведения о Пятигорском? Жив ли он? Я дружил с одним
интереснейшим человеком (годился мне в отцы) - теневым бизнесменом 70-х, 80-х и
он мне рассказывал, что Пятигорский был знаменитостью среди киевских
\enquote{теневиков}.

\end{itemize} % }
