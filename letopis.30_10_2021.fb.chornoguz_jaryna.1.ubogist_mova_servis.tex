% vim: keymap=russian-jcukenwin
%%beginhead 
 
%%file 30_10_2021.fb.chornoguz_jaryna.1.ubogist_mova_servis
%%parent 30_10_2021
 
%%url https://www.facebook.com/yaryna.chornohuz/posts/4541685852588861
 
%%author_id chornoguz_jaryna
%%date 
 
%%tags jazyk,mova,obschestvo,obsluzhivanie,ukraina,ukrainizacia
%%title Скажіть мені, будь ласка, скільки ця убогість буде тривати тут?
 
%%endhead 
 
\subsection{Скажіть мені, будь ласка, скільки ця убогість буде тривати тут?}
\label{sec:30_10_2021.fb.chornoguz_jaryna.1.ubogist_mova_servis}
 
\Purl{https://www.facebook.com/yaryna.chornohuz/posts/4541685852588861}
\ifcmt
 author_begin
   author_id chornoguz_jaryna
 author_end
\fi

Знову стикнулася з ситуацією, яка змусила думати про те, як убезпечити себе в
нашій країні від того, щоб тобі раптом бува не надали послуги російськомовним
бидлостильним сервісом, коли ти щось замовляєш. Щоб коли ти замовляєш техніку,
яка важлива для виконання тобою завдання, тобі раптом не написали в
гарантійному талоні «Ирина» замість «Ярина», а потім огризались на твоє
зауваження з цього приводу «та я пішу клієнта как слишу па тєлєфону». Коли ти
просиш пояснити українською, як називаються ті налаштування, в які тобі треба
зайти, бо ти тупо не в‘їжджаєш, коли тобі це пояснюють російською, а в
працівника челюсть і мозок не здатні перелаштуватися і він кидає слухавку.

Наче б ухвалили Закон про державну мову давно, навіть участь трохи взяла в
цьому процесі.

І все одно ти мусиш кожного разу починати ще цю війну за своє право на
адекватний україномовний сервіс, аби користуватися купленим товаром, який тобі
потрібен для виконання бойового завдання. А в тебе є ще своя пряма війна, у
конкретному значенні цього слова, аби ще з усіма навколо воювати, щоб тобі
надавали послуги твоєю рідною мовою - мовою держави, за чию безпеку ти ризикуєш
своїм життям ніби як. До слова - коли я сказала працівнику, що я військова,
знаходжуся далеко не в тилу  і не маю фізичної можливості та часу бігати по
сервісних центрах - у нього просто нуль не масу. Ви думаєте він на українську
перейшов? Ні. Вийди з автобуса ще 150 раз, у цих людей нічого в башні не
міняється. Русскій мір он как хрєн по ходу - запах не виводиться нічим.

Я трохи закинула цю справу через службу на контракті - звертатися СУТО до тих
магазинів та Інтернет-магазинів, які надають послуги українською мовою, бо
іноді треба купити для військової справи деякі речі і про це немає часу думати.
Тепер я це відновлю. Україні потрібні мережі магазинів та інших сервісів
обслуговування, які матимуть україномовне обслуговування за свою фішку. І я
впевнена, що заробляли б вони цим багато.

Наразі на українському ринку досі україномовний клієнт змушений сам
вибудовувати собі базу з лояльних до його мови закладів і сервісів, де з ним
відразу заговорять українською, видадуть україномовний гарантійний талон та не
перероблятимуть твоє ім’я, прізвище  на російський лад.

Скажіть мені, будь ласка, скільки ця убогість буде тривати тут?

\ii{30_10_2021.fb.chornoguz_jaryna.1.ubogist_mova_servis.cmt}
