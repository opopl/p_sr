% vim: keymap=russian-jcukenwin
%%beginhead 
 
%%file 05_06_2021.fb.bilchenko_evgenia.1.impressionizm
%%parent 05_06_2021
 
%%url https://www.facebook.com/yevzhik/posts/3954497694585311
 
%%author Бильченко, Евгения
%%author_id bilchenko_evgenia
%%author_url 
 
%%tags 
%%title БЖ. Мой импрессионизм
 
%%endhead 
 
\subsection{БЖ. Мой импрессионизм}
\label{sec:05_06_2021.fb.bilchenko_evgenia.1.impressionizm}
\Purl{https://www.facebook.com/yevzhik/posts/3954497694585311}
\ifcmt
 author_begin
   author_id bilchenko_evgenia
 author_end
\fi

БЖ. Мой импрессионизм.

Слишком много на этом поле угольных синяков:
Больше, чем на моём лице вдавленных васильков -
Оттисков от прыщей. Помните, Валь и Коль,

Бурение нефти кожи - детскую дурь и боль,
Дабы нравиться мальчикам? Говорила ж бабушка: \enquote{Не дави!
Любые угри, даже на носу, - не препятствие для любви}.

Истинно, истинно, что в любви каждый урод есть царь.
Васильки выросли вместе с нами - синячищами на сердцах.
Рана - такой же прыщ. Сначала сочится кровью.
Потом набирается и вредит внешности и здоровью.

После мы выжимаем соки из памяти, быть красивыми
Возжелав: остаются пятна, небосводные, черно-синие.
И снова утро сквозь них растёт, погибая за честь, алея.

Когда я вижу багровый шов у неба на лбу, я млею:
Это значит, что будет чёрное - иудово на осине.
А потом - Стаханов и все дела. И Победа. И ярко-синие
Васильки...

5 июня 2021 г.

\ifcmt
  pic https://scontent-cdt1-1.xx.fbcdn.net/v/t1.6435-9/195287868_3954497641251983_3064179416172064077_n.jpg?_nc_cat=103&ccb=1-3&_nc_sid=8bfeb9&_nc_ohc=DwG1uFspWmMAX9KVqF-&_nc_ht=scontent-cdt1-1.xx&oh=08de7b54150be75e8a5f9294b78643b2&oe=60E0490D
\fi

\begin{itemize}
\iusr{Виталька Ульяненко}

Я так понял, пост надо читать полностью, дабы понять смысл прочитанного.

\iusr{Евгения Бильченко}

Виталька Ульяненко ну, да)

\iusr{Ирина Жуковская}

Женечка, Ваши стихи не из сора растут... Просто из синих рубцов. И из алых. Незаживших...

\iusr{Евгения Бильченко}

\textbf{Ирина Жуковская} спасибо, Ирочка!

\iusr{Світлана Яценко}

Чому Ви так міцно вчепилися за ту посаду професорки? Вільному поету це важливо?

\iusr{Евгения Бильченко}

\textbf{Світлана Яценко} да, чтобы вас раздражать. Зарабатываю я как свободный учёный и
вольный поэт раза в три больше. Но именно ярость нацистов убеждает меня:
\enquote{Упрямься дальше, БЖ, доведи их до открытой фатальной ошибки}.

\iusr{Евгения Бильченко}

\textbf{Світлана Яценко} они ж ко мне тайно ходят) те 55 студентов, чьи подписи вы
проигнорировали. И после увольнения, если вы красиво спляшете, я у них стану
живым Бродским. Кроме того, Европарламент даёт 60 дней на публичный ответ:
университет не ответил. И это вышло в ЕП за пределы нордической группы. В
общем, спасибо, реально. Спасибо неоднократно: за хуторянство, за ссору с ЕП и
имидж университета в мире, за неспособность доказать мою профнепригодность, за
обратный эффект, который судилище вызвало у массово пропускающих теперь пары
студентов, за отказ молодых четверокурсников поступать в магистратуру (а это -
ноль лицензии), в общем, за все ветки, которые за собой пилят украинские
\enquote{арийцы}. Могла б - приплачивала б вам.

\iusr{Світлана Яценко}

\textbf{Евгения Бильченко} Даремно це все. Ви помиляєтесь. Погляньте, ніхто
давно на Ваші пости не реагує. Ви давно всім \enquote{ворогам} байдужі. Ви ліпше від
інших знаєте про це, і саме тому продовжується підігрівати навколо себе. Нових
\enquote{Месій} було і буде... Так влаштований СВІТ

\iusr{Евгения Бильченко}

\textbf{Світлана Яценко} Так последуйте моим умным врагам: не реагируйте) всё
равно они все поняли, что со мной давление бесполезно. Свою функцию они уже
выполнили, за это им отдельная благодарность. А до вас никак не доходит.

\iusr{Світлана Яценко}

\textbf{Евгения Бильченко} Ні, мила моя, це до Вас не доходить. Ви погляньте, я
, зовсім Вам незнайома людина, провокую, а Ви миттю реагуєте. Отже, більше
нічим зайнятися))).  Озирніться, Ви ВСІМ, окрім "55" за Вашим же зізнанням,
давно байдужі. Лишайтесь з Богом, а про долю Бродського((, то Вам ще дуже
далеко

\end{itemize}
