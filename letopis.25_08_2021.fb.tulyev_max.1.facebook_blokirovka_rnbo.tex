% vim: keymap=russian-jcukenwin
%%beginhead 
 
%%file 25_08_2021.fb.tulyev_max.1.facebook_blokirovka_rnbo
%%parent 25_08_2021
 
%%url https://www.facebook.com/mt6561/posts/4465951440134699
 
%%author Тулиев, Максим
%%author_id tulyev_max
%%author_url 
 
%%tags facebook,internet,rnbo,strana_ua,ukraina
%%title Ну шо, фейсбучные френды, пора тут прощаться
 
%%endhead 
 
\subsection{Ну шо, фейсбучные френды, пора тут прощаться}
\label{sec:25_08_2021.fb.tulyev_max.1.facebook_blokirovka_rnbo}
 
\Purl{https://www.facebook.com/mt6561/posts/4465951440134699}
\ifcmt
 author_begin
   author_id tulyev_max
 author_end
\fi

Ну шо, фейсбучные френды, пора тут прощаться.

НКРЗІ разослал всем операторам и провайдерам (зачем-то) требование блокировать
ресурсы согласно списку из последнего решения РНБО (полный текст ниже).
Особенно подчёркивается необходимость блокировать странички в Фейсбуке. Которые
блокировать без блокировки Фейсбука целиком невозможно. Никак. Вообще. И походу
требование жёсткое. И походу большинство упоротых операторов его выполнит.

=== cut ===

Згідно з частиною третьою статті 5 Закону України «Про санкції» рішення щодо
застосування санкцій щодо окремих іноземних юридичних осіб, юридичних осіб, які
знаходяться під контролем іноземної юридичної особи чи фізичної особи -
нерезидента, іноземців, осіб без громадянства, а також суб'єктів, які
здійснюють терористичну діяльність (персональні санкції), передбачених пунктами
1 - 21, 23 - 25 частини першої статті 4 цього Закону, приймається Радою
національної безпеки та оборони України та вводиться в дію указом Президента
України. Відповідне рішення набирає чинності з моменту видання указу Президента
України і є обов'язковим до виконання.

Указами Президента України від 20 серпня 2021 року № 375/2021 та № 376/2021
уведено в дію рішення Ради національної безпеки і оборони України від 20 серпня
2021 року «Про застосування персональних спеціальних економічних та інших
обмежувальних заходів (санкцій)». 

Вказаними рішеннями Ради національної безпеки і оборони України серед видів
обмежувальних заходів, що застосовуються до фізичних та юридичних осіб
передбачено санкції, що відповідають принципам їх застосування, встановленим
Законом України «Про санкції», а саме блокування інтернет-провайдерами доступу
до відповідних веб-ресурсів, у тому числі до їх субдоменів та інших
веб-ресурсів, що забезпечують доступ до зазначених далі посилань/ресурсів,
зокрема щодо:

Товариство з обмеженою відповідальністю ''ІНФОРМАЦІЙНЕ АГЕНТСТВО ''ШАРІЙ.НЕТ'' (https://sharij.net);

Шарій Анатолій Анатолійович, Шарій (Бондаренко) Ольга, Олексіївна, Бондаренко
Алла Іванівна (https://sharij.net, https://sharij.com.ua,
https://www.youtube.com/channel/UCVPYbobPRzz0SjinWekjUBw,
https://www.youtube.com/user/SuperSharij,
https://www.facebook.com/anatolijsharij, https://www.facebook.com/sharijnet,
https://www.facebook.com/bondarenkoolyaa,
https://www.youtube.com/.../UCIp1KqtkGR11qSNsimJyV1fpJEBd5Q,
https://www.facebook.com/sharij.official, t.me/ASuperSharij,
twitter.com/sharijnet, twitter.com/anatoliisharii, t.me/OlgaSharij);

Товариство з обмеженою відповідальністю ''Ласмак'', Товариство з обмеженою
відповідальністю ''Маджері'', Товариство з обмеженою відповідальністю ''Смарт
медіамережа'' (https://strana.ua, Страна.UA - Новости Украины, @stranaua,
vk.com/ stranaua, ok.ru/stranaua), а також https://strana.news, та інших
веб-ресурсів, які забезпечують доступ до порталу Cтрана.ua, (товарного знаку,
його графічного відтворення).

Зазначені Укази Президента України набирають чинності з дня їх опублікування.

Контроль за виконанням рішень Ради національної безпеки і оборони України,
введеного в дію цими Указами, покладено на Секретаря Ради національної безпеки
і оборони України.

Пунктами 3 зазначених рішень Ради національної безпеки і оборони України
доручено Кабінету Міністрів України разом зі Службою безпеки України,
Адміністрацією Державної служби спеціального зв'язку та захисту інформації
України, Національним банком України, за участю Національної комісії, що
здійснює державне регулювання у сфері зв'язку та інформатизації, забезпечити
реалізацію і моніторинг ефективності персональних спеціальних економічних та
інших обмежувальних заходів (санкцій), передбачених пунктами 2 вказаних рішень.

Разом з тим, звертаємо увагу операторів, провайдерів телекомунікацій, що
статтею 39 Закону України «Про телекомунікації» передбачено обов’язок
здійснювати діяльність у сфері телекомунікацій відповідно до законодавства, за
умови включення до реєстру операторів, провайдерів телекомунікацій, а у
визначених законом випадках також за наявності відповідних ліцензій та/або
дозволів.

Слід зазначити, що зобов'язання операторів, провайдерів телекомунікацій
додержуватися вимог законодавства передбачено також:

пунктом 39 Правил надання та отримання телекомунікаційних послуг, затверджених
постановою Кабінету Міністрів України від 11.04.2012 № 295;

пунктом 3 Правил здійснення діяльності у сфері телекомунікацій, затверджених
рішенням НКРЗІ від 19.11.2019 № 541, зареєстрованих в Мін’юсті України
28.12.2019 за № 1309/34280.

З огляду на вказане інформуємо операторів, провайдерів телекомунікацій, що під
час проведення планового (позапланового) заходу державного нагляду (контролю)
щодо дотримання суб’єктом ринку телекомунікацій вимог законодавства у сфері
телекомунікацій, уповноваженими НКРЗІ для здійснення державного нагляду
перевіряється, зокрема, вимоги рішень Ради національної безпеки і оборони
України, які набрали чинності в установленому законом порядку, щодо обмеження
або припинення надання телекомунікаційних послуг і використання
телекомунікаційних мереж загального користування та інших санкцій в сфері
телекомунікацій, встановлених відповідно до пунктів 9, 25 частини першої статті
4 Закону України «Про санкції», що відповідають принципам їх застосування, в
тому числі заборони Інтернет - провайдерам надання послуг з доступу
користувачам мережі Інтернет до певних ресурсів/сервісів, блокування Інтернет -
провайдерами доступу до ресурсів, у тому числі до їх субдоменів (рішення НКРЗІ
від 29.09.2020 № 367 «Про затвердження уніфікованої форми акта, складеного за
результатами проведення планового (позапланового) заходу державного нагляду
(контролю) щодо дотримання суб’єктом ринку телекомунікацій вимог законодавства
у сфері телекомунікацій», зареєстроване в Мін’юсті 20.11.2020 за № 1151/35434).

При цьому, частиною першою статті 75 Закону України «Про телекомунікації»
встановлено, що особи, винні у порушенні законодавства про телекомунікації,
притягуються до цивільної, адміністративної, кримінальної відповідальності
відповідно до закону.

Зокрема, статтею 145 Кодексу України про адміністративні правопорушення
встановлено, що порушення умов і правил, що регламентують діяльність у сфері
телекомунікацій та користування радіочастотним ресурсом України, передбачену
ліцензіями, дозволами, повідомленням про початок здійснення діяльності у сфері
телекомунікацій, - тягне за собою накладення штрафу на посадових осіб
підприємств і організацій усіх форм власності у розмірі від п’ятисот до однієї
тисячі неоподатковуваних мінімумів доходів громадян.

Ті самі дії, вчинені повторно протягом року після накладення адміністративного
стягнення за порушення, передбачені частиною першою цієї статті, - тягнуть за
собою накладення штрафу на посадових осіб підприємств і організацій усіх форм
власності у розмірі від однієї тисячі до двох тисяч неоподатковуваних мінімумів
доходів громадян.

Крім того, згідно зі статтею 148-2 Кодексу України про адміністративні
правопорушення порушення порядку та умов надання послуг зв'язку в мережах
загального користування - тягне за собою накладення штрафу на посадових осіб у
розмірі від п'ятдесяти до ста неоподатковуваних мінімумів доходів громадян.

Враховуючи наведене, просимо здійснювати свою діяльність з урахуванням викладеного вище.

=== сге ===
