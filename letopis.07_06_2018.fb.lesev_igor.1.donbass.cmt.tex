% vim: keymap=russian-jcukenwin
%%beginhead 
 
%%file 07_06_2018.fb.lesev_igor.1.donbass.cmt
%%parent 07_06_2018.fb.lesev_igor.1.donbass
 
%%url 
 
%%author_id 
%%date 
 
%%tags 
%%title 
 
%%endhead 
\subsubsection{Коментарі}

\begin{itemize} % {
\iusr{Матвей Кублицкий}
боюсь что и со следующим президентом Украины нельзя будет договориться о ЛДНР. Слишком болезненный вопрос для внутренней политики

\begin{itemize} % {
\iusr{Игорь Лесев}
ЛДНР вернуть даже сложнее, чем Крым. Во втором случае, по крайней мере, нет кровников, а Донбасс всех повязал на крови

\iusr{Матвей Кублицкий}
\textbf{Игорь Лесев} 

плюс активное выталкивание ЛДНР в рублевую зону и разрыв экономических связей.
Если уж думать про возврат Донбасса - то вначале замораживание ситуации лет на
20 минимум, чтоб эмоции затихли, а потом уже переговоры. Кстати, в Грузии так и
планировалось, но Мишка-дурачок всё испортил

\iusr{Игорь Лесев}
\textbf{Матвей Кублицкий} здравые идеи, но здесь такое не в тренде. У нас принято друг друга эмоционально унижать, при этом с двух сторон. А потом удивляться, чего это оппоненты такие упертые. Но с ЛДНР пока что жопа тотальная.

\iusr{Матвей Кублицкий}
экономически у ЛДНР есть огромный плюс: отсутствие огромного внешнего долга и сохраненные связи с РФ.

\iusr{Игорь Лесев}
\textbf{Матвей Кублицкий} нет там никаких плюсов... заводы стоят, потому что а) сырьевая база в Криворожском бассейне осталась и единый цикл разбит и б) продукцию нельзя реализовывать на внешних рынках, а на внутреннем - они прямые конкуренты РФ. Так что ждет их разруха и тотальная деиндустриализация.

\iusr{Матвей Кублицкий}
\textbf{Игорь Лесев} не соглашусь. их заводы просто скупят российские конкуренты и вопрос с продажами за границу будет решен. какие-то навернуться, какие то продолжат работу. можем конкретно по предприятиям пройтись. А на Украине деиндустриализация уже произошла, к сожалению...

\iusr{Игорь Лесев}
так а чего соглашаться или не соглашаться, если в Донецке работы сейчас нет. Люди за 10 тыс. руб. считают круто работать. Чуть глубже заехать - там вообще мрак. Это реалии

\iusr{Матвей Кублицкий}
\textbf{Игорь Лесев} а в Сумской области или Ровенской? тоже все бегут за границу работать. Щас эта проблема всей Украины. низкие зарплаты. я же про перспективу лет в пять

\iusr{Игорь Лесев}

В Сумской или Ровенской области у чела есть паспорт, не летит над головой, а
также есть возможность вариативности - поехать в Киев, заняться бизнесом,
свалить накуй на заработки или вообще с концами, не опасаясь за свой дом,
который, кстати, можно еще и легально продать. В ЛДНР у пипла непонятки во всем


\iusr{Матвей Кублицкий}
\textbf{Игорь Лесев} 

даже не попытаюсь спорить. на сегодня именно так. Но при заморозке конфликта -
я же предположил лет пять - ситуация начнет менятся. Ну естессно, если местные
власти тупить не начнут. Они просто весь бизнес перестроят на россию.

\iusr{Матвей Кублицкий}
\textbf{Игорь Лесев} Кстати, как пример - абхазы. Сколько я читал что они еще будут завидовать грузинам. Фиг там. Живут в своем мирке и в ус не дуют. 2 млн туристов в год на 300 тысяч населения. Да и вся сельхозпродукция Россией скупается на корню.

\iusr{Игорь Лесев}

два миллиона там вряд ли, но а кули Абхазии с ее положение еще делать? растить
апельсины и принимать туристов. В ЛДНР туристы будут ехать также, как и в
Челябинск

\iusr{Матвей Кублицкий}
\textbf{Игорь Лесев} 

там больше двух миллионов каждый год. А Донбасс - металлургия и уголь. плюс
шикарная земля для сельского хозяйства. Наши только и ждут отмашку, чтоб там
поля засеять.

\iusr{Игорь Лесев}

я такой же встречный вопрос могу задать - а какие заводы работают под завязку?
донецкий металлургический, например, практически стоит... гугл вам в помощь, а
лучше пообщайтесь с местными

\iusr{Игорь Лесев}
\textbf{Матвей Кублицкий} если там будут поля засеивать, то Донбасс через несколько лет превратится в разновидность Сумской области

\iusr{Матвей Кублицкий}
\textbf{Игорь Лесев} потому как война - вот и стоит. а после войны его россияне к рукам махом приберут и запустят. а что в Сумской области с селским хозяйством?

\iusr{Игорь Лесев}

то же, что и в Черкасской или Винницкой... вопрос не в с/х, а в ментальности,
которая и отличает с/х регионы от индустриальных. Это вопрос цивилизационного
маркера

\iusr{Матвей Кублицкий}
\textbf{Игорь Лесев} 

ну если уходить в рассуждениях так далеко... То тут во всем мире оказывается
такое разделение. останется лишь несколько центров промышленности. и много
аграрных супердержав. Пример прибалтики и польши показателен. Украина вылетела
из соревнования. Россия пытается удержаться


\iusr{Елена Максюченко}
\textbf{Матвей Кублицкий} так прямо и ждут, все вокруг заминировано

\iusr{Матвей Кублицкий}
\textbf{Елена Максюченко} щас саперы российские Сирию разминируют и будут болтаться без дела. вот их и направят куда надо @igg{fbicon.laugh.rolling.floor}{repeat=3} 

% -------------------------------------
\ii{fbauth.maksjuchenko_elena.zaporozhie.ukraina.alchevsk}
% -------------------------------------

\textbf{Матвей Кублицкий} 

давайте пройдемся, Алчевский металлургический, для работы двух доменных,
кислородно конвертерного цеха, прокатного реально электроэнергии не хватает,
при запуске потоки перенаправляют, город без света, вода через день в городе,
на несколько часов, средний возраст в городе 50лет, молодежь пока есть, но
уезжает. На енакивском металлургическом крупная авария, кто и за какие деньги
будет восстанавливать, большой вопрос, в Луганске почти ничего не
работает... список можно продолжать, но мне нравится ваш оптимизм, устойчивый.
очень печально все

\iusr{Елена Максюченко}
\textbf{Матвей Кублицкий} "щас" это в категории вечности?

\iusr{Матвей Кублицкий}
\textbf{Елена Максюченко} 

ну электроэнергия - не проблема. это шикарный бизнес. так что линию точно с
ростовской области протянут. а по заводам - тут Игорь довольно точно сказал про
непонятки во всем. У вас ведь реально на сегодня даже нормальной власти нет.
Так что тут просто надо выжидать дальше и других вариантов пока не видно.
Сочуствую

\iusr{Марина Иванищенко}
\textbf{Игорь Лесев} ЕМЗ, КХЗ только в Енакиево. ЕМЗ (бывший в метинвесте), не на полную мощность, но тем не менее. Работает. И да. Я местная

\iusr{Елена Максюченко}
\textbf{Марина Иванищенко} Марин, что то случилось у вас, оборудование сгорело, мож уже сделали, друзья рассказывали, что серьезно, и да, я рядом, 70 км, но не сейчас

\iusr{Елена Максюченко}
\textbf{Матвей Кублицкий} прям завидую, вы такой оптимист!

\iusr{Марина Иванищенко}
\textbf{Елена Максюченко}, 

завод работает. Кадры решают всё, видимо, какие-то кадры понабирали по
обьявлению. Деталей не знаю. Но завод работает. Гудит, дымит родной. и соседи на
работу ходят. Будем жить

\iusr{Елена Максюченко}
\textbf{Марина Иванищенко} ну и слава богу, Енакиево привет, много там моих друзей, до июня 14 работала в Дарах природы, в Горловке, жила в Енакиево 2 года, на выходные, домой, в Алчевск

\iusr{Марина Иванищенко}
\textbf{Елена Максюченко} я в Алчевске училась)) привет Алчевску. Ваш метал завод заработал?

\iusr{Елена Максюченко}
\textbf{Марина Иванищенко} заработал, читайте выше, выплавили, прокатали, остановили. Мы сейчас в Запорожье, работа и очень старенькая мама, сама уже не справится
 · 3 г. · Отредактировано
Марина Иванищенко
Елена Максюченко, у нас, конечно, получше, конечно, чем в Алчевске. Хотя, понять не могу, у вас ни один снаряд, кроме авианалета на перевальск не упал. Наш завод только в 14 г на три месяца работал в тихом режиме. До прошлогодней блокады на полную мощность работал.
 · 3 г.
Елена Максюченко
Марина Иванищенко упал, к сожалению, и мой дом зацепил, а в остальном да, ниче хорошего, электроэнергии не хватает, да и воды нет
 · 3 г. · Отредактировано
Марина Иванищенко
Елена Максюченко, в Алчевске всю войну челентанно работало.) там войны не было. С водой беда там.лена, я через день общаюсь с Алчевском
 · 3 г.
Елена Максюченко
Марина Иванищенко а я каждый день, а Челентано работало и будет работать, там всегда были цены не высокие
 · 3 г.
Марина Иванищенко
Елена Максюченко в Енакиево только в 15 г возобновилось. У нас город вымерал в 12 дня. Нельзя сравнивать
 · 3 г.
Елена Максюченко
Марина Иванищенко да уж Марина, я все знаю, все плохо
 · 3 г.
Марина Иванищенко
Елена Максюченко все уже нормально. Детские площадки полны детьми, кафе отдыхающими.я со своим мопсом вечером выхожу гулять, молодёжь целуется, гуляет.красивые.моя кохана дитина в меде в Донецке учится. А в Донецке сейчас невероятно красиво. Розы
 · 3 г.
Елена Максюченко
Марина Иванищенко и в Луганске, сегодня рассказыали

\end{itemize} % }

\end{itemize} % }
