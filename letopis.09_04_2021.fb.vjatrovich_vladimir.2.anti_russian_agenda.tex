% vim: keymap=russian-jcukenwin
%%beginhead 
 
%%file 09_04_2021.fb.vjatrovich_vladimir.2.anti_russian_agenda
%%parent 09_04_2021
 
%%url https://www.facebook.com/volodymyr.viatrovych/posts/10219320768789459
 
%%author 
%%author_id 
%%author_url 
 
%%tags 
%%title 
 
%%endhead 
\subsection{ARA - Anti-Russian Agenda}
\Purl{https://www.facebook.com/volodymyr.viatrovych/posts/10219320768789459}

ARA.

Зовнішня політика за визначенням є дипломатичною. Але в момент наростання
небезпеки для країни - може і має стати гранично чіткою та навіть гострою.
Такою, як політика Черчилля, спрямована на перемогу над Третім Райхом. Такою,
як політика Рейгана проти «імперії зла» СРСР. Такою ж чіткою врешті має стати
політика України щодо Росії. 

Основою зовнішньої політики України має стати Anti-Russian Agenda.  

ARA - це єднання Україною світу для захисту свободи. 
ARA - це інформаційна робота про загрозу Росії для усього вільного світу. 
ARA - це участь України у міжнародних об’єднаннях, які опираються на цінності свободи, що ворожі для Росії
ARA - це творення міжнародних коаліцій з метою послаблення Росії. 
ARA - це скорочення її впливів аж до повної ізоляції Росії у світі. 
ARA - це підтримка визвольних рухів поневолених народів в середині Росії. 

Росія - не лише наш Карфаген, а й усього вільного світу.
