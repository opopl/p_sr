%%beginhead 
 
%%file 30_12_2022.fb.oshovskyj_viktor.kyiv.likar.akusher_ginekolog.1.ukra_na___tse_zavzhd
%%parent 30_12_2022
 
%%url https://www.facebook.com/viktor.oshovskyy.1/posts/pfbid0CrZm4xZZNE8KXoA4UgKjvtymwBRD2ezAAHaEeNFmiffLSZsgpwbQbFAQ5R7imoc3l
 
%%author_id oshovskyj_viktor.kyiv.likar.akusher_ginekolog
%%date 30_12_2022
 
%%tags medicina,volonter,dity
%%title Україна - це завжди про межовий досвід
 
%%endhead 

\subsection{Україна - це завжди про межовий досвід}
\label{sec:30_12_2022.fb.oshovskyj_viktor.kyiv.likar.akusher_ginekolog.1.ukra_na___tse_zavzhd}

\Purl{https://www.facebook.com/viktor.oshovskyy.1/posts/pfbid0CrZm4xZZNE8KXoA4UgKjvtymwBRD2ezAAHaEeNFmiffLSZsgpwbQbFAQ5R7imoc3l}
\ifcmt
 author_begin
   author_id oshovskyj_viktor.kyiv.likar.akusher_ginekolog
 author_end
\fi

\#встигнути\_подякувати

Україна - це завжди про межовий досвід, як влучно сказав наш неповторний
філософ \href{https://www.facebook.com/volodymyr.yermolenko}{Volodymyr Yermolenko}: \enquote{можливо, в цьому суть України - пройти по краю
життя, зачепити край смерті, не впасти в безодню смерті і повернутися з радістю
- знову до життя}. І мені не спаде на думку кращого визначення. Втім, лікарі
прикордоння відчули це особливо гостро. 

Колеги з Запорізького пологового №3 не припинили свою роботу ані на мить, хоча
розташовані безпосередньо на підстрільному краю міста. Справа в тому, що це
єдиний пологовий будинок області, не враховуючи перинатальний центр, що має
повний пакет дитячої допомоги НСЗУ, а також технічні і ресурсні можливості
надавати допомогу новонародженим. Сюди їдуть народжувати з усіх окупованих та
прифронтових сіл та містечок. І, попри потрощені будинки та вирви на суміжних
вулицях - хіба можна інакше? Авжеж - ні. 

Особливо, коли є ми з вами - здатні допомогти і підтримати відчайдухів, що
б'ються за наше майбутнє так само відважно, як воїни ЗСУ. Дякую всім колегам,
хто долучився до вирішення негайних проблем неонатального відділення, але
найбільше - нашим великим друзям з
\href{https://www.facebook.com/AkvelonUkraine}{Akvelon Ukraine} та ініціативи
United With Ukraine, що допомогли придбати компресор на якому тепер працюють
всі прилади ШВЛ та запас украй важливих катетерів для передчасних немовлят. Що
є темрява зовні, коли стільки світла всередині? 

Лікарям Херсонської обласної дитячої лікарні - повезло ще менше. Їхні долі та
долі маленьких пацієнтів тимчасово потрапили в руки болотяних покидьків. Це
вартує окремого допису - як колеги продовжували працювати між чисельними
допитами, під постійним психологічним тиском. Як правдами й неправдами
попереджали вивезення дітей з лікарень та інтернатів області. Як протидіяли
пропаганді серед маленьких громадян. Це страшні, але героїчні історії нашої
медицини, які ще варто буде озвучити та вшанувати. 

Ніщо не могло затьмарити радість звільнення, окрім, хіба власне темряви, яка
стала вірною, але підступною приятелькою для колег. Знеструмлення відділення
після чергового обстрілу - це зупинена робота апаратів штучного дихання,
інфузоматів, нагрівальних столиків, пульсоксиметрів. Це смерть та інвалідність
маленьких підопічних. Генератор, один на лікарню, не здатний швидко покрити всі
потреби безперебійно та з ідеальною синусоїдою струму, що потрібна для
примхливого обладнання. Автономія була можлива лише за наявності власних
електричних підстанцій, які забезпечать безперервну роботу відділення. І цей
проект нам теж вдалося закрити доволі швидко за допомогою друзів, які
долучилися до придбання, провівши майстер-класи у сферах, де вони є
неперевершеними. 

Дякую \href{https://www.facebook.com/kseniya.kryachko}{Kseniya Kryachko} за
вишуканість і красу, не знаю чи врятує вона світ, але зробить перебування в
ньому однозначно приємнішим. Дякую
\href{https://www.facebook.com/profile.php?id=100006566076129}{Ксения
Хажиленко} за таїнство нового життя, що виходить з твоїх рук - підхопимо, не
підведемо. Дякую \href{https://www.facebook.com/elena.susidko.5}{Elena Susidko}
за неприборкуване джерело енергії, що заряджає та надихає всіх довкола. 

Віднині три потужні станції зможуть перекрити всі потреби дитячого відділення у
випадку, коли безмозкі імпотенти знов захотять побрязкати зброєю. 

Дякую всім присутнім. Гайда до праці.
