% vim: keymap=russian-jcukenwin
%%beginhead 
 
%%file 27_09_2021.fb.blazhnova_rekuha_julia.1.mysli_posle_boja_usik
%%parent 27_09_2021
 
%%url https://www.facebook.com/yuliya.blazhnovarekukha/posts/2933701570201951
 
%%author_id blazhnova_rekuha_julia
%%date 
 
%%tags __sep_2021.usik.pobeda.dzhoshua,sport,ukraina,usik_aleksandr
%%title Мысли после боя
 
%%endhead 
 
\subsection{Мысли после боя}
\label{sec:27_09_2021.fb.blazhnova_rekuha_julia.1.mysli_posle_boja_usik}
 
\Purl{https://www.facebook.com/yuliya.blazhnovarekukha/posts/2933701570201951}
\ifcmt
 author_begin
   author_id blazhnova_rekuha_julia
 author_end
\fi

Мысли после боя 

\#хроникаджошуаусик 

Пресс-конференция закончилась в два часа ночи

Последним штрихом было небольшое общение с Джошуа после пресс-конференции,
оставившее приятный флёр) Он классный)

Дорога домой была длинной

Безумный бесконечный Лондон не спал той ночью

Под утро вышедшие из клубов, сидели на асфальте, шли, ехали. Метросексуалы,
бруталы, парни в белых рубашках и парни в женской одежде, безумный калейдоскоп
девушек в которых течёт кровь их родителей со всего мира и в эту тёплую почти
летнюю ночь их объединяли красивые открыто-откровенные наряды

\ifcmt
  pic https://scontent-yyz1-1.xx.fbcdn.net/v/t1.6435-9/242975005_2933701546868620_6224062916756917751_n.jpg?_nc_cat=110&ccb=1-5&_nc_sid=8bfeb9&_nc_ohc=s1CgNAL4qXUAX_MZyK0&_nc_ht=scontent-yyz1-1.xx&oh=4c2d204e9c758ed52d1052607be3aa26&oe=6179C9A0
  @width 0.8
\fi

Смотрела на них и думала, как им повезло, что у них сегодня есть возможность
болеть за спорт без лишних бэкграундов 

Сегодня они болели за Джошуа. А утром...А утром они снова будут болеть за
Джошуа. С досадой поражения. Но без ненависти. Без классовой, социальной,
рассовой, религиозной. Без любой ненависти 

Им не надо нервно следить за своими звёздами футбола и бокса, говорят они на
национальном английском языке или нет 

Не надо следить ходят они в церковь РПЦ или УПЦ

Не надо ждать, что бесплатно покажут бокс на BBC. Не покажут) Кстати, за
содержание которого платит каждое домохозяйство Британии

Не надо ждать от своих спортсменов чёткой политической позиции, которая
совпадает с текущей национальной доктриной

Это всё несомненно важно, но безусловно лучше, когда нет всех этих мытарств и
терзаний

Да, британцы тоже могут накинуть верёвку памятнику на шею и скинуть его в реку
в рамках современных тенденций за ошибки прошлого, но они умеют ценить героев
настоящего времени. Особенно спортсменов 

Поддерживать, верить, уважать 

Александр Усик сказал на пресс-конференции после победного боя с Энтони Джошуа,
что самая дорогая награда для него в жизни была одна - золото на Олимпиаде 2012

Тогда был гопак на ринге в Лондоне. И в субботу был гопак на ринге. Опять в
Лондоне) 

Однажды он рассказывал, что прилетел из Лондона с Олимпиады 14 августа, а 16
августа умер его отец

Они не успели встретиться после Олимпиады, он вложил своё олимпийское золото
отцу в руку. Уже не живому отцу...

Самое дорогое, что у него есть - золото Олимпиады 

Золото добытое под флагом Украины 

У него отец был военным. У братьев Кличко военный. У Андрея Шевченко военный

Мы все знаем украинскую реальность. Не трудно представить детство каждого.
Маленький населённый пункт, семья военного. Дисциплина. Безденежье. Спорт.
Перспективы 

Ежедневный труд простых пацанов, которые догребли до вершины Олимпа. Таки
перспективные оказались) 

Каждый шёл туда для себя. Но шли под флагом Украины 

И по их фамилиям идентифицируют Украину и украинцев миллионы людей по миру 

Ту Украину, которая всю свою историю сначала превозносит гетьманов, а потом
отрекается от них 

Только история такая интересная штука, что имена,  вписанные в неё однажды
золотом, остаются навсегда независимо от колебаний общественного мнения

Эти три довольных лица в раздевалке лондонского Tottenham Hotspur Stadium 25
сентября 2021 точно вошли в историю Украины  @igg{fbicon.flag.ukraina}
@igg{fbicon.hands.applause.yellow} 

Украинцы Андрей Шевченко, Виталий Кличко, Александр Усик

P.S. Благодарю за эксклюзивное фото  @igg{fbicon.face.happy.two.hands} 

Upd: кому показалось, что это про "какаяразница", вам показалось. Проходим
мимо, не создаём ажиотаж

\ii{27_09_2021.fb.blazhnova_rekuha_julia.1.mysli_posle_boja_usik.cmt}
