% vim: keymap=russian-jcukenwin
%%beginhead 
 
%%file 22_12_2018.stz.news.ua.mrpl_city.1.gostinica_kontinental_ekskurs_v_istoriu
%%parent 22_12_2018
 
%%url https://mrpl.city/blogs/view/gostinitsa-kontinentale-kskurs-v-istoriyu
 
%%author_id burov_sergij.mariupol,news.ua.mrpl_city
%%date 
 
%%tags 
%%title Гостиница "Континенталь" - экскурс в историю
 
%%endhead 
 
\subsection{Гостиница \enquote{Континенталь} - экскурс в историю}
\label{sec:22_12_2018.stz.news.ua.mrpl_city.1.gostinica_kontinental_ekskurs_v_istoriu}
 
\Purl{https://mrpl.city/blogs/view/gostinitsa-kontinentale-kskurs-v-istoriyu}
\ifcmt
 author_begin
   author_id burov_sergij.mariupol,news.ua.mrpl_city
 author_end
\fi

Это здание знали и знают все мариупольцы. Живущие ныне знают его как \href{https://mrpl.city/blogs/dk-molodezhnyj}{Дворец
культуры \enquote{Молодежный}}. Кто постарше - помнят, что здесь был Дворец культуры
комбината \enquote{Азовсталь}. Одно время он именовался \enquote{Дворец культуры и техники
завода \enquote{Азовсталь}}. Старожилы, переступившие 80-летний рубеж, расскажут, мол,
сразу после Второй мировой войны здесь находился Клуб металлургов. Что первый
этаж был занят спортзалом, где тренировались боксеры, акробаты и другие
спортсмены. А позже, с конца 50-х годов, опустевший спортзал занял магазин
\enquote{Гастроном}. Что же касается престарелых аборигенов – мариупольцев в третьем
или даже в более старшем поколении - то они скажут: \enquote{Здесь была гостиница
\enquote{Континенталь}, мне дедушка рассказывал}. Существуют непроверенные сведения,
что строительство этого здания было завершено в 1897 г.

\textbf{Читайте также:} 

\href{https://mrpl.city/news/view/v-mariupole-dk-molodezhnyj-prevratyat-v-otel-kontinental}{%
В Мариуполе ДК \enquote{Молодежный} превратят в \enquote{Отель Континенталь}, Роман Катріч, mrpl.city, 07.10.2018}

% Начало XX в.
\ii{22_12_2018.stz.news.ua.mrpl_city.1.gostinica_kontinental_ekskurs_v_istoriu.pic.1}

Заместитель директора Мариупольского краеведческого музея Р. П. Божко, многие
годы посвятившая изучению истории родного края, писала: \enquote{Несомненным
лидером в области гостиничного бизнеса была \enquote{Континенталь}. В конце
XIX века ее содержал мещанин А. Македон. Она располагалась в доме Елизаветы
Томазо}. (Мариуполь и его окрестности: взгляд из XXI века: - Мариуполь, изд.
\enquote{Рената}, 2006). Через какое-то время аренду дома под гостиницу
перекупил Д. А. Калери, номера которой занимали второй и третий этажи.

На первом этаже были магазины, но кому они принадлежали и чем они торговали,
установить пока не удалось. Часть этого этажа и подвал арендовали братья
Гольдрин для своей электротипографии. Кстати, именно в этом заведении было
отпечатано, по меньшей мере, восемьдесят выпусков почтовых открыток с
различными видами нашего города. Во всяком случае, столько их перечислено в
каталоге коллекционера и краеведа Леонида Морозова \enquote{Мариуполь на видовых
открытках}. В этом же заведении в 1911 году был отпечатан сборник стихотворения
Дмитрия Афанасьевича Лухманова, в то время помощника начальника Мариупольского
торгового порта. Он также известный мореплаватель, сотрудник Наркомата морского
флота, преподаватель морского дела, писатель-маринист.

\textbf{Читайте также:} 

\href{https://mrpl.city/news/view/kak-priglasit-zvezdu-v-mariupol-master-klass-ot-vedushhego-intervyu}{%
Как пригласить звезду в Мариуполь? Мастер-класс от ведущего, Яна Іванова, mrpl.city, 22.12.2018}

Пожалуй, в первом десятилетии к основному зданию была сделана пристройка.
Наверху соорудили концертный зал, а внизу было отведено место для ресторана.
Приходилось слышать, что акустика зала была столь хороша, что певцы и
оркестранты во время своих выступлений обходились без микрофонов. Впрочем их
тогда и не было. Сейчас не вспомнить, в каком номере газеты \enquote{Мариупольская
жизнь} промелькнуло объявление о гастролях женского симфонического оркестра под
управлением женщины-дирижера...

Слом эпох. 19 июля (1 августа) 1914 года Германия объявила войну России. На
следующий день – 20 июля (2 августа) Россия объявила войну Австро-Венгрии -
союзнице Германии. Началась Первая мировая война. Мариуполь находился в
глубоком тылу. Но война и здесь давала о себе знать. Прежде всего, мобилизация
военнообязанных в армию. В город стали прибывать беженцы из западных губерний
Российской империи. На заводе Никополь-Мариупольского горного и
металлургического общества был пущен броневой прокатный стан. С началом войны в
Российской империи был принят \enquote{сухой закон}, запрещавший производство и продажу
спиртных напитков. Мариупольский казенный винный склад, известный нам как
ликеро-водочный завод, остановился. В опустевших помещениях поместили военный
госпиталь. 15 марта (28 марта) 1915 года в концертном зале гостиницы
\enquote{Континенталь} с лекцией на военную тему выступил преподаватель Чугуевского
училища полковник Савченко. 15 ноября 1916 года на Георгиевской улице
состоялось открытие Екатеринославского губернского учебно-ремесленного
дома-приюта для инвалидов Первой мировой войны. Теперь в этом здании находится
Мариупольский краеведческий музей.

2 марта (по старому стилю) 1917 года телеграфист станции Мариуполь получил
телеграмму, что царь Николай II отрекся от престола, что свершилась Февральская
революция. В Мариуполе, как и во всей бывшей Российской империи, началась
борьба за власть. Центральная рада для усмирения направила в наш город 200
гайдамаков, которых разместили в гостинице \enquote{Континенталь}. Не вдаваясь в
подробности, можно сказать, что 30 декабря 1917 г. толпа, поверившая агитации
большевиков, двинулась на штурм гостиницы. Силы были неравными, и гайдамакам
пришлось сложить оружие. Большевики пришли к власти. Отметим только, что в 1920
году в стенах гостиницы \enquote{Континенталь} размещался штаб начальника морских сил
Черного и Азовского морей.

\textbf{Читайте также:} 

\href{https://mrpl.city/news/view/budet-li-obedinenie-mariupolskih-pravoslavnyh-foto}{%
Будет ли объединение мариупольских православных?, Ярослав Герасименко, mrpl.city, 21.12.2018}

% Дом труда. 20-е годы XX в.
\ii{22_12_2018.stz.news.ua.mrpl_city.1.gostinica_kontinental_ekskurs_v_istoriu.pic.2}

Пропустим несколько страниц истории, поскольку они не касаются \enquote{Континенталя}.
После Гражданской войны первый этаж \enquote{Континенталя} был занят магазинами
Церабкопа – Центрального рабочего кооператива. А номера гостиницы на втором и
третьем этажах были заняты отделами различных профсоюзов. С начала
строительства завода \enquote{Азовсталь} гостиница была передана строящемуся заводу. С
1933 года здесь был организован клуб металлургов. Пока не удалось найти
материалы о деятельности этого клуба.

22 июня 1941 года. Гитлеровская Германия напала на Советский Союз.

А 8 октября 1941 года Мариуполь был занят немцами. Началась оккупация. К
сожалению, сведений о событиях, связанных с Клубом металлургов завода
\enquote{Азовсталь} в трагические дни оккупации, не удалось разыскать. Гитлеровцы
сожгли здание этого клуба. И только к 7 ноября 1946 года удалось восстановить
первую очередь Дворца культуры металлургов завода \enquote{Азовсталь}. В заново
отделанном концертном зале состоялся концерт самодеятельных артистов. Авторами
проекта восстановления Дворца были архитекторы Мариупольского филиала Гипромеза
А. А. Крещановский и В. К. Царапкин.

\textbf{Читайте также:} 

\href{https://mrpl.city/blogs/view/v-dk-molodezhnyj-pohitili-dzhenni}{%
В ДК \enquote{Молодежный} похитили Дженни!, ДК Молодежный, mrpl.city, 30.10.2018}

Много можно узнать об этом здании из многочисленных публикаций научных
сотрудников Мариупольского краеведческого музея и краеведов. Для продолжения
повествования воспользуемся сведениями из книжки Аркадия Дмитриевича Проценко
\enquote{Азовстальские истории}. Аркадий Дмитриевич – почетный гражданин Мариуполя,
член Союза журналистов Украины, краевед и писатель, коллекционер, библиофил. Но
в данном случае - главное, что он более сорока лет руководил Дворцом культуры
\enquote{Азовсталь}. Более того, задолго до того как стал директором, он занимался
пением в вокальном коллективе Дворца.

\ii{22_12_2018.stz.news.ua.mrpl_city.1.gostinica_kontinental_ekskurs_v_istoriu.pic.3}

В книжке Аркадия Дмитриевича рассказано о творческих коллективах Дворца. Одна
из глав книги посвящена встречам здесь с замечательными людьми. Классиком
украинской литературы Иваном Ле, композиторами Юрием Милютиным и Дмитрием
Кабалевским, выдающимся музыкантом Мстиславом Ростроповичем, киноартистами
Борисом Андреевым и Марком Бернесом.

К этому нужно добавить, что во Дворце культуры была художественная студия,
которой руководил талантливый художник и педагог Яков Иванович Никодимов. Среди
его учеников - заслуженные художники Украины Олег Ковалев и Евгений Скорлупин,
члены Национального союза художников Украины Юрий Муравьев, Александр
Бондаренко, бывший заместитель главного архитектора города Нина Горлышева,
профессиональные художники Георгий Скрипник, Эдуард Сахаров и многие другие.

При Дворце культуры были духовой оркестр под руководством Николая Васильевича
Попова, который не только научил многих ребят играть на музыкальных
инструментах, но отвлек от улицы. Народный драматический театр, которым в
разное время руководили артист П. И. Виноградов, заслуженный артист Туркмении
А. Я. Плетнев, заслуженный работник культуры УССР Л. А. Бессарабов. Этому
коллективу А. Д. Проценко посвятил книгу \enquote{Народному драматическому театру
\enquote{Азовстали} 70 лет}.

\ii{22_12_2018.stz.news.ua.mrpl_city.1.gostinica_kontinental_ekskurs_v_istoriu.pic.4}

Когда к комбинату \enquote{Азовсталь} был присоединен \enquote{Маркохим}, у
предприятия оказалось два Дворца культуры. Какой из них оставить за
объединенным предприятием? Выбор пал на маркохимовский очаг культуры, открытый
1 мая 1958 года. А известный многим поколениям мариупольцев Дворец культуры
комбината \enquote{Азовсталь} остался как бы бесхозным.

Не будем углубляться в подробности, но финал оказался прекрасным. Нашлись одни
мудрые люди, принявшие правильное решение, а люди с добрым сердцем дали
средства для ремонта исторического здания.

И теперь мариупольская молодежь получила великолепное место для развития своих
способностей и талантов. Между прочим, не многие города нашей страны могут
похвастаться таким дворцом.

\textbf{Читайте также:} 

\href{https://mrpl.city/news/view/prevrashhenie-molodezhnogo-v-otel-kontinental-v-mariupole-zajmet-ot-shesti-mesyatsev}{%
Превращение \enquote{Молодежного} в \enquote{Отель Континенталь} в Мариуполе займет от шести месяцев, Роман Катріч, mrpl.city, 25.11.2018}
