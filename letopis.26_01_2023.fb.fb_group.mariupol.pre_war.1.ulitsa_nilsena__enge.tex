%%beginhead 
 
%%file 26_01_2023.fb.fb_group.mariupol.pre_war.1.ulitsa_nilsena__enge
%%parent 26_01_2023
 
%%url https://www.facebook.com/groups/1233789547361300/posts/1397309117676008
 
%%author_id fb_group.mariupol.pre_war,elena_mariupolskaja
%%date 26_01_2023
 
%%tags mariupol,istoria,azovstal
%%title УЛИЦА НИЛЬСЕНА (ЭНГЕЛЬСА), 39
 
%%endhead 

\subsection{УЛИЦА НИЛЬСЕНА (ЭНГЕЛЬСА), 39}
\label{sec:26_01_2023.fb.fb_group.mariupol.pre_war.1.ulitsa_nilsena__enge}
 
\Purl{https://www.facebook.com/groups/1233789547361300/posts/1397309117676008}
\ifcmt
 author_begin
   author_id fb_group.mariupol.pre_war,elena_mariupolskaja
 author_end
\fi

УЛИЦА НИЛЬСЕНА (ЭНГЕЛЬСА), 39

В предвоенные годы дома выше двух этажей в Мариуполе можно было пересчитать по
пальцам одной руки. У некоторых из них даже были неофициальные собственные
имена. Тот, что у городского сада – «Коксохимовский», а что за сквером –
«Азовстальский». Адрес последнего – улица Энгельса, 39. Построен он, вероятно,
в году тридцать седьмом. Приходилось слышать байку, что стены его сложены из
кирпича взорванной церкви Святой Равноапостольной Марии Магдалины, украшавшей
сквер. В это трудно поверить. На сохранившейся фотографии руин храма можно
разглядеть лишь груды кирпичного боя, да куски кладки, спаянные намертво
известковым раствором высочайшего качества, откуда кирпич, не разрушив его,
достать невозможно. Впрочем, чем черт не шутит.

В этом доме жили работники «Азовстали» со своими семьями. В том числе и семья
начальника доменного цеха Александра Ивановича Кочеткова. От его дочери –
Валентины Александровны – довелось узнать, что в том крыле здания, что выходит
на проспект Мира, на первом этаже до войны был магазин «Гастроном», на втором и
третьем — квартиры малосемейных сотрудников, на четвертом – комнаты молодых
специалистов. Поскольку именно эта часть здания наиболее приметна, посвятим ей
последующее повествование...

Во время оккупации Мариуполя, религиозным общинам было разрешено открыть церкви
в опустевших клубах и магазинах. Там, где был «Гастроном» устроили
Свято-Преображенский собор. Туда же были перенесены мощи митрополита Игнатия,
до того находившиеся в подвале краеведческого музея. Гитлеровцы, отступая из
города в сентябре сорок третьего года, предали пятиэтажку сожжению, огонь не
пощадил и мощи. Историю спасения их частиц можно прочитать в воспоминаниях отца
Василия Мултых.

После освобождения нашего города азовстальцы восстановили помещение, в котором
еще недавно был собор, и устроили там клуб металлургов завода. Там работали
кружки художественной самодеятельности, в частности драматический кружок,
проводились собрания, вечера отдыха. Мариупольский живописец Иван Егорович
Тютьма вспоминал, что в 1944 году в клубе была устроена выставка произведений
местных художников. Ее инициатором был Алексей Иванович Мищенко,
художник-карикатурист выездной редакции газеты «Правда» на восстановлении
завода «Азовсталь». Среди участников выставки были Василий Овечкин, Лель
Кузьминков, Леонид Гади, Иван Тютьма и сам Мищенко.

В октябре 1946 года для клуба металлургов было восстановлено при активном
участии комсомольцев завода «Азовсталь» здание бывшей гостиницы «Континенталь».
А в освободившемся помещении, о котором выше шел разговор, приступили к работам
по переоборудованию его для универмага. Универмаг был открыт 6 мая 1947 года.

У каждого горожанина свои воспоминания об этом предприятии советской торговли.
Кто-то запомнил, наверное, припорошенный пыльцой бытовой холодильник
«ЗИС-Москва». Этот товар продавался плохо – и дорого, и как бы в хозяйстве был
ни к чему, поскольку хранить особо в нем было нечего. Но если кто-то и
приобретал это чудо бытовой техники тех лет, то ставил его на видном месте
горницы, накрывал его салфеткой, украшенной вышивкой или мережкой, водружал
какую-нибудь безделушку.

Запомнились, вероятно, и китайские товары, заметим, не в пример теперешним, —
высочайшего качества. Изящные зонтики с бамбуковыми спицами, обтянутыми шелком,
украшенным нежнейшим акварельным рисунком. В тон им веера из того же шелка и
палочек из сандалового дерева. Интересно, что купленные шестьдесят лет назад,
они и сейчас источают тонкий пряный запах. А еще продавались невесомые
шерстяные кофточки, несносимые бостоновые брюки, и летние — хлопчатобумажные
бежевые и голубоватые. Все это «богатство» было оснащено этикетками с надписью
«Дружба», иероглифами и значком в виде двух переплетенных колец.

Мальчишек-подростков более всего привлекало левое крыло магазина, где стояли
велосипеды. Они подолгу глазели на солидные немецкие «Диаманты», изящные цвета
кофе с молоком «Латвело», тяжеловесные двухколесные машины «Украина» и
«Прогресс» производства Харьковского велозавода. Но больше всего их внимание
привлекал подростковый чехословацкий велосипед «Мифа», блистающий голубым лаком
рамы, хромированными крыльями, колесами и, естественно, рулем. Все это
великолепие дополнялось ярко-оранжевой кожаной сумкой для инструмента,
притороченной к раме. Даже после снижения цен 1 марта 1950 года эта «голубая»
мечта стоила 696 рублей – сумма абсолютно неподъемная для большинства семей
послевоенного времени. Откуда известна цена? А сохранилась газета за 1 марта
1950 года, где против строки «велосипеды и запчасти к ним» красным карандашом
детской рукой написано – «Мифа, 696 р.». Мечта осталась неосуществленной.

25 августа 1965 года на проспекте Ленина перед покупателями распахнул свои
двери только что построенный универмаг «Украина». Вскоре после этого в бывшем
уже теперь универмаге строители приступили к переоборудованию торгового зала.
Одновременно сносили перегородки на втором этаже, где находились жилые
помещения. В результате город получил двухэтажный магазин «Детский мир»…

В плане дом, о котором здесь идет речь, похож на несколько искаженную букву
«П». В послевоенные годы со стороны сквера были построены два четырехэтажных
жилых дома (их адреса проспект Мира, 63/1 и 63/2). Так образовалась довольно
обширная замкнутая площадка. Виталий Александрович Алимов, чье детство прошло в
доме 63/2, вспоминал, что в 50 – 60 годы благодаря энтузиастам их двора на
площадке был устроен фонтан, установлены бюсты Александра Пушкин и Максима
Горького, расставлены лавочки. В летнюю пору вечерами там устраивались концерты
самодеятельности детей.

В декабре 1991 года был основан «Первый Украинский Международный Банк (ПУМБ)».
В июле 1997 года зарегистрирован его филиал в Мариуполе. Филиал разместился в
основательно реконструированном крыле здания, где последовательно были магазин
«Гастроном», Свято-Преображенский собор, клуб металлургов завода «Азовсталь»,
универмаг, магазин «Детский мир».
