% vim: keymap=russian-jcukenwin
%%beginhead 
 
%%file topics.vojna.poezia
%%parent topics.vojna
 
%%url 
 
%%author_id 
%%date 
 
%%tags 
%%title 
 
%%endhead 

\section{Вірші, Пісні}
\label{sec:topics.vojna.poezia}

\begin{itemize} % {

\item \hyperlink{01_03_2022.fb.fb_group.story_kiev_ua.1.molitva}{%
Молитва, Анатолий Каганович, Київські Історії, facebook, 01.03.2022%
}

\item \hyperlink{03_03_2022.fb.kostrikov_andrej.odessa.1.stih_russkogo_soldata}{%
Стихотворение русского солдата, Андрей Костриков, facebook, 03.03.2022%
}

\item \hyperlink{16_03_2022.fb.kostrikov_andrej.odessa.1.kovcheg}{%
КОВЧЕГ (утопическое стихотворение пятилетней давности), Андрей Костриков, facebook, 16.03.2022%
}

\item \hyperlink{30_03_2022.fb.balatska_vira.gostomel.1.orki_chernobyl}{%
Послали орків на Чорнобиль. Наказ короткий: захопіть!, Мої вірші для всіх. Балацька-Гузієнко Віра, %
facebook, 30.03.2022%
}

\item \hyperlink{23_04_2022.fb.jenny_lev.1.matusju_prokynsja}{%
Матусю! Прокинься! - Він гладив її по обличчю, Jenny Lev, facebook, 23.04.2022%
}

\item \hyperlink{24_04_2022.fb.jenny_lev.1.razgovor_na_oblakah}{%
РАЗГОВОР НА ОБЛАКАХ, Jenny Lev, facebook, 24.04.2022%
}

\item \hyperlink{26_04_2022.fb.gouzar_olena.1.ne_znaly_scho_tak_ii_ljubymo}{%
Ми й не знали, що так її любимо, Olena Gouzar, facebook, 26.04.2022%
}


\end{itemize} % }
