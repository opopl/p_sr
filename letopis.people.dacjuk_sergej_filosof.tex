% vim: keymap=russian-jcukenwin
%%beginhead 
 
%%file people.dacjuk_sergej_filosof
%%parent people
 
%%url 
 
%%author 
%%author_id 
%%author_url 
 
%%tags 
%%title 
 
%%endhead 
\section{Сергей Дацюк}
\label{sec:people.dacjuk_sergej_filosof}

\url{http://www.uis.kiev.ua/~xyz/vir_bio.html}

СЕТЕВАЯ БИОГРАФИЯ
Как житель Сети я родился в конце марта 1997 года. К началу апреля я научился ходить... по сети. Говорить... в Сети начал 15 мая. В этот день вышел мой перевод "Декларации Независимости Киберпространства" Джона Барлоу. А так же - мое послесловие к ней "Парадоксальная интенция свободы в Internet". В тот же день вышли мои резюме "Теории виртуальной реальности" на русском и английском языках.
Первое вступление в полемику и диалог с Сетью начал на сетевом пространстве Журнала.Ру 28 июня статьями "От идеологии к теории Internet" и "Ноу-хау виртуальных технологий".
5 июля стартовал мой первый периодичный проект "Культурные провокации".
В октябре-ноябре 1997 года - цикл статей в "Непогоде".
В декабре 1997 года я исподволь начал приближаться к сетевому журналу XYZ
В феврале 1998 года вышла моя работа "Виртуальный анализ масс-медиа" в "Русском журнале".
С моим именем в Сети связаны разные истории. "Вечерний Интернет" 20 мая назвал меня "читателем из Киева", "борхесом" Сети, затем после цикла моих статей в "Непогоде" в своем опусе писатель Егор Простоспичкин объявил меня своей выдумкой, а вскоре и вообще был создан "Робот "Сергей Дацюк"", которому приписывались все мои сетевые работы и объявлялся конкурс на звание "Сергея Дацюка". Соответственно всем этим экспериментам я ответил в эссе "Герметический корпус". Дальнейшее развитие событий было описано в работе "Виртуалография робота Сергей Дацюк" и "ИНТЕРАКТИВНАЯ ДЕПЕРСОНАЛИЗАЦИЯ АВТОРА" (К проблеме гиперавторства - казус "Робот Сергей Дацюк"), которая была опубликована в Русском журнале, также как и ранее статья "Авторство и наука в Интернете".
Хронологическое описание событий и статей начального периода вы можете найти в архивах 1997 и 1998 года.
Всю мою историю публикаций и событий в Интернет вы можете посмотрете в рубрике "Философия Интернет".
Анализ нынешней ситуации в поисковых системах и каталогах при поиске строки "Сергей Дацюк" свидетельствует о том, что работы (тексты и другие проекты), принадлежащие "Роботу Сергей Дацюк", являются очень распространенными. Меж тем ко многим этим работам я не имею никакого реального отношения, кроме того, что, во-первых, в бренде "Робот Сергей Дацюк" (торговой марке, которая мне не принадлежит) использовано мое имя и фамилия (мое отчество "Аркадьевич" на этапе создания этого программного комплекса его создателям было не известно, поэтому встречается "Сергей Сергеевич" и тому подобное), а, во-вторых, некоторые из работ "Робота Сергей Дацюк" составлены из отрывков моих реальных работ. Во многих случаях некоторые работы, подписанные просто "Сергей Дацюк", составлены из моих действительных работ, так что читатели путают их с моими собственными работами. Поэтому в настоящее время к работам, являющимся моими (то есть собственно авторскими), можно отнести работы в Интернет, оригиналы которых размещены в разных тематических разделах серверов http://www.uis.kiev.ua/ и http://xyz.org.ua/.
На основе данного многолетнего эксперимента в Интернет, который затеян Андреем Черновым с "Роботом Сергей Дацюк" в 1997 году, с целью выяснить возможность деавторизации некоторой совокупности работ с реальным авторством, можно утверждать:
1) Что обилие компиляций текстов, созданных программным комплексом "Робот Сергей Дацюк", порождающих информационный шум вокруг совокупности текстов реального автора, деавторизует эту совокупность только для нецелевых аудиторий (людей, которые натыкаются на какую-либо работу случайно), при этом создавая интригу неожиданного узнавания в будущем действительного положения дел ("а ведь есть реальный автор")
2) Для целевых аудиторий компиляции текстов "Робота Сергей Дацюк создают дополнительный промоушн авторских текстов, вынуждая целевые группы обратиться к первоисточнику и прочитать аутентичные тексты.
Если вы все же тяеготеете к реальности, можете прочесть часть моей реальной биографии, касающейся Internet.

Виртуально Ваш on line Сергей Дацюк
\url{xxyz@ukr.net}
