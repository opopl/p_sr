% vim: keymap=russian-jcukenwin
%%beginhead 
 
%%file link.04_12_2021.fb.staliv_aleksandr.1.upravljat_i_spravljatsja
%%parent 04_12_2021.fb.zharkih_denis.1.politika_tendencii.cmt
 
%%url 
 
%%author_id 
%%date 
 
%%tags 
%%title 
 
%%endhead 


\href{https://m.facebook.com/story.php?story_fbid=4602500199844261&id=100002529780098}{%
Управлять и справляться, Александр Ста́лив, facebook, 04.12.2021%
}

\begin{multicols}{2}

Управлять и справляться. 

Управлять и справляться.

Вот, что особо важно. Управление.

Там, где управление, там и справляются.

Справляются потому, что делают всё по Прави Божией - по правилам (нормам)
законов естествознания - в соответствие (пропорционально) состоянию и действию
(механизму и механике) среды обитания, то есть Вселенной, Системы Вселенной
относительно Её средоточия, неотъемлемой и неотделимой составной частью и
принадлежностью Которой все мы являемся.

Это действительно (реально) по-Божески, по-людски и по-человечески, по-свойски,
по-доброму, участливо и счастливо, жизненно.

А борьба, борьба это удел дикой природы, животных, зверей, хищников, но не
людей, не человеков.

Борьба это не жизнь (по Прави Божией, то есть в соответствие (пропорционально)
состоянию и действию (механизму и механике) Вселенной), а выживание по понятию
от себя, каждый.

И мы, современники, мы поэтому и выживаем, не живём, потому что не имеем
достаточной естественнонаучной определённости жизни, чтобы жить. Не имеем.

Отсюда, по причине неопределённости, и борьба до иступления и самоизничтожения,
погибели, смерти, после которой, когда иметь подсистемное в Системе Вселенной
естествознание, жизни - не существует. Всё существует один единственный и
неповторимый (уникальный) раз. В том числе и жизнь каждого из нас, людей
земного шара.

Чтобы иметь правильное (нормальное) представление о действительности
(реальности), и соответственно (пропорционально) правильно (нормально) себя
располагать во времени и в пространстве, и тем самым-то и жить, не тужить, для
этого и необходим подсистемный в Системе Вселенной естественнонаучный подход.
Он правовой (нормативный). Он космополитический. Он принципиальный
(изначальный).

Все названные чуть выше особенности этого и такого подхода - исходя из пределов
величин (параметров) среды обитания - Вселенной, Системы Вселенной по их
происхождению одних из других (по производным) относительно Её средоточия,
неотъемлемой и неотделимой составной частью и принадлежностью Которой все мы
являемся - ещё раз подчёркиваю.

Об этом у меня за десятилетия написаны десятки тысяч различных произведений -
больших и малых - насущных.

Так вот, любая борьба осуществляется по причине неопределённости.

Борьба в человеческом обществе это непроизводительная трата жизненных сил,
времени жизни.

Вы только об этом задумайтесь, насколько неприятна борьба в повседневном быту.
Это просто катастрофа - переворот.

Согласны?

И вот, в частности, борьба с болезнями - вопиющая катастрофа всего
человечества, то, что сейчас происходит и существует, есть.

Это неправильно (ненормально), не по Прави Божией, то есть - не в соответствие
(не пропорционально) состоянию и действию (механизму и механике) среды обитания
- Вселенной.

Понимаете?

Это не управление. Это не справляются. Поэтому и борются.

Понимаете?

Вот в чём суть.

Болезни же естественно необходимо и следует просто лечить. Лечить посредством
человеческой благоразумной и благородной, доброй деятельности.

И всё. Точка. Вы согласны?

Думайте. Решайте. Выбор за каждым. Свой собственный выбор.

Жизненности вам, всем и каждому.

С особым почтением.
\end{multicols}
