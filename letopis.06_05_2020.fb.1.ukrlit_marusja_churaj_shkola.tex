% vim: keymap=russian-jcukenwin
%%beginhead 
 
%%file 06_05_2020.fb.1.ukrlit_marusja_churaj_shkola
%%parent 06_05_2020
 
%%url https://www.facebook.com/groups/1423469381075591/permalink/2873069822782199/
 
%%author 
%%author_id 
%%author_url 
 
%%tags 
%%title 
 
%%endhead 

\subsection{Не буде толку з цієї освіти. І, що найгірше, з України теж толку не буде. Через таку освіту}
\label{sec:06_05_2020.fb.1.ukrlit_marusja_churaj_shkola}
\Purl{https://www.facebook.com/groups/1423469381075591/permalink/2873069822782199/}

Подивилася онлайн урок для 11 класу з української літератури «Інтимна та громадянська лірика Ліни Костенко».
Картинка уроку – вау! Натюнінгована Новопечерська школа. Технічні навороти, таке інше …
Молода вчителька – супер! Усміхнена. Симпатична. Навчена. Слова від зубів відскакують. Пультик у руках.
Все таке модне, сучасне, европейське.
А зміст?
ЗМІСТ? ДЕ?
Весь урок розбирали, яка щаслива авторка, закохана. (чудово, 11 клас теж закоханий)
Шмат уроку розбирали, хто такі Фавн, Парнас, Пегас, Амур, тощо. (ок, культурна спадщина інших народів). Табличку з Пегасами, Амурами розглядали, обирали правильну відповідь: Фавн – це Бог лісу, моря, кохання? (Інтерактивчик, чудово).
Картинки подивилися світових художників з зображенням богів. (нормально, торкнулися мистецтва)
Голос Богдана ступки послухали, поезію «Крила» у його виконанні – клас!
Ну а тепер громадянська лірика. І отут починається наша українська освіта, яка дає нуль для України, нуль для громадянського становлення дітей, нуль для патріотичних почуттів до своєї країни та нашої історії, КОСМОПОЛІТИЗМ повний.
Громадянська лірика на кінець уроку. Громадянській ліриці відведено кілька штрихів. «Маруся Чурай»
- Твір був заборонений, - каже вчителька одну фразу. І все?
То скажи дітям, ЧОМУ заборонений? Хто забороняв Ліну Костенко? Навіщо забороняли? Що не так було з твором «Маруся Чурай», що його заборонили?
- Твір має історичне тло, - і все?
То скажи, яке те історичне тло? Про які події пише Ліна? Яке значення ті події мають СЬОГОДНІ? Проведи паралелі з сучасністю.
- Твір зобразив не лише любовні стосунки, а й визвольні змагання українців, – і знову одне речення.
То скажи, що це за визвольна боротьба? Від чого визволялися і хто визволявся?
- Домашнє завдання. Прочитати «Марусю Чурай». Скласти порівняльну характеристику двох родин: родини Марусі і родини Грицька.
А подумати над питанням, чому помилували Марусю? Чому не стратили, не повісили?
А подумати над питанням, чим Маруся була для українців у тих визвольних змаганнях?
А подумати над питанням, (міжпредметні зв’язки, історія) що то була за визвольна боротьба?
А запропонувати як домашнє завдання інтерактивчик дітям, такий як з грецькими/римськими богами, виберіть правильну відповідь: українці боролися за городи, за зменшення податків, за волю/незалежність. Українці боролися за незалежність від кого? Від поляків? Від росіян?
То чим усе-таки була Маруся в цій боротьбі, і чому її помилували? В чому полягає трагедія Марусі? Адже це трагедія: бути пташкою такого високого польоту - і закохатися з такими страшними наслідками в когута.
ПРО ЩО був цей урок? Красивий урок. З красивою вчителькою. З красивими словами. З красивою дошкою. З красивою школою. З красивими картинками.
ПРО ЩО?
Та діти повинні випурхувати з цього уроку повні енергії не лише красиво кохати/любити, а й з бажанням творити для УКРАЇНИ, для УКРАЇНСЬКОГО народу. Як робить це Ліна Костенко. Як робила це героїня твору, Маруся. Любити на повну і творити для своєї країни на повну!
Про ЩО НАША ОСВІТА?
Освітяни скажуть, та то ж лише перший урок. У програмі аж 5 годин відведено на вивчення творчості Ліни. Там і розглянуть.
Не розглянуть. Немає в програмі в КОМПЕТЕНТНОСТЯХ нічого про патріотичні почуття на основі твору «Маруся Чурай». Знаєте, що є?
Учень/учениця:
висловлює власну думку про особливості історичного мислення письменниці в «Марусі Чурай»;
визначає жанр твору, основні проблеми;
пояснює роль деталей, психологічну природу вчинків Марусі Чурай та інших персонажів;
висловлює власне ставлення до вчинку головної героїні.
осмислює естетичну вартість художнього твору;
формує уявлення про прекрасне як чинник гармонізації власного внутрішнього світу;
висловлює міркування, проводячи аналогії з сучасним життям.
Хочете знати, як зникає Україна? Отак зникає. Непомітно. Через виховання просто людей. Не українців. Навіть такі геніальні твори, як «Маруся Чурай» так злизати – це ще треба вміти. Чергова освітня диверсія. Це я все про освіту, не про вчительку. І не смійте її цькувати. Вчителька все зробила, як написано в програмі. Вона молоденька і талановита – її так навчили. І це страшно. А скільки в українській освіті ще диверсій проти України.
Міністерство, яке злизує патріотичні твори, треба розганяти, а програму переробляти. Суцільна гниль і боягузтво. При совку було більше патріотизму в програмі, ніж тепер.
