% vim: keymap=russian-jcukenwin
%%beginhead 
 
%%file 23_10_2021.fb.bilchenko_evgenia.1.marafon_konferencia_chelovechnost.cmt
%%parent 23_10_2021.fb.bilchenko_evgenia.1.marafon_konferencia_chelovechnost
 
%%url 
 
%%author_id 
%%date 
 
%%tags 
%%title 
 
%%endhead 
\subsubsection{Коментарі}

\begin{itemize} % {
\iusr{Виктор Савинов}
Лисичка!

\begin{itemize} % {
\iusr{Евгения Бильченко}
\textbf{Виктор Савинов} Рябина

\iusr{Евгения Бильченко}
\textbf{Виктор Савинов} у Малахова тоже фон рыженький)
\end{itemize} % }

\iusr{Алексей Бажан}

Все какие-то унылые и серьезные, а на Вашей бывшей кафедре зажигают не
по-детски: "Цікавим виявляється те, що погляди А. ЛаВея щодо природи Сатани,
можна порівняти з поглядами Г. Сковороди на природу Бога. Українські дослідники
Ю. Федів та Н. Мозгова говорять, що «не дивлячись на неортодоксальність
ставлення Г.С. Сковороди до Бога, до релігії, можна стверджувати, що світогляд
мисли- теля носить чітко виражений пантеїстичний характер. Бога, як
всемогутньої сили, що стоїть над природою і людьми, не існує. Він тотожний
природі і має різні імена «природа», «натура» тощо. Він – всюдисущий, невидима
натура» (Федів \& Мозгова, 2001, pp. 128–129). Тобто Сатана в розумінні А. ЛаВея
– це фактично Бог в поглядах Г. Сковороди." Пикантно, что аффтор благодарит за
ценные советы и поддержку тт. Хамитова и Крылову. 

\url{https://www.facebook.com/lbogoyavlensky1/posts/1255899134862650}

\end{itemize} % }
