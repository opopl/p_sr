% vim: keymap=russian-jcukenwin
%%beginhead 
 
%%file 22_11_2021.fb.uljanov_anatolij.1.maidan.3
%%parent 22_11_2021.fb.uljanov_anatolij.1.maidan
 
%%url 
 
%%author_id 
%%date 
 
%%tags 
%%title 
 
%%endhead 
\subsubsection{3}

Я знаю, что по этому поводу есть другие точки зрения, но оставлю их на совести
тех, кто ещё находит в себе силы для праздничных речей о ценностях свободы и
прав человека. Я понимаю, что пишущие гуманитарии мыслят «ценностями», а не
ценами. Как и то, что, живя в Украине, ты вынужден рассказывать себе некоторое
количество сказки, которая помогает сделать реальность чуть более выносимой. 

Этой механике миллионы лет. Когда человеку страшно, он выдумывает бога,
оперирует метафизическими категориями – например, говорит о \sem{гуманизме и
достоинстве}, словно не замечая ползущего по польскому полю соотечественника; о
«независимости» в кредит; ну и, конечно, Враге, на которого наше инфантильное
«гражданское общество» списывает все украинские беды.

Соседство с Агрессором даёт нам возможность откладывать человечность до
победного конца. Конец этот всегда не за горами, и наступит как только на месте
России возникнет Франция, благословляющая нас на членство в чужом военном блоке
у своих границ. Ну а пока – дай денег, Европа, иначе сам Сатана пройдёт по нам
к вам в круасан гуманизма.
