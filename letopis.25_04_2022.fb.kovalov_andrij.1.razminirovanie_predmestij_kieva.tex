% vim: keymap=russian-jcukenwin
%%beginhead 
 
%%file 25_04_2022.fb.kovalov_andrij.1.razminirovanie_predmestij_kieva
%%parent 25_04_2022
 
%%url https://www.facebook.com/andriy.kovalov/posts/5119957371423761
 
%%author_id kovalov_andrij
%%date 
 
%%tags 
%%title Розмінування передмістя Києва
 
%%endhead 
 
\subsection{Розмінування передмістя Києва}
\label{sec:25_04_2022.fb.kovalov_andrij.1.razminirovanie_predmestij_kieva}
 
\Purl{https://www.facebook.com/andriy.kovalov/posts/5119957371423761}
\ifcmt
 author_begin
   author_id kovalov_andrij
 author_end
\fi

Сапери \href{https://www.facebook.com/112btro}{112 Бригада територіальної
оборони} міста Києва
\href{https://www.facebook.com/TerritorialDefenseForces}{Територіальна оборона
ЗС України} працюють над розмінуванням передмістя Києва. Щодня. Навіть на
Великдень. Війна триває. У хлопців немає вихідних і часу на перепочинок, бо
роботи ще дуже багато.

\ii{25_04_2022.fb.kovalov_andrij.1.razminirovanie_predmestij_kieva.pic.1}

На фото: інженерна рота 112 бригади ТрО ЗСУ зачищає ліс, де стояла реактивна
артилерія російських терористів з так званої «37 отдєльной ґвардєйской
мотострелковой брігади» в/ч 69647 з міста Кяхта, що у Бурятії. Відстань від
військового містечка у Кяхті до цих позицій у передмісті Києва - 6400
кілометрів. Навіть якщо їхати весь час на автомобілі і не зупинятися - дорога
займе від 3 до 4 днів. Деякі російські терористи довше їхали в Україну, ніж
встигли тут повоювати  @igg{fbicon.face.tears.of.joy} 

\ii{25_04_2022.fb.kovalov_andrij.1.razminirovanie_predmestij_kieva.pic.2}

Бурятські терористи з Кяхти обстілювали «Градами», «Ураганами» і «Буратінами»
цивільну забудову мирних українських сіл і містечок на Київщині.
Використовували вони і заборонені міжнародними конвенціями касетні боєприпаси. 

Позиції цих бойових бурятів влучними пострілами знищили наша реактивна
артилерія Збройних сил України. 

Зараз сапери 112 бригади розпочали обстеження цвинтарів в околицях Києва, які
були під російською окупацією. Адже цього тижня за традицією українці підуть на
цвинатрі, щоб навідати могили родичів, але там можуть залишатися
вибухонебезпечні предмети. Це лише снаряди, чи ракети, які не розірвалися.
Також російські терористи свідомо мінували цвинтарі. Будьте обережні! Бережіть
себе та своїх рідних!

\ii{25_04_2022.fb.kovalov_andrij.1.razminirovanie_predmestij_kieva.pic.3}

Слава Збройним Силам України! 

Слава Україні! 

Ми переможемо! 

Фото: Андрій Ковальов, \href{https://www.facebook.com/nick.tymchenko1}{Nick Tymchenko}

\ii{25_04_2022.fb.kovalov_andrij.1.razminirovanie_predmestij_kieva.pic.4}
\ii{25_04_2022.fb.kovalov_andrij.1.razminirovanie_predmestij_kieva.pic.5}

\ii{25_04_2022.fb.kovalov_andrij.1.razminirovanie_predmestij_kieva.cmt}
\ii{25_04_2022.fb.kovalov_andrij.1.razminirovanie_predmestij_kieva.cmtx}
