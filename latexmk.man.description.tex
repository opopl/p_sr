% vim: keymap=russian-jcukenwin
%%beginhead 
 
%%file man.description
%%parent body
 
%%endhead 

\section{DESCRIPTION}

\vspace{0.5cm}
 {\ifDEBUG\small\LaTeX~section: \verb|man.description| project: \verb|latexmk| rootid: \verb|p_saintrussia| \fi}
\vspace{0.5cm}

Latexmk completely automates the process of compiling a LaTeX document.
Essentially, it is like a specialized  relative  of  the  general  make
utility,  but  one  which determines dependencies automatically and has some
other very useful features.  In its basic mode  of  operation  la- texmk  is
given the name of the primary source file for a document, and it issues the
appropriate sequence of commands to generate a .dvi, .ps, .pdf and/or hardcopy
version of the document.

By  default  latexmk will run the commands necessary to generate a .dvi file.

Latexmk can also be set to run continuously with a suitable  previewer.  In
that case the latex program (or one of its relatives), etc, are re- run
whenever one of the source files is modified, and the previewer au- tomatically
updates the on-screen view of the compiled document.

Latexmk  determines  which  are  the  source files by examining the log file.
(Optionally, it also examines the list of input and output files generated  by
the  -recorder  option  of modern versions of latex (and pdflatex, xelatex,
lualatex,  etc).   See  the  documentation  for  the -recorder  option  of
latexmk below.)  When latexmk is run, it examines properties of the source
files, and if any have been changed since  the last document generation,
latexmk will run the various LaTeX processing programs as necessary.  In
particular, it will repeat the run of  latex (or  a  related program)) often
enough to resolve all cross references; depending on the macro packages used.
With  some  macro  packages  and document classes, four, or even more, runs may
be needed. If necessary, latexmk will also run bibtex, biber, and/or
makeindex.   In  addition, latexmk can be configured to generate other
necessary files.  For exam- ple, from an updated figure file it can
automatically generate  a  file in  encapsulated  postscript  or another
suitable format for reading by LaTeX.

Latexmk has two different previewing options.  With the simple -pv  op- tion,
a  dvi,  postscript  or pdf previewer is automatically run after generating the
dvi, postscript or pdf version  of  the  document.   The type  of  file  to
view is selected according to configuration settings and command line options.

The second previewing option is the  powerful  -pvc  option  (mnemonic:
"preview continuously").  In this case, latexmk runs continuously, regularly
monitoring all the source files to see if any have changed.  Every  time a
change is detected, latexmk runs all the programs necessary to generate a new
version of the document.  A good previewer will  then automatically update its
display.  Thus the user can simply edit a file and, when the changes are
written to disk, latexmk completely automates the  cycle  of  updating  the
.dvi (and/or the .ps and .pdf) file, and refreshing the previewer's display.
It's not quite WYSIWYG,  but  usefully close.

For  other previewers, the user may have to manually make the previewer
update its display, which can be (e.g., with some versions of xdvi  and
gsview) as simple as forcing a redraw of its display.

Latexmk  has  the  ability  to print a banner in gray diagonally across
each page when making the postscript file.  It  can  also,  if  needed, call
an  external  program to do other postprocessing on generated dvi and
postscript files.  (See the options -dF and -pF, and the documentation  for
the  \verb|$dvi_filter|  and  \verb|$ps_filter|  configuration variables.)
These capabilities are leftover from older versions of latexmk, but are
currently  non-functional.  More flexibility can be obtained in current
versions, since the command strings for running  latex,  pdflatex,  etc can
now be configured to run multiple commands.  This also extends the possibility
of postprocessing generated files.

Latexmk is highly configurable, both from the command line and in
configuration  files,  so  that  it can accommodate a wide variety of user needs
and system configurations.  Default values are set  according  to the operating
system, so latexmk often works without special configuration on MS-Windows,
cygwin, Linux, OS-X, and other UNIX  systems.   See the  section
"Configuration/Initialization  (rc)  Files", and then the later sections "How
to Set Variables in Initialization Files",  "Format of  Command
Specifications", "List of Configuration Variables Usable in Initialization
Files", "Custom Dependencies", and "Advanced  Configuration"

A  very annoying complication handled very reliably by latexmk, is that LaTeX
is a multiple pass system.  On each run, LaTeX reads in  information
generated on a previous run, for things like cross referencing and indexing.
In the simplest cases, a second run of LaTeX  suffices,  and often  the log
file contains a message about the need for another pass.  However, there is a
wide variety of add-on  macro  packages  to  LaTeX, with  a variety of
behaviors.  The result is to break simple-minded determinations of how many
runs are needed and of  which  programs.   Latexmk has a highly general and
efficient solution to these issues.  The solution involves retaining between
runs  information  on  the  source files,  and a symptom is that latexmk
generates an extra file (with extension \verb|.fdb_latexmk|, by default) that
contains the source file  information.


  
