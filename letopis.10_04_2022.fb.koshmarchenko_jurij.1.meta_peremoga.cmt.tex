% vim: keymap=russian-jcukenwin
%%beginhead 
 
%%file 10_04_2022.fb.koshmarchenko_jurij.1.meta_peremoga.cmt
%%parent 10_04_2022.fb.koshmarchenko_jurij.1.meta_peremoga
 
%%url 
 
%%author_id 
%%date 
 
%%tags 
%%title 
 
%%endhead 
\zzSecCmt

\begin{itemize} % {
\iusr{Марина Абрамова}
Сил.

\iusr{Yuliya Byelinska}
Ми всі солдати

\iusr{Юрій Кошмарченко}
\textbf{Yuliya Byelinska} 

до речі, трошки вагаюся щодо цього

все ж таки солдати і фронт — це там, де стріляють і є прямий ризик загинути

а, скажімо, інформаційний фронт — це ти сидиш на дивані в трусах і їж
шоколадку, трошки інше концептуально

\iusr{Юрій Кошмарченко}
а тут, до речі текст про те, чому не можна себе зараз знецінювати:

\href{https://platfor.ma/topic/ya-roblyu-zamalo-yak-buty-iz-vidchuttyam-nepotribnosti-u-chas-vijny/}{%
«Я роблю замало»: як бути із відчуттям непотрібності у час війни, Наталя Шевчук, platfor.ma, 06.04.2022%
}

\iusr{Karyna Lytvynenko}

Блін, було б реально смішно виділити один день на грандіозну сварку, назвати це
«Один день національної сварки» і прям лаятися, кидатися помідорами чи тим, що
залишиться, кричати, як ненавидиш всі подушки, бо єдина зручна залишилась у
тебе вдома, сперечатися, де було краще ховатися — в підвалі чи за двома стінами
й інше, інше.

\begin{itemize} % {
\iusr{Olga Pobedash}
\textbf{Karyna Lytvynenko} 

це точно, такий день національної сварки нам потрібен @igg{fbicon.face.smiling.eyes.smiling}  Якщо ще поєднати з
іншою, притаманною нашій українській ДНК рисою, почуттям гумору, вийде веселе
свято і велика туристична аттракція. Треба запам'ятати цю ідею.

\iusr{Юрій Кошмарченко}

наскільки я пам'ятаю, в різних країнах є частково схожі штуки

головне не дійти до такої версії: 

\href{https://uk.wikipedia.org/wiki/%D0%A7%D0%B8%D1%81%D1%82%D0%BA%D0%B0_(%D1%84%D1%96%D0%BB%D1%8C%D0%BC)}{%
Чистка (фільм), wikipedia.org%
}

\iusr{Olga Pobedash}
\textbf{Юрій Кошмарченко} 

ой ні, я думаю, Карина щось зовсім інше мала на увазі. Зі сварками нашими
дійсно треба щось робити, якщо серйозно. Конструктивно і весело. Конструктив -
прививати мистецтво ведення дискусії, критичне мислення, використовувати факти,
враховувати іншу думку, чи принаймні намагатись її зрозуміти.

\iusr{Юрій Кошмарченко}
\textbf{Olga Pobedash} ох, так
дуже важлива робота на роки

\iusr{Olga Pobedash}
\textbf{Юрій Кошмарченко} 

і це з дитсадочків і шкіл треба починати. Це дійсно має стати великою місією.
Не тільки виховувати людей з почуттям власної гідності (це by default), а й з
повагою до гідності іншої людини. Мені здається, це важливо для виховання
людяності.

Ну і почуття гумору - це must. Чим старше я становлюсь, тим більше розумію, як
важливо мати почуття гумору  @igg{fbicon.smile} 

\iusr{Lin Helly}
\textbf{Карина Литвиненко} сваритися та викривати все негативне в наших політиках, це мабуть національна ідея))

\end{itemize} % }

\iusr{Anton Kartavtsev}
Роби поруч з президентом  @igg{fbicon.heart.green} 

\iusr{Юрій Кошмарченко}
\textbf{Anton Kartavtsev} та це на Банкову йти, мені лінь

\iusr{Таля Винарська}
Про скидання нюдсів забули. Це теж важливо  @igg{fbicon.face.grinning.squinting} 

\iusr{Юрій Кошмарченко}
\textbf{Таля Винарська} це фундамент перемоги!

\iusr{Инна Кожухарь}
Дрібна переможенька - краплина в річці ПЕРЕМОГИ і самоповаги

\iusr{Svitlana Granovska}

А можна посилання на цей текст, дякую? Таке само відчуття... до перемоги
@igg{fbicon.fist.raised} @igg{fbicon.flag.ukraina}

\begin{itemize} % {
\iusr{Юрій Кошмарченко}
\textbf{Svitlana Granovska} еее  @igg{fbicon.smile} 
ось посилання:
\url{https://www.facebook.com/jumma/posts/10228098714159237}

\iusr{Svitlana Granovska}
\textbf{Юрій Кошмарченко} дякую @igg{fbicon.hands.pray}
\end{itemize} % }

\iusr{Юлия Дудина}

Дякую. Доречно дійсно в наших умовах. Щоб не робив, постійно почуваєшся якимось
недопатріотом, недоукраїнцем чи ще щось, бо ти ж не на фронті  @igg{fbicon.face.unamused} 

\end{itemize} % }
