% vim: keymap=russian-jcukenwin
%%beginhead 
 
%%file 23_06_2021.fb.nicoj_larisa.1.voloshki_kiev_jazyk.cmt
%%parent 23_06_2021.fb.nicoj_larisa.1.voloshki_kiev_jazyk
 
%%url 
 
%%author 
%%author_id 
%%author_url 
 
%%tags 
%%title 
 
%%endhead 
\subsubsection{Коментарі}

\begin{itemize}

\iusr{Андрій Смолій}

Пані Ларисо, все ж в Києві багато хто тепер говорить українською. Років 20
назад ситуація була дуже погана. Зараз на мій слух десь 30-35\% вже є
україномовних.

От сьогодні був в кількох місцях, зі мною всі крім 1 людини говорили
українською.  Так що поступово (звісно не так як нам би хотілось), проте йдемо
своїм шляхом.

\iusr{Лариса Ніцой}

Андрій Смолій то ж не мої слова, то думка дітей ))

\iusr{Ser Sanych}

Андрій Смолій а раніше що 10\%?

\iusr{Андрій Смолій}
\textbf{Ser Sanych} в часи коли я вчився в школі було набагато менше

\iusr{Тетяна Засаднюк}
\textbf{Ser Sanych} В Миколаєві теж плачевна історія з мовою,торгівля майже повсюди ігнорує..

\iusr{Фрося Синичкина}
\textbf{Андрій Смолій} А у нас у Дніпрі тяжко. Майже вся молодь \enquote{баєть
па масквацкі}, а особливо упороті так це \enquote{яжематирі}.

\iusr{Татьяна Селиверстова Лемешева}

\textbf{Андрій Смолій} пане Андрію, хочеться скоріше, бо це неволя, сама пособі
відчуваю, це так,як серце в клітці зачинене, б'ється крилами, щоб вирватись.
Не має з ким і розмовлять буває, а як говориш, то така реакція буває
неадекватна.  Взнаєш людину, наче і не знав....

\iusr{Віталій Варчак}

\textbf{Андрій Смолій} 35 процент на 2,5 мільйонний Київ доволі велика
кількість. Нажаль у Львові росте більша кількість російськомовних . Пригадую ще
десь в 2010 році на парах викладач з історії говорив що у Львові проживає 150
тис росіян, і це рахуйте ще було до війни з Росією, зараз не побоюсь цього
слова майже половина центру Львова - російськомовна( Львів дійсно за останні
роки після війни став більш русифікованим, десь мабуть тисяч 200 на 720
тисячний Львів точно російськомовні 😟

\iusr{Дарія Матвіїв-Веклин}
\textbf{Віталій Варчак} не видумуйте! Панаєхало трохи, а ще колишні окупанти васпрялі духом.

\iusr{Ірина Бойчук}

Дуже багато російськомовних в Луцьку на Волині, це просто не можна слухати,
працюю на атракціонах в касі, всім російськомовним відповідаю тільки рідною,
надіюсь розуміють...

\iusr{Игорь Олевский}
\textbf{Ірина Бойчук}
А як же закон про мову, обслуговувати на мовi клiента?

\iusr{Ірина Бойчук}
\textbf{Игорь Олевский} Перегляньте ще раз цей закон про мову від 16 сіня 2021р.

\iusr{Игорь Олевский}
\textbf{Ірина Бойчук}
Переглянув... так, не справедливо и суперечить Конституцii.

\iusr{Ira Malynovska}
Волинська обл теж має село Волошки)

\iusr{Ольга Небесник}
Як це круто, коли діти хочуть іти на зустріч з письменницею і купувати- читати українські книжки про Україну !!!

\iusr{Наталія Лозинська}
Пані Ларисо! Ви - неймовірна! Добре, що ви у нас Є!

\iusr{Iryne Mango}
Село Волошки е і у Волинській обл! Дуже гарне та затишне!

\iusr{Оксана Підопригора}
Не всі в Києві говорять російською, так само як не всі в Рівному говорять українською

\iusr{Лариса Ніцой}
\textbf{Оксана Підопригора} Згодна. Я передала слова дітей

\iusr{Леся Черпак}
Ваша робота, пані Ларисо, заслуговує поваги. Здоров'я Вам, наснаги і успіхів. Наша мова того вартує!

\iusr{Igor Ostapuk}
П. Ларисо, є село з такою назвою і у нас на Волині. Недалечко від Колодяжного.

\iusr{Константин Попов}
мi так нє будєм

\iusr{Sergii Bilokur}
Пані Лариса помиляється, не усі хто приїздить до Києва говоре російською і хто там живе теж)))

\iusr{Лариса Ніцой}
Sergii Bilokur ви не уважний. Слова належать не мені.

\iusr{Sergii Bilokur}
\textbf{Лариса Ніцой} гадав це Ваша історія

\iusr{Людмила Полонська}
\textbf{Sergii Bilokur} не говоре,а говорить

\iusr{Sergii Bilokur}
\textbf{Людмила Полонська} ну добре, тоді говорять

\iusr{Вертелецкий Валерий}

Російської ці діти не знають, на відміну від пані Лариси, це добре. Не дуже
гарно, що вони не знають іншої мови. Треба бути причетним хоч до
англо-саксонської цивілізації. Хуторянство ні до чого доброго не доведе.

\end{itemize}
