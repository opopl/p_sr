% vim: keymap=russian-jcukenwin
%%beginhead 
 
%%file 17_12_2020.sites.ru.zen_yandex.asov_alexandr.1.skazanija_pjatigorskogo_knjazhestva
%%parent 17_12_2020
 
%%url https://zen.yandex.ru/media/id/5eeaf725a3dca453cfdd4b58/skazaniia-piatigorskogo-kniajestva-5fce5054c9e0347434d5c17d
 
%%author 
%%author_id asov_alexandr
%%author_url 
 
%%tags istoria,pjatigorsk
%%title Сказания Пятигорского княжества
 
%%endhead 
 
\subsection{Сказания Пятигорского княжества}
\label{sec:17_12_2020.sites.ru.zen_yandex.asov_alexandr.1.skazanija_pjatigorskogo_knjazhestva}
\Purl{https://zen.yandex.ru/media/id/5eeaf725a3dca453cfdd4b58/skazaniia-piatigorskogo-kniajestva-5fce5054c9e0347434d5c17d}
\ifcmt
	author_begin
   author_id asov_alexandr
	author_end
\fi

\index[cities.rus]{Пятигорск, Россия!Сказания Пятигорского княжества, 17.12.2020}

\begin{leftbar}
  \begingroup
    \em\Large\bfseries\color{blue}
Со времён легендарной Русколани дошло немало удивительных и прекрасных
сказаний, повествующих о горах Приэльбрусья и Пятигорья: об Эльбрусе, о Большом
и Малом Бермамытах, о горах Бештау и Машук и иных близлежащих. И во всех сих
сказаниях повествуется о том, как в стародавние времена жили князья (или боги),
обратившиеся в эти горы.
  \endgroup
\end{leftbar}


\ifcmt
  pic https://avatars.mds.yandex.net/get-zen_doc/3047751/pub_5fce5054c9e0347434d5c17d_5fcf374cadaafd0b10ebd06a/scale_1200
	caption Пятигорск (начал XX в.) и статуя Буса. Коллаж А. Асова
\fi

Тогда хозяином этих мест был грозный бог-великан Эльбрус. У него была жена, или
— в других сказаниях — дочь, по имени Машук. Молодой витязь Бештау влюбился в
неё, а Эльбрус противился сей любви. И потому Машук сбежала с Бештау. Эльбрус
бросился в погоню, пустив по следу псов Большого и Малого Бермамытов. У
Подкумка погоня настигла беглецов, была битва, в коей все погибли и обратились
в горы.

Исток сей легенды — суть звёздный миф времён ведических. И изначально в нём
речь шла о боге Дые, который властвовал над миром, а его чертог полагался на
Эльбрусе. Герой Бештау (имя значит «Пять вершин») — суть пятиликий Велес. Машук
— это дочь Дыя лунная богиня Дива.

В «Русских Ведах» сей миф излагается так. Бог Велес враждовал с Дыем и низверг
его в Пекло. Но потом Дый поднялся на землю и устроил великий пир в знак
примирения с Велесом. Для сего пира закололи огромного быка «в тени коего могли
отдыхать сто обычных коров с быками», навезли пива великие бочки. И Велес не
смог удержаться, прибыл на этот пир.

На том пиру он и встретил дочь Дыя «дивную» красавицу Диву и страстно влюбился
в неё. Велес закружил её в танце. Тогда Дый грозно воскликнул, так что по небу
прокатился гром: «Между нами Медведь явился! Он погибнет от тяжких ран! Пусть
помчатся за ним Собаки по земле и по небосводу!»

Слова эти значили, что в силу вступает звёздный закон. И как за Большой
Медведицей по небу мчится созвездие Гончих Псов, так и за Велесом-Медведем
ринутся псы Дыевы. И тут Велес был отравлен «ослиным мозгом», влитым в его
чашу, и потом заточён на Чёрном Алатыре.

Важно заметить, что горы в окрестностях Эльбруса носят имена, которые мы можем
найти не только на земле, но и на небе. Сам Эльбрус являет собой как бы
небесную ось, исходящую из Полярной звезды. И если смотреть от Эльбруса, от
легендарного «Храма Солнца», то в ночь на летнее солнцестояние можно видеть над
Бермамытами созвездие Гончих Псов, над Бештау — Волопаса (Велеса), над Машуком
— Большую Медведицу. Рядом — созвездие Малого Льва, и оно висит над горою
Спящий Лев (Развалка), над горою Змейка находится созвездие Змея Тритона.

\ifcmt
  pic https://avatars.mds.yandex.net/get-zen_doc/1878023/pub_5fce5054c9e0347434d5c17d_5fcf3464adaafd0b10e6f5fa/scale_1200
	caption Вид на Большой и Малый Бермамыты от г. Тузулук. Фото А. Асова.
\fi

Подобная звёздная легенда сохранилась и у балкарцев, живущих в окрестностях
Эльбруса. Она дана в книге «Нарты. Героический эпос карачаевцев и балкарцев»
(М., 1994). Балкарского Велеса зовут Ёрюзмеком, он носит волшебную волчью
шкуру, иногда и сам обращается в Звёздного Волка (звезду Сириус из созвездия
Большого Пса). Ёрюзмека также приглашают на пир (той) шайтаны. Они хотят
отравить его ослиным мозгом. И балкарский Дый по имени Фуко (Пако) также
гневается на гостя, который хочет похитить его дочь и говорит: «Пусть собаки
всегда его будут преследовать!»

В балкарской легенде прямой связи образов нартов и богов с горами Пятигорья
нет, это древнейший звёздный миф. А у их соседей, кабардинцев, легенды
повествуют именно о великанах, ставших горами. К тому же кабардинские легенды
переносят нас во времена Русколани. И за образами «владетельного князя»
Эльбруса, его дочери (или жены) Машук, витязя Бештау видны лики исторических
князей.

В этих легендах князь Эльбрус живёт в замке из голубого хрусталя и белого
мрамора, расположенном на утёсе у озера Тамбукан, что близ Пятигорска. Машук
оказывается сестрой князя Казбека, убитого Эльбрусом. И он её берет насильно в
жёны.

У князя Эльбруса есть сын — Арслан, который также влюбился в красавицу Машук,
увидев её на празднике урожая. Арслан втайне от отца стал дарить Машук
самоцветы, осыпал её цветами, а также подарил её фату, сотканную из туманов.

Но вот враги напали враги на этот край. Арслан, уезжая на войну, надел на палец
любимой перстень — талисман с бирюзой. Как-то Эльбрус увидел на пальце Машук
тот перстень и взревновал. В гневе он пригрозил жене смертью, отнял перстень и
зашвырнул его. Покатился перстень со звоном по ущельям, упал и превратился в
Кольцо-гору, что и поныне есть в Кисловодске.

Воротился с войны Арслан, и ночью втайне встретился с любимой. И тогда Машук
ему сказала: «Давай от грозного отца, от старого мужа моего, убежим туда, в
степи, где пшеницу сеют, или ещё дальше, к кочевникам. Возьмём у отца верблюда
для кочевья, а на случай, если придётся на земле пахать, захватим с собою и
быка». Арслан согласился.

Проснулся Эльбрус, огляделся, а возле него нет молодой жены. Схватил заветный
алмазный меч, который назывался «Содрогайся», кинулся в погоню с верными
гончими псами, Большим и Малым Бермамытом. И настиг беглецов в долине Подкумка.

И скрестили мечи отец и сын, а Машук отбежала в сторону и укрылась фатою,
сотканною из туманов. Арслан ударил старика по седой голове, раскроил её
надвое, но великан, собрав последние силы, взмахнул алмазным мечом и рассёк
Арслана на пять частей. Рухнул сын-великан, обратился горой Бештау, что значит
«пять гор». А шлем, упавший с его головы, стал горою Железной.

И тогда Эльбрус обернулся к Машук и ударил её кинжалом. Отброшенный в сторону
клинок стал Кинжал-горою, а на месте удара образовался Провал. И так Машук
стала каменеть и обернулась горою. И само имя Арслан (Лев) по велению Эльбруса
обернулось горой Спящий Лев (ныне Развалка). Умирает в степи Машук, цепенеет её
тело, обращается в камень, а из глаз горячие слёзы льются. Желает Машук
подползти к Бештау, надеясь, что её слёзы, её ласки исцелят его смертельные
раны, но Лев не пускает. «Ужаль его!» — стала просить Машук проползавшую мимо
змею. Но Лев только лапой махнул и отшвырнул гадину. Стала она горой Змейкой.
Окаменели также сопровождавшие беглецов Верблюд и Бык, и они также стали
горами.

Оба верных пса Большой и Малый Бермамыты каменистыми горами встали у ног
Эльбруса. Только любовь Машук не умерла. Льются её горячие, горькие слёзы, не
иссякает их целебная сила. Может быть, они, как живые воды, исцелят глубокие
раны Бештау? А пока они врачуют каждого, кто приблизится к источникам этих
слёз… (Пересказ по записям Е. Польской, К.И. Шаленко и др., см. «Кабардинские
легенды и сказки», Ставрополь, 1915).


\ifcmt
  pic https://avatars.mds.yandex.net/get-zen_doc/2431229/pub_5fce5054c9e0347434d5c17d_5fcf34d6adaafd0b10e7b7c0/scale_1200
	caption Гора Бештау. Фото А. Асова
\fi

Приведу также запись легенды о горах Пятигорья из сборника А. Сушилина
«Пятигорье» (Пятигорск, 1927) «Рассказ старого горца Измаила Адигей Бека о том,
откуда появились горы Эльбрус, Машук и Бештау».

\begingroup
	\em\large\color{blue}
«Давно, эх как давно, кунак, это случилось… Гор этих не было, кругом были
степи безбрежно широкие с изумрудной, сочной травой, и в степях этих
паслись бесчисленные табуны князя Эльбруса. Хороши были виноградники у
князя, а вино было такое, что довольно было одного стакана, чтобы сердце
человека загорелось огнём… Знаешь, чем поливал он виноградные лозы? Сначала
кровью льва или тигра, водились они когда-то в здешних краях, потом кровью
осла, а когда уже поспевать начал виноград — резал свиней и их кровью
поливал свои виноградники. Выпьешь первый стакан — загорится сердце как у
льва, выпьешь второй — голова отяжелеет и можешь ты наделать глупостей, а
выпьешь третий — будешь валяться как свинья, где попало!

Кони его были чистокровные, огромные стада баранов паслись в степях, а дома
было столько добра, не счесть… И ковры Ирана, и самоцветные каменья из
Индии, и пряности из Багдада. А оружие какое было?! Кинжалы, шашки
дамасской стали, калёные стрелы из крепкого, тонкого бамбука привезли ему
из Монголии. Эх, была у него одна шашка. Огонь, не шашка! Ещё прадеду
Эльбруса, могущественному Талаат-Беку служила она, и так и передавалась из
рода в род. Шашка эта рубила железо как дерево, и один взмах её перерубал
тушу быка. Крепко любил шашку князь и никогда не расставался с нею и даже,
ложась спать с молодой женой, тоненькой красавицей Машукой, клал её с
правой стороны. Так и спал — между подругой сердца и подругой битв.

Хороши утренние и вечерние зори, чудесные звёзды небесные, но звёзды чёрные
глаз Машуки были ярче ночных светил… Крепко, со всем пылом поздней страсти,
любил князь жену и баловал её безмерно. Но не один Эльбрус любил красавицу
Машук. Часто приезжал к нему в гости его друг, Бештау-бек. Ах, и красив же
был молодой Бештау! Стройный, высокий, с тонкой талией и широкими плечами.
Орлиный с горбинкой нос, тёмные бархатные глаза. Часто останавливались на
нём глаза Машук, и ярко вспыхивали тогда зарницы на её щеках.

И вот однажды, когда поблекли огни заката, во двор поместья Эльбруса въехал
молодой Бештау, возвращавшийся с охоты. Проезжая мимо женской сакли,
захотел он взглянуть на дорогое лицо. Ласково встретил Эльбрус охотника, и
вот, после ужина, когда ушёл Бештау спать на увитый виноградом балкон, а
муж уснул, разметалась в постели прекрасная Машук — душная была ночь. Не
может уснуть, тянет её безотчётно на воздух, в тенистый, густой сад с
вековыми деревьями… Вот уже идёт она, облитая лунным светом, к обрыву в
конце сада, где далеко-далеко, на дне, журчал всегда так задумчиво и нежно
холодный ключ… Бештау видел, как из дома вышла Машук, видел, как скрылась
она среди густой листвы сада. Как молодой барс, легко и быстро спрыгнул с
балкона молодой джигит и бросился в глубину сада по освещённой луной
дорожке.

Душная, страшная ночь! Не спится князю… А вот проснулся он, но нет рядом
Машук и не тянется к князю лилейная рука… Вскочил с ложа князь, сразу
вспомнил перехваченные за ужином взгляды жены и гостя — и, схватив шашку,
выскочил на балкон… Никого! Вихрем мчится к обрыву князь, сжимая шашку, и
вдруг видит две тени, так близко стоящие друг ко другу! Кровь бешеным
потоком залила обезумевший мозг, и удар шашки свалили на землю Машук…

— Опомнись, Эльбрус! — воскликнул Бештау. — Посмотри, ты разрубил её сердце
и — ты взгляни только — не кровь льётся, а голубая чистая вода! Это знак её
невинности! О Машук, Машук! Сердце моё разрывается на части от горя!

Как изваяние стоял побледневший Эльбрус, опустив шашку. И как гнев природы
вдруг загудела, закачалась земля. Страшное тогда было землетрясение. Земля
вздулась от могучих подземных ударов, и там, где стоял Эльбрус, — появилась
огромная гора со снежными вершинами, а вдали — Машук с провалом, на дне
которого и сейчас ты увидишь голубую воду — доказательство её невинности, а
где упал с разорвавшимся от горя на пять частей сердцем Бештау — стоят пять
печальных вершин.

\endgroup

\ii{17_12_2020.sites.ru.zen_yandex.asov_alexandr.1.skazanija_pjatigorskogo_knjazhestva.comments}
