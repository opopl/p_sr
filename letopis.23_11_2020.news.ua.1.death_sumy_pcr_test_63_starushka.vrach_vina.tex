% vim: keymap=russian-jcukenwin
%%beginhead 
 
%%file 23_11_2020.news.ua.1.death_sumy_pcr_test_63_starushka.vrach_vina
%%parent 23_11_2020.news.ua.1.death_sumy_pcr_test_63_starushka
 
%%url 
%%author 
%%author_id 
%%tags 
%%title 
 
%%endhead 

\subsubsection{Знакомые считают, что виновата врач}
\label{sec:23_11_2020.news.ua.1.death_sumy_pcr_test_63_starushka.vrach_vina}

\index[deaths.rus]{Юрченко, Любовь Павловна!18 ноября 2020, коронавирус, Путивль}

В среду, 18 ноября, на пороге Путивльской центральной районной больницы
скончалась 63-летняя \textbf{Любовь Юрченко}. Она пришла туда, чтобы ее
госпитализировали, однако лечь в больницу просто не успела.

\enquote{Она только закричала, упала и умерла. Не стало моего близкого человека, я еще
не могу осмыслить это}, \dshM плачет кума женщины Ольга Бордакова.

Она рассказывает, что Любовь Павловна примерно две недели болела. \enquote{Я ей
позвонила, она мне говорит, что чувствует себя плохо, у нее температура 39.
Однако таких симптомов, как отсутствие обоняния, у нее не было, не было и
ломоты в теле. Но она все время жаловалась на слабость. Я ее спрашивала, почему
же тебя не направят на ПЦР-тест? Оказывается, не было показаний для него. Она
наблюдалась у частного семейного врача Аллы Чижик, с ней была заключена
декларация. Потом ей вроде тест сделали, но в понедельник, когда она пришла,
результаты потеряли}, \dshM говорит Бордакова.

\ifcmt
pic https://i.obozrevatel.com/gallery/2020/11/19/gettyimages-1209359910.jpg
caption Женщине долго не делали ПЦР-тест
\fi

По словам родственницы, в среду, 18 ноября Любови Юрченко должны были также
провести КТ. \enquote{Она сама уже позвонила и записалась на обследование, но там
сказали приходить только через семь дней. Как раз в среду \dshM когда она и
умерла}, \dshM продолжает Бордакова.

Еще одна знакомая Юрченко, Светлана Шаповалова, возмутилась в соцсетях. \enquote{Это
вопиющий беспрецедентный случай, в голове не укладывается. Знаю лично умершую
женщину и хождение по мукам к горе-врачу. Больная умоляла о направлении на
ПЦР-тест, и он действительно в понедельник был взят, но таинственно исчез.

Сегодня, видя, что пациентка еле жива, сатурация упала, ей все же взяли тест на
месте, прокапали, вручили направление на госпитализацию и... выпроводили. Никто
не искал ей место в больнице ни ранее, ни в тот день. Почему ни в понедельник,
ни сегодня ей не вызвала скорую семейный врач на госпитализацию, а вытолкала
домой? Почему за две недели врач, закрывая один больничный и открывая второй,
не задала себе вопрос, а может, бронхит уже должен был бы пройти?} \dshM пишет
Шаповалова.

\ifcmt
pic https://i.obozrevatel.com/gallery/2020/11/19/screenshot115.png
caption Знакомая умершей считает, что врач затянула время. Источник: Facebook Светланы Шаповаловой
\fi

