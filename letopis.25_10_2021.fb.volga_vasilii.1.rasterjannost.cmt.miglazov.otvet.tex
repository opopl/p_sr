% vim: keymap=russian-jcukenwin
%%beginhead 
 
%%file 25_10_2021.fb.volga_vasilii.1.rasterjannost.cmt.miglazov.otvet
%%parent 25_10_2021.fb.volga_vasilii.1.rasterjannost.cmt
 
%%url 
 
%%author_id 
%%date 
 
%%tags 
%%title 
 
%%endhead 
\paragraph{Иван Миглазов - У меня есть ответ! Надо бороться и Победить!}

\begin{itemize} % {

% -------------------------------------
\ii{fbauth.miglazov_ivan.odessa.ukraina.doneck}
% -------------------------------------

У меня есть ответ! Надо бороться и Победить! И плевать я хотел на то, знает
Путин что делать или не знает! Он пусть занимается своей страной! А я буду
делать всё возможное, чтобы наша Украина наконец избавилась от этой нечисти
нацистской и заокеанской и по настоящему Возродилась!!!

\begin{itemize} % {
\iusr{Василий Александрович Волга}
а как бороться будем, Иван?

\iusr{Галина Дегтярева}
\textbf{Иван Миглазов} Нашему теляти да б вовка зьисты.

\iusr{Иван Миглазов}
\textbf{Василий Александрович Волга} смело! С 2018 года, как могу, так и борюсь! А вообще с «Партии Регионов» всё началось и с ней должно всё и закончится! Вернее, «назад в будущее»! Учитывая все грехи и ошибки!

\iusr{Иван Миглазов}
\textbf{Галина Дегтярева} а и сожрём нахрен! И волка и демона!!!
\end{itemize} % }

\iusr{Евгений Рулев}

А почему Путин должен знать, что делать с такой Украиной. Он что назначенный за
тамошний беспредел? Делайте что хотите, Путин то почему должен знать всё?

\begin{itemize} % {
\iusr{Василий Александрович Волга}
не должен. я просто цитирую его.

\iusr{Галина Дегтярева}
\textbf{Евгений Рулев} Да хотя бы только потому, чтобы натовские ракеты не
стояли в пяти минутах подлёта к Москве. Я уж не говорю о брошенных русских
людях.
\end{itemize} % }

\iusr{Евгений Гарец}
Вы глубоко заблуждаетесь, утверждая Его растерянность.

\iusr{Галина Бумах}
Мне кажется , не всё так прямолинейно. И не думаю , что действительно не знает .

\iusr{Сергей Соколов}

Когда кто-то пытается проследить за мыслью Путина, советую вспомнить российские
спецслужбы, прекрасно подготовленных дипломатов, десятки институтов изучающих
международные проблемы, политологов, российских политиков, которые собирают
информацию для Путина и формируют мнение России в мире. И второе. Когда русский
царь ловит рыбу, Европа может ждать», - сказал Александр III, Российский
император.

\begin{itemize} % {
\iusr{Галина Дегтярева}
\textbf{Сергей Соколов} Последним Александром III был Брежнев.
\end{itemize} % }

\iusr{Ирина Трифонова}

Что думает Путин, знает только сам Путин.  @igg{fbicon.wink}  Щас прям вот все вам расскажи,
ага...  @igg{fbicon.face.tears.of.joy} 

\begin{itemize} % {
\iusr{Василий Александрович Волга}
" а че сказать-то хотел?"))

\iusr{Ирина Трифонова}
\textbf{Василий Александрович Волга} поживём - узнаем  @igg{fbicon.beaming.face.smiling.eyes} 
\end{itemize} % }

\iusr{Sergey Novikov Lisovsky}
Там несколько вариантов развития событий прорабатываются постоянно по логике вещей. Все знают что делать..

\begin{itemize} % {
\iusr{Василий Александрович Волга}
возмоножно. мы ведь обсуждаем ответ Путина. не более.
\end{itemize} % }

\iusr{Александр Добровольский}
Будет война. Вот и всё.

\iusr{German Gorozhanski}

Если статистика о 60-ти процентах т.н. "недовольных Путиным", приведенная
Погребинским, верна, то среди этих недовольных немало тех, кто недоволен
бездеятельностью РФ и Путина в отношении Украины. А это совсем другие
"недовольные". Плюс не каждый готов ответить о своей лояльности Путину, -
боятся люди.


\iusr{Вадим Годун}

о чем можно говорить,когда совсем недавно , Путин размышлял о красных
линиях,которые к сожалению ,уже давно перешагнуты американцами. Очень печально.
Но все близится к "большому шухеру..."

\begin{itemize} % {
\iusr{Василий Александрович Волга}
и?

\iusr{Василий Александрович Волга}
где "шухер"?

\iusr{Paul Isaev}
\textbf{Василий Александрович Волга} там же, где и вступление Украины в НАТО. Зачем раскрывать карты раньше времени? @igg{fbicon.shrug} 
\end{itemize} % }

\iusr{Андрей Василич}

Есть по крайней мере два вида ответа «не знаю». Один «не знаю» - не имею
вариантов, поэтому не знаю что делать. И другой: вариантов несколько, не знаю,
какой выбрать...

\iusr{Кирилл Ольгинский}

Это хорошо, что он не знает. Значит, вышел из зоны комфорта (если действительно
не знает и действительно ему казалось, что знал). Если вышел из зоны комфорта,
значит, открывается окно возможностей для решений.

\begin{itemize} % {
\iusr{Василий Александрович Волга}
Ну, это интересная мысль!

\iusr{Евгений Рулев}
\textbf{Кирилл Ольгинский} 

Украинцы думают, что Путин только и думает о них. Он о России думает. И
потерять равновесие недопустимо. Штурвал корабля он держит крепко.

\end{itemize} % }

\iusr{Петр Иванов}

Согласен с теми кто пишет что безусловно план есть - точнее разные алгоритмы
действий в зависимости от ситуации. Другой вопрос что форсирования однозначно
не будет. Ни с одной стороны ни с другой. И этот вопрос (вступления в НАТО)
вероятно растянется на 10ки лет. Так что не факт что при нашей жизни что то
изменится

\iusr{Евгений Рулев}

Можно предположить, что на Украине на фоне экономической катастрофы и разных
бедствий произойдёт общая деградация и провал государственности. А там хаос,
махновщина и башня рухнет. Путин будет с холма наблюдать, что будет
происходить.

\begin{itemize} % {
\iusr{Василий Александрович Волга}
уже восемь лет ждем этого, да все мимо. четвертый год экономика Украины показывает уверенный рост.
\end{itemize} % }

\iusr{Юлия Хатковская}
Сколько лет была руина в18 веке? Около 25. Вы даже не в середине пути.

\begin{itemize} % {
\iusr{Галина Дегтярева}
\textbf{Юлия Хатковская} Мерси утешили.

\iusr{Юлия Хатковская}
\textbf{Галина Дегтярева} ,а мы кувыркались с начала 90 где-то до 10 года. 20 лет. И заметьте, зрелые демократии наших борцов за свободу и независимость кормили, лечили и снабжали всем необходимым.
\end{itemize} % }

\iusr{Михаил Уханов}

Ответ уже был у Гоголя. И Россия и Украина ожесточаются и готовятся к войне.

\begin{itemize} % {
\iusr{Ирина Капырина}
\textbf{Михаил Уханов} Украина готовится к земле, так правильнее и честнее. Аминь.

\iusr{Михаил Уханов}
\textbf{Ирина Капырина} было два брата: так и Украина расколется на две части.

\iusr{Дмитрий Белоногов}
\textbf{Михаил Уханов} боюсь, что именно так.
\end{itemize} % }

\iusr{Сергей Бурналейт}

А мне вот интересно - 60\% населения за НАТО, а 40\% против. А шо, был референдум
или это спрашивали у тех, у кого кружевные трусики в кармане до эвропы лежать?

\iusr{Ирина Капырина}

Я думаю, там уже все планы прописаны, что да как и главное, когда. Приоткроется окно возможностей, и будет вскрыт нужный конвертик.

\iusr{Константин Валентинович}

Вариант только один-вторжение.. увы.. и Крым был украинским пока не заговорили о
создании там баз нато и разрыве договора аренды для чмф... не видеть этого
преступно ,но когда загоняют в угол надо защищаться ...

\begin{itemize} % {
\iusr{Aleksandr Fareniuk}
\textbf{Константин Валентинович} ПО ПРОСЬБЕ ГРАЖДАН УКРАИНЫ, А ОНА БУДЕТ ИЛИ УЖЕ ЕСТЬ.
\end{itemize} % }

\iusr{Марина Хлебникова}

Путин что сказал? Что Россия и Украина переплетены семьями, как нигде. Что тут
делать? Военными действиями..это даже не обсуждается. Уезжать в Россию? Да.
Вариантов нету.. Именно НЕТУ. Мне незачем тут писать " трактаты", россияне
видели и слышали многое за восемь лет. И если многие украинцы говорят, что
правительство украинское плохое, ай-яй, тарифы повышает и прочее.... То Россия
это агрессор, а то!

Так зачем России такое на шею? Не надо. Уж извините. Хорошая была республика,
жалко возгордилась чрезмерно. Прямо напрашивается сравнение...

\begin{itemize} % {
\iusr{Oleg Dobrolyubov}
\textbf{Марина Хлебникова} Уезжать в РФ вместе с вокзалами ,домами, дорогами,полями и городами !

\iusr{Aleksandr Fareniuk}
\textbf{Oleg Dobrolyubov} 100 \%
\iusr{Aleksandr Fareniuk}
\textbf{Марина Хлебникова} 

ПОЧЕМУ Я ДОЛЖЕН УЕЗЖАТЬ КУДА-ТО , Я ОБАЖАЮ СВОЮ РОДИНУ. ПРОВЕСТИ ЧЕРЕЗ ТРИБУНАЛ
ВСЮ ЭТУ МАЙДАННУЮ НЕЧЕСТЬ ,А ЗАБЛУДШИХ ЧЕРЕЗ ДЕНАЦИФИКАЦИЮ.

\iusr{Марина Хлебникова}
\textbf{Aleksandr Fareniuk} проведите

\iusr{Aleksandr Fareniuk}
\textbf{Марина Хлебникова} ТАК ТОМУ И БЫТЬ.

\iusr{Марина Хлебникова}
\textbf{Aleksandr Fareniuk} ну, Лейте словесную воду и дальше. Вопрос поставлен, я лично, на него ответила. Конкретно.
Можете так жить, живите. Насильно же Вас никто не гонит с Украины. Обожайте на здоровье.

\iusr{Aleksandr Fareniuk}
\textbf{Марина Хлебникова} А ВЫ УВЕРЕНЫ ЧТО УКРАИНА ОСТАНЕТСЯ УКРАИНОЙ ?

\iusr{Марина Хлебникова}
\textbf{Aleksandr Fareniuk} В каком смысле пишите?

\iusr{Aleksandr Fareniuk}
\textbf{Марина Хлебникова} В ПРЯМОМ .ПУТИН СКАЗАЛ ЧТО МЫ ОДИН НАРОД .ВЫ С НИМ НЕ СОГЛАСНЫ?

\iusr{Марина Хлебникова}
\textbf{Aleksandr Fareniuk} почему не согласна то? Вы с чего вопросы такие задаете? Я что, не написала этого выше?
Я про Фому, а Вы мне про Ерему.

\iusr{Марина Хлебникова}
\textbf{Aleksandr Fareniuk} так один народ и выход жить в России. В чем вопрос? Волшебников не ждите, они в сказках. И спокойной уже ночи.

\iusr{Aleksandr Fareniuk}
\textbf{Марина Хлебникова} НЕТ ,ВЫ ПРО ТО ЧТО Я ДОЛЖЕН КУДА-ТО ПЕРЕЕЗЖАТЬ. У МЕНЯ ТАМ ПОКОЯТСЯ ДЕДЫ ПРАДЕДЫ .
\end{itemize} % }

\iusr{Данил Пасечник}

Давайте будем объективны. Россия никогда не знала, что делать с Украиной.
Например, вступление в войну с Польшей в 1654м (спустя 6 лет как идет польская
гражданская). В чем Россия была тогда уверена? Возвращение Смоленска и
Чернигова (до смутного времени российские) - достаточно веское оправдание для
вступления в войну. В чем еще могла быть уверена? Что русские люди (бывшие
польские граждане), переприсягнувшие русскому царю будут хранить ей верность.
Присягу приняло некоторое количество русских, иные русские воевали на стороне
Польши (про присягу киевской знати и духовенства ничего не известно). Россия не
была даже уверена, чей Киев, который по андрусовскому перемирию считался
польским, но все 20 лет ею не контролировался. Россия не знала, что делать с
Киевом, но заплатила за все потери, которые потерпела Польша в связи с утратой
Киева (он просто был выкуплен). Россия точно не знала что делать с северным
причерноморьем, которое отошло к ней по итогу победы над Турцией. Россия не
знала, что делать с правобережьем, которое ей досталось по итогу развала самой
Польши. Если оценивать ретроспективно, история может выглядеть как цепь
успешных событий. Но в конкретный исторический момент, "здесь и сейчас" может
оказаться, что никто не знает, что делать, а решения принимаются как ответ на
вызовы и провокации

\begin{itemize} % {
\iusr{Марина Хлебникова}
\textbf{Данил Пасечник} с какой Украиной еще Россия в 1654 м году не знала, что делать?

\iusr{Данил Пасечник}
\textbf{Марина Хлебникова} с восставшей частью Польши

\iusr{Марина Хлебникова}
\textbf{Данил Пасечник} с какой Украиной?
\end{itemize} % }

\iusr{Владимир Симонов}

Да, не знает что делать. И это печально ... Президент такой страны как Россия не
может «не знать»... Можно предположить разочарование своей роли, или усталость от
той же взятой на себя функции мирового развития... Но там в Валдае было одно
примечательным: на вопрос о будущем мира В. Путин ответил важностью внимания к
человеку. Это контрапункт. И в этом Знание. Значительно более весомого любого
«незнания».

\iusr{Леонид Леонидис}

У Путина не бывает безвыходных ситуаций, как относиться к странам НАТО он
знает, Думает над выходом, который устроит всех и при котором Украина
сохранится.

\iusr{Владимир Симонов}

Естественно Владимир Путин знает то что должен. И ,реализуя сверхзадачу
эволюционного развития человечества, он говорит о том, что важнее и НАТО, и
Америки и всего наносного и вторичного. Относительно новейшего: индивидуальной
ответственности каждого. Каждого!!! Время упования на эгрегоры и их силу ушло.
Что на смену? Другое... Эгрегорная связь государств трансформировалась. Новейшее
-в новом мироустройстве. И здесь бессилен и Путин, и Байден ,
и все все все.

\iusr{Владимир Симонов}

Избранность не в том чтобы демонстрировать свою исключительность,
но а обретении осознанного качества прозрачневения ,
синхронно эволюционными задачам вселенной.

\iusr{Лариса Портнова}

Он столько лет на виду,что некоторые его черты характера уже не новы. И
растерянность в их перечень не входит. А осторожность входит

\iusr{Владислав Сивачук}
Ну а там сказать нельзя?

\iusr{Максимук Виктор}

Василий Александрович,

ПРОРОЧЕСТВО ПО УКРАИНЕ (о. Дмитрия Смирнова).

Церковный раскол на Украине будет на совести Константинопольского патриархата,
которым сейчас манипулируют США. В ситуации с Украиной Константинополь -
среднее звено в цепи церковного раскола, первым звеном является Госдепартамент
США.

Раскольники и их поддержавший Фанар не веруют ни в какого Бога, забывают о нем,
поэтому жесткие последствия и кровопролития в стране могут произойти. Но при
этом патриарх Варфоломей, поддержавший начало раскола, не планирует каяться,
пока выполняет приказ Запада.

Патриархат Варфоломея затеян в пику России с подачи США. Так что не будет от
него никаких извинений - разве есть в Госдепе хоть один сотрудник, который
жалеет людей?

Фактически Фанар, затеявший всю ситуацию под дудку США, не сможет выйти сухим
из воды.

Патриарх Константинополя уже оказался в патовой ситуации: он никак не сможет
выйти из конфликта, не потеряв лицо. Американские власти через Церковь хотят
устроить кровавую бойню на Украине. Последствия могут быть схожи с кровавой
бойней в Ливии.

\ifcmt
  ig https://scontent-lhr8-1.xx.fbcdn.net/v/t1.6435-9/248888273_3071857379694156_2887553251056958792_n.jpg?_nc_cat=110&ccb=1-5&_nc_sid=dbeb18&_nc_ohc=DwPmiJh5y_4AX8rTTl5&_nc_ht=scontent-lhr8-1.xx&oh=a1c20b9bcd247eb53c148560e1c01abf&oe=619E35BF
  @width 0.4
\fi

\iusr{Nataliya Makashutina}

Ну, аообще он чётко сказал, что "...легальных методов, для смены власти в
Украине, не осталось.". Т.е...

\iusr{Максимук Виктор}

Василий Александрович, вы вспомните, как до сих пор все полиНтологи куражились,
что Украину в НАТО никто не возьмёт, совершенно не понимая смысла истории, а
ведь всё наоборот, непременно возьмёт и не только Украину, но и Грузию и
Молдавию и возможно и ещё кого-то. Идёт третья мировая всеми возможными
способами с Православием и Россией, чтобы наконец создать НМП по примеру
Китая...

\begin{itemize} % {
\iusr{Maya Miyagi}
\textbf{Максимук Виктор} НМП это что ?

\iusr{Лариса Портнова}
Новый мировой порядок
\end{itemize} % }

\iusr{Светлана Жучкова}

А мне показалось что Путин говорил о самой Украине- не может он выйти и на весь
мир провозгласить планы России по Украине- хоть и формально- но независимому
госсударству???? Он сказал что не представляет что делать жителям Украины с
такой властью-правовым путем ее не убрать, оппозиция сидит под лавкой- слова
сказать не может....- естественно в такой ситуации неизвестно как убрать этих
беспредельщиков.

\iusr{Елена Иванова}
\textbf{Василий Александрович}, 

я думаю, что ответ у Путина есть. И даже не один. Если исходить из того, что
Украина сама себе помочь не может, то все ответы плохие или очень плохие для
Украины. Я рискну предположить, что прежде всего рассматривается вариант с
"традиционными штампами".

\begin{itemize} % {
\iusr{Василий Александрович Волга}
Ну, я не умею читать мыслей, Елена, не обучен. Я просто слушаю то, что люди говорят. Путин говорит, я слушаю, и пытаюсь понять сказанное им.
\end{itemize} % }

\iusr{Святослав Ющенко}

Думаю, что он знает что делать. Просто не может дать ответ в данный момент,
т.к. это перспектива будущего и планы на это у России скорее всего имеются,
просто здесь есть множестов противодействий, которые он и не озвучит. Потому,
это скорее печальные мысли, чем незнание что делать.

\iusr{Маргарита Хан}

Всё он знает, другое дело, что он показывает внешне...значит он хочет, чтобы
думали, что он в растерянности @igg{fbicon.index.pointing.up}

\iusr{Николай Григорьев}

Такое уже было. Чтобы убрали ядерные ракеты из Турции, такие ракеты привезли на
Кубу. Толпе не обязательно докладывать, что и куда привезут.

\iusr{Виктория Куколкина}

Где растерянность и где Путин. Не с его "институтами" теряться.

\begin{itemize} % {
\iusr{Владимир Коваль}
\textbf{Виктория Куколкина} 

А тем более с его оружием. Лет семь тому назад на такой вопрос он отвечал, что
украинцы сами разберутся и судя по его ответу, он начинает понимать что они не
торопятся разбираться и этого он понять не может, точно как и автор
публикации..

\end{itemize} % }

\iusr{Лариса Палагина}
отправлять войска спасать зад украины он не будет. медведев все сказал..С А М И

\iusr{Maya Miyagi}
Не верю , что не знает. План точно есть.

\begin{itemize} % {
\iusr{Василий Александрович Волга}

возможно. слишком, видать, он перспективный. 8 лет прошло, а воз и ныне там.
многие говорят, что в этом план и состоит, чтобы не иметь никакого плана.

\end{itemize} % }

\iusr{Людмила Александрова}
А есть ли силы на самой Украине, что бы было на кого опереться в помощи?
Т.н. ОПЗЖ не только себя не проявляет, но откровенно сидит "в домике"
И зачем за таких впрягаться?

\iusr{Катя Костюковская}

Россия не «не знает». Тут большинству безразлично. Союз распался давно, Украина
- во всяком случае юридически - независимое государство, которое к России имеет
отношение только границами и трубой.

\begin{itemize} % {
\iusr{Владимир Коваль}
\textbf{Катя Костюковская} Разве?
\end{itemize} % }

\iusr{Vadim Tokarev}
Не всегда нужно говорить открыто о своих планах. Зе Путину - не пример.

\iusr{Василий Коростенец}

Враньё многие мои знакомые изменили своё мнение став положительно относится к
России . Так что пизд@@@т ваши социологи и Вы вместе с ними .

\begin{itemize} % {
\iusr{Василий Александрович Волга}

везет Вам со знакомыми. Я мы все больше смотрим на социологию и результаты
голосования, которые показывают, что на прошлых выборах в горсовет Киева,
например, политики русофобы набрали в общем зачете 87\%

\end{itemize} % }

\iusr{Коля Кузнецов}

Это примерно как «не знать», что делать с чирием, который создаёт сильный абсцесс.
Перспектива, что он сам рассосётся - маловероятна. Остаётся вопрос - какого уровня будет хирургическая операция.

\iusr{Светлана Лаврухина}

Наивно думать, что у Путина нет плана или ответа. Есть. Он старается
минимизировать ущерб стране, которой он служит в качестве президента и стране,
которой плевать на собственный народ.

\begin{itemize} % {
\iusr{Василий Александрович Волга}
да мы не против. мы ведь просто обсуждаем его ответ: "Я не знаю, что делать"

\iusr{Светлана Лаврухина}
\textbf{Василий Александрович Волга} , это он надеется на дискуссию.
\end{itemize} % }

\iusr{Елена Самойленко}

Больше занят вопросом как всех насильно добровольно вакцинировать. И как
быстрее избавится от пенсионеров и бедных

\begin{itemize} % {
\iusr{Larisa Kuprina}
\textbf{Елена Самойленко} Для этого он организовал материальную помощь семьям с детьми, пенсионерам, военным и прочая!

\iusr{Елена Самойленко}
\textbf{Larisa Kuprina} да, именно для этого лишил пенсионеров пенсии на 5 лет, но сунул подачку в 10 000,0. Миллион забрал 10 000,0 сунул как собакам. Хорошая экономика
\end{itemize} % }

\iusr{Иван Алексеев}

Вопрос должен быть адресован не ему, а украинцам. И у нас есть на него ответ -
эмиграция. Печаль, беда, но это факт. Если мы будем оглядываться на то, что
скажут соседи - этот ответ останется единственным.

\iusr{Леонид Зарапа}

В данный момент и здесь - Американцы нас переиграли с Украиной, но как говорят
Русские - Ещё не вечер. Крымская война в 1854 году была проиграна Россией, но
статус КВО она себе всё равно вернула, через время. Так же мы проиграли Японцам
в 1905 году и Цусиму и Порт Артур и Сахалин, но в 1945, через 40 лет мы снова
восстановили свой статус КВО, да ещё и с контрибуцией Курильских островов.
Так-что по Украине наш статус КВО будет восстановлен. Проверено неоднократно
историей.

\iusr{Константин Филиппишин}

Его неуверенность мне говорит о том что он пока не решился. Но сценарии, думаю,
давно обработаны. Вопрос в том, как во времени будет ситуация развиваться. Но в
любом случае это будет тяжёлый выбор

\iusr{Лариса Забродина}
Не думаю, что у В.В. Нет плана Б.

\iusr{Dmitry Serebrov}

Вы плохо знаете повадки разведчика. Это первое. Важнее другое: когда жители
Украины сами начнут решать свою судьбу? Или всем нравится нынешнее положение и
состояние их страны?...

\begin{itemize} % {
\iusr{Светлана Чемоданова}
\textbf{Dmitry Serebrov} Вы себе не представляете, в каком мы восторге! Особенно русские, особенно повершившие в 2014-м.

\iusr{Dmitry Serebrov}
\textbf{Светлана Чемоданова} 7 лет терпеть ЭТО - сродни мазохизму...

\iusr{Светлана Чемоданова}
Мы тоже думали, что 7 лет предательства - должно быть неуютно на сердце. Но могли бы хоть не пинать оставшихся в этом концлагере.

\ifcmt
  ig https://scontent-lhr8-1.xx.fbcdn.net/v/t39.30808-6/249239159_277838474249287_5958270640605857727_n.jpg?_nc_cat=106&ccb=1-5&_nc_sid=dbeb18&_nc_ohc=qBpTU8ocytsAX_DcgeH&_nc_ht=scontent-lhr8-1.xx&oh=1d3d424df81970afd1adb22ff1b6cc5e&oe=617E6F6D
  @width 0.7
\fi

\end{itemize} % }

\iusr{сергей шашкин}

Не будьте так наивны - знает. Он четко сказал, что это уже не зависит от народа
Украины, а на сколько и куда шагнут упоротые сейчас не может предсказать никто.
Россия будет действовать по результатам.


\end{itemize} % }

