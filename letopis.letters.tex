% vim: keymap=russian-jcukenwin
%%beginhead 
 
%%file letters
%%parent body
 
%%url 
 
%%author_id 
%%date 
 
%%tags 
%%title 
 
%%endhead 

\ii{letters.27_01_2023}
\ii{letters.06_02_2023}
\ii{letters.14_02_2023}
\ii{letters.18_02_2023}

%https://www.facebook.com/permalink.php?story_fbid=pfbid0sbMsAukawJAnZeA8fdWBvWhfxh4pusncRbwfHWN2sQYRxNnjbGn4FkQ5fDmWm4ajl&id=100087641497337

Надіє, дуже важлива книга! Мені, якщо чесно, як людині, що на щастя не бачила
наочно всього цього пекла, хоча я багато читав і читаю постів про Маріуполь, і
я відчуваю ваш біль, який так, відчуваєш, навіть якщо всього цього не бачив,
важко коментувати, оскільки подумки питаєш себе, що і як варто казати людям, що
все це пережили. Але... я мушу зізнатись в такому. Ще до війни я бачив вже
приблизно все те, що було в Маріуполі... у фільмі. Польському фільмі. Фільм
називається Місто 44 (Город 44). Страшний фільм насправді, і я в той момент
зовсім не думав, що всі ті жахіття, що були показані у тому фільмі, колись
будуть повторені вже в Україні. І цей фільм про трагедію Варшавського
повстання. Так, трагедія такого ж масштабу...  Хоча інша країна, інша історична
епоха, інші конкретні обставини. Так, у 1944 році, коли Варшава була під
німецькою окупацією, поляки підняли повстання проти нацистів, яке було в
результаті надзвичайно жорстоко придушене. Надзвичайно жорстоко. Місто було в
вогні, зруйновано, близько 100 000 поляків було в результаті вбито. Поляки
дотепер пам'ятають про це. Але... Варшава... якщо ходиш вулицями цього міста -
красивого, гарного міста... не відчуваєш, що тут колись коїлось... Так
заведено, що Міста мають внутрішню, так, внутрішню велику Силу... яка все
подолає і переможе! І так! В кінці фільму - якщо знайдете час колись подивитись
- дуже зворушливо. По-перше... (якщо подивититесь, побачите самі... - все
розказувати наперед напевне не треба... )  І по друге, що стосується власне
Міста Варшава. Кінець фільму... Палаюче знищене місто зникає... І постає
Варшава наших днів... І оцей перетворення із палаючого міста у сучасне мирне
усміхнене європейське місто колись мене надзвичайно зворушив, аж до сліз... І
так само Маріуполь. Місто залікує свої рани, так само як і маріупольці. І
колись, через десятиліття те, що відбувалось, залишиться лише у книжках та у
відео та фільмах, та пам'яті народу. Але приклад поляків дуже показовий. Вже
скільки років пройшло, а вони зняли цей фільм зовсім недавно, хоча стільки
років вже пройшло! Але вони пам'ятають, так, і дотепер, і будуть пам'ятати
завжди! 

от ви кажете, що зоряні війни, то є нудно... знаєте. А я почну здалека... з
Природи. Ось природа... виходите на вулицю... вдихаєте свіже повітря...
радієте... Сонечко світить... чи не задумувались ви про те, як все прекрасно,
взаємопов'язано між собою. Сонце світить - дощ ллє. Земля всмоктує в себе
вологу, і дає життєдайні сили, щоби прекрасна квіточка - скажімо, десь у Києві
на Співочому Полі - або у бабусі в селі -  проросла. Потім з'являються люди,
висаджують квіти, створюють парки, як от парк Кейкенхов в Голландії, або ж
дитячий парк Веселка в Маріуполі - справжня міні Голандія у прекрасному місті
біля Моря... Так от. Сонце світить, вітер віє... дощ ллє... Саме по собі Сонце
- то є величезна розжарена куля, на якій немає ніякого життя, там панує лише
Вогонь і Смерть... Так, Величезний Страшний Вогненний Океан під назвою Сонце,
розміром із тисячу таких планет як Земля... Але... світло від Сонця досягає
Землі, і дає в результаті життя прекрасній тендітній квіточці... Одне - велике,
неймовірно велике, дуже гаряче і смертоносне - дає в результаті життя чомусь
дуже малому, тендітному, крихкому... Так... В Природі, у Всесвіті все
взаємопов'язане... мале пов'язане з великими... мертве - з живим, гаряче - з
холодним... А як можна назвати те, что все це уособлює собою? Одні кажуть -
Матір-Природа, інші кажуть - Бог, Творець. А в цьому фільмі центральним
лейтмотивом проходить ідея Сили... Сили, що все об'єдную між собою... Дуже
філософський фільм насправді. І показовий діалог з передостанньої, восьмої серії
фільму (Останні Джедаї)... 

Таймкод - 49-10

Люк - тянись до неї своєю душею...  що ти бачиш?
Рей - острів... Життя... Смерть та тлін... початок нового Життя... тепло...
холод... мир... насилля... 
Люк - а що їх об'єднує?
Рей - баланс... і енергія... Сила...
Люк - а в тобі самій...
Рей - і в мені самій... та ж сама Сила... 
Люк - це і є твій урок... Сила не є власністю джедаїв...
ідея, що зі смертю джедая вмирає Світло, є хибною... ти відчуваєш це!? 

Ну от. У Всесвіті все пов'язано. Як колись писав англійський поет-романтик Вільям Блейк, 

Бачити світ у зерні піску, Небо — у квітці синій, В своїй долоні безкрайність усю І вічність в одній годині.

А Сила - то є назва тому, що все об'єднує
в одне ціле, і далекі галактики, і розжарене Сонце, і тендітну квіточку десь у бабусі на городі.
Ось у цьому власне і є основна ідея фільму, ідея Сили, так, ідея Сили.

%https://t.me/maripol_hope/1291

Дуже влучно сказано про останню крапку, що поставить Життя! Бо дійсно, так і
буде! Життя переможе Смерть. І щоби це сталось насправді... потрібно...
навчитись смерть перемагати у собі, в душі, в серці... перемагати відчай,
розпач, відчуття безнадії... Розумію, що для маріупольців це важко зараз... Я -
киянин... дивлюсь все це зі своєї сторони... Але... знаєте. Я от десь недавно
прочитав у одного автора - він каже про нескінченний постійний потік фото  в
телеграмі про зруйнований Маріуполь. Я скажу так... Це звичайно дійсно так.
Але... Інтернет - така штука... що тут ти насправді сам вибираєш, що ти можеш
дивитись... Я, коли заходжу в телеграм або фейсбук, звичайно, також бачу багато
фото або історій поганих про Маріуполь. Але... також! Нескінченний, дійсно
реально нескінченний потік радісних, гарних фото, історій про Маріуполь на
різних ресурсах та в абсолютно різних ракурсах! З яких на нас дивиться саме
Життя! Так, саме Життя, бурхливе невгамовне Життя, а не Смерть! Більше 200
років Життя, акумульованого у цих прекрасних та різноманітних фото чудового
міста біля моря, так - 200 років - бо фото, відео, історії, враження - це
акумуляція, сумарний ефект величезної Духовної Сили Міста, що стоїть біля Моря
вже більше двох століть! у порівнянні з цим роком смерті... Що ж переможе
врешті решт? 200 років Життя, чи рік (два-три) занепаду та смерті? Я думаю,
Життя! Так, Життя! Українське Життя в Українському Маріуполі! 💛💙💛💙

% 23.02.2023
% https://t.me/maripol_hope/1307

Доброго ранку! 

Щодо книги - то є велика подія. Обов'язково прочитайте, книжка сама по собі вас
не вб'є!  Так, можливо, буде боляче читати, але це лише книжка. Її завжди можна
відкласти в сторону, почитати пізніше, повернутись, перегорнути сторінки,
закрити, подумати, уявити, згадати...  потім знову повернутись до тих чи інших
рядків... Книжка - то є велика духовна річ, яка дійсно здатна лікувати душевні
рани! Але... у мене тут є такі думки. Щодо повернення у звільнений Маріуполь.
Скажу таке, це напевне не дуже вам сподобається, що я скажу. Зауважу наперед,
що я ніколи не був у Маріуполі, і тому... як це парадоксально не звучить, він
для мене такий самий живий - на фото, відео, як і до війни. Відсторонена
позиція, так сказати, тут допомагає це усвідомлювати.  Так, я співчуваю
маріупольцям душею і серцем, але я також знаю - що справжній Маріуполь живий,
хоча він і залишився таким лише у пам'яті та у фото та відео. А щодо звільнення
Маріуполя. Ви мрієте про повернення, уявляєте як все буде, як ЗСУ звільнять
місто... от можна уявити собі, ...  найоптимістичний варіант... російські
війска розбиті вщент, путін вмер або його вбито, військова перемога України
здобута, росія підписує принизливий для себе мир, навіть погоджується на
репарації, і виводить свої війська взагалі... Ви купляєте квиток... радісно
їдете на поїзді Київ-Маріуполь, наприклад... і знаєте. тут на вас чекає
смертельна небезпека. Абсолютно і без жартів.  А чому... Тому що ви приїдете в
напів-мертве місто, в місто-примару. ЗСУ може звільнити місто фізично, але
звільнити місто в душі і серці - то є справа персональна кожного, справа
духовна, інтимна, тут ніякий хаймарс не допоможе. Так от... Повернетесь... В
місто, яке стало Кладовищем, і в місто, яке є лише Тінню Маріуполя. В місто,
яке зараз насичене невидимою страшною духовною отрутою... І це є страшна
небезпека для душі, для серця... Для тих, хто вживу знав його зовсім іншим...
Оскільки це може просто вбити на місці... як тільки ви приїдете туди,
пройдетесь вулицями, дійдете до тих місць, які колись робили вас щасливими... і
усвідомите на повну і всерйоз, що повернення до того минулого, яким ви його
пам'ятаєте, вже ніколи не буде... В таких випадках людина може просто вмерти на
місці від атомного вибуху у серці та душі, або зійти з розуму і назавжди...
Так, ваше найрідніше місто може вас вбити, якщо ви душевно, духовно не станете
як слід сильними, готовими до цього повернення !  Так що...  Звільнення,
повернення Маріуполя у душі, у серці повинне передувати справжньому поверненню
у реальності.  І чим більше спогадів, чим більше всього, що вас поєднує із
Маріуполем, тим страшніша ця небезпека, і тим більша ймовірність, що без
відповідної розумової і душевної підготовки, духовної роботи, мертве отруєне
місто вас просто вб'є на місці, як удар блискавки в дім без громовідводу.

%Дякую.. 

%Я теж про це думаю! Додому хочется, але.. Я вирішила після деокупації поїхати
%на тиждень, подивитися, придивитися ЯК мені, дитині .. Ми зможемо там жити!? А
%який контингент навколо буде? Не відразу повертатися. Можливо там буде жахлива
%енергетика💔😣 Але як то кажуть ми не перші .. багато міст відновили після війн
%і зараз це квітучі міста!


Доброго дня! Щодо контингенту, відповідь на коментар вище... як там буде... не
в цьому річ, абсолютно не в цьому. А в Ваших персональних відносинах, так,
інтимних душевних відносинах із Вашим Містом... Розумієте. Неважливо насправді,
який там буде контингент на момент вашого приїзду у звільнене місто. Головним
тут є ваша зустріч із Містом один-на-один.  І так... щодо контингенту.
Припустимо, що... ось...  Маріуполь звільнено, більше того, всі зрадники та
колаборанти або заарештовані, або просто повтікали...  Зібрався цілий потяг
друзів, маріупольців, всі щасливі, радісні, нарешті ми їдемо додому!!! Поїзд
прибуває на розгромлений вокзал...  перші гнітючі думки... руйнування усюди,
незважаючи на те, що місто вже наше, і гордо майорить вже наш найгарніший у
світі прапор над будівлею мерії... потім, ви всі розходитесь... або гуляєте
разом...  спогади, минулі жахи оживають... ось тут було оце, ось тут було це.
Тут у дворі поховані мої друзі, там заживо згоріла бабуся... Тут був будинок із
муралом...  нема вже ні будинку, ні муралу, ні тих людей, що там жили....
Розумієте...  І Місто починає вас душити, всю вашу прекрасну компанію.
Безжалісно душити, безжалісно бомбардувати вашу душу, ваше серце. Починає
душити всіх вас поодинці, жорстко, безжалісно розбиває вашу єдність... Береться
за кожного, і починає роздирати ваші душі, ваші серця на частини... Мертве,
зруйноване - хоча і звільнене місто, але ще наскрізь отруєне місто... береться
за кожного маріупольця, і починає творити над ним (нею) свій Страшний Суд...
Спогадів, з'єднаних із жорсткою реальністю, все більше і більше, як смертоносна
сніжна лавина в Карпатах зимою, вони сходять на вас, зносячи все на своєму
шляху... вони починають душити вас звідусіль...  Так!  Пекло, через яке ви
колись пройшли, і з якого ви вибрались живими, і яке зараз залишилось лише у
книжці Надії та ваших спогадах та фото і відео в інтернеті, повертається
вживу... В результаті три можливі варіанти - або ви берете квиток і наспіх
тікаєте з міста щодуху, щосили, щоби туди більше ніколи не повернутись... або
ж...  Місто просто вбиває вас насмерть...  через інфаркт, інсульт, просто
вбиває на місці... або ж.. Ваша Внутрішня Сила, так, Ваша Духовня Броня - дає
відсіч демонам темряви і відчаю... (бажаний варіант) Ви виходите переможцем
врешті решт... І Місто відпускає свої сталеві єжові рукавиці від вас, і каже
вам - залишайся тут, друже. Залишайся тут, я, Українське Місто Маріуполь,
найгарніше місто біля моря у світі, буду з тобою скільки ти забажаєш. Залишайся
тут, вкладуй свою Душу, своє Серце в мене, в Місто Маріуполь, і я тобі віддячу
сповна! Будуть тобі і фонтани, і новий парк Веселка, і новий ДрамТеатр, і
радісні веселі новорічні трамваї, і просто трамваї, і тролейбуси, і може навіть
метро! все буде, не сумнівайся! Ось так я це все бачу...
