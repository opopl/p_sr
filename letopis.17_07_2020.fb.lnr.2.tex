% vim: keymap=russian-jcukenwin
%%beginhead 
 
%%file 17_07_2020.fb.lnr.2
%%parent 17_07_2020
 
%%endhead 
\subsection{Майданщики подрались с полицией и провели ритуал в розовом дыму}
\label{sec:17_07_2020.fb.lnr.2}
\url{https://www.facebook.com/groups/LNRGUMO/permalink/2854306238014265/}
  
\vspace{0.5cm}
{\ifDEBUG\small\LaTeX~section: \verb|17_07_2020.fb.lnr.2| project: \verb|letopis| rootid: \verb|p_saintrussia|\fi}
\vspace{0.5cm}

Потасовкой с полицией и кострами с песнопениями под стенами украинского
парламента ознаменовался второй день акции в защиту закона о тотальной
украинизации.

Как передаёт корреспондент «ПолитНавигатора», конфликт с протестующими
разгорелся на фоне того, что правоохранители отказались пропускать
«активистов», которые несли с собой брёвна для «ватры», которая представляла
собой сложенные шалашом дрова и лозунг: «Не подливайте масла в огонь, не
трогайте язык!»

Толкотня и взаимные проклятия фактически переросли в потасовку, пока в конфликт
не вмешались извне и пообещали, что костёр разжигать не будут.

Огонь действительно никто не разжигал, протестующие ограничились файерами, тем
не менее канистра с горючим стояла возле импровизированного очага.

Майданщики в розовом дыму стучали в вёдра и распевали ритуальные песни.

Напомним, что протесты вызваны попыткой рассмотреть в парламенте законопроект
продлить срок права получения образования на русском языке до 1 сентября 2023
года.

Вадим Москаленко 
