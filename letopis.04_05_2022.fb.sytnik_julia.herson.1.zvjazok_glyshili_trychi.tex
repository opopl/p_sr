% vim: keymap=russian-jcukenwin
%%beginhead 
 
%%file 04_05_2022.fb.sytnik_julia.herson.1.zvjazok_glyshili_trychi
%%parent 04_05_2022
 
%%url https://www.facebook.com/Juliya.Sytnik/posts/1612836979087374
 
%%author_id sytnik_julia.herson
%%date 
 
%%tags 
%%title Зв'язок глушили вже тричі
 
%%endhead 
 
\subsection{Зв'язок глушили вже тричі}
\label{sec:04_05_2022.fb.sytnik_julia.herson.1.zvjazok_glyshili_trychi}
 
\Purl{https://www.facebook.com/Juliya.Sytnik/posts/1612836979087374}
\ifcmt
 author_begin
   author_id sytnik_julia.herson
 author_end
\fi

Зв'язок глушили вже тричі. Першого разу ми дуже перелякалися. На вулиці майже
не було людей. Сусіди не відчинили двері, навіть ті, що бачили нас у камери. І
ми пішли шукати тих, хто відчинить. І знайшли. Коли ми дізналися, що у Лєрки
працює провідний інтернет, щастю не було меж. Я одразу записала всім купу
голосових, щоб не гаяти час та почала дзвонити близьким, щоб заспокоїти. Також
всі ми, ті хто залишився, та ті, хто виїхав почали обмінюватися адресами та
винаходити нові шляхи для спілкування. Ті, кому за 30 почали згадувати
дитинство чи молодість без інтернету та мобілок. Одним із наших \enquote{винаходів}
стали записки в поштових скриньках, де ми домовлялися про час і місце зустрічі,
або обмінювалися новинами.

\ii{04_05_2022.fb.sytnik_julia.herson.1.zvjazok_glyshili_trychi.pic.1}

- картопля сьогодні по 15 грн. на Мира, біля ринку, може ще встигнете купити.

- зайдемо після обіду о 15.00 

- взяла вам молочку, зайди, як прочитаєш. 

- зайняли вам чергу на молокозаводі.

Другого разу зв'язку не було цілий день, але у багатьох був інтернет, тому люди
сильно не переймалися. 

А от на третій раз, почалася тотальна паніка. 1 день. Оператори мереж в
смартфонах відсутні, інтернету немає ніде ніякого. Люди виходять на вулиці,
знайомляться, шукають спільні рішення. Працює сарафанне радіо. Так ми швидко
дізналися про всі точки у місті, де можна "прорватися" у світ, а також про
скайнет.

Ми натикаємося на хлопця, який відправляє смс. І просимо відправити і нам з
його смартфону. 

- А якщо буде відповідь, щось написати?

- Напишіть, що ви просто незнайомий херсонець. 

Чоловік розмовляє по телефону!! 

- Ви що кудись додзвонилися?

- Так, зі старого телефону. Я воєнний звязківець. - і дістає з десяток таких же
моделей того часу. - Прилетів у Херсон вчора на особистому літаку, забирати
сина з полону. 

- а приземлилися куди? У Чорнобаївку?

- місця треба знати!

2 день. Купа людей \enquote{висить} на кораблі на Лілеї, бо там є зв'язок. Люди
скупчуються в певних місцях для зв'язку з рідними. Напруження ізоляції спало.

\ii{04_05_2022.fb.sytnik_julia.herson.1.zvjazok_glyshili_trychi.cmt}
