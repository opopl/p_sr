% vim: keymap=russian-jcukenwin
%%beginhead 
 
%%file slova.sistema
%%parent slova
 
%%url 
 
%%author 
%%author_id 
%%author_url 
 
%%tags 
%%title 
 
%%endhead 
\chapter{Система}
\label{sec:slova.sistema}

%%%cit
%%%cit_head
%%%cit_pic
%%%cit_text
Сюрпризы начались сразу же: выяснилось, что на станции не открылась крышка
отсека научной аппаратуры и отказали шесть из восьми бортовых вентиляторов
станции. Это ставило под угрозу большинство научных экспериментов и могло
вызвать серьезные осложнения в работе \emph{системы жизнеобеспечения
космонавтов}.  В первопроходцы выбрали самый опытный экипаж. За плечами
командира Владимира Шаталова и бортинженера Алексея Елисеева было уже по два
полета. Исключение составлял лишь инженер-исследователь Николай Рукавишников,
который еще не бывал в космосе. Среди основных задач миссии значились стыковка
и доставка первых космонавтов
%%%cit_comment
%%%cit_title
\citTitle{«Они были обречены» 50 лет назад погиб экипаж «Союза-11». Кто
виноват в главной трагедии советской космонавтики?: Космос: Наука и техника:
Lenta.ru}, Сергей Варшавчик, lenta.ru, 30.06.2021
%%%endcit

%%%cit
%%%cit_head
%%%cit_pic
%%%cit_text
Что будет. Разобраться в ожидаемых перипетиях перехода различных \emph{систем},
в том числе общественно-политических, нам поможет \emph{теория систем} –
сравнительно новая научная дисциплина, созданная на стыке некоторых
фундаментальных и прикладных наук, среди которых есть и философия, и
математика. Как видно из ее названия, эта теория занимается изучением различных
классов, видов и типов \emph{систем}, исследует их основные принципы и
закономерности их поведения, процессы их функционирования и развития, в том
числе процессы перехода \emph{систем} из одного установившегося состояния в
другое. Отвлекаясь от конкретной природы изучаемой \emph{системы} и основываясь
на формальных взаимосвязях между ее различными компонентами и на характере их
изменений под воздействием различных внешних факторов, \emph{теория систем}
позволяет выстроить унифицированные закономерности, управляющие
функционированием системных объектов.  Как учит нас эта теория, процесс
перехода от одной \emph{системы} к другой разделяется на несколько этапов
%%%cit_comment
%%%cit_title
\citTitle{Украина после Зеленского}, 
Sergius Ak, hvylya.net, 07.12.2021
%%%cit_url
\href{https://hvylya.net/analytics/243091-ukraina-posle-zelenskogo}{link}
%%%endcit
