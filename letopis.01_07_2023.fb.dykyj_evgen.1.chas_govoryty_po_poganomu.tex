%%beginhead 
 
%%file 01_07_2023.fb.dykyj_evgen.1.chas_govoryty_po_poganomu
%%parent 01_07_2023
 
%%url https://www.facebook.com/evgen.dykyj/posts/pfbid0coha5LnMAjpcniJRc6BjbhDrDaQE1oKzu9QBdxPWWYYGyyfNxEt828VXCNd38CdQl
 
%%author_id dykyj_evgen
%%date 01_07_2023
 
%%tags 
%%title ЧАС ГОВОРИТИ ПО-ПОГАНОМУ?
 
%%endhead 

\subsection{ЧАС ГОВОРИТИ ПО-ПОГАНОМУ?}
\label{sec:01_07_2023.fb.dykyj_evgen.1.chas_govoryty_po_poganomu}

\Purl{https://www.facebook.com/evgen.dykyj/posts/pfbid0coha5LnMAjpcniJRc6BjbhDrDaQE1oKzu9QBdxPWWYYGyyfNxEt828VXCNd38CdQl}
\ifcmt
 author_begin
   author_id dykyj_evgen
 author_end
\fi

ЧАС ГОВОРИТИ ПО-ПОГАНОМУ?

Ворог контролює небо, перевага у авіації над нашою у понад десять разів. Ми
економимо кожний снаряд, ворог витрачає їх в рази більше. Союзники надали нам
бронетехніку із розрахунку приблизно на штат двох американських дивізій. Перед
нам фронт завдовжки півтори тисячі кілометрів, сотні квадратних кілометрів
щільних мінних полів, та ешелонована оборона з трьох ліній укріплень. Це – та
реальність, в якій розпочався і наразі відбувається наш наступ.

Реальність, в якій жодна із армій наших союзників ніколи не наступала, навіть
не пробувала нічого подібного. В їхній теорії це неможливо. Теорія насправді
мудра та вірна, в таких умовах наступати не варто. Але в нашій реальності ми
просто не маємо вибору – тому наступаємо. І попри всі ці умови та попри
передбачення теорій, досі продовжуємо наступ та щодня зубами вигризаємо не
просто чергові метри нашої землі, а чергові позиції, чергові ворожі опорники та
блоки, тобто потроху, повільно але впевнено стираємо з карти першу з трьох
ліній ворожої оборони, разом з її оборонцями. І невблаганно наближаємося до
другої – ще щільнішої...

Тим часом у іншій реальності ситий гарвардський професор політології, який
бачив війну виключно в телевізорі, бере до рук калькулятора, і ділить відстань
до Севастополя на відстань, яку за добу пробили наші бійці. І з поважним видом
пише, що до звільнення Криму нам лишилося біля 16 років. 

Він не здатний збагнути, що на війні відстані не лінійні – буває один кілометр,
який доводиться проламувати місяцями, а далі за лініями оборони лежать сотні
км, які потім проходяться за добу – якщо перед тим ти здолав отой перший
кілометр. Це зрозуміло кожному, хто воював, але що до того пихатому гарвардцю,
він бачить карту в Інеті та вважає це достатнім для мудрих заключень.

Цю маячню друкують впливові видання, і вона вливається у цілий хор інших
публікацій, загальний зміст яких зводиться до одного: союзники, які не надали
нам навіть одної десятої від тих ресурсів, які за їхніми військовими доктринами
необхідні для розпочатої нами наступальної кампанії, \enquote{розчаровані} низьким
темпом нашого просування, який \enquote{не виправдовує їхніх очікувань}.

У нас із очікуваннями все в порядку, принаймні у армії, та у тих у тилу, хто
хоч трохи дотичний до війни. Ніхто не чекав повторення \enquote{харківського дива}
минулої осені, адже умови кардинально відрізнялись. Не те що не було ефекту
несподіванки – навпаки, ворогові наперед були очевидні всі можливі напрямки
наступу, їх не так багато, і рівно всі ті 8 чи 9 місяців, що ми збирали ресурси
для наступу (зокрема буквально поштучно виклянчували у союзників броню та
арту), ворог окопувався, будував бетоновані доти, мінував підступи до них, і
готувався відбивати наші атаки. Завдання відпочатку було на межі можливого,
якщо не за цією межею, і наразі сюрпризом є не повільний темп наступу, а швидше
те, що він в цілому поки що проходить більш-менш успішно (за теоретично
неможливих передумов), і що втрати ворога в ході нашого наступу значно
перевищують наші (знову ж таки, всупереч теорії, де мало би бути навпаки).

Тож склалась парадоксальна ситуація, коли на Південному та Східному фронтах
обстановка краща, аніж на безкровному, але від того не менш важливому
\enquote{Західному фронті}. Здається, там ми наразі не те що не в наступі, а маємо
навпаки, відбивати потужну інформаційну атаку, чи навіть наступальну кампанію,
яка визначає зміни громадської думки, від якої в свою чергу напряму залежить
постачання нам життєво необхідних ресурсів, та те, як скоро нас стануть
примушувати до програшної для нас \enquote{заморозки конфлікту}.

Це наразі стає чи не головною нашою проблемою, тож варто розібратись із суттю
проблеми, та спробувати знайти можливі рішення.

Для початку відкинемо емоції, та відмовимось від пози «ображених на союзників»,
адже ображатись нам реально нема на кого та немає за що. Союзники такі які вже
є, інших у нас немає, ми їх не переробимо і не змусимо дивитись на світ нашими
очима, відчувати наш біль та думати нашими мозками.

Могло не бути і таких союзників та навіть такої допомоги – можемо пригадати як
світ свого часу відмовив у допомозі спершу УНР, потім УПА, і порівняти із
ставленням до нас зараз. А ще можемо порівняти обсяги допомоги нам із обсягами
нашої допомоги народу Сирії або раніше народу Чечні під час російських
бомбардувань – щоб не виникало питання, як же можна так байдуже сприймати нашу
трагедію: от рівно так само як ми сприймали трагедії інших. Своя сорочка ближча
до тіла всім та завжди, не виняток ні ми, ні союзники.

Також не забуваємо, що не союзники заважали нам готуватись до цієї війни. Ми
самі навіть ті вісім років, коли війна вже йшла, чомусь не готувались до того
що вона стане тою ж по суті, але зовсім іншою за масштабом – хоч це було досить
очевидно. І саме тому ми тепер кожний снаряд мусимо випрошувати у союзників,
хоч за 8 років могли мати всі необхідні запаси і зброї, і техніки, і БК, так
само як і навчений армійський резерв.

Ми самі всі ті 3 чи 4 місяці, коли розвідки союзників вже малювали нам на карті
всі стрілочки, за якими потім дійсно заїхали інтервенти, замість робити вздовж
цих стрілочок засідки, впоперек стрілочок лінії оборони, завчасно озброювати
ТрО та провадити вчасну мобілізацію, готувались смажити шашлики на \enquote{травневі} –
і саме тому всі наші перемоги у цій війні оплачені незрівнянно вищою ціною,
аніж могло б і мало би бути. Власне саме тому зараз нам знадобилась ця
відчайдушна та надскладна наступальна кампанія – хоч утримати Південь навесні
2022 було б незрівнянно легше, аніж тепер відбивати його назад.

Тож не валимо на Байдена, Трампа та Саллівана всі гріхи, і не вважаємо себе
\enquote{білими та пухнастими} – ми дійшли до сьогоднішнього стану, зокрема до повної
залежності від зовнішньої допомоги, та до необхідності звільняти з-під
окупантів майже кожний п’ятий квадратний метр нашої країни, спершу через низку
своїх помилок, і лише потім до наших проблем додалися дивні \enquote{червоні лінії} в
головах наших союзників. Але наразі ми вже робимо все можливе і неможливе, і
тепер саме ці \enquote{червоні лінії} та \enquote{таргани} у головах стейкхолдерів є найбільшою
перешкодою на шляху до нашої перемоги.

Що ж не так? Не вистачає аргументів? Ми непереконливі чи незрозумілі? Наче ні.
Наші аргументи абсолютно переконливі та зрозумілі – для тих, хто взагалі
готовий їх чути та розглядати. Однак скидається на те, що ми вже вичерпали
ресурс тих людей на Заході, хто в принципі готовий був нас почути.

Спершу ми вичерпали ресурс емпатії та співчуття. Цей ресурс взагалі більш
гуманітарний та дуже персоніфікований, і майже не впливає на масштаби та
характер державної військової допомоги. Емпатія – це про те, щоб прийняти наших
біженців. Це про західних волонтерів, які власним коштом возять нам фури
гуманітарки. Про добровольців, які кинули все та приїхали до нас воювати. Але
це навіть не про Джавеліни чи Стінгери, не кажучи вже про Абрамси та Атакмси.

Далі ми мобілізували всіх тих, хто сприймає нас не так емоційно, натомість
раціонально розуміє і російську загрозу для себе, і те, як весь світовий
порядок залежить від результату цієї війни. Балтійські країни та Польща,
Британія та Скандинавія віддали нам все, що могли надати, і навіть трохи
більше, Шольц став \enquote{яструбом} та витрусив з Бундесверу ледь не все, що там не
встигли згноїти та здати на брухт за роки бездіяльності, а Франція саме з цих
міркувань надала нам принаймні чимало зброї, і раптом стала нашим лобістом у
НАТО.

І от лишилась далека заокеанська країна, яка допомагає нам більше за всіх, але
водночас найбільше стриножує нас купою заборон та обмежує номенклатуру
поставок. Власне, саме позиція керівників цієї однієї далекої країни є
визначальною і стосовно нашого невступу до НАТО (так, схоже що за десять днів у
Вільнюсі нас чекає величезний \enquote{облом} – замість запрошення до Альянсу нам
запропонують фіговий листок чергових \enquote{безпекових гарантій}, мало чим відмінний
від Будапештського), і стосовно ненадання нам літаків, дальнобійних ракет тощо,
і стосовно ганебної заборони переносити війну на терени ворога.

Ми здолали безліч перешкод, які ще пару років тому видавалися нездоланними, але
зараз здається вперлись у непрохідну стіну в голові дідуся Байдена. І ця стіна
– не його персональний \enquote{забобон}, все на жаль набагато гірше.

Марш \enquote{вагнерів} по росії нарешті зробив очевидним для західного політикуму те,
що ми знаємо здавна – рашка не лише не всесильна, вона насправді є конструкцією
з лайна та паличок, і цілком може розпастись у дуже недалекій часовій
перспективі. І от саме ця перспектива змусила ту частину західного політикуму,
яка наразі повністю формує позицію Білого Дому, нарешті остаточно зняти маски
та показати свої справжні мотивації.

У першому коменті лінк на статтю у Нью Йорк Таймс, у другому коменті  лінк на
її український переклад. Автор – Томас Фрідман, тричі лауреат Пулітцерівської
премії, тобто світило журналістики, за поглядами один із \enquote{лідерів думок} в
таборі демократів. Читаючи його, можемо уявити що читаємо думки Саллівана,
Кірбі чи Байдена. Стаття коротенька, можна почитати всю, щоб загалом уявити, що
коїться у голівоньках нинішніх керманичів найпотужнішої світової держави. Але
головне для нас сформульоване у останньому абзаці, цитую дослівно:

\enquote{Якщо він (Путін) переможе, російський народ програє. Але якщо він програє, а
його наступником стане безлад, програє весь світ}.

Висновок очевидний – звісно ж, хай краще програє \enquote{російський народ} (про наш
народ загалом немає згадки), аніж \enquote{весь світ}. Тобто перемога пу та збереження
його режиму подається як менше зло, а альтернатива малюється винятково як хаос
із купою нових неконтрольованих режимів із шматками ядерного арсеналу в руках у
кожного з численних пригожиних та кадирових.

Саме цей страх визначає межі, до яких нас готові підтримувати у Білому Домі, і
здається ми вже в цю межу вперлись. Час визнати вкрай неприємний факт: головний
союзник банально боїться нашої перемоги, а відповідно – хоче її уникнути.

Саме це найкраще пояснює всі ті повороти у \enquote{недодопомозі} нам, які можуть
здатись нам дивними, нелогічними та непослідовними. Насправді це якраз дуже
послідовна політика, хоч водночас і аморальна, і недалекоглядна. І частиною
цієї ж політики є хор журналістів та гарвардських \enquote{ікспєрдів}, які \enquote{висловлюють
розчарування} темпами нашого наступу.

Суть цієї політики проста: не дати перемоги пу – так, дати нам перемогти рашку
– ніт, бо ж \enquote{програє весь світ}. І здається ніякі наші продумані раціональні
аргументи не здатні пробити цей страх, адже ірраціональні страхи належать до
найсильніших людських мотивацій.

Ми можемо скільки завгодно апелювати до цінностей, але це вже було, і не дало
результату. Можемо пояснювати, що насправді така позиція не мудра, що це не
\enquote{прагматичний розрахунок}, а всього лише небажання лінивих та ситих боягузів
покинути зону комфорту. І що із зони комфорту їм вийти все одно доведеться,
раніше чи пізніше, і що саме в прагматичних інтересах Штатів прискорити нашу
незаперечну, переконливу перемогу, та вже наперед готуватись брати під контроль
процеси при неминучих пертурбаціях на росії. Але ми все це вже говорили, та
вперлись у неготовність це чути.

Боюсь, що оскільки ми маємо справу із глибоко закоріненими страхами, здолати їх
може лише інший страх, ще більш потужний. І здається, нам час додавати до
арсеналу нашої аргументації \enquote{страшилки}, які перекриють в головах союзників
їхній жах перед \enquote{невідомим майбутнім росії}.

Певно, час малювати союзникам картинки можливого майбутнього, на фоні яких
умовна пригожинська рашка виглядатиме не найгіршим варіантом.

Варто пояснити нашому другові Борелю (друг без лапок, він реально робить все що
може), що він не зовсім правий, коли каже що без поставок західної зброї ми \enquote{за
кілька днів програємо війну, та перетворимось на нову Білорусь}. Так,
програємо, але не за дні, а за довгі місяці. За ці місяці загинуть сотні тисяч,
а ще мільйонів 15-20 поповнять табори біженців у Європі. А потому ми станемо не
\enquote{другою Білоруссю}, а \enquote{другим Афганістаном}, де прямо біля кордону ЄС роками та
навіть десятиліттями не стихатиме кривава партизанська війна, з усіма
«зручностями» для сусідів.

Варто передати нашому ворогові Орбану, що якщо його зусилля допомогти перемозі
росії раптом таки виявляться успішними, і ми відповідно до його прогнозу
програємо війну, то останнім наказом нашій мільйонній армії, загартованій у
боях із другою армією світу, буде відступити до нейтральних країн Європи з
усією зброєю та технікою, і там інтернуватись. І що маршрут цього відступу
пролягатиме не через ті країни, які нам допомагали, а виключно через Угорщину.
А ще по дорозі ця армія матиме завдання забезпечити гуманітарний коридор для
виїзду мільйонів цивільних біженців, так само через Угорщину. Зрештою, чи не
мадярам знати як це виглядає – тисячекілометровий збройний марш через чужі
\enquote{вимушено гостинні} землі у пошуках \enquote{обітуваної землі} 🙂 історія інколи робить
парадоксальні кульбіти, і їм варто про це замислитись 🙂

Варто пояснити всім цим гарвардсько-пулітцерівським \enquote{ікспєрдам}, що примушена
до \enquote{недоперемоги} та перемир'я Україна, частина якої залишиться під окупацією,
звісно ж не зможе стати частиною НАТО та ЄС, в неї ніхто не інвестуватиме тощо
– і що ми зробимо це не лише своєю трагедією, але і їхнім потужним головним
болем.

Ця недобита та покинута напризволяще країна попри злидні вкладатиме всі ресурси
у війну, зокрема кістми ляже щоб зробити собі ядерну зброю, і має дуже великі
шанси на успіх. Ну а якщо ЦРУ таки вдасться цьому запобігти, то вже що-що, а
так звану \enquote{брудну бомбу} у нас клепатимуть чи не в кожному підвалі, і з цим
точно ніхто ніц не поробить. Боротися зі злиднями за відсутності репарацій та
інвестицій доведеться так само будь-якими методами, зокрема стати одним із
світових хабів по торгівлі нелегальною зброєю, іншими товарами небажаними у
«культурних» країнах, пралкою для відмивання тіньових коштів тощо.

То може тоді Білому Дому простіше віддати нас на поталу росії, аніж мати
проблеми з такою скаліченою ПТСРною країною? Теж ні, бо країни може не бути,
але українців занадто багато, щоб їх всіх фізично знищили росіяни.

Мільйон ветеранів з унікальним бойовим досвідом, з відчуттям що їх зрадили, без
роботи та перспектив, роз'їдеться світом складати успішну конкуренцію
\enquote{вагнерам}, або ж сформує такі злочинні угруповання, що балканські та
італійські мафії нервово куритимуть у коридорі.

Діти загиблих бійців виростуть біженцями у сиротинцях Європи, і плекатимуть
ненависть не лише до окупантів, але і до тих хто нас зрадив у вирішальний час.
І оскільки дістати москву їм буде важкувато, злість каналізується у помсту тим,
до кого можна дотягнутися – західним політикам, які свого часу приклались до
нашої поразки. Якщо із очікуваних 20 мільйонів українських біженців терористами
– месниками стане лише одна сота доля процента, матимемо двотисячне підпілля із
добре інтегрованих в західних країнах людей – куди там тій Аль-Каїді...

Всі ці та подібні жахливі картинки мають стати перед очима фрідманів,
салліванів та байденів, та стати для них більш реальними нічними жахами, аніж
дуже абстрактний \enquote{безлад на росії}. І привести їх до розуміння того, що єдиний
спосіб уникнути всіх цих кошмарів – це переможна Україна, яка з допомогою
Заходу вщент розбила російські війська, звільнила всі території, вступила до
НАТО та ЄС – і у відповідь має надалі бути чемною, контрольованою, дотримувати
правил «клубу» та погоджувати свої дії з союзниками та інвесторами.

Звісно, далеко не всі такі меседжі може публічно озвучувати наша держава (хоч
успішний досвід посла Мельника доводить, що часто варто бути нечемним та навіть
брутальним, аби таки бути почутим). Але ж крім офіційних речників є ще преса,
експертні спільноти та безліч інших каналів комунікації та донесення меседжів.
Певно, час цим каналам перестати боятись роздратувати союзників, і навпаки –
почати доносити до них інфу про нашу версію \enquote{апокаліптичного безладу}.

Так, від цього нас не стануть більше любити. Але потенціал любові та співчуття
ми вже використали по максимуму, більшого з цих шляхетних почуттів хороших
людей ми не витиснемо. А для перемоги нам потрібно більше, і набагато більше.
Час достукатись до нехороших людей, до тих, хто і так нас не любить, і
достукатись через те єдине, що достатньо їх мотивує – апокаліптичні страхи. Не
вийшло домовитись по-доброму? Що ж поробиш, поговоримо по-поганому. Адже у нас
нема вибору – нам потрібна лише перемога.
