% vim: keymap=russian-jcukenwin
%%beginhead 
 
%%file 19_12_2021.fb.fb_group.story_kiev_ua.1.detstvo_shkola
%%parent 19_12_2021
 
%%url https://www.facebook.com/groups/story.kiev.ua/posts/1822208344642625
 
%%author_id fb_group.story_kiev_ua,fedjko_vladimir.kiev
%%date 
 
%%tags deti,detstvo,kiev,pamjat,shkola,sssr
%%title Київ: Дитинство і перші шкільні роки
 
%%endhead 
 
\subsection{Київ: Дитинство і перші шкільні роки}
\label{sec:19_12_2021.fb.fb_group.story_kiev_ua.1.detstvo_shkola}
 
\Purl{https://www.facebook.com/groups/story.kiev.ua/posts/1822208344642625}
\ifcmt
 author_begin
   author_id fb_group.story_kiev_ua,fedjko_vladimir.kiev
 author_end
\fi

Київ: Дитинство і перші шкільні роки 

Мій мозок і моя психіка дошкільного періоду були повністю захищені від
зовнішніх інформаційних і ідеологічних впливів радянського суспільства. Я був
домашньою дитиною, яка не знала, що таке «дитячий садок» і не відчула на собі
усі «прєлєсті» радянського дошкільного виховання, яке практикувалося там. Мені
не довелося зубрити безглузді віршики про Леніна на зразок:

\obeycr
«Я маленькая девочка,
Я в школу не хожу,
Я Ленина не видела,
Но я его люблю!»,
\restorecr

і брати участь у театралізованих постановках, присвячених радянським святам: 23
лютого, 8 березня, 1 травня і 7 листопада, що були обов’язковим атрибутом в
дитячих садках.

\begin{multicols}{2} % {
\setlength{\parindent}{0pt}

\ii{19_12_2021.fb.fb_group.story_kiev_ua.1.detstvo_shkola.pic.1}
\ii{19_12_2021.fb.fb_group.story_kiev_ua.1.detstvo_shkola.pic.1.cmt}

\ii{19_12_2021.fb.fb_group.story_kiev_ua.1.detstvo_shkola.pic.2}
\ii{19_12_2021.fb.fb_group.story_kiev_ua.1.detstvo_shkola.pic.2.cmt}

\ii{19_12_2021.fb.fb_group.story_kiev_ua.1.detstvo_shkola.pic.3}
\ii{19_12_2021.fb.fb_group.story_kiev_ua.1.detstvo_shkola.pic.3.cmt}

\ii{19_12_2021.fb.fb_group.story_kiev_ua.1.detstvo_shkola.pic.4}
\ii{19_12_2021.fb.fb_group.story_kiev_ua.1.detstvo_shkola.pic.5}
\ii{19_12_2021.fb.fb_group.story_kiev_ua.1.detstvo_shkola.pic.5.cmt}

\ii{19_12_2021.fb.fb_group.story_kiev_ua.1.detstvo_shkola.pic.6}
\ii{19_12_2021.fb.fb_group.story_kiev_ua.1.detstvo_shkola.pic.7}

\ii{19_12_2021.fb.fb_group.story_kiev_ua.1.detstvo_shkola.pic.8}
\ii{19_12_2021.fb.fb_group.story_kiev_ua.1.detstvo_shkola.pic.8.cmt}
\end{multicols} % }

1954-й... Перший раз у перший клас...

10 січня мені виповнилося 7 років! Останнє вільне літо і безтурботне дитинство
закінчилося – в серпні мене почали збирати в школу. Пам’ятаю, що першою
покупкою була шкільна форма. Кітель і штани коричневого кольору, дві сорочки –
біла і коричнева, чорні черевики, коричневий ремінь з абревіатурою «СШ»
(середня школа) на пряжці. В наступному поході в універмаг («Шкільний базар»)
купили чорний дермантиновий портфель, Буквар, зошити в «клітинку» і в «дві
косих», щоденники, альбоми для малювання, альбоми для запису нот (на уроках
співів), чорнильницю-невиливайку (яку носили в окремій торбинці), ручки, пера
(«піонерські» – коричневого кольору з зірочкою), кольорові олівці, гумку для
стирання, порошок для приготування фіолетових чорнил.

***

Ми проживали на вулиці Мало-Житомирській, у будинку № 5, а школа моя (СШ № 6)
була зовсім поруч – на Михайлівській площі, у приміщенні колишньої гімназії
(зараз там Дипломатична академія). Це була красива будівля. Просторі класні
кімнати з великими вікнами, високі стелі, довгі і широкі коридори, великий
актовий зал, у якому проходили різні урочистості. В класах свіжопофарбовані
парти темно-коричневого кольору; стільниця – чорного кольору. Дві дошки. Одна
чорного кольору, одна – коричневого, розлініяна в «дві косих» для уроків
каліграфії. На дошках писали білою крейдою. Стирали написане – вологою
ганчіркою. 

Характерною особливістю школи була обов’язкова наявність бюстів Леніна і
Сталіна у вестибюлі та живих квітів біля них, їх же портретів у кожному класі,
і їх же портретів у Букварі.

\ii{19_12_2021.fb.fb_group.story_kiev_ua.1.detstvo_shkola.pic.9}

Дорога до школи займала хвилин десять спокійної ходи і мені всього двічі
потрібно було переходити дорогу. І обидва рази біля будівлі міського управління
міліції, де водії проїжджали перехід дуже тихо. Тому моя рідня колективно
відвела мене в школу один раз – 1 вересня, а з наступного дня я вже ходив у
школу самостійно.

\begin{multicols}{3} % {
\ifcmt
  ig https://scontent-frt3-1.xx.fbcdn.net/v/t39.30808-6/269445490_4817995141593239_252951152227935947_n.jpg?_nc_cat=102&ccb=1-5&_nc_sid=b9115d&_nc_ohc=BB0OHUuAmikAX9XiXEX&_nc_ht=scontent-frt3-1.xx&oh=00_AT_B7XVOi1alOm2r0AZ7hl918EubCT3m7tPLUBuZaSi1mQ&oe=61C65788
  @width 0.2
\fi
\columnbreak
\iusr{Надежда Владимир Федько}

Французький художник-академіст Вільям Бугро написав картину, без якої сьогодні
важко уявити ТОП найромантичніших творів мистецтва. «Купідон і Психея в
дитинстві» - оригінальна назва живопису, названа колись помилково «Перший
поцілунок». У наші дні ця зворушлива картина широко відома саме під таким
найменуванням.

\columnbreak

\ifcmt
  ig https://scontent-frt3-1.xx.fbcdn.net/v/t39.30808-6/269672560_4817995264926560_5599881032538062677_n.jpg?_nc_cat=102&ccb=1-5&_nc_sid=b9115d&_nc_ohc=mY7vnSK5_5IAX8uwvES&_nc_ht=scontent-frt3-1.xx&oh=00_AT_K8VOcIr6jKRQkimp3IFGlemJi5Mq0RNXJqCZ-9zRrCQ&oe=61C7CD72
  @width 0.2
\fi

\columnbreak

\end{multicols} % }

Фотографій з того урочистого дня у мене, на жаль, не збереглося! 

Пам’ятаю, що десь за тиждень до школи мене підстригли. Тоді в моді були
чоловічі зачіски «бокс» і «полубокс» (рос. транскрипція), які в побуті
іменувалися «під фріца», а фактично були копією зачіски «гітлерюгенд».

І от я гордовито іду в школу! Я – в центрі, по бокам від мене – мама, її старша
сестра (моя тьотя), бабуся і моя старша двоюрідна сестра. В одній руці у мене
портфель і торбинка з чорнильницею, а в другій – великий букет квітів. 

Урочиста лінійка, перший дзвінок, нас розводять по класам і розсаджують попарно
за парти: хлопчик – дівчинка. Хлопчиків було більше, тому декому дівчинка не
дісталася. Мені повезло! Надійка П., симпатична дівчинка з розумними очима і
гарно заплетеними косами, стала моєю першою шкільною подружкою. 

***

Вчителька знайомиться з нами. Називає себе, потім починає викликати учнів по
прізвищам у журналі. Названий встає і голосно говорить «Я», розповідає про
своїх батьків. 

Коли пролунало моє прізвище, то я встав, назвав маму по імені та по батькові,
сказав ким вона працює, а щодо батька, то назвав його по імені та по батькові і
сказав, що він живе і працює на Півночі. Вчителька якось дивно подивилася на
мене і більше нічого не запитувала. Тепер я розумію, що очевидно вона подумала,
що мій батько репресований і відбуває покарання в таборах на Колимі. Це не було
рідкістю в ті часи. 

[В дитинстві щодо відсутності батька для мене було пояснення, що «він полярник
і працює на далекій Півночі». Щомісяця від батька приходили грошові перекази з
короткими текстами на звороті. Мама мені ці перекази показувала і читала
вітання. Як я потім дізнався від тьоті, то при розлученні батьки домовилися, що
вони не будуть оформляти сплату аліментів через суд; я залишаюся на прізвищі
батька – Федько, а мама повертається на своє дівоче прізвище – Костенко.] 

В цей же день формуються органи управління: складається список чергових по
класу, з дівчаток призначаються сандружинниці, завданням яких є щоденна
перевірка перед початком занять чистоти рук, шиї і вух, підстрижки. 

***

В моєму класі було 35 хлопчиків і дівчаток, з яких тільки нас троє вміли
читати, писати і рахувати до 100. Всі інші були повністю безграмотні. Це
виявилося за тиждень після початку занять. Оскільки нам було нудотно на уроках,
коли всі інші вчаться читати по складах і писати спочатку окремі елементи
літер… «нажим» – «волосинка»…, а потім цілі літери, то нашу трійцю посадили
разом у дальній куток і дозволили читати будь-які книжки, які ми приносили з
дому. Таким чином у нас утворився своєрідний читацький клуб. Ми читали свої
книжки, а на перервах і після уроків обмінювалися думками про прочитане. Наша
трійця складалася з двох хлопчиків і одної дівчинки, тому майже одразу у нас
виникло суперництво за увагу дівчинки!

З дитинства сімейне виховання прищепило мені повагу до дівчат! Рідні
підкреслювали, що з дівчатами треба бути ввічливим і ні в якому разі не
ображати їх, пропускати першими в дверях, поступатися місцем, брати на себе
більш важку роботу при прибиранні в класі. Важливим елементом статевого
виховання була настанова, що дівчатам можна говорити компліменти і дарувати
квіти, дівчат можна обіймати і цілувати, що категорично забороняється відносно
хлопців!

За перший тиждень я вже звик до своєї подружки по парті, вона мені подобалася
і, як мені стало зрозуміло, я теж їй сподобався. Але вона була неграмотна і нас
розлучили. Наша трійця сиділа на двох партах у дальньому кутку класу. Вчителька
була досвідченим педагогом. Тому вона розпорядилася, щоб наша партнерка по
грамотності один день сиділа зі мною, а другий – з грамотним хлопцем нашої
трійки. 

О, перші романтичні стосунки! Перші шкільні романи!

Моє серце розривалося між трьома дівчатами. Двома однокласницями і сусідською
дівчинкою, Іринкою М., яка вчилася в іншій школі.

***

В першому ж класі, чи то 22 вересня, чи то 22 жовтня (точно не пам’ятаю), був
створений «гурт жовтенят». Нам всім видали нагрудний значок – п’ятикутну
червону зірку з портретом малолітнього Леніна, яку потрібно було носити на
шкільній формі з лівого боку, і розбили на п’ятірки – символ п’ятикутної зірки.
Як правило в «зірочці» кожне жовтеня посідало одну з посад – командир,
квітникар, санітар, бібліотекар і фізкультурник. 

Нашу «грамотну трійцю» спеціально розкидали по трьох «зірочках», призначивши
кожного бібліотекарем. А моя перша подружка по парті потрапила в мою «зірочку»
і отримала посаду квітникаря. Ми обоє були цим дуже задоволені.  

Одразу після призначення мене бібліотекарем я закликав членів нашої «зірочки»
до створення «зіркової бібліотеки»! В основі лежало суто прагматичне бажання –
розширити коло літератури для власного прочитання за рахунок принесених
однокласниками книг в бібліотеку. Мої письменні товариші вловили підтекст моєї
ідеї і гаряче її підтримали. Класна керівниця побачила в моїй ініціативі
власний інтерес і теж підтримала. 

Через декілька днів завгосп разом з учителем фізкультури притягли в клас велику
шафу, яку вчителька відвела під класну бібліотеку. Мою ідею створення «зіркової
бібліотеки» вона вже піарила, висловлюючись сучасною мовою, як власну.

Слід нагадати, що в ті часи навчання у першому – четвертому класах провадила
одна вчителька, яка була і класною керівницею. За нею та учнями закріплювалося
класне приміщення, яке ми всі повинні були підтримувати у чистоті і порядку.
Класна оранжерея поповнювалася батьками, а призначені квітникарі повинні були
слідкувати за квітами... 

Клас вчився читати, бібліотека поступово наповнювалася книгами...

В часи мого дитинства і юності «зачитати книжку», тобто не повернути її
господарю, не вважалося гріхом. В крайньому разі замість «зачитаної»
поверталася інша. Також не вважалося гріхом не повернути книгу в бібліотеку або
просто поцупити книгу. Бібліотека вимагала замість втраченої принести
рівноцінну або сплатити вартість втраченої у п’ятикратному розмірі. 

Чи грішив і я? Так, грішив. Пару-трійку разів брав участь у крадіжках цікавих
книг у магазині. Хтось один з нашої банди (не плутати з жовтенятською
«зірочкою»), частіше за все це були дівчатка, відволікав увагу продавця, а ми
вже експропріювали книжку, що нам сподобалася. Вкрадена книжка була спільною
власністю і ходила по рукам, поки всі наші її не прочитають. Потім, по
домовленості, ця книжка залишалася у когось з наших або вимінювалася на іншу.

Піонерія

Потужна ідеологічна атака на наші дитячі мізки почалася восени 1956 року, коли
я вже навчався у третьому класі. Оскільки нам виповнилося 9 років, то настав
час приймати нас у піонери.

Знову почалися читання розповідей про Леніна і про піонерську організацію, яка
носила його ім’я. Але читала нам вже не вчителька, а піонери з IV– VI класів!
Ключовою темою читань  були піонери-герої! А культовою фігурою, піонером № 1 –
Павлик Морозов!

В квітні 1957 року, напередодні дня народження Леніна, нас усім класом (оптом)
прийняли в піонери.

Покарання і заохочення...

Вся радянська система побудована на системі покарань і заохочень. І школа не
була виключенням з цього правила! Процес навчання в радянській школі також
супроводжувався системою покарань і заохочень!

Покарання були індивідуальні і колективні. 

Ієрархія індивідуальних покарань була досить різноманітною:

\obeycr
- Запис у щоденник зауважень вчителя. Наприклад, «Крутився на уроці»; «Неуважно слухає пояснення вчителя»; «Переписується записочками з іншими учнями»; «Зробив голуба і запустив в класі під час уроку»; «Смикав дівчину за коси», «Спізнився з перерви на (**) хвилин» тощо... 
- Встань за партою на (5 – 10 – 15 ) хвилин.
- Встань за партою до кінця уроку. 
(Безумовно, що писати стоячи було незручно і, крім того, почерк різко погіршувався, що автоматично знижувало оцінку за написане). 
- Поклади щоденник на стіл і стань у куток.
- Поклади щоденник на стіл і стань у куток носом до стіни.
- Поклади щоденник на стіл і вийди за двері.
- Поклади щоденник на стіл, іди до завуча/директора і розкажи про свою погану поведінку.
- Виклик батьків до класної керівниці.
- Виклик на педраду з батьками.
- Виключення з школи на (1 або 3 дні, на тиждень). 
\restorecr

Всі накладені покарання записувалися у щоденник! (А коли заводився новий
щоденник, то у ньому в обов’язковому порядку нумерувалися всі сторінки, щоб не
можна було вирвати аркуш!) Батьки повинні були зробити відмітку, що вони
ознайомилися із заувагою вчителя. В кінці тижня виставлялася оцінка за
поведінку по п’ятибальній системі. Батьки повинні були розписатися у щоденнику
і в понеділок класний керівник перевіряла усі щоденники на наявність підпису
батьків.

В табелі успішності, який видавався після закінчення кожної чверті, також
виставлялася оцінка за поведінку. Для тих, кому виповнилося 12 років, наявність
виключень зі школи на декілька днів чи тиждень, та оцінка за поведінку «4» або
«3», могло потягнути за собою постановку на облік у дитячій кімнаті міліції або
навіть відправку у спеціальну школу-інтернат. До речі, ставили на облік і
відправляли у спецшколу також за вчинення крадіжок та злісне хуліганство. 

Колективне покарання – весь клас залишався після уроків на 1-2 години, під час
яких класний керівник читала нам нудну мораль! А ми, тишком-нишком, хто як
умів, за цей час примудрялися зробити частину уроків, наприклад, вирішити
завдання з математики.

***

Система заохочень була набагато менш масштабною:

- Оцінка за поведінку «5».

- «Похвальний лист» після закінчення навчального року.

***

Покарання в школі тягнули за собою відповідну реакцію батьків – покарання
вдома!

У нашій сім’ї фізичні покарання не практикувалися! За різні домашні і шкільні
провинності мене могли позбавити солодкого чи морозива, на крайній випадок
поставити на годинку носом у куток. 

А від шкільних товаришів мене часто доводилося чути про покарання ременем по
голим сідницям та стояння на колінах у кутку. Деяких ставили на коліна на
розсипану гречку чи горох. Це було і з хлопчиками, і з дівчатами.

***

Оскільки вже зайшла розмова про покарання, то розкажу одну із власних історій... 

«Джульєтта» з паралельного класу (Галинка Л.)...

Якось за якусь провину мене виставили за двері. В коридорі вже вешталися
чоловік п’ять таких же, як і я, вигнанців з інших класів. А біля вікна стояла
дівчинка з паралельного класу і плакала. Я підійшов до неї і запитав, чого вона
плаче, хто її образив. Крізь сльози вона розповіла, що вчителька вигнала її з
класу за те, що вона не виказала подругу, яка надіслала їй записочку. Тепер
вчителька зробить запис про покарання у щоденник, а тато за найменшу провину
шмагає її ременем, а потім ще ставить на коліна у куток.

Мені стало жаль дівчинку і я, не знаючи, як її втішити, поцілував її в мокрі і
солоні від сліз губи. Очевидно дівчинці це сподобалося... Вона одразу перестала
плакати, витерла обличчя піонерським галстуком і сказала: «Давай поцілуємося ще
раз!» 

Мене не потрібно було вмовляти! Ми поцілувалися ще раз, причому вона була
досить емоційна!

В цей час до нас підбіг один з вигнанців і почав голосно репетувати:

\obeycr
«Тили-тили тесто,
Жених и невеста!
Тесто засохло,
А невеста сдохла!» 
\restorecr

Подібного нахабства я, звісно, стерпіти не міг. Спіймав пацана за піджак,
вліпив йому декілька потиличників і закінчив екзекуцію потужним пєндєлем під
зад!

Він, рюмсаючи, побіг... Я отримав ще один поцілунок в нагороду! І тут пролунав
грізних голос моєї класної керівниці: «Це що таке!!!!!!!!!!!??? А!?». 

Виявляється, що пацан побіг в учительську і розповів про все моїй класній
керівниці... та прийшла розібратися... і застукала нас «на гарячому».

Ми отримали записи в щоденниках, що повинні з’явитися у п’ятницю, після уроків,
з батьками на педраду, де будуть розбиратися з нашою поведінкою.

Дожили до п’ятниці і ось ми на педагогічній раді! Директор, декілька вчителів,
обидві класні керівниці, Галинка з батьком, я з бабусею, мати побитого мною
хлопчика.

Моя класна дама з пафосом і обуренням розповідає про наші гріхи... Вірніше, про
мої гріхи... Погано вів себе на уроці, був виставлений за двері, цілувався з
дівчинкою, побив хлопця з молодшого класу... Пара вчителів доповнили звинувачення
розповідями про мої минулі вихватки. Правда всі зауважували, що я один з
найуспішніших учнів – ніколи не мав «трійок», і ніколи не прогулював уроків. 

Нарешті дали слово мені... Оскільки мені нічого не загрожувало (як я вже говорив,
в нашій сім’ї фізичні покарання не практикувалися), то я взяв всю вину на себе.
Сказав, що поцілував Галинку силою, що вона опиралася...

Хтось з учителів насмішкувато промовив «Теж мені Ромео і Джульєтта… Підрости ще
треба!»

І тут зовсім несподівано Галинка сказала, що Джульєтті теж було дванадцять
років! На пару хвилин наступила тиша…

Обуренню вчителів не було меж!.. Вони одностайно засудили наш романтизм і
постановили поставити нам обом «трійку» по поведінці за тиждень!

Педрада закінчилася, ми вийшли в коридор. Несподівано батько Галинки сказав, що
йому сподобалася моя поведінка на педраді – я поводився, як справжній чоловік!
Тому він дозволяє мені приходити до них додому і готувати уроки разом з
Галинкою… Але… Тут він грізно зиркнув і голосом, в якому дзвенів метал, сказав,
що якщо він спіймає нас за поцілунками, то відшмагає ременем по голим сідницям
обох до посиніння сідниць! І на підтвердження своїх слів потряс перед нами
кінчиком ременя своїх штанів!

Чесно кажучи, нас це розсмішило, оскільки уроки звичайно готувалися вдень, а
він приходив з роботи десь о восьмій годині вечора. Крім того, сам вираз «якщо
він спіймає нас за поцілунками» не означав категоричної заборони цілуватися.

***

Якщо була гарна погода, то після уроків ми йшли гуляти на Володимирську гірку і
гуляли там години до шостої вечора. Потім ішли до Галинки готувати уроки. А
якщо погода була погана, то ми одразу після школи йшли до неї. Уроки – уроками,
а поцілунки – обов’язково! Хочу зізнатися, що на той час обіймашки і поцілунки
нас вже хвилювали!  

Якось я запитав Галинку, чому вона на педраді сказала, що Джульєтті було
дванадцять років? Відповідь виявилася прозаїчною. Під час літніх канікул
Галинка була у піонерському таборі. І її навчив цілуватися піонервожатий, якому
було років п’ятнадцять. Він розповідав їй романтичну історію про Ромео і
Джульєтту, якій було 12 років; говорив, що вона дуже схожа на Джульєтту і…
навчив цілуватися. Галинці історія сподобалася, а ще більше сподобалося
цілуватися. Вона була дуже чуттєвою дівчиною.

О часи, о звичаї, друг Гораціо!

***

P.S. Тематичні ілюстрації підібрані з Інтернету.

***

\ii{19_12_2021.fb.fb_group.story_kiev_ua.1.detstvo_shkola.cmt}
