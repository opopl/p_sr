% vim: keymap=russian-jcukenwin
%%beginhead 
 
%%file 08_01_2022.fb.zharkih_denis.1.detskaja_literatura_savenkova
%%parent 08_01_2022
 
%%url https://www.facebook.com/permalink.php?story_fbid=3161926050687457&id=100006102787780
 
%%author_id zharkih_denis
%%date 
 
%%tags chelovek,deti,donbass,kontorovich_aleksandr,literatura,savenkova_faina
%%title Что такое детская серьезная литература?
 
%%endhead 
 
\subsection{Что такое детская серьезная литература?}
\label{sec:08_01_2022.fb.zharkih_denis.1.detskaja_literatura_savenkova}
 
\Purl{https://www.facebook.com/permalink.php?story_fbid=3161926050687457&id=100006102787780}
\ifcmt
 author_begin
   author_id zharkih_denis
 author_end
\fi

На фоне скандалов с писателем Фаиной Савенковой, которая объявлена
спецоперацией Кремля, вышла ее книга в соавторстве с фантастом Александром
Конторовичем. Так вот, эта книга реальный шаг вернуть серьезную детскую
литературу, которую мы практически потеряли со времен СССР. 

\ii{08_01_2022.fb.zharkih_denis.1.detskaja_literatura_savenkova.pic.1}

Что такое детская серьезная литература? Это когда с детьми не сюсюкают, не
впаривают им бездумный экшен или ужастики, а ставят перед ними весьма взрослые
вопросы. Вот авторы эти вопросы и ставят. Главные вопросы книги - Власть или
любовь? Долг или сострадание? Система или порыв? И ответы авторы не суют в рот
юношеству, а по взрослому разбирают, давая все нюансы власти, верности,
справедливости и чести. 

Много ли сегодня такой литературы? И как-то язык не поворачивается назвать ее
детской, поскольку в книге содержится сатира на современную бюрократию и
судопроизводство. Какую именно - украинскую, российскую, европейскую или
американскую, пусть читатели сами решат. Пусть посмотрят на людей-механизмов,
которые имеют всего одну функцию, а, если и присмотреться, то и не одной. Пусть
посидят на жестком троне правителя, который раб чего-то бездушного, которое и
не существо вовсе, и не совсем механизм. Пусть почувствуют, что иногда в этих
механизмах просыпаются человеческие чувства и они идут на благородные поступки.
Пусть поймут, что власть часто бездушна, а потому может убить живое, и то, для
чего она, в принципе, предназначена. Потому, что если книга написана с умом и
любовью, ее стоит почитать.

\begin{itemize} % {
\iusr{Вадим Павский}
\enquote{Мир которого нет} это точно детское название?))))

\iusr{Oleg Marmazinsky}
Денис, а серьезные телеканалы? Я про детские даже не заикаюсь...

\iusr{Наталья Асуховская}
Согласна с вами!!!!!

\iusr{Andrey Kichatov}

\obeycr
\enquote{ПРАВДА} - 23.09.2018
\enquote{ЛЯЧНО}
СКАЗАТЬ ПО-УКРАИНСКИ, ЛЯЧНО
СТАНОВИТСЯ МНЕ ИНОГДА-
НА НОЛЬ ПОМНОЖИЛИ УДАЧНО
ТЕ, НАШИ ЛУЧШИЕ, ГОДА.
КОГДА УЧИЛИСЬ, ПЕСНИ ПЕЛИ,
ТВОРИЛИ, СТРОИЛИ И В ВЫСЬ
ЦВЕТНЫЕ ШАРИКИ ЛЕТЕЛИ
С НАДЕЖДОЮ И ВЕРОЙ В ЖИЗНЬ.
\restorecr

\iusr{Sergei Litvinov}
«Те, кто читает книги, всегда буду управлять теми, кто смотрит телевизор».

\iusr{Петр Терескович}
Судя по Вашему описанию своеобразный детский вариант идеи матрицы.
\end{itemize} % }
