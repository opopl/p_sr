% vim: keymap=russian-jcukenwin
%%beginhead 
 
%%file 24_10_2021.fb.dolinskij_eduard.1.miroslav_simchich
%%parent 24_10_2021
 
%%url https://www.facebook.com/eduard.dolinsky/posts/4826757800689583
 
%%author_id dolinskij_eduard
%%date 
 
%%tags 1944,istoria,oun,prestuplenie.vojennoje,simchich_miroslav.upa.ubijca,ukraina,upa
%%title Симчич был дважды осужден за убийства мирных людей и военные преступления
 
%%endhead 
 
\subsection{Симчич был дважды осужден за убийства мирных людей и военные преступления}
\label{sec:24_10_2021.fb.dolinskij_eduard.1.miroslav_simchich}
 
\Purl{https://www.facebook.com/eduard.dolinsky/posts/4826757800689583}
\ifcmt
 author_begin
   author_id dolinskij_eduard
 author_end
\fi

Верховная Рада Украины 238 голосами приняла обращение к Президенту о присвоении
звания Героя Украины военному преступнику, организатору и исполнителю массового
убийства мирного польского населения 98-летнему Мирославу Симчичу. 

В материалах уголовного дела указано, что глава сотни УПА Симчич лично отдал
приказ на уничтожение польского населения села Пистень в Ивано-Франковской
области, включая женщин, детей, стариков и их домов. 

\ifcmt
  ig https://scontent-frx5-1.xx.fbcdn.net/v/t1.6435-9/248415888_4826757467356283_5163580678925770665_n.jpg?_nc_cat=105&ccb=1-5&_nc_sid=730e14&_nc_ohc=feHLBJOSB0gAX_nuQlS&_nc_ht=scontent-frx5-1.xx&oh=edc99c54149f50863daef7407ef05f28&oe=619D096D
  @width 0.4
  %@wrap \parpic[r]
  @wrap \InsertBoxR{0}
\fi

23 октября 1944 года отряд Симчича напал на село Троице. При этом было убито
более 80 человек - 66 поляков, 14 украинцев и один русский. Среди убитых были
женщины и дети. 

Убийцы сожгли 49 польских домов и забрали все вещи принадлежавшие убитым -
одежду, продукты питания. 

Симчич был дважды осужден за убийства мирных людей и военные преступления. 

Это сочли достаточным основанием, чтобы наградить его орденами "За заслуги"
(Ющенко)и «Свободы» (Порошенко), присвоить звание почетного гражданина Львова и
Коломыи. 

Симчичу при жизни установили памятник, а в 2017 году суд города Косова
реабилитировал военного преступника Мирослава Симчича.

На фото: Мирослав Симчич на фоне памятника себе.
