% vim: keymap=russian-jcukenwin
%%beginhead 
 
%%file 21_11_2021.fb.bilych_andrij.tjachiv.ukraina.1.den_gidnosti_i_svobody.pic.14.cmt
%%parent 21_11_2021.fb.bilych_andrij.tjachiv.ukraina.1.den_gidnosti_i_svobody
 
%%url 
 
%%author_id 
%%date 
 
%%tags 
%%title 
 
%%endhead 

\iusr{Олександр Михайлик}

Так, Михайлівський монастир тоді став прихистком і захистом. Якби не його
служителі, які прмхистили людей (і били на сполох у дзвони), невідомо, що було
б...


\iusr{Orusyna Gerasumchyk}
Це церква справжня

\iusr{Володимир Іванчо}
ФІЛАРЕТ СИЛА ЯК БИ ДОНЬОГО СТАВИЛИСЯ. А ЛЮДЕЙ ПРИХИСТИТИ ТРЕБА БУЛО МАТИ МУЖНІСТЬ.

\iusr{Skidenets Olga}
Це коли стало зрозуміло, яка церква від Бога.

\iusr{Юрий Кравчук}
Ось чому нам потрібна Наша Українська Церква! А не московський талібан!

\iusr{Uli Ana}
Я тут не була... але це назавжди залишиться в моїй голові

\iusr{Надія Халемендик}
Слава Богу за все

\begin{itemize} % {
\iusr{Татьяна Адамова}
\textbf{Надія Халемендик} за що? З а смерті ?

\iusr{Лилия Алёхина}
\textbf{Татьяна Адамова} написано ж - ЗА ВСЕ. Якщо Вам нема за що дякувати Богові - то біда...
\end{itemize} % }

\iusr{Галина Гуцол}
Справжня церква захищає своїх прихожан і дає їм притулок

\iusr{Наталия Долевская}
Так, Михайлівський братії вдячність і уклін.

\iusr{Віктор Погорілий}
Так, Софія фсбешна.

\begin{itemize} % {
\iusr{Olena Derevska}
\textbf{Віктор Погорілий} В Софії музей.

\iusr{Anastasia Nekrasova}
\textbf{Віктор Погорілий} 

лавра нажаль теж. Селила тодi дiвчину(я риелтор) яка студенкою худграфу жила в
общазi на террiторii лаври, так ось, вона бачила на власнi очi вночi людей в
чорному комуфляжу з гвинтiвками, яких заводили служки до звiницi в лаврi.

\iusr{Irina Kukhtyk}

Згадую той час. Дзвін дзвонів ніколи не забуду. Це було так страшно,
неймовірно, щиро і правдиво і пробуджувало надію, сподівання, віру.

\iusr{Антон Ходоренко}
О. Лелик

\iusr{Natali Demina}

Поруч в молебельному домі розташовувався хірургічний центр, де стояло п'ять
столів, куди приїжджали кияни і забирали постраждалих до дому,наша маленька
родина привозила каву лікарям. Коли почались розстріли, загиблих клали біля
альтанки. Приїзжали з від усюд лікарі, надавали допомогу, були втомлені але усі
тримались разом і чим могли, тим і допомогали. Багато спогадів, дома сидіти не
могли. СЛАВА УКРАЇНІ! Слава Вам люди за те що не простили і повстали, це перша
ознака того, що ми вільні. Щира Вам Вдячність кожному

\iusr{Євген Шашко}

Михайловский Златоверхий в ночь с 10 на 11 декабря бил в набат, когда
\enquote{Беркут} готовился к штурму.  Последний раз в набат монастырь бил в 13 веке, во
время монгольской осады.

\end{itemize} % }
