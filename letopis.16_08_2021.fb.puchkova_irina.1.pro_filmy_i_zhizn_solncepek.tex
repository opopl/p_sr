% vim: keymap=russian-jcukenwin
%%beginhead 
 
%%file 16_08_2021.fb.puchkova_irina.1.pro_filmy_i_zhizn_solncepek
%%parent 16_08_2021
 
%%url https://www.facebook.com/permalink.php?story_fbid=2952035388344503&id=100006142128029
 
%%author 
%%author_id puchkova_irina
%%author_url 
 
%%tags donbass,film.solncepek.donbass,moskva,rossia,vojna
%%title Про фильмы и жизнь
 
%%endhead 
 
\subsection{Про фильмы и жизнь}
\label{sec:16_08_2021.fb.puchkova_irina.1.pro_filmy_i_zhizn_solncepek}
 
\Purl{https://www.facebook.com/permalink.php?story_fbid=2952035388344503&id=100006142128029}
\ifcmt
 author_begin
   author_id puchkova_irina
 author_end
\fi

Про фильмы и жизнь.

Из разговора со знакомыми москвичами. Сегодня. Обедая в шумной сетевой
кафешке, обсуждаем новые киноленты. 

- Всё эти фильмы, типа "Солнцепека" - это художественное преувеличение. Ну как
"Рембо" или типа того... Таких ужасов, конечно, не было...

Спокойно отвечаю:

- Такие ужасы были!

С ироничной усмешкой:

\obeycr
- А вот ты же, признайся, этого не видела.
- Я это видела.
- Что ты видела, Господи Боже мой, вот такой кошмар? Ты весёлая, жизнерадостная, позитивная...
\restorecr

Я перебиваю:

- То есть вы считаете, что если люди видели смерть, то они больше не могут
жить и быть жизнерадостными? Я была в городе всю первую страшную часть войны.
Я с подругой детей её вывозила, когда мы ехали, а навстречу БТР, а мы не можем
понять  кто в нем. И что они с нами сделают, понимаешь? Когда по городу
прилетает, а линии ещё толком нет, едешь и хрен поймёшь, кто навстречу: наши
или нацбаты! Всё было вперемешку. И убьют тебя сейчас или пропустят мимо. И
подруга орёт "Что делать?!?", а ты понимаешь, что, блин, ничего уже. Просто
молиться, чтоб наши... 

- Ты не рассказывала. Подожди, но этого же не может быть чтоб прям - убьют!?! 

- Блин, я рассказывала!! Вы не хотели слышать. И может быть, что прям убьют. И
убивали. И наших знакомых так. И машины разбитые стояли на дорогах, едешь,
вдруг машина раскуроченная и ограбленная. А людей нет... 

- Нет, война это страшно, но это же все таки не та война, все таки не совсем война... 

- Как не та? А та - это какая? На "нетой" что ли снаряды не взрываются? На
нашей - ещё как. Моя свекровь в эпицентр боя попадала! Ползла оттуда, чтоб не
задело, автоматные очереди над головой. А прилетать - так вон к моим
горловчанам и не переставало! Месяца не было чтоб не упало в посёлке!! 

- Так а как там живут?!?! - глаза круглые у людей. 

- А вот так и живут, понимаешь! Если человек не говорит вам нон-стоп о беде
своей, то только потому, что знает, что вы все равно не поймёте! Да, каждый у
нас знал и знает, что такое война. Да, мы носились с рассказами об этом первые
годы. Но смысл? Вы ж знать не хотите! 

- Почему не хотим... Но просто... трудно поверить... 

  Вот что ответить??? 

Да,  трудно!!! А вы попробуйте. И если дончане не говорят только о своей
трагедии, то не значит, что они не помнят, не видели, что это неправда. 

Правда. Эта боль навсегда. 

Мы научились просто не навязывать нашу трагедию вам, тем кому она не нужна.
Кто не особо хочет верить в то, что война - всегда одинаково ужасна. Не бывает
"не такой". Всегда смерть, боль, ужас и безнадега. 

Мы просто знаем цену жизнерадостности. И понимаем, что она может окончиться в
любой момент. 

От того каждый момент и ценим теперь...

\ii{16_08_2021.fb.puchkova_irina.1.pro_filmy_i_zhizn_solncepek.cmt}
