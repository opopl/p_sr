% vim: keymap=russian-jcukenwin
%%beginhead 
 
%%file 14_02_2022.fb.fb_group.story_kiev_ua.2.kievskij_hlebushek
%%parent 14_02_2022
 
%%url https://www.facebook.com/groups/story.kiev.ua/posts/1861633027366823
 
%%author_id fb_group.story_kiev_ua,dubinina_oksana
%%date 
 
%%tags hleb,kiev
%%title Киевский хлебушек и хрустящие корочки
 
%%endhead 
 
\subsection{Киевский хлебушек и хрустящие корочки}
\label{sec:14_02_2022.fb.fb_group.story_kiev_ua.2.kievskij_hlebushek}
 
\Purl{https://www.facebook.com/groups/story.kiev.ua/posts/1861633027366823}
\ifcmt
 author_begin
   author_id fb_group.story_kiev_ua,dubinina_oksana
 author_end
\fi

Киевский хлебушек и хрустящие корочки

\ii{14_02_2022.fb.fb_group.story_kiev_ua.2.kievskij_hlebushek.pic.1}

Помните, как в детстве мы шли по маминой просьбе в ближайший хлебный магазин за
свежей выпечкой?

До сих пор в памяти этот манящий аромат! Почти всегда хлеб приносился домой с
обгрызенной корочкой. Невозможно ж было удержаться и не съесть хрустящую и
особо вкусную корку еще тёплого хлеба! У белого батона съедались «жопки», у
украинского отламывался узкий край по кругу.  

А улыбающаяся киевская паляница - мы говорили «паляницЯ»- прям зазывала
отломить и схрумкать, словно лопнувший от удовольствия, её большой полукруглый
козырёк корки.

Наш хлеб не плесневел странными цветами радуги((, а лишь подсыхал.

В последнее время традиционная украинская хлебопекарная промышленность
считается убыточной, а ей на «пятки наступает» выпечка из быстрых дрожжей и
разных ускоренных технологий… 

Хочется пожелать нашим пекарям не унывать и радовать нас вкусным настоящим
хлебом!

P.S. Вдохновило на приятные воспоминани фото из семейного архива Светланы
Петровой.

\ii{14_02_2022.fb.fb_group.story_kiev_ua.2.kievskij_hlebushek.cmt}
