%%beginhead 
 
%%file 13_02_2023.fb.fb_group.mariupol.pre_war.7.lyutii_2018__daleka_
%%parent 13_02_2023
 
%%url https://www.facebook.com/groups/1233789547361300/posts/1409173469822906
 
%%author_id fb_group.mariupol.pre_war,sokyrko_jevgen.mariupol
%%date 13_02_2023
 
%%tags mariupol,mariupol.pre_war,2018
%%title Лютий 2018... Далека вже історія..
 
%%endhead 

\subsection{Лютий 2018... Далека вже історія..}
\label{sec:13_02_2023.fb.fb_group.mariupol.pre_war.7.lyutii_2018__daleka_}
 
\Purl{https://www.facebook.com/groups/1233789547361300/posts/1409173469822906}
\ifcmt
 author_begin
   author_id fb_group.mariupol.pre_war,sokyrko_jevgen.mariupol
 author_end
\fi

Лютий 2018...

Далека вже історія... Колись в дитинстві фотографувались у сквері на фоні
фонтана та на фоні не зовсім зрозумілого пам'ятника. 

Потім настала докорінна перебудова сквера: повиймали таки залишки
німецько-румунського цвинтара, з під пам'ятника забрали вбитих занадто
\enquote{революційних активістів}. Вирубали плакучі шовковиці, традиційний фонтан,
театральні ліхтарі. Простір для всіх перетворився у простір \enquote{сковорідки} зі
схованим фонтаном, який скоріше для молоді. Хоча стало зручно проводити
концерти. Їх було багато, гарні і \enquote{смачні}...

Тепер все перевернуто дикою агресивною насильницькою силою, колективним
вбивцею...

І наш улюблений театр разом з нашими маріупольцями, і величезні вже ялини
навкруги нього, і фонтан, і дитячі площадки...   Та ще не кінець процесу...
Все буде відновлено...

Буде ще Новий Маріполь! І наші прапори, як у кращі часи...
