% vim: keymap=russian-jcukenwin
%%beginhead 
 
%%file 06_07_2021.fb.kachan_evgenij.1.gos_gimn_ukraina
%%parent 06_07_2021
 
%%url https://www.facebook.com/permalink.php?story_fbid=5733117696730202&id=100000960862758
 
%%author_id kachan_evgenij
%%date 
 
%%tags gimn,gosudarstvo,ideologia,obschestvo,strana,ukraina
%%title Задумывались ли вы когда-нибудь, что такое государственный гимн?
 
%%endhead 
 
\subsection{Задумывались ли вы когда-нибудь, что такое государственный гимн?}
\label{sec:06_07_2021.fb.kachan_evgenij.1.gos_gimn_ukraina}
 
\Purl{https://www.facebook.com/permalink.php?story_fbid=5733117696730202&id=100000960862758}
\ifcmt
 author_begin
   author_id kachan_evgenij
 author_end
\fi

Задумывались ли вы когда-нибудь, что такое государственный гимн? На мой взгляд,
гимн – это некий месседж, программная декларация государства и его народа –
другим государствам, Миру. Это заявление Миру о нас, и о нашем в нем месте.

Причем это не только представление нас Миру, но и некий образ, некое
самоощущение, формирующееся у народа, представление людей о себе самих.

Взглянув с этой точки на гимн Украины, я, признаться, пришел в ужас… Начнем по порядку…

«Ще не вмерла України і слава, і воля.» Что это? Что мы заявляем Миру такими
словами? Это наша жизненная программа? Невозможно существовать в состоянии «Ще
не вмерла»… По крайней мере долго… 

«Ще нам, браття українці, усміхнеться доля.» - констатация факта
обездоленности? Признание того, что сейчас все хуже некуда, но вот когда-нибудь
все обязательно будет хорошо?

«Згинуть наші вороженьки, як роса на сонці,

«Запануєм і ми, браття, у своїй сторонці.» - признание того факта, что мы не
хозяева в собственном доме? Да это скорее гимн «правительства в изгнании» или
беженцев, тоскующих на чужбине за родным домом…

«Душу й тіло ми положим за нашу свободу,

І покажем, що ми, браття, козацького роду.» - Свобода – наша главная ценность и
мы готовы за нее сражаться. Пожалуй единственный позитивный месседж в главной
части гимна (целиком гимн практически никогда не исполняется).

Какой же образ и мироощущение создает лично во мне наш гимн? Образ израненного,
обессиленного козака, отстреливающегося последними патронами от толп врагов,
которые вот-вот завалят его своими телами… Он-то может и спасется чудом, но
родная хата сожжена, враги заполонили вишневый сад и шансов отвоевать это все
назад практически нет. Тоска. Героическая. И чувство, что нас предали. Все
предали.

Мир, услышав такую историю, должен, пожалуй, тяжело вздохнуть и пустить скупую
слезу – «были ж вот ведь герои, рыцари»… БЫЛИ. Когда-то.

Не кажется ли вам, что наша с вами задекларированная жизненная программа очень
успешно воплощается в реальности? А может стоит ее пересмотреть – нашу
жизненную программу и, собственно, самоощущение? А то, знаете ли «Ще не вмерла»
самоощущается так себе… 

Государство «Украина» критически нуждается в перезапуске. Даже я бы сказал в
переСоздании, переОсновании. И первое, с чего бы я начал – с гимна. С такой
программой жить нельзя. Существовать – можно. Недолго. А хочется, чтобы Украина
жила и процветала. 

ИМХО, разумеется.

P.S. Да, у меня тоже ком в горле и подкатывают слезы, когда я слышу наш Гимн…
Но сказанного выше это никак не меняет.

\ii{06_07_2021.fb.kachan_evgenij.1.gos_gimn_ukraina.cmt}
