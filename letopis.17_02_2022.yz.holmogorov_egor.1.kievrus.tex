% vim: keymap=russian-jcukenwin
%%beginhead 
 
%%file 17_02_2022.yz.holmogorov_egor.1.kievrus
%%parent 17_02_2022
 
%%url https://zen.yandex.ru/media/holmogorow/suscestvovala-li-kievskaia-rus-620e007dff2d5a753ed7a43e
 
%%author_id holmogorov_egor
%%date 
 
%%tags istoria,kievrus
%%title Существовала ли "Киевская Русь"?
 
%%endhead 
 
\subsection{Существовала ли \enquote{Киевская Русь}?}
\label{sec:17_02_2022.yz.holmogorov_egor.1.kievrus}
 
\Purl{https://zen.yandex.ru/media/holmogorow/suscestvovala-li-kievskaia-rus-620e007dff2d5a753ed7a43e}
\ifcmt
 author_begin
   author_id holmogorov_egor
 author_end
\fi

В конституции Украины нет ни одного слова с корнем \enquote{рус}. И, тем не
менее, официальный Киев то и дело пытается доказать и нам и самому себе две
взаимоисключающие вещи: с одной стороны, что украинцы это отдельный от русских
народ, а с другой, что именно Украина и есть подлинная наследница
\enquote{Киевской Руси}.

И здесь как нигде наглядно видно, что современная Украина – это целиком и
полностью конструкт советского времени. Это от Ленина она получила свою
территорию. Это от Сталина она получила насильственную украинизацию
образования. Это от назначенного большевиками главой советской исторической
науки Бориса Дмитриевича Грекова она получила концепт «Киевской Руси» как
некоего якобы существовавшего в средние века государства.

\ii{17_02_2022.yz.holmogorov_egor.1.kievrus.pic.1}

Разумеется, нигде, ни в одной древней летописи нет ничего даже отдаленно
напоминающего понятие «Киевская Русь». В летописях есть только понятие
«Киевская область» или «волость», — небольшой регион. Именно в таком значении
его и употребил в XIX веке историк Михаил Максимович – «Киевская Русь» - как
понятие однопорядковое с «Суздальская Русь», «Червоная Русь» и так далее. 

\ii{17_02_2022.yz.holmogorov_egor.1.kievrus.pic.2}

И в таком же значении этот термин употреблял знаменитый историк Сергей
Михайлович Соловьев: \enquote{Несмотря на то что Юго-Западная Русь, преимущественно
Киевская область, была главною сценою древней нашей истории, пограничность ее,
близость к полю или степи, жилищу диких народов, делала ее неспособною стать
государственным зерном для России, для чего именно природа приготовила
Московскую область; отсюда Киевская область (Русь в самом тесном смысле)
вначале и после носит характер пограничного военного поселения, остается
страною козаков до полного государственного развития, начавшегося в Северной
Руси, в стране источников}.

В работах выдающегося русского историка, в известном смысле отца современной
русской национальной историографии, Василия Осиповича Ключевского понятие
«Киевская Русь» приобрело свойство хронотопа. Поскольку Ключевский придавал
большое значение процессам колонизации, перемещения населения, то зафиксировав
сдвиг в XII веке значительной части южнорусского населения на северо-восток, с
Днепра на Волгу, он начал в своем «Курсе русской истории» отличать «старую
Киевскую Русь» от «новой верхневолжской Руси». «Киевская Русь» у Ключевского
это не столько место, и уж тем более не государство. Это период истории,
заканчивающийся в XII веке.

\ii{17_02_2022.yz.holmogorov_egor.1.kievrus.pic.3}

Создатель украинского национализма Михайло Грушевский термином «Киевская Русь»
не пользовался. У него существовала украинская «Руська держава», история
которой продолжалась в Галицко-Волынской Руси и в Речи Посполитой, и логично
приводила к нему самому в австрийский Львов. Эта история и была для него
\enquote{историей Украины}.

«Киевская Русь» как некая существовавшая в средние века держава
конструировалась советской историографией, причем практически специально под
потребности созданной большевиками УССР, столица которой перенесена была в 1934
г. из Харькова в Киев. Впервые в том смысле, в котором употребляли этот термин
советские учебники и современные украинские пропагандисты, «Киевская Русь» была
введена в работах представителя «школы Ключевского» Бориса Дмитриевича Грекова,
идейного врангелевца, пошедшего после занятия красными Крыма на службу
большевикам.

\ii{17_02_2022.yz.holmogorov_egor.1.kievrus.pic.4}

Перед Грековым стояла непростая задача – приладить русскую историю под
советские потребности, ниспровергнув русофобские идеи главного
марксистско-ленинского историка Михаила Покровского. «Говорить о едином
\enquote{русском государстве} в Киевскую эпоху можно только по явному недоразумению» –
утверждал Покровский, целью которого было в духе раннего большевизма всемерно
подорвать древность и престиж русского государства, дерусифицировать русскую
историю.

И вот когда коммунистам стало понятно, что без опоры на тысячелетнюю Россию они
не выживут, Покровский был посмертно проклят и объявлен уклонистом. Сталин и
Жданов отказались от русофобских идей «школы Покровского» и потребовал
совместить концепцию национальной русской истории и большевистскую идеологию.

Термин \enquote{Киевская Русь} был политически канонизирован в 1936 году в связи со
скандалом вокруг постановки в Московском камерном театре оперетки \enquote{Богатыри} на
либретто \enquote{пролетарского поэта} Демьяна Бедного. В духе присущего ему русоедства
и воинствующего безбожия Ефим Придворов глумился над богатырями, князем
Владимиром, Крещением Руси. Князь «винища греческого вылакал, спьяну смуту и
народе сделал... старая вера пьяная была, а новая и того пуще».

Глава Совнаркома Вячеслав Молотов возмущенно ушел после первого акта со
словами: \enquote{Безобразие! Богатыри ведь были замечательные люди!}. Вскоре Политбюро
ЦК ВКП(б) приняло грозное постановление:

\begin{zzquote}
"14 ноября 1936 г.

№ 44. п. 202 — О пьесе «Богатыри» Демьяна Бедного.

...опера-фарс Демьяна Бедного, поставленная под руководством А.Я.Таирова в
Камерном театре с использованием музыки Бородина,

а) является попыткой возвеличения разбойников Киевской Руси, как положительный
революционный элемент, что противоречит истории и насквозь фальшиво по своей
политической тенденции;

б) огульно чернит богатырей русского былинного эпоса, в то время как главнейшие
из богатырей являются в народном представлении носителями героических черт
русского народа;

в) дает антиисторическое и издевательское изображение крещения Руси,
являвшегося в действительности положительным этапом в истории русского народа,
так как оно способствовало сближению славянских народов с народами более
высокой культуры..."	
\end{zzquote}

\ii{17_02_2022.yz.holmogorov_egor.1.kievrus.pic.5}

Пьеса была снята с репертуара, Демьян Бедный исключен из партии (слыханное ли
дело - исключение из партии за неуважение к Крещению Руси!), а понятие
\enquote{Киевская Русь} было освящено авторитетом Политбюро, хотя нетрудно заметить,
что употребояется оно снова в \enquote{эпохальном} смысле, как синоним древнейшего
этапа истории Руси. 

При этом, в то же время, из советской историографии требовалось убрать термин
Маркса «Империя Рюриковичей». Поскольку этот термин Маркс употребил в
маньякально русофобской работе «Разоблачение дипломатической истории XVIII
века», которая теперь стала считать подрывной, то на его использование наложено
было вето.

Кстати сказать, украинские пропагандисты регулярно утверждают что в этой работе
Маркса, в 4-й главе содержится утверждение, что «Россия получившая свое
название в XVIII веке нагло претендует на историческое наследие Руси, созданной
на 800 лет раньше». Тут смешно не то, что украинцы ссылаются на авторитет
Маркса, памятники которому посносили в процессе «декоммунизации», а то, что
каждый заглянувший в 4-ю главу марксова сочинения ничего подобного там не
обнаружит.

\ii{17_02_2022.yz.holmogorov_egor.1.kievrus.pic.6}

Маркс, конечно, Россию ненавидел, но считал её древней страной и никак
древнерусского периода от Российского царства и Империи не отличал. Зато Киеву
отвесил между делом такую плюху: «Сам Киев, древняя столица, перестав быть
резиденцией великого князя, превратился в заурядный город и был предоставлен
своей собственной судьбе».

На замену марксовой «Империи Рюриковичей» Борис Дмитриевич Греков и соорудил в
1939 году «Киевскую Русь», опубликовав соответствующую монографию. При этом с
первых же слов предисловия становится понятно, что под Киевской Русью Греков
подразумевает всё тот же «Киевский период» в истории России, что и
дореволюционные историки.

Книга начинается так: «Чем объяснить хорошо известный факт, что русский народов
своем былинном эпосе отводит самое видное место именно Киевскому периоду своей
древней истории?» То есть «Киевская Русь» в интерпретации Грекова это просто
киевский период истории русского народа. Связь дискурса Грекова с недавним
\enquote{богатырским} скандалом совершенно очевидна.

Период до удельной раздробленности, то есть время Владимира и Ярослава. Скажем
эпоху «Слова о полку Игореве» Греков «Киевской Русью уже не считал и полагал,
что Киев в этот период утратил свое значение.

\ii{17_02_2022.yz.holmogorov_egor.1.kievrus.pic.7}

«Под Киевским периодом истории ни в коем случае нельзя разуметь период уделов с
его разобщенностью отдельных княжений и княжескими усобицами, как это делают и
Майков и отчасти Ключевский. Время уделов нельзя называть Киевским хотя бы по
той причине, что Киева как политического центра уже тогда не было, он
стушевался и решительно затерялся среди других местных центров…»

А дальше, говоря об удельном периоде, Греков наносит прямо-таки смертельный
удар тем, кто использует сконструированное им понятие для украинской
пропаганды:

«Если можно говорить в это время о единстве русского народа, то лишь только в
смысле этническом. Политического единства, хотя бы в относительной форме
Киевского государства, в это время уже не было. Русская народная масса, где бы
она ни находилась, какой бы власти она ни подчинялась, не забывала никогда
своей этнической общности, но политического единства, хотя бы в относительной
форме Древнерусского государства, в это время уже не было».

То есть по мысли Грекова киевское государство уже прекратилось, а этническое
единство русского, именно русского народа продолжало в удельный период
существовать. Никаких тебе украинцев или, хотя бы, протоукраинцев.

Первое издание труда Грекова содержало и прямой выпад в адрес отца-основателя
украинского национализма Грушевского: \enquote{Грушевский путем явной фальсификации
источников пытается приспособить [историю Руси] к истории одной только
Украины}. Иными словами, введший в историографию концепт \enquote{Киевская Русь}
академик Греков прямо обвинял крупнейшего представителя украинской
националистической историографии Грушевского в том, что тот фальсификатор, а
также Греков специально подчеркивал, что понятие \enquote{Киевской Руси} не относится к
одной лишь Украине. Интересно, что в позднейших изданиях грековского труда
обвинение Грушевского в фальсификации было снято - советская компилиментарность
к украинскому национализму, пусть даже в его буржуазной форме, вступила в свои
права.

\ii{17_02_2022.yz.holmogorov_egor.1.kievrus.pic.8}

Даже в советский период, когда академик Греков был главой советской
исторической науки, термин «Киевская Русь» не пользовался всеобщим признанием.
Другая группа советских историков, например Владимир Мавродин и его ученики,
пользовалась не понятием «Киевская Русь», а понятием «Древняя Русь», или
«Древнерусское государство». У филологов понятие «Киевская Русь» вообще не
вошло в оборот – они использовали исключительно термин «древнерусская
литература».

То есть обосновывать украинские националистические притязания на понятии
«Киевской Руси» это дремучий совок, к тому же пдлохо переваренный. Этот термин
был создан в русской историографии, в советский период. И декоммунизированной
Украине его употреблять не к лицу.

Впрочем, как мы давно заметили, при всем своем формальном антикоммунизме,
украинские националисты охотно пользуются подарками Ленина — территорией УССР,
и продуктом сталинской украинизации школы – мовой. Почему бы им не пользоваться
и советскими историографическими терминами. Ничего нового.

Главное, чтобы мы сами в России не запутались. Термин «Киевская Русь»
искусственный и из употребления должен быть безоговорочно выведен.
