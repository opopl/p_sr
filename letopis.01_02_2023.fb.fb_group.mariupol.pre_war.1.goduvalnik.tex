%%beginhead 
 
%%file 01_02_2023.fb.fb_group.mariupol.pre_war.1.goduvalnik
%%parent 01_02_2023
 
%%url https://www.facebook.com/groups/1233789547361300/posts/1401064853967101
 
%%author_id fb_group.mariupol.pre_war,kipcharskij_viktor.mariupol
%%date 01_02_2023
 
%%tags mariupol
%%title Годувальник
 
%%endhead 

\subsection{Годувальник}
\label{sec:01_02_2023.fb.fb_group.mariupol.pre_war.1.goduvalnik}
 
\Purl{https://www.facebook.com/groups/1233789547361300/posts/1401064853967101}
\ifcmt
 author_begin
   author_id fb_group.mariupol.pre_war,kipcharskij_viktor.mariupol
 author_end
\fi

Годувальник

Році так у 2004-му році один студент \enquote{поплив} на захисті лабораторних роботах
з \enquote{Теорії різання} (не думайте погано – з різання металів!). Аби якось
спантеличити (ошелешити) його аби він не намагався \enquote{прослизнути}, я у
присутності студентської групи задав йому питання:

- Що ж Ви ставите мене у незручне становище, годувальник Ви мій?

Він не зрозумів. Група теж не зрозуміла. 

- Ви ж мене і мою сім'ю годували весною 1983-го року...

- Та я ж тоді ще не народився!

Він трохи розслабився... Довчасно...

Весною 1983-го року в в магазинах Іллічівського (де я жив) і Жовтневого (де я
працювава) районах Жданова (тимчасова назва Маріуполя з 1948-го по 1989-й
роки) не стало картоплі. Від слова зовсім. Ринок відреагував на це різким
підвищенням цін. Тоді у мене, інженера зі стажем неповний рік, заробітну
платня була аж 115 рублів на місяць. Тобто, 115 мені нараховували, а потім
знімали податкові, профсоюзні, комсомольські і ще щось (дякувати Богу, хоч не
знімали \enquote{холостяцькі} або як їх ще називали \enquote{податок на яйця}, який платили
неодружені бездітні чоловіки, бо в нас вже була дитина, але через це дружина
отримувала 35 \enquote{дитячих} рублів). Тобто, ринкові ціни у сімейний бюджет не були
закладені. А от на Лівому березі у деяких магазинах картопля була! Чому? Тому
що їм овочі постачав Азовстальський ОРС (отдел рабочего снабжения) – чи то
більше восени запасли, чи то менше згноїли у сховищах. Мій колега Ігор з
дружиною Галею мешкав саме на Лівому березі. Так от, дізнавшись що мені
потрібна картопля, Ігор зателефонував Галі і попросив її заздалегідь зайняти
чергу (!!!) в овочевому. Після роботи ми з Ігорем під'їхали до магазину,
підійшли до Галі і, ігноруючи натяки \enquote{Вас тут не стояло!!!} стали перед Галею.
Коли підійшла черга, ми взяли належні \enquote{в одні руки} 3 (три!) кілограми
картоплі, тобто на усіх 12 кг! Мабуть ви зраділи \enquote{Ага він рахувати не
може!!!}. Ну то рахуйте самі: мені 3 кг, Ігорю 3 кг, Галі 3 кг і ще 3 кг отому
самому студенту, який ще був у маминому (вже помітному) животику!

Усю картоплю віддали мені, мотивуючи це тим, що \enquote{Собі ми й завтра купимо}. І я
трамвайчиком з Пашковського до Левченка, а там пішочком повз Перші ворота аж
до П'ятих (зупиночний пункт 1256, якщо хтось в курсі)... 

Отак ще до свого народження Сашко годував мою сім'ю. 

Не скажу, що він одразу став зразковим студентом, але допуск до іспиту він отримав. (Картопля таки допомогла?)

Фото надано Оленою Сугак.

Навздогін: Дякую усім, хто допоміг відновити мою пам'ять і встановити
місцезнаходження того овочевого, поруч з яким жили мої друзі. Нажаль, той
район \enquote{звільнили}... 

Дуже \enquote{звільнили}...

Від будинків... 

Від людей...

Де мої друзі-годувальники?
