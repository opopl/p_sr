% vim: keymap=russian-jcukenwin
%%beginhead 
 
%%file 04_11_2022.fb.kaspruk_iryna.lviv.1.jak_ty_pyshesh
%%parent 04_11_2022
 
%%url https://www.facebook.com/irkaspruk/posts/pfbid0fKofzJtos4RdSPY8GXmMv89L7ze85CjsPUk2UgoLd3RNrk8zbud8tH7U5Kr7fHUol
 
%%author_id kaspruk_iryna.lviv
%%date 
 
%%tags 
%%title Розкажи мені, як ти пишеш? Як взагалі народжуються твої вітання, листи чи просто вірші?
 
%%endhead 
 
\subsection{Розкажи мені, як ти пишеш? Як взагалі народжуються твої вітання, листи чи просто вірші?}
\label{sec:04_11_2022.fb.kaspruk_iryna.lviv.1.jak_ty_pyshesh}
 
\Purl{https://www.facebook.com/irkaspruk/posts/pfbid0fKofzJtos4RdSPY8GXmMv89L7ze85CjsPUk2UgoLd3RNrk8zbud8tH7U5Kr7fHUol}
\ifcmt
 author_begin
   author_id kaspruk_iryna.lviv
 author_end
\fi

- Ір, ти ж реально класно це робиш!

- Ти не хочеш писати на замовлення?

- Тобто безкоштовно? Так не має бути, не знецінюй себе.

- Ти не вмієш себе продавати! (Тут прошу зрозуміти правильно)

- Твої вітання справді унікальні і дуже щирі. А те, що вони ще й віршовані -
взагалі  @igg{fbicon.flame} 

\ii{04_11_2022.fb.kaspruk_iryna.lviv.1.jak_ty_pyshesh.pic.1}

Ці та інші фрази різних людей "ходять" за мною кілька днів. Спочатку я
відганяла їх. Потім задумалась над тим, чому вони мені згадуються саме зараз.
Згодом почала аналізувати все,  намагаючись дивитися на себе і свою творчість
очима сторонньої людини. 

 @igg{fbicon.speech.baloon} : 

- Розкажи мені, як ти пишеш? Як взагалі народжуються твої вітання, листи чи
просто вірші? - питає якось подруга, яка і запустила ланцюжок цих думок. 

Гаразд, розкажу. Пишу їй і паралельно сама усвідомлюю, що це ж насправді
Робота. Так, творча. Так, я це люблю. Так, я завжди кажу, що це не моя заслуга,
а щедрий Божий дар. Але...

 @igg{fbicon.diamond.orange.small}  Все починається зі знайомства. Щоб написати щось для людини, потрібно її
відчути. А найчастіше  відчути обох - ту, яка просить написати, і ту, для якої
творимо. Знайомство у переписці чи дзвінках займає немало часу - від 30 хв до
періодичного спілкування упродовж кількох днів. 

 @igg{fbicon.diamond.orange.small} Мої пояснення щодо інформації, яку я хотіла б отримати для створення вітання.
У кожному випадку намагаюсь відчути щось особливе, не банальне, оту "хімію",
яка об'єднує людей. 

 @igg{fbicon.diamond.orange.small} Тепер треба з цим пожити. Іноді рядки
 народжуються одразу, а іноді я проводжу з ними кілька днів. Їх не виманиш
 примусом, бо муза - ще та примхлива пані, яка з легкістю може обернутися і
 піти надовго @igg{fbicon.face.upside.down} 

 @igg{fbicon.diamond.orange.small} Пишу. Хто вже мав досвід творчого тандему зі мною, підтвердить, що пишу я не
скупі 2-3 стовбчика, а добротних 6-8 по 4 рядки. Іноді більше. 

 @igg{fbicon.diamond.orange.small} Далі вичитую. Спочатку мовчки, а далі - вголос. Коли вірш звучить, стає краще
помітно моменти, які варто поправити. Редагую. Знову перечитую. І аж тоді, коли
мені все подобається, надсилаю вірш замовнику.

Скільки це займає часу? Від 2-3 годин, якщо маю можливість зануритись в процес
у тиші та на самоті. Іноді - довше. 

 @igg{fbicon.diamond.orange.small}  Зворотний зв'язок від замовника. Редагуємо, якщо є потреба. Я завжди з
радістю сприймаю пропозиції щось змінити чи дописати, адже для мене важливо,
щоб людина, для якої пишемо, відчула, що вітання створене саме для неї, не
шаблонно, не банально, іноді з гумором, іноді доволі сентиментально. Плюсуйте
ще трохи часу. 

Далі. Якщо це листи:

 @igg{fbicon.diamond.orange.small}  Заздалегідь залучаю моїх казкових фей і невтомних помічниць - ілюстраторку
Olenka Osipchuk  та графічну дизайнерку Oksana Tsebrivska . Так, Олена створює
унікальну ілюстрацію для листів, щоб малятам (і не тільки) було приємно, цікаво
і захопливо читати лист. Оксана вміло поєднує текст із зображенням та працює
над довершеним виглядом листа. З пакуванням та тематичними наліпками допомагає
теж вона. І з друком теж, ага @igg{fbicon.sparkles}  Додайте сюди ще деякі з описаних вище аспектів,
враховуючи, що замовник тепер - я @igg{fbicon.smile}  Ось такий у нас чудовий тандем. 

 @igg{fbicon.speech.baloon} :

- Коли ти це все встигаєш? - все випитує невгамовна дівчина. 

Спочатку за звичкою пробую пручатись, що "тут нічого особливого", але подальші
мої слова, які пишу, трохи приводять до тями... 

Отже, пишу:

 @igg{fbicon.diamond.orange.small} коли спить молодший син. А спить він хіба вночі, денних снів практично немає
(або я просто не можу фізично вмістити у годину денного сну ВСЕ);

 @igg{fbicon.diamond.orange.small} коли чоловік йде на прогулянку з дітьми, а я залишаюсь вдома. Чи навіть не
так... Я не йду на прогулянку з ними, бо маю щось писати (хоча дуже хочу);

 @igg{fbicon.diamond.orange.small} щоразу, коли замість прийняти ванну/почитати книгу/подивитись
фільм/побавитись з дітьми/просто, врешті решт, поспати я обираю написати для
когось вітання. Бо я це люблю🫶

@igg{fbicon.speech.baloon} :

- Ти розумієш, що ти крадеш у дітей маму, у чоловіка - дружину, у сЕбе - себЕ і
той дорогоцінний час для себе, якого і так вічно бракує? Крадеш, бо коли ти
береш за це оплату - це робота і є фінансова винагорода. - розмова набирає
обертів і я не маю що сказати... І поступово визріває розуміння, що таки так,
це - робота. Непроста, тривала, інтелектуальна і щоразу унікальна. О, як
нескромно заспівала  @igg{fbicon.face.upside.down}  

Отож, починаю вчитись називати вартість своєї праці. Це непросто, це вихід за
звичні межі, але я спробую. 

Звісно, щоб бути чесною до кінця, були, є і будуть люди, для яких мої вітання
залишаться подарунком від щирого серця @igg{fbicon.heart.yellow}  Тож, якщо я
так вам кажу - не пручайтесь) 

А що скажете ви? Як ви уявляли процес написання віршів/вітань/листів? Чи
змінилося сприйняття після моєї розповіді? Чи готові ви платити за таку
послугу, чи це у мене вже росте  @igg{fbicon.crown} ?

***

Доречі, роздуми мої дуже перегукуються з темою місяця у Місце щасливих людей .
Ярослава Павлюковець підняла тему грошей - вміння приймати, заробляти,
витрачати, примножувати свій дохід. Долучайтесь, ще не пізно і точно кожен
знайде щось для себе. Тут цікаві спікери, живі зустрічі, класні завдання,
нейрографіка і просто чудові знайомства з різними, надихаючими людьми! 

***
Момент впіймала Іра Лан  @igg{fbicon.heart.yellow} 
***

Хтось дочитав до кінця? @igg{fbicon.face.grinning.smiling.eyes} 

\#sonyachna \#сонячні\_вірші \#сонячні\_роздуми \#листи

\ii{04_11_2022.fb.kaspruk_iryna.lviv.1.jak_ty_pyshesh.orig}
\ii{04_11_2022.fb.kaspruk_iryna.lviv.1.jak_ty_pyshesh.cmtx}
