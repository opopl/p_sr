% vim: keymap=russian-jcukenwin
%%beginhead 
 
%%file 21_11_2022.fb.fb_group.story_kiev_ua.1.vojna_zhizn
%%parent 21_11_2022
 
%%url https://www.facebook.com/groups/story.kiev.ua/posts/2072411239622333
 
%%author_id fb_group.story_kiev_ua,drobot_tatjana.kiev
%%date 
 
%%tags 
%%title Война беспощадно разделила жизнь на до и после
 
%%endhead 
 
\subsection{Война беспощадно разделила жизнь на до и после}
\label{sec:21_11_2022.fb.fb_group.story_kiev_ua.1.vojna_zhizn}
 
\Purl{https://www.facebook.com/groups/story.kiev.ua/posts/2072411239622333}
\ifcmt
 author_begin
   author_id fb_group.story_kiev_ua,drobot_tatjana.kiev
 author_end
\fi

В период кapaнтинa два года назад наша квартира сияла чистотой и порядком. 

Я хорошо помню то время. Внезапно закрыли все, на работу в салон стало ходить
не нужно, праздники и встречи с друзьями ограничились четырьмя стенами нашей
киевской квартиры. 

В квартире вместе со святым углом появился еще один. Санитарный. 

С масками, перчатками и антисептиками.

Я перебирала шкафы и выбрасывала старые вещи, разобрала хлам с залежами времен
нашего конфетно-букетного периода, окна были вымыты дважды за неделю, а в доме
каждый день запахло свежей выпечкой. 

\ii{21_11_2022.fb.fb_group.story_kiev_ua.1.vojna_zhizn.pic.1}

Я даже открыла свой кейтеринг, закупала инвентарь по всему миру, писала книгу с
быстрыми рецептами, дело быстро пошло на лад и я наслаждалась новым проектом.

...В моем гардеробе много платьев и туфель на каблуке. 

С 24 февраля я не прикасалась к шкафу, где висят мои платья.

Джинсы и кроссовки стали пропуском в подвешенное тревожное время. 

Война с особым остервенением разделила мой гардероб на до и после. 

Платья и каблуки задвинуты в дальний угол и отныне наречены довоенными. 

С 24 февраля я вычеркнула многих авторов из ежедневно читаемых. 

Война сбросила внешние яркие покрывала с их никчемных темных душ. 

Книги не стали довоенными, они стали злобно предательскими.

С 24 февраля кейтеринг стал непонятной довоенной роскошью с четким ограничением
по комендантскому часу. 

С 24 февраля при прощании мы обнимаемся с клиентами салона как в последний раз. 

Желаем друг другу не хорошего вечера, а мирного неба.

С 24 февраля комедии и фильмы сменились на новости и сводки, от которых бросает
то в жар, то в холод.

Война беспощадно разделила жизнь на до и после.

Война оголила наши души, кому-то подарила еще десять фантомных рук и ментальных
сердец, чтобы обнять каждого нуждающегося, а у кого-то, как в Иисусовой притче,
«увяло, и, как не имело корня, засохло»; и с этим корнем ушло все последнее
доброе, вечное..

Война разбила дышащие на ладан отношения, порвала там, где тонко, сбросила
маски, и еще прочнее зацементировала семьи, где на пьедестале уважение и любовь
на первом месте.

Мы сейчас все сдаем свой собственный жизненный экзамен. 

Он самый главный, государственный. 

Каждый день мы ищем то, ради чего нам стоит жить. И дарим надежду тем, кто
совсем отчаялся.

Мы уже смеемся над отключениями и вывешиваем мемы с лайфхаками как на свечах
сварить борщ.

Вчера я вспомнила весной сказанное мной выражение, что Киев стал черным от
горя. 

Нет. Он не стал черным. 

Он застыл на время, как ледяная фигура на ВДНХ, он стал жестче и прочнее, чем
закаленная сталь, вместо ярких платьев и кокетливых сумочек он носит берцы и
тактические рюкзаки, а вместо помад и букетов ищет фонари и генераторы.

Это временно.

Чтобы потом достать ящик шампанского из угла, где мною написан гриф «За
Победу!»

P.S. Кстати, вчера мы были с электричеством и я успела спечь к Рождеству
штоллены @igg{fbicon.croissant}

\ii{21_11_2022.fb.fb_group.story_kiev_ua.1.vojna_zhizn.orig}
\ii{21_11_2022.fb.fb_group.story_kiev_ua.1.vojna_zhizn.cmtx}
