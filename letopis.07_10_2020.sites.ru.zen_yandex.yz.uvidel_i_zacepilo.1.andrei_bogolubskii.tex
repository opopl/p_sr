% vim: keymap=russian-jcukenwin
%%beginhead 
 
%%file 07_10_2020.sites.ru.zen_yandex.yz.uvidel_i_zacepilo.1.andrei_bogolubskii
%%parent 07_10_2020
 
%%url https://zen.yandex.ru/media/zacepilo/triapka-ikona-i-lestnica-chto-ostalos-ot-osnovatelia-rossiiskogo-gosudarstva-5f7df59e1e2da6289ecd750a
 
%%author 
%%author_id yz.uvidel_i_zacepilo
%%author_url 
 
%%tags 
%%title Тряпка, икона и лестница: что осталось от основателя Российского государства
 
%%endhead 
 
\subsection{Тряпка, икона и лестница: что осталось от основателя Российского государства}
\label{sec:07_10_2020.sites.ru.zen_yandex.yz.uvidel_i_zacepilo.1.andrei_bogolubskii}
\Purl{https://zen.yandex.ru/media/zacepilo/triapka-ikona-i-lestnica-chto-ostalos-ot-osnovatelia-rossiiskogo-gosudarstva-5f7df59e1e2da6289ecd750a}
\ifcmt
	author_begin
   author_id yz.uvidel_i_zacepilo
	author_end
\fi

\begin{leftbar}
  \begingroup
    \em\Large\bfseries\color{blue}
В ночь на 29 июня 1174 года на лестнице в Боголюбском замке был зарублен мечами
Великий князь Андрей Юрьевич — человек, сделавший северо-восток страны центром
государства и тем самым направивший Россию по её нынешнему вектору развития.
Восемь с половиной веков назад — это почти сказочные времена, но от великого
политика осталось кое-что материальное.
  \endgroup
\end{leftbar}

\index[names.rus]{Боголюбский, Андрей!Святой благоверный князь}

Когда большевики в феврале 1919 года вскрыли раку с мощами святого благоверного
князя Андрея Боголюбского из Успенского собора во Владимире, они обнаружили
скелет мужчины с множественными рублеными повреждениями — и обрывки шёлковой
золототканой одежды. Один из этих обрывков сейчас выставлен в 10 зале
Исторического музея в Москве, загляните полюбопытствовать. По сути, это
единственная сохранившаяся личная вещь Очень Важного Человека.

\ifcmt
  pic https://avatars.mds.yandex.net/get-zen_doc/175604/pub_5f7df59e1e2da6289ecd750a_5f7df5d4b4802f1c4a19311c/scale_1200
  width 0.4
	caption Фрагмент ткани из гробницы Великого князя Владимиро-Суздальского Андрея Юрьевича Боголюбского (1111-1174). Государственный исторический музей, Москва
	fig_env wrapfigure
\fi

Историки 19 века называли Андрея Боголюбского «первым великорусским князем»,
имея в виду, что именно с его бегства во Владимир-на-Клязьме в 1155 году
следует отсчитывать разделение русо-славянской общности на три народа —
великорусский, белорусский и малороссийский. Главная «героиня» упомянутого
бегства и центр этой большой истории — знаменитая Владимирская икона Божией
Матери.

Эта икона была прислана в 1130 году из Византии в подарок киевскому князю
Мстиславу Великому и хранилась в Богородичном монастыре Вышгорода в 16 км от
Киева. Когда в 1155 году ростово-суздальский Юрий Долгорукий (привет,
москвичи!) стал Великим князем Киевским, он посадил своего старшего (на тот
момент) сына Андрея в Вышгороде — чтоб был под рукой.

А дальше всё как в сказке: однажды ночью Вышгородская Богоматерь явилась Андрею
Юрьевичу и распорядилась везти её куда-нибудь, подальше отсюда. Молодой (ему
было 34 года) князь подозрительно подробно собрался, взял семью, дружину, и
двинул в земли, где вырос. Уже в конце пути Богоматерь опять пришла к герою и
велела ставить её не в Ростове или Суздале, древних центрах княжества, а в
молодом Владимире-на-Клязьме.

\ifcmt
  pic https://avatars.mds.yandex.net/get-zen_doc/3963198/pub_5f7df59e1e2da6289ecd750a_5f7df7701e2da6289ed0afa1/scale_1200
  width 0.3
	caption Владимирская икона Божией Матери, Константинополь, первая треть XII века. Государственная Третьяковская галерея, храм Святителя Николая в Толмачах
	%fig_env wrapfigure
\fi

Нынешние историки говорят: будущий Боголюбский хотел домой и независимости от
отца, и всего этого он добился, разыграв гениальную историю с иконой. Любое
государство базируется на мифе, и Андрей буквально привёз этот миф с собой.
Перед этой иконой венчались на царство все русские цари до последнего.

Надо сосредоточиться и зафиксировать: на этой доске 104×69 см, как на
фундаменте, построено государство, в котором я живу. Сильный предмет, в общем.

Впрочем, Андрей и был гением. Он был и железным руководителем, за несколько лет
выстроившим независимое княжество, и (предположительно) слагал религиозные
гимны. В 12 километрах от Владимира он поставил великокняжеский замок
Боголюбый, самое большое каменное сооружение на Руси той эпохи. «Замок» —
потому что строили его каменщики германского императора Фридриха Барбароссы, то
ли нанятые за огромные деньги, то ли присланные Рыжебородым по дружбе (летописи
говорят разное). 

От замка в ансамбле нынешнего Боголюбского монастыря сохранились каменный
переход на полуциркульных арках и квадратная в плане белокаменная башня с
винтовой лестницей (между прочим, единственное на сейчас гражданское сооружение
домонгольской эпохи у нас в стране). На этой лестнице соратники и прикончили
Великого князя в июне 1174 года. Видите, вон эти части — они в левой части
нынешнего здания?

Боголюбский, как все гении, был жесток с людьми и правил, как диктатор.
Составился комплот; напившись для храбрости вина в погребе замка, заговорщики
выломали дверь в спальне князя. Как свидетельствует летопись, 63-летний Андрей
был ещё силён, и хоть меч у него выкрали заранее, яростно отбивался.

Ему даже удалось сбежать от нападавших, но в Лестничной башне силы оставили
раненого князя, он опустился на ступени. Первый подоспевший заговорщик, боярин
Пётр Кучкович, неким рубяще-режущим орудием типа сабли нанёс Андрею «обширную
рану левого надплечья и плечевого сустава, с разрубом лопатки, головки плечевой
кости и крупной подключичной артерии». Это уже — результаты советской
экспертизы. 

\ifcmt
tab_begin cols=3
	caption Андрей Боголюбский, великий князь

  pic https://avatars.mds.yandex.net/get-zen_doc/1708669/pub_5f7df59e1e2da6289ecd750a_5f7df63c1e2da6289ece9587/scale_1200
  width 0.3
	caption Сохранившийся фрагмент Боголюбского замка (левая часть сооружения, 1160-е годы) в составе церкви Рождества (справа, 1758 год). Владимиро-Суздальский музей-заповедник

  pic https://avatars.mds.yandex.net/get-zen_doc/235990/pub_5f7df59e1e2da6289ecd750a_5f7df66fdee66543988990bf/scale_1200
  width 0.35
	caption «Убиение Андрея Боголюбского», Федор Бруни, гравюра, 1839 год. Государственный исторический музей, Москва

  pic https://avatars.mds.yandex.net/get-zen_doc/3540570/pub_5f7df59e1e2da6289ecd750a_5f7df5f215099c198a60644e/scale_1200
  width 0.2
	caption Скульптурная реконструкция облика Андрея Боголюбского, академик Михаил Герасимов, 1939 год. Государственный исторический музей, Москва
tab_end
\fi

С 1919-го и по 1987 год мощи князя были экспонатами музеев (сейчас их вернули
во Владимирский Успенский собор). В 1939 году антрополог и скульптор Михаил
Герасимов восстановил по черепу облик Андрея Юрьевича. Смотрю на бюст и верю —
этот бился до последнего.

И, кстати, не все в Великом Владимирском княжестве были так уж злы на Андрея. В
2015 году реставраторы обнаружили на южной абсиде Спасо-Преображенского собора
Переславля-Залесского — граффити, церковное проклятие. Два столбца текста,
обведенных в общую рамку с крестом сверху: «Петр Кучкович, Амбал, Яким, Ивка,
Петрко, Стырята (всего 20 имён) <...> Си суть убийцы великого князя Андрея, да
будут прокляты».

