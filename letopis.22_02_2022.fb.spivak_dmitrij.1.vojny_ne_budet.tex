% vim: keymap=russian-jcukenwin
%%beginhead 
 
%%file 22_02_2022.fb.spivak_dmitrij.1.vojny_ne_budet
%%parent 22_02_2022
 
%%url https://www.facebook.com/SpivakDmytriy/posts/5237015802984102
 
%%author_id spivak_dmitrij
%%date 
 
%%tags __feb_2022.putin.priznanie
%%title ВОЙНЫ НЕ БУДЕТ
 
%%endhead 
 
\subsection{ВОЙНЫ НЕ БУДЕТ}
\label{sec:22_02_2022.fb.spivak_dmitrij.1.vojny_ne_budet}
 
\Purl{https://www.facebook.com/SpivakDmytriy/posts/5237015802984102}
\ifcmt
 author_begin
   author_id spivak_dmitrij
 author_end
\fi

ВОЙНЫ НЕ БУДЕТ. 

Этот шаг был ожидаем. К сожалению, ни Луганск ни Донецк давно уже был не нужен
никому, в том числе и  нашей власти. Там давно уже нет ничего украинского.
Рубли, российские законы, российское ТВ, Программа обучения  в школах, налоги.
Людей давно бросили на произволяще. 

Вчера де-юре закрепилось лишь то, что и так было всем известно. Никто в Киеве
не собирался идти алгоритмом Минска, а другого альтернативного плана Б, который
мы обещали представить миру с декабря 2019 года,  у Зеленского нет. 

Более того, знаю из достоверных источников, что в окружении нашего Президента
положительно восприняли вчерашнюю новость. Мол, баба с возу.... 

И никаких обязонов перед Западом по Минским соглашениям. Мол, агрессор их
признал, а теперь что мы можем сделать... 

Что ж, Ставки поднимаются на невероятную высоту. Ничего не изменится.
Переговоры Запада и России все равно будут идти. Итоги все равно к лету выйдут
на новую формулу миропорядка и коллективного Договора.  

Военным путём никто нам не поможет вернуть эти территории. Все эти санкции
против ЛДНР - больше для проформы. Пострадают не в Кремле, а обычные люди. 

Но Жители этих районов не интересуют Путина. Впрочем, как и мы с Вами не
интересуем Запад. 

Их игра мне понятна. Мне непонятна позиция Украины. Мы оказались вообще ни к
чему не готовы. Даже торговые отношения с Россией прекратить не можем. Слишком
зависимы в энергетике, а альтернативу не создали. Потому, что коррупция
зашкаливает. Воруют и грябят ежедневно. 

Кстати, и дипломатические отношения мы тоже навряд ли прервём. А что тогда? 

Все остальное - словоблудие и лозунги, никак не меняющие ситуацию. 

Да и четкую позицию Президента так никто и не понял... 

Ещё раз. Надо всем успокоиться. Войны не будет. Кусок территории мы потеряли.
Людей Донбасса с нашей гуманитарной политикой мы потеряли ещё раньше. Только не
хотим в этом признаться. 

У меня нет вопросов к Путину или Западу. С ними все понятно и Все
прогнозируемо. Они так играют. Кто как умеет. Путин агрессивно. Запад иначе. 

У меня вопрос, что будем делать мы в этой ситуации? 

Кроме патриотических призывов и заявлений. 

И ещё раз - Войны не будет!!!!
