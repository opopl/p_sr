% vim: keymap=russian-jcukenwin
%%beginhead 
 
%%file 16_01_2022.stz.news.ua.hvylya.1.anatomia_vraga.4.sakralnost_vlasti
%%parent 16_01_2022.stz.news.ua.hvylya.1.anatomia_vraga
 
%%url 
 
%%author_id 
%%date 
 
%%tags 
%%title 
 
%%endhead 

\subsubsection{Сакральность власти - как основа легитимности принятия решений}

Необходимо заметить, что клерикальный фактор православной церкви, и сегодня
успешно работающий на общегосударственную стратагему московской экспансии, уже
в конце XV века, выписал долгосрочную концепцию державности Москвы не как
исторического \enquote{выскочки}, но как \enquote{третьего Рима}, базируясь на
браке Ивана III и Софьи Палеолог - племяннице последнего императора Византии
Константина XI Палеолога.

Концепция \enquote{третьего Рима} стала удобным обоснованием проистечения
\enquote{святости} единоначальной власти. \enquote{Продолжение традиций
погибшей Византии}, являющейся потомком античного Рима. Московские великие
князья, а в последствии цари, начиная с первого официально венчаного Ивана IV,
становились \enquote{прямыми потомками римских цезарей и православных
императоров} Константинополя в 1453 году, покоренного турками -османами.

В народном сознании московитов эта величавая конструкция \enquote{сохранения
святого наследия погибшей Византии в царях}, становиться фундаментом
псевдоправославной, \enquote{единственно верной} экспансионистской философии
\enquote{защиты} Москвы не только от \enquote{хаоса} востока, но и от
\enquote{непонятных свобод} запада латинян, как мировоззренческой угрозы
основам государства. Более того, в коллективном бессознательном, византийская
концепция становится подтверждением \enquote{сакральности} (природности)
вертикали власти что в контексте нынешнего, продолжающегося и кажущегося многим
\enquote{иррациональным} противостояния РФ коллективному западу, очень
актуально. Не даром в своих письмах польскому королю Стефану Баторию уже Иван
IV писал, дескать, \enquote{ты царь \enquote{выборный} - не настоящий}.

