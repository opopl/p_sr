% vim: keymap=russian-jcukenwin
%%beginhead 
 
%%file 04_03_2022.stz.news.ua.pravda.1.reportazh_z_kieva
%%parent 04_03_2022
 
%%url https://www.pravda.com.ua/articles/2022/03/4/7328163
 
%%author_id rjasnyj_dmytro,news.ua.pravda
%%date 
 
%%tags 
%%title Репортаж з Києва. Війна очима журналіста
 
%%endhead 
 
\subsection{Репортаж з Києва. Війна очима журналіста}
\label{sec:04_03_2022.stz.news.ua.pravda.1.reportazh_z_kieva}
 
\Purl{https://www.pravda.com.ua/articles/2022/03/4/7328163}
\ifcmt
 author_begin
   author_id rjasnyj_dmytro,news.ua.pravda
 author_end
\fi

\ifcmt
  ig https://i2.paste.pics/58a4f47fab0e85d9991ce699a55ca20b.png
  @wrap center
  @width 0.8
\fi

\begin{zznagolos}
Дорога з південного передмістя Києва в центр і сьогодні займає біля години. Але
якщо раніше ви витрачали цей час на корки, то зараз – на черги на блокпостах, і
огляд втрат, які понесло місто.

Цей репортаж народився абсолютно випадково. За кілька днів до поїздки я
познайомився в місцевій теробороні з фотографом \enquote{НВ} Наталією Кравчук. Їй було
важливо побачити Київ, де вона не була майже тиждень з моменту початку війни.
Мені – завезти нині дефіцитний вантаж одному зі столичних бізнесменів, завдяки
якому у хлопців з моєї тероборони з’явилось п’ять бронежилетів і гроші на
якісне обладнання блокпостів. 

Перше, з чим ви зіштовхнетесь на шляху з околиці в центр – новий маршрут в
районі аеропорту \enquote{Жуляни}. Він пов'язаний з укріпленням цього об’єкту, тому аби
не спрощувати життя ворогу, тут обійдемось без фотографій. 	
\end{zznagolos}

% 1-2
\ii{04_03_2022.stz.news.ua.pravda.1.reportazh_z_kieva.pic.1}

Зі знімками підбитого будинку на проспекті Лобановського все навпаки. Його
просто необхідно фотографувати для наповнення фактажем майбутнього процесу в
Гаазі. Тут поміж битої цегли, бетону і арматури – книжки, дитячі малюнки,
фотографії. На одній з них зі зворотної сторони підпис \enquote{Киев. 1989 год}. Чомусь
здається, що автор цього рядка, якщо ще живий, найближчим часом не готовий як
раніше користуватись російською мовою. Сьогодні вона викликає відразу.

% 3-5
\ii{04_03_2022.stz.news.ua.pravda.1.reportazh_z_kieva.pic.2}

% 6-8
\ii{04_03_2022.stz.news.ua.pravda.1.reportazh_z_kieva.pic.6_8}

Поряд з розбитим будинком – черга в Новус. Містяни тримаються в ній як графиня,
що програла свій капітал в карти – пригнічені, але ввічливі і з прямою спиною.

Наступна \enquote{пам'ятка} на вашому шляху – згорілий автомобіль поліції неподалік
Південного залізничного вокзалу. Він був знищений від уламків ракети, яка була
нещодавно збита в цьому районі. Встигаємо зробити буквально пару фото, і майже
одразу отримуємо емоційну рекомендацію від місцевих силовиків \enquote{видалити фото і
їхати далі}.

\ii{04_03_2022.stz.news.ua.pravda.1.reportazh_z_kieva.pic.9_10}

На Львів наразі люди згодні їхати звичайними приміськими електричками. У
лаунж-залі очікувань, куди в ці дні неможливо попасти навіть за гроші, теж
знижені вимоги до життя: жінки в шубах самостійно наливають собі воду з
п’ятилітрових банок води в пластикові стакани.

На самому вокзалі людей відносно небагато. Але це хибне сприйняття. Буквально
за день до цього тут було не проштовхнутись. Тобто, майже усі, хто хотів,
виїхали? Ні. Просто з цього тижня столичний вокзал де-факто змінив юрисдикцію
державного підпорядкування – інакше впоратись з потоком людей, які місцями, на
жаль, вже почали втрачати обличчя, було неможливо.

\ii{04_03_2022.stz.news.ua.pravda.1.reportazh_z_kieva.pic.11_13}

З вокзалу ми їдемо в центр міста до вищезгаданого бізнесмена. Віддаємо обіцяне.
Але це точно той випадок, коли він зробив для нас (нашої тероборони) значно
більше, ніж ми для нього. 

\ii{04_03_2022.stz.news.ua.pravda.1.reportazh_z_kieva.pic.14_16}

Звідси ми маємо повертатись назад – до південної околиці. Так і робимо, з
невеликим \enquote{крюком} - заїжджаємо на телевишку, яка нещодавно зазнала удару.
Місцевий представник розповідає, що по ній було нанесено одразу чотири удари.
Один з них попав в саму станцію: чорний слід  - це саме від нього. 

Три інших розбили супутню інфраструктуру. Разом з окулярами одного з місцевих
співробітників, і серцями його рідних. Він загинув.

\ii{04_03_2022.stz.news.ua.pravda.1.reportazh_z_kieva.pic.17_19}

Незадовго до нас на території телевишки були іноземні журналісти. Один з них
закляк над калюжею крові. Склалось враження, що це не через страх, а через
розуміння безглуздості цієї і тисяч інших людських трагедій в цій війні.

\ii{04_03_2022.stz.news.ua.pravda.1.reportazh_z_kieva.pic.20_21}

Впевнений, що так думає кожен в цивілізованому європейському світі. Україна є
його частиною навіть в стані війни – повертаючись до околиці міста через
проспект Перемоги, ми зустріли розбите відділення \enquote{Нової пошти}, з полиць якої
за тиждень війни не вкрали посилки. 

Вони обов'язково дістануться своїх адресатів. 

\ii{04_03_2022.stz.news.ua.pravda.1.reportazh_z_kieva.pic.22_24}

Дмитро Рясний, УП
