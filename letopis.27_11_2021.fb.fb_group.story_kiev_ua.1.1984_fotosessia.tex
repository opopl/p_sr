% vim: keymap=russian-jcukenwin
%%beginhead 
 
%%file 27_11_2021.fb.fb_group.story_kiev_ua.1.1984_fotosessia
%%parent 27_11_2021
 
%%url https://www.facebook.com/groups/story.kiev.ua/posts/1805985932931533
 
%%author_id fb_group.story_kiev_ua,fedjko_vladimir.kiev
%%date 
 
%%tags 1984,fotografia,fotosessia,kiev
%%title 1984 рік. Історія однієї фотосесії...
 
%%endhead 
 
\subsection{1984 рік. Історія однієї фотосесії...}
\label{sec:27_11_2021.fb.fb_group.story_kiev_ua.1.1984_fotosessia}
 
\Purl{https://www.facebook.com/groups/story.kiev.ua/posts/1805985932931533}
\ifcmt
 author_begin
   author_id fb_group.story_kiev_ua,fedjko_vladimir.kiev
 author_end
\fi

1984 рік. Історія однієї фотосесії...

Я належу до корінних киян. І покидав Київ тільки на час службі в радянській
Армії у 1966-1969 роках. З середини 70-х моя духовна ніша – Київське
хореографічне училище, Оперний театр, художники і скульптори Києва. Брав участь
у фотовиставках, а також у роботі над фотоальбомом \enquote{Зустрічі з балетом} про
український балет.

\ii{27_11_2021.fb.fb_group.story_kiev_ua.1.1984_fotosessia.pic.1}

Доньці дев’ять років і вона полюбляє роздивлятися мої фотографії, особливо
балету. Якось ввечері я переглядав портретну фотозйомку однієї із своїх творчих
подружок – балерини Валерії, учениці випускного класу Галини Кирилової,
художнього керівника хореографічного училища, відзняту напередодні. Юля сиділа
поруч і теж розглядала фотографії. Затримавши погляд на одному з портретів,
вона сказала:

- Тату, я теж хочу портрет!

- Доживемо до неділі – сфотографую. (Розмова відбувалася у понеділок). Тільки в
суботу помий голову, щоб для зйомки волосся було свіже.

Тиждень пролетів швидко. В суботу я повернувся з театру пізно, коли Юля вже
спала. Дружина передала мені нагадування доньки про фотографування в неділю.


***

Вранці поснідали. Смакую каву. Надійка розчесала Юлі довге волосся…

Ставлю світильники за класичною студійною схемою. Юля каже:

- Повісь чорний фон. 

Вішаю фон, ставлю широкоформатну камеру на штатив, вмикаю світильники... На
підлозі кладу серветку – місце, де повинна стояти модель.

\ii{27_11_2021.fb.fb_group.story_kiev_ua.1.1984_fotosessia.pic.2}

Юля заходить в кадр і рукою проводить по грудям, показуючи мені, де повинна
проходити нижня межа кадру – знімаємо крупний план. 

- Я готова!

При фотозйомках портретних серій у мене був ритуал – перед першим кадром я
хрестив камеру, опускався на одне коліно і цілував балерині руку. Надійка і Юля
про це знали. 

За звичкою хрещу камеру і... Юля простягає мені руку для поцілунку. Стримую
посмішку – ритуал є ритуал. Опускаюся на коліно, цілую руку. Знімаю перший
кадр... 

Юля міняє поворот голови.

- Я готова!

Ще один кадр... Донька знову міняє поворот голови.

- Я готова!

У мене виникає відчуття, що я працюю з досвідченою моделлю, яка чітко знає, як
вона хоче виглядати на фотографії. 

П’ятий кадр...

- Повісь світлий фон, а я переодягнусь.

Поки я замінюю фон, Юля повертається у Надійчиному пальто з пухнастим комірцем.
Пальто довге і волочиться по підлозі. Але для фотозйомки це не має значення –
ми знімаємо крупні плани. Донька стає в кадр...

- Я готова.

Трошки змінює нахил голови...

- Я готова.

Роблю кадр і Юля говорить:

- Сім фотографій! Досить. 

У мене виникли задумки зробити ще декілька сюжетів і я пропоную продовжити
фотозйомку, але донька невмолима. 

***

Наступного дня проявив плівки, надрукував фотографії. Доньці портрети
сподобалися, але ще фотографуватися вона не захотіла. 

***

\ii{27_11_2021.fb.fb_group.story_kiev_ua.1.1984_fotosessia.cmt}
