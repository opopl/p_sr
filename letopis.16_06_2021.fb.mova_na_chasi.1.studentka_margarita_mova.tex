% vim: keymap=russian-jcukenwin
%%beginhead 
 
%%file 16_06_2021.fb.mova_na_chasi.1.studentka_margarita_mova
%%parent 16_06_2021
 
%%url https://www.facebook.com/infomova/posts/3772944306150982
 
%%author Мова на Часі
%%author_id mova_na_chasi
%%author_url 
 
%%tags harkov,jazyk,mova,skovoroda_grigorii,studenty,ukraina,ukrainizacia
%%title Почему вы насилуете меня, мою психику и мои нервы?» — студентка Маргарита обматюкала українську мову
 
%%endhead 
 
\subsection{Почему вы насилуете меня, мою психику и мои нервы?» — студентка Маргарита обматюкала українську мову}
\label{sec:16_06_2021.fb.mova_na_chasi.1.studentka_margarita_mova}
\Purl{https://www.facebook.com/infomova/posts/3772944306150982}
\ifcmt
 author_begin
   author_id mova_na_chasi
 author_end
\fi

«Пишу курсовую работу на е\#\%чем украинском языке…Почему вы насилуете меня, мою
психику и мои нервы?» — студентка Маргарита обматюкала українську мову
Маргарита Миколаївна під ніком rrito.burrito оприлюднила розповідь в Інстаграмі
про свою огиду до української. Дівчина з смт Слобожанське, яка вчиться в ХНПУ
ім. Г. Сковороди і працює фельдшеркою на швидкій, упевнена, що її утискають
через заборону писати курсову російською. 

«Пишу курсовую работу на е\#\%чем украинском языке и вот понимаю, что
украинская власть, она меня ущемляет, ущемляет мои права!» — почала зі
скандальної заяви Маргарита.

«Что значит, я не могу писать свою научную работу, в которой мои заключения,
мои выводы, в которой я описываю, как я проводила эту работу, да. Почему я не
могу написать её на русском языке? Что, на украинском это будет более
информативно или более развёрнуто? Так нет!» — обурилася дівчина в наступній
розповіді. 

«Материалы, да, я брала для этой работы, вообще на английском языке были, я их
переводила. И щас я должна писать на украинском: ну так, х\%й сосать — не в рот
е\#\%ться, ребят! Типа нравится украинский язык, — переводите, пожалуйста.» 

Відтак Маргарита відразу ж заявила про те, що українська мова ґвалтує її
психіку й нерви, водночас бажаючи переписати курсову роботу навіть англійською,
аби не українською.

«Почему вы насилуете меня, мою психику и мои нервы? Я не хочу, я проще на
английском её напишу, ещё раз переведу на английский и напишу её на английском,
чем на украинском!» — скандально поскаржилася студентка. 

Керівник Олексій Дейкун: Студентка українського вишу з українською мовою освіти
в Україні відкрито зневажає українську ж мову, вдається до найвульгарніших
порівнянь і заявляє про «мовне зґвалтування»! Безсумнівно, більшого,
нахабнішого та одіознішого глузування з нашої мови українське суспільство ще не
чуло. Від імени нашої спільноти закликаємо всі компетентні органи, ЗМІ,
активістів і небайдужих громадян негайно відреагувати на ці антиукраїнські
заяви Маргарити.

КНП ХОР \enquote{Центр екстреної медичної допомоги та медицини катастроф},
Уповноважений із захисту державної мови, просимо відреагувати на ці
антиукраїнські заяви!
