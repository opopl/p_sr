%%beginhead 
 
%%file 09_01_2023.fb.dedova_natalia.mariupol.1.dyakuyu_l_karts__ole
%%parent 09_01_2023
 
%%url https://www.facebook.com/permalink.php?story_fbid=pfbid02CknBnsn9Sx3z2iCCyiHhddbDiFJD1kXNnJJ2ivon2u2aBUN69YcXQ9t7ga7R3xjil&id=100007662284921
 
%%author_id dedova_natalia.mariupol
%%date 09_01_2023
 
%%tags mariupol,teatr.mariupol.drama,mariupol.golosy_myrnyh,__mar_2022.mariupol.dramteatr.udar,16_03_2022
%%title Дякую лікарці Олені Матюшиної за розмову, за рятування в театрі сотен маріупольців
 
%%endhead 

\subsection{Дякую лікарці Олені Матюшиної за розмову, за рятування в театрі сотен маріупольців}
\label{sec:09_01_2023.fb.dedova_natalia.mariupol.1.dyakuyu_l_karts__ole}

\Purl{https://www.facebook.com/permalink.php?story_fbid=pfbid02CknBnsn9Sx3z2iCCyiHhddbDiFJD1kXNnJJ2ivon2u2aBUN69YcXQ9t7ga7R3xjil&id=100007662284921}
\ifcmt
 author_begin
   author_id dedova_natalia.mariupol
 author_end
\fi

\#голосимирних
\#театр
\#авіабомба

❣️ Дякую лікарці Олені Матюшиної за розмову, за рятування в театрі сотен
маріупольців, за опіку над сиротою Сашком, чия матуся загинула на польовій
кухні 16 березня під час авіаудару. Наші - справжні!Послухайте, що і як
розповідає пані Олена. 

▪️Місяць я відходила від шокового стану і тиждень просто плакала. Забрати таку
кількість людей ... 

▪️1/3 або 1/2 міста загинула. Як порахуєш ті братські могили? Десь 200 тисяч. 

▪️Мкр Східний. Приватний будинок. В голові дві дати 24 лютого та 16 березня,
коли обстріляли театр. 

▪️Лежала під ковдрою на підлозі. Цілу добу обстріли. Через три доби попередили,
залишатися тут небезпечно. Йде руйнування та знищення людей. 

\ii{09_01_2023.fb.dedova_natalia.mariupol.1.dyakuyu_l_karts__ole.video.1}

▪️Черьомушки. Мама та донька там. + Я та чоловік. Страшний обстріл. Бомба
потрапила в сусідній під'їзд. 5 поверхів знищено. Сильна пожежа. Декілька машин
гасили пожежу кілька днів. З моря нас будуть обстрілювати. Будинки палали, як
сірники. Ти не можеш зрозуміти, куди прилетить. 

▪️Будинок розстрілюють, перебрались на роботу чоловіка до театру. Забрала
доньку. Мама відмовилася. 2 тижня до 16 березня ми там були. 

Познайомилися з Евгения Забогонская-Афендикова Сергей Забогонский . В театрі -
півтори тисячі людей. Опалення - немає. Люди мирні - похилого віку, багато з
дітьми... Лежали на підлозі. На тому, у кого що було. Коли я побачила цю
картину, жахнулась. Я лікарка. І якщо зараз не почати надавати медичну
допомогу, люди почнуть помирати. Чоловік організував польову кухню. 

▪️Йдеш коридорами театру і дивишся, як на кого не стати. 

▪️В мене 30 років стажу. Я лікарка загальної медичної реабілітації. 

▪️Людей заносили, виносили. Шокі, голодні обмороки, втрата крові, опіки. 

▪️Величезна подяка родині Забогонських. Список ліків. Вони знаходили. В мене
була необхідна аптека. 

▪️На прийомі була дуже велика кількість дітей від 2 місяців. Холодно. Бронхіти,
пневмонії, високі гарячки. Їжа пропадала, були поноси, рвоти. Нечиста вода,
неякісна їжа, відсутність людським умов. 

▪️Вперше в житті стикнулася з голодними обмороками. Люди не їли 5 діб. 

▪️Гострі стани. Осколки, перебиті судини, фонтанувала кров.  

▪️Привозили із зруйнованих будинків. Бомба потрапила в будинок двох пенсіонерів.
Вона при бомбардуванні впала з 8 на 6 поверх, зламавши руки. А її чоловік
згорів на 8 поверсі. 

▪️Психічний стан людей. Декілька вагітних. Відсутність води, тепла, загрози.
Були транквілізатори та заспокійливе. З психічними розладами зверталося дуже
багато людей. 

▪️Потім люди з пораненнями. 15 років, дівчинка супроводжувала пацієнта в 2
лікарню. Гостре поранення черевної порожнини. Вона була з нами після операції.
Перев'язки, антибіотики. Вибору не було. 

▪️Скільки пройшло людей, не можу уявити. Неможливо було записати. На 360
градусів обслуговували всіх. Медичний пункт на 1 поверсі. Я та дві медсестри.
При нам летальних випадків не було. 

▪️16 березня. 2 поверх. 10/15 хвилин відпочивала. Піднялася до себе. Не
пам'ятаю, коли їла. Вікно було вибито в кімнаті. Спали в одязі. Донька
заколачувала вікно. Пролунав страшний вибух. Я опинилася на підлозі. Чи жива
моя дитина? Половини кімнаті немає. Стіни, стеля - купа будівельного сміття.
Кричали люди. Пожежа.

▪️Прийшов знайомий доньки. Кров, голова. Схопила перекис. Займалася ним. Через
груду переліз хлопець. \enquote{Це бомба. Загроза завалів, пожежа}. 

▪️Олена з сином Сашком жила поблизу нас. Вона загинула на польовій кухні. Де
дитина? Сашко живий. Я забрала хлопця, ми взяли його під опіку. Крім мами в
нього - нікого. І зараз Сашко з нами у Франції. 

▪️Кухня і далі - дуже багато вбитих і поранених.

▪️ Дісталися на Черьомушки. Забрала дітей, маму, цього пораненого хлопця.
Виїхали. 

▪️Блокпости. Бачила зеків з наколками. Соціальні відходи. До 15 чоловік їх. До
трусів роздягали при -10. Пораненого хотіли забрати. 

▪️Три коти. Два загинули. Одного вивезли. 

▪️Люди ховалися у підвалах, кидали бомбу і двері закривали. Як можна порахувати
загиблих?  ❤️🩹 Повне інтерв'ю вже незабаром. 

🇺🇦 Друзі, якщо Ви готові розповісти свою історію, пишіть у приват або в
коментарях. Світ має почути кожного! 

Більше інтерв'ю 👇

\url{https://civilvoicesmuseum.org/}
