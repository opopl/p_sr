%%beginhead 
 
%%file 23_04_2023.fb.bludsha_maryna.kyiv.1.423_teplyj_sonechnyj_den_cvyntar_bajkove
%%parent 23_04_2023
 
%%url https://www.facebook.com/maryna.bludsha/posts/pfbid05Te3dRafgMAjqX42TnjTgAsXM5tWbkPPcHsN1ucUaG5CbjkCS47wBoidQwhLu255l
 
%%author_id bludsha_maryna.kyiv
%%date 23_04_2023
 
%%tags kiev,kiev.cemetery.bajkove,paton_boris,paton_evgen,glushkov_viktor.kibernetik,lesja_ukrainka,melnychenko_volodymyr.hudozhnik.skulptor.stina_pamjati,kiev.skulptura.stina_pamjati,kultura
%%title 423 дні Сьогодні був дуже теплий і сонячний день на цвинтарі
 
%%endhead 

\subsection{423 дні Сьогодні був дуже теплий і сонячний день на цвинтарі}
\label{sec:23_04_2023.fb.bludsha_maryna.kyiv.1.423_teplyj_sonechnyj_den_cvyntar_bajkove}

\Purl{https://www.facebook.com/maryna.bludsha/posts/pfbid05Te3dRafgMAjqX42TnjTgAsXM5tWbkPPcHsN1ucUaG5CbjkCS47wBoidQwhLu255l}
\ifcmt
 author_begin
   author_id bludsha_maryna.kyiv
 author_end
\fi

423 дні

Сьогодні був дуже теплий і сонячний день на цвинтарі. Ходила з подругою на
Байкове кладовище прибирати могилу її бабусі. Страшно, якщо чесно, писати тут
про це, бо розумію, що зараз для багатьох це дуже болісна тема і свіжі рани.
Для мене натомість кладовище поки що, як і раніше, - це дуже приємний досвід. Я
люблю бувати там, тож залюбки йду на цвинтар за найменшої нагоди, у якому б
місті, селі чи країні не була. Там мені дуже спокійно і водночас дуже цікаво,
бо навколо сотні людей та їхніх історій. А сьогодні було ще багато сонця і
зелені, бо Байкове - це старе кладовище, заросле деревами, кущами і квітами. І
цікавих запитань від Діми, онука моєї подруги, якому 5 років.

- Чому всі ці люди тут лежать?

- Для чого тут їхні фотографії?

- Чому тут не можна шуміти?

Ми прибирали сімейну могилу, де поховані Дімина прабабуся і родина її брата.
Дружина брата не бронювала собі місце під пам'ятник, тому її фото в звичайній
рамці обмотали файлами, щоб не промокло, і приклеїли до скла в рамці, на якому
вже написали епітафію. Вода все одно просочилась у поліетилен, тож поки ми не
звільнили рамку від файлів, роздивитися фото було неможливо. "Їй було 45 років,
коли помер чоловік. Хто тоді думає про смерть? Віра планувала жити вічно.
Вродлива була до остовпіння", - розповідає моя подруга. Віра померла, коли їй
було вже за 90. Зі світлини в рамці на мене дивиться статна жінка вже в
поважному віці серед клумб. На моєму фото Діма тримає цю рамку в руках,
спираючись на пам'ятник прабабусі Зіни.

Цікаво, що в нашій традиції поховання майже всі ми можемо розраховувати на
пам'ятник посмертно. І наша цінність для тих, хто нас ховатиме, буде виражена в
короткій посвяті, без трудового стажу і переліку наших важливих проєктів.
Напевно, тому, що нас ховають родичі, а не колеги. Хоча, звісно, є винятки -
письменники, народні артисти, державні діячі та військові, чиїх могил багато на
Байковому. На могилі Діминої прабабусі Зінаїди Михайлівни ця посвята така:
"Єдиній і незабутній мамі й бабусі". І вона справді незабутня - я чула стільки
історій про неї від подруги, хоча їй уже за 60 і вона сама бабуся. І хоча
бабусі Зіни немає на цьому світі вже понад 30 років, вона жива в серцях своєї
доньки, онук і правнуків.

- Я й Дімі так кажу, коли він питає, чи помре. Кажу, що він завжди житиме в
моєму серці, а я в його.

Бережіть себе та Україну.

Доброї нам усім ночі.

%\ii{23_04_2023.fb.bludsha_maryna.kyiv.1.423_teplyj_sonechnyj_den_cvyntar_bajkove.cmt}
