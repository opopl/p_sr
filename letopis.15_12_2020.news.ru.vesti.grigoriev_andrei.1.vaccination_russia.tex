% vim: keymap=russian-jcukenwin
%%beginhead 
 
%%file 15_12_2020.news.ru.vesti.grigoriev_andrei.1.vaccination_russia
%%parent 15_12_2020
 
%%url https://www.vesti.ru/article/2499221
 
%%author Григорьев, Андрей
%%author_id grigoriev_andrei
%%author_url 
 
%%tags vaccination,covid,russia,vaccine
%%title Новый этап борьбы с коронавирусом: старт вакцинации в России
 
%%endhead 
 
\subsection{Новый этап борьбы с коронавирусом: старт вакцинации в России}
\label{sec:15_12_2020.news.ru.vesti.grigoriev_andrei.1.vaccination_russia}
\Purl{https://www.vesti.ru/article/2499221}
\ifcmt
	author_begin
   author_id grigoriev_andrei
	author_end
\fi

\index[rus]{Коронавирус!Вакцинация!Россия, 15.12.2020}

\ifcmt
  pic https://cdn-st1.rtr-vesti.ru/vh/pictures/xw/307/975/9.jpg
  width 0.5
\fi

\begin{leftbar}
	\begingroup
		\em Более миллиона россиян лечатся сейчас от коронавирусной инфекции, из них 80
процентов амбулаторно. Чтобы обеспечить бесплатными лекарствами тех, кто болеет
дома, правительство выделит дополнительно почти три миллиарда рублей.
Об этом во вторник, 15 декабря, сообщил Михаил Мишустин на заседании
координационного совета по борьбе с коронавирусом.

Минздрав объявил о старте вакцинации по всей стране. Почти 30 тысяч человек уже
успели сделать прививку. Потребности регионов в вакцине и ее поставки
будут тщательно контролировать. График составлен на ближайшие три
месяца.
	\endgroup
\end{leftbar}

Новый этап борьбы с коронавирусом — в правительстве теперь ежедневно фиксируют
не только количество заразившихся, но и ведут учет тех, кто получил вакцину.

"Важно обеспечить, конечно, ритмичные поставки вакцины в регионы, важно
также оперативно распределять их. В пятницу я поручал главам регионов
обеспечить работу по пунктам вакцинации во время январских праздников.
Организацию этого процесса, конечно, необходимо держать на личном
контроле", — отметил Михаил Мишустин.

Масштабная вакцинация идет уже во всех российских регионах. Каждый фельдшер
знает, что ампулы хватает на пятерых, а перед уколом "Спутник V" нужно
обязательно смешать, но не взбалтывать

Температурный режим (хранить препарат нужно при минус восемнадцати) накрепко
усвоен не только врачами, но и транспортными службами. Третьей части
медицинских учреждений страны пришлось закупать специальное холодильное
оборудование.

"Во всех регионах развернуто в общей сложности 89 медицинских организаций,
которые принимают оптовые партии вакцины внутреннего распределения в
регионе. Общее количество медицинских организаций, которые на сегодня
задействованы в проведении вакцинации, 1229", — рассказал министр
здравоохранения Михаил Мурашко.

Чем больше в России прививочных пунктов, тем ближе коллективный иммунитет.
Главврач городской больницы № 1 города Курска Ольга Тутова рассказала, что
каждый вакцинированный вносится в реестр вакцинированного населения. В нем уже
28,5 тысячи фамилий. Процедура является добровольной.

В Калининградской области до середины января будут работать только продуктовые
магазины, а в Краснодарском крае считают, что пока поводов отказываться от
зимнего туристического сезона нет. Канатные дороги ежедневно дезинфицируют, в
общественном транспорте осуществляются проверки масочного режима.

"Аэрофлот" обещает на своих рейсах тех, кто не наденет маску, отсаживать от
всех остальных. По прилету таким нарушителям грозит штраф. Пресс-секретарь
авиакомпании Юлия Спивакова пояснила, что такая практика является общемировой
и, разумеется, вынужденной.

За рассадкой теперь следят и в Госдуме. Все потому, что среди депутатов опять
больше десятка заболевших, причем пятеро заразились повторно. В причинах
пытаются разобраться ученые.

"Да, действительно, случаи повторные могут быть, но это крайне редко", —
прокомментировал главный внештатный специалист Минздрава России по инфекционным
болезням Владимир Чуланов.

При этом в целом по стране показатели заболеваемости постепенно выравниваются.

"Я бы хотела отметить, что достигнутая стабилизация — она достаточно хрупкая.
Сегодня от каждого из нас зависит соблюдение требований, которые позволят
закрепить результат и идти дальше к снижению этих цифр", — заявила руководитель
Роспотребнадзора, главный государственный санитарный врач России Анна Попова.

В Бурятии открылись для планового лечения поликлиники, которые раньше были
загружены исключительно ковидными больными. В Оренбурге и вовсе возобновили
плановые хирургические операции.

Все больше пациентов переносят коронавирус на дому, но и для них предусмотрена
помощь от государства.

"Правительство продлевает выдачу бесплатных лекарств людям, которые
заболели коронавирусом и лечатся дома под наблюдением врачей. Сейчас таких
пациентов большинство. Правительство уже выделяло субъектам России более 5
миллиардов рублей на закупку необходимых препаратов. Дополнительно направим
регионам еще почти 2,8 миллиарда рублей. Такое решение уже принято.
Федеральные средства позволят обеспечить нужными лекарствами еще не менее
350 тысяч человек с коронавирусом", — отметил Мишустин.

Доставлять наборы медикаментов по квартирам будут добровольцы, которые
занимаются этим с весны. Волонтерский корпус с каждым месяцем только растет.
