% vim: keymap=russian-jcukenwin
%%beginhead 
 
%%file 03_12_2020.fb.donbass.1.prosypajsja_narod
%%parent 03_12_2020
 
%%url https://www.facebook.com/groups/546902252129720/permalink/1876284625858136/
 
%%author 
%%author_id 
%%author_url 
 
%%tags 
%%title Просыпайся, Великий Народ!
 
%%endhead 
 
\subsection{Просыпайся, Великий Народ!}
\label{sec:03_12_2020.fb.donbass.1.prosypajsja_narod}
\Purl{https://www.facebook.com/groups/546902252129720/permalink/1876284625858136/}

\enquote{Моя хата с краю...} - это выражение приписывают украинцам как характеристику
их пассивности, трусости и разобщённости. Однако выражение это злонамеренно
обрезано, потому что имело продолжение: \enquote{... Первым ворога встречаю!}, что
полностью меняет весь смысл. Украина - у края Руси - первой атакуется врагами.

Поэтому украинцам (впрочем, как и россиянам с белорусами) тщательно обрезается
связь с общими для всех русских людей корнями, ибо другая (тоже урезанная
поговорка) целиком звучит так: \enquote{И один в поле воин, если он по-русски скроен}.
Сознание украинцев подверглось мощной атаке, направленной на искоренение всего
русского, чтобы сдали добровольно родину свою. Сегодня многие из моих сограждан
скорее признают себя марсианами, чем русскими. Но... вторая поговорка даёт нам
уверенность, что и на этот раз мы одолеем врага. Просыпайся, Великий Народ! 

\href{https://www.youtube.com/watch?v=JCMNTGN-kZ8&feature=youtu.be&fbclid=IwAR2ndmSH5vgjsy0N7Vbft8uqBuxCT0-piJHinaY17WmbsxXUSdfLme7_Tjs}{Просыпайся, Великий Народ!}

\ifcmt
pic https://external.fiev6-1.fna.fbcdn.net/safe_image.php?d=AQBOLg-LH2Xxgi5E&w=500&h=261&url=https%3A%2F%2Fi.ytimg.com%2Fvi%2FJCMNTGN-kZ8%2Fmaxresdefault.jpg&cfs=1&ext=jpg&_nc_cb=1&_nc_hash=AQD2084hC1t3b-Go
\fi

