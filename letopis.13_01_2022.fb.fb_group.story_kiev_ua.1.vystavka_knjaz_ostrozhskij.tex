% vim: keymap=russian-jcukenwin
%%beginhead 
 
%%file 13_01_2022.fb.fb_group.story_kiev_ua.1.vystavka_knjaz_ostrozhskij
%%parent 13_01_2022
 
%%url https://www.facebook.com/groups/story.kiev.ua/posts/1839150412948418
 
%%author_id fb_group.story_kiev_ua,poluektov_igor
%%date 
 
%%tags istoria,kiev,knjaz_ostrozhskij.konstantin,lavra,vystavka
%%title Виставка про володаря всієї Русі, що московитів переміг
 
%%endhead 
 
\subsection{Виставка про володаря всієї Русі, що московитів переміг}
\label{sec:13_01_2022.fb.fb_group.story_kiev_ua.1.vystavka_knjaz_ostrozhskij}
 
\Purl{https://www.facebook.com/groups/story.kiev.ua/posts/1839150412948418}
\ifcmt
 author_begin
   author_id fb_group.story_kiev_ua,poluektov_igor
 author_end
\fi

Виставка про володаря всієї Русі, що московитів переміг.

Про поховання в Успенському соборі Києво-Печерської лаври.

Прямо зараз в Лаврі відкрита виставка, яка присвячена князям Острозьким і яку
ви не можете пропустити. 

Там ще багато цікавого (наприклад, головний скарб української історії -
оригінал Золотої пекторалі з Товстої Могили), але про все по порядку.

В минулому році я не лише досліджував родовід Острозьких, писав про них кілька
постів, але і подорожував деякими містами їх слави.

\ii{13_01_2022.fb.fb_group.story_kiev_ua.1.vystavka_knjaz_ostrozhskij.pic.1}

В Дубно дивився і знімав одне з 7 чудес України - один з найстаріших замків
країни. А в Острозі не лише засновану ними академію, замок, церкви, вежі, але і
шикарний пам’ятник. Який символізує споконвічні українські принципи
державності: \enquote{Армія, Мова, Віра}. Адже на ньому фігури трьох видатних
представників роду:

1. великий гетьман і переможець московитів та інших орд Костянтин Іванович (з
мечем), 

2. засновник академії, фундатор книгодрукування, розвитку мови та наук
Василь-Костянтин (з біблією) 

3. визнаний святим Федір (в лаврі покоїться як преподобний Феодосій) (у схимі).

І яка ж радість була у мене, коли дізнався, що і в мій рідний Київ повертається
пам’ять про князів Острозьких. В місце, де вже майже 500 років спочиває великий
полководець ХVI ст. Костянтин Іванович - в Печерську лавру.

% 4-6
\ii{13_01_2022.fb.fb_group.story_kiev_ua.1.vystavka_knjaz_ostrozhskij.pic.2}

В Успенському соборі можна побачити шикарний, європейського зразка (яким і був
Київ п’ять століть тому) надгробок. Над місцем, де було поховання князя
Костянтина Івановича Острозького. 

Він і розвивав Київ європейським містом, а відповідно і традиція його поховання
була такою, як і інших монархів в Європі. Майже чотири сотні років вишуканий
надгробок був головною окрасою і гордістю Успенського собору, нагадував про
видатний спадок і героїчну долю цього героя нашої історії. Доки не сталась
біда, не прийшли зайди з півночі.

% 7-9
\ii{13_01_2022.fb.fb_group.story_kiev_ua.1.vystavka_knjaz_ostrozhskij.pic.3}

Цей спадок всіляко намагались знищити і піддати забуттю: за часів імперії
накривали балдахіном, приховували, а за совка взагалі підірвали разом із
собором. 

% 10-12
\ii{13_01_2022.fb.fb_group.story_kiev_ua.1.vystavka_knjaz_ostrozhskij.pic.4}

Бо не могли стерпіти ту \enquote{золоту добу} української історії, коли тут
розвивались науки, відкривались перші у східній Європі навчальні заклади, було
започатковане книгодрукування, відновлювались пам’ятки, було організовано
шикарне військо, що і татар і московитів розбивало, коли правили на Русі
Костянтин і Василь Острозькі.  

Хоча і зараз, з певних політичних мотивів, за чіткою командою з Москви, хтось
дуже не хоче, щоб кияни знали про те, що надгробок видатного українського князя
відновлений. Про це ви не знайдете жодних вказівників чи написів у Лаврі, про
це не скаже касир, про це може знати лише той, хто знає історію (або читає мої
пости :)). Більше того, прохід до цієї частини церкви наче ненавмисне
захаращений церковним безладдям (на фото). 

\raggedcolumns
\begin{multicols}{2} % {
\setlength{\parindent}{0pt}

\ii{13_01_2022.fb.fb_group.story_kiev_ua.1.vystavka_knjaz_ostrozhskij.pic.13}
\ii{13_01_2022.fb.fb_group.story_kiev_ua.1.vystavka_knjaz_ostrozhskij.pic.13.cmt}

\ii{13_01_2022.fb.fb_group.story_kiev_ua.1.vystavka_knjaz_ostrozhskij.pic.14}

\ii{13_01_2022.fb.fb_group.story_kiev_ua.1.vystavka_knjaz_ostrozhskij.pic.15}

\ii{13_01_2022.fb.fb_group.story_kiev_ua.1.vystavka_knjaz_ostrozhskij.pic.16}
\ii{13_01_2022.fb.fb_group.story_kiev_ua.1.vystavka_knjaz_ostrozhskij.pic.17}

\end{multicols} % }

Але ніщо не зупинить ідеї, час якої настав  @igg{fbicon.wink}  

Брехня завжди програє. Правда перемагає. Справжня історія повертається!

Острозький знову в Києві, на своєму законному місці, в святині до збереження та
відновлення якої доклався і він і його син. 

Є щось символічне і глибоке в тому, що найвидатніші герої нашої історії, такі
як К.Острозький, К-В.Острозький, Іван Мазепа і інші, яких за наказом з кремля
тривалий час оббрехували, хаяли та навіть когось і проклинали в стінах цього
храму, тепер повертаються на своє законне місце. Повертаються як переможці!

Зверхньо дивляться зі стінописів величні постаті цих українських героїв на
нікчемних брехунів в рясах, що прикриваючись вірою вчиняли беззаконня,
безрассудство, що намагались знецінити і піддати забуттю сотні років нашої
історії. Історії, яка була набагато краща і цікавіша, ніж все, що було за часів
кремлівської окупації. Саме тому так важливо для них було приховувати той
період. Який тривав з часів перемоги над Ордою в 1320-х роках, коли ми
звільнили Київ, і аж до приходу сюди окупантів з півночі майже п’ять століть
потому. Бо і тривалість того періоду, і порівняння якості життя та важливості
доби були завжди не на користь кремлівців.

.

Це був коротенький вступ, а тепер дам цитати, які краще за мене розкриють
масштаб і велич особистості некоронованого короля Руси-України, якого в XVI ст.
поважала вся Європа.

Відома дослідниця шляхти, д.і.н. Наталя Яковенко:

\enquote{Костянтин Острозький став людиною, яка опинились в потрібний час в потрібному
місці. Можна сказати, що він посідав в ієрархії Великого Князівства Литовського
другий, або навіть перший уряд}.

Дослідник родоводу Острозьких, к.і.н. Ігор Тесленко:

\enquote{Костянтин Іванович мав одну з ключових посад. І, як видно з історичних джерел,
Він мав великий вплив на королів. Як свідчать шляхтичі Волинські, Київські,
Брацлавські, які жили на початку XVI ст.: під Костянтином Івановичем ходила вся
Русь}.

Історія Польська Мартіна Кромера:

\enquote{І щоби знати, слід почати з одвічної проблеми – Тартарії, та трьох армій під
командуванням володаря Константина Острозького, князя народу битв, істино,
найдостойнішого у пам’яті всіх віків, що зупинив напади гунів, знищив їх}.

За свідченнями провідної литовської дослідниці, д.і.н. Генуте Кіркенє: 

\enquote{Враховуючи руське (українське) походження князя, отримання Острозьким
постійного призначення на командування було великим визнання його як
військового діяча і талановитого воєначальника, його здібностей}.

В дусі епохи, не завжди об’єктивно оцінював життєвий шлях Острозьких П.Куліш,
але навіть він визнавав, що: \enquote{Національна стійкість його значила багато для
нашої Руси, уже захитаної в своїй національності}.

Іван Огієнко у монографії \enquote{Князь Костянтин Острозький і його культурна праця}
пише, що: 

\enquote{Він мав помітні бої з татарами, волохами та московцями і всі їх вигравав}.

Представник Папи Римського, апостольський легат Якоб Пізо після переможної
війни проти московитів писав в 1514 році: 

\enquote{Князь Костянтин може бути названий кращим воєначальником нашого часу, він 33
рази ставав переможцем на полі битви… в бою він не поступається хоробрістю
Ромулу}. 

Королівський секретар Йодок Людвіг Децій в реляції 1527 року, що вийшла в
Нюрнберзі, досить докладно описав дії Острозького в одній із знаменних битв вже
немолодого гетьмана, назвав того:

\enquote{воєначальником, що талантом своїм перевершив всіх вождів світу}.

Костянтина Івановича Острозького називали другим Ганнібалом, Пірром і Сципіоном
руським. 

А як відомо, і як ми бачимо з тієж реляції Я. Пізо, або синхронної їй від Яна
Лаского (\enquote{Про народ рутенів}), тоді цілком і точно розділяли рутенів (русинів
або українців) і московитів (яких ще називали москами або мосхами),
протиставляючи ці два різні народи (які багато років воювали між собою). І на
той час ніхто не називав московитів русинами, русами, руськими, рутенами тощо.
Це було б так само, якщо б московитів хтось почав називати українцями. Ця
химерна термінологічна помилка буде примусово нав’язана століттями пізніше. Про
це я вже наводив певно з сотню іноземних джерел і цей факт більше не потребує
зайвого доведення.

Думаю цього достатньо, щоб зрозуміти, що перед нами не просто герой свого часу,
а одна із головних постатей української історії.

.

Про його військовий досвід:

Петро Кулаковський:

\enquote{Перший військовий досвід протистояння татарам Костянтин Острозький здобув у
1480-х рр., коли брав участь у виправах свого батька. Без сумніву, молодий
Костянтин взяв участь у відбитті татарського нападу 1491 р., під час якого
кочівники спалили Печерську церкву в Києві, Пречистенський собор у Володимирі,
спустошили Галич. Вирішальна битва відбулася під Заплавом, де за керівництва
князя Семена Юрійовича Гольшанського християни отримали перемогу}.

Вже з середини 1490-х рр. він почав відігравати одну з ключових ролей на Волині
щодо організації відсічі татарам. Острозький уславився не лише оборонними
діями, в 1497 році очолив успішний похід під Очаків, ще ряд авмібних виправ. А
потому отримав призначення Великим Гетьманом Литовським і Руським.

«Татарські напади повторювалися тоді щорічно. Наприклад, коли в 1508 р.
К.Острозькому довелося протистояти татарській загрозі, то за свідченням Марціна
Бельського і Алессандро Гваньїні, вони вдерлися на саму Волинь. В організації
опору взяли участь як частини під керівництвом К.Острозького, так і козацькі
загони під керівництвом козака Полоуса. Діяли вони автономно – частину татар
розбив гетьман, частину – козаки. Відбулося це вже під час повернення татар з
Волині, оскільки А.Гваньїні повідомляє, що і гетьманські підрозділи, і козацькі
частини відняли у татар здобич, тобто награбоване майно. Густинський літопис
зазначає, що К.Острозький розбив татар аж під Слуцьком» - там же. Географія
битв вражає. А ще і окреслює межі володінь Русі того часу.

Мав ще багато видатних звитяг, але набув загальноєвропейської слави після
блискучої перемоги над московським військом під Оршею 8 вересня 1514 року. Коли
московити загрожували поглинути не лише більшу частину Литви і Русі, але і
захопити столицю - Вільно, а разом з цим і Київ, бажаючи вкрасти у нас майже
два століття розвитку. 

Це найвідоміша перемога, бо керовані ним війська в кілька разів поступались за
чисельністю, але за рахунок геніальності полководця, розгромили московську
армію та взяли в полон величезну кількість бояр та воєначальників. Майже на
сторіччя визначила статус-кво на кордонах Московської і Литовсько-Руської
держав. В грудні 1514 року князь К. Острозький увійшов у Вільно як видатний
переможець. Його зустрічали триумфально, мов римських імператорів. 

Це була нехай і дуже важлива, але лише одна з великих перемог Гетьмана.
Наприклад, в 1527 році він здобуде не менш геніальну над татарами під
Ольшаницею. Після чого вже відкриватиме ногою парадні ворота Кракова, де його
чекатиме тріумфальний в’їзд і королівський прийом з найбільшими почестями.

.

«Унікальна особливість Костянтина Івановича як фундатора та мецената полягала у
тому, що він зводив та матеріально підтримував храми поза межами своїх
володінь. Саме як великий політик ця персона дуже цікава, але мало досліджена.
Відомо, що князь  мав особливу повагу з боку короля та привілей — червону
печатку. Та часто перебував на монаршому дворі у позавоєнний час. У цієї людини
були серйозні амбіції як одного з лідерів Руського народу у Великому князівстві
Литовському», - свідчить Ігор Тесленко.

Один з моїх улюблених істориків, к.і.н. О.Алфьоров додає і резюмує:

\enquote{Князь Костянтин — це передусім полководець, який вигравав битви проти
московитів. І це дуже ріже очі тим, хто сьогодні не бажає бачити жодної
української перемоги. І, безперечно, ті, хто не хоче бачити перемоги українців
— будь-де, хоч у політиці, хоч на військовій арені — ті не хочуть бачити й
Костянтина Острозького як постать, яка має бути наслідувана. А його називали
«Імператором». Імператор — не в сенсі монарх, а як в античній історії —
«відомий полководець, переможець». Він, безумовно, був одним із найбільш
блискучих стратегів в історії України. Це саме та постать, яку ми маємо
наслідувати у справах військових, шляхетських, меценатських, політичних}.

З великими почестями похований в Успенському соборі Києво-Печерського
монастиря. Його надгробок – один з шедеврів українського ренесансу, після
окупації Києва московитами, спочатку завішувався балдахіном, а пізніше загинув
під час вибуху собору у 1941 р. 

Але нарешті, попри весь спротив Москви, сьогодні цей український шедевр,
надгробок видатного князя, відновлено в тому самому місці, де знаходився він до
того трагічного дня, де 400 років до того спочивав видатний герой української
історії.

І саме тому, кожен, хто називає себе українцем, має піти, низенько вклонитись і
покласти квіти біля тепер вже, можна так говорити, пам’ятника К.І.Острозькому
(першої частини відтворення, бо будуть ще і шикарні шати на всю стіну, як
колись).

Хоча знаходиться він по ліву руку від центрального входу, але так як той
закритий, то зайшовши через ліві (відкриті) двері храму, ви тримаєтесь постійно
правої сторони і знайдете так те, що шукаєте. 

Виставку ж, яка присвячена князям Острозьким, ви знайдете там же поруч, у
виставкових залах, за кілька метрів від входу в собор.

.

Епітафія в Успенському Соборі переписана Атанасієм Кальнофойським, 1638 р.:

\enquote{Здобувши над Москвою і татарами шістдесят три перемоги, кров'ю забарвлені -
Рось, Дніпро, Ольшанка, оздобив і заснував багато замків, багато монастирів,
багато святих церков, які в князівстві Острозькім і в столичному місті
Великого Князівства Литовського Вільні – вимурував. Другу Гетсиманію - Дім
Пречистої Діви Печерської щедро обдарував і у ньому після смерті був
покладений. Для немічних притулки, для дітей школи, для людей лицарських в
Академії Марсовій списи з шаблями залишив. Сципіонові руському Костянтину
Івановичу князю Острозькому, гетьману Великого Князівства Литовського}.

Епітафія в Успенському Соборі з рукописних збірників Софії Київської, XVIII
ст.:

\enquote{...благовірний і христолюбивий освічений і могутній князь Костянтин Іванович
Острозький, славний в усій Русі, син роду князів руських, воєвода Троцький,
гетьман Великого князівства Литовського цей відомий хоробрістю і подолав і
переміг за допомоги всесильного Бога супостатів. Він же з юних літ і до
старості з вірою служив, і немало великих битв виграв, всі 60, а великих Три:
на Білій Церкві, біля Орші (біля Дніпра), і Ольшаниці ріці...}.
