% vim: keymap=russian-jcukenwin
%%beginhead 
 
%%file topics.kiev.poezia.pestun_maksim.arhistratig_mihail
%%parent topics.kiev.poezia
 
%%url 
 
%%author_id 
%%date 
 
%%tags 
%%title 
 
%%endhead 
\url{https://stihi.ru/2020/06/01/6934}

Архистратиг Михаил
Максим Пестун
Когда осядет шум вчерашний,
И входит ночь в свои права
На город темный, город спящий
Свой дивный свет струит Луна

И старой улицей мощённой,
Прищурясь от витрин огня,
Бредёт Aрхангел возмущённый,
Но лишь небесный наш судья…

Здесь все чужое год за годом.
Да что там годы!..
Каждый день
Растут гостиницы-уроды,
Где старый сад дарил нам тень

И ТРЦ, как пятна туши,
На город выплеснулись вдруг,
Они воруют наши души
Разнообразием услуг

Рисунок города бледнеет,
Холмы святые не видны,
Зажат гранитом Днепр мелеет,
Покрыты пробками мосты

Поля, что раньше простирались
Насколько глаз мог охватить,
Застройке массовой поддались,
Как будто не бывало их…

Там раньше чащи и озера
Служили домом для зверья,
А нынче лишь немым укором
Деревья редкие торчат

И горизонт вокруг сереет
Однообразья мишурой,
И взгляд уставший не согреет
Природы дивною игрой

Лицо Архангела темнеет.
Рука сжимает твердь копья.
Дракон коррупции наглеет -
Ему и небо не судья!
