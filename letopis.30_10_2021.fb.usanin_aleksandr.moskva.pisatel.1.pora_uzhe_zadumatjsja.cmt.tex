% vim: keymap=russian-jcukenwin
%%beginhead 
 
%%file 30_10_2021.fb.usanin_aleksandr.moskva.pisatel.1.pora_uzhe_zadumatjsja.cmt
%%parent 30_10_2021.fb.usanin_aleksandr.moskva.pisatel.1.pora_uzhe_zadumatjsja
 
%%url 
 
%%author_id 
%%date 
 
%%tags 
%%title 
 
%%endhead 
\subsubsection{Коментарі}
\label{sec:30_10_2021.fb.usanin_aleksandr.moskva.pisatel.1.pora_uzhe_zadumatjsja.cmt}

\begin{itemize} % {
\iusr{Бхакта Игорь}

Зашёл в селе в продуктовый магазин... за молоком.. забрал последнюю пачку с
витрины.. при этом на полках больше ста наименований различных алкогольных
напитков.. в т. ч слабоалкогольных для так сказать начинающих... Витрины завалены
колбасой.. йогурта же одно наименование .. вот и вся картина маслом как
говориться .. не нужны ни цытаты ни аналитики ни предсказатели.. просто
посмотреть на ассортимент и все становится ясным и понятным куда все катится и
что ничего хорошего не следует из этого ожидать

\begin{itemize} % {
\iusr{Александр Усанин}
\textbf{Бхакта Игорь} Мир катится туда, куда его катят. Нам нужно катить его в правильном направлении.

\iusr{Андрей Мошненко-Высоцкий}
Хм.. В селе, в магазине водка, хм-м. Зачем в селе столько водки. Там есть свои самогонные аппараты и самогон. Сами выгнали и сами потребляют. Им паленка магазинная не зачем. Странный магазин и село.

\iusr{Gennady Kazachkov}
\textbf{Бхакта Игорь} йогурт и молоко быстро разбирают)
\end{itemize} % }

\iusr{Татьяна Ражная}

\ifcmt
  ig https://scontent-frx5-2.xx.fbcdn.net/v/t39.1997-6/p480x480/105941685_953860581742966_1572841152382279834_n.png?_nc_cat=1&ccb=1-5&_nc_sid=0572db&_nc_ohc=DkwTfYKMYwYAX-Jgl55&_nc_ht=scontent-frx5-2.xx&oh=e664bbb1aabeafa6660f78433b6922e0&oe=61A436CB
  @width 0.2
\fi

\iusr{Александр Ста́лив}
\textbf{Александр Усанин}  @igg{fbicon.face.smiling.eyes.smiling}  @igg{fbicon.heart.sparkling} 

Наши предки в веках и тысячелетиях не задумывались особо, и мы не задумываемся,
всем некогда, и тогда и ныне, не задумывались и не задумываемся в повседневной
потребительской текучке о том, кто такой и почему человек и что такое
человечность.

Человек - чутким челом век.

Что это значит, человек?

Вместилище чего человек?

Когда человек становится и является человеком?

Что входит в суть человечности?

Вот, что особо важно. Это.

Определённость человека. Определённость человечества. Определённость в пределах
величин (в параметрах) среды обитания, то есть Вселенной, Системы Вселенной
относительно Её средоточия, неотъемлемой и неотделимой составной частью и
принадлежностью Которой все мы являемся.

Подсистемый в Системе Вселенной, естественнонаучный, космополитический,
принципиальный (изначальный) подход, в котором принцип (по-русски начало) это
начало (принцип) Вселенной, Системы Вселенной, это средняя точка, середина,
центр, ядро, начало (принцип), средоточие Вселенной.

Отсюда следует осуществлять изначальный (принципиальный) отсчёт и расчёт
естествознания жизнеобеспечения населения земного шара, каждого человека
непосредственно (конкретно). Из определённости в пределах величин (параметров)
по их происхождению одних из других (по производным) относительно средоточия
Вселенной.

Это самый доскональный (идеальный) смысловой (логический) деловой подход.
Потому что так есть в действительности (реальности). Есть бессознательно и
безусловно, безоговорочно, непромысленно (объективно), вне воли всего живого
сущего, по воле Бога, то есть в соответствие (пропорционально) состоянию и
действию (механизму и механике) Вселенной, Системы Вселенной относительно Её
средоточия. Здесь мы - неотъемлемая и неотделимая составная часть и
принадлежность. Что особо важно знать и понимать, учитывать вовремя и к месту -
здесь и сейчас, сразу же, сходу, мгновенно и беспрепятственно, безотказно для
себя и для других, для всех. Этим ЖИТЬ. Жить и не тужить, добра и счастья
наживать любовью и процветать в доходе и достатке каждого и всех - во благости
благодати Божией - в том самом соответствии (в пропорции) состоянию и действию
(механизму и механике) среды обитания - Вселенной, Системы Вселенной
относительно Её средоточия.

Всему своё место в своё время - свойственность всеобщих и частных величин
(параметров) среды обитания по их происхождению одних из других (по
производным) относительно средоточия Вселенной.

Это главное правило (норма) жизни. Запомните.

Всему своё место в своё время. Свойственность.

(Окончание ниже, в комментарии).

\begin{itemize} % {
\iusr{Александр Ста́лив}
\textbf{Александр Усанин}  @igg{fbicon.heart.sparkling}  @igg{fbicon.face.smiling.eyes.smiling} 

(Окончание. Начало смотрите и читайте выше).

Особо замечу и подчеркну: Вселенная в свойственности вещей (материи) вечна и
бесконечна, настолько Она соразмерна и соответственна (пропорциональна),
согласованна и сообразна (синхронизирована и синхронна) сама в себе
относительно своего средоточия. И мы в Ней, ещё раз замечу и подчеркну, мы в
Ней - неотъемлемая и неотделимая составная часть и принадлежность.

И наше естественное предназначение во Вселенной, как и естественное
предназначение всего живого сущего в Ней, - ЖИТЬ. Жить и не тужить добра и
счастья наживать любовью и процветать в доходе и достатке каждого и всех - во
благости благодати Божией - в том самом чутком, осознанном и целеустремлённом,
ответственном, добровольном соответствии (в пропорции) Её, Вселенной, состоянию
и действию (механизму и механике).

Вот, что главное. В этом суть понятия и определения человек - чутким челом век,
человечность - чутким челом вечность. Соответствие (пропорция) среде обитания -
Вселенной, Системе Вселенной относительно Её средоточия.

Всё это я не читал. Всё это я пишу свой собственной жизнью, частным образом, в
пределах своего полномочия разрабатываю фундаментальную и прикладную науку по
общим наименованием "Определённость жизни" - естествознание жизнеобеспечения
населения земного шара - науку ЖИТЬ, жить и не тужить, добра и счастья наживать
любовью и процветать в доходе и достатке каждого и всех - во благости благодати
Божией, то есть в соответствии (в пропорции) состоянию и действию (механизму и
механике) среды обитания - Вселенной, Системы Вселенной относительно Её
средоточия, неотъемлемой и неотделимой составной частью и принадлежностью
Которой все мы являемся.

Имея естественнонаучную, подсистемную в Системе Вселенной определённость жизни,
мы, человечество и каждый человек, в состоянии по образцу Вселенной,
Вселенной-блокчейн (как говорится: "по образу и подобию Божиим") налаживать и
упорядочивать общение и отношения, и, стало быть, - свой быт, ведение
отраслевого хозяйствования (экономики) государства, всех государств земного
шара.

Вот о чём речь. Насколько это существенно и значимо, важно, необходимо в
достатке всем и каждому, Александр Евгеньевич.

Жизненности Вам, Вашим родным и близким людям.

Жизненности России и всему человечеству.

С особым почтением.

\iusr{Владимир Яковлев}
\textbf{Александр Ста́лив} в совокупности двух частей повествования, поддерживаю автора во взглядах в метафизическом сегменте диалектики и интерполяции гармонии общего и частного.
Это во многом корреспондируется с принципиальными подходами первой части, являющейся в целом, методологией Социал-патриотизма.

\iusr{Александр Ста́лив}
\textbf{Владимир Яковлев} Спасибо.
Просто в один комментарий по числу знаков не помещается.
С особым почтением.

\iusr{Александр Ста́лив}
\textbf{Владимир Яковлев} 

Нам необходим простой человеческий язык. Современный язык крайне сложный и не
воспринимается людьми.

И основа простоты языка - чувства, родные поместные самобытные испокон веков
слова, не заимствованные из иных языков.

Мы сами того не зная и не замечая в поиске "лучшей жизни" усложнили общение и
отношения, и в итоге в огромной степени утратили связь с природой, с
действительностью (реальностью) окружающей среды.

Ещё на рубеже восьмидесятых и девяностых годов я поставил перед собой задачу
разобраться в жизни, поделиться с другими своим пониманием её основ.

Что делал и делаю в продолжение многих десятилетий.

Родной язык это язык ощущений и чувств, предчувствия - язык природы. Он -
основа определённости человека и общества. Он заключает и несёт в себе
взаимосвязь прошлого, настоящего и будущего (нави, яви и прави).

Именно и только целеустремлённо познавая и постигая родной язык с подачи
общества, государства, мы сможем по-настоящему сблизиться и породниться, стать
народом. Счастливым народом.

Иначе, видим, кто мы, как относимся и общаемся, осуществляем хозяйствование
(экономику).

С особым почтением.
\end{itemize} % }

\iusr{Владимир Рыжков}

\ifcmt
  ig https://scontent-frx5-1.xx.fbcdn.net/v/t39.30808-6/248687262_126405146436955_7095056900824848194_n.jpg?_nc_cat=100&ccb=1-5&_nc_sid=dbeb18&_nc_ohc=uVaDenqbqjAAX8iTz6h&_nc_ht=scontent-frx5-1.xx&oh=9a2fe18c2baeb99c0f4eb3810a87d5e9&oe=61A3A294
  @width 0.4
\fi


\end{itemize} % }
