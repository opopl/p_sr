% vim: keymap=russian-jcukenwin
%%beginhead 
 
%%file letters.mariupol.bespalova_natalia
%%parent letters
 
%%url 
 
%%author_id 
%%date 
 
%%tags 
%%title 
 
%%endhead 

https://www.facebook.com/ivan.ivan.kyiv/posts/pfbid0Byx4dEDaz86yDnBCbS3Jdt9S1c8o872LQzntAvxyikGn1MgchAbT1wAEQJG7feRfl

18:12:49 05-05-23
Іван Іван
Наталія Беспалова 

дякую за Вашу відповідь! знаєте... все те, що Ви сказали, я вже знаю... я вже
тонни інформації прочитав про Маріуполь... і книжку Надії, і щоденники різних
людей, наприклад от щоденник Віктора Павловича Віктор Кіпчарський - і також
багато чого записав... в мене вже всього того реально сотні файлів і тисячі
сторінок тексту і фото на ноуті... одних тільки Кроликів з тої виставки в 2019
більше 200 фото і вони вже тута викладені, сукупно позбирав, я ж програміст...
крім того, я досліджую зв'язок тої виставки з виставкою на Софії в 2018 в
Києві... але... таких як я, відносно мало. Ну тобто. Людей, які ніколи не були
в Маріуполі, але прочитали тонну всього про Маріуполь, і блокадний, і мирний, і
культуру, і тих же писанок і кролів... Розумієте... Апріорі то мене ніщо не
змушує займатись саме Маріуполем, я це роблю за покликом серця і душі, бо я
бачу, яка страшна біда сталась... Що я хочу тут звернути увагу... Окремо ще
якось пізніше напишу по пунктах окремо... але... ось дивіться. Так, страшна
неймовірна трагедія, страшний злочин. Але. В Києві наприклад на звичайному
житєйському рівні треба дуже старатись, щоби бодай щось відкопати про
Маріуполь. Це дійсно так, бо я раніш багато разів бігав по книгарням і по
Петрівці... і реально нічого не було... Немає книжок, немає карт географічних,
немає путівників... Про все інше на світі - є до біса, а от про Маріуполь -
немає... От якимсь дивом собі карту одну Маріуполя дістав... Так! Їх - книжок
про Маріуполь - просто тупо немає, розумієте. Одна невеличка книжка Надії
Сухорукової чи Шишацького чи (ще один автор опублікував шоденник як книжку,
вона в мене є, але не з собою в селі, називається Прощание с Мариуполем 2022,
Дневник Прощания, скрін додаю нижче, я її собі купив, так) тут загальної
картини не змінює, хоча це дуже хороший крок вперед, тому що хоча б в деяких
книгарнях люди вже бачать слово "Маріуполь", хоча б і затиснуте поміж сотень
інших назв... Так от... Немає абсолютно книжок про мирний туристичний
Маріуполь, немає книжок - а тут потрібні томи, реально ТОМИ, розумієте,
ФОЛІАНТИ, товсті серйозні томи... на тисячі сторінок, про трагедію блокадного
Маріуполя, щоб усе разом було зафіксовано послідовно і чітко... бо якщо
загинуло 130 000, то бодай треба записати хоча б 1000 в одну книжку, для
початку... хто були ці люди, чим займались, за яких обставин загинули... Так, є
безліч фб-публікацій... але... поки вся та інформація на фейсбуку або
телеграмі... не збережена і не записана як Книжка, хоча б і в електронному виді
спочатку, вона не може слугувати надійним документальним свідоцтвом... а
чому... а тому що пости на фб зникають за мить, і навіть цілі аккаунти...
Розумієте, немає фактично книжок, ось так щоби ти прийшов в книгарню і без
проблем собі щось взяв про Маріуполь... А якщо є... їх треба реально
вишукувати... Ну а як наслідок, ніхто на масовому рівні через Маріуполь не
переймається, тому що власне кажучи, Маріуполя немає як слід в культурному
просторі наприклад того ж Києва, де я живу... і маємо-те-що-маємо. Навіть от
якщо взяти группу Маріуполь Довоєнний, хто там постить та коментує - власне в
основному тільки маріупольці і все. А де ж всі інші?! Ну добре... я поки тут
закінчу... а щодо книжок навіть цілий пост написав, ще у березні, ось...
https://www.facebook.com/ivan.ivan.kyiv/posts/pfbid02XCmNpqepMNpZFkrtdU8MW6SFzkntZRas6CLRRZxySzeUNWK8wQ3bLemtx3R2JmHHl


так... я знаю... тому і зробив такі от відосики... і пісню от підібрав...
але... справа тут не тільки в самій оцій пісні... тому що інші пісні теж варто
послухати... і я люблю слухати різні пісні... і думаю, позаписую ще... а справа
в тому... що я зараз Вдома. Вдома у себе в селі... І тут зараз немає війни,
немає шахедів, ППО і т.д. І ніхто у мене не забрав ні маму, ні тата, ні брата,
ні сестру, ні сімью мою... і квартира моя у повному порядку... і будиночок в
селі також... і роботу я не втратив... як працював... так і працюю в Києві...
але... я знаю... багато інших дуже-дуже постраждало... і багато інших просто
загинуло... Так... багато інших, зокрема і Ви... я знаю... пережили дуже-дуже
багато лиха. Неймовірно багато... я вже начитався всього... І... розумієте...
Це така собі моя спроба передати Вам та іншим от це відчуття Дому... розумієте.
От таке от тепле і просте відчуття, що ти Вдома... От знаєте, як кажуть -
Ласкаво просимо, відчувайте себе як Вдома... І вдома - оце от ключове слово
тут... Тому що... От я розумію... що у мене все начебто в порядку... але... у
багатьох інших далеко не все так добре... Я про це вже писав, що майже у всіх
нас не Всі Дома, точно от той діалог Аліси і Кота згадайте... І от. Я думаю...
напевне все це звучить якось дуже наївно, але... Ну короче, навіщо оті
відосіки... Неначе от... я запросив Вас до себе в гості... сказав...
приїжджайте в село... покажу вам город... тюльпанчики, там, ліс є неподалеку,
знаєте, як там чудово... Ось до цього до речі відео із серії Вітер Віє... А
щодо Вогню і Пічки і Пісень... ну... короче, Пічка в селі - це одна із моїх
улюблених штук тута. Ну тобто одна із таких моїх сокровенних штук, що ти
приїжджаєш після шумного Києва, в якому до біса людей і яке весь час кудись
дико мчить... в тихе спокійне Село-де-нічого-немає, можеш собі пічку запалити,
пічка тріщить, круто! Розумієте... немовби я запросив Вас у гості... ось ми
сіли разом біля пічки... розмовляєм... щось там обговорюєм... короче, це моя
така собі спроба поділитись частинкою свого Дому... І також... Вогонь... я
думаю, що Вогонь як стихія у багатьох визиває дуже негативні емоції через те,
що Вогонь зруйнував їх будинки, що Вогонь на війні вбиває людей і т.д... але
насправді Вогонь може бути також абсолютно мирним... як от в пічці Вдома,
розумієте... Так що тут справа не тільки в пісні самій по собі і в словах, а
також в тому... що начебто я запросив Вас до себе в гості... ми сіли коло
пічки... слухаємо музику... і немає ніякої війни... немає ні війни... ні росії,
ні путлера, нічого немає... окрім власне Пічки, що тріщить... та з якої
періодично вивалюється гаряче вугілля... так що його треба нашвидкоруч руками
кидати назад (гаряче, чорт забирай!), також... українських Пісень і ну власне
ласкавого домашнього Вогню, якось так...
