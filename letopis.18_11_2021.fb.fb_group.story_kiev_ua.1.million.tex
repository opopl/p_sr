% vim: keymap=russian-jcukenwin
%%beginhead 
 
%%file 18_11_2021.fb.fb_group.story_kiev_ua.1.million
%%parent 18_11_2021
 
%%url https://www.facebook.com/groups/story.kiev.ua/posts/1800365970160196
 
%%author_id fb_group.story_kiev_ua,fedjko_vladimir.kiev
%%date 
 
%%tags dengi,kiev,sssr
%%title Я бачив один мільйон рублів СРСР! Я доторкнувся до цього мільйона!
 
%%endhead 
 
\subsection{Я бачив один мільйон рублів СРСР! Я доторкнувся до цього мільйона!}
\label{sec:18_11_2021.fb.fb_group.story_kiev_ua.1.million}
 
\Purl{https://www.facebook.com/groups/story.kiev.ua/posts/1800365970160196}
\ifcmt
 author_begin
   author_id fb_group.story_kiev_ua,fedjko_vladimir.kiev
 author_end
\fi

Я бачив один мільйон рублів СРСР! 

Я доторкнувся до цього мільйона! 

Я порахував, на скільки б мені вистачило цих грошей, якби вони у мене були!

Відносно майбутньої професії – юрист – я визначився в армії, оскільки часу на
роздуми було досить багато — життя на три роки було чітко розписано: учбова
дивізія (військова спеціальність, погони молодшого сержанта) і подальша служба
в Республіці Куба. 

Повернувшись в Київ, поступив вчитися на вечірній факультет. А працювати мене
направили (через військкомат) в спецпідрозділ МВС по охороні державних установ,
в тому числі і відділень Держбанку СРСР.

\begin{multicols}{2} % {
\setlength{\parindent}{0pt}

\ii{18_11_2021.fb.fb_group.story_kiev_ua.1.million.pic.1}
\ii{18_11_2021.fb.fb_group.story_kiev_ua.1.million.pic.1.cmt}

\ii{18_11_2021.fb.fb_group.story_kiev_ua.1.million.pic.2}

\ii{18_11_2021.fb.fb_group.story_kiev_ua.1.million.pic.3}
\ii{18_11_2021.fb.fb_group.story_kiev_ua.1.million.pic.3.cmt}

\ii{18_11_2021.fb.fb_group.story_kiev_ua.1.million.pic.4}
\ii{18_11_2021.fb.fb_group.story_kiev_ua.1.million.pic.4.cmt}
\end{multicols} % }

Призначене мені місце чергування – відділення Держбанку Печерського району на
Хрещатику, біля площі Ленінського комсомолу (нині Європейська площа).

Як «молодого» мене поставили (вірніше, посадили) на пост біля входу у сховище
для грошей. Це на підземному поверсі. Клітка з металевих грат, в якій
броньовані двері в сховище. Доступ в клітку з двох сторін: з ліфту, яким
спускаються відповідальні працівники банку та зі сходів, якими спускаються
постові і черговий по об’єкту, який кожні дві години робить обхід і перевіряє
чи не спить постовий. Між сходами і кліткою маленька камера. В ній стоїть
письмовий стіл з телефоном внутрішнього зв’язку і стілець. Камера відділена від
сходів металевими дверима, які замикаються ззовні (зі сходів) черговим по
об’єкту. З камери в клітку також двері з металевих грат, які замикаються вже
постовим. Між кліткою і шахтою ліфту також металеві двері з грат, які теж
замикаються постовим.

Чергування на посту двозмінне: зранку до обіду і з обіду до закриття відділення
банку. Озброєння на чергуванні: 9-мм пістолет Макарова і дві обойми з
патронами. 

Старші колеги цей пост не любили, оскільки опалення там не було і температура
була постійна – десь +16 — +17. З собою на пост можна було брати термос з
гарячим чаєм, канапки, книжки.

Служба була поставлена так. Постовий на першій зміні заступав на охорону
сховища о 8-й годині ранку. В присутності чергового по об’єкту перевіряв
цілісність 3-х пластилінових печаток на дверях сховища, розписувався в прийомі
поста в спеціальній книзі. Після чого черговий виходив і замикав за собою
двері, що ведуть на сходи. Читати дозволялося, чим я і користувався. Десь на
початку десятої ліфтом спускалися до сховища три особи: керуючий банком,
головний бухгалтер і головний касир. Коли я бачив, що ліфт спустився, двері
ліфта відчинилися і в ліфті стоять троє працівників, то підходив до дверей
клітки і вітався з працівниками банку. Вони показували мені три посвідчення у
розгорнутому вигляді. Я передивлявся посвідчення (звіряв фотографії з
обличчями, перевіряв дату дійсності посвідчення) і відкривав двері. Втрьох вони
підходили до дверей сховища, перевіряли цілісність печаток, зривали їх,
вставляли три ключі, відкривали сховище і заходили в нього, закриваючи за собою
двері. Через деякий час вони виходили, котячи за собою відкритий візок, на
якому акуратним кубом лежали гроші різними номіналами в банківській упаковці.
Закривали двері сховища, пломбували печатками. Я перевіряв печатки, проводжав
їх до ліфта і закривав за ними двері. 

До кінця зміни я вже нікого крім чергового не бачив. Сидів і готувався до
лекцій – читав, робив виписки, писав конспекти. 

На другій зміні працівники банку повертали гроші в сховище. А ритуал проходив
за тим же сценарієм, що я описав вище.

З часом працівники банку звикли до мене, а я до них. І одного разу, коли вони
викотили візок з грішми, я запитав: «А скільки тут лежить грошей?»

Керуючий банком мені відповів: «Рівно мільйон рублів!» 

З його дозволу я двома руками доторкнувся до мільйона рублів! Заплющив очі і
декілька секунд так простояв.

Вони поїхали нагору, а я взяв аркуш паперу і став рахувати…

Мені двадцять два з половиною роки. І у мене є мільйон рублів! Якщо б я їх
зараз мав, то по скільки рублів в день я міг би витрачати протягом 50 років
життя, до 72,5!?

50 років життя – це, грубо кажучи 18 200 діб (364 х 50 = 18200).

1 000 000 рублів ділимо на 18 200 і маємо 54,94 рублі щоденно протягом 50 років
життя!

Чесно кажучи, я не повірив очам і декілька разів перераховував!

А моя заробітна платня на той час становила 140 рублів на місяць!

Наступним кроком моїх досліджень були мрії, як я витрачав би гроші!

Я вважав, що мені необхідно мати наступні речі:

• власний будинок з садом і басейном;

• власні автомашини – 2-3...

• мотоцикли – 2-3...

• штат домашніх працівників: камердинер (для догляду за гардеробом), повар з
помічниками, офіціантки, прибиральниці, садівник, водії і, безумовно, керуючий
штатом працівників.

Але при всьому цьому у мене не виникало жодних думок щодо зміни професії!

P.S.

Я прожив 50 років щасливо і без мільйона.

\ii{18_11_2021.fb.fb_group.story_kiev_ua.1.million.cmt}
