%%beginhead 
 
%%file 02_10_2019.fb.fb_group.mariupol.biblioteka.korolenka.1.festival__z_kra_ni_v
%%parent 02_10_2019
 
%%url https://www.facebook.com/groups/1476321979131170/posts/2461671670596191
 
%%author_id fb_group.mariupol.biblioteka.korolenka,kibkalo_natalia.mariupol.biblioteka.korolenko
%%date 02_10_2019
 
%%tags mariupol,mariupol.pre_war,festival,kultura
%%title Фестиваль "З країни в Україну"
 
%%endhead 

\subsection{Фестиваль \enquote{З країни в Україну}}
\label{sec:02_10_2019.fb.fb_group.mariupol.biblioteka.korolenka.1.festival__z_kra_ni_v}
 
\Purl{https://www.facebook.com/groups/1476321979131170/posts/2461671670596191}
\ifcmt
 author_begin
   author_id fb_group.mariupol.biblioteka.korolenka,kibkalo_natalia.mariupol.biblioteka.korolenko
 author_end
\fi

Фестиваль «З країни в Україну».  Хто не чув про цю фундацію соціальних
іновацій? Така велика культурно-просвітницька подія проходить в Україні вже
вшосте. Цього року у вересні він мандрує через Вугледар, Волноваху та Мангуш. І
у смт Мангуш вирушили волонтерами фестивалю співробітники Центральна міська
публічна бібліотека ім. В. Г. Короленка м. Маріуполь. Бібліотечна локація
складалась із  буккросінгу, затишної «книжкової» фотозони. А ще  бібліотекарі
запропонували гостям фестивалю невеличкий мозковий штурм - завдання
«Розгадай-ка».  За допомогою загадкового коду треба було розгадати зашифровані
вислови видатних людей про книгу. Запросили бібліотекарі мешканців Мангушу і на
цікавий майстер-клас в техниці скрапбукінг «Книжкова закладка для улюбленої
книжки». Бібліотекарі, звісно ж, завжди і різними способами намагаються
якнайбільше передати любов до Книги!
