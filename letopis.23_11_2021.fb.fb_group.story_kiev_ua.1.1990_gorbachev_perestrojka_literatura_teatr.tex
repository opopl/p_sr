% vim: keymap=russian-jcukenwin
%%beginhead 
 
%%file 23_11_2021.fb.fb_group.story_kiev_ua.1.1990_gorbachev_perestrojka_literatura_teatr
%%parent 23_11_2021
 
%%url https://www.facebook.com/groups/story.kiev.ua/posts/1804226679774125
 
%%author_id fb_group.story_kiev_ua,fedjko_vladimir.kiev
%%date 
 
%%tags gorbachev_mihail,kiev,kultura,literatura,perestrojka,sssr,teatr
%%title 1990-й... Горбачовська «перебудова» зносить цензурні шлюзи в літературі і театрі
 
%%endhead 
 
\subsection{1990-й... Горбачовська «перебудова» зносить цензурні шлюзи в літературі і театрі}
\label{sec:23_11_2021.fb.fb_group.story_kiev_ua.1.1990_gorbachev_perestrojka_literatura_teatr}
 
\Purl{https://www.facebook.com/groups/story.kiev.ua/posts/1804226679774125}
\ifcmt
 author_begin
   author_id fb_group.story_kiev_ua,fedjko_vladimir.kiev
 author_end
\fi

1990-й… Горбачовська «перебудова» зносить цензурні шлюзи в літературі і театрі. 

Іду на спектакль Київського державного хореографічного училища, який
відбудеться у Жовтневому палаці. Вигляд у мене богемний: шкіряний піджак,
батнік, джинси Levi Strauss, туфлі-мокасини; на плечі кофр з фотоапаратурою.
Ниряю в «трубу» (підземний перехід на площі Жовтневої Революції; нині Майдан
Незалежності), щоб перед спектаклем випити «подвійну половинку» кави.

У переході до мене звертається чоловік років тридцяти, теж стильно одягнений.
Представляється режисером новоствореного модернового театру [здається його
прізвище було Петраков] і дає запрошення на прем’єру спектаклю «Москва –
Пєтушки» за твором Венедикта Єрофєєва. Дякую за запрошення і продовжую свій
щлях.

З культовим твором Венедикта Єрофєєва я вже встиг познайомитися з щойно
випущеної у Москві книги, яку дав мені прочитати (на дві доби!) знайомий
режисер Молодіжного театру.

Прем’єра спектаклю відбувалася у приміщенні Театру ляльок (колишня синагога
Бродського). 

Зал заповнений повністю! Гасне світло... Наростає шум потяга, що наближається...
Відкривається завіса... На сцені декорація у вигляді купе електрички. В купе двоє
осіб.


\begin{multicols}{2} % {
\setlength{\parindent}{0pt}

\ii{23_11_2021.fb.fb_group.story_kiev_ua.1.1990_gorbachev_perestrojka_literatura_teatr.pic.1}
\ii{23_11_2021.fb.fb_group.story_kiev_ua.1.1990_gorbachev_perestrojka_literatura_teatr.pic.1.cmt}

\ii{23_11_2021.fb.fb_group.story_kiev_ua.1.1990_gorbachev_perestrojka_literatura_teatr.pic.2}
\ii{23_11_2021.fb.fb_group.story_kiev_ua.1.1990_gorbachev_perestrojka_literatura_teatr.pic.2.cmt}

\ii{23_11_2021.fb.fb_group.story_kiev_ua.1.1990_gorbachev_perestrojka_literatura_teatr.pic.3}
\ii{23_11_2021.fb.fb_group.story_kiev_ua.1.1990_gorbachev_perestrojka_literatura_teatr.pic.3.cmt}

\ii{23_11_2021.fb.fb_group.story_kiev_ua.1.1990_gorbachev_perestrojka_literatura_teatr.pic.4}
\ii{23_11_2021.fb.fb_group.story_kiev_ua.1.1990_gorbachev_perestrojka_literatura_teatr.pic.4.cmt}

\ii{23_11_2021.fb.fb_group.story_kiev_ua.1.1990_gorbachev_perestrojka_literatura_teatr.pic.5.erofeev}
\ii{23_11_2021.fb.fb_group.story_kiev_ua.1.1990_gorbachev_perestrojka_literatura_teatr.pic.5.cmt}
\end{multicols} % }

Головний герой, що виглядає як БОМЖ, звертається до залу...

«По всій землі, від Москви до Пєтушков, п'ють ці коктейлі і досі, не знаючи
імені автора, п'ють «Ханаанський бальзам», п'ють «Сльозу комсомолки», і
правильно роблять, що п'ють. Ми не можемо чекати милостей від природи. А щоб
взяти їх у неї, треба, зрозуміло, знати їх точні рецепти: я, якщо ви хочете,
дам вам ці рецепти. Слухайте!»

В залі здіймається легкий шум... Декілька осіб встають і йдуть на вихід...

«Пити просто горілку, навіть з горлечка, - в цьому немає нічого, крім томління
духу і суєти. Змішати горілку з одеколоном – в цьому є відомий каприз, але
немає ніякого пафосу. А ось випити склянку «Ханаанського бальзаму» – в цьому є
і каприз, і ідея, і пафос, і крім того ще метафізичний натяк,
Який компонент «Ханаанського бальзаму» ми цінуємо понад усе? Ну звичайно,
денатурат. Але ж денатурат, будучи тільки об'єктом натхнення, сам цього
натхнення начисто позбавлений. Що ж, в такому разі, ми цінуємо в денатураті
понад усе? Ну звичайно, голе смакове відчуття. А ще понад той міазм, який він
дарує. Щоб цей міазм відтінити, потрібна хоч крихта пахощів. З цієї причини в
денатурат вливають в пропорції 1:2:1 оксамитове пиво, найкраще «Останкінське»
або «Сенатор», і очищену політуру.

\ii{23_11_2021.fb.fb_group.story_kiev_ua.1.1990_gorbachev_perestrojka_literatura_teatr.pic.6.pamjatnik}
\ii{23_11_2021.fb.fb_group.story_kiev_ua.1.1990_gorbachev_perestrojka_literatura_teatr.pic.6.cmt}

Не буду вам нагадувати, як очищається політура. Це кожне дитя знає. Чомусь
ніхто в Росії не знає, від чого помер Пушкін, – а як очищається політура – це
всякий знає.

Коротше, записуйте рецепт «Ханаанського бальзаму». Життя дається людині один
раз, і прожити його треба так, щоб не помилитися в рецептах:

Денатурат – 100 г.

Оксамитове пиво – 200 г.

Політура очищена – 100 г.»

Ледь тихо загула вентиляція і по залу поплив запах денатурату! 

В залі сміх і легкий шум. Хтось голосно кричить: «Антісовєтчіна!», «Ганьба!»
(«Позор!», в оригіналі російською).  

З десяток осіб встають і виходять із зали. 

Головний герой продовжує…

«Отже, перед вами «Ханаанський бальзам» (його в просторіччі називають
«чорнобуркою») – рідина справді чорно-бурого кольору, з помірною міцністю і
стійким ароматом. Це навіть не аромат, а гімн. Гімн демократичної молоді. Саме
так, тому що в тому, хто випив цей коктейль, визрівають вульгарність і темні
сили. Я стільки раз спостерігав!..»

В залі сміх і легкий шум. Хтось голосно кричить «Ганьба!» («Позор!», в
оригіналі російською).  

Герой робить паузу і продовжує…

«А щоб визріванню цих темних сил хоч якось запобігти, є два засоби. По-перше,
не пити «Ханаанський бальзам», а, по-друге, пити замість нього коктейль «Дух
Женеви».

У ньому немає ні краплі шляхетності, але є букет. Ви запитаєте мене: в чому
загадка цього букета? Я вам відповім: не знаю, в чому загадка цього букета.
Тоді ви подумаєте і запитаєте: а в чому ж розгадка? А в тому розгадка, що
«Білий бузок» [«Белая сирень», рос.], складову частину «Духа Женеви», не слід
нічим заміняти, ні жасміном, ні шипром, ні конвалією. 

«У світі компонентів немає еквівалентів», як казали старі алхіміки, а вони-то
знали, що говорили. Тобто «Конвалія срібляста» [«Ландыш серебристый», рос.] –
це вам не «Білий бузок», навіть в моральному аспекті, не кажучи вже про букет.

«Конвалія», наприклад, розбурхує розум, тривожить совість, зміцнює
правосвідомість. А «Білий бузок» – навпаки того, заспокоює совість і примиряє
людину з виразками життя».

Гуде вентиляція і зала наповнюється свіжим повітрям!

«У мене було так: я випив цілий флакон «Конвалії сріблястої», сиджу і плачу.
Чому я плачу? – тому що маму згадав, тобто згадав і не можу забути свою маму.
«Мама», – кажу. І плачу. А потім знову: «Мама», – кажу, і знову плачу. Інший
би, хто дурнішийо, так би сидів і плакав. А я? Взяв флакон «Бузку…» – і випив.
І що ж ви думаєте? Сльози обсохнули, безглуздий сміх здолав, а маму так навіть
і забув, як звати по імені-по батькові.

І тому, як мені смішний, хто, готуючи «Дух Женеви», в засіб від пітливості ніг
додає «Конвалія срібляста»!

Слухайте точний рецепт:

«Білий бузок» – 50 г.

Засіб від пітливості ніг – 50 г.

Пиво «Жигулівське» – 200 г.

Лак спиртовий – 150 г.»

Поки герой надиктовує рецепт, вентиляція розносить по залі запах «Білого
бузку»! В залі сміх, легкий шум...

На сцені зміна персонажів... Попутник головного героя покидає купе, а замість
нього з’являється новий попутник, до якого звертається герой...

«Але якщо людина не хоче даремно топтати світобудову, нехай він пошле до свиней
і «Ханаанський бальзам» і «Дух Женеви». А краще нехай він сяде за стіл і
приготує собі «Сльозу комсомолки». Пахучий і дивний цей коктейль. Чому пахучий,
ви дізнаєтеся потім. Я спочатку поясню, чому він дивний».

В залі крики «Ганьба», «Антисовєтчина», «Ганьба»... Пару десятків глядачів
встають і йдуть на вихід.

Актори не звертають на це уваги і герой продовжує...

«П'ючий просто горілку зберігає і здоровий глузд, і тверду пам'ять, або навпаки
– втрачає разом і те, і інше. А у випадку з «Сльозою комсомолки» просто смішно:
вип'єш її сто грам, цієї сльози, – пам'ять тверда, а здорового глузду наче й не
було. Вип'єш ще сто грам – і сам собі дивуєшся: звідки, взялося стільки
здорового глузду? І куди поділася вся тверда пам'ять?

Навіть сам рецепт «Сльози» пахущий. А від готового коктейлю, від його
пахучості, можна на хвилину зомліти! Я, наприклад, втрачав свідомість!

Лаванда – 15 г.

Вербена – 15 г.

Одеколон «Лісова вода» – 30 г.

Лак для нігтів – 2 г.

Зубний еліксир – 150 г.

Лимонад – 150 г.

Приготовану таким чином суміш треба двадцять хвилин помішувати гілкою
жимолості. Інші, правда, стверджують, що в разі потреби можна жимолость
замінити берізкою. Це невірно і злочинно. Ріжте мене вздовж і поперек – ви мене
не примусите помішувати берізкою «Сльозу комсомолки», я буду помішувати її
жимолостю. Я просто розривався на частини від сміху, коли при мені помішують
«Сльозу» не жимолостю, а берізкою...»

В залі гучний сміх, аплодисменти… Хтось кричить «Ганьба!»… Йому відповідають:
«Не подобається – іди геть!» Декілька десятків глядачів покидають залу 

Герой з посмішкою дивиться в зал і продовжує…

«Але про «Сльозу…» досить. Тепер я пропоную вам останнє і найкраще. «Вінець
праць вище всіх нагород», як сказав поет. Коротше, я пропоную вам коктейль
«Сучий потрох», напій, що затьмарює все. Це вже не напій – це музика сфер. Що
найпрекрасніше в світі? –  боротьба за звільнення людства. А ще прекрасніше ось
що (записуйте):

Пиво «Жигулівське» – 100 г.

Шампунь «Садко – багатий гість» – 30 г

Аерозоль для очищення волосся від лупи – 70 г

Клей БФ – 12 г

Гальмівна рідина – 35 г

Дезинсекталь для знищення дрібних комах – 20 г

Все це тиждень настоюється на тютюні сигарних сортів – і подається до столу ...

Мені приходили листи, до речі, в яких дозвільні читачі рекомендували ще ось що:
отриманий таким чином настій ще відкидати на друшляк Тобто – на друшляк
відкинути і спати лягати... Це вже чорт знає що таке, і всі ці доповнення і
поправки – від в'ялості уяви, від нестачі польоту думки, ось звідки ці
безглузді поправки...

Отже, «Сучий потрох» поданий на стіл. Пийте його з появою першої зірки,
великими ковтками. Вже після двох келихів цього коктейлю людина стає настільки
одухотвореною, що можна підійти і цілих півгодини з відстані півтора метрів
плювати їй в харю, і вона нічого тобі не скаже».

Зал регоче і аплодує! В залі залишилася половина глядачів!

***

На жаль, фотозйомку цього спектаклю (в негативах) я дав знайомому режисеру,
оскільки у нього теж були якісь задуми щодо цього твору. А потім було не до
цієї теми.

Рецептура у спогадах (в перекладі з російської) наводиться по записах із
щоденника (виписки з книги).

Я згадав про цей спектакль тому, що потім «Москва – Пєтушки» ставили на своїх
сценах багато театрів, але після розвалу СРСР вже ніколи не було такої
атмосфери в залах.  

***

З рукопису спогадів «Погляд у минуле», написаних для особового фонду Федька В.Ф
у Державному архіві Київської області.

\ii{23_11_2021.fb.fb_group.story_kiev_ua.1.1990_gorbachev_perestrojka_literatura_teatr.cmt}
