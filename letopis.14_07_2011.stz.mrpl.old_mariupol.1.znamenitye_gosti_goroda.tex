% vim: keymap=russian-jcukenwin
%%beginhead 
 
%%file 14_07_2011.stz.mrpl.old_mariupol.1.znamenitye_gosti_goroda
%%parent 14_07_2011
 
%%url http://old-mariupol.com/znamenitye-gosti-goroda
 
%%author_id mrpl.old_mariupol,jaruckij_lev.mariupol
%%date 
 
%%tags 
%%title ЗНАМЕНИТЫЕ ГОСТИ ГОРОДА
 
%%endhead 
 
\subsection{ЗНАМЕНИТЫЕ ГОСТИ ГОРОДА}
\label{sec:14_07_2011.stz.mrpl.old_mariupol.1.znamenitye_gosti_goroda}
 
\Purl{http://old-mariupol.com/znamenitye-gosti-goroda}
\ifcmt
 author_begin
   author_id mrpl.old_mariupol,jaruckij_lev.mariupol
 author_end
\fi

\begin{quote}
\em\bfseries
Что делал в нашем городе адмирал Литке и побывал ли в Мариуполе Айвазовский?

Краеведческие исследования порой бывают похожи на детектив. Но если в финале
пос­леднего все сюжетные узлы непременно развязываются, то краеведу нередко
приходится признать, что поиск ушел в песок, а мучившая его загадка осталась
неразгаданной.
\end{quote}

Мне всегда казалось странным, что среди бесподобных марин Ивана Константиновича
Айвазовского (1817-1900), — а я имел счастье любоваться ими в немалом
количестве в раз­личных коллекциях, в том числе и в знаменитой галерее в
Феодосии, — ни разу не попался мне ни один азовский сюжет. Неужели великий
художник-маринист, чья жизнь и творче­ство столь тесно связаны с Черным морем,
так ни разу не взглянул на его скромного соседа — Азовское? Может ли быть,
чтобы несравненный поэт моря, избороздивший на кораблях тысячи миль \enquote{свободной
стихии} на разных широтах, так и не увидел самое маленькое в мире и уже этим
одним уникальное море, не оценил, не восхитился его неброской красотой и
прелестью?

\enquote{Кто жил и мыслил, тот не может} не считать весьма сомнительной оптимистическую
сентенцию: \enquote{Кто ищет, тот всегда найдет}. Но в этом случае в ходе моих поисков
неожиданно (как всегда) блеснула надежда, что Айвазовский все-таки побывал в
Мариуполе. Был в биографии Ивана Константиновича эпизод, когда он непременно
должен был отправиться в Мариуполь.

Этот эпизод связан с приездом в наш город великого князя Константина
Николаевича. Второй сын Николая I и родной брат Александра II побывал у нас
дважды — в 1845 году и еще в 1876-м. Его первый приезд в наш город интересен не
только из-за весьма незаурядной личности великого князя, который с Мариуполем,
помимо сказанного, связан и через своего сына Константина Константиновича,
купившего по баснословной по тем временам цене у Архипа Ивановича Куинджи его
шедевр \enquote{Лунная ночь на Днепре}. В не меньшей степени интересна для нас свита,
сопровождавшая великого князя в путешествии 1845 года. А входили в эту свиту
такие знаменитости, как Ф. П. Литке и И. К. Айвазовский.

\ifcmt
  ig https://i2.paste.pics/PKLAF.png?trs=1142e84a8812893e619f828af22a1d084584f26ffb97dd2bb11c85495ee994c5
	@caption Федор Литке
  @wrap center
  @width 0.7
\fi

Федор Петрович Литке (1797-1882) — выдающийся русский ученый и мореплаватель,
адмирал и академик, талантливый организатор науки. Он стоял у колыбели Русского
геогра­фического общества, многие годы был президентом Академии наук и своими
трудами внес весомый вклад не только в отечественную — в мировую науку. \enquote{Имя
его будет жить вечно среди географов и моряков}, — так пишут о нем в наши дни.
И если мы, по примеру других городов, станем составлять список выдающихся
людей, когда-либо посетивших Мариуполь, одно из первых почетных мест в этом
перечне несомненно займет Ф. П. Литке.

В 1832 году Николай I вызвал Федора Петровича и назначил его главным
воспитателем своего второго сына, Константина, которому в ту пору исполнилось
пять лет. Император пожелал, чтобы его второй сын со временем не формально, а
фактически, профессионально хорошо подготовленным мог стать во главе
Российского флота.

Сначала Литке проводил со своим воспитанником 12 часов в сутки, но вскоре няню
отправили в отставку, и Федор Петрович перешел на круглосуточную бессменную
вахту. Удивительно ли, что великий князь необычайно привязался к своему
наставнику. \enquote{У меня теперь три отца, — писал шестнадцатилетний Константин
Федору Петровичу, — вездесущий отец небесный, Папа, который в то же время и мой
Государь, и Ты, который всегда печется о моем счастии...}.

В систему воспитания будущего морского министра России Литке включил, конечно,
ежегодные плавания, например вокруг Скандинавии. Но в 1845 году, когда
Константин окончил курс обучения и перед лицом своего венценосного родителя
блестяще выдержал экзамен, запланировал Федор Петрович путешествие из
Петербурга в Константинополь и далее по Мраморному и Эгейскому морям. Из
столицы они сухим путем добрались до Николаева, где их любезно встретил
адмирал Лазарев. Здесь же к свите Константина присое­динился
двадцативосьмилетний Айвазовский, к этому времени уже знаменитый художник,
академик. В этом же 1845 году Константин вместе со своей свитой посетил
Мариу­поль, на что у нас имеется документальное подтверждение. Значит, и
Айвазовский побывал в нашем городе?

\ifcmt
  ig http://old-mariupol.com/wp-content/uploads/2011/07/%D0%90%D0%B9%D0%B2%D0%B0%D0%B7%D0%BE%D0%B2%D1%81%D0%BA%D0%B8%D0%B9.jpg
	@caption Иван Айвазовский
  @wrap center
  @width 0.7
\fi

Но тут начинаются сложности.

Из Николаева в Одессу путешественники прибыли на пароходе \enquote{Бессарабия} 31 мая.
Затем направились в Севастополь, где Константина торжественно встретил весь
Черноморский флот. А уже 6 июня \enquote{Бессарабия} во главе эскадры под
командованием Литке (в нее входили еще фрегат \enquote{Флора}, бриг \enquote{Персей} и шхуна
\enquote{Дротик}) встала на якорь в Босфоре. Совершенно немыслимо предположить, что
великий князь за шесть дней, которые отделяют его от прибытия в Одессу до
швартовки в Константинополе с предварительным заходом в Севастополь, сумел
выкроить время для путешествия по Азовскому морю с посещением, в частности,
Мариуполя.

Г. И. Тимошевский в книге \enquote{Мариуполь и его окрестности} (1892) зафиксировал с
подробностями два пребывания в городе Константина, и достоверность этих
сведений не вызывает никаких сомнений. Жаль только, что авторитетный источник,
называя 1845 год, не указывает месяц и день визита.

Остается одно: предположить, что великий князь посетил Мариуполь не до, а после
путешествия к берегам Турции и по Архипелагу.

Мы точно знаем, что о приезде Константина Мариуполь был заблаговременно
предупрежден. Отцы города, повторюсь, по обыкновению навели блеск, в том числе
городской голова Иван Алексеевич Чабаненко приказал снести с лица земли остатки
запорожской крепости (как жаль!). Его высочество остановился в доме Чентукова.
В благодарность за постой великий князь подарил хозяину дома золотую табакерку.

Г. И. Тимошевский, не будучи уроженцем Мариуполя, писал о рассматриваемом
эпизоде со слов старожилов через 47 лет после визита Константина. При этом
смущает, что он, рассказывая об этом визите, не упоминает столь громкое имя
Айвазовского. Но точно так же он не упоминает и Федора Петровича Литке, тоже
очень именитого и известного человека. Возможно, что старые мариупольцы были в
свое время так захвачены столь исключительным фактом посещения их города сыном
государя императора, что блестящая свита, сопровождавшая Константина, осталась
в тени. Тем более, что о знаменитом живописце, как и о прославленном адмирале
и ученом они по невежеству своему слыхом, должно быть, не слыхивали. И
остались Литке и Айвазовский вне памяти встречавших великого князя
мариупольцев, как некогда и Пушкин во время посещения нашего города генералом
Раевским.

Извивы исторических сюжетов прихотливы и неожиданны до невероятности. Ну,
скажите, разве мог бы кто-нибудь помыслить в 1845 году, что через 82 года
Федор Петрович Литке вновь окажется в этом городе? Но факт: почти через век \enquote{Ф.
Литке}, линейный ледорез, построенный в 1909 году в Великобритании,
отшвартовался у пирса Мариупольского порта.

Зима 1927 года выдалась суровой, и немало судов на подходе к Мариуполю застряли
во льдах замерзшего Азовского моря. Сохранился рапорт капитана Н. М. Николаева
(этот документ обнаружил краевед Юрий Наумов, ныне покойный) начальнику
Мариупольского морского порта о том, что \enquote{Ф. Литке} только за февраль вывел из
порта 14 и ввел 6 судов. В следующем 1928 году \enquote{Ф. Литке} не менее успешно
боролся с ледяным панцирем, сковавшим Азовское море, давая возможность
мариупольским морякам продолжать навигацию и в зимнее время.

Из Мариуполя \enquote{Ф. Литке} после ремонта в Севастополе ушел на Дальний Восток,
что­бы стать ледоколом арктического флота. Скажу, не боясь высокопарности: то
был путь в бессмертие. В 1934 году \enquote{Ф. Литке} под началом уже известного нам
капитана Н. М. Николаева при научном руководстве ученого-полярника Владимира
Юльевича Визе, участво­вавшего, к слову сказать, в знаменитой экспедиции к
Северному полюсу нашего земляка Г. Я. Седова, совершил первое в истории
сквозное плавание Северным морским путем.

Отважного \enquote{Ф. Литке} списали только в 1958 году, но на карте Арктики два
острова обозначены именем Федора Петровича. И Мариуполь может не без гордости
сказать, что не раз на своем веку видел Федора Петровича Литке — человека и
парохода.

А через несколько лет после того, как \enquote{Ф. Литке} навсегда покинул наш город, в
Мариуполе поселился родной племянник Ивана Константиновича Айвазовского, сын
его сестры Екатерины Константиновны, Семен Иванович Вениосов.

Но это тема для отдельного рассказа.

\vspace{0.5cm}
\textbf{Лев Яруцкий}
\vspace{0.5cm}

\textbf{\enquote{Мариупольская старина}.}
