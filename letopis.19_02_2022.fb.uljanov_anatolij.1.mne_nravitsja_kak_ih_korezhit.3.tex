% vim: keymap=russian-jcukenwin
%%beginhead 
 
%%file 19_02_2022.fb.uljanov_anatolij.1.mne_nravitsja_kak_ih_korezhit.3
%%parent 19_02_2022.fb.uljanov_anatolij.1.mne_nravitsja_kak_ih_korezhit
 
%%url 
 
%%author_id 
%%date 
 
%%tags 
%%title 
 
%%endhead 

\subsubsection{3}

Самое удивительное в наших патриотах – это обида за недостаток патриотизма в
тех из нас, кому они плевали в лицо, и для кого их флаг, к огромному сожалению,
стал символом невозможности быть собой в своей стране.

Что в этом флаге должно привлекать? Какие права эта тряпка даёт? Где свобода? И
зачем без свободы для всех Украина? Какую общность мы можем чувствовать с теми,
кто рассказывает нам про то, что мы – оккупанты, и язык наш – собачий?

Нет, ты не можешь рассчитывать на безоговорочную поддержку тех, кого ты либо
расчеловечивал, требуя потерпеть твою «положительную дискриминацию», либо
помалкивал, пока у тебя на глазах их душили? Закон за законом, улица за улицей,
памятник за памятником власть ксенофобов вытирала о нас ноги, делая вид, что
ничего не происходит; думая, что мы – глупы, слепы, и не видим, к чему это
идёт.   

Майдан исключил миллионы нас, и потому не стоит удивляться тому, что теперь
исключённые люди не желают сливаться в милитаристском засосе со своими
«патриотическими» угнетателями – теми, кто означает для нас удушье.

Те, кто кричали «Чемодан-Вокзал-Россия» не в праве обвинять тех из нас, кто от
отчаяния прислушался к этому крику, и кто встречает российский танк цветами.
Это – трагедия. Но в этой трагедии виновны, в первую очередь, те, кто отчуждал
людей за их особенности и отличия.

Украина стала для нас живодёрней, где злоба душит наши мечты. Где невозможно
жить и работать свободно. Говорить то, что думаешь. Где медиа закрывают. Где
партии запрещают. Где символы наших семей – вне закона, а насилие – правило.

Что и кого может привлекать в этом чудище, в которое превратили нашу страну
олигархи, и их челядь? Во имя чего мы должны брать в руки оружие и умирать за
флаг нашего исключения? 

Нет, человек и наша жизнь для нас важнее, чем страна. Уж пусть простят. И да,
мы видим, что страна для них важнее человека. А раз так, то нам не по пути. 

Мы не хотим гибнуть за империи, олигархов, и колониальные элиты, которые
толкают нас в окопы, пакуя, меж тем, чемоданы со стикером «Лондон».

