% vim: keymap=russian-jcukenwin
%%beginhead 
 
%%file 06_02_2021.fb.fb_group.story_kiev_ua.1.pokazhu_vam_gorod
%%parent 06_02_2021
 
%%url https://www.facebook.com/groups/story.kiev.ua/posts/1592682624261866
 
%%author_id fb_group.story_kiev_ua,lebedev_fedor.kiev.videoblogger
%%date 
 
%%tags film,gorod,kiev,kino
%%title "ПОКАЖУ ВАМ ГОРОД" – 10-МИНУТНЫЙ ФИЛЬМ О КИЕВЕ, КОТОРЫЙ НАВЕРНОЕ, НИКТО ИЗ ВАС НЕ ВИДЕЛ
 
%%endhead 
 
\subsection{\enquote{ПОКАЖУ ВАМ ГОРОД} – 10-МИНУТНЫЙ ФИЛЬМ О КИЕВЕ, КОТОРЫЙ НАВЕРНОЕ, НИКТО ИЗ ВАС НЕ ВИДЕЛ}
\label{sec:06_02_2021.fb.fb_group.story_kiev_ua.1.pokazhu_vam_gorod}
 
\Purl{https://www.facebook.com/groups/story.kiev.ua/posts/1592682624261866}
\ifcmt
 author_begin
   author_id fb_group.story_kiev_ua,lebedev_fedor.kiev.videoblogger
 author_end
\fi

\enquote{ПОКАЖУ ВАМ ГОРОД} – 10-МИНУТНЫЙ ФИЛЬМ О КИЕВЕ,  КОТОРЫЙ НАВЕРНОЕ, НИКТО ИЗ ВАС
НЕ ВИДЕЛ.

У этого фильма интересная история. Снимался он по заказу Киевского городского
бюро путешествий и экскурсий. Интересно то, что фильм заказали не
профессиональной студии, а мне, - тогда ещё 21-летнему преподавателю
кино-кружка Киевского Дворца пионеров и школьников им.Н.Островского. И я,
вместе со своими воспитанниками, – 14-16-ти летними детьми в августе 1982 года
(почти 39 лет назад!), отснял это кино на 35-мм.киноплёнке. 

\ii{06_02_2021.fb.fb_group.story_kiev_ua.1.pokazhu_vam_gorod.pic.1}

Почему заказали именно мне, а не профессиональной киностудии? Потому, что годом
ранее, я с этими же школьниками сделал фильм о Киево-Печерской Лавре. Снимали
для себя, во время занятий. Лавра находилась близко от Дворца пионеров и мы
туда с детишками ходили на съёмки. Директору Лавры (Киево-Печерского
заповедника) та сделанная кинокартина очень понравилась и он за 5 тысяч рублей
(огромные по тем временам деньги) купил у Дворца пионеров 3 фильмокопии,
которые стали крутить в кинотеатрах перед сеансами.

\ii{06_02_2021.fb.fb_group.story_kiev_ua.1.pokazhu_vam_gorod.pic.2}

В кинотеатре, фильм о Лавре увидел директор бюро путешествий и экскурсий. Он
приехал ко мне прямо в институт (где я проходил зимнюю сессию как
студент-заочник), и  предложил снять фильм о Киеве для бюро путешествий. Мы
договорились, что он заплатит мне 850 рублей (во Дворце я получал 120 руб.).
Подписали договор, я взял отпуск и вскоре, со своими
воспитанниками-добровольцами приступили к съёмкам. Бюро путешествий оплачивало
прокат киноаппаратуры на киностудии им. А. Довженко, транспорта и даже вертолёта,
который мы арендовали на часик у ГАИ. Так что и в Советском Союзе оказывается,
можно было неплохо «шабашить»!

\ii{06_02_2021.fb.fb_group.story_kiev_ua.1.pokazhu_vam_gorod.pic.3}

Четырём мальчикам, помогавшим мне на съёмках, было очень интересно работать на
съёмочных площадках ассистентами. Ещё больший восторг они испытали, когда я,
получив расчёт, заплатил им деньги, - от 70 до 150 рублей, исходя из их
персональных  трудовых затрат.

\ii{06_02_2021.fb.fb_group.story_kiev_ua.1.pokazhu_vam_gorod.pic.4}

Через год я получил диплом киноведа и ушёл из Дворца, устроившись
кинорежиссёром на студию «Укртелефильм». Но это уже другая история. 

\ii{06_02_2021.fb.fb_group.story_kiev_ua.1.pokazhu_vam_gorod.pic.5}

Весной я хочу начать снимать новый фильм о Киеве со своими учениками, которых
обучаю видеоблогерству. Хочу совместить фотографии старого Киева 19-20 веков с
кадрами, снятыми с этих же точек, но уже сегодня.

\ii{06_02_2021.fb.fb_group.story_kiev_ua.1.pokazhu_vam_gorod.pic.6}

Друзья! Буду признателен, если поделитесь качественными фотографиями старого
Киева, которые можно использовать в новом кино!

И не судите строго о нашем фильме «Покажу вам город...», – всё-таки, её делали
школьники, а главному режиссёру-оператору тогда стукнул 21 годик...

Фёдор Лебедев.

\url{flebedev@ukr.net}

\ii{06_02_2021.fb.fb_group.story_kiev_ua.1.pokazhu_vam_gorod.pic.7}
\ii{06_02_2021.fb.fb_group.story_kiev_ua.1.pokazhu_vam_gorod.pic.8}
