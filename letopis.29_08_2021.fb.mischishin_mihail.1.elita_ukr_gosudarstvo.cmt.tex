% vim: keymap=russian-jcukenwin
%%beginhead 
 
%%file 29_08_2021.fb.mischishin_mihail.1.elita_ukr_gosudarstvo.cmt
%%parent 29_08_2021.fb.mischishin_mihail.1.elita_ukr_gosudarstvo
 
%%url 
 
%%author_id 
%%date 
 
%%tags 
%%title 
 
%%endhead 
\subsubsection{Коментарі}

\begin{itemize}
%%%fbauth
%%%fbauth_name
\iusr{Andrei Raschuk}
%%%fbauth_url
%%%fbauth_place
%%%fbauth_id
%%%fbauth_front
%%%fbauth_desc
%%%fbauth_www
%%%fbauth_pic
%%%fbauth_pic portrait
%%%fbauth_pic background
%%%fbauth_pic other
%%%fbauth_tags
%%%fbauth_pubs
%%%endfbauth
 
Так будет всегда, если строить национальное государство, а не демократическое

%%%fbauth
%%%fbauth_name
\iusr{Андрей Лукашов}
%%%fbauth_url
%%%fbauth_place
%%%fbauth_id
%%%fbauth_front
%%%fbauth_desc
%%%fbauth_www
%%%fbauth_pic
%%%fbauth_pic portrait
%%%fbauth_pic background
%%%fbauth_pic other
%%%fbauth_tags
%%%fbauth_pubs
%%%endfbauth
 

Напоминаю, в декабре 1991 года люди проголосовали за Независимость Украины на
условиях «Декларации о государственном суверенитете Украины», в которой
ОТСУТСТВУЕТ должность президента Украины.

\begin{itemize}
%%%fbauth
%%%fbauth_name
\iusr{Михаил Мищишин}
%%%fbauth_url
%%%fbauth_place
%%%fbauth_id
%%%fbauth_front
%%%fbauth_desc
%%%fbauth_www
%%%fbauth_pic
%%%fbauth_pic portrait
%%%fbauth_pic background
%%%fbauth_pic other
%%%fbauth_tags
%%%fbauth_pubs
%%%endfbauth
 
\textbf{Андрей Лукашов} Да, наверное, Вы правы. Одного премьера нам хватило бы.
Надежды на то, что в Украине появится президент-реформатор, и ему нужна будет
независимость, уже практически нет. Правда, период сравнительного неплохого
правления Кучмы связан именно с сильным конституционно президентом. Дальше его
полномочия начали урезать в пользу Рады и правительства. А, как только Янукович
их восстановил, то его сразу же и не стало на посту президента. После чего
восстановленный статус сильного президента революционеры почему-то тут же и
отменили.
\end{itemize}

%%%fbauth
%%%fbauth_name
\iusr{Feran Tarek}
%%%fbauth_url
%%%fbauth_place
%%%fbauth_id
%%%fbauth_front
%%%fbauth_desc
%%%fbauth_www
%%%fbauth_pic
%%%fbauth_pic portrait
%%%fbauth_pic background
%%%fbauth_pic other
%%%fbauth_tags
%%%fbauth_pubs
%%%endfbauth
 

При Ельцине в 90-ых Запад тоже активно поддерживал Россию, пока Ходорковские,
Гусинские, Березовские и иже с ними лихо присваивали народное добро и
передавали управление активами западным партнерам.

Как только Россия в 2000-ых решила пойти своим путем, а не скакать на задних
лапах перед Западом, мгновенно стала изгоем.

Появился список Магнитского, секторальные санкции, а Крым просто стал очередным
поводом для дальнейшего противостояния.

Видимо Украине нужно определиться, либо становиться вассалом Запада и послушно
поставлять сырье и трудовые ресурсы, либо прыгать под крыло России.
Самостоятельно устоять Украине практически невозможно.

И нужно еще одно понять-перспектива у Украины не как у Польши, а как у
Афганистана.

\begin{itemize}
%%%fbauth
%%%fbauth_name
\iusr{Viktor Yeryomin-Khokhljuk}
%%%fbauth_url
%%%fbauth_place
%%%fbauth_id
%%%fbauth_front
%%%fbauth_desc
%%%fbauth_www
%%%fbauth_pic
%%%fbauth_pic portrait
%%%fbauth_pic background
%%%fbauth_pic other
%%%fbauth_tags
%%%fbauth_pubs
%%%endfbauth
 
\textbf{Feran Tarek}
У них всё раздеребанено, как в Украине. Только "семей" побольше, а коррупция ещё похлеще.

%%%fbauth
%%%fbauth_name
\iusr{Михаил Мищишин}
%%%fbauth_url
%%%fbauth_place
%%%fbauth_id
%%%fbauth_front
%%%fbauth_desc
%%%fbauth_www
%%%fbauth_pic
%%%fbauth_pic portrait
%%%fbauth_pic background
%%%fbauth_pic other
%%%fbauth_tags
%%%fbauth_pubs
%%%endfbauth
 
\textbf{Feran Tarek}

В целом согласен, кроме - прыгать под крыло России.
Теоретически понятно. Практически сейчас это трудно реализуемо. Мне кажется,
для Украины был бы сейчас очень неплох - более спокойный и взвешенный путь
Финляндии, Норвегии, Вьетнама и т.д., не сильно влазящих в разборки России и
Запада, а теперь - и Китая. Да, Россия - рядом. И это тот фактор, который в
Украине нельзя не учитывать. То, что наши революционеры позволили сделать из
себя вторую Грузию и Прибалтику - так мы получим размеры Прибалтики и
грузинскую войну у себя в итоге. Но все почему-то надеются, что будет так, как
было раньше, когда Прибалтика росла на российском транзите и выведении средств
из России. А из Грузии при Саакашвили пытались лепить витрину западных реформ
любыми средствами и, забывая подтирать внутри. Россия серьезным образом
окрепла. А мы верим всем этим байкам западным. Как дети, честное слово.
\end{itemize}

%%%fbauth
%%%fbauth_name
\iusr{Алла Круглова}
%%%fbauth_url
%%%fbauth_place
%%%fbauth_id
%%%fbauth_front
%%%fbauth_desc
%%%fbauth_www
%%%fbauth_pic
%%%fbauth_pic portrait
%%%fbauth_pic background
%%%fbauth_pic other
%%%fbauth_tags
%%%fbauth_pubs
%%%endfbauth
 
Не хотят, не собираются это делать и не будут, к огромному нашему сожалению. 😞

\begin{itemize}
%%%fbauth
%%%fbauth_name
\iusr{Михаил Мищишин}
%%%fbauth_url
%%%fbauth_place
%%%fbauth_id
%%%fbauth_front
%%%fbauth_desc
%%%fbauth_www
%%%fbauth_pic
%%%fbauth_pic portrait
%%%fbauth_pic background
%%%fbauth_pic other
%%%fbauth_tags
%%%fbauth_pubs
%%%endfbauth
 
\textbf{Алла Круглова} Да, Аллочка. Пришел к такому же выводу.
\end{itemize}

% -------------------------------------
\ii{fbauth.grigorjev_ivan.moskva}
% -------------------------------------

Я может обижу чем-то украинцев, но они никогда не строили и не хотели строить
государство. Более 300 лет им обьясняли из Москвы как строить.теперь уже 30 лет
обьясняют из Вашингтона. И никакие партаппаратчик Кравчука никого в космос не
отправляли, они выполняли команды из центра. Украина не будет нормальным
государством, пока не развалится и не начнет сама собираться. Всё что было у
Украины было сделано не ею, да при её участии, но без её воли. Промышленность
уже развалили, медицину тоже, теперь теряются территории. Вот когда останется
что-то маленькое, но то, что будет представлять ценность для своих граждан, и
если граждане смогут этой ценностью увлечь других, тогда и сможет возникнуть
украинское государство.

% -------------------------------------
\ii{fbauth.shabrina_svetlana.kiev.ukraina}
% -------------------------------------
 
Никак не пойму, зачем и почему называют их "элитой"? Что в них элитарного )))

% -------------------------------------
\ii{fbauth.rozhankovskij_vladimir.moskva.rossia.ekspert.finansy}
% -------------------------------------

Думаю, что потому, что политика является производной от экономики, а не наоборот

% -------------------------------------
\ii{fbauth.abroskin_vladimir.rossia}
% -------------------------------------

Все, казалось бы, есть для построения успешного, процветающего государства -
огромная территория, богатая земля, прекрасный климат, трудолюбивый,
талантливый, патриотичный народ, мощное наследство империи и т.д. и т.п. А
государства не получается! В чем же дело? На мой взгляд, одна из существенных
причин - отказ от общей с Россией истории. 

Ведь история - мощнейший неисчерпаемый государствообразующий ресурс, духовный
стержень, на который нанизываются все государственные институты. 

История - питательная среда для формирования элиты, осознающей свою
ответственность за судьбу страны! 

Украина сама порвала в клочья, изрезала на куски все свое славное прошлое. И
первый шаг - отказ от русского имени! Ну нельзя же всерьез относиться к
двуглавым эвфемизмам-уродцам в виде Украины-Руси или Руси-Украины! 

Смешны стенания украинцев о том, что Россия украла их русское имя. С такой
радостью и готовностью от дорогого не отказываются! А ведь можно было строить
независимое государство, не отрекаясь от общей с Россией историей! 

Вам не нравится российская азиатчина? Вы - еуропейцы! Прекрасно, стройте чисто
европейский вариант России, свободный от влияния Кавказа, Средней Азии, Сибири!
Ведь в вашей общей с Россией историей был и такой вектор! Используйте его! Вас
воротит от российской имперскости, от вовлеченности России в дела Сирии,
Венесуэлы, Африки. Да, это крест России, который она обязана нести, если хочет
оставаться Россией. Вам это претит? Пожалуйста! Сбросьте эту ношу, на вас нет
бремени имперских сверхрасходов. Был и такой период в богатейшей русской
истории.

Стройте альтернативную Русь по образцу Швейцарии! Что мешает-то? Да то, что
смыслом вашего государства является АнтиРоссия, а не иная, лучшая на ваш взгляд
Россия!

\index{АнтиРоссия}

\begin{itemize}
%%%fbauth
%%%fbauth_name
\iusr{Михаил Мищишин}
%%%fbauth_url
%%%fbauth_place
%%%fbauth_id
%%%fbauth_front
%%%fbauth_desc
%%%fbauth_www
%%%fbauth_pic
%%%fbauth_pic portrait
%%%fbauth_pic background
%%%fbauth_pic other
%%%fbauth_tags
%%%fbauth_pubs
%%%endfbauth
 
\textbf{Vladimir Abroskin} 

Да. Но надо понимать, что сейчас мы в значительной степени ведомы. Украинцы,
как ребенок в переходном возрасте, сам себе выдает чужие, вздорные и часто
подлые мысли за свои. Но они нам чужие. Со временем большая часть из нас это
поймет. Меньшая и не желающая меняться, предпочитающая жить ненавистью, -
оторвётся от Украины и уйдет в свое самостоятельное плавание.
\end{itemize}

% -------------------------------------
\ii{fbauth.klochko_andrej.genichesk.ukraina.opzzh}
% -------------------------------------

В Украине, тех кто хотел создать достойное государство, почему-то системно
убивают! Вывод. Для решения задачи описанной в публикации, необходим сильный
силовой блок иначе убьют. Поэтому без учёта исторического опыта не реально
прийти к успеху, а историческая ретроспектива всегда вращается вместе с
Россией!

\begin{itemize}
%%%fbauth
%%%fbauth_name
\iusr{Михаил Мищишин}
%%%fbauth_url
%%%fbauth_place
%%%fbauth_id
%%%fbauth_front
%%%fbauth_desc
%%%fbauth_www
%%%fbauth_pic
%%%fbauth_pic portrait
%%%fbauth_pic background
%%%fbauth_pic other
%%%fbauth_tags
%%%fbauth_pubs
%%%endfbauth
 
\textbf{Андрей Клочко} 

Увы, на моей памяти погибли в авариях Вячеслав Черновол
и Кузьма Скрябин. Один был в состоянии построить европейскую Украину без
угнетения русских. Другой попадал в струю шоуменов: Кличко, Зеленский и т.д.,
но был на голову выше их (в моем понимании) в оценке того, что происходит в
Украине. И то, что он говорил и собирался делать, я думаю, могло сработать и
развивать Украину. Оба ушли неожиданно.

%%%fbauth
%%%fbauth_name
\iusr{Андрей Клочко}
%%%fbauth_url
%%%fbauth_place
%%%fbauth_id
%%%fbauth_front
%%%fbauth_desc
%%%fbauth_www
%%%fbauth_pic
%%%fbauth_pic portrait
%%%fbauth_pic background
%%%fbauth_pic other
%%%fbauth_tags
%%%fbauth_pubs
%%%endfbauth
 
\textbf{Михаил Мищишин} 

идея федеративной Украины В. Черновола была и остаётся
актуальной и сегодня для строительства успешного государства. Также вы забыли
вспомнить Е. Кушнарёва, который разделял такую же идею. Что позволило бы
Украине адекватно соседствовать со всеми странами Евразийского континента.

%%%fbauth
%%%fbauth_name
\iusr{Михаил Мищишин}
%%%fbauth_url
%%%fbauth_place
%%%fbauth_id
%%%fbauth_front
%%%fbauth_desc
%%%fbauth_www
%%%fbauth_pic
%%%fbauth_pic portrait
%%%fbauth_pic background
%%%fbauth_pic other
%%%fbauth_tags
%%%fbauth_pubs
%%%endfbauth
 
\textbf{Андрей Клочко} 

Да. Евгений Кушнарёв. Здравый и сильный политик. И тоже
нелепая, неожиданная смерть. На охоте. По Кушнареву и Черноволу, кстати, видно,
что политик из Харькова и Киева и политик из Черкасс, Львова и Киева могут
прийти к одним и тем же или очень похожим выводам. Если только, они -
проукраинские политики. По сути. А не по антуражу, как у Ющенко, например, и
его бравой швейковской команды, яку вин позбырав на смитныках, в чем честно и
признался.

\end{itemize}

% -------------------------------------
\ii{fbauth.zadorozhnaja_natalia.kiev.ukraina.doneck}
% -------------------------------------

Трудно комментировать, потому что суть событий, приведших к данности, отражена объективно и точно.

И главная мысль безукоризненна - пока нынешние бесталанные управленцы (суть
марионетки) ради трамбовки карманов, пренебрегающие интересами собственных
граждан и страны, будут у власти - добру здесь не быть.

И какие союзы несут в себе экономически более выгодные перспективы, любому
думающему читателю очевидно.

Но до момента, когда возникнет осознанная самостоятельность и приоритетом
станут интересы именно Украины и её людей, ошеломлённых размахом коррупции,
уничтожения и безнаказанности тех, кто вместо развития страны её рвёт на части,
как свиньи у бюджетного корыта, ничего не изменится. Нет к тому никаких
предпосылок.

Политические импотенты, послушные кукловодам за возможность личных благ без
наказания, за нетронутые счета, выведенные оглушительные суммы, обескровившие
банковскую систему и экономику, уничтожившие в угоду заокеанским или
европейским опекунам собственную страну, в моём понимании при правильном
подходе должны и ответить за сделанное и вернуть взятое. Все! От Гонтаревой до...
включая владельцев Новой кузни и коксохимов с электростанциями.

Государственный сектор экономики, стратегические отрасли, энергетика и недра и
то, что из них извлечено, не может быть источником обогащения персоналий. Это
абсурд. Возведенный ими в ранг закона для безнаказанного грабежа.

Это не вопрос «сильной руки» - это тонкий голос здравого смысла. Любой бизнес
частный всегда и исключительно в интересах владельца.

Границы допустимого, формат!!!

Никто из растащивших страну по карманам не создавал газотранспортную систему,
не прокладывал ж/д полотна, не создавал шахт и заводов, сталеплавильных печей,
электростанций, заводов тяжелого машиностроения и кондитерских фабрик. Никто!!!
Они просто взяли по праву сильного и ничего не сделали для страны!

Рекламные игры в гуманитария при неудавшемся проекте создания отдельной
автономии и вышедших из-под контроля последствий; тарифные (откровенно
безосновательные) циничные грабежи всех людей ради тупо выдавливания последнего
из их карманов в свой - что ещё нужно сделать, как уничтожить всё здесь от
нормальной жизни каждого до закрытых заводов и уехавшей молодёжи, ставшей вдруг
нормой открытого угнетения с сегрегацией населения по языковым или этническим и
религиозным признакам, чтобы люди очнулись???!!!

Не знаю.

Погружение в морок продолжается.

А зная примеры истории новейшей и не только, нынешнюю усталость и разочарование
с отчаянием большинства, бодренькие глупости тупоголовых (и смешно
самонадеянных в этой тупости) соросят, и весь размах абсурда, который обрядили
в фиговый листок закона, можно сказать уверенно - ничего хорошего нас здесь не
ждёт.

Ничего!

Речевки и лозунги про «понад усэ» - пустопорожняя болтовня на фоне минус 300000
населения ежегодно по стране в балансе смертность-рождаемость, сократившегося и
так не менее чем на 15 млн. человек, ежегодных отъезжающих отсюда в огромном
количестве, 65\% молодёжи, которая намеревается это же исполнить при первой
возможности, подорожавшие вдвое продукты за последний год и перекрывшие в этом
европейские цены на них же кое в чём и вдвое - всё это сулит большую беду. Как
и возросшее раздражение людей, кои составляют более 60\% населения, которых
правильно назвать русскоговорящие украинцы, тем, что их лишили принудительно
конституционно закреплённого права и возможности учить своих детей и внуков в
школах и ВУЗах на том языке, который вдруг - нате вам билет на шоу
«тупой-ещё-тупее» - стал «языком врага». Язык не Достоевского и Пушкина, не
Гумилева и Чехова, не ... - а «язык врага»!

Полный привет. Мозги где?

Всё. Сносит...

Правы. Во всём.

Пожелания мои о самостоятельности - утопия. Здесь или одним лижут или перед
другими в прогибе стоят. А те своё дело и интересы блюдут. Они то точно знают,
что самостоятельными, сильными, союзными меж собой самими мы им не нужны и
опасны!!! Потому что и территория не самая маленькая, и ресурс (был и пока в
остатке кое-что есть, но уже растащили-распродали во множестве от боевого
оснащения до...) вполне, а особенно то, что в недрах и почвы уникальные с
климатическими условиями вместе (особенно сейчас это будет остро ощущаться и
этот год показал наглядно, как перемены могут начаться мгновенно), а вот
людишек многовато, лишние им... и у них выходит, благодаря нашим подлецам! Нашим
шлюхам от политики.

% -------------------------------------
\ii{fbauth.karpij_vladimir.jagotyn.ukraina}
% -------------------------------------

Сколько в мире по-настоящему независимых государств? Уверен, что с десяток и то
с натяжкой. Подавляющее большинство -  квази-государства с атрибутикой и
границами, но с контролируемой элитой. Сможет ли к власти в Канаде прийти
человек, не преданный делу Вашингтона? Этого не может быть, потому что этого не
может быть никогда. Что из себя представляет оччень независимая Словакия?
Финансовая система - подконтрольна, энергетика - подконтрольна, добыча полезных
ископаемых - подконтрольна. Своего ничего нет - всё продано богатым европейцам.
Та же картина и в других Болгариях, Румыниях, Македониях, Словениях, Хорватиях
и даже Австриях с Даниями-Грециями. Страсть как независимые датчане оказались
весьма чуткими к пожеланиям заокеанского старшего брата и врала всему миру с
честными-пречестными глазами, что российский "Севпоток-2" страсть как он вреден
для экологии, и эта вредность существовала до тех пор, пока не надо было давать
разрешение польскому Балтик Пайп . Что, датчане не понимали, что позорятся по
принуждению? Но это называется независимостью малых государств. Зачем миру
такое государство как Киргизия, например, или Таджикистан? Для вскармливания
элит и грабежа собственных граждан. Собственно, для этого государства и
создаются. Я веду к тому, что так или иначе государство должно быть или само
империей или быть частью чужой империи. Третьего не дано. Украина должна быть
частью чьей-то империи. И момент сегодняшний - это определение вектора.
Армрестлинг Запада и Востока на плодородных почвах Неньки. Из этого следует
вывод: государство в Украине всегда будет не для людей. Элита нашего
государства всегда будет петь с чужого голоса. А наше пограничное
западо-восточное геоположение влечёт за собой вот этот армрестлинг. В иных
условиях Украина уже давно бы определилась. Но Галичина не даст Украине жить
спокойно. Думаю, что никогда.

\begin{itemize}
%%%fbauth
%%%fbauth_name
\iusr{Михаил Мищишин}
%%%fbauth_url
%%%fbauth_place
%%%fbauth_id
%%%fbauth_front
%%%fbauth_desc
%%%fbauth_www
%%%fbauth_pic
%%%fbauth_pic portrait
%%%fbauth_pic background
%%%fbauth_pic other
%%%fbauth_tags
%%%fbauth_pubs
%%%endfbauth
 
\textbf{Володимир Карпій} 

Да. Можно быть зависимым, как, например, Прибалтика. Но подпитываться от ЕС и
иметь зарплаты в 1000 евро и выше. Мало. Прибалты уезжают в Англию, Германию,
Францию, США. Их места занимают наши. Но даже при таком абсолютно зависимом и
вассальном государстве что-то да народу перепадает. У нас зависимость еще
больше. До народа не доходит ничего, кроме проблем. Ту же возросшую цену на
газ, как следствие ошибок во внешней и внутренней политике - преспокойно
переложат на плечи потребителей, придумают дополнительные платежи вроде платы
за доставку и т.д. Мы получили шоковую терапию без....европейских зарплат и
пенсий. У облуживающих олигархов они уже есть, и давно. В этом-то и проблема.
Если уж продаетесь, как последние сучки - так поделитесь с малоимущими. Или их
мало в Украине? Но наша власть продается исключительно для себя.\Smiley[1.0][yellow]. И своей
обслуги. А о народе мы вполне достаточно печемся по телевизору.\Smiley[1.0][yellow]. И это,
кстати, прообраз возможного будущего мира. Промывка мозгов Министерством
счастья. Но на практике хорошо живут только работники ТНК (в самом широком
смысле этого слова: заводы, фабрики, научные учреждения, власть, медиа,
политики...) Остальным - ежедневная порция виртуального счастья и вирши про
незалежнисть. Независимость в Украине - это такой фальшборт, прикрывающий новые
ужасные преступления власти против своего народа, в том числе, и - на крови,
сделанные якобы во имя независимости, которой...нет. И в мире, действительно,
очень мало независимых стран. Все зависят друг от друга. По настоящему важно
то, как правительства распоряжаются имеющимся у них ресурсом. И в чью пользу. А
называть это можно как угодно, Вы правы. Можно и в зависимой стране жить
хорошо. И в независимой - плохо. Вообще, в мире есть немало красноречивых
примеров независимых и успешных стран. Это США, Китай, Сингапур и даже Россия.
И Беларусь. И давление на них - это именно что попытка вернуть их в лоно
зависимости от Запада. И, чтобы понять, что такое незалежнисть, достаточно
сравнить Украину с ними. Мы видим, что в их случае народ завоевывает
независимость. А его власть и элита успешно используют полученный ресурс
независимости и отсутствие колониальной ренты во благо своим согражданам.
Уровень жизни в России и даже Беларуси выше, чем в Украине. Вот, собственно, и
все про реформы, захидный шлях, рынок, демократию, свободу и прогресс. В
конечном итоге всегда все просто. В Украине же мы имеем фикцию на фикции. Люди
вполне реально и мучительно умирают за независимость, которой ...нет. А
полученный зависимой властью ресурс используется не во благо народа. А для
обслуживания власти и ее подтанцовки. Вот это украинцам, на мой взгляд, и важно
держать перед глазами. Проще говоря, нас дурять. Скажу ещё проще. Оце и е
зрада.\Smiley[1.0][yellow].

%%%fbauth
%%%fbauth_name
\iusr{Володимир Карпій}
%%%fbauth_url
%%%fbauth_place
%%%fbauth_id
%%%fbauth_front
%%%fbauth_desc
%%%fbauth_www
%%%fbauth_pic
%%%fbauth_pic portrait
%%%fbauth_pic background
%%%fbauth_pic other
%%%fbauth_tags
%%%fbauth_pubs
%%%endfbauth
 
\textbf{Михаил Мищишин} 

Мы плавно подошли к весьма опасному, острому, как лезвие Gillet, вопросу: что
для конкретного человека важнее - нормально жить, не парясь где у тебя столица,
или "заради незалежності топитиму соломою, а не імперським газом". Я когда-то
писал о прочитанной мной статье Г.Померанца в "Вопросах философии" об Эльзасе и
Лотарингии в период между 1870 - 1918. Там жили люди, которым довелось быть под
Францией (до франко-прусской войны) под Германией, снова под Францией (после
Первой мировой). И что? Они жили, занимались бизнесом, общались. Эльзасец месье
Дюбуа выпекал булочки и круассаны как при французах, так и при немцах. А
эльзасец же герр Шульц держал мясную лавку, не расстраиваясь, что раньше
платили франками, а теперь - марками. И главное для них было  - налоги. И жили
они для своих семей, заботились о детях и престарелых родителях. Были, конечно,
патриоты, но основная мысль в статье: люди должны жить не для прапора и герба,
а для себя. И нельзя, чтобы цвет прапора ставал на пути человеческого
благосостояния. Гербы и прапоры - это орудие дьявола, имеющее своей целью
рассорить народы, заставить их убивать друг друга. Убивать за то, чтобы их
грабили мироеды из Парижа , а не из Берлина. Герб и флаг - это отрава мозга,
выкручивающая человеческое сознание так, что сдирание с тебя последней рубашки
во имя этих герба и флага воспринималось, как патриотизмом и высшая гражданская
доблесть.


%%%fbauth
%%%fbauth_name
\iusr{Михаил Мищишин}
%%%fbauth_url
%%%fbauth_place
%%%fbauth_id
%%%fbauth_front
%%%fbauth_desc
%%%fbauth_www
%%%fbauth_pic
%%%fbauth_pic portrait
%%%fbauth_pic background
%%%fbauth_pic other
%%%fbauth_tags
%%%fbauth_pubs
%%%endfbauth
 
\textbf{Володимир Карпій} 

Да. Именно так я и понимаю независимость. Правда, с поправкой - при ней люди
должны жить лучше. По крайней мере, в рамках одного-двух поколений и в
обозримом будущем. Если этого не происходит, людей просто обманывают. И под
лозунги о свободе еще больше закабаляют и грабят. Независимость - это
отсутствие колониальной ренты в виде грабежа полезных ископаемых, денег,
человеческих ресурсов и т.д. Если колониальную ренту платить уже не надо, и
страна реально независима - это должно как-то отражаться в повышении уровня
жизни людей. Вот и все. Именно это я и пытаюсь сказать. Но мы живем в период
красивых, высоких слов и низких дел тех, кто их произносит.\Smiley[1.0][yellow]. И, главное,
Владимир, селянина такое положение устраивает! У него есть огород, закрутки на
зиму, он охотно идет на фронт умирать за якобы независимость. Себя он
обеспечит, и с голоду не умрет. В крайнем случае поедет на заработки за рубеж,
зарежет свинью, бычка, качок на зиму набьет:). Но, шоб гимн и флаг - были
его... И он готов так жить вечно! У него - все хорошо!:). Нормально!:). И
правильно.\Smiley[1.0][yellow]. Не дурный прыдумав робыты з Украины аграрну супердержаву.\Smiley[1.0][yellow].
\end{itemize}

\end{itemize}

