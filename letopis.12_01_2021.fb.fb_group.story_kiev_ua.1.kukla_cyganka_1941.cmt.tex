% vim: keymap=russian-jcukenwin
%%beginhead 
 
%%file 12_01_2021.fb.fb_group.story_kiev_ua.1.kukla_cyganka_1941.cmt
%%parent 12_01_2021.fb.fb_group.story_kiev_ua.1.kukla_cyganka_1941
 
%%url 
 
%%author_id 
%%date 
 
%%tags 
%%title 
 
%%endhead 
\zzSecCmt

\begin{itemize} % {
\iusr{Арт Юрковская}

А еще если вспомнить еврейские погромы 1905 г. и петлюровские.

\begin{itemize} % {
\iusr{Мария Каменцова}
\textbf{Арт Юрковская} . К сожалению, о столь далеких временах у меня нет историй, бабушки и дедушки не сохранили... Эту я из мамы тоже с трудом вытянула!  @igg{fbicon.smile} 

\iusr{Elena Samoylenko}
\textbf{Art Yurkovska} зачем так далеко. 50-е годы 20 века. Борьба передового пролетариата с космополитизмом. Дело врачей.
\end{itemize} % }

\iusr{Надежда Кохан}
Это изумительная история!

\begin{itemize} % {
\iusr{Мария Каменцова}
\textbf{Надежда Кохан} . Благодарю вас. Я долго думала и сомневалась, стоит ли ее публиковать, но все же решилась...  @igg{fbicon.smile} 

\begin{itemize} % {
\iusr{Надежда Кохан}
\textbf{Мария Каменцова} Вы не против, если я размещу эту историю на израильском сайте? Вдруг найдётся семья Лии?

\iusr{Мария Каменцова}
\textbf{Надежда Кохан} . 

С огромной радостью! Мама отказалась, она была уже очень старая (мы поздние
дети), любила покой, который она, несомненно, заслужила, – не захотела
поднимать эту историю, я записала ее для себя, как и много других, я их еще
покажу тут... Кроме того, она просто не знала (или забыла?..) фамилию этих
людей. Но было бы интересно что-либо о них узнать! Все же связь тут,
получается, у нас с ними КРОВНАЯ...

\iusr{Ольга Гондза}
\textbf{Мария Каменцова} вот и у меня так же получилось. Но думаю - главное, что об этом знаем мы и наши дети.

\iusr{Мария Каменцова}
\textbf{Ольга Гондза} . Мы хранители этой информации, и наше дело – постараться ее передать, обнародовать и сохранить...  @igg{fbicon.smile} 

\iusr{Надежда Кохан}
\textbf{Мария Каменцова} я сегодня поделилась Вашей историей. Будем надеяться, что у нее будет продолжение @igg{fbicon.hands.pray} 

\iusr{Мария Каменцова}
\textbf{Надежда Кохан} . Я думаю, будет трудно найти какие-либо следы этой чуть ли не столетней истории. Но, как говорится, чудеса – это по части Бога!  @igg{fbicon.smile} 

\iusr{Надежда Кохан}
\textbf{Мария Каменцова} всегда есть надежда на Чудо!

\iusr{Мария Каменцова}
\textbf{Надежда Кохан} . 

А кроме того, я рассказала только об одной семье, которая нашла у нас тогда
приют... Их, рассказывала мама, было множество! Всех прятали, давали какую-то
одежду ( в Бабий яр евреев доставляли, в холодное осеннее время, полностью
голыми), делились куском хлеба или ложкой горохового супа (почему-то только
горох тогда был в относительном изобилии!), помогали искать родственников или
знакомых, которые соглашались приютить и спасти... Дети бегали по городу с
записками, добирались и в близлежащие поселки!  @igg{fbicon.smile} 

\end{itemize} % }

\iusr{Наталия Слободян}
\textbf{Надежда Кохан},и если будет продолжение, пожалуйста, опубликуйте в группе.

\iusr{Надежда Кохан}
\textbf{Наталия Слободян} это обязательно!!!

\iusr{Мария Каменцова}
\textbf{Надежда Кохан} . 

Тут меня в комментах спросили (еще и с гневом!), почему спасенная семья не
попыталась связаться с нами, оказать помощь со своей стороны... Думаю, ответ
будет и для вас интересен!  @igg{fbicon.smile}  " Тогда другое время было... Даже простая
переписка с родственниками или знакомыми из-за границы могла привести к
печальным последствиям для \enquote{советских граждан}, и я знаю такие случаи лично.
Письма могли не доставляться, какие-либо попытки связи – блокироваться,
позвонить из-за границы было попросту невозможно. А в 1991 году, после развала
Союза и снятия всех ограничений, мы переехали и сменили адрес. Найти нас было
уже трудно. Старшее поколение спасенных к тому времени могло уже уйти
естественным путем, следующее – не знать всего точно... Да и чем, собственно,
они могли нам тут помочь?! Мы вовсе в помощи не нуждались, и возможно, не
приняли бы её! Мама, как я написала в конце рассказа, отказалась искать этих
людей из-за давности событий, или по иной причине...  @igg{fbicon.smile}  Не стоит сердиться на
семью тети Лии, скорее всего, они не смогли связаться с нами по каким-то
серьезным причинам... Там, в Израиле или в Америке, нередки теракты, они могли
попросту погибнуть или пострадать! Не думаю, что причина в неблагодарности.
Могло, к сожалению, случиться всякое... А вам спасибо за ваше горячее
неравнодушие!" @igg{fbicon.heart.suit}

\iusr{Ірина Оснач}
Думаю, варто уточнити, в якій саме школі працювала тьотя Лія. В 70-й?

\end{itemize} % }

\iusr{Helen Terliga}
Какие сообразительные и смелые дети

\iusr{Мария Каменцова}
\textbf{Helen Terliga} . Мамина жертва была велика. Но спасенные жизни того стоили...

\iusr{Larisa Kirova}
Чудесная история... На таких людях держится мир

\begin{itemize} % {
\iusr{Мария Каменцова}
\textbf{Larisa Kirova} . Сколько будущих детей было спасено ценой ужасной гибели одной куклы...  @igg{fbicon.wink} 

\iusr{Larisa Kirova}
\textbf{Мария Каменцова} это правда
\end{itemize} % }

\iusr{Римма Майская}
Очень интересная история, действительно на таких людях мир держится!!

\iusr{Мария Каменцова}
\textbf{Римма Майская} . Сколько будущих детей было спасено ценой ужасной гибели одной куклы...  @igg{fbicon.wink} 

\iusr{Anna Kolesova}
Страшные времена были и спасибо вашей семье за спасение людей. Тот, кто спас одну жизнь, спасает весь мир.

\iusr{Мария Каменцова}
\textbf{Anna Kolesova} . Каждый человек – целая Вселенная.  @igg{fbicon.smile} 

\iusr{Маргарита Марченко}
Спасибо за рассказ!

\iusr{Анна Анна}

в процессе чтения, потеряла Ваш рассказ, лента прошла... Искала с целью
дочитать его. Спасибо, это очень интересная история!

\iusr{Мария Каменцова}
\textbf{Анна Анна} . 

Маленькие герои и праведники потом живут целую большую жизнь, не думая о том,
что они совершили подвиг и прошли по грани бытия и смерти...

\iusr{Irina Sokol}

Есть очень классный рассказ писательницы которая приехала в США в 1994 году и
пишет по-английски, ее зовут Лара Вапняр, рассказ называется «Евреи в моем
доме» или может «В моем доме-евреи», поищите может есть перевод на русский,
если нет, то я как-нибудь найду время, переведу, и поставлю в группе.

\begin{itemize} % {
\iusr{Мария Каменцова}
\textbf{Irina Sokol}. Было бы очень интересно прочитать! Наверное, тема совпадает с моей!  @igg{fbicon.smile} 

\iusr{Irina Sokol}
\textbf{Maria Kamentsova} 

ну рассказов про это время много, просто у неё очень классно написано. Я найду
время и переведу. Уже заказала книгу в библиотеке, к сожалению на интернете
даже английской версии нет чтоб куда-то в транслит вставить и облегчить задачу,
так что буду старым способом с листа переводить.


\iusr{Людмила Мозговая}
\textbf{Irina Sokol} заранее спасибо.

\iusr{Vera Faynberg}
\textbf{Irina Sokol} Спасибо! только что нашла на Google. I will try to download it.

\begin{itemize} % {
\iusr{Irina Sokol}
\textbf{Vera Faynberg} на английском или на русском? Если на русском ставьте в группу, если на английском, пришлите мне пожалуйста, я переведу для участников.

\iusr{Vera Faynberg}
\textbf{Irina Sokol} In English, но я тоже попробую поискать на русском. Если найду, пошлю Вам.

\iusr{Irina Sokol}
\textbf{Vera Faynberg} спасибо, если найдёте на русском ставьте в группу, а пока перейдите мне пожалуйста на английском, я заброшу в Google translate и отредактирую.

\iusr{Vera Faynberg}
\textbf{Irina Sokol} попробую это сделать сейчас.

\iusr{Vera Faynberg}
\textbf{Irina Sokol} I just sent it to you in messenger. Let me know please if you see it.

\end{itemize} % }

\end{itemize} % }

\iusr{Сергей Сереженко}
Очень впечатляет когда ходишь по этой улице сегодня

\end{itemize} % }
