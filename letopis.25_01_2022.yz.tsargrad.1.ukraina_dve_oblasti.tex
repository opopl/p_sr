% vim: keymap=russian-jcukenwin
%%beginhead 
 
%%file 25_01_2022.yz.tsargrad.1.ukraina_dve_oblasti
%%parent 25_01_2022
 
%%url https://zen.yandex.ru/media/tsargrad.tv/ukraine-ostaviat-vsego-lish-dve-oblasti-vse-ostalnoe-russkie-zemli-61efbd5ba4e98573a61f573d
 
%%author_id tsargrad
%%date 
 
%%tags istoria,ukraina
%%title Украине оставят всего лишь две области. Всё остальное - русские земли
 
%%endhead 
 
\subsection{Украине оставят всего лишь две области. Всё остальное - русские земли}
\label{sec:25_01_2022.yz.tsargrad.1.ukraina_dve_oblasti}
 
\Purl{https://zen.yandex.ru/media/tsargrad.tv/ukraine-ostaviat-vsego-lish-dve-oblasti-vse-ostalnoe-russkie-zemli-61efbd5ba4e98573a61f573d}
\ifcmt
 author_begin
   author_id tsargrad
 author_end
\fi

Ситуацию вокруг Украины нагнетают с каждым днём: \enquote{Россия нападёт на Украину. Не
сегодня, так завтра!} – любимая тема рассуждений для западных СМИ и политиков.
При этом другой рукой Запад подталкивает Зеленского к нападению на Донбасс.
Полное впечатление, что скоро там всё равно полыхнёт – как Воронья слободка в
\enquote{Золотом телёнке}, которая просто не могла не сгореть, потому что очень многие
этого хотели. Вот только переживёт ли Незалежная такую политическую авантюру
своего руководства – большой вопрос. Особенно если вспомнить, из каких земель
сложился \enquote{пазл} нынешней Украины и на какие её территории в случае чего будут
претендовать её соседи.

\ifcmt
  ig https://avatars.mds.yandex.net/get-zen_doc/4480952/pub_61efbd5ba4e98573a61f573d_61efbd8eefaff71e04617b72/scale_1200
  @caption ФОТО: VIRTIS/SHUTTERSTOCK.COM
\fi

Чтобы немного разрядить столь напряжённую за последнее время обстановку вокруг
Украины, анекдот в тему, из недавнего прошлого, мало отличающегося от
настоящего:

\begin{zznagolos}
Путин звонит Порошенко и спрашивает: – Ще не вмерла Украина? Порошенко в ответ
смеётся: – Ще не вмерла. Путин: – Хорошо, я тогда попозже позвоню...	
\end{zznagolos}

Смех – смехом, но если не о предсмертных судорогах, то о смертельно опасной
болезни Украины говорить можно давно. Унаследованную от СССР мощную экономику
(60-е место в мире) Незалежная, прямо скажем, раздербанила и профукала. Госдолг
вырос в десятки раз. Население с советских 51 миллиона человек только
официально упало до 42, а фактически (по оценкам многих экспертов) – до 36, да
и то в лучшем случае. Потери в разы бо́льшие, чем в годы Великой Отечественной и
пресловутого \enquote{голодомора} вместе взятых.

Выход у украинских временщиков один – маленькая война. И не то чтобы
\enquote{победоносная}, но позволяющая \enquote{обнулить долги} и запустить на
свою территорию для \enquote{защиты от российской агрессии} войска НАТО. Так,
первый министр госбезопасности ДНР, исполнительный директор Союза добровольцев
Донбасса Андрей Пинчук в комментарии \enquote{Первому русскому} подчеркнул:

\begin{zzquote}
Сегодня имеет место высочайший уровень вероятности войны. Режим президента
Зеленского шатается, и для него война – единственная возможность удержать
власть. Но Зеленский – лишь кукла в руках Запада, которому нужна война. Чтобы
\enquote{сдержать} Россию, начавшую перезагрузку ОДКБ и дипломатическое
наступление. И преодолеть фактор раскола в Европе по отношению к России. Для
прекращения \enquote{излишней самостоятельности} ряда стран под предлогом
\enquote{атлантического единства} необходимо дружить против общего врага.	
\end{zzquote}

Вот только закончиться эта авантюра может для Украины... полным распадом этой
страны, а точнее, недогосударства.

\subsubsection{Границы Украины: откуда что взялось?}

Можно сколько угодно повторять завиральные теории одного из творцов украинского
псевдоисторического мифа Михаила Грушевского о никогда не существовавшей в
реальности \enquote{козацкой государственности}. Но факты – упрямая вещь (причём в
данном случае не только в отношении Незалежной). Практически все крупные города
Украины построены во времена нашей общей русской государственности (будь то
Киевская Русь, Российская Империя или СССР).

Так, Харьков основан царём Алексеем Михайловичем в 1630-х годах и заселялся
эмигрантами в Россию из \enquote{польского} правобережья Днепра. С городом Сумы
(основан в 1655 году) – та же история. Императрицей Екатериной II во второй
половине XVIII века основаны Днепропетровск (Екатеринослав), Луганск, Херсон,
Николаев, Одесса, Севастополь, Симферополь, Мариуполь, Кривой Рог, Запорожье.
Добавьте сюда основанный в 1754 году Елизаветой Петровной с недавних пор
именуемый Кропивницким Кировоград (Елисаветград) и Александром II Донецк
(1869), и что тогда от Украины останется?

Ах, да – Чернигов, старинный русский город, известный с X века. Но и он в
составе России был с 1503 года (исключая период 1611–1653 годов, когда его
отобрали и разорили поляки). Киев, кстати, юридически – тоже не украинский: его
поляки передали России в 1667 году по условиям Андрусовского перемирия, сначала
на время, а потом – продали за немалые по тем временам деньги (146 тысяч
рублей). Так что и здесь, пардон, это наша \enquote{нерушимая частная собственность}.
