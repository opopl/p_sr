% vim: keymap=russian-jcukenwin
%%beginhead 
 
%%file 21_11_2020.news.ua.strana.1.maidan_7_let
%%parent 21_11_2020
 
%%url https://strana.ua/news/302150-evromajdan-kohda-nachalas-revoljutsija-dostoinstva-prichiny-i-khronolohija-sobytij.html
 
%%author 
%%author_id 
%%author_url 
 
%%tags maidan
%%title От "хорошего настроения" до массовых убийств. Почему в Украине семь лет назад случился Евромайдан
 
%%endhead 
 
\subsection{От \enquote{хорошего настроения} до массовых убийств. Почему в Украине семь лет назад случился Евромайдан}
\label{sec:21_11_2020.news.ua.strana.1.maidan_7_let}
\Purl{https://strana.ua/news/302150-evromajdan-kohda-nachalas-revoljutsija-dostoinstva-prichiny-i-khronolohija-sobytij.html}

\ifcmt
pic https://strana.ua/img/article/3021/50_main.jpeg
caption Массовые протесты в Украине. Евромайдан 
\fi

Сегодня --- седьмая годовщина Евромайдана.\Furl{https://strana.ua/news/301411-hodovshchina-majdana-kak-otmetjat-7-letie-revoljutsii-dostoinstva-i-den-svobody.html}

Хотя это событие от нас все дальше, сложно найти другое, так сильно
повлиявшее на современную историю Украины.

Поэтому напомним вкратце, что происходило в ноябре 2013 --- феврале 2014. И
каков истинный смысл тех событий.

\subsubsection{Предпосылки Евромайдана}

Президент Виктор Янукович в ноябре 2013 года внезапно повернул вспять процесс
ассоциации с Евросоюзом. Власти поняли, что соглашение с ЕС в перспективе лишит
Украину российского рынка, но европейским это не компенсируется.

ЕС же отказывался идти на уступки --- как в увеличении квот для украинского
экспорта, так и в предоставлении украинской экономике инвестиций, которые бы
сгладили потерю старых рынков.

Закономерный вопрос --- зачем тогда вообще было начинать всю историю с
Евросоюзом?

Причин здесь было две. Первая --- многовекторность. Янукович, который в первый
год своего президентства подписал Харьковские соглашения с Россией и вызвал вал
обвинений в пророссийской политике, хотел \enquote{диверсифицировать геополитические
риски} и усилиться теперь на западном направлении. Против чего не возражали и
многие группы в Партии регионов. 

Заигрывание регионалов с Европой в свою очередь должны было, по их мнению,
сподвигнуть Москву на некие уступки --- к примеру, по цене на газ. 

Вторая причина --- Запад хотел вовлечь Украину в ассоциацию, чтобы обнулить
возможность вхождения нашей страны в Таможенный союз с Россией, что тогда
предлагала Киеву Москва. То есть это был элемент геополитической
конкуренции. Кроме того, Соглашение ставило Украину в экономическую (за
которой следовала и политическая) зависимость от ЕС и Запада в целом. Чего
у Януковича сначала не понимали. А когда поняли (в том числе посчитав
убытки для экономики) --- процесс начали сворачивать. 

Кроме того, после ряда переговоров Януковича и Путина Москва наконец-то
пообещала дать Киеву то, чего украинские власти давно добивались, --- скидку
на газ и крупный кредит. Причем без всяких политических условий (вроде
вступления в Таможенный союз). Да и вообще почти без условий, кроме одного
– отказаться от Соглашения об ассоциации с ЕС.

В итоге 21 ноября Кабмин приостановил подготовку к подписанию документа. 

\subsubsection{Начало протестов и первая радикализация}

21 ноября журналист Мустафа Найем бросил в Facebook клич выходить на
Майдан, что и стало формальным началом протестов. 

\enquote{Встречаемся в 22:30 под монументом Независимости. Одевайтесь тепло,
берите зонтики, чай, кофе, хорошее настроение и друзей}, --- написал Найем в
Facebook.

\ifcmt
pic https://strana.ua/img/forall/u/0/92/b1a935572021340caf9619e2e29904ad[1].png
\fi

Правда, были они сначала не слишком впечатляющими. И вряд ли настолько
отвлеченная вещь, как ассоциация с ЕС (которая не тождественна вступлению
в Евросоюз), могла бы вывести на улицы сотни тысяч людей. Но в дело
включились силы, которые начали раскручивать протесты. 

С одной стороны это Запад, который был недоволен срывом Соглашения. С
другой --- прозападная и националистическая оппозиция --- Яценюк, Кличко,
Тягнибок и другие.

Но главное, что против Януковича, по сути, восстала часть Партии регионов.
Олигархи, которые имели свои доли в этой политической силе, были
недовольны тем, что вес набирает \enquote{Семья} президента. И негласно
подключились к новым протестам --- прежде всего, за счет своих СМИ. 

Поэтому протесты достаточно быстро переросли в массовые. В том числе после
явных провокаций или же просто непродуманных шагов со стороны власти. Уже
30 ноября \enquote{Беркут} избил оказавших сопротивление протестующих на Майдане.
Все это СМИ (включая те, которые контролировались \enquote{регионалами})
транслировали в прямом эфире, подавая как \enquote{избиение детей} (хотя
студентов, как потом выяснилось, среди пострадавших было немного). 

После этого на улицы вышло гораздо больше людей. И уже первого декабря
протесты вошли в стадию радикализации. Протестующие пытались штурмовать
администрацию президента, нападали на кордоны милиции, избивали
правоохранителей. В итоге их разогнал \enquote{Беркут}. 

То есть, с одной стороны, это было \enquote{боевое крещение} праворадикалов,
которые вышли на Майдан (и политики, поддержавшие протесты, тут же
попытались от них отмежеваться).

С другой --- многие в Украине поняли, что протест \enquote{за Европу} отнюдь не
мирный. Что уже тогда начало работать на раскол страны --- попытки
националистов свергнуть Януковича силовым путем явно не находили понимания
на юго-востоке.

После этих событий стало понятно, что быстрой победы над властью не будет.
Но и Майдан уже имел серьезную поддержку, и зачистить его силой было,
конечно, возможно, но Запад к тому времени протесты уже открыто поддержал.
И требовал от Януковича силовых методов не применять.

Параллельно все больший вес набирали националисты. 8 декабря \enquote{свободовцам}
позволили свалить памятник Ленину на Бессарабке.

Чего бы не могло произойти без молчаливого содействия властей, которые
решали свои узко-тактические задачи, пытаясь демонизировать \enquote{Свободу}
накануне довыборов в Раду по одному из киевских округов. Там нужно было
завести в парламент Виктора Пилипишина и не пустить туда \enquote{свободовца} Юрия
Левченко --- и это удалось.

Об издержках --- усилении праворадикалов на Майдане и в целом в Украине –
тогда, видимо, никто не думал.  

\subsubsection{Разворот к России }

Янукович в ответ на протесты и западную их поддержку начал резко
разворачивать внешнеполитический вектор (о чем, как писалось выше, он еще
до того договорился с Путиным).

17 декабря он привез из Москвы масштабное соглашение --- о почти двойной
скидке на газ, 15 миллиардах кредита без всяких формальных условий, а
также инвестициях России в украинские высокотехнологические отрасли –
строительство кораблей, самолетов и так далее. 

И все это --- без вступления в Таможенный союз. То есть власть Януковича
получила весомую поддержку с востока.

После этого стало окончательно очевидно, что быстрого свержения президента
не будет и дело затягивается.

Лидеры Майдана оказались перед угрозой его маргинализации и потери
поддержки.

Но тонус сопротивления постоянно поддерживался различными загадочными
событиями. Вроде избиения 26 декабря бывшей журналистки Татьяны Черновол
на Бориспольской трассе.

По официальной версии, это была бытовуха: Черновол ввязалась в дорожный
конфликт. Но на Майдане заявили об атаке властей на активистку. Что лишь
подлило масла в пламя протестов. 

Впрочем, фактор Черновол и ему подобные не могли долго поддерживать градус
протестов --- и в начале января они более-менее стабилизировались,
превратившись в позиционный конфликт. Было понятно, что таким образом
власть не взять.

Регионалы решили воспользоваться этой паузой, чтобы перейти в наступление.

16 января они проголосовали за законы, которые ужесточали ответственность
за все, чем занимались майдановцы: хождение в балаклавах, поездки
Автомайдана домой к чиновникам и депутатам, нападения на милицию и так
далее. 

Большинство приняло эти законы, видимо, будучи уверенным в том, что
протесты сдуваются и таким образом их можно додавить.

На Майдане также понимали это и назвали законы \enquote{диктаторскими}. 

Но поначалу его вожди выглядели растерянными, не понимая, что делать
дальше. 

И тут в дело вступила радикальная часть майдановцев, которая перевела
протест в силовую стадию, атаковав правоохранителей на улице Грушевского.
Начали массово применяться \enquote{коктейли Молотова}.

Со стороны это выглядело настолько дико, что руководство протестами –
Яценюк, Кличко, Тягнибок --- от акции тут же отмежевались, не поддержал
происходящее и Запад.

Такая реакция была вполне понятна. Атака на силовиков давала повод властям
разогнать Майдан. 

Но президент Янукович, вопреки ожиданиям, не дал такого приказа, видимо,
рассчитывая при поддержке Запада как-то успокоить ситуацию. Тем более, что
и с Запада к нему шли настоятельные просьбы воздержаться от радикальных
действий.

По итогу силовое противостояние на Грушевского затянулось на несколько
дней. А радикалы, которые атаковали правоохранителей, стали в глазах
майдановцев не провокаторами, а героями.  

Окончательный же перелом произошел 22 января. Рано утром в тот день при
непонятных обстоятельствах были убиты активисты Майдана --- армянин Нигоян и
белорус Жизневский. По официальной версии сегодня, их убили некие
сотрудники МВД --- при этом подозреваемых в деле нет. По версии предыдущих
властей --- стреляли с трех метров, то есть свои --- провокаторы, целью
которых была дальнейшая радикализация протестов.

Как бы то ни было, гибель этих двух майдановцев сильно подорвала позиции
власти. На западе и в центре страны начались захваты обладминистраций. А в
Киеве, напуганный первыми смертями Янукович, с одобрения Запада, пошел на
переговоры с лидерами оппозиции. 

Их первыми итогами стали уступки со стороны президента --- в конце января он
вынудил уйти в отставку премьера Николая Азарова (вместо него и.о.
премьера стал представитель \enquote{Семьи} вице-премьер Сергей Арбузов, что
вызвало дополнительный раздрай внутри Партии регионов). 

Более того, после этого Янукович предложил пост премьера Яценюку. Что,
вероятно, было  попыткой пойти на компромисс с Западом и расколоть
майдановский лагерь.

Однако со стороны это выглядело как проявление слабости со стороны
президента. Да и на протестах уже настолько сильны были позиции радикалов,
что никто на такие условия пойти не согласился. 

У оппозиции к тому времени созрел иной план --- заставить президента и
Партию регионов вернуть Конституцию 2004 года (с уменьшением полномочий
президента в пользу парламента и правительства), сформировать свое
правительство и, фактически, взять власть в стране, а потом договориться с
Януковичем о его уходе. 

Но самого Януковича такой план не устраивал, и он пытался маневрировать,
уговаривая Запад как-то повлиять на оппозицию. Запад в ответ склонял
Януковича к новому геополитическому развороту --- от России к ЕС. 

Москва же, видя метания украинской власти, заморозила очередной транш из
согласованного в декабре кредита. Эти деньги были нужны для стабилизации
гривны.

В общем, ситуация зашла в тупик, из которого стороны попытались выйти
силовым путем.

18 февраля майдановцы пытались штурмовать Раду и продавить возврат к
Конституции образца 2004 года.

Но силовики отбили штурм и пошли в наступление. До 20 февраля они
зачистили все прилегающие к Майдану улицы, оставив под контролем
протестующих лишь саму площадь.

Все ждали приказа на зачистку эпицентра протестов, силовики были к ней
готовы. Вечером 18 февраля они начали продвижение и заняли часть площади,
но были остановлены стеной из горящих шин и получили приказ остановиться
на занятых позициях.

К тому времени с обеих сторон уже были погибшие и раненые.

Одновременно в городе действовали подконтрольные власти титушки, они убили
журналиста \enquote{Вестей} Вячеслава Веремия.

На следующий день прошла информация, что власти собираются объявить режим
Антитеррористической операции и зачистить Майдан. Но вместо этого было
объявлено о начале переговоров Януковича и лидеров оппозиции, которые
заключили \enquote{перемирие} (впрочем, \enquote{Правый сектор} тут же заявил, что его не
признает). А 20 февраля в Киев должны были приехать министры иностранных
дел Польши, Германии и Франции (позже к ним присоединился представитель
России Лукин), которые должны были согласовать компромиссное политическое
решение.

Но 20 февраля утром произошла бойня на улице Институтской, в которой
погибло более 50 человек --- как протестующих, так и милиционеров.

По данным материалов уголовного дела (его расследовали уже после Майдана),
которые публиковала \enquote{Страна}, первые выстрелы прозвучали со стороны
протестующих (их осуществляла \enquote{группа Парасюка}). Также эту информацию
подтверждал львовский майдановец Иван Бубенчик. 

То есть первыми открыли стрельбу майдановцы, что и спровоцировало всю
последующую бойню.

Но сейчас за убийства судят только \enquote{беркутовцев}. По официальной версии,
они открыли огонь по майдановцам и убили на Институтской десятки людей. Но
сами \enquote{беркутовцы} это опровергают. Более того, еще в 2017 году появились
свидетельства граждан Грузии, которые заявили, что они стреляли по
майдановцам по приказу лидеров оппозиции. 

Как бы то ни было, но массовые убийства резко поменяли политический
расклад в пользу Майдана. Запад начал грозить Януковичу санкциями, а
провластное большинство в Раде окончательно развалилось.

На этом фоне деморализованный президент согласился вернуть Конституцию
2004 года в обмен на то, что он остается у власти как минимум до осени
2014 года (когда должны пройти досрочные выборы президента).

Это и было основой компромиссного соглашения, которое было заключено между
президентом и лидерами оппозиции 21 февраля при посредничестве глав МИД
Франции, Польши и Германии (представитель России уехал до его подписания).

Но уже вечером на Майдане выступил сотник Парасюк и заявил, что его
выполнять не будут. А так как, по тому же соглашению Янукович вывел
правительственные войска из Киева, то фактически весь центр столицы
оказался в руках майдановцев (которые, к тому времени, уже обзавелись
оружием). 

В таких условиях Янукович решил спешно покинуть столицу и уехал в Харьков,
где 22 февраля должен был пройти съезд депутатов юго-востока. Но соратники
президента из Партии регионов уже чувствовали, что он проигрывает, а
потому фактически не оказали президенту никакой поддержки, начиная уже
договариваться с лидерами Майдана.

В Киеве же с утра 22 февраля майдановцы перехватили контроль над
парламентом, назначили спикером Турчинова, а затем объявили о прекращении
полномочий Виктора Януковича. Последнее решение было полностью
неконституционным и окончательно разрушало договоренности от 21 февраля.
Но Запад его фактически признал.

Оказавшись без какой-либо поддержки, Янукович уехал через Крым в Россию. 

На этом \enquote{Революция достоинства} формально была окончена, и начинался новый
– и уже куда более трагический --- этап в жизни страны. 

\subsubsection{Крым, Донбасс, война. Итоги Евромайдана}

Одним из первых решений новой власти стала отмена 23 февраля закона о
языках, который давал русскому статус регионального. 

Это событие --- да и в целом захват власти в Киеве
радикально-националистическими силами ---  запустили процесс полураспада
Украины. На юго-востоке начались массовые протесты, которые поддержала
Россия. 

Дальнейшее известно. В Крыму выступления против Киева стали катализатором
аннексии и включения в состав РФ. А на Донбассе вызвали к жизни появление
\enquote{ДНР} и \enquote{ЛНР} и уже более чем шестилетнюю войну.

Сторонники Майдана говорят, что их протесты вообще никак не связаны с
такими плачевными для Украины последствиями --- мол, их причиной стали
исключительно действия России. 

Однако очевидно, что Майдан изначально не сплотил общество, а расколол
его. Стало понятно, что в случае его победы нелегитимным путем (то есть не
через выборы), миллионы людей на юго-востоке не примут смену власти. Что и
произошло. 

Для Украины же в целом протесты 2013-2014 годов означали одно --- полную
смену геополитического курса, конец многовекторности и движение в русле
политики Запада.

Учитывая, что долгие годы страна была завязана на российские рынки и
энергоносители из России, столь резкий разворот (вместе с началом войны)
вызвал экономический коллапс, обвал экспорта и нарастающую бедность.  

Итогом стал многократный рост коммунальных тарифов, падение гривны,
производства, отмена многих социальных стандартов. И как следствие этих
факторов --- массовая эмиграция населения. 

Впрочем, на официальном уровне сегодняшний день остается праздником --- Днем
достоинства и свободы. И президент Зеленский уже записал свои
поздравления. 

О последствиях, к которым этот день привел, нынешняя власть, так же, как и
прежняя, предпочитает не вспоминать.

