% vim: keymap=russian-jcukenwin
%%beginhead 
 
%%file posts.init
%%parent posts
 
%%url 
 
%%author_id 
%%date 
 
%%tags 
%%title 
 
%%endhead 

ЯКБИ ТАРАС ШЕВЧЕНКО ЖИВ ЗАРАЗ...
(І як це пов'язано між собою із Ненацьком, БознаДе, і ВзагаліЩоЦеЗаМаячня)
ТО...
Оце от натрапив щойно... скопіював у Сергій Іванович. Але, як я бачу, цей текст є в стрічці в інших людей... Мене це вразило трохи, а чому, стане пізніш. Але спочатку... вставлю тут частину того тексту... 
якби Тарас Григорович Шевченко жив зараз і вів би фейсбук, то це було б так. 
Пише Тарас Григорович у фейсбуку: 
➤
"Садок вишневий коло хати. 
Хрущі над вишнями гудуть. 
Плугатарі з плугами йдуть. 
Співають ідучи дівчата. 
А матері вечерять ждуть."
Коментарі: 
➤
"Вишні біля хати не треба саджати, біля хати треба саджати горіх, бо горіх то дуже від мух добре". 
➤
"Слухайте, чому вечерю завжди має готувати мати? А мати що, не напрацювалася за день? Ось у мене в родині завжди готує чоловік - и знаєте, так смачно готує!" 
(комент до цього коменту: "Він що в тебе, гей, що на кухні порається? Кухня - то не для чоловіків!")
Текст Martina Doe
Тарас Шевченко
Дивлюся, аж світає...
Осип Курилас, 15 грудня 1918 p.
ЦЕ Я ДО ЧОГО. 
А того... А я так думаю, от одразу прийшла думка... Що, якщо Тарас Шевченко жив би зараз, то....
Ееееееем. Справа то вся в тому... що... якщо Тарас Шевченко 
жив би зараз, то... він не був би Тарасом Шевченко... От як це. Якщо людина зветься Тарас Шевченко, то значить ця людина - Тарас Шевченко? Ну... так... і не так... одночасно...  Ну тобто. Він звичайно називався би Шевченко Тарас Григорович. громадянин України, народився такого то дня, такого то місяця, року скажімо 1970. Також, Шевченко Тарас Григорович
скажімо мав би машину Шкода Октавія, любив би книжки, мав би художню освіту, навіть вів би фейсбук... і т.д. і т.д. але це... не був би Тарас Шевченко. А як це так!? Тарас Шевченко і вже не Тарас Шевченко!?
Хм... а уся справа знаєте в чому. В Асоціаціях і Історичному Контексті, тобто Зв'язку з власне історією самого Тараса Григоровича, ну тобто що він народився, був кріпаком, його викупили, він закінчив Академію Художеств в Петербурзі... потім, був членом Кирило-Мефодіївського Товариства, його арештували, заборонили писати та малювати, потім, зіслали в Оренбург, він був простим солдатом, потім, це щодо його життя, і творчість, те, за те його ми власне любимо, його невмируще Слово +
