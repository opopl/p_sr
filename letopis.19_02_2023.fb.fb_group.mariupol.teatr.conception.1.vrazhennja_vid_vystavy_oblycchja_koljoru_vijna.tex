%%beginhead 
 
%%file 19_02_2023.fb.fb_group.mariupol.teatr.conception.1.vrazhennja_vid_vystavy_oblycchja_koljoru_vijna
%%parent 19_02_2023
 
%%url https://www.facebook.com/groups/conception.theatre/posts/2230945583735555
 
%%author_id fb_group.mariupol.teatr.conception,sosnovskij_evgenij.mariupol
%%date 19_02_2023
 
%%tags 
%%title Враження від вистави "Обличчя кольору війна"
 
%%endhead 

\subsection{Враження від вистави "Обличчя кольору війна"}
\label{sec:19_02_2023.fb.fb_group.mariupol.teatr.conception.1.vrazhennja_vid_vystavy_oblycchja_koljoru_vijna}
 
\Purl{https://www.facebook.com/groups/conception.theatre/posts/2230945583735555}
\ifcmt
 author_begin
   author_id fb_group.mariupol.teatr.conception,sosnovskij_evgenij.mariupol
 author_end
\fi

Наш глядач Ігор Р. про свої враження від вистави \enquote{Обличчя кольору війна} (18
лютого, Національний академічний драматичний театр ім. Лесі Українки):

\begin{quote}
\em\enquote{Театральне мистецтво раніше не було пов'язане з особистим болем. Так, серед
багатьох вистав я назву десяток так званих сльозогінних, але лиш за рахунок
авторського задуму і майстерної гри акторів, проте вони не торкалися близького.
А Маріуполь я знаю сонячним, солоним від морського повітря, промисловим, бо
навіть здалеку я десятки разів відчував наближення до міста. В Маріуполі у мене
друзі і купа знайомих. Були? Були. Тепер вони, можливо, знаходяться на... на
половині планети. Хтось перелетів аж за океан. І це тільки ті, про кого я щось
знаю.

На сцені – маріупольські актори. В залі – теж частина тих, хто знає місто краще
за мене, і ще –знає його таким, яким не знаю я. П'єса маріупольців про самих
себе в ті березневі дні 2022 року. Я не скажу, що нас – українців можуть
приголомшити якісь нечувані факти. Ми вже рік живемо в суцільному потоці новин
про злочини на нашій землі, вчинені сусідами. Вони здійснюються прямо зараз,
коли ми ходимо вулицями Києва, виконуємо нашу роботу, купуємо перепічку поблизу
Хрещатику або жартуємо з колегами. Всі новини, побачені відео з фронту – вони
емоційні, але те, як актори грають самих себе...

Далі – світлини і прохання не лишатися спостерігачами. Навіть знаходячись в
тилу, ми здатні хоч на щось. Віддати заради перемоги – щось, та й можна.
Підтримати тих, в кого є особливі потреби поруч із нами – хоча би і так. Знайти
своє місце в країні важливо, доки ця країна ще є у нас і, сподіваюся, буде.}
\end{quote}

\emph{Фото: Ігор Рубцов}

\#ТеатрConception \#ConceptionTheatre \#ОбличчяКольоруВійна
