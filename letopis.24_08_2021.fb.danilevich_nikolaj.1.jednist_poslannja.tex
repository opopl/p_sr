% vim: keymap=russian-jcukenwin
%%beginhead 
 
%%file 24_08_2021.fb.danilevich_nikolaj.1.jednist_poslannja
%%parent 24_08_2021
 
%%url https://www.facebook.com/mykola.danylevych/posts/4586690444685401
 
%%author Данилевич, Николай
%%author_id danilevich_nikolaj
%%author_url 
 
%%tags edinstvo,nacia,nezalezhnist,ukraina,zelenskii_vladimir
%%title Послання внутрішньої єдності
 
%%endhead 
 
\subsection{Послання внутрішньої єдності}
\label{sec:24_08_2021.fb.danilevich_nikolaj.1.jednist_poslannja}
 
\Purl{https://www.facebook.com/mykola.danylevych/posts/4586690444685401}
\ifcmt
 author_begin
   author_id danilevich_nikolaj
 author_end
\fi

Послання внутрішньої єдності.

Загалом, мені сподобалася сьогоднішня промова Президента В. Зеленського з
нагоди Дня незалежності. Зазвичай, це стратегічний текст, яким глава держави
хоче послати якісь ідеї своєму народу.  

Цьогорічна промова містить в собі об’єднавчі ідеї. Видно, що Президент і його
команда намагаються повернути в країну ідеологію внутрішньої єдності,
інклюзивності, і вийти з лекал крайньої правої ідеології, яка панувала в нашій
країні при попередньому очільнику держави. 

\ifcmt
  pic https://www.president.gov.ua/storage/j-image-storage/20/13/61/a7d09147380a584b2a1e8c70e6be3b2e_1629797504_extra_large.jpeg
  width 0.4
\fi

1. Найперше, Президент звертається: «Шановні громадяни України! Дорогий український народе!».

Не «Шановні українці!», а «громадяни України!». Є різниця.  

Нижче в промові Президент В. Зеленський уточнює: «Сьогодні вперше в історії
хочу назвати тих, хто живе в Україні, щоб ми нарешті зрозуміли, що український
народ – це українці, кримські татари, караїми, кримчаки, росіяни, білоруси,
молдавани, болгари, угорці, румуни, поляки … А всі ми – громадяни України.
Збірна України. Одна команда». 

Тобто, Україна багатоетнічна, і всі ми є членами однієї великої сім’ї. Немає
поділення на «українців» і «неукраїнців», «патріотів» і «ватників», на
«правильних» і «неправильних». «… такі різні, зі сходу й заходу, україномовні
та русскоговорящие, маємо бути однією родиною. Бо об'єднує нас Україна», -
говорить далі Президент. В минулі роки Президент якось сказав крилату фразу «Ми
– різні, але рівні». Цього року приблизно те саме, але іншими словами, і більш
розлого. 

2. В своїй промові Президент намагається дистанціюватися від всього того, що
розділяє. Уникає крайнощів. В промові згадуються різні регіони України, кожен з
яких робить свій внесок в наше спільне державне життя. І ось, зокрема, важливий
пасаж: «Рівненщина подарувала нам першого Президента України. Чернігівщина –
другого. Сумщина – третього. Донеччина подарувала нам …» 

І тут, (хто дивився пряму трансляцію), Президент зробив паузу. Чи назве він
Януковича? «…Леоніда Бикова та Сергія Бубку». 

Всі засміялися. Зеленський не згадав Януковича. Але, що важливо! Він не згадав
далі і Порошенка, який сидів поряд з іншими екс-Президентами (Кучмою і
Ющенком), яких згадав Зеленський. 

В промовах високих людей слід звертати увагу не лише на те, що сказано, але й
на те, про що (і про кого) не сказано / не згадано.

Чому не згадав Януковича – це зрозуміло. А от Порошенка не згадав, на мій
погляд, тому що той також уособлює собою іншу – крайню праву ідеологію,
ідеологію меншості, хай і пасіонарної, але меншості жителів України.
Відчувається, що Президент намагається шукати ті ідеологічні речі, які сприяють
формуванню загальноукраїнської єдності і уникати тих, які цьому не сприяють.
Здається, що нинішній Президент таки намагається формувати певний ідеологічний
центризм. Принаймні в офіційних текстах це відчувається.  Не просто одразу
розвернути країну в сторону центризму після стількох років іншої ідеології, але
якісь спроби робляться. 

3. Приємно було чути згадки про християнські корені нашої країни, про
Православ’я, про Київ, з якого «… починалося православ’я, тут взяла початок
старослов’янська мова, правонаступницею якої є сучасна українська мова. Тут
почалась і наша державність. Зародження нашої державності ми будемо відзначати
в день розквіту нашої державності – у день Хрещення Київської Русі – України. І
про все це я підписую сьогодні відповідний указ».

Таким чином, Президент встановив День української державності, в прямому ефірі підписавши відповідний указ.

Святкуватиметься тепер це свято – 28 липня, в день пам’яті св. кн. Володимира.
За останні роки це свято широко відзначається православними, проводяться
масштабні хресні ходи. Тому, навіть якщо держава святкуватиме це свято
формально, протокольно, то Церква про це не забуде і святкуватиме всенародно,
бо це співпадає з церковним святом. 

4. В церковній сфері, на жаль, немає сьогодні єдності. Тому, Президент згадавши
про християнство загалом, не згадав про якусь Церкву чи конфесію окремо. Ні про
Томос, ні про ПЦУ, ні про Константинополь, незважаючи на те, що
Константинопольський патріарх Варфоломій стояв на Майдані серед запрошених
гостей, - Президент не згадав.

Був від Президента лише натяк на необхідність міжконфесійного миру, а також на
необхідність плекання єдності в різноманітності, в тому місці, де було сказано: 

«Без Лаври та Софії, костелу Святого Миколая й собору Святого Юра, ханської
мечеті в Бахчисараї та синагоги у Дніпрі Україна не така багатогранна», -
цитата Президента.

P. S. Очевидно, що плекання єдності країни – це завдання для кожного
Президента. В нашому українському випадку Президент має своїми словами і діями
об’єднувати, або хоча б утримувати баланс між різними силами в країні. Мені, як
громадянину України, близькі ці об'єднавчі ідеї, які сьогодні прозвучали з уст
Президента. Близькими ці ідеї є й для нашої Церкви, адже такі самі ідеї УПЦ
висловлювала й раніше. Хотілося б, щоб наша влада не лише декларувала, але й на
ділі мала сили і можливості реалізовувати їх. І ми її в цьому підтримаємо. 

P. P. S. Вітаю всіх, хто мене читає, з Днем незалежності нашої України,
громадянами якої ми є. Ми любимо нашу країну, бо ми є її органічною частиною.
Ми можемо по-різному дивитися на її сучасний стан, на її майбутнє, на шляхи її
розвитку, і маємо не лише право, а навіть обов’язок відкрито говорити про це,
переживати за нашу державу, брати участь в розбудові нашої країни, дискутувати,
не погоджуватися з іншими думками, відстоювати своє бачення, адже всі ми –
громадяни України, але разом з тим, ми покликані поважати один одного в нашій
іншості, поважати думку і позицію інших громадян країни, дослуховуватися,
шукати компроміси. А також спільно, кожен на своєму місці, працювати для її
розвитку та процвітання. 

Мені було приємно сьогодні бачити, як Президент нагороджував простих солдат,
простих вчителів, простих людей з різних сфер нашого життя, адже саме прості
працівники щоденно і непомітно рухають прогрес, розвивають нашу країну.

Бажаю, щоб всі ми намагалися чути один одного, розуміти і поважати один одного,
і тим самим будувати єдність, мирне і благополучне спільне життя нашої країни. 

Зі святом! З Днем незалежності!
