% vim: keymap=russian-jcukenwin
%%beginhead 
 
%%file 27_12_2021.fb.fb_group.story_kiev_ua.1.odnoj_krovi_ty_i_ja.cmt
%%parent 27_12_2021.fb.fb_group.story_kiev_ua.1.odnoj_krovi_ty_i_ja
 
%%url 
 
%%author_id 
%%date 
 
%%tags 
%%title 
 
%%endhead 
\zzSecCmt

\begin{itemize} % {
\iusr{Ирине Вильчинская}

я тоже люблю базары. Причем, на базаре смотрю не только на товар, но и на лицо
продавца, на его поведение. От злых, недоброжелательных или матерящихся ухожу
сразу. Базар - это совсем другая атмосфера \enquote{шопинга}! Базар - это
ОТНОШЕНИЯ.

\iusr{Елена Мирошниченко}
Отлично написано.

\iusr{Віталій Пономаренко}
Не все базарные собаки бесполезны, есть такие крысоловы, что котам фору дадут!

\iusr{Ирине Вильчинская}
\textbf{Віталій Пономаренко} это правда!

\iusr{Надежда Горицкая}
\textbf{Люблю Владимирский}, 

но покупаю там только мясо. Собак нет, охрана, котов нет. Раньше, лет 10 назад
к куриным ларькам приходили собаки с Байкового, был один огромный рыжий добряк,
умел проходить с людьми дорогу по светофору, и приводил с собой ещё 2 девочки.
Потом в 10-м отловили и все. Уничтожили. Сейчас в одном ларьке появилась кошка.


\iusr{Андрей Пожарский}
Да нет уже на базарах добродушных тетек. Да и базаров уже почти нет.

\begin{itemize} % {
\iusr{Галина Полякова}
\textbf{Андрей Пожарский} В Москве, может быть, и нет. Не знаю. А у нас в Киеве есть.

\iusr{Андрей Пожарский}
\textbf{Галина Полякова} где?
\end{itemize} % }

\iusr{Светлана Власова}
Да, и я тоже предпочитаю рыночки. Где ты знаешь, и тебя ждут

\iusr{Тамара Ар}

Люблю Дарницкий старый базар с большим павильоном послевоенной постройки начала
50 х, расположенный возле большого парка

\begin{itemize} % {
\iusr{Ирина Попова}
\textbf{Тамара Ар} , единственный, который сохранил рыночный колорит, мы с правого берега туда ездили!
\end{itemize} % }

\iusr{Наталія Громова}

А я не люблю базары, да и давно это не те базары что были, я помню Житний -
жила на Подоле,сейчас на Виноградаре, Одни перекупщики, обвешивают в наглую,
выбираеш одно, умудряются всучить, ил вбросить то что викинуть им жалко, или
бомжу отдать, какие там добродушные - могут если не по них - и матом обложить.
Наглые.и злые . Так что давно на рынок не хожу, рядом три маркета рядом, все
тоже - что и на рынке, даже дешевле можно купить,

\begin{itemize} % {
\iusr{Галина Полякова}
\textbf{Наталія Громова} Каждому - свое.

\iusr{Наталія Громова}
\textbf{Галина Полякова} я лиш высказала свое личное мнение.
\end{itemize} % }

\iusr{Мила Галактионова}
Базары и собаки - вещи несовместимые. В цивилизованных странах...

\begin{itemize} % {
\iusr{Irina Malshenkowa}
\textbf{Мила Галактионова} в цивилизованных странах собаки ездят в метро, ходят на базары, в торговые центры, кафе и рестораны
\end{itemize} % }

\iusr{Arcadia Olimi}

 @igg{fbicon.hands.applause.yellow} Спасибо!  @igg{fbicon.heart.beating} 
И слава Богу хоть там о бездомных собачах заботятся @igg{fbicon.exclamation.mark.double}
И базарная торговка и забота о собаче - понятия несовместимые  @igg{fbicon.heart.sparkling} 

\begin{itemize} % {
\iusr{Галина Полякова}
\textbf{Arcadia Olimi} Очень даже совместимые. Практика это доказывает.

\iusr{Arcadia Olimi}
\textbf{Галина Полякова} 

Вы не поняли, я имела ввиду название - баз. торговка и забота о собачах
@igg{fbicon.exclamation.mark.double}

\iusr{Галина Полякова}
\textbf{Arcadia Olimi} 

Простите, видимо, я что-то недопоняла. Вообще-то название выделено крупным
шрифтом. Это цитата из Киплинга. \enquote{Мы с тобой одной крови}.

\iusr{Arcadia Olimi}
\textbf{Галина Полякова}

Цитату помню прекрасно)))
Я же начала с того, что слава Богу, на рынке собач подкармливают  @igg{fbicon.heart.beating} 
\end{itemize} % }

\iusr{Петр Кузьменко}

Именно походы за продуктами и не только за покупками на \enquote{свой} базар отличают
киевлянина от просто жителя столицы. Я уже имел честь несколько раз писать в
нашей замечательной группе о родном для меня, как и для многих подолян, Житнем
базаре. Ваш рассказ живой, тёплый, добрый, предновгодний и очень киевский.
Спасибо! @igg{fbicon.hands.shake} 

\iusr{Maryna Chemerys}

Ми завжди ходили, а потім їздили у Критий ринок на Бесарабку. Ще донедавна
купувала там свіжий сир - найсмачніший у Києві - і м'ясо. Але вже років два
туди не ходжу

\iusr{Старр Диггер}

Мало кто верит, но на Бессарабке самое недорооое и качественное мясо в Киеве.
Все остальное покупать не стоит из-за дороговизны.

\begin{itemize} % {
\iusr{Тамара Кузнецова}
\textbf{Старр Диггер} не-не-не! Я, как раз, это хорошо знала. Особенно, ближе к закрытию рынка.

\begin{itemize} % {
\iusr{Галина Полякова}
\textbf{Тамара Кузнецова} Когда-то я жила на Круглоуниверситетской. Вы правы! Можно кое-что купить!

\iusr{Тамара Кузнецова}
\textbf{Галина Полякова} Ой! Были соседями. Я - на Бассейной.

\iusr{Тамара Яцык}
\textbf{Тамара Кузнецова} Это потому, что там нет холодильника.

\iusr{Тамара Кузнецова}
\textbf{Тамара Яцык} так вот оно что!!! @igg{fbicon.grin} 

\iusr{Старр Диггер}
\textbf{Тамара Яцык} Он есть. Но денег стоит.

\iusr{Старр Диггер}
Да и лишний ночлег понижает профит.

\end{itemize} % }

\iusr{Тарас Кирейко}

На Бессарабке в вокресенье после часу дня мясо дешевело, ибо понедельник был
санитарный день. А холодильники с 1913 года. А сейчас там всюду контрабандная
любая ИКОРКА.

\iusr{лариса погосова}
\textbf{Старр Диггер}
100\%

\end{itemize} % }

\iusr{Тамара Кузнецова}

Хороший рассказ. Снова почувствовала себя киевлянкой, хотя из Киева уехала 11
лет назад. У меня ближайший базар был, как раз, Бессарабский рынок, но ездила я
на Лесной. Конечно же, из за цен.

\iusr{Ольга Македонская}
На житнем рынке недавно кошек потравили

\begin{itemize} % {
\iusr{Тамара Яцык}
\textbf{Ольга Македонская} А неужели ещё есть Сенной рынок?

\begin{itemize} % {
\iusr{Ольга Македонская}
\textbf{Тамара Яцык} спасибо, я с житним перепутала,

\iusr{Ольга Македонская}
\textbf{Тамара Яцык} удалось спасти кормящую кошку с четырьмя котятами, двое красавцы, похожи на сиамских, можете бронировать кому интересно.

\iusr{Ольга Македонская}

\ifcmt
  tab_begin cols=2,no_fig,center,no_height,width=0.3,resibox=0

     pic https://scontent-frx5-1.xx.fbcdn.net/v/t39.30808-6/270113094_2617140425096329_1966100832083799221_n.jpg?_nc_cat=105&ccb=1-5&_nc_sid=dbeb18&_nc_ohc=yq_UWtGg65cAX9wdQZW&_nc_ht=scontent-frx5-1.xx&oh=00_AT-Uft5AQP3hILB7NuLJQGrr3tmk9xUxNEOWQF6hv2OIbA&oe=61CFFA38

		 pic https://scontent-frt3-1.xx.fbcdn.net/v/t39.30808-6/270035033_2617140501762988_7689350106852548479_n.jpg?_nc_cat=107&ccb=1-5&_nc_sid=dbeb18&_nc_ohc=_gZsAVJ3ATUAX_pC_Vx&_nc_ht=scontent-frt3-1.xx&oh=00_AT9j6MD58OV4uBOXJQ2DjewmGFMtUvdkdqtSidU7Xco7Tg&oe=61CF0602

  tab_end
\fi

\end{itemize} % }

\end{itemize} % }

\iusr{Елена Супрун}

\ifcmt
  ig https://i2.paste.pics/be6d4038592f08462f334d46896b3af9.png
  @width 0.2
\fi

\iusr{лариса погосова}

Жила всю жизнь на Шота Руставели. Мой базар -это Бессарабка. Знала продавцов по
Много лет.

Теперь только Творог и мясо!

Знакомы по 20 лет...

Езжу к ним специально!

\begin{itemize} % {
\iusr{Наташа Гресько}
\textbf{лариса погосова} 

наша семья с Бессарабки, и папа с детства умел торговаться на рынке
@igg{fbicon.face.savoring.food} (я так и не научилась).

\iusr{лариса погосова}
\textbf{Наташа Гресько}
В какой школе учились?

\iusr{Наташа Гресько}
\textbf{лариса погосова} папа или я?

\iusr{Наташа Гресько}
\textbf{лариса погосова} 

отец с братом много школ поменяли, но было это очень давно: из тех, что помню,
86-я. а я в соседней 117-й училась (не только, правда, в ней).

\end{itemize} % }

\iusr{Таня Сидорова}

Гарно написано, але я не люблю базари. Давно там вже стоять перекупщики. Мені
спокійніше купувати в супермаркеті, може тому, що не вмію торгуватись, а це зараз
на має сенсу

\iusr{Viktoria Terpylo}
Респект \enquote{базарной республике} 

отнеслись по -человечески к псу!!!

Я жила на Лесном а соответственно и рынок Юность был поблизости, хотя и Лiсова
рядом и всякие базары на перекрестках были. Но Юность это что то.... От соленого
огурца и до шурупа с гвоздями. Я своего мужа (итальянца)как повела на
экскурсию, так он прозрел....

На следующий день отпросился сам пойти на экскурсию. Как и любому мужику ему
были интересны строительно-электро-товары. Был приятно удивлен что базар
работает до вечера


\iusr{Любовь Сыроватка}
Прочитала начало , дальше не хочу читать .

\iusr{Майя Степовая}
Щосуботи їздимо на сільський базар на Виноградар. Море задоволення від спілкування... Дякую за розповідь!

\iusr{Ольга Сергиенко}
Это тоже мой базар! хорошу помню этого пса!
Куда же он пропал?

\iusr{Наташа Гресько}

очень хорошо знала и этого пса - Рыжика, и его уже покойную, увы, хозяйку. он
исчез в день аномального снегопада (в марте 13-го, кажется?). тоже частенько
бываю на этом рынке, хотя знакомых продавцов почти не осталось.

\iusr{Алла Волонтирець}

По Вашему рассказу можно снять фильм...

\iusr{Мария Иванова}
Котик с Березняковского рынка)

\ifcmt
  ig https://scontent-frx5-1.xx.fbcdn.net/v/t39.30808-6/270171087_4261820043924007_22445825280476613_n.jpg?_nc_cat=110&ccb=1-5&_nc_sid=dbeb18&_nc_ohc=2xjo_ZGdS6UAX8vneQu&_nc_ht=scontent-frx5-1.xx&oh=00_AT_CnoLyK4fjRFl4CgNFIkVFj0Ty3eR9cI7QiOlXJe7K3w&oe=61CF03D5
  @width 0.2
\fi

\end{itemize} % }
