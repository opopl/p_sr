% vim: keymap=russian-jcukenwin
%%beginhead 
 
%%file slova.festival
%%parent slova
 
%%url 
 
%%author 
%%author_id 
%%author_url 
 
%%tags 
%%title 
 
%%endhead 
\chapter{Фестиваль}

%%%cit
%%%cit_head
%%%cit_pic
\ifcmt
  pic https://zaxid.net/resources/photos/news/202107/1523278_2119426.jpg?202107291639&fit=cover&w=720&h=480&q=65
  width 0.4
	caption «Убогий жайворонок» у Крячківці
\fi
%%%cit_text
Минулого року ми були на \emph{фестивалі} «Хліб» під Харковом. Неймовірні люди його
роблять: завзято, натхненно, створюють форпост української мови. Люди, що
роблять \emph{фестиваль} «Хліб», викупили квартиру Шевельова з Будинку «Слово» і
роблять там резиденцію. І такі маленькі приватні ініціативи пробують тримати
нашу культуру на достойному рівні. Хочеться долучатись до цих спільнот.
Ще одна ознака нашого проекту ― це спроба, з одного боку, долучитися до плеяди
таких людей, а з іншого ― їм допомогти, зрозуміти, що ми не одинокі краплинки.
У цьому теж для мене місія і радість. І, напевне, теж така краплинка ―
наступний рік ювілейний, трьохсотріччя Сковороди
%%%cit_comment
%%%cit_title
\citTitle{«Нам би хотілось закрутити всю Україну на цій осі»}, 
Катерина Сліпченко, zaxid.net, 29.07.2021
%%%endcit

