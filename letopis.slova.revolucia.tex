% vim: keymap=russian-jcukenwin
%%beginhead 
 
%%file slova.revolucia
%%parent slova
 
%%url 
 
%%author 
%%author_id 
%%author_url 
 
%%tags 
%%title 
 
%%endhead 
\chapter{Революция}

Смена формы правления и общественного устройства путём голосования за какую-то
бумажку невозможна. Общественное устройство меняется только путём
\emph{Революции}. Она может происходить либо \enquote{сверху}, либо
\enquote{снизу}. Для \emph{Революции} \enquote{сверху} у Зеленского нет ни
доверенных кадров, ни организационного ресурса, ни на 100\% лояльного силового
аппарата. Ну, а для \emph{Революции} \enquote{снизу} у украинского общества нет
ни сплочённости, ни способности к самоорганизации, ни чёткого видения
совместного лучшего будущего. Так что, всех всё устраивает, а потому,
\enquote{деолигархизация} отменяется,
\textbf{Никакой деолигархизации в Украине быть не может},
Даниил Богатырев, strana.ua, 05.06.2021

Теперь, после предупреждения, о сути. Последние несколько лет (точнее около
10-15) в мире происходит достаточно много \emph{революционных} событий. Нет,
скажет часть читателей – политические революции происходили и раньше.
Соглашусь. Но отмечу важную деталь: раньше речь шла о смене персоналий в
политических системах либо выборе одной из известных и апробированных систем
для отдельно взятого общества. Сейчас ситуация иная – лихорадит авторитарные и
автократические режимы, лихорадит военные хунты, лихорадит либеральные
демократии.  И \emph{революционные} процессы сегодня явно либо не явно, но
ставят вопрос о трансформации существующей политической системы во что-то
новое. То есть, старые лекала, алгоритмы, работают не всегда. В некоторых
государствах процессы идут мягко, через выборы и приход к власти
\enquote{внесистемных сил}, в некоторых через волнения и войны,
\citTitle{Как \enquote{работают} революции: этапы, результаты, сроки}, Ігар Тышкевіч, censor.net, 08.06.2021

Как \enquote{работают} революции: этапы, результаты, сроки, Ігар Тышкевіч, youtube.com, 06.06.2021

