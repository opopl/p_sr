%%beginhead 
 
%%file 15_05_2022.fb.gelfand_maja.izrail.chess.zhurnalist.1.igor_i_galina_yarmon
%%parent 15_05_2022
 
%%url https://www.facebook.com/maya.gelfand/posts/pfbid02eHGbPG7cWc6CAL1TAGXpN6WPkAJSfJGxXF2pvNHfrWkV3p3DJAX7JY5d4LeKyTfyl
 
%%author_id gelfand_maja.izrail.chess.zhurnalist
%%date 15_05_2022
 
%%tags mariupol,mariupol.war,chess,jarmonov_igor.mariupol,izrail,bezhenec
%%title Игорь и Галина Ярмоновы
 
%%endhead 

\subsection{Игорь и Галина Ярмоновы}
\label{sec:15_05_2022.fb.gelfand_maja.izrail.chess.zhurnalist.1.igor_i_galina_yarmon}

\Purl{https://www.facebook.com/maya.gelfand/posts/pfbid02eHGbPG7cWc6CAL1TAGXpN6WPkAJSfJGxXF2pvNHfrWkV3p3DJAX7JY5d4LeKyTfyl}
\ifcmt
 author_begin
   author_id gelfand_maja.izrail.chess.zhurnalist
 author_end
\fi

Галина прижимает к груди небольшую красную сумочку. 

- Здесь все наши богатства, - говорит она. – Все, что помогло нам выжить.

Из богатств – паспорта, обернутые в целлофан, свидетельство об инвалидности,
книжка почетного жителя Мариуполя со следами сажи, две пачки \enquote{Parliament}.

- Как вы думаете, что это? – спрашивает Галина.

- Выглядит, как сигареты, - отвечаю.

- Нет! Это валюта! – говорит Галина, улыбаясь.

В осажденном Мариуполе деньги потеряли ценность, зато сигареты и водка стали
самым ходовым товаром. На них можно было обменять кусок хлеба, немного муки,
банку чистой воды. Полукилограммовая палка колбасы стоила дороже – за нее
требовали золотые серьги. Но муж запретил Галине их продавать. 

- Игорь – настоящий глава семьи, - рассказывает мне Галина. – Он принимает все
решения. Как шахматист, он умеет просчитывать ходы вперед.

Муж Галины – Игорь Ярмонов, пятикратный чемпион мира по шахматам среди людей с
нарушением опорно-двигательного аппарата, легендарный шахматный композитор. В
Израиль он приехал, чтобы принять участие в чемпионате мира, который проходит в
Ашдоде. В прежние времена мы бы говорили о турнирах, медалях и кубках. Сегодня
мы говорим только о том, как удалось вырваться из ада.

-  21-го февраля мы гуляли по набережной Азовского моря. Знаете, какая красивая
у нас набережная! Там детские площадки, лавочки, клумбы. Все красиво, ухожено.
Мы тогда радовались солнцу, теплу. Было так хорошо! – вспоминает Галина. –
Через три дня началась война. А 2-го марта отключили электричество, воду,
связь. Мы оказались отрезанными от всего мира. Со всех сторон город был окружен
войсками. Это была западня, из которой невозможно было выбраться.

До войны в Мариуполе проживало шестьсот пятьдесят тысяч человек. Те, кто сумел,
уехали в первые сутки, а это около полумиллиона жителей. Игорь – инвалид первой
группы, он с трудом передвигается и почти не говорит. Галина для него – руки,
ноги, язык. В первые дни после начала бомбежек они не успели покинуть город. А
потом это было уже невозможно. 

- Сначала Игорь не выпускал меня на улицу. Там лежали трупы. Мы только смотрели
из окна и ждали, пока их уберут. К тому же, у нас были продукты – обычные
запасы, которые есть в каждом доме. Этого хватило на первые дни. А еще нам
подсказали набрать ванну с водой, и это нас спасло. Эту воду мы пили целый
месяц. 

Когда запасы продуктов начали иссякать, Галина была вынуждена пойти \enquote{на охоту}.
Она вспоминает, как в первую ночь, как только отключили электричество, все
магазины и аптеки были разграблены. Мужчины, из местных, выносили продукты,
технику, лекарства, оставляя после себя груду битого стекла. Потом кто-то
продавал награбленное, кто-то оставлял себе. Чтобы не умереть от голода, Галина
ходила по домам и умоляла помочь. Иногда удавалось раздобыть что-то из
съестного, но бывало, она возвращалась домой с пустыми руками. И тогда
приходилось кипятить в котелке воду из ванны и пить пустой кипяток. Боевые
действия шли беспрерывно. Во время вылазок можно было угодить под обстрел,
получить пулю в лоб от снайпера, наступить на мину, сгореть от попадания
фосфорной бомбы. Позже Галина научилась определять приближение ракеты: если
звук высокий, значит, она далеко; если гул нестерпимо громкий, значит, нужно
бежать и искать укрытие.

- Мы разводили костер во дворе и на нем готовили пищу. Наш сосед по дому,
который знал меня сорок лет, сказал: без дров не приходи! А где я дрова-то
найду? И я собирала щепки, сухие ветки, приносила книги, мебель. Иначе меня бы
отогнали от костра, и я бы не смогла приготовить даже кашу. А Игорю нужно
питаться, ведь он человек очень нездоровый. Да и я тоже. 

Акации, тополя и берёзы были уничтожены в первую очередь. Потом приехали войска
ДНР и спилили многолетние ели: из них они сделали камуфляж для своих танков. Те
жители, которые остались в осажденном городе, старались помогать друг другу. У
них даже появился свой \enquote{сын двора} - двухлетний Максим. Из своих скудных
запасов соседи выделяли ему еду: кто сухари, кто молоко. Главное, чтобы мальчик
был сыт. Однажды раздобыли мороженую картошку и сварили из нее суп. Получился
сладковатый неаппетитный кисель. Но и его съели, конечно. Голод, холод и страх
– постоянные спутники жизни  в оккупированном Мариуполе.

- У нас выбило все стекла, мы заклеили окна скотчем, опустили жалюзи и заложили
проемы декоративными подушками, которые взяли с дивана. Было очень холодно. Мы
надевали по три свитера, собрали все одеяла, которые были в доме, а сверху
укрывались овечьей шкурой. Но все равно не могли согреться. Мы только сидели и
молились, чтобы мы выжили. Игоря спасали шахматы, а меня – забота о нем. 

Наконец, в начале апреля Галине удалось связаться с сестрой Светланой, живущей
в Москве. Она прошла все инстанции, но смогла эвакуировать Ярмоновых из
Мариуполя. Помогли и связи в шахматном мире – Игорь в городе человек известный
и уважаемый. Спустя почти полтора месяца после оккупации Игорь и Галина смогли,
наконец, почувствовать себя в безопасности, поесть и помыться. 

- Нам некуда идти, - говорит Галина. – Наш дом разбомблен. Город по-прежнему
оккупирован. Страна воюет. В Россию, к сестре, мы, конечно, не поедем, об этом
не может быть и речи. Жить в стране-агрессоре для нас невозможно.

Через два дня закончится чемпионат, но Ярмоновым некуда возвращаться. У них
больше ничего не осталось. Все богатства умещаются в маленькой красной
сумочке, испачканной сажей и пропахшей костром. 

Игорь и Галина подали запрос в израильское МВД с просьбой признать их беженцами
и предоставить статус временного проживания в Израиле, но пока не получили
ответ. Я очень прошу тех, кто может помочь этой семье, откликнуться. Им
действительно очень нужна наша поддержка.

%\ii{15_05_2022.fb.gelfand_maja.izrail.chess.zhurnalist.1.igor_i_galina_yarmon.cmt}
