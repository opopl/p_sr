% vim: keymap=russian-jcukenwin
%%beginhead 
 
%%file slova.donbass
%%parent slova
 
%%url 
 
%%author 
%%author_id 
%%author_url 
 
%%tags 
%%title 
 
%%endhead 

Он также рассказал, что в съёмках хроники медицинских будней им помогал
реальный врач из Сибири, а точнее с Крайнего Севера. Медик болеет за
\emph{Донбасс}, помогает и бывает здесь. Сам о себе он говорит – просто доктор,
просто «Потапыч», \textbf{Премьера песни от группы «Зверобой» – Шанс на
спасение}, miaistok.su, 18.06.2020

Мир и воссоединение с \emph{Донбассом} еще впереди, Андрей Ермолаев, strana.ua,
30.05.2021

\enquote{Почта ЛНР} определила победителей конкурса \enquote{Люблю тебя, мой
край родной, Донбасс!}, lug-info.com, 31.05.2021

Детей и стариков \emph{Донбасса} жаль больше всего.. Гончаренко все равно ответит за
то, что делал и делает.. не верю,что ему не страшно.. Сны наверное снятся..
комментарий, \textbf{Опять Шария достали. Что дальше?} Анатолий Шарий, youtube.com, 31.05.2021

02 июня 2014г. - день, когда был окончательно потерян \emph{Донбасс},
\textbf{Ровно семь лет назад Украина навсегда потеряла Донбасс, - Помните женщин с оторванными ногами у здания Луганской ОГА?} 
Андрей Головачев, strana.ua, 02.06.2021

Можно сколько угодно играть в игру под названием Минские Соглашения, формула
Штайнмайера, план Козака-Ермака и т.д., но приходится признать, что Донбасс
ушел и нет силы, которая могла бы его вернуть,
\textbf{Ровно семь лет назад Украина навсегда потеряла Донбасс, - Помните женщин с оторванными ногами у здания Луганской ОГА?} 
Андрей Головачев, strana.ua, 02.06.2021

\enquote{Точка невозврата для \emph{Донбасса}}. Семь главных вопросов об авиаударе ВСУ по Луганску 7 лет назад,
Максим Минин, strana.ua, 02.06.2021

Так получилось, что почти вся моя жизнь и воспоминания связаны с войной, из-за
чего у меня нет сожалений и печали о прошлом. Я живу настоящим и изредка думаю
о будущем, в котором есть место наивной и глупой мечте, вызывающей улыбку.
Вполне реальная, согревающая и почти осязаемая она позволяет не отчаиваться
даже в самые тяжёлые времена. Я хочу, чтобы в небе \emph{Донбасса} летали пассажирские
самолёты, а не бумажные фонарики. Любая мечта должна превращаться в реальность.
Так должно быть и я верю, что так и будет,
\textbf{Фаина Савенкова. Верить и надеяться}, antifashist.com, 02.06.2021


Гибель людей на \emph{Донбассе} говорит о том, что наше общество больное,
Денис Жарких, strana.ua, 04.06.2021

Гибель людей на \emph{Донбассе} говорит о том, что наше общество больное. И, что
человек, что общество, приступают к лечению, когда чувствуют себя больными.
Именно чувствуют, но для этого надо эти самые чувства иметь. К сожалению, мы
продолжаем развивать бесчувственное общество потребителей, паразитов, не
имеющих чувств, а только желудок.  Вакциной от бесчувствия является творчество,
искусство, красота. И в этом плане творчество маленькой девочки Фаины
Савенковой лучик света в этом огромном хаосе преступных желаний политиков и
безразличия масс. В творчестве Фаины нет пафоса, нет конъюнктуры, нет
романтизации войны - это творчество чистого ребенка повзрослевшего во время
войны более большинства взрослых,
\textbf{Гибель людей на \emph{Донбассе} говорит о том, что наше общество больное},
Денис Жарких, strana.ua, 04.06.2021

Посмотрите внимательно: видите контур «незалежной»? Так там и Крым, и \emph{Донбасс}
на старом месте – как было до госпереворота, пардон, до «революции гидности». А
еще, наябедничаю, там и «кричалки» бандеровские, которые возродил майдан 2014
года и закрепил на государственном уровне, шелком вышиты. Да-да: «Слава
Украине! Героям слава!» - как и положено настоящему арийско-патриотическому
государству (правда, почему-то второй лозунг, очень похожий на лозунг
нацистской Германии, спрятан в тылу, на спине, да еще и с изнанки вышит)),
\citTitle{Форма не главное – главное содержание! И в спорте тоже...}, Мысли Бабы Яги, zen.yandex.ru, 07.06.2021

Слов о \enquote{раковой опухоли} мы там не нашли (хотя именно на дневники Гончара как
на источник этой цитаты ссылался в недавнем интервью Кравчук). Но отыскали два
других упоминания о Донбассе.  В опубликованной версии дневников Гончара есть
запись от 1993 года. Там писатель действительно нелестно отзывается о регионе и
пишет, что \enquote{Донбасс - это люмпенизированный край}.  Написано это было не просто
так, а во время масштабной забастовки в Донецкой и Луганской областях лета 1993
года. Это была крупнейшая (до первого Майдана) акция протеста в истории Украины
и сильно потрясла страну, во многом поменяв политический расклад, приведя во
власть в Киеве донецких промышленников во главе с Ефимом Звягильским.  На этом
фоне в июне 1993 года Гончар полагал, что Донбасс \enquote{спровоцирован и обдурен
шовинистами}. И своей забастовкой не дает угля днепропетровским металлургам,
которые рискуют остаться без доменных печей.  По мнению писателя, шахтеры
шантажируют Украину и у них \enquote{нет чувства рабочей чести},
\citTitle{Донбасс как раковая опухоль. Есть ли это в дневниках Олеся Гончара},
Максим Минин; Екатерина Терехова, strana.ua, 08.06.2021

