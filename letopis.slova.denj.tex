% vim: keymap=russian-jcukenwin
%%beginhead 
 
%%file slova.denj
%%parent slova
 
%%url 
 
%%author 
%%author_id 
%%author_url 
 
%%tags 
%%title 
 
%%endhead 
\chapter{День}
\label{sec:slova.denj}

%%%cit
%%%cit_head
%%%cit_pic
%%%cit_text
Новый миф утверждает, что 22 июня вспоминать нечего, поскольку \enquote{для украинцев
война началась в 1939 году}. Только я не знаю другого такого \emph{дня}, который бы
обозначил перелом в судьбе конкретно украинцев. Зато мы все знаем \emph{день}, который
разделил жизнь 170 миллионов людей на \enquote{до} и \enquote{после}. И беду сделал общей, и
войну, и Победу. Понятно, что всегда есть кто-то особенный. Так и здесь есть
те, кто жил в Польше. Или те, кто ликовал и радовался, бежал помогать строить
\enquote{новый порядок}, рассчитывая на персональную миску и будку. Или за свою личную
обиду спешил мстить, но уляпался по шею. У них действительно свой счет времени
и в общий смысл они никак не укладываются.  Да только какое нам до них дело?  У
сотен миллионов живущих внутренние часы безо всяких подсказок отсчитывают те
бесконечные минуты самого длинного \emph{дня}. Это живет в нашей крови. И его
нельзя отменить постановлением очередного бестолкового правительства
%%%cit_comment
%%%cit_title
 \citTitle{У сотен миллионов живущих дата 22 июня 1941 года - в крови}, 
 Дмитрий Заборин, strana.ua, 22.06.2021
%%%endcit

%%%cit
%%%cit_head
%%%cit_pic
%%%cit_text
12 июля, еще один летний \emph{день}, изматывающая жара, от которой некуда деваться.
Однако, во второй половине \emph{дня}, все, кто хоть как-то интересуется политикой,
начисто забывают о жаре. Информационное пространство взорвано статьей Владимира
Путина «Об историческом единстве русских и украинцев». Еще бы! В кои-то веки
Владимир Владимирович «взялся за перо». Такого ранее не замечалось, ни перед
вторжением в Грузию, ни перед донбасской авантюрой. Накануне аннексии Крыма,
как помнится, Владимир Владимирович тоже ничего не писал, и исторических прав
современной Российской Федерации на Крымский полуостров не обосновывал. А тут
вдруг вот такой ход… И действительно, есть от чего призадуматься
%%%cit_comment
%%%cit_title
\citTitle{«Историческое единство русских и украинцев» и внешнеполитические метания Украины}, 
Игорь Балута, analytics.hvylya.net, 28.07.2021
%%%endcit

%%%cit
%%%cit_head
%%%cit_pic
%%%cit_text
Сегодня был первый \emph{день} "транспортного коллапса". 21 октября вступили в силу
новые правила межобластных перевозок. Выехать за пределы области можно лишь с
ковид-сертификатом или тестом. Правда, железнодорожники с утра сделали
послабления - на электричках можно ездить по старому, даже в другие области.
Поэтому люди, у которых нет ковид-паспортов и денег на тесты (экспресс-тест на
вокзале можно сдать за 350 гривен) ехали на перекладных - с пересадками.  А вот
что касается автобусов, то тут ситуация намного хуже. За \emph{сегодняшний день}
отменили 50-60\% рейсов, а завтра может быть еще хуже. Люди массово
переметнулись к нелегадьным перевозчикам и на бла-бла-кар. В итоге ценники там
взлетели втрое
%%%cit_comment
%%%cit_title
\citTitle{Как ездил транспорт в Украине по новым правилам 21 октября}, Людмила Ксенз, strana.news, 21.10.202
%%%endcit

%%%cit
%%%cit_head
%%%cit_pic
\ifcmt
  pic https://img.strana.news/img/article/3589/facebook-papers-novyj-61_main.jpeg
  @width 0.4
\fi
%%%cit_text
Модераторы рассматривали такой контент как отдельные фрагменты вместо того,
чтобы признать это целостной кампанией.  NBC News рассказало, кто именно давал
указания по контенту в день штурма Капитолия.  В этот \emph{день} указания о
приоритетности контента давал лично главный технический директор Facebook Майк
Шрепфер. Указывается, что во внутренних чатах его посылы были встречены резкими
ответами сотрудников.  "Я изо всех сил пытаюсь сопоставить мои ценности с моей
работой здесь. Я пришел сюда, надеясь произвести изменения и улучшить общество,
но все, что я видел, - это атрофия и отказ от ответственности", - комментирует
свою работу в компании в тот период один из неназванных работников
%%%cit_comment
%%%cit_title
\citTitle{Разжигает ненависть от Азии до Африки. Почему Facebook одновременно атаковали крупнейшие мировые СМИ}, 
Максим Минин; Екатерина Терехова, strana.news, 26.10.2021
%%%endcit

%%%cit
%%%cit_head
%%%cit_pic
\ifcmt
  pic https://strana.news/img/forall/u/0/0/232823_3.jpg
  @width 0.4
\fi
%%%cit_text
Вчерашний \emph{день} был днем 77-й годовщины освобождения Украины от нацистской
оккупации. Но большинство политических сил праздновать годовщину не стали.
Традиционно широко отметила ее только "Оппозиционная платформа – За жизнь", да
еще отметились отдельные "слуги народа".  Власть тоже высказалась. Но как!
Владимир Зеленский, например, в своем тексте поздравления забыл упомянуть о
том, кто именно освобождал Украину – что весьма показательно демонстрирует
эволюцию человека, который два с половиной года назад с гордостью говорил о
деде-красноармейце
%%%cit_comment
%%%cit_title
\citTitle{Киевский полулокдаун, "анонимные" освободители Украины, украино-немецкий скандал. Итоги "Страны"}, 
, strana.news, 29.10.2021
%%%endcit

%%%cit
%%%cit_head
%%%cit_pic
%%%cit_text
...Повернувшись додому, Гриць не міг отямитися \emph{днів} три. Лежав, на
ліжку, згадував розмову з легендарним супутником. Хотів піти до брата Бориса,
який працював у Інституті кібернетики, але стримався: неодмінно пошлють його до
психіатра. Треба самому розібратися у всьому.  Попливли звичайні робочі
\emph{дні}. Гриць у вільний час внадився до бібліотеки, почав вивчати проблему
оптимізму, радості, сміху, гумору. Минали тижні, місяці, але з чужих джерел
хлопець не міг напитися того трунку, що його відчув на гірській стежині. Мудрі
фрази, дотепні афоризми, авторитетні рецепти — діаметрально протилежні. Він
глибоко відчув, що осягнення радості — не інтелектуальна проблема, і ніхто не
може створити формулу сміху або дати алгоритм оптимізму.  Гриць звернув увагу
на джерела радості, які у всі віки називалися просто — народною усмішкою,
гумором. Цей пошук привів його до гумористичного часопису «Хрін». Згодом він
став позаштатним кореспондентом, писав веселі, дотепні гуморески, що викликали
схвалення читачів і редакції. Редактор обіцяв підшукати місце в штаті, агітував
завершити університетську освіту, отримати диплом
%%%cit_comment
%%%cit_title
\citTitle{Вогнесміх}, Олесь Бердник
%%%endcit
