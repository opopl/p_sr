% vim: keymap=russian-jcukenwin
%%beginhead 
 
%%file 18_11_2021.fb.medvedeva_olesja.1.donbass_detstvo
%%parent 18_11_2021
 
%%url https://www.facebook.com/olesia.medvedieva/posts/1741758206015294
 
%%author_id medvedeva_olesja
%%date 
 
%%tags borsch,chelovek,detstvo,foto,pamjat,ukraina,zhizn
%%title Темнота, холод и шахтерский посёлок
 
%%endhead 
 
\subsection{Темнота, холод и шахтерский посёлок}
\label{sec:18_11_2021.fb.medvedeva_olesja.1.donbass_detstvo}
 
\Purl{https://www.facebook.com/olesia.medvedieva/posts/1741758206015294}
\ifcmt
 author_begin
   author_id medvedeva_olesja
 author_end
\fi

Темнота, холод и шахтерский посёлок.

В одном из блогов про нехватку угля в Украине, я поделилась своими
воспоминаниями о зимах своего детства и отрочества. Это вызвало довольно бурную
реакцию в комментариях. Штошш, давайте углубимся и поностальгируем. 

\ifcmt
  ig https://scontent-frt3-2.xx.fbcdn.net/v/t39.30808-6/257752905_1741756296015485_6165955771728614991_n.jpg?_nc_cat=101&ccb=1-5&_nc_sid=8bfeb9&_nc_ohc=KG0n72TdbhQAX9rTPA2&_nc_ht=scontent-frt3-2.xx&oh=b2fddc07450642728c9c83d4049da784&oe=619F1AF7
  @width 0.4
  %@wrap \parpic[r]
  @wrap \InsertBoxR{0}
\fi

Я родилась в шахтерском посёлке микрорайона Комсомолец, расположенном между
одноимённой шахтой и шахтой 5. Город Горловка. Донецкая область. Семья
шахтеров, инженеров и учителей. Дом, в котором мы жили, построил прадед в 50х
годах прошлого века. Отопление - печь, которая топится, ожидаемо, дровами и
углём. Ванной комнаты в доме не было. Туалет на улице. Мыться можно в большом
алюминиевом тазике, предварительно нагрев воду в 8ми литровых чугунных
кастрюлях на печи, которую надо растопить. Челендж ещё тот. И до того как мне
исполнилось 10, хозяйственными занималась моя прабабушка, и я в эти все нюансы
не вникала. Но когда ее не стало, хозяйством стали заниматься все. 

Зимы на Донбассе самые суровые в Украине, а лето самое горячее и сухое. С
годами печь стала греть хуже, поэтому зимой, дополнительно, мы включали
электрические обогреватели. И когда были эти веерные отключения, от 2 до 12
часов, я выла волком от холода. 

Утро для меня казалось адом. Мне нужно было ехать в школу в другой конец
города. Встать надо было в 6. Света нет. Нет и тепла. Печь за ночь остыла.
Чтобы умыться не ледяной водой, надо было расколупать и растопить печь, ибо
электрическая не работает (привет веерные отключения), предварительно почистив
ее от жужалки. Потом в мороз за дровами в сарай и нарубить топором (ибо мороз)
уголь в другом сарае. И только тогда грелась вода. 

Но была и романтика. Мамина игра за пианино при свете свечей. Уроки при свечах,
гадание на воске @igg{fbicon.laugh.rolling.floor}, кроссворды при свечах с бабушкой и многое другое. 

Мозг человека устроен так, что плохое забывается и остаётся романтика и
ностальгия. Но на самом деле была просто жесть. Повторить эти все подвиги
выживания не хотелось бы. 

На фото мне 1,5 года. Я проснулась, стою на пороге своего дома, обескуражена
тем, что прабабушка Марта кладет в борщ капусту и лук, и очень обиделась на
неё. В миске та самая капуста. До сих пор не люблю борщ @igg{fbicon.monkey.see.no.evil} 

\ii{18_11_2021.fb.medvedeva_olesja.1.donbass_detstvo.cmt}
