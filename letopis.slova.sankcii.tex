% vim: keymap=russian-jcukenwin
%%beginhead 
 
%%file slova.sankcii
%%parent slova
 
%%url 
 
%%author_id 
%%date 
 
%%tags 
%%title 
 
%%endhead 
\chapter{Санкции}
\label{sec:slova.sankcii}

%%%cit
%%%cit_head
%%%cit_pic
%%%cit_text
Секретарь совета национальной безопасности и обороны Украины Алексей Данилов
прокомментировал слова экс-главы МВД Арсена Авакова о том, что решения о
\emph{санкциях} СНБО принимаются без подготовки и "за пять минут".  На брифинге в
среду, 10 ноября, Данилов фактически подтвердил это заявление Авакова.  "По
документам с грифом секретно - мы не можем выдавать их заранее, они выдаются
сразу перед заседанием. И потом после заседание из изымают.  Голосование есть
персональным. Если человек не согласен с тем или иным решением - он может
проголосовать как угодно. Голосование есть тайным, но люди сами могут сказать,
как они голосовали. Например, как Разумков сказал, что не голосовал за \emph{санкции}
против Медведчука", - объяснил Данилов.  То есть секретарь СНБО подтвердил, что
"секретные" папки раздаются сразу перед заседанием и у членов СНБО нет времени
их изучить
%%%cit_comment
%%%cit_title
\citTitle{Данилов подтвердил слова Авакова, что у членов СНБО нет времени изучить вопрос по санкциями перед голосованием}, 
Эллина Либцис, strana.news, 10.11.2021
%%%endcit

%%%cit
%%%cit_head
%%%cit_pic
%%%cit_text
\emph{Санкционная} машина СНБО затрагивает все больше украинских граждан и бизнесов. И
чем дальше, тем больше участников \emph{санкционных} списков начинают оспаривать
санкции в судебном порядке. В том числе, и интернет-газета "Страна".  На днях
Верховный суд принял к рассмотрению иск главреда "Страны" Игоря Гужвы к
президенту страны Владимиру Зеленскому о признании незаконным указа о введении
\emph{санкций} СНБО.  Подобные иски в последние месяцы поступают в Верховный суд
"пачками". Однако их рассмотрение блокируют госорганы. Дело в том, что
документы с обоснованиями для введения \emph{санкций} приравниваются к "гостайне":
истцам и судьям не дают к ним доступ. И в то же время, госорганы не
сертифицируют для суда специальное оборудование, необходимое для рассмотрения
дел под грифом "секретно".  Среди судей и экспертов крепнет убеждение, что во
всех этих "секретных" и "тайных" документах вообще нет никаких доказательств
обоснованности введения \emph{санкций}, поэтому власти под разными предлогам и не
хотят, чтоб их увидели в судебном процессе.  О том, что в реальности никаких
доказательств вины участников \emph{санкционного списка} на СНБО перед рассмотрением
вопроса о \emph{санкциях} не предоставляется, заявлял недавно и экс-глава МВД Арсен
Аваков
%%%cit_comment
%%%cit_title
\citTitle{"Судьи говорят про справки СБУ - там же совсем ничего нет". Как власть саботирует суды по санкциям СНБО}, 
Анастасия Товт, strana.news, 14.11.2021
%%%endcit
