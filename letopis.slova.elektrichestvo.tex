% vim: keymap=russian-jcukenwin
%%beginhead 
 
%%file slova.elektrichestvo
%%parent slova
 
%%url 
 
%%author 
%%author_id 
%%author_url 
 
%%tags 
%%title 
 
%%endhead 
\chapter{Электричество}

%%%cit
%%%cit_head
%%%cit_pic
%%%cit_text
\begin{itemize}
\item - Спасибо, Марина.- я беру кружку с обломленной ручкой, пью маленькими глотками.
\item - Хорошо хоть с дровами проблем нет.
\item - Да,-горько улыбаюсь я,- развалины кругом. Вон на соседней улице взрывной волной деревянный дом разметало, так что ночью не замерзнем.
\item - Скоро 18.00,- деловито говорит Марина,- на час дадут \emph{электричество}. Надо успеть телефоны подзарядить. И радиоприёмник. Простираться бы, но ведь машинку осколками посекло, не подлежит восстановлению.
\item - Марина,-говорю я,- подзаряди и мой. Хорошо? Может удастся дозвониться до Екатеринбурга, у меня там сестра.
\item - Бедная,- в голосе Марины столько печали,- говорят что Екб фосфором (фосфорные боезаряды) бомбили.
Я молча допиваю кружку, собираюсь идти на соседнюю улицу за дровами. Ночь обещает быть холодной, надо готовиться
\end{itemize}
%%%cit_comment
%%%cit_title
\citTitle{Такие разные разговоры с подростками...}, Мак Сим, zen.yandex.ru, 11.06.2021
%%%endcit

