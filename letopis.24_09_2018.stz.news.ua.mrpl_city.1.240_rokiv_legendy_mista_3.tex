% vim: keymap=russian-jcukenwin
%%beginhead 
 
%%file 24_09_2018.stz.news.ua.mrpl_city.1.240_rokiv_legendy_mista_3
%%parent 24_09_2018
 
%%url https://mrpl.city/blogs/view/do-240-richchya-mariupolya-legendi-mista-chastina-iii
 
%%author_id demidko_olga.mariupol,news.ua.mrpl_city
%%date 
 
%%tags 
%%title До 240-річчя Маріуполя: Легенди міста. Частина III
 
%%endhead 
 
\subsection{До 240-річчя Маріуполя: Легенди міста. Частина III}
\label{sec:24_09_2018.stz.news.ua.mrpl_city.1.240_rokiv_legendy_mista_3}
 
\Purl{https://mrpl.city/blogs/view/do-240-richchya-mariupolya-legendi-mista-chastina-iii}
\ifcmt
 author_begin
   author_id demidko_olga.mariupol,news.ua.mrpl_city
 author_end
\fi

\ii{24_09_2018.stz.news.ua.mrpl_city.1.240_rokiv_legendy_mista_3.pic.1}

\vspace{0.5cm}
\begin{raggedright}
{\bfseries\em\enquote{Великим ворогом істини часто є не брехня - навмисна, удавана і ганебна,

а міф – стійкий, захоплюючий і ефемерний}.}

\emph{Джон Фіцджеральд Кеннеді}
\end{raggedright}
\vspace{0.5cm}

Остання серія легенд про Маріуполь була підготовлена завдяки розвідкам
етнографа \textbf{Якова Новицького}, письменника \textbf{Андрія Кокотюхи}, краєзнавця \textbf{Аркадія
Проценка}, українського журналіста та краєзнавця \textbf{Валерія Шептухи}, письменника і
краєзнавця \textbf{Івана Костирі} та, як завжди, розповідям маріупольців. Легенд про
наше місто набагато більше, ніж мені вдалося показати. І вони будуть
продовжувати з'являтися, адже Маріуполь – сучасне місто, що може привабити не
тільки історичними пам'ятками та старовинною архітектурою, але й унікальними
загадками.

Таємниці та легенди Маріуполя надають можливість повного занурення в чарівну
атмосферу минулого, відкриває місто з абсолютно іншого боку – казкового і
незрозумілого.

\begin{center}
\textbf{Таємниці курганів}
\end{center}

\ii{24_09_2018.stz.news.ua.mrpl_city.1.240_rokiv_legendy_mista_3.pic.2}

У центрі Маріуполя на перетині проспекту Будівельників і вулиці Урицького
розташований курган, в народі його називають \enquote{Зелена гірка}, на
старовинних картах – \enquote{Дід}. Висота його – шість метрів. Таких високих
курганів в приазовських степах одиниці. У 1992 році під час будівництва
водопроводу в новому житловому масиві будівельники наткнулися на один з них. І,
мабуть, вирішили, що це звичайний земляний насип. А коли розвалили вже половину
кургану, виявилося, що це стародавнє поховання. Будівництво було припинено,
міська влада запросила археологів, виділили гроші на дослідження зруйнованої
частини кургану.

Степові кургани зберігають свої таємниці. Якщо людина помирала, в могилу,
незалежно від його рангу, ставили горщики з їжею. Щоб, йдучи в інший світ, він
міг взяти з собою їжу на доріжку. Сучасні поминки – продовження цієї традиції.
А день поминання і відвідування кладовищ після Пасхи недаремно називається
Червоною гіркою: \emph{гірка} – це курган, а \emph{червона} – значить,
святкова.

Поховавши померлих, насипали курган, куди потім приходили здійснювати обряди. І
сьогодні місцеві жителі просять не чіпати кургани, які служать для них
своєрідним духовним пам'ятником. Люди шанують кургани й дають їм імена:
\emph{Орлова могила, Нікітіна могила}.

\textbf{Читайте також:} \emph{В Мариуполе появится первый туристический центр}%
\footnote{В Мариуполе появится первый туристический центр, Олена Онєгіна, mrpl.city, 23.09.2018, \url{https://mrpl.city/news/view/v-mariupole-poyavitsya-pervyj-turisticheskij-tsentr}}

Існує легенда, що стародавні поховання можуть наслати прокляття на тих, хто їх
потривожить: якщо зловмисник розриє курган в пошуках наживи або ж, наткнувшись
на курган випадково, зрівняє його з землею.

Біля кургану Дід розташований ресторанчик, побудували його на місці іншого
кургану. Всі, хто заходив сюди, страждав від головного болю, яка за словами
відвідувачів, була сильніше звичайного похмілля.

Втім, кургани, не розраховуючи на педантичність нащадків, тисячі років самі
охороняють свої таємниці. Багато курганів захищені міцними кам'яними крепсидами
– магічними кільцями, спорудженими навколо священного місця. Не один бульдозер
зламався, коли намагалися розорати непокірний пагорб. Союзники курганів – час і
сама природа. Саме вони допомагають заховати від людства великі таємниці.

\begin{center}
\textbf{Козацькі скарби на теренах Маріуполя}
\end{center}

Друга легенда теж пов'язана з курганами, що були розташовані на території
Маріуполя.

\ii{24_09_2018.stz.news.ua.mrpl_city.1.240_rokiv_legendy_mista_3.pic.3}

Етнограф \textbf{Я. Новицький} виявив наступну легенду, пов'язану з маріупольським
курганом:

\begin{quote}
\em\enquote{Вранці при сході сонця вибіг ватажок на маріупольський курган, встав, а тінь
так і простягнулася на захід. Тоді він велів копати яму там, куди потяглася
тінь від голови. Викопали яму, спустили на ланцюгу казан з грошима, щоб ніхто
не брав, поки вдруге не повернуться з роз'їзду. Потім засипали гроші землею,
осідлали коней і піддалися в степ.

Через роки стали з'їжджатися. Багато хто повернувся, а ватажка немає, і скарб
не взято до цього часу. В скарбниці, кажуть, лежать червінці – верху мідні, а
поверх мідних - срібні талери, щоб не зіпсувалися. Гроші закопали, щоб не
залишилося сліду, їх затоптали коні. А прикмета така: як сонце сходить – встань
на курган і дивись на свою тінь; визнач те місце, де закінчується тінь від
голови, біжи і копай. Зверху буде залізне кільце, тягни й витягнеш шворінь;
копай глибше, потім дужка та заповітний казан}.
\end{quote}

Збереглася ще одна легенда. Одного разу до маленького хлопчика під'їхав на
швидкому коні старий козак з двома пістолетами за поясом. Він показав йому
камінь поблизу кургану і розповів наступне: \enquote{Дивись на мене, мені 104 роки і,
можливо, я вже останній з козаків. Цей камінь поклав я разом з товаришами.
Всього нас було 12 запорожців. Під каменем яма, а в ямі три бочки: в одному
мідь, в іншому срібло, а в третьому золото. Гроші закляті, їх мені без
товаришів брати не можна, а тобі – інша справа, бери, як свої}.

Невідомо, чому хлопчик так і не викопав скарби, які, можливо, досі охороняє
таємничий курган...

\textbf{Читайте також:} \emph{Любава, Идрис и Игорина: как называют своих детей в 2018-м жители Донетчины?}%
\footnote{Любава, Идрис и Игорина: как называют своих детей в 2018-м жители Донетчины?, Яна Іванова, mrpl.city, 21.09.2018, \url{https://mrpl.city/news/view/lyubava-idris-i-igorina-kak-nazyvayut-svoih-detej-v-2018-m-zhiteli-donetchiny}}

\begin{center}
\textbf{Святий Харлампій}
\end{center}

\ii{24_09_2018.stz.news.ua.mrpl_city.1.240_rokiv_legendy_mista_3.pic.4}

Збереглася легенда про \textbf{священномученика Харлампія}. Старі джерела розповідають,
що під час переселення греків з Криму, коли епідемія стала косити паству
митрополита Ігнатія, переселенці звернули свої молитви до святого Харлампія.
Збереглася легенда, що саме завдяки заступництву Харлампія хвороби припинилися.
Відтоді цей святий став особливо шанованим маріупольськими греками.

\begin{center}
\textbf{Легенди про красуню Азу}
\end{center}

Одну з легенд про Азовське море я вже розповідала раніше. Однак ще залишилося
безліч легенд, присвячених нашому самобутньому та унікальному морю. Більшість з
них присвячена доньці рибалки Азі. Найбільш розповсюдженою є історія, пов'язана
з трагічною загибеллю дівчини. Легенда розповідає, що Аза жила на самому березі
нашого моря зі старим батьком. І була такою гарною, що всі хлопці очей з неї не
зводили. Вона ж ні на кого не звертала уваги, бо, кажуть, була занадто гордою.
Ще й хвалилася, що їй ніхто не подобається.

\ii{24_09_2018.stz.news.ua.mrpl_city.1.240_rokiv_legendy_mista_3.pic.5}

Ось всі хлопці, які жили поблизу, змовилися і прийшли разом до Ази. Вони
запропонували їй вибрати між ними собі нареченого. Красуня глянула на них,
подумала-подумала, а потім і каже: \enquote{Будете змагатися. Хто з вас своїх товаришів
поборе, той і буде моїм судженим}.

\ii{24_09_2018.stz.news.ua.mrpl_city.1.240_rokiv_legendy_mista_3.pic.6}

І почали хлопці боротися. Один все ж вийшов переможцем з цих змагань, але Аза
відмовила йому, та ще й почала насміхатися над хлопцями. Обдурила суперників.
Розсердилися вони на гордячку, взяли та й утопили її в морі.

Дотепер, коли підходить вода до берега, з моря чути чи то плач, чи то стогін.
Старі люди кажуть, що це втоплена красуня Аза плаче за своїм незнайденим
судженим. І море начебто названо на її честь – Азовським.

\textbf{Читайте також:} \emph{Історія і легенди Азовського моря}%
\footnote{Історія і легенди Азовського моря, Ольга Демідко, mrpl.city, 23.06.2018, \url{https://mrpl.city/blogs/view/istoriya-i-legendi-azovskogo-morya}} %
\footnote{Internet Archive: \url{https://archive.org/details/23_06_2018.olga_demidko.istoria_i_legendy_azovskogo}}

\begin{center}
\textbf{Історія однієї благородної справи}
\end{center}

У 1901 році в маріупольській друкарні Л. Шпарбера було надруковано твір
повітового голови дворянства \textbf{П. Каменського} \enquote{Істо\hyp{}рія одного дня. Достовірна
оповідь} – друге видання неймовірної історії, що відбулася в Маріуполі 6 квітня
1863 року. Книга присвячена маріупольцю Г. Ільяшенку, який вирішив провчити
колишнього члена грецького суду Логафетова за зловживання довіреною йому
владою. Виявивши неабиякі акторські здібності, він представився уповноваженим
імператора і розпорядився \emph{\enquote{за грабежі та вбивство і взагалі за всі зловживання
позбавити (Логофетова) всіх прав стану зі засланням в алтайські заводи у вічні
працівники, а маєток його продати з публічного торгу і задовольнити всіх
боржників...}}. Місцева влада, налякана \enquote{начальницькою} особою, завзято виконала
всі вказівки, заарештувала Логафетова і навіть поголила йому півголови. Але, на
жаль, благородна справа В. Ільяшенка було розкрита. Чи правдива ця історія, чи
все ж таки це життєва вигадка, ми вже не дізнаємося... Але те, що в Маріуполі у
всі часи намагалися боротися проти несправедливості, може бути цілковитою
правдою.

\begin{center}
\textbf{Піджак з золотими ґудзиками}
\end{center}

\ii{24_09_2018.stz.news.ua.mrpl_city.1.240_rokiv_legendy_mista_3.pic.7}

Цю легенду мені розповіли маріупольські старожили, яку запам'ятали з дитинства.
Герой історії все своє життя накопичував фінанси та боявся, що хтось у нього
відбере його ж гроші. Він не був доброзичливим ні для рідних, ні для сусідів.
Дуже гордий, пихатий, жорстокий і небачено скупий – головні характеристики
Семена Павловича, який наказав після смерті поховати себе в піджаку із золотими
ґудзиками з морським орнаментом. Він залишив після себе мало приємних спогадів,
тому після смерті його родичі зробили все по залишеній інструкції. Але ніхто і
подумати не міг, що Семен Павлович зовсім не помер - він впав у стан глибокого
забуття і його летаргічний сон змусив усіх повірити в те, що його більше немає.
Після похоронної церемонії й поховання Семен Павлович і не думав приходити в
себе, але під ранок, коли вже нікого не було поруч і йому довелося відкрити очі
в дерев'яній труні та відчути всі жахи похованого живцем, відбулося
переродження Семена Павловича. Історія про те, як все ж таки вдалося довести,
що це він, все той же скнара, не так цікава... У те, що він не помер, повірити
вдалося легше. А ось в те, що він став щедрий і доброзичливий, перетворився в
душу компанії й кращого дідуся всієї вулиці, повірити було складніше всім тим,
хто його знав до переродження. Але все ж це сталося. А ось піджак з золотими
ґудзиками так і залишився похованим...

\textbf{Читайте також:} 

\begin{itemize}
\item \emph{До 240-річчя Маріуполя: Легенди міста. Частина II.}%
\footnote{До 240-річчя Маріуполя: Легенди міста. Частина II., Ольга Демідко, mrpl.city, 19.09.2018, \url{https://mrpl.city/blogs/view/do-240-richchya-mariupolya-legendi-mista-chastina-ii}} %
\footnote{Internet Archive: \url{https://archive.org/details/19_09_2018.olga_demidko.mrpl_city.240_rokiv_legendy_mista_2}}

\item \emph{До 240-річчя Маріуполя: Легенди міста. Частина I.}%
\footnote{До 240-річчя Маріуполя: Легенди міста. Частина I., Ольга Демідко, mrpl.city, 12.09.2018, \url{https://mrpl.city/blogs/view/do-240-richchya-mariupolya-legendi-mista-chastina-i}} %
\footnote{Internet Archive: \url{https://archive.org/details/12_09_2018.olga_demidko.mrpl_city.240_rokiv_legendy_mrpl_1}}

\end{itemize}

\begin{center}
\em\bfseries Судячи з маріупольських легенд, наше місто усіяне скарбами. Залишається лише
дочекатися щасливчиків, які їх знайдуть...
\end{center}
