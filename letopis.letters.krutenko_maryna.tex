% vim: keymap=russian-jcukenwin
%%beginhead 
 
%%file letters.krutenko_maryna
%%parent letters
 
%%url 
 
%%author_id 
%%date 
 
%%tags 
%%title 
 
%%endhead 

%08:28:57 14-04-23
Доброе утро, Марина! Прочитал весь Ваш дневник, спасибо большое, что решились
все это опубликовать, сочувствую тому, что Вам пришлось пережить.

Меня зовут Иван, я киевлянин, программист и живу и работаю в Киеве. Много
сейчас читаю про Мариуполь, у меня есть также несколько книжек про Ваш
замечательный Город. К сожалению, до войны не успел побывать в Мариуполе... Ну
вот как то выпал Мариуполь из моего поля зрения, и очень жаль, бывал во многих
городах Украины, а вот в Мариуполе - ни разу... Но... я принимаю близко к
сердцу, все, что произошло с Мариуполем и с мариупольцами. Страшная,
невообразимая трагедия...  которая забрала ваших близких, родных, детей,
родителей, раскидала вас, мариупольцев, по всему миру... Тем не менее, я
уверен, лучшие времена для Мариуполя еще впереди, и отчаиваться точно не
надо... Потому что Город - это не только постройки, улицы, парки, сами по
себе... Город - это также Духовная Сущность, в том числе - это книги, память,
фото, истории, люди...  И я как раз из того Города, Киева то есть, - который
как раз много-много раз разрушали, сжигали, разоряли, здесь творились ужасные
трагедии и бедствия, но Киев каждый раз восставал, залечивал свои раны, и
становился только краше! И я уверен, что Мариуполь тоже обязательно восстанет
из пепла, как птица-Феникс, и снова засияет, да, прекрасная Жемчужина-Город на
берегу Азовского Моря!

... Да, как я сказал выше, я программист. В свободное время от работы... я
также занимаюсь систематизацией разных публикаций о войне, у меня есть
отдельный проект Летописи Войны, который я веду - пока что сам - довольно уже
давно, собственно говоря, я сохраняю от забвения посты на фейсбуке, потом
раскидываю их по авторам, темам. Технология, которую я использую, называется
LaTeX (не знаю, имеете ли вы отношение к физике или математике, но это та
технология, которая повсеместно используется учеными в научном мире для записи
и публикации своих статей и исследований ). Так вот.  Я сохраняю в печатном
виде, то, что я читаю, считаю важным для сохранения...  Про Мариуполь я в
последнее время тоже сохраняю, и довольно много. И про довоенный мирный
Мариуполь, и для ужасы войны. У меня есть телеграм канал
https://t.me/kyiv_fortress_1, там уже довольно много выложено, так же как и на
этом фейсбук - аккаунте, также я постепенно выкладываю собранные  материалы - с
уважением к авторам - я всегда указываю, кто и когда что создал - на
https://archive.org - это интернет сайт, посвященный сохранению всего в
Интернете, ссылка на меня там https://archive.org/details/@kyiv_chronicler.
Зачем все это делается... знаете... Мариуполь физически то убили... но
Мариуполь не умер. Мариуполь остался в Духовном Пространстве, и это очень очень
важно... Но если не записывать... если этим не заниматься... есть риск, что
Мариуполь умрет духовно... и это будет уже навсегда. А духовная смерть - это...
еще страшнее, чем смерть физическая... Поэтому я этим и занимаюсь, хотя я и
киевлянин, вообще то говоря. Хочется, знаете, тоже приехать в наш Украинский
Возрожденный Мариуполь, как гость, в свое время ) Надеюсь, все о чем я пишу,
Вам будет интересно. Если Вы не против, я был бы рад также опубликовать в том
числе Ваши посты дневника (они уже все сохранены у меня) в том виде, в котором
я записываю публикации - в таком виде их нельзя удалить по чьей то прихоти, как
это происходит очень часто на фейсбуке, и их можно было бы посмотреть все
вместе как одну книжку (я вчера выложил сборку по Мариуполю довоенному, в
альбоме на моей странице и также как ссылку для скачивания pdf-файла, который
можно просматривать по всему содержанию, вот здесь все можно найти
https://www.facebook.com/media/set/?set=a.216045831035752&type=3)

Я послал Вам запрос на дружбу в фейсбуке, не знаю, принимаете ли Вы новые
запросы, но буду рад подружиться с Вами здесь на фб.

С уважением, Иван.
