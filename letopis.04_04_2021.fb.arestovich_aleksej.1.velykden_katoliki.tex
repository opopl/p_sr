% vim: keymap=russian-jcukenwin
%%beginhead 
 
%%file 04_04_2021.fb.arestovich_aleksej.1.velykden_katoliki
%%parent 04_04_2021
 
%%url https://www.facebook.com/alexey.arestovich/posts/4202688599795188
 
%%author 
%%author_id 
%%author_url 
 
%%tags 
%%title 
 
%%endhead 

\subsection{Сегодня Пасха у христиан западного обряда единой, соборной, апостольской Церкви}
\url{https://www.facebook.com/alexey.arestovich/posts/4202688599795188}

- Сегодня Пасха у христиан западного обряда единой, соборной, апостольской Церкви. 
В своё время, я начинал, как воинствующий атеист и гордился гордостью атеиста:
- иметь смелость жить и радоваться в бессмысленном мире, пока жалкие, трусливые верующие ищут себе утешение в придуманных сказках.
И, конечно же, чтобы высмеивать их покачественнее, я решил изучить эти придуманные сказки.
Так вот ничего более страшного в своей реалистичности, в предельном знании жизни и человека, чем эти сказки, в подлунном мире не существует. 
Евангелие демонстрирует такую беспощадную правду про человека, такое пронзительное знание его натуры, одновременно ничтожной и великой, что никакой скепсис атеистов и близко не может сравнится со скепсисом «...этих сказок».
Я верю, потому, что я скептик. И критик. 
Такой скептик и критик, что резонанс, соответствующий мере моего неверия, я нашёл только в вере.
Вопреки распространённому мнению, разум не ищет знания. 
Разум ищет сомнения.
А в чем же ему сомневаться?
Да во внутреннем чувстве художественной правоты сюжета. 
Внутреннее чувство правоты сюжета, как свет, прорывается сквозь все уговоры скептицизма, заключающиеся в том, что идеального не существует, а если существует - то не сбывается. 
Пасха - кульминация сюжета.
От реализма здесь: предательство, подлость, мучения, смерть, отречение, крах. 
От идеализма: великая надежда, и великая радость - главная для человека и человечества.
Принято считать, что чувством, противоположным страху, есть смелость.
Но это не так.
Смелость, мужество, отвага, стойкость - это средства борьбы со страхом. 
Противоположным страху чувством есть радость.
Иначе невозможно объяснить, почему жалкие, раздавленные, перепуганные (реальные) люди - апостолы, сломавшиеся после распятия настолько, что боялись подойти к могиле, начинают после Пасхи демонстрировать чудеса стойкости и мужества, выходящие далеко за пределы смелости камикадзе.
Исторически засвидетельствованные чудеса.
Что-то они узнали. Что-то они испытали.
И видно, размер этой радости был таков, что ее хватило им и для последующего восхождения на крест. 
Само чувство вечного ужаса перед небытием, свидетельствует, что нечто вечное существует.
Я желаю Вам прочувствовать, пережить основное содержание пасхального сюжета.
- Вечной смерти нет. 
А вечная жизнь - вот она:


\ifcmt
  pic https://scontent-amt2-1.xx.fbcdn.net/v/t1.6435-9/169256866_4202685476462167_3174348296917311838_n.jpg?_nc_cat=111&ccb=1-3&_nc_sid=8bfeb9&_nc_ohc=2FW-mtekIhsAX_Kvrg-&_nc_ht=scontent-amt2-1.xx&oh=7a5bbbced08bffa40f909d7dbf6cb9b2&oe=608EF332
  width 0.4
\fi

