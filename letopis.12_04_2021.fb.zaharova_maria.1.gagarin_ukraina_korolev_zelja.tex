% vim: keymap=russian-jcukenwin
%%beginhead 
 
%%file 12_04_2021.fb.zaharova_maria.1.gagarin_ukraina_korolev_zelja
%%parent 12_04_2021
 
%%url https://www.facebook.com/maria.zakharova.167/posts/10225854709584095
 
%%author 
%%author_id 
%%author_url 
 
%%tags 
%%title 
 
%%endhead 

\subsection{«Без лишней скромности и Украины» }
\Purl{https://www.facebook.com/maria.zakharova.167/posts/10225854709584095}

Зеленский о юбилее первого полёта человека в космос высказал  мысль: «Без
лишней скромности скажу, что без Украины он был бы невозможен. Украинские
конструкторы играли ключевую роль в реализации советской космической программы.
К фундаментальным расчетам ученые добавили веру в то, что можно дотянуться до
звезд. И у них все получилось».

Сколько лет уже ничего не мешает украинским конструкторам с фундаментальными
расчетами и верой покорять космические просторы. И мы рады, что всё получается,
даже если и в мечтах. 

Россия никогда не делила славу космических побед. Спутники, ракеты и улыбка
Гагарина – это всё было общее и принадлежало всем 15 республикам Союза
Советских Социалистических Республик: России, Азербайджану, Армении,
Белоруссии, Грузии, Казахстану, Киргизии, Латвии, Литве, Молдавии,
Таджикистану, Туркмении, Узбекистану, Украине и Эстонии. Учёным, рабочим,
учителям, врачам, военным и всем, кто работал на покорение невероятной цели. 

Жаль, что ярчайшими достижениями современного  киевского  режима стали неуёмная
злоба, русофобия и реинкарнация националистической идеологии с нацисткой
символикой.
