% vim: keymap=russian-jcukenwin
%%beginhead 
 
%%file 14_02_2022.tg.tkachev_jurij.1.tragedia_sovremennoj_ukrainy
%%parent 14_02_2022
 
%%url https://t.me/dadzibao/5269
 
%%author_id tkachev_jurij
%%date 
 
%%tags ukraina
%%title Трагедия современной Украины
 
%%endhead 
 
\subsection{Трагедия современной Украины}
\label{sec:14_02_2022.tg.tkachev_jurij.1.tragedia_sovremennoj_ukrainy}
 
\Purl{https://t.me/dadzibao/5269}
\ifcmt
 author_begin
   author_id tkachev_jurij
 author_end
\fi

Трагедия современной Украины заключается в том, что она никому не нужна, однако
в то же время каждый участник геополитической игры не хочет, чтобы она
досталась кому-либо другому. Пассивы уже давно превзошли активы, и в кратко- и
среднесрочной перспективы \enquote{интегрировать Украину} означает \enquote{попасть на
очень-очень большие деньги} с совершенно неясной перспективой оных денег
возвращения. 

Именно этим и объясняется политика США, в настоящий момент ситуативно (!)
включивших Украину в свою сферу влияния. По сути Штаты действуют по принципу
ликвидационной комиссии: распродать остатки ликвидного имущества, остатки
просто распилив на металлолом, полученное лавэ вывести. Логика проста: сделать
разницу между активами и пассивами ещё выше, а шансы на то, что кто-то
покусится на Украину в будущем - ещё меньше. Создавать, строить здесь что-то
попросту нерационально: с появлением тут чего-то годного появляется и повод для
соперничества. А США, несмотря на всю свою спесь и реальные возможности,
понимают, по какому тонкому льду они со своими украинскими вассалами тут ходят.

Впрочем, России Украина тоже не нужна. Территорий у России хватает и у самой;
климат похуже, но это такое. Население? Актив, но актив легко спиливаемый и
вывозимый на контролируемые территории. Чему, кстати, американская политика в
Украине только способствует. Причём ажиотажного спроса на выходцев из Украины в
России нет: демографическую яму можно успешно засыпать мигрантами из Средней
Азии, которые выходят дешевле.

Транзитный потенциал? Порты Украины уже в целом заменяют новыми проектами в
Краснодарском крае. Про трубопроводы сами всё знаете. Транспортные коридоры
прокладываются через Белоруссию-Польшу, а то и вовсе через Балтийское море. В
южном направлении - через Чёрное. Благо, после \#Крымнаша безопасность на этом
направлении гарантирована.

Промышленность? Всё важное уже импортозаместили (плохо ли, хорошо ли, вопрос
десятый; важно, что деньги потрачены, теперь их нужно отбивать). То, что
осталось, имеет в России своих конкурентов. За крайне редкими, исчезающе малыми
исключениями, которые явно не стоят геморроя.

Оговоримся: речь идёт о современной России в её нынешней парадигме. Так-то с
Украиной Россия стала бы существенно сильнее, но это в 20-30-летней перспективе
и после огромных, иссушающих остальную Россию вложений. И кстати я бы не спешил
осуждать российское руководство за то, что оно не думает лет на 20-30 вперёд с
учётом того, что в эти сроки вполне вероятна новая мировая война.

Есть, правда, чисто политические штуки: тот же вопрос вступления Украины в
НАТО. Который на самом деле в реальной повестке дня не значится. По описанной
выше причине: Украина США по большому счёту стратегически не нужна. Дискурс
«ракет под Белгородом» тухлый изначально: «ракеты под Псковом» ту же проблему
закрыли бы с тем же успехом, и заметьте, размещать их там никто не спешит. То
есть, это такая игра: Россия чисто во внутриполитических целях разводит
перформанс под названием «мы мешаем принять Украину в НАТО», а США ведут
ответную партию «мы мешаем России мешать принять Украину в НАТО». Вот для таких
мелодрам класса «Б» Украина представляет почти идеальную съёмочную площадку. 

Есть, конечно, реально важные вопросы. Например, поставки газа в Европу.
Недаром первым, что интересует те же США в Украине, является труба. И недаром
попытки Тимошенко заигрывать с Кремлём по этой теме стоили ей пожизненного бана
в Вашингтоне. Но и тут основное правило соблюдается: в стратегической
перспективе США заинтересованы в том, чтобы украинская труба не работала, а
Европа покупала американский сжиженный газ.  Ну и вот из вышеизложенного я
вслед за Семёном Ураловым (\url{https://t.me/su2050/755}) делаю печальный вывод:
война в Украине почти неизбежна. Не сейчас, а вообще. И даже не факт, что с
участием России. Потому что лучшего способа обнулить активы, чем война, пока
ещё не изобрели. Едва ли не единственная наша надежда на то, что та самая
третья мировая, которая обнулит всю предыдущую повестку и нарисует карту мира
заново, случится раньше.

Хотя преспективка тоже, признаем честно, так себе.
