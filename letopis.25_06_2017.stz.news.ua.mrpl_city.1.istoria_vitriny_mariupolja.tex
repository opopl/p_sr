% vim: keymap=russian-jcukenwin
%%beginhead 
 
%%file 25_06_2017.stz.news.ua.mrpl_city.1.istoria_vitriny_mariupolja
%%parent 25_06_2017
 
%%url https://mrpl.city/blogs/view/istoriya-vitriny-mariupolya
 
%%author_id burov_sergij.mariupol,news.ua.mrpl_city
%%date 
 
%%tags 
%%title История: витрины Мариуполя
 
%%endhead 
 
\subsection{История: витрины Мариуполя}
\label{sec:25_06_2017.stz.news.ua.mrpl_city.1.istoria_vitriny_mariupolja}
 
\Purl{https://mrpl.city/blogs/view/istoriya-vitriny-mariupolya}
\ifcmt
 author_begin
   author_id burov_sergij.mariupol,news.ua.mrpl_city
 author_end
\fi

\ii{25_06_2017.stz.news.ua.mrpl_city.1.istoria_vitriny_mariupolja.pic.1}

Прошло чуть больше пяти лет после кровавой и разрушительной войны с фашистской
Германией, прошедшей огненным смерчем и по Мариуполю, и многое в нашем городе
изменилось к лучшему. Все меньше на улицах оставалось погорелок. Частные дома
восстанавливали их владельцы, остовы жилых строений, некогда принадлежавших
го­родскому коммунальному хозяйству, а также оказавшихся бесхозными, передавали
промышленным предприятиям и строительным организациям, а те, в свою очередь,
распре­деляли своим работникам, выдавали ссуды, по возможности снабжали
строительными материалами. Ну а уж превращать прокопченные стены в более или
менее пригодные для обитания жилища приходилось их будущим обитателям.

Многоэтажные дома поднимали из руин крупные заводы - имени Ильича,
\enquote{Азовсталь}, коксохимический - своими силами или с привлечением строительных
трестов. Здесь, конечно, в первую голову следует упомянуть коллектив треста
\enquote{Азовстальстрой} и его руководителя Александра Павловича Поборчего, на наш
взгляд незаслуженно забытого. А ведь именно на его плечи легла, казалась бы,
неподъемная задача - восстановить \enquote{Азовсталь} руками полуголодных, плохо одетых
и обутых строителей, значительную часть которых составляли молодые люди из сел
и поселков, избежавшие по возрасту призыва в армию и угона в Германию в
рабство. Постепенно начали возводить и новые дома на месте тех строений,
восстанавливать которые считали невозможным.

В числе первых возрожденных крупных зданий в центре Мариуполя был дом на углу
проспекта Республики и Советской улицы (ныне Харлампиевской). Он стоит через
дорогу от бывшей гостиницы \enquote{Континенталь}, и теперь тоже бывшего Дворца
культуры комбината \enquote{Азовсталь}. Впрочем, зачем столь пространные объяснения?
Ведь у этого здания есть конкретный адрес: Харлампиевская, 15. Первый этаж
этого дома был отведен под парикмахерскую, хлебный магазин и магазин тканей, а
также гастроном. Вход в это торговое заведение находился со стороны улицы
Советской. Еще сравнительно недавно в этом помещении торговали съестными
продуктами и горячительными напитками, продолжая под новой вывеской традиции
гастронома. Сейчас в нем разместилось отделение одного из многочисленных
банков. В пору открытия гастронома его торговый зал казался огромным. Но не это
преимущество было главным. Более всего поражали витрины. В одной из них висел
макет плоской банки с черной икрой,   размеры этой банки были величиной с
тазик приличных размеров, а под этим деликатесом были разложены вполне
натурально изготовленные из папье-маше или гипса муляжи окороков, головок сыра
и другой аппетитной снеди.

В другой же витрине укрепили также макет огромной коробки папирос \enquote{Казбек}.
Пожилые люди, наверное, помнят рисунок на этой коробке: черный силуэт всадника
на фоне голубых Кавказских гор с вершинами, покрытыми снегом. Чуть впереди
стоял манекен ростом с пятилетнего ре­бенка. Манекен был одет в самый
настоящий, только маленького размера, однобортный мужской костюм, сшитый из
серого в полоску трико. Голова манекена изображала приятной наружности молодого
человека, очень похожего на исполнителя ролей соблазнителей в средней руки
аме­риканских фильмах: набриолиненные иссиня-черные волосы с идеальным косым
пробором, тщательно подстрижен­ные бакенбарды, тонкая полоска усов над верхней
губой. Но не \enquote{красота} этого искусственного человечка привлекала внимание
зевак. Главное было то, что человечек время от времени поворачивался на
небольшой угол и движе­нием паралитика подносил к губам папиросу, зажатую между
ука­зательным и средним пальцами правой руки. В этот самый момент на конце
папиросы загорался красный огонек. Подержав таким образом некоторое время
папиросу, \enquote{курильщик} судорожно опускал руку и поворачивался в прежнее
положение. Все это действо приводило в неописуемый восторг юных граждан
тогдашнего Мариуполя, не оставляя, конечно, равнодушными людей более зрелого
возраста.

Раз уж разговор пошел о витринах, нельзя не вспомнить те, что находились по обе
стороны от двери, над которой красовалась вывеска со странной для людей, не
знавших украинский язык, надписью: \enquote{Держфото}. Читатели-старожилы наверняка
уже догадались, что речь идет здесь о фотоателье на проспекте Ленина, 24,
соседствующем с централь­ной сберкассой. В витринах постоянно были выставлены
три-четыре великолепно выполненных крупноформатных фотопортрета. На них были
изображены то кто-нибудь из очаровательных мариупольских девушек, то фактурный
старик с окладистой бородой, то моряк в фуражке с \enquote{крабом}, а то и улыбающееся
личико ребенка. Внизу под стеклом находилось множество фотографий открыточного
формата с лицами наших сограждан. Экспозиция витрины довольно часто менялась.

Все это были работы высококлассных художников-фото­графов Федора Петровича
Науменко и Льва Исааковича Сосны. Трудно себе представить более разных людей,
чем они. Разная внешность, разные темпераменты, разные биографии и
происхождение. Но было нечто, что их объединяло: безукоризненное владение
своей профессией, необык­новенное трудолюбие и хороший вкус... Коль скоро речь
пошла об этих мариупольских фотохудожниках, приведем цитату из
иллюстрированного каталога \enquote{Мариуполь на видовых открытках} известного
коллекционера Л. П. Морозова: \enquote{В начале 50-х годов киевский трест \enquote{Укрфото}
выпустил нумерованную серию свыше десяти черно-белых открыток. Фотографии
сюжетов сделаны Ф. Науменко и Л. Сосной}.

У входа в кинотеатр \enquote{Победа} было окно, в котором досужие прохожие и те, кто
заранее пришли на сеанс, могли удовлетворить свое любопытство. Там выставлялись
портреты кинозвезд того времени, цветные рекламные киноплакаты или коллажи,
отпечатанные на глянцевой фотобумаге, на темы анонсируемых художе­ственных
фильмов. Эти коллажи были составлены, как правило, из портрета главного героя
фильма и нескольких кадров из той же кинокартины. В период предвыборных
кампаний в эту своеобразную витрину помещали отпечатанные в два цвета на
небольших листах бумаги биографии кандидатов в депутаты.

Были еще и газетные витрины - плоские застеклен­ные ящики, укрепленые на двух
стойках. В них вывешива­лись центральные, республиканские и областные газеты.
Такая витрина была установлена и рядом с редакцией \enquote{Приазовского рабочего}.
Естественно, что в ней находился свежий номер главного городского
периодического издания. Его вывешивали сразу после выхода из типографской
машины. Так что все желающие могли ознакомиться с новостями, можно сказать, с
рассвета.

Как-то не припоминается, чтобы в наши дни кто-нибудь надолго останавливался у
витрин. А вот в прошлом около них то и дело топтались девчонки и мальчишки,
рассматривая и невиданных раз­меров папиросные коробки, и человечка-курильщика,
и муляжи ветчины, кусков вареной колбасы с вкраплениями сала, головок
голландского сыра (самое интересное, что эти продукты мало кто из них
пробовал).

Тогда ведь не было телевизоров, а тем более захватыва­ющих компьютерных игр,
которые сейчас умудряются \enquote{впихнуть} даже в мобильные телефоны. Молодости же
свойственны любопытство, стремление к новым зрительным впечатлениям. Это
стремление ребятня послевоенных лет удовлетворяла у витрин или в кино. Но за
кино надо было платить, а витрины были удовольствием абсолютно бесплатным.
