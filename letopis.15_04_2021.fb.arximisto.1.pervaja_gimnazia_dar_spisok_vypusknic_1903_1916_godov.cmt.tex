% vim: keymap=russian-jcukenwin
%%beginhead 
 
%%file 15_04_2021.fb.arximisto.1.pervaja_gimnazia_dar_spisok_vypusknic_1903_1916_godov.cmt
%%parent 15_04_2021.fb.arximisto.1.pervaja_gimnazia_dar_spisok_vypusknic_1903_1916_godov
 
%%url 
 
%%author_id 
%%date 
 
%%tags 
%%title 
 
%%endhead 

\qqSecCmt

\iusr{Лариса Колесниченко}

Учебно-воспитательный комплекс \enquote{гимназия-школа} 1 находится по адресу
ул. Воинов-Освободителей (на Левом)

\iusr{Cathrin Dakova}

Анна! Вы не только талантливая художница и замечательная учительница, но ещё и
любитель истории! Удивительный человек, однако!

\iusr{Татьяна Дубинина}

Могут-ли горожане посетить музей при Мариинской ним
гимназии?

\begin{itemize} % {
\iusr{Yaroslav Fedorovskyi}
\textbf{Татьяна Дубинина}

вы можете согласовать посещение музея с \textbf{Ганна Дежец.}

\begin{itemize} % {
\iusr{Татьяна Дубинина}
\textbf{Yaroslav Fedorovskyi}, спасибо большое. Реально с друзьями давно это обсуждали.

\iusr{Yaroslav Fedorovskyi}
\textbf{Татьяна Дубинина}, у вас есть предки, которые учились в этой гимназии?

\iusr{Татьяна Дубинина}
\textbf{Yaroslav Fedorovskyi}, нет, просто интересуюсь историей города в котором проживаю.

\iusr{Yaroslav Fedorovskyi}
\textbf{Татьяна Дубинина}, приходите в музей)

\iusr{Татьяна Дубинина}
\textbf{Yaroslav Fedorovskyi}, обязательно организую группу друзей и будем!

\iusr{Yaroslav Fedorovskyi}
\textbf{Татьяна Дубинина}, класс!

\iusr{Лилиана Тельнова}
\textbf{Татьяна Дубинина}, добрый день! Хочу присоединиться. Ответить можете в мессенджер

\iusr{Yaroslav Fedorovskyi}
\textbf{Лилиана Тельнова} , приходите)

\iusr{Валерия Каткова}

Были семьёй на экскурсии - интересно, познавательно, хранитель музея очень
приятный человек. Получите удовольствие!

\end{itemize} % }

\iusr{Татьяна Дубинина}

Да, Лилиана будем определяться с датой.

\end{itemize} % }

\iusr{Natalia Matuzna}

Подскажите пожалуйста, где можно найти список выпускников мужской гимназии Мариуполя?

\iusr{Дмитрий Читах}

Подскажите, а списки 1917-18 не сохранились?

\begin{itemize} % {
\iusr{Yaroslav Fedorovskyi}
\textbf{Дмитрий Читах}, к сожалению, не сохранились. В 1918 году выпуск гимназисток не состоялся из-за событий 1917 года.

\begin{itemize} % {
\iusr{Galina Flora}
\textbf{Yaroslav Fedorovskyi} моя бабушка Асланова Евдокия Николаевна, 1899г.р. училась в гимназии в 1917-1919 г., а затем преподавала там же. Аттестат сохранился и копия его находится в музее.

\iusr{Galina Flora}
\textbf{Yaroslav Fedorovskyi} очень благодарна Вам за списки! Нашла сестру своего дедушки Сёмкину Анастасию, которая окончила гимназию в 1912г, а потом преподавала там же ( в 1 школе) до 1941г. В музее находится ее портрет - Кужельная Анастасия Георгиевна. 
\end{itemize} % }

\iusr{Дмитрий Читах}

Ярослав, моя прабабушка Душка Евфимия Марковна, 1899 г.р., заканчивала
гимназию. У меня сохранилось фото, где она запечетлена с гимназистками,
датированное 3-м декабря 1918 года.

\begin{itemize} % {
\iusr{Yaroslav Fedorovskyi}
\textbf{Дмитрий Читах}, если можно, поделитесь фотографии и документами с музеем, пожалуйста.
Если есть история жизни вашей прабабушки, пожалуйста, поделитесь тоже.

\iusr{Galina Flora}
\textbf{Дмитрий Читах} моя бабушка Асланова Евдокия Николаевна, 1899г.р. возможно училась с Вашей прабабушкой в 1917-1919 г. Аттестат с печатью, подписями всех преподавателей сохранился, передала в музей вместе с фотографиями. Дмитрий, очень хотелось бы увидеть вашу фотографию гимназисток. 
\end{itemize} % }

\iusr{Дмитрий Читах}

Ярослав, не вопрос, но для начала хотелось бы уточнить учебное заведение. В Мариуполе ведь не одна гимназия была?

\begin{itemize} % {
\iusr{Yaroslav Fedorovskyi}
\textbf{Дмитрий Читах}, ул. Греческая 44. Здание бывшей Мариинской женской гимназии.

\iusr{Іванова Яна}
\textbf{Дмитрий Читах} в Мариуполе была еще Александровская мужская гимназия.
\end{itemize} % }

\iusr{Дмитрий Читах}

Вы не поняли. Насколько я знаю, было несколько женских гимназий.

\begin{itemize} % {
\iusr{Yaroslav Fedorovskyi}
\textbf{Дмитрий Читах}, я понимаю вас. Приносите фотографии в музей или поделитесь в личных сообщениях фотографией. Попробуем определить.
\end{itemize} % }

\iusr{Дмитрий Читах}

\href{https://www.facebook.com/groups/rodnoimariupol/permalink/2226607087648152}{%
На фото моя прабабушка Душка Евфимия Марковна (1899-1985), Дмитрий Читах, Рідний Маріуполь, 05.03.2020}
\footnote{На фото моя прабабушка Душка Евфимия Марковна (1899-1985). Это последний курс гимназии. Она сидит в центре, скрестив руки. Прабабушка - гречанка, дочь доктора и заведующего аптекой в Старой Ласпе Душка Марка Марковича (1866-1937). Прабабушку я хорошо помню. Каждое лето гостил у нее в Мариуполе, на Спартаковском переулке. Евфимия Марковна была интеллегентной, образованной женщиной. Всю жизнь проработала учителем в школе. За доблестный труд, перед войной её дважды представляли к ордену Ленина, но, так как она была женой репрессированного, то награду так и не получила. Бабушка имела очень сильный характер и до последнего дня была полна энергии. Это был очень трудолюбивый и творческий человек. Царствие ей небесное и вечная память!,\par \url{https://www.facebook.com/groups/rodnoimariupol/permalink/2226607087648152}}

\iusr{Дмитрий Читах}

Фото могу выслать на Вайбер или электронную почту. Я живу в Виннице, в Мариуполе бываю редко. 

\begin{itemize} % {
\iusr{Yaroslav Fedorovskyi}
\textbf{Дмитрий Читах}, отправил email в личные сообщения.
\end{itemize} % }

\iusr{Дмитрий Читах}

Я знаю, но прабабушка там вряд ли училась )

\iusr{Дмитрий Читах}

Галина, вполне возможно. Моя прабабушка на фото сидит, скрестив руки.

\iusr{Дмитрий Читах}

\ifcmt
  tab_begin cols=1,no_fig,center,separate

     pic https://scontent-ams2-1.xx.fbcdn.net/v/t1.6435-9/174145287_4100184913360434_1607502554365708324_n.jpg?_nc_cat=105&ccb=1-7&_nc_sid=dbeb18&_nc_ohc=KWl6bEPTAF8AX_YaCG7&_nc_ht=scontent-ams2-1.xx&oh=00_AfCL8bbbBL1y3RWOUf9l9-BC1UPoEEoHVa45ptXhZb5Zuw&oe=65483FE0

		 pic https://scontent-ams4-1.xx.fbcdn.net/v/t1.6435-9/174275745_4100186683360257_3645233300831104794_n.jpg?_nc_cat=102&ccb=1-7&_nc_sid=dbeb18&_nc_ohc=Ko0mIwwfqUAAX_xdlJe&_nc_ht=scontent-ams4-1.xx&oh=00_AfBePalKySubASAkGmwAGC5GHBhQDaAXjDEgXToRxh8Fgg&oe=6548383D

  tab_end
\fi

\end{itemize} % }
