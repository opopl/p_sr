%%beginhead 
 
%%file 27_08_2023.fb.mariupol.muzej.mkm.1.zustrich_impuls_vidrodzhennja
%%parent 27_08_2023
 
%%url https://www.facebook.com/100093184796939/posts/pfbid01AGzhk9drFY3RYtkZkife2WKx3bBdXgj9MM6YN8WBXuHuBqLpRp8jnmBZYqgvGJRl
 
%%author_id mariupol.muzej.mkm
%%date 27_08_2023
 
%%tags 
%%title Зустріч "Імпульс відродження"
 
%%endhead 

\subsection{Зустріч \enquote{Імпульс відродження}}
\label{sec:27_08_2023.fb.mariupol.muzej.mkm.1.zustrich_impuls_vidrodzhennja}

\Purl{https://www.facebook.com/100093184796939/posts/pfbid01AGzhk9drFY3RYtkZkife2WKx3bBdXgj9MM6YN8WBXuHuBqLpRp8jnmBZYqgvGJRl}
\ifcmt
 author_begin
   author_id mariupol.muzej.mkm
 author_end
\fi

📣 Маріупольці в декількох містах взяли участь в офлайн та онлайн-зустрічі
\enquote{Імпульс відродження}

🌟Цікавий та важливий діалог про сенс мистецтва в умовах війни та його вплив на
нашу культуру, патріотизм і психологічне здоров'я пройшов разом з одеським
художником Анатолієм Кравченком та іншими фахівцями.

🔗Захід транслювався для учасників в різних містах: Одеса, Чернівці, Тернопіль,
Кропивницький, Київ та ін.

🎭Говорили про важливість мистецтва як засобу відгуку на сучасні події, про те,
як арт-терапія може допомогти тим, хто зазнав травми.

❤️Спеціальна подяка нашим спікерам - Анатолію Кравченку, Стеллі Мунтян та Аліні
Тишковій за цінні думки і ділові поради.

Завжди важливо розуміти думки та пріоритети тих, для кого створюються події.
Тож, просимо вас поділитись враженнями за посиланням:
\url{https://forms.gle/ERombcSatBbo5p9n7}

Також можна залишити відгук про тему зустрічі.

Ваша думка має велике значення для нас! 📝

🤝Разом ми зробимо більше для розвитку мистецтва та культури в Україні!

Організатори події: Маріупольський краєзнавчий музей та молодіжна організація
\enquote{Митецький Гарт}.

\#mkmmariupol \#mariupol \#маріупольськиймузей \#маріуполь
\#відродимомаріупольськиймузей \#художник \#захід
