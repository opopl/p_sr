% vim: keymap=russian-jcukenwin
%%beginhead 
 
%%file poetry.rus.mykola_dudar_1950.analytics.zvychajne_chudo
%%parent poetry.rus.mykola_dudar_1950
 
%%url http://maysterni.com/publication.php?id=148279
%%author 
%%tags 
%%title 
 
%%endhead 

\subsubsection{Звичайне Чудо Миколи Дудара}
\label{sec:poetry.rus.mykola_dudar_1950.analytics.zvychajne_chudo}
\Purl{http://maysterni.com/publication.php?id=148279}

\paragraph{Наталка Пасічник з Миколою Дудар}

В українській поезії є безліч чудових письменників, яких важко віднести до
якоїсь конкретної генерації. Хоча б тому, що писати починають пізніше за своїх
ровесників, але одразу пропонують світові зрілу лірику. Одним із таких авторів
є Микола Дудар. Його вірші - лаконічні, стислі, трохи інфантильні (це я
по-доброму!) і завжди залишають враження "звичайного чуда":

\paragraph{Камінь}

Вродився каменем лежачим…
І наче сню. А наче й ні.
Я наче є. Немає наче.
Спіткнеться хтось -- завжди пробачу…
Так рік за роком. День по дні.
Їх незліченно .Та напрочуд
Всі непомітні. Не мої.
То ворон сяде -- дзьоба точить.
То жабка щось своє торочить.
То шурхотне луска змії…
І раптом якось після зливи…
І раптом якось по весні…
Мене підняв ти, Майстре сивий!
Я знаю:має статись диво.
Тому так боляче мені…

Можна багато говорити про впливи (гадаю, що таких віршів у своїх доробках не
посоромилися б ні Микола Вінграновський, ні Леонід Талалай, а саме їхні
інтонації я інколи вбачаю у цій поезії), але як на мене, то головною ознакою
Дударевих віршів є його безстрашність здатися не таким, як очікують інші, майже
дитяча безпосередність:
* * *

Ревнуючи,сповідую свій страх…
Тому в очах набрякла тінь спокути.
Мабуть отак і жив ночами Бах,
До звуків нескінченності прикутий…
Церковний хор навіється в село.
У димарях затісно стане вітру…
Прийдеш і скажеш: «Так уже було --
Світ вічний і минущий, мов півлітра»…
І гул торнадо, й скрекіт канонад –
Все перекриють нетутешні звуки…
Скрізь буде Бах .І дух його --- як брат.
І час тектиме нам з тобою в руки…

А ще Дудар трохи пафосний, але цей пафос не дратує, а розчулює:
* * *

Я знаю: ви -- створіння неземне.
І все, що я скажу вам -- не до речі...
Ви вчасно попередили мене:
Нам небезпечно бачитись під вечір...
Заклавши камінь у безсоння храм,
Коли ще ціла вічність до світання,
Не можу відповісти на питання:
Чому я знов іду назустріч вам...
Я знаю як пройти отой рубіж
Щоб вас не зачепити ненароком
В моєму серці наче гострий ніж...
До вас лишилося пройти півкроку
Які ж вони важкі ці перепони
О дай мені дійти і роздивитись
Упасти до колін і помолитись -
Допоки ще дзвенять церковні дзвони

Найкращі рядки - завжди випадкові, ненадумані, а тому, цитуючи самого автора,
скажу: "Світ вічний і минущий, мов півлітра". Врешті-решт так і є.
