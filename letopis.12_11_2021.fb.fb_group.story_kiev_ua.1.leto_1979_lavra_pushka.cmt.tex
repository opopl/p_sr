% vim: keymap=russian-jcukenwin
%%beginhead 
 
%%file 12_11_2021.fb.fb_group.story_kiev_ua.1.leto_1979_lavra_pushka.cmt
%%parent 12_11_2021.fb.fb_group.story_kiev_ua.1.leto_1979_lavra_pushka
 
%%url 
 
%%author_id 
%%date 
 
%%tags 
%%title 
 
%%endhead 
\zzSecCmt

\begin{itemize} % {
\iusr{Tatiana Thoene}
Жаль, такую историческую находку потеряли.

\begin{itemize} % {
\iusr{Alexey Paschenko}
\textbf{Tatiana Thoene} кто-то теряет, а кто-то находит....

\iusr{Tatiana Thoene}
\textbf{Alexey Paschenko} и снова теряет ))
\end{itemize} % }

\iusr{Alexey Paschenko}

Ну всё, после этой публикации охотники за кладами хлынут с ломами и лопатами
наперевес, в дополнение к металлоискателям, искать эту пушку. Она же может
десятки тысяч долларов стОить!

\iusr{Yuri Zmeelov}

Ещё в начале семидесятых там ещё оставались т.н. \enquote{Генеральские дачи}. Со
стороны подъездной дорожки забор был если и не высокий, то, по крайней мере
надежный, а вот между участками - символические. Нас там гоняли с собаками
когда мы перелазили через заборы и сливы с деревьев воровали. Между участками в
тот день гоняла стая собак. Один раз я, единственный, прозевал момент, не успел
добежать до забора и меня окружила эта внушительная свора. Все с ужасом, уже
из-за забора наблюдали за происходящим. Интуитивно я резко присел по-собачьи и
опять же по собачьи протяжно завыл. Мощный кобель, предводитель стаи, сначала
охел, резко затормозил, сел передо мной (морда в морду!) и пристально, не
отрываясь смотрел мне в глаза. Его бока сочувственно вздрагивали и через минуту
он стал выть вместе со мной. Потом по очереди присоединилась вся стая. Я
потихоньку, не переставая выть, бочком, приближался к спасительному забору. К
вою присоединились все собаки округи. Стоило мне на секунду расслабиться и
замолчать, как вожак (а за ним и вся стая!) вздыбливала холки и начинала рычать
готовясь разорвать меня на куски! А эти сс¥ки )) сидят на заборе и ржут! )).
Короче! Когда до забора остался один прыжок я его сделал!)) Сразу же мои
друзья-однокласники услышали от меня неожиданно много очччень интересных слов,
но ржали мы потом вместе и долго! ))

\iusr{Valentina Urban}

\ifcmt
  ig@ name=scr.hands.applause
  @width 0.2
\fi

\iusr{Любовь Белоцерковец}
Помню ....
а там, где сейчас Родина - Мать, были рвы и кустарники. Мы, чтобы сократить
путь, пробирались по ним от Чешской до парка Примакова...

\begin{itemize} % {
\iusr{Leonid Dukhovny}
\textbf{Lyubov Belotserkovets} 

Мне надо было срочно попасть с Русановки в район пл.Леси Украинки. Удалось
быстро поймать такси и, когда я назвал адрес, шеф, мило улыбнувшись, сказаг:
\enquote{Да знаю это место - ЗАЖОПЬЕ !} Так я узнал народное название
старинного киевского района, расположенного между спиной монумента Родины-
матери и спиной великой поэтессы Леси Украинки. А вы говорите Печерск, Евбаз,
Демеевка, Подол...!

\begin{itemize} % {
\iusr{Юрий Руденко}
\textbf{Leonid Dukhovny} междужопье.

\iusr{Любовь Белоцерковец}
\textbf{Leonid Dukhovny} ну, это вы уже перегнули палку))). Хоть и были чащи за Суворовским училищем! Признаю!!!Никогда Зажопье не называли)))
Междужопье называли потому, что два памятника открыли, памятник Леси Украинки в 70-х годах, и Родину - мать в 80-х)))
Когда построили Русановку, мы ее называли Киевской Венецией!

\iusr{Любовь Белоцерковец}
\textbf{Юрий Руденко} Междужопье - это название появилось позже... Когда построили Родину - мать...

\iusr{Юрий Руденко}
\textbf{Leonid Dukhovny} в 80-х.

\iusr{Каченко Евгений}
\textbf{Leonid Dukhovny} междужопье, если уж быть точным

\iusr{Михаил Харченко}
\textbf{Leonid Dukhovny}, в моей памяти - \enquote{междужопье}: территория между указанными памятниками...

\iusr{Leonid Dukhovny}
\textbf{Lyubov Belotserkovets} Пожалуйста, давайте без \enquote{загибов}. Если вы не слышали, значит такого быть не может?!
\end{itemize} % }

\iusr{Leonid Dukhovny}

Дорогие мои оппоненты Вы очевидно, правы, но мой таксист сказал именно
ЗАЖОПЬЕ, что меня очень развеселило и я вспомнил московское ЗАРЯДЬЕ. Верю, что
в ваших кругах именовали МЕЖДУ.... Но в наших палестинах бытовало ЗА... Мой
старинный приятель Владимир Баранников, известный собиратель андеграудных
песен, именно так называл этот район, где стоял его родовой особняк !... Где-то
у меня должно быть фото front-yard его здания, где на скамеечке. в кустах
цветов, сидим я, он, Никита Джигурда и известный высоцковед
Епштейн..... Впрочем, о чём спор, господа...

\end{itemize} % }

\iusr{Мария Иванова}
На фото район Зверинецкого кладбища?

\iusr{Раиса Сухарева}
А красота то какая, лепота. Величественный Днепр. Представляю нашу южную ночь,
разлита нега, тепло, луна и тишина...

\iusr{Мария Тимошенко}

В 1950 годах мы жили ул Цитедельная 15 помню все что рассказано там был дом
Чуйкова я в одном классе занималась с сыном маршала школа 90 открыта 1954 г кто
помнит первую учительницу Анастосию Владимировну моя девичья фамилия Глузман
Мария

\begin{itemize} % {
\iusr{Мамэд Рустамов}
\textbf{Мария Тимошенко}
Маиии глазаааа
ЦитЕдельная
АнастОсию
А знаки препинания?
Цимес, а не каммент

\begin{itemize} % {
\iusr{Sasha Ty}
\textbf{Мамэд Рустамов}
Це ви до чого?  @igg{fbicon.thinking.face} 
А як до того щоб порахувати вік Марії? @igg{fbicon.face.grinning.big.eyes} 
А то Цимес, цимес ...

\iusr{Мамэд Рустамов}
\textbf{Sasha Ty}
І шо?
\#какаяразніца кіко їй років?
Хоча - так.
Вона елеганцько зруйнувала міф, про якісну савєцьку освіту

\iusr{Мария Тимошенко}
\textbf{Мамэд Рустамов} если не можешь высказывать по русски и писать так и не пробуй какая то ерунда

\iusr{Мамэд Рустамов}
\textbf{Мария Тимошенко}

\ifcmt
  ig https://i2.paste.pics/028c871ec091b2a2d4cabfa80b533f52.png
  @width 0.4
\fi

\end{itemize} % }

\end{itemize} % }

\iusr{Игорь Гаврилов}

Не случайно поселения что были в тех местах назывались одно Запещерка, а другое
Батарейка. Названия очень логичные, учитывая расположение за Дальними пещерами, и
на месте давней крепостной батареи. Я хорошо знал эти места, а мой дед там даже
жил в двадцатые годы прошлого века.


\iusr{Клим Форманчук}
Интересно бы сейчас с той точки сделать фото и сравнить.

\iusr{Евгения Бочковская}

Это фото с высотки усадьбы на бывшей ул.Церковной. (потом ул.Верхняя). Я этот
вид рисовала, будучи ребёнком. А внизу просматриваются фонари уже построенной
автострады. Это уже послевоенное фото. А на плато горы над улицей, была
зенитная часть. Зачехлённые зенитки стояли сверху, прямо над улицей, стволами в
небо.

Очень хорошо помню ! И не удивительно, что на этой высотке что-то ещё в земле
могло сохраниться.

\begin{itemize} % {
\iusr{Григорій Цілуйко}
\textbf{Евгения Бочковская} - в зенітно-ракетному дивізіоні у 1969 році починав свою службу мій брат - Цілуйко Іван Олександрович .
\end{itemize} % }

\iusr{Олег Гончаров}

Интересно было бы узнать, что чего находил в Киеве, в 70х в районе ул. Урицкого
откопал своего рода троицу: самовар, пулемет и австрийский крейцер, символично и
знаково оказалось.

\begin{itemize} % {
\iusr{Евгений Гройсман}
\textbf{Олег Гончаров} отец в 50х жил на ямской. Из выкопанного у него была коллекция крестов немецких, штык и офицерская сабля.

\iusr{Олег Гончаров}
\textbf{Евгений Гройсман} первая мировая, перевязочный пункт?
\end{itemize} % }

\iusr{Yuri Zmeelov}

А я ещё помню как такси поднималось от Наводницкой вверх к Лавре. Машина, гремя
колесами по брусчатке (мягкой радиальной резины ещё не было), проезжала через
одни, а, затем, по извилистому участку, через вторые ворота. Ворота были узкие
(проехать могла только одна машина) и когда с другой стороны ворот появлялась
встречная водители жестами решали кто проедет первым.

\begin{itemize} % {
\iusr{Олександр Попов}
\textbf{Yuri Zmeelov} Московские ворота - они и сейчас стоят, только закрытые и через них никто не ездит - рядом ресторан Царское село.

\ifcmt
  ig https://scontent-frt3-2.xx.fbcdn.net/v/t39.30808-6/255700643_4792528434131362_7363529627142256739_n.jpg?_nc_cat=103&ccb=1-5&_nc_sid=dbeb18&_nc_ohc=Ml_WkkcYR6oAX_U6dQS&_nc_ht=scontent-frt3-2.xx&oh=00_AT8JssMLy_p4G9bhUqRpOpQInFJwgyDsIQYcPoqqsjilnQ&oe=61CCC6BD
  @width 0.4
\fi

\begin{itemize} % {
\iusr{Yuri Zmeelov}
\textbf{Олександр Попов} Знаю. Просто ярко вспомнились некоторые детали.

\iusr{Олександр Попов}
\textbf{Yuri Zmeelov} Согласен, я тоже вспоминаю часто, как мы проезжали через них на своей Победе...
\end{itemize} % }


\iusr{Олександр Попов}
Тут мы и ездили...

\ifcmt
  ig https://scontent-frx5-1.xx.fbcdn.net/v/t39.30808-6/256755018_4792926607424878_2874720376921953836_n.jpg?_nc_cat=110&ccb=1-5&_nc_sid=dbeb18&_nc_ohc=ZOA7H5qyX0QAX-nskDA&_nc_ht=scontent-frx5-1.xx&oh=00_AT9XNkqxJtnjB3IQ3fv9wxhv0fJ6TCdg9gD5iyWWC31AEg&oe=61CB9483
  @width 0.4
\fi

\end{itemize} % }

\iusr{Xenya Tazkaya}

Там был ещё вход в подземный ход, такая небольшая арочка в земле, сверху
украшенная кирпичами. Высотой 30 см в самом высоком месте. Мы собрались
компанией и полезли туда. Там было подземелье в форме креста, причем в одном
месте кладка была более новая. Думаю, это бы какой-то подземный ход, из
крепости или из Лавры. Потолки были высокие, не помню, чтобы мы шли сгибаясь.
Вылезли, мечтая разобрать \enquote{новую} кладку. А там нас ждали два мужика, возможно
хотели попугать или поприставать. Наш визит явно не прошел незамеченным. Но 4
суровые школьницы, держащие в руках по половинке кирпича мужиков не вдохновили.
 @igg{fbicon.smile}  Но больше мы туда не вернулись ...

\begin{itemize} % {
\iusr{Yuri Zmeelov}
\textbf{Xenya Tazkaya} 

Насчет подземелий в форме креста. это были пороховые склады Киевской крепости.
В один из них мы в детстве лазили. Не уверен, но это похоже где-то здесь
50.4229434,30.5429793,2184 или 50°25'42.3"N+30°33'08.1"E Сейчас это частная
территория. Помню что холм в котором был этот подземный ход выглядел в форме
четырехгранной. усеченной пирамиды, а сам холм хорошо просматривался из окон 27
трамвая шедшего по Старонаводницкой.

\begin{itemize} % {
\iusr{Xenya Tazkaya}
\textbf{Yuri Zmeelov} , да, этот холм был виден из трамвая.

\iusr{Yuri Zmeelov}
\textbf{Xenya Tazkaya} А его место там где я указал ?
\end{itemize} % }

\end{itemize} % }

\iusr{Nadia Baranovsky}
Сколько нового узнаешь о Киеве, когда читаешь комментарии. Спасибо всем

\iusr{Микола Данилейко}
І все?


\end{itemize} % }
