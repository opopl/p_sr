% vim: keymap=russian-jcukenwin
%%beginhead 
 
%%file 02_12_2020.news.ru.lenta_ru.1.karabah.vozvraschenie_zemel
%%parent 02_12_2020.news.ru.lenta_ru.1.karabah
 
%%url 
 
%%author 
%%author_id 
%%author_url 
 
%%tags 
%%title 
 
%%endhead 
\subsubsection{Возвращение земель}

Азербайджанские войска вошли в Лачинский район, носивший название Кашатахский в
составе непризнанной Нагорно-Карабахской республики (НКР, Республика Арцах).
Это последняя территория, переданная Баку в рамках соглашения о прекращении
военных действий в Карабахе, подписанного 9 ноября лидерами России, Армении и
Азербайджана. По условиям перемирия, Баку также получил Кельбаджарский (ранее
Шаумянский), Агдамский районы и сохранил за собой территории НКР, занятые в
ходе боев.

В Азербайджане итоги войны восприняли как победу над «кровным врагом». Народ
высыпал на улицы отпраздновать случившееся, несмотря на комендантский час. По
официальной версии Баку, Нагорный Карабах был оккупирован армянами, и взятие
территорий под контроль для азербайджанцев является их освобождением.

\ifcmt
pic https://icdn.lenta.ru/images/2020/12/01/17/20201201172622032/pic_0d2df2240c5f327fff375387aaf5c635.jpg
caption «Би-би-си» Фото: Aziz Karimov / Reuters 
\fi

Президент Азербайджана Ильхам Алиев охотно делится подробностями исторической
онлайн-встречи: он рассказал, что хотел заставить армянского премьера Никола
Пашиняна так же публично подписать заявление, «как и после окончания Второй
мировой войны ставленники Гитлера подписали акт о капитуляции перед военными
руководителями союзных стран». Однако, по его словам, президент России Владимир
Путин попросил не настаивать на этом, и азербайджанский лидер согласился,
поскольку премьер Армении был «уже достаточно унижен и наказан».

\begin{leftbar}
\emph{Я сказал, что не возражаю. Пусть подписывает там, где хочет: в темной комнате,
на складе, вдали от камер пусть подпишет. Так и случилось}\par
				\textbf{Ильхам Алиев}\par
				\textbf{президент Азербайджана}\par
\end{leftbar}

Азербайджанский лидер уже поспешил лично посетить некоторые из возвращенных
территорий. Он пообещал, что все разрушенные города будут восстановлены, а
карабахские беженцы Азербайджана, вынужденные покинуть свои дома во время
первой войны 1990-х годов, уже скоро смогут вернуться домой. Баку строит
амбициозные планы по развитию новых земель: строительство автомобильных и
железных дорог, развитие солнечной энергетики и туризма, который процветал там
в советские годы. Работы уже начались.

И судя по официальным заявлениям, Азербайджан не прочь оплатить их из кармана
соседа. Алиев уже пообещал, что совместно с международными организациями и
экспертами подсчитает нанесенный Азербайджану ущерб за 30 лет «оккупации» и
потребует от Армении компенсации. По его словам, армяне проводили на
азербайджанских землях настоящий террор, уничтожая все, что только можно.

\begin{leftbar}
\emph{Эти воры вырубают и вывозят леса, чтобы продать их в Армении. Поджигают дома,
которые строили не они. Они поджигают школы, убивают домашний скот,
который не могут увезти с собой. Посмотрите, с кем, с какими дикарями
мы были лицом к лицу!}\par
\textbf{Ильхам Алиев}\par
обращение к нации 25 ноября (цитата «Sputnik Азербайджан»)	
\end{leftbar}

В Карабахе на принадлежащих теперь Баку территориях в последние дни
действительно горели дома, и уничтожалось все, что нельзя забрать с собой.
Только вот жившие здесь столетиями этнические армяне совершенно не согласны с
Алиевым по поводу того, кто из них дикарь. Карабахцы до сих пор пересказывают
истории о зверствах азербайджанских солдат на почве межрелигиозной и
межэтнической ненависти в 1980-1990-е годы и уверены, что это повторится.

\ifcmt
pic https://icdn.lenta.ru/images/2020/12/01/18/20201201184525211/pic_2bca9d8c793897e10fed11afa5bd2ab2.jpg
caption Карабахский конфликт 1990-х годов, город Мартакерт Фото: Р. Мангасарян / РИА Новости
\fi
