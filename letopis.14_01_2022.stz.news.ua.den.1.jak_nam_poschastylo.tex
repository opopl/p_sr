% vim: keymap=russian-jcukenwin
%%beginhead 
 
%%file 14_01_2022.stz.news.ua.den.1.jak_nam_poschastylo
%%parent 14_01_2022
 
%%url https://day.kyiv.ua/uk/blog/suspilstvo/yak-nam-poshchastylo
 
%%author_id ljubka_andrii
%%date 
 
%%tags ukraina,rossia,kazahstran,maidan2
%%title Як нам пощастило!
 
%%endhead 
\subsection{Як нам пощастило!}
\label{sec:14_01_2022.stz.news.ua.den.1.jak_nam_poschastylo}

\Purl{https://day.kyiv.ua/uk/blog/suspilstvo/yak-nam-poshchastylo}
\ifcmt
 author_begin
   author_id ljubka_andrii
 author_end
\fi

Спостерігаючи за подіями в Казахстані, я не міг позбутися однієї нав’язливої
думки: як нам пощастило! Адже на місці казахів могли бути ми, наша держава.
Зрештою, саме такий сценарій для нас і готували – і лише наша наполегливість,
рішучість і дрібка фарту врятували Україну.

Немає сумнівів, що Янукович і його кремлівські покровителі готували в 2014-му
пекельний сценарій для України. Але в підсумку – можливо, через те, що
кремезний Янукович насправді виявився боягузом і передчасно втік з країни – нам
вдалося уникнути окупації всієї України, втративши (сподіваюсь, тимчасово) лише
Крим і частину Донбасу.

Безумовно, приклад сусідньої Білорусі теж дуже актуальний для України. Скільки
разів наша молода й нестійка демократія могли збитися на манівці, піти не тим
шляхом, закінчившись фазою кривавого авторитаризму? Адже сценарій жорсткої
диктатури міг бути реалізованим і в нашій країні – він навіть починав
розкручуватися, коли Небесна Сотня врятувала нас. Активна й свідома частина
українців або сиділа б у тюрмах, або була б змушена емігрувати.

Зрештою, уже кілька років спостерігаючи за новинами з Росії, я невимовно
тішуся, що ми тепер по цей бік нової «залізної завіси». Врятовані від
кремлівської маячні й імперської пропаганди, ми живемо мовби в іншому світі, на
іншій планеті, і головна ознака нашого життя на фоні цих маніакальних сусідів –
нормальність. І за цю нормальність я готовий багато чого віддати.

Тільки тепер, через 8 років після Революції Гідності, побачивши, як
шизофренічно й криваво розвиваються події в Росії, Білорусі й Казахстані, що ще
три декади тому перебували з Україною в одній державі, можна чітко ствердити:
Майдан таки переміг! Так, можливо, нам не до кінця вдалося пебороти корупцію, а
рівень добробуту в нашій країні все ще не такий, як хотілося б; так, можливо,
наші президенти дратують нас – когось теперішній, а когось і попередній, та
головне – що вони змінюються, що ми реально впливаємо на владу в своїй країні.

Ми живемо в нормальній країні – яку можна й треба критикувати, але яка дуже
якісно відрізняється від багатьох наших колишніх «співкамерників» по СРСР. Ми
дорого платимо за цю нормальність і свою незалежність, ми розділені й
пересварені, часом розчаровані, інколи злі на повільність реформ, але все ж у
нас є розкішне відчуття, що країна таки рухається в правильному напрямку – хай
навіть черепашачими кроками!

Уже за тиждень ми знову поринемо в траур, згадуючи невинних жертв, розстріляних
на київському Майдані вісім років тому. Їхні чисті й шляхетні душі довіку
будуть для нас тягарем, адже мова про героїв, що власними життями заплатили за
нашу свободу. Ці смерті для нас є і будуть довічним кармічним боргом, адже ми
зобов’язані перед Небесною Сотнею клятвою побудувати успішну Україну.

Напередодні цих чорних днів хочеться сказати таке: страшна ціна, яку Україна
заплатила Майданом і війною за власну незалежність, не була даремною. Бо
Революція Гідності таки перемогла – і на фоні постсовєтського простору ця
перемога з кожним роком стає все  очевиднішою. Ми зберегли і нашу гідність, і
нашу державу, і демократію – хіба ж це не диво?!

