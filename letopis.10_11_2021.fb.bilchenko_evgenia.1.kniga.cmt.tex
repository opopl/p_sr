% vim: keymap=russian-jcukenwin
%%beginhead 
 
%%file 10_11_2021.fb.bilchenko_evgenia.1.kniga.cmt
%%parent 10_11_2021.fb.bilchenko_evgenia.1.kniga
 
%%url 
 
%%author_id 
%%date 
 
%%tags 
%%title 
 
%%endhead 
\subsubsection{Коментарі}
\label{sec:10_11_2021.fb.bilchenko_evgenia.1.kniga.cmt}

\begin{itemize} % {
\iusr{Алексей Бажан}

Контроль одна их трех главных тем Берроуза, а через него индастриала. Ладно,
как это не креативно, все равно не совсем академично. Но есть же еще хотя бы
Leçons des ténèbres, где более чем полностью и нетривиально рассказано про
власть как контроль над производимой нею же неразличимостью/неопределенностью,
и что делать Партизану

\begin{itemize} % {
\iusr{Евгения Бильченко}
\textbf{Алексей Бажан} 

Со времён Берроуза и расчетов Римского клуба прошло больше трёх десятков лет,
пожинаем плоды. Если интересно, насколько академично, приобретаем мою версию за
1000 рэ в любом интернет-магазине Питера и Москвы. Есть совсем жёсткая
академическая версия вот этого всего не для всех, ее можно купить у меня за
донаты, помочь мне, буду рада. Реально рада, потому что я на идею без гонораров
работаю и без работы нахожусь.

\iusr{Евгения Бильченко}
\textbf{Алексей Бажан} есть ещё третья версия: написать самому, что делать партизану, мы вам сами заплатим тогда))
\end{itemize} % }

\iusr{Ольга Матвиенко}
Жень, браво! Ни убавить, ни прибавить.

\begin{itemize} % {
\iusr{Евгения Бильченко}
\textbf{Ольга Матвиенко} будущее покажет6 что там и как. будущее неотвратимо, жестоко и справедливо. проверено. И, потом ,мы всё равн овсе умрём.
\end{itemize} % }

\iusr{Denis Dunaev}

Единственный украинский интернет-магазин книг, где продаётся "Сентиментальное
насилие..." - дуркует. Пытался там заказать эту книгу, требует авторизации, при
авторизации не принимает адрес электронной почты и таким образом книгу я не
смог купить. Прошу у автора \textbf{Евгения Бильченко} номер карточки, и или через НП,
или при встрече в храме.

\begin{itemize} % {
\iusr{Евгения Бильченко}
\textbf{Denis Dunaev} Не дуркует. Боится. Мне звонил тамошний рук, просил, чтобы Киев я не отсвечивала. Батюшка, родной, я вам напишу в личку.
\end{itemize} % }

\iusr{Владислав Сушков}
Как хорошо написано предисловие - даже я понял ))) а что, серьёзно рецензировать будут Жижек и Кляйн?

\end{itemize} % }
