% vim: keymap=russian-jcukenwin
%%beginhead 
 
%%file topics.ukraina.rossia.ugroza_napadenia.preface
%%parent topics.ukraina.rossia.ugroza_napadenia
 
%%url 
 
%%author_id 
%%date 
 
%%tags 
%%title 
 
%%endhead 
\subsection{Передмова}

Тут ми зібрали деякі матеріали щодо так званого вторгнення Росії в Україну.
Зараз у нас вже місяць лютий 2022 року, і ми пишемо ці рядки, сидячи в затишній
Київській квартирі десь на Оболоні. Вечоріє. Сонце опускається все нижче і
нижче над Вічним Городом Ярослава і Володимира - Містом-Героєм Києвом,
запалюються вечірні ліхтарі і потихеньку падає температура надворі.

\ifcmt
  ig https://socportal.info/media/contentimages/1195d1e08bfd0e8f.jpg
	@wrap center
	@width 0.9
\fi

Добре, тут слід зауважити таке. Якщо Україна претендує на спадщину Київської
Русі, то вона також претендує на назву Русь, логічно ж, еге? А якщо так, то
Україна = Русь = Росія = Россия = Раша = Russia, тому що Росія - це всього
навсього грецький (візантійський) варіант слова Русь. А англійською Русь =
Russia (Руссія). А тому найперше треба зрозуміти, про що власне іде мова. Це як
треба розуміти - Україна, тобто Мала (Київська) Росія збирається напасти на
саму себе, чи як?  Чернігів збирається вторгнутись в Одесу, а Львів у Харків?
Ви тут спитаєте, про що тут пишуть ці божевільні?  А ми скажемо, що важливо
точно формулювати свої твердження. Бо інакше можна опинитись в болоті, і
копирсатись в ньому все своє свідоме життя, без надії коли-небудь з нього
вибратись. На даний момент ми таки в цьому болоті все еще знаходимось, але дай
Боже, ми все ж з нього колись виберемось. А для цього необхідно мати більш чітке та
послідовне мислення. Бо, як казав ще Декарт - Cogito, Ergo Sum - тобто, мислю,
значить, існую.

Так що, панове, формулюйте точніше свої твердження. Нам здається, більш точним
було б для початку таке формулювання: ймовірне вторгнення військ Російської
Федерації на території Київської Держави.

А далі...

А далі Ви знайдете різноманітні думки від різних людей щодо цього самого
вторгнення, а також наші власні думки щодо цього всього.

І знаєте... є три речі, які можна робити нескінченно:

\begin{itemize} % {
\item Блукати вуличками Києва;
\item Вивчати історію Києва;
\item Спостерігати за вторгненням Росії в Україну - найдовшим по часу в історії людства.
\end{itemize} % }

Приємного читання, незнайомий Читачу! Сподіваємось, тобі ця збірка стане в нагоді.

\clearpage
\subsubsection{Дивлюсь я на небо, та й думку гадаю... Чому я не сокіл, чому не
літаю?}

На нашу думку, думку пересічних Киян - вся ця істерія щодо вторгнення - є
просто психоз. Психоз хворого Західного світу, демонізувавшого до крайнощів в
своїй хворобливій уяві Російську Федерацію, і зокрема її президента -
Володимира Володимировича Путіна, який для нас не є ні Богом, ні злим демоном,
а в першу чергу просто людиною, до речі, досить вихованою і стриманою людиною.
Ми про Путіна взагалі мало думаємо, якщо що. Ну, є там в Кремлі Путін, ну і що?
Зате у нас, простих Киян, є наш золотоверхий Київ, є Собор Святої Софії, є парк
Шевченка, є найпрекрасніша у світі диво-ріка Дніпро, і ще багато чого є, чого
на жаль, громадяни в нашій державі не цінують і не люблять, чому доказ є досить
понурий, на сьогодення, стан нашого найголовнішого міста - міста Києва.

І істерія щодо \enquote{вторгнення} є також психоз украй хворого українського
суспільства... травмованого самим словом \enquote{Росія}, нездатного спокійно,
без емоцій, чути або казати це слово... суспільства, яке втратило твердий грунт
під ногами, внаслідок цього ставши надзвичайно вразливим до найрізноманітніших
інформаційних вірусів - ура-патріотизму з безглуздими кричалками, масовому
цькуванню діячів культури, спорту та науки, вірусу ненависті та русофобії,
вірусу втрати людської подоби як такої... А причина всьому цьому, на нашу
думку, є те, що Україна як суспільство втратила відчуття цінності Києва, так,
тисячолітнього Города Києва, для своєї культури, історії та свого майбутнього.
І зокрема у вас, так звані патріоти-активісти, і ви, так звані
\enquote{захисники} України, воїни так званої АТО, що насправді є незаконна
братовбивча громадянська війна, - незаконна і по конституції України, і просто
з погляду загальнолюдських цінностей, - совісті, любові, милосердя та
взаєморозуміння, - і яка, дай Боже, все ж таки колись закінчиться, - і ви,
свідки \enquote{святого} Майдану та так званої \enquote{революції гідності},
Київ, його тисячолітня історія, духовність, культура, пов'язана з ним - на
жаль, то є очевидно пусте місце...

Якщо ж Ви не вірите нашим словам про відсутність цінності Києва у сучасній
Україні, відсутності Києва у масовій свідомості нашого народу, і за старою
звичкою почнете кричати про всюдисущого злого Путіна і кремлівську пропаганду,
ну що ж, по-перше ми Вам скажемо, що Ви просто хворі, а по-друге, Ви можете
пересвідчитись самі в цьому, що Київ явно перебуває в досить поганому стані, і
це є великий сором для України і для нашого народу. От, наприклад, візьміть і
пройдіться самі вулицями Києва, брудними та розбитими, повними недопалків та
сміття; сядьте у брудну закопчену маршрутку, яка буде підстрибувати сотні раз
від поганих доріг; також загляньте на центральний вокзал Києва, подивіться
самі, в якому зараз стані площа перед вокзалом.  Зробіть це, і може, Ви все ж
почнете про щось замислюватись, і врешті-решт почнете щось робити конкретно для
Стольного Города Києва, так, саме для конкретного Києва, а не для якоїсь уявної
примарної жовто-блакитної України. Бо ж з тисячолітнього золотоверхого Києва
починається Україна; і зміниться Київ - зміниться і Україна.

І так, багато хто кричить: Українапонадусе! Славаукраїнігероямслава!
Славанаціїсмертьворогам! Путіннападе! Кримнаш! Росіяворог!  Чемоданвокзалросія!
Весьсвітзнами! Хтонескачетоймоскаль! Але мало хто з таким же натхненням
співає... Як тебе не любити, Києве Мій... Мало хто з таким же натхненням
перегортає сторінки книг, вишукуючи в них згадки про Київ.  А тим часом... А
тим часом Київ занепадає.  Брудний, засмічений Київ. І Київ понурий,
засмучений... Засмучений байдужістю громадян України до Серця своєї Країни -
вічно молодого Києва, що занепадав та відроджувався; палав у вогні війни, а
потім знову розквітав, як чудо-квітка.  Так може якраз зараз і наступає добра
нагода повернутись серцем і душею до Києва?

А як думаєте Ви?

\ifcmt
  tab_begin cols=2,no_fig,center

		 pic https://avatars.mds.yandex.net/i?id=9667cd9b8439e284a88399d7840124b8-5709479-images-thumbs&n=13
     pic https://avatars.mds.yandex.net/i?id=bf0b712e7993a05772436b1ad02889bd-3701558-images-thumbs&n=13
		 %pic http://www.gornitsa.ru/images/products/15/264/15264668-s2.jpg

  tab_end
\fi


\ifcmt
  tab_begin cols=2,no_fig,center

     pic https://restoranoved.ru/netcat_files/3/1/Kiev.jpg
		 pic https://avatars.mds.yandex.net/i?id=e0beba5a7bd41972c79951e43245237b-5755341-images-thumbs&n=13

  tab_end
\fi

\raggedcolumns
\begin{multicols}{3} % {
\setlength{\parindent}{0pt}

\obeycr
Двадцать второго июня,
Ровно в четыре часа
Киев бомбили, нам объявили
Что началася война.
\smallskip
Война началась на рассвете
Чтоб больше народу убить.
Спали родители, спали их дети
Когда стали Киев бомбить.
\smallskip
Кончилось мирное время,
Нам расставаться пора,
Я уезжаю и обещаю
Верным вам быть навсегда.
\smallskip
И ты смотри,
С чувством моим не шути,
Выйди подруга, к поезду друга,
Друга на фронт проводи.
\smallskip
Вздрогнут колеса вагона,
Поезд помчится стрелой,
Ты мне с перрона — я с эшелона
Грустно помашем рукой.
\smallskip
Рвалися снаряды и мины,
Танки гремели броней,
Ястребы красны в небе кружили,
Мчались на запад стрелой.
\smallskip
Началася зимняя стужа
Были враги у Москвы,
Пушки палили, мины рвалися
Немцев терзая в куски.
\smallskip
Кончился бой за столицу
Бросились немцы бежать
Бросили танки, бросили мины,
Несколько тысяч солдат.
\smallskip
Пройдут года,
Я снова увижу тебя,
Ты улыбнешься,
К сердцу прижмешься,
Вновь поцелуешь меня.
Ты улыбнешься,
К сердцу прижмешься,
Вновь поцелуешь меня.
\smallskip
Пройдут года,
Я снова увижу тебя,
Ты улыбнешься,
К сердцу прижмешься,
Вновь поцелуешь меня.
Ты улыбнешься,
К сердцу прижмешься,
Вновь поцелуешь меня. 
\restorecr
\end{multicols} % }

\ifcmt
  ig https://187011.selcdn.ru/thumbnails/photos/7/8/f/78fa1b791a4aaa43_1024.jpg
  @wrap center
  @width 0.6
\fi
