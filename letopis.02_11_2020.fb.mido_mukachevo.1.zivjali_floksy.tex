% vim: keymap=russian-jcukenwin
%%beginhead 
 
%%file 02_11_2020.fb.mido_mukachevo.1.zivjali_floksy
%%parent 02_11_2020
 
%%url https://www.facebook.com/mido.mukachevo/posts/3578322692234548
%%author 
%%tags 
%%title 
 
%%endhead 

\subsection{Зів’ялі флокси}
\Purl{https://www.facebook.com/mido.mukachevo/posts/3578322692234548}
\Pauthor{Дочинець, Мирослав}

Був кінець 80-тих, я працював у обласній молодіжній газеті, і був відряджений
записати спогади про видатного художника. Його друг юності, старий
сором’язливий чоловічок, жив на околиці Мукачева, вутла хижка впиралася в
замкову гору. Сад геть здичавів, зате як буйно цвіли, як навісно пахли флокси
– білі, рожеві, блакитні. Ми сиділи на ґанку, пили шипшиновий узвар і
спроквола розмовляли. Розповідав старий скупо, але цікаво. Слова, думки,
епізоди – наче картинні мазки. Мене аж заколисала ця тиха бесіда, а може, від
п’янкого аромату флоксів, що горнулися до колін… Наприкінці я запитав:

«Ви ж разом навчалися малярству в Європі, а чому не стали художником?»

«Я теж дещо малюю, – знічено відповів він, – але … як би це сказати… не те, що треба. Їм».

Я зміряв його протяжним запитальним поглядом. І тоді він підвівся й повів мене
до напівтемної прохолодної кімнати. Зі стін, наче крізь серпанок віків, на
мене суворо глянули очі святих. Аж здригнувся, але ноги самі повели від
полотна до полотна. Це був неймовірно сильний портретний живопис, і зовсім не
іконостасний, не строго канонічний. Так я підійшов до Нього. Ранковий
туманець, дерева, роса на плодах, знайомі флокси, і Він стоїть біля хвіртки –
такий молодий, такий бадьорий, радісний, такий близький. Наче запрошує в сад.
Ісус! Я стояв як укопаний. Довго і мовчки.

«Любиться вам?» – тихо запитав художник, про присутність якого я й забув.

«Дуже», – зізнався я.

Правда, не сказав, що кілька днів тому мене, прости Господи, призначили в
газеті заввідділом агітації, пропаганди та атеїзму.

Потім він провів мене до хвірки, зазираючи в очі якось по-іншому, довірливіше.
А коли прощався, затримав довше мою долоню. Ніколи не забуду його очей. Статтю
я писав з непідробним надихом, бо перед очима стояли ті лики, що загадково
сяяли в напівсутені майстерні-келії. Про них, звісно, я не згадав, бо «їм»,
редакторам, би це не сподобалося.

Стаття «Коли цвітуть флокси», казали, вдалася, її навіть передрукувала
республіканська газета. Згодом у редакції мені передали, що телефонував той
чоловік, дякував і просив завітати до нього. Але газетярщина і молодість несли
мене на своїх шалених хвилях, ніяк не випадало провідати його. Зібрався аж
пізньої осені. Замок Паланок похмуро бовванів згори. Хвіртка в дикий сад була
смішно приторочена цепом до стовпця, висів замок. Сад похнюпився, флокси
зів’яли. Вистомила голову з-за паркану сусідка:

«Помер сердешний. Одійшов тихо, як і жив... І треба ж…За дев’ять днів по
смерти впала хвіртка. Мій син її прив’язав, аби пси не пхалися…»

 Історії не кінець. Десь через півроку мене розшукала статечна дама з шармом
 нетутешності.

«Мій батько дещо заповів вам», – і повела до автомобіля з угорськими номерами.
І дістала з багажника полотно. Я впізнав його одразу – Ісус у вранішньому
тумані. Посеред шумної вулиці, де неслися в різні боки машини, запахло
флоксами…

Мирослав Дочинець. "Різнотрав'я".
