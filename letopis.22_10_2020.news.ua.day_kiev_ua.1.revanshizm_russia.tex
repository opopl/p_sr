% vim: keymap=russian-jcukenwin
%%beginhead 
 
%%file 22_10_2020.news.ua.day_kiev_ua.1.revanshizm_russia
%%parent 22_10_2020
%%url https://day.kyiv.ua/uk/blog/suspilstvo/revanshyzm-yak-rosiyska-nacionalna-ideya
 
%%endhead 

\subsection{Реваншизм як російська національна ідея}

\url{https://day.kyiv.ua/uk/blog/suspilstvo/revanshyzm-yak-rosiyska-nacionalna-ideya}

Ігор Яковенко, 22 жовтня, 2020 - 19:40

«Усіх знищимо!  Одним ударом!  Скільки б їх не було: 10 мільйонів, 100
мільйонів! Столиць їхніх не буде! Ні Києва, ні Брюсселя!» — звично волає в
студії головного державного телеканалу країни «Россия 1» лідер парламентської
фракції.  На обличчях ведучих Скабеєвої і Попова блаженні усмішки. Всі
«експерти» в студії теж щасливі.  Це не вистава, не сатирична програма, а
головне суспільно-політичне ток-шоу, яке всі останні три роки отримувало
телевізійну премію «ТЕФІ» в цій номінації.

Людина, яка постійно і публічно закликає країну, що має ядерну зброю, знищити
десятки і сотні мільйонів європейців і американців, 27 років засідає в
російському парламенті разом з очолюваною ним фракцією таких самих негідників,
за них голосують, востаннє понад 4 мільйонів осіб.  Немає жодних сумнівів, що
на думських виборах 2021 року пан Ж. знову протягне свою партію до Держдуми.
Для цього він не вилазить із федеральних ефірів. Для цього навколо нього
створено атмосферу визнання. «Ось тут я змушений погодитися з Володимиром
Вольфовичем», — таким є постійний приспів «експертів», зокрема «лібералів за
викликом», «українців за викликом», «американців за викликом», а останнім часом
і «білорусів за викликом». Пан Ж. став нормою. «Про що Путін шепоче, про те Ж.
кричить», — ось та формула, яка дозволяє цій карикатурі на Гітлера стати нормою
російської політики.

На заклик Ж. знищити столиці європейських міст єдине заперечення знайшлося у
декана Третьякова, який заперечливо вигукнув: «Київ — наше місто!» На що Ж.
негайно заперечив, що «Київ — отруєне місто, і там ненавидять росіян».

Формальним приводом для чергового нападу керованої істерики став оголошений
Координаційною радою народний ультиматум.  Коли на великому екрані студії «60
хвилин» з’явилось обличчя Світлани Тихановської, реакція «експертів» була
такою, ніби зграю бісів рясно полили свяченою водою.

«Подивіться, яка у неї мерзенна фізіономія!»  — кричав Ж. — Це ж відьма!
Чудовисько!  Стервозна баба!  Нормальна жінка сидить із чоловіком і дітьми, а
вона вештається Європою і зустрічається з главами держав!  Те, що вона
пропонує, це в чистому вигляді склад злочину!  Це як більшовики кричали: геть
самодержавство!  Чим закінчиться?  Країні кінець!»

Давши вичерпну естетичну, моральну та юридичну оцінку лідеру білоруського
протесту, лідер російської парламентської фракції перейшов до конструктивних
пропозицій. «Потрібно спецназ направити, викрасти її з Вільнюса, привезти і
повісити в центрі Мінська!  Всіх заарештувати і вранці повісити!  У Лукашенка
кращий шанс — увійти в Росію шістьма областями!  Вибори не потрібні — він їх
програє!  Забудьте слово «Білорусь»!  Шість областей Росії: Брестська,
Вітебська, Гомельська, Гродненська, Могилевська і Мінська!»

Присутні в студії в цілому були згодні з Ж. в оцінках Тихановської і з його
планом урегулювання ситуації в Білорусі. «Згоден, що Тихановська — відьма, але
ж ці технології працюють!»  — із сумом вигукнув «політолог» Володимир Аватков.
Інший «політолог», Галкін, дорікнув Росії за те, що вона «не працює з
інтелектуальним, експертним і культурним співтовариством на потсрадянському
просторі».

Свій погляд на події в Білорусі виклав декан Вищої школи журналістики МДУ
Третьяков.  «Коли я дивлюся на європейських політиків, я бачу політичних
звірів! Ці мерзотники хочуть знищити Білорусь!»  — повідомив декан і перейшов
до оцінки Світлани Тихановської, в якій він повністю підтримав Ж. «Вона або
цілковита дурепа, або провокаторка! Повністю згоден із Володимиром Вольфовичем,
що її треба вилучити з будь-якої країни і судити в Мінську!»  Ведуча Скабеєва
діловито поцікавилась у Третьякова, чи вважає він, що Тихановську треба
повісити, на що декан тут же відреагував: «Я не пропонував повісити цю
провокаторку, але її заклики — це націонал-зрада, і за це треба судити!» Ж.,
який після виконання своєї чергової арії трохи задрімав, при слові «повісити»
стрепенувся і хрипко заволав: «Тільки одне — вішати!  вішати! і вішати!»

Багато хто із супротивників путінського режиму зневажливо відмахується від
того, що відбувається в російському телевізорі, вважаючи, що всі ці шоу
ненависті дивляться тільки домогосподарки, для яких телевізор звучить фоном.
Одна хвилина реклами після 18.00 на телеканалі «Россия 1» коштує 980 тисяч
рублів.  Програми «60 хвилин» Скабеєвої з Поповим і «Вечір» Соловйова
переповнені не тільки ненавистю, а й рекламою, за ефективністю якої
рекламодавці стежать дуже пильно. Тож уся ця отрута доходить до мільйонів
адресатів і отруює мізки й душі росіян.

Уся величезна машинерія російських інформаційних військ уже котрий рік б’є в
одну точку, вбиваючи в суспільну свідомість реваншизм як головну російську
національну ідею. Головне державне інформагентство Росії, «РИА Новости»,
нещодавно опублікувало «установчу» статтю Петра Акопова під назвою «У Росії
немає вибору на пострадянському просторі».  Ось її основні тези:

«Розвивайте свою Російську Федерацію — і заспокойтеся вже, говорили нам.  І це
було безсовісною брехнею.  Розвивайте свою Російську Федерацію — і заспокойтеся
вже, говорили нам.  І це було безсовісною брехнею.  Адже Російська Федерація не
може розвиватися сама по собі — вона лише частина історичної Росії.  Можна,
звичайно, відмовитися від своєї історії, але тоді немає відповіді на запитання,
чому Україна, тобто те, звідки і почалася Русь, може бути незалежною державою,
а Дагестан або Далекий Схід не можуть. У нинішньої Росії у вигляді Російської
Федерації просто немає вибору: вона може бути або повноцінним продовженням
історичної Росії — СРСР, або не бути взагалі».

«Нам потрібно збирати, відновлювати свій простір, повертатися на стовпову
дорогу російської історії, виправляти історичний вивих», — так формулює
«історичну місію» російського народу автор агентства «РИА Новости», що повністю
узгоджується з тим, про що кричить телевізор і що реально робить останніми
роками Путін. І поки він і його режим при владі, під загрозою перебувають
суверенітет і безпека не лише тих 14 країн, які мали нещастя входити до складу
СРСР.  Путінська інформаційна обслуга дуже любить цитувати фразу свого кумира
про те, що «кордони Росії ніде не закінчуються». Яку країну Путін і його
придворні міфотворці призначать на роль «частини історичної Росії», залежить
тільки від сьогочасної кон’юнктури і співвідношення сил на даний момент у
конкретній точці планети.
