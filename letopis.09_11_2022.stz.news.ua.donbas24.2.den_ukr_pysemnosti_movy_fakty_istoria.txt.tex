% vim: keymap=russian-jcukenwin
%%beginhead 
 
%%file 09_11_2022.stz.news.ua.donbas24.2.den_ukr_pysemnosti_movy_fakty_istoria.txt
%%parent 09_11_2022.stz.news.ua.donbas24.2.den_ukr_pysemnosti_movy_fakty_istoria
 
%%url 
 
%%author_id 
%%date 
 
%%tags 
%%title 
 
%%endhead 

Ольга Демідко (Маріуполь)
Україна,Мова,Українська мова,День української писемності та мови,Свято,date.09_11_2022
09_11_2022.olga_demidko.donbas24.den_ukr_pysemnosti_movy_fakty_istoria

День української писемності та мови — цікаві факти та історія свята

9 листопада в Україні традиційно відзначають День української мови — однієї з
головних цінностей нашої країни

Українська мова має багату історію, адже вона є однією з найдавніших у світі.
Її неодноразово забороняли і намагалися викорінити, а над її носіями вчиняли
геноцид. Наша мова вистояла і наразі визнана однією з найкрасивіших мов у
світі. А в умовах повномасштабної війни День української писемності та мови має
особливе значення, адже Росія не лише вбиває наших людей, знищує міста, але й
зазіхає на головні державні цінності та символи України, які ідентифікують нас
як українців. Саме українська мова є ключовим символом, головним генератором і
генетичним кодом нації.

Читайте також: У Дії з’являться дипломи та атестати: коли можна скористатися

Історія свята

День української писемності та мови було встановлено указом Президента України
Леоніда Кучми від 9 листопада 1997 року «Про День української писемності та
мови» на підтримку «ініціативи громадських організацій та з урахуванням
важливої ролі української мови в консолідації українського суспільства». Дата
була вибрана зовсім не випадково, адже 9 листопада вважається Днем вшанування
пам’яті Преподобного Нестора-літописця. За деякими дослідженнями, саме з його
праць починається українська писемність. Нестор-Літописець був ченцем
Києво-Печерського монастиря, відомим київським літописцем та письменником,
послідовником Кирила та Мефодія, які створили слов’янську писемність. Він є
автором «Повісті минулих літ».

Носіями української мови є 40−45 мільйонів людей, більшість з яких мешкають на
території України. Поширена наша мова також у Молдові, Польщі, Румунії,
Словаччині, Казахстані, Аргентині, Бразилії, Великій Британії, Канаді, США та
інших країнах, де проживають українці. Українська мова є другою або третьою
слов’янською мовою за кількістю мовців і входить до 30 найпоширеніших мов
світу. Після початку повномасштабного вторгнення Росії в Україну кількість
українців, які почали говорити вдома українською, значно виросла. Водночас
жителі нашої країни суттєво знизили споживання російськомовного контенту,
більше 40% опитаних взагалі перестали дивитися російські серіали та слухати
російську музику.

Читайте також: Мовні норми вступають в силу — яких правил треба дотримуватися

Головні традиції святкування

Цього дня в Україні:

пишуть Радіодиктант національної єдності, який започаткований у 2000 році;
покладають квіти до пам’ятника Нестору-літописцю;
відзначають найкращих популяризаторів українського слова;
заохочують видавництва, які випускають літературу українською мовою;
стартує Міжнародний конкурс знавців української мови імені Петра Яцика — конкурс проводиться за підтримки Міністерства освіти та науки України та Ліги українських меценатів. Щорічна кількість учасників — понад 5 млн із 20 країн світу.

Читайте також: Школи, виші та садочки України переходять на дистанційку: деталі

Цікаві факти про українську мову

1. Українська дуже тісно пов’язана зі старослов’янською — спільною мовою
предків всіх сучасних слов’ян, так само як санскрит, є мовою, найближчою до
спільної мови перших індоєвропейців (арійців). Цікаво, що в українській мові є
багато слів, які майже ідентичні словам у санскриті: «повітря», «кохати»,
«кінь», «дерево», «вогонь».

2. Санскритолог, дослідник, філолог-мовознавець, поет В. Кобилюх довів, що
українська мова сформувалася в Х-IV тисячоліттях до нашої ери. Саме тому
походження найважливіших українських слів слід шукати саме в санскриті, а не в
російській, німецькій, турецькій, грецькій та інших мовах, які виникли значно
пізніше.

3. Наприкінці XVI — на початку XIX століть в Україні використовувалася особлива
система письма («козацький скоропис»), де написання деяких букв відрізнялися
від прийнятих у кирилиці.

4. Згідно зі словником Національної Академії Наук України сучасна українська
мова налічує близько 256 тисяч слів і включена до списку мов, які успішно
розвиваються.

5. Відповідно до видання «Короткий словник синонімів української мови», в якому
розроблено 4279 синонімічних рядів, на сьогодні найбільшу кількість синонімів
має слово «бити» — 45 синонімів.

Читайте також: В Україні запускають Google Знання: чому можна навчитися на
освітньому хабі

6. Українську мову протягом різних історичних періодів називали по-різному:
про́ста, руська, русинська, козацька тощо. Історично найуживанішою назвою
української мови до середини XIX ст. була назва «руська мова».

7. Офіційно вважається, що після видання «Енеїди» Івана Котляревського,
українська мова була прирівняна до літературної мови. Українського поета і
драматурга Івана Котляревського по праву вважають основоположником нової
української мови.

8. В українській мові й донині збереглися назви місяців з давньослов’янського
календаря — сiчень (час вирубки лісу), лютий (люті морози), березень (тут існує
кілька тлумачень: починає цвісти береза; брали березовий сік; палили березу на
вугілля), квітень (початок цвітіння берези), травень (зеленіє трава), червень
(червоніють вишні), липень (початок цвітіння липи), серпень (від слова «серп»,
що вказує на час жнив), вересень (цвітіння вересу), жовтень (жовтіє листя),
листопад (опадає листя з дерев), грудень (від слова «груда» — мерзла колія на
дорозі).

9. В українській мові найбільша кількість слів починається на літеру «П», а
найменш уживаною літерою українського алфавіту є літера «Ф».

10. Головною особливістю української мови є те, що вона багата на зменшувальні
форми. Зменшувально-пестливу форму має, як не дивно, навіть слово «вороги» —
«вороженьки».

Текст підготовлено за наступними джерелами: zno.ua, zn.ua, vsviti.com.ua,
glavcom.ua.

Раніше Донбас24 розповідав, про особливості всеукраїнського Радіодиктанту
національної єдності у 2022 році.

Ще більше новин та найактуальніша інформація про Донецьку та Луганську області
в нашому телеграм-каналі Донбас24.

ФОТО: з відкритих джерел.
