% vim: keymap=russian-jcukenwin
%%beginhead 
 
%%file 18_02_2022.stz.news.ua.den.1.nastup_bude
%%parent 18_02_2022
 
%%url https://day.kyiv.ua/uk/blog/polityka/nastup-bude
 
%%author_id ljubka_andrii,news.ua.den
%%date 
 
%%tags rossia,ugroza,ukraina,vtorzhenie
%%title Наступ буде
 
%%endhead 
 
\subsection{Наступ буде}
\label{sec:18_02_2022.stz.news.ua.den.1.nastup_bude}
 
\Purl{https://day.kyiv.ua/uk/blog/polityka/nastup-bude}
\ifcmt
 author_begin
   author_id ljubka_andrii,news.ua.den
 author_end
\fi

16 лютого, яке наші західні союзники називали днем імовірного вторгнення Росії,
нова фаза агресії не розпочалася. Дехто навіть зітхнув з полегшенням – мовляв,
пронесло. Я не знаю, чи наступу в цей день вдалося уникнути завдяки
геополітичному таланту Байдена і твердій позиції Заходу; невідомо, чи справді в
Кремлі планували почати велику війну саме в цей день. Зрештою, це не має аж
такого принципового значення, бо насправді нас аж ніяк не пронесло – наступ
буде.

Так, ви правильно прочитали – наступ буде. 21 лютого, 17 березня чи 24 липня –
ці дати я вибрав навмання. Бо нам усім потрібно усвідомити, що Росія готова
напасти на нас будь-якого дня, кожного дня, щохвилини. Поки в Кремлі панує
доктрина російського імперіаліазму й відбудови СРСР – ми будемо жити під
постійною загрозою! І зауважте: не до кінця перебування Путіна у владі, бо
прізвище президента не є вирішальним, а саме до кінця панування ідеї
російського імперіалізму. Влада може змінитися, але навіть опозиціонери там
заражені вірусом імперськості – і теж прагнуть задушити Україну в «братських»
обіймах.

Моя теза проста: Росія – це ворог України, і війна триває на всіх фронтах
щодня. Так, є більш чи менш інтенсивні фази цієї війни, є показники
концентрування військ біля кордонів, але насправді війна – це не лише танки і
зброя. Точніше, зброя буває різною, і під час війни проти нас Росія
використовує весь арсенал своїх інструментів впливу. Від газового тиску,
пропаганди, дискредитації в світі – до армії на Донбасі, від дестабілізації
внутрішньої ситуації і підкупу українських політиків – до можливості
застосувати проти України ядерну зброю.

Якщо широкого наступу не відбулося, то це не означає, що плани підкорити
Україну скасовано. Просто в цей день порахували, що вигідніше скористатися
іншими засобами, а війська можна буде застосувати пізніше. Кремль вичікує
слушного моменту, щоб завдати якомога болючішого удару: це може бути внутрішня
криза чи політичний розкол в Україні, тимчасове послаблення монолітності
Заходу, економічна криза, стихійне лихо, та будь-що – звір застиг перед
стрибком.

Тому нам треба навчитися жити з відчуттям, що наступ буде. Інформаційний
наступ, кібер-атака, провокації, терористичні акти, спроби змінити уряд на
проросійський, зрештою, масований військовий наступ і бомбардування Києва – все
це може початися кожної миті, адже зброя в агресора напоготові. Російська
імперія не може існувати без України – і ми мусимо бути пильними аж до моменту
краху Росії.

Тому треба готуватися й нарощувати м’язи. Це не означає, що вільні гроші мають
спрямовуватися виключно на армію. Бо Україна не може собі дозволити щорічно
віддавати такий колосальний відсоток ВВП тільки на оборону – нам потрібні нові
лікарні, дороги, адекватні зарплати працівникам бюджетної сфери. Україна мусить
вкладати гроші в оборону, а не тільки в Міністерство оборони; а наша оборона –
це добра якісна освіта, це наукові розробки, бо варто не купувати, а самим
виготовляти зброю, це реформа держуправління і успіхи в боротьбі з корупцією.
Саме це робить нас сильними і є довгостроковою інвестицією у власну безпеку.
