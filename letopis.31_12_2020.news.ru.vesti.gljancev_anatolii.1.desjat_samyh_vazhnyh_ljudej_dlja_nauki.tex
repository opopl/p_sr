% vim: keymap=russian-jcukenwin
%%beginhead 
 
%%file 31_12_2020.news.ru.vesti.gljancev_anatolii.1.desjat_samyh_vazhnyh_ljudej_dlja_nauki
%%parent 31_12_2020
 
%%url https://www.vesti.ru/nauka/article/2505956
 
%%author 
%%author_id gljancev_anatolii
%%author_url 
 
%%tags 
%%title Десять самых важных для науки людей в 2020 году: версия журнала Nature
 
%%endhead 
 
\subsection{Десять самых важных для науки людей в 2020 году: версия журнала Nature}
\label{sec:31_12_2020.news.ru.vesti.gljancev_anatolii.1.desjat_samyh_vazhnyh_ljudej_dlja_nauki}
\Purl{https://www.vesti.ru/nauka/article/2505956}
\ifcmt
	author_begin
   author_id gljancev_anatolii
	author_end
\fi

Редакция самого престижного в мире научного журнала Nature составила список из
десяти людей, оказавших самое большое влияние на науку в 2020 году. Одни попали
в эту десятку благодаря научным работам, другие – из-за своего таланта к
руководству, третьи стали примером личного мужества.

\subsubsection{Тедрос Гебрейесус: управлять ВОЗ в пандемию}

\ifcmt
  pic https://cdn-st3.rtr-vesti.ru/vh/pictures/xw/309/003/0.jpg
	caption Тедрос Адханом Гебрейесус возглавляет ВОЗ в трудное время борьбы с пандемией COVID-19.
  width 0.4
\fi

Открывает список генеральный директор Всемирной организации здравоохранения
(ВОЗ) Тедрос Адханом Гебрейесус (Tedros Adhanom Ghebreyesus).

Он возглавил организацию в 2017 году. Когда в 2018 году в Демократической
республике Конго произошла вспышка геморрагической лихорадки Эбола, Гебрейесус
несколько раз лично посещал эту страну. Это заболевание смертельно опасно и
крайне заразно, так что руководитель рисковал жизнью, направляясь в очаг
эпидемии. Под руководством Гебрейесуса ВОЗ организовала помощь местным медикам.
Тогда врачи не допустили широкого распространения смертельно опасного вируса.

Но это была лишь репетиция, настоящим испытанием для организации стала пандемия
COVID-19. Напомним, что первые новости о появлении нового вируса начали
появляться в конце декабря 2019 года. 31 января 2020 года Гебрейесус от лица
ВОЗ объявил чрезвычайную ситуацию в области общественного здравоохранения. В
марте он признал происходящее пандемией, то есть сильнейшей эпидемией, вышедшей
за пределы отдельных стран и континентов.

ВОЗ под руководством Гебрейесуса вносит огромный вклад в борьбу с инфекцией.
Организация вырабатывает рекомендации для правительств и населения,
координирует испытания новых препаратов и принимает другие меры. На фоне
захлестнувшей мир лавины противоречивых и недостоверных сообщений организация
остаётся источником проверенных сведений о коронавирусе. Сведения постоянно
обновляются, так как копилка знаний человечества пополняется буквально каждый
день, но ВОЗ в этом бушующем океане информации остаётся столпом надёжности и
достоверности.

Nature отмечает, что Гебрейесус проявляет завидную выдержку даже в политически
напряжённых ситуациях. Хороший тому пример – его спокойная реакция на угрозы
президента США Дональда Трампа прекратить финансирование ВОЗ. Пандемия – не
время для конфликтов, демонстрирует своим поведением руководитель ВОЗ. В это
трудное время всем важно сотрудничать друг с другом ради борьбы с общим врагом
– вирусом.

\subsubsection{Верена Мохопт: безопасность на краю света}


\ifcmt
  pic https://cdn-st4.rtr-vesti.ru/vh/pictures/xw/309/003/1.jpg
	caption Верена Мохопт оберегала полярников от многочисленных опасностей экспедиции.
  width 0.4
\fi

Верена Мохопт (Verena Mohaupt) была координатором логистических операций
Междисциплинарной дрейфующей обсерватории по изучению арктического климата
(MOSAiC \Furl{https://psl.noaa.gov/mosaic}).

Эта экспедиция началась в сентябре 2019 года и закончилась в октябре 2020 года.
Около трёхсот полярников работали в бесконечной тьме и холоде полярной ночи. Её
сменил полярный день, и лёд начал таять под солнечными лучами.

В задачи Мохопт и её подчинённых входило обеспечение безопасности всех
участников экспедиции. Это было непросто. Люди рисковали получить обморожение,
стать жертвой белых медведей, угодить в полынью или заработать нервный срыв
из-за многих месяцев работы в экстремальных условиях и изоляции от остального
мира. Полярникам угрожали штормы и подвижки льда.

Ещё до начала экспедиции Мохопт подготовила своих подопечных ко всем возможным
неожиданностям. Например, люди тренировались выбираться из холодной воды с
помощью одних только ледорубов. Во время работы руководитель постоянно следила
за безопасностью людей. Иногда ей приходилось принимать решение, когда белый
медведь стоял в нескольких шагах и нюхал воздух.

Блестящая работа Мохопт и её коллег помогла участникам экспедиции сделать свою
работу и благополучно вернуться домой.


\subsubsection{Гонсало Мораторио: выявить врага}

\ifcmt
  pic https://cdn-st2.rtr-vesti.ru/vh/pictures/xw/309/001/3.jpg
	caption 3 Гонсало Мораторио Усилия Гонсало Мораторио позволили предотвратить широкое распространение COVID-19 в Уругвае.
  width 0.4
\fi

Уругвайский вирусолог Гонсало Мораторио (Gonzalo Moratorio) с недавних пор стал
знаменитостью в своей стране. Люди подходят к нему на улицах и благодарят за
то, что он сделал для борьбы с коронавирусом SARS-CoV-2.

В 2020 году Мораторио впервые возглавил собственную лабораторию. И тут началась
пандемия. Вирусолог одним из первых среди своих коллег осознал всю серьёзность
ситуации. Уже к марту 2020 года он и его сотрудники разработали собственный
тест на коронавирус. Это позволило Уругваю не зависеть от закупок тестов за
рубежом, цены на которые были очень высокими.

Команда Мораторио предложила программу масштабного тестирования населения. В
сотрудничестве с Министерством здравоохранения Уругвая вирусологи создали в
стране сеть диагностических лабораторий и обучили их персонал.

К концу мая в Уругвае проводилось уже 800 тестирований в день. Сегодня это
число составляет около 5000 тестов в день, из которых около 30\% выполняются с
помощью разработки Мораторио. Результат налицо: по состоянию на 10 декабря 2020
года в стране с населением 3,5 миллиона человек от COVID-19 умерли всего 87
человек.

\subsubsection{Ади Утарини: повелительница хороших комаров}

\ifcmt
  pic https://cdn-st2.rtr-vesti.ru/vh/pictures/xw/309/003/3.jpg
	caption Команда Ади Утарини добилась важного успеха в защите населения от лихорадки денге.
  width 0.4
\fi

Индонезийская исследовательница Ади Утарини (Adi Utarini) и её команда в 2020
году одержали большую победу над лихорадкой денге. От этой болезни, переносимой
комарами, страдает до 400 миллионов человек в год.

Чтобы остановить передачу вируса, биологи выпускали в окружающую среду комаров,
заражённых безопасной для человека бактерией вольбахией (Вести.Ru подробно
рассказывали о том, как работает эта технология). Эксперимент проводился в
крупном индонезийском городе – Джокьякарте. В итоге биологи уменьшили число
случаев лихорадки денге в "подшефных" районах города на 77\%.

Отдельной задачей было договориться с властями, а также заручиться поддержкой
местных жителей. Люди, не понимающие сути нового метода, могли воспротивиться
идее выпускать вблизи их жилья целые тучи комаров. В деле связей с
общественностью Утарини проявила недюжинные таланты. Она использовала
объявления в СМИ, граффити, личные встречи и даже конкурс короткометражных
фильмов.

Беспочвенные опасения были развеяны столь успешно, что жители города часто сами
просили учёных провести подобную работу в их районе. Теперь "хорошие" комары
выпускаются по всему городу.


\subsubsection{Кэтрин Янсен: вакцинировать человечество}


\ifcmt
  pic https://cdn-st3.rtr-vesti.ru/vh/pictures/xw/309/003/4.jpg
	caption Кэтрин Янсен руководила испытаниями вакцины от COVID-19, сегодня одобренной к применению в нескольких странах.
  width 0.4
\fi

Кэтрин Янсен (Kathrin Jansen), сегодня работающая в компании Pfizer – один из
признанных авторитетов в деле разработки вакцин. Именно под её руководством был
разработан знаменитый "Гардасил" – первая в мире вакцина против вируса
папилломы человека, заражение которым может привести к раку. Также команда
Янсен улучшила вакцину "Превенар-13", вдвое увеличив количество штаммов
менингококка, от которых она защищает.

Сегодня "Превенар-13" занимает первое, а "Гардасил" – второе место в списке
самых продаваемых вакцин мира. Но вакцина от COVID-19 от Pfizer компаний
BioNTech, испытанная под руководством Янсен, имеет все шансы обойти оба
препарата.

Испытания этой разработки шли в трудных условиях пандемии. Однако они уложились
в рекордно сжатые сроки: всего 210 дней.

Новинка стал первой в мире вакциной от какого-либо патогена, созданной на
основе матричной РНК и при этом разрешённой к применению на людях. Используя
недостаточно "обкатанную" технологию, Янсен сыграла ва-банк и выиграла. В ходе
клинических испытаний вакцина продемонстрировала эффективность более 90\% и уже
одобрена в нескольких странах, включая США и Великобританию.

\subsubsection{Чжан Юнчжэнь: начало борьбы}

\ifcmt
  pic https://cdn-st4.rtr-vesti.ru/vh/pictures/xw/309/003/5.jpg
	caption Чжан Юнчжэнь и его команда первыми опубликовали в Интернете геном коронавируса SARS-CoV-2, сделав его доступным учёным всего мира.
  width 0.4
\fi

Разработка тестов на коронавирус SARS-CoV-2 и вакцин против него была бы
невозможна без расшифровки генома этого вируса. Поэтому в список Nature вошёл
китайский учёный Чжан Юнчжэнь (Zhang Yongzhen). Именно он первым опубликовал
геном патогена в Интернете, сделав его доступным для вирусологов всего мира.

Исследователь и его команда опубликовали геном SARS-CoV-2 уже 11 января 2020
года, вскоре после начала вспышки в Ухане. Чжан Юнчжэнь утверждает, что не знал
о распоряжении властей Китая от 3 января, запрещающем публиковать информацию о
новом вирусе, но догадывался, что некоторые представители власти могли быть
этим недовольны.

Некоторые СМИ сообщали, что Юнчжэнь и его сотрудники были наказаны за свой
смелый и столь важный для человечества шаг, хотя сам исследователь опровергает
это.

\subsubsection{Чанда Прескод-Вайнштейн: тёмная материя и права чернокожих}


\ifcmt
  pic https://cdn-st2.rtr-vesti.ru/vh/pictures/xw/309/007/7.jpg
	caption Чанда Прескод-Вайнштейн совмещает работу в теоретической астрофизике с борьбой за права чернокожих.
  width 0.4
\fi

Чанда Прескод-Вайнштейн (Chanda Prescod-Weinstein) – астрофизик-теоретик. Она
занимается тёмной материей и другими загадками Вселенной. Прескод-Вайнштейн
работает на факультете физики и астрономии Университета Нью-Гэмпшира. Это
делает её, вероятно, первой в США чернокожей женщиной, имеющей постоянную
научную должность в области теоретической космологии или теории элементарных
частиц.

Прескод-Вайнштейн вспоминает, что часто оказывалась единственным чернокожим
физиком в комнате. Это побудило её бороться с расизмом в науке и в обществе в
целом.

В начале июня 2020 года она вместе с другими учёными организовала "Забастовку
за жизнь чернокожих". Толчком к этому послужили несколько нашумевших трагедий,
в частности, убийства Джорджа Флойда и Ахмауда Арбери.

К движению присоединились самые авторитетные научные организации США,
насчитывающие сотни тысяч членов. Среди них Американский геофизический союз,
Американское физическое общество, Американское химическое общество и
Американская ассоциация развития науки, издающая Science – один из самых
престижных в мире научных журналов.

Прескод-Вайнштейн и её соратники призывают научные учреждения принять меры по
искоренению расизма, который всё ещё встречается в академической среде самых
разных стран.

\subsubsection{Ли Ланьцзюань: оправданная жёсткость}

\ifcmt
  pic https://cdn-st2.rtr-vesti.ru/vh/pictures/xw/309/009/7.jpg
	caption Ли Ланьцзюань вовремя настояла на введении жёсткого карантина в Ухане, предотвратив быстрое распространение коронавируса по Китаю.
  width 0.4
\fi

Ли Ланьцзюань (Li Lanjuan) – китайский врач и эксперт в области
здравоохранения. Именно она порекомендовала властям организовать строгий
карантин в Ухане, когда там распространился коронавирус.

Уже 23 января 2020 года въезд и выезд из города был запрещён, движение
общественного транспорта остановлено. Жителей обязали оставаться в своих домах.

Семьдесят шесть дней строгого карантина в городе с населением почти 12
миллионов человек. Такая мера, вероятно, не имела прецедентов в истории. Но,
как показали дальнейшие события, она безусловно была оправданной. Карантин в
Ухане предотвратил молниеносное распространение вируса по всему Китаю, что
повлекло бы огромное количество жертв. Кроме того, благодаря этому шагу, другие
страны получили возможность сколько-нибудь подготовиться к появлению первых
заболевших.

К слову, сама Ли Ланьцзюань осталась в заблокированном Ухане и стала одним из
символов борьбы с инфекцией. Местные жители с уважением называли 73-летнего
эпидемиолога "бабушкой Ли".

\subsubsection{Джасинда Ардерн: здесь не место для вируса}


\ifcmt
  pic https://cdn-st4.rtr-vesti.ru/vh/pictures/xw/309/006/3.jpg
	caption Премьер-министр Новой Зеландии Джасинда Ардерн приняла решения, защитившие страну от коронавируса.
  width 0.4
\fi

Премьер-министр Новой Зеландии Джасинда Ардерн (Jacinda Ardern) вовремя приняла
меры, предотвратившие распространение коронавируса в этом государстве.

К 14 марта 2020 года только шесть новозеландцев имело положительный тест на
коронавирус. Тем не менее решения Ардерн были достаточно жёсткими. Она закрыла
порты для круизных судов (и это в стране, для которой туризм – одна из основных
статей дохода), ввела двухнедельный карантин для всех прибывающих в страну и
так далее.

В дальнейшем премьер-министр тоже действовала решительно. На борьбу с
коронавирусом Новая Зеландия направила 20\% своего ВВП. Для большинства
государств это немыслимая цифра.

Эти меры оправдали себя. В стране с населением почти в пять миллионов человек
от COVID-19 умерли всего 25 человек. Для сравнения: в США смертность от этого
заболевания на тысячу человек более чем в 170 (!) раз больше.

\subsubsection{Энтони Фаучи: борьба с пандемией и политикой}


\ifcmt
  pic https://cdn-st2.rtr-vesti.ru/vh/pictures/xw/309/006/5.jpg
	caption Энтони Фаучи выступает за ужесточение мер против распространения COVID-19 в США.
	width 0.6
\fi

Энтони Фаучи (Anthony Fauci) – директор Национального института изучения
аллергических и инфекционных заболеваний США. Этот всемирно известный
инфекционист отстаивал свою точку зрения на борьбу с COVID-19, даже когда
недоброжелатели угрожали его жизни.

На своём посту эксперт консультировал шестерых президентов США по вопросам
борьбы с инфекционными заболеваниями, включая ВИЧ. Когда разразилась пандемия
COVID-19, Фаучи было уже 79 лет. Тем не менее он работал без выходных, не
только вырабатывая управленческие решения, но и лично занимаясь лечением
пациентов.

Фаучи выступал и выступает за ужесточение мер против распространения
коронавируса. В этом его взгляды резко разошлись с позицией президента США
Дональда Трампа. Последний стремился избежать решений, наносящих ущерб
американской экономике.

Трамп пригрозил Фаучи увольнением. Некоторые недоброжелатели учёного даже стали
угрожать его жизни, так что властям пришлось обеспечить эксперта охраной. Тем
не менее специалист продолжает настаивать, что ради спасения человеческих
жизней нужно поступиться интересами экономики.

Таким был для науки 2020 год в лицах по версии журнала Nature. Из десяти
названных персон только три попали в список не в связи с пандемией COVID-19.
Это и понятно: она, бесспорно, стала самым заметным событием уходящего года.

Тем не менее жизнь науки в 2020 году не ограничивалась борьбой с вирусом. Ранее
Вести.Ru рассказывали о десяти научных прорывах года по версии журнала Science \Furl{https://smotrim.ru/article/2505557}
и о том, чем нас в уходящем году порадовала российская наука.\Furl{https://smotrim.ru/article/2503126} Писали мы и о
самых странных исследованиях 2020 года.\Furl{https://smotrim.ru/article/2504492}


