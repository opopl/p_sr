% vim: keymap=russian-jcukenwin
%%beginhead 
 
%%file slova.ukraina
%%parent slova
 
%%url 
 
%%author 
%%author_id 
%%author_url 
 
%%tags 
%%title 
 
%%endhead 

Так и живем в \emph{Украине}: удочек нет у власти для простых \emph{украинцев},
и рыбы нет для страждущих, \textbf{Еще никому из президентов не удалось объединить
Украину}, Александр Гончаров, strana.ua, 24.05.2021

Євробачення 2021: \emph{Україна} посіла п'яте місце, beztabu.net, 23.05.2021

\emph{Украину} нужно лишить права оценивать выступления участников от
страны-агрессора: это аморально, Игорь Кондратюк, obozrevatel.com, 23.05.2021

Уберите российский сапог и \emph{украинская} культура побьет все мировые
рекорды, Елена Кудренко, obozrevatel.com, 23.05.2021

\emph{Украинские} туристы подрались в самолете: мама младенца озвучила свою
версию конфликта, Марина Петик, obozrevatel.com, 24.05.2021

Не просто \emph{Україна}, а \emph{Україна}-Русь. Як Михайло Грушевський захищає
державні кордони?  radiosvoboda.org, 23.05.2021

Навіть попри те, що Путін не почав нову фазу завоювань в \emph{Україні} –
останній епізод кілька тижнів тому змусив людей у Вашингтоні і Києві
понервувати,
radiosvoboda.org, 24.05.2021

«Дайте нам грошей!»: большая \emph{украинская} мечта о великой халяве, from-ua.com, 24.05.2021

«Евровидение» в Европрессе: Италия — победитель, Германия и Британия  «пасут
задних», а \emph{Украину} назвали «кошмарной танцевальной угрозой», sharij.net,
24.05.2021

Зато выступление GoA от \emph{Украины} с песней «Шум» назвали одним из лучших
выступлений вечера,
sharij.net, 24.05.2021

Паралимпийская сборная \emph{Украины} заняла первое место на чемпионате Европы,
завоевав 37 золотых медалей, sharij.net, 23.05.2021

Кладбище империй: Чему научила США и \emph{Украину} война в Афганистане,
sharij.net, 24.05.2021

Для \emph{Украины}, где миллионы людей с разных сторон, Холодная война долго не сможет быть холодной,
\textbf{Дело Протасевича - еще один повод раздуть Холодную войну}, Денис Жарких, strana.ua, 24.05.2021

Вот я читаю посты некоторых российских патриотов, и не нахожу особой разницы
между ними и патриотами \emph{украинскими},
\textbf{Дело Протасевича - еще один повод раздуть Холодную войну}, Денис Жарких, strana.ua, 24.05.2021

Тобто. вже рік тому можна було бачити, що Північний потік 2 для \emph{України}
- програна справа, \textbf{Наш МИД живет в каком-то выдуманном мире}, Владимир
Воля, strana.ua, 24.05.2021

У келіях Печерського монастиря запалав світильник \emph{української} культури. Саме
тут бере свій першопочаток давня \emph{українська} література, художнє мистецтво,
медицина. Нестор-літописець – перший історик \emph{України}-Руси, автор Повісті, що є
головним джерелом вивчення \emph{української} історії, Агапій – перший відомий лікар,
Аліпій – перший живописець...
radiosvoboda.org, 06.05.2017

Шановні туристи, ми не заперечуємо вагомості внеску цього київського духовного
центру до скарбниці всіх східнослов'янських культур, але наголошуємо, що Лавра
– це передусім феномен \emph{української} культури».
radiosvoboda.org, 06.05.2017

Відвідувачі можуть запитати, чому у тексті Повісті Нестора Літописця (мова
твору церковнослов'янська) древня обитель іменується лише Печерською. Один із
найкращих знавців давньоруських літописів професор Василь Яременко стверджує –
це лише доказ того, що розмовною мовою автора була \emph{українська}. Саме тому
«\emph{українська} лексика ллється суцільним потоком у Повісті: жито, колодязь,
подружжя, туга, печера…» – аргументує професор.
radiosvoboda.org, 06.05.2017

\emph{Украина} была одной из главных составных частей Советского Союза — чрезвычайно
развитой в экономическом отношении, многолюдной и комфортной республикой.
\emph{Украинцы} по сути являлись второй имперской нацией, а выходцы из республики со
своими многочисленными группами поддержки порою оказывались на самых вершинах
власти, \textbf{Надоела Украина? Ежедневное «иди и смотри»}, Константин Кеворкян, ukraina.ru, 24.05.2021

Скоро в \emph{Украину} ворвется лето: синоптики заявили о \enquote{шикарной} погоде и сделали детальный прогноз, obozrevatel.com,
25.05.2021

На реках и водоемах с начала года утонуло более 300 \emph{украинцев}: причины жуткой
статистики, obozrevatel.com, 25.05.2021

\emph{Україна} і ЄС готують рішення про припинення польотів до і над Білоруссю.
Якими можуть бути інші санкції? radiosvoboda.org, 25.05.2021

«Це незграбно і підло. Винуватці зриву спецоперації \emph{українських}
спецслужб мають бути встановлені і покарані. А спроби їх «відмазати» марні», –
написав у твітері колишній міністр, radiosvoboda.org, 25.05.2021

\emph{Украина} может стать главным карателем Белоруссии, vz.ru, 25.05.2021

Сама \emph{Украина} попадет в патовую ситуацию, ведь для нее полеты над Россией
уже и так закрыты, а теперь закроется небо еще и над Белоруссией. Облетать
\emph{украинцам} придется много и долго, vz.ru, 25.05.2021

Важной является позиция \emph{Украины}, которая очень часто действует в русле западных политиков,
vz.ru, 25.05.2021

Уж если борщ не \emph{украинский}, Царит не наша правота – Расколот мир! В нём нет
единства, Повсюду мрак и темнота, odnarodyna.org, 25.05.2021

Бойцы, хранитель святыни, Над нами реет вечный стяг!  Неспевший славу \emph{Украине}
Державе нашей – лютый враг, odnarodyna.org, 25.05.2021

Эй, Макаревич, дорогуша, Ты лоб, пожалуйста, не морщь!  Ты \emph{украинцам} плюнул в
душу, А заодно попал нам в борщ!
odnarodyna.org, 25.05.2021

Но независимой державой \emph{Украйне} быть уже пора: И знамя вольности
кровавой, Я подымаю на Петра, Александр Пушкин, Полтава

Під впливом «Історії Русів» писали свої твори на \emph{українську} тематику
Кіндрат Рилєєв, Микола Гоголь, Тарас Шевченко. Використовував «Історію Русів» і
Пушкін, пишучи «Полтаву». Саме з цього \emph{українського} твору з'явилися і в
творі Пушкіна мотиви героїчної \emph{української} історії, мотиви боротьби
\emph{України} за незалежність, radiosvoboda.org, 25.05.2021

Храня суровость обычайну, Спокойно ведал он \emph{Украйну}, Молве, казалось, не
внимал, И равнодушно пировал, Александр Пушкин, Полтава

На вашому YouTube каналі одне із рекордсменів за переглядами – відео
\enquote{Чи матюкались давні \emph{українці}?}. Давайте розставимо крапки над і
щодо лайок в \emph{українській} мові. Чи винен в цьому умовний колективний
путін?  pravda.com.ua, 25.05.2021

Остап \emph{Українець}: Твоя індивідуальна \emph{українська} мова не має бути такою, як
вимагає неіснуючий \enquote{золотий стандарт}, pravda.com.ua, 25.05.2021

Часом ви дозволяєте собі говорити страшні речі. Про те, що \emph{українська}
мова не наймилозвучніша в світі, що вона не походить безпосередньо від
фінікійського письма, що лайки прийшли в неї не виключно як запозичення з
російської, а \emph{українською} не лише з мальвами розмовляють. Чому ви досі
не в списку ворогів нації? pravda.com.ua, 25.05.2021

Якщо розбирати, яке слово де-факто прийшло з \emph{української}, а яке
запозичене, вийдуть дуже прикольні речі. Вийде, що слово \enquote{щур} в
\emph{українській} мові скоріш за все з російської, а російське слово
\enquote{крыса} скоріш за все з \emph{української}, просто свого часу ми ними
помінялися, pravda.com.ua, 25.05.2021

Верховный суд \emph{Украины} принял решение о том, что в Ровенской области настоятели
двух захваченных храмов УПЦ останутся проживать вместе со своими семьями в
церковных домах, fraza.com, 25.05.2021

Сладкие парочки: Россия и \emph{Украина}, mikle1.livejournal.com, 21.11.2009

Даже я понял, что он лично виноват в том, что Россия до сих пор не развалилась
на демократические Чечни и Запопинские республики Окрайны, Китайны и прочая.
Еще он виноват в \emph{украинском} голодоморе 1932 года, монголо-татарском
нашествии и проигрыше Мазепы в 1709 году. Змею, укусившую вещего Олега тоже
тренировал он, mikle1.livejournal.com, 21.11.2009

Такие вот культы личности с обратными знаками. 2 страны, 2 личности, 2 системы
ценностей и 2 результата. Все смелые, прогрессивные, правильные, модные,
продвинутые, революционные и прочая говорят, что лучше \emph{украинский}
вариант, mikle1.livejournal.com, 21.11.2009

Не возвращайся в \emph{Украину}!  Угроза жизни дышит в спину, В затылок дышит
подлецо – Фашизма подлое лицо! Юнна Мориц, www.owl.ru

Василь Шкляр: «\emph{Українською} заговорять навіть в Африці, якщо без неї неможливо
буде повноцінно жити», globlvillage.com, 15.07.2020

Тим часом, \emph{Україна} вже не вперше у своїй історії опинилася сам на сам з
найкровожерливішим агресором. Соромно бачити, як плазує перед Росією Франція,
як стримано поводиться Німеччина, наче вони досі не оговталися від давніх
поразок у війнах з цим монстром, Василь Шкляр, globlvillage.com, 15.07.2020

Зеленский уже и не скрывает, что он является марионеткой в руках иностранцев и
все делает в их интересах, а не интересах \emph{Украины} и \emph{украинцев}.
Вот честно и открыто рассказывают как будут продавать землю иностранцам о чем
ранее молчал, Ростислав Кравец, strana.ua, 25.05.2021

\enquote{Руслан и Людмила} в вольном пересказе первого премьер-министр
\emph{Украины} Витольда Фокина. Цікаво. Чудово, www.skadtalk.cc, 21.06.2020

\enquote{Господи милосердний, \emph{Україна} – це не Раша}, Ірина Фаріон,
obozrevatel.com, 16.06.2019
