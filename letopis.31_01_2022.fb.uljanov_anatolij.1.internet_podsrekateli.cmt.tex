% vim: keymap=russian-jcukenwin
%%beginhead 
 
%%file 31_01_2022.fb.uljanov_anatolij.1.internet_podsrekateli.cmt
%%parent 31_01_2022.fb.uljanov_anatolij.1.internet_podsrekateli
 
%%url 
 
%%author_id 
%%date 
 
%%tags 
%%title 
 
%%endhead 
\zzSecCmt

\begin{itemize} % {
\iusr{Владимир Ищенко}

Спасибо! За последние годы было мало поводов почувствовать гордость за
причастность к Украине, но и этот они пытаются испоганить.

\iusr{Вячеслав Чеш}

Ох, мені подобається як чувак з Лос-Анджелеса розказує як тут ми, кредитні
хом'яки, в 37 км від роскордону будемо в гастропабах сидіти. Да, звісно, це ж
мабуть тобі виїздити пакуючи родину та морських свинок з міста, що в 37 км від
кордону, якщо твої братушки повторять 2014-й. А. От пам'ятаю, був у нас в 2018
вишкіл, ну знаєш, яким кінцем, кредитному хом'яку з АК стріляти. І от пара з
наших потім до ЗСУ пішла. Я з одним потім спілкувався. Більше мене заробляє. Та
й електриком, мабуть, теж більше. І ще в колі друзів ветеран є, на js кодує. І
ти не повіриш, теж більше. Я вчитель якщо що. Давай розказуй про мєнєєімущіх.
Толік, іди в ЛА попротестуй, хулі ти людям з зп в 420 баксів розказуєш як їм
батьківщину любити. Знавець, сука, гастропабів.

\iusr{Алексей Блюминов}

я кстати не думаю что Зеленский одумался и что з этим стоит какой то разворот в
политике. Просто Клоун испугался.

\iusr{Владимир Михайловский}
\textbf{Алексей Блюминов} Так оно и есть.

\iusr{Владимир Могилевский}

Он случайно. Если только попробуют вякать не по тренду, ему и этой стране
конец. Сразу узнаете что все не так и еще много интересного.

Несмотря на претензию на какой-то особый путь, Украина легко может повторить
эпизоды из недавней истории России. В какой-то момент, Ельцин сказал, что у нас
"нет национальных интересов, только общемировые (американские)", а тогдашний
министр обороны руководил армией из Майями. Когда период выживания закончился,
национальная буржуазия окрепла, осознала, что у нее есть свои интересы и
началась какая-то самодеятельность, вот тут начались проблемы.

Дружелюбное бормотание \enquote{водка-горбачев-перестройка} сменилась злобным враньем,
подставами и провокациями.

В эпоху когда общественное мнение моделируется в соцсетях и можно изменить
сознание целых стран за год-два, шансов на какую-то самостоятельность нет. Не
даром, первое что сделали \enquote{демократические} лидеры майдана, это вырубили
яндекс, вконтакте и начали войну.

\iusr{Юрий Лукшиц}
Эх. Чем дальше, тем страннее.

\iusr{Dmytro Stezhko}

Так на самом деле да, военная истерика раздутая на пустом месте, может быть
вполне результатом договорняка Байдена и Путина, чтобы заставить Украину
выполнить Минск-2 который просто разорвет страну и Зеленского превратив Украину
в Боснию 2.0

\iusr{Olex Butryk}
Гарний пост

\iusr{Dmitriy Krasnov}

Здесь Анатолий в самую точку. Россия - однозначно стратегический враг и так
далее, но лично у меня впервые за постмайданный период возникло желание, чтобы
Украина послала нахуй всех западных партнёров, не как подыгрывание России, а
просто потому что заслужили, заебали. Имею опыт общение с иностранцам
ежедневно, слышу от них о том, что здесь уже чуть ли не война, и, что они,
разумеется, сюда не хотят лететь, вот-вот планировали, но их посольства
советуют им от этого отказаться. Это выбешивает, но такова реальность. Будто
психологичное насилие. Я скорее хейтер Зеленского, но здесь солидарен с ним.
Единственное, Анатолий, меня как твоего давнего почитателя, удивляет твой
нарратив в стиле «не нужно злить Россию», мол вот были бы всегда внеблочные, не
злили бы Россию, и было бы счастье. Когда тебе возражают, ты аргументируешь
а-ля «после майдана Украина начала активно смотреть в сторону НАТО». Напомню,
до майдана Украина и вовсе была под полным контролем России - лояльный к ним
президент, вся власть в целом, даже военные министр россиянин - и что? А
ничего, Россия все больше поглощала Украину, экономическое давление было ПРИ
ЯНУКОВИЧЕ, задолго до майдана, когда они показушно, с подъебоб, запрещали ввоз
некоторых наших товаров и в таком роде. Нынешняя Россия видит Украину как часть
России, не более - неоспоримо и очевидно, а повод они всегда найдут. Это как
члены братства, которые нападали на тебя в Киеве, а на вопрос «за что?» в медиа
ехидно говорили о якобы твоём гомосексуализме (хотя ты давным-давно в
гражданском браке с девушкой), мол это повод дать тебе пизды, за извращение,
хотя били тебя за твою позицию, которая была для них неугодной. Неприятная, но
параллель

\begin{itemize} % {
\iusr{Алексей Баскаков}
\textbf{Dmitriy Krasnov} 

у вас абсолютно этноцентричная позиция человека, который уважает и любит как
своего дедушку-бандероца, так и самого Бандеру. Поэтому вам и не понятна
позиция автора, которая бесконечная далека от этнократических идей начала
прошлого и конца позапрошлого веков.

\iusr{Dmitriy Krasnov}
\textbf{Алексей Баскаков} 

да, молодец, Бандеру углядел в моем тексте. Если ещё два раза перечитать,
маленького Путина в трусиках можно будет увидеть))

\iusr{Rostyslav Pelekhovych}
на зло Америці підіграємо Росії. дєцкій сад

\iusr{Алексей Баскаков}
\textbf{Dmitriy Krasnov} 

человек, который любит Бандеру, не может адекватно воспринимать тексты,
написанные вне идеологии этнонационализма.

\iusr{Владимир Могилевский}

А сейчас Украина не под полным контролем? Когда еще она была настолько не
самостоятельной? Даже при СССР самодеятельности было больше.


\iusr{Rostyslav Pelekhovych}
\textbf{Vladimir}, а від кого вона залежна? Від тих кого нв днях президент публічно закликав не займатись нагнітанням паніки?

\iusr{Владимир Могилевский}
\textbf{Rostyslav Pelekhovych} в Конституции Украины написано от кого.

\iusr{Rostyslav Pelekhovych}
\textbf{Vladimir}, ну так продовжить думку, не соромтесь: від кого і чому більше ніж за счср

\iusr{Владимир Могилевский}
\textbf{Rostyslav Pelekhovych} От Америки и мирового империализма. А ваш национализм только инструмент.

\iusr{Rostyslav Pelekhovych}
\textbf{Vladimir}, ну от я і питаю, в СССР перший секретар КПУ міг публічно критикувати Москву, серйозно? )

\iusr{Dmitriy Krasnov}
\textbf{Алексей Баскаков} я не знаю о чем вы, где вы в тексте увидели Бандеру? Перечитайте))

\iusr{Dmitriy Krasnov}
\textbf{Алексей Баскаков} 

вот вы все как под копирку. Вам самим не дает покоя Бандера, ибо вам насрали им
же в голову, и с вами бесполезно спорить - предсказуемо и скучно. Знаю, о чем
вам подобные будут молвить дальше. Главное, высрать нарратив о Бандере, образ
которого ебет вас всех в жопу ведь «деды воевали», а потом повторять это с
видом одухотворителя-интеллигента, сидя на бутылке и под бояркой)

\iusr{Константин Станиславский}
\textbf{Dmitriy Krasnov} 

Что за чушь, когда это Украина была под полным контролем России. Это тогда
когда Путен просил Януковича, не возбуждать уголовное дело против Тимошенко? И
тот его послал. И Юля села. Или возможно тогда когда правительство Азарова
подало иск в Стокгольмский арбитраж? А при Порошенке Украина его выиграла и
получила с России 3 миллиарда долларов?

Или тогда когда во всю шли переговоры с Евросоюзом об ассоциации, и Кремль чуть
ли не каждый день предупреждал о том что это скажется на двусторонних
отношениях? А потом в Вильнюсе таки не подписали, ибо условия были такие что
Янукович понял - это просто катастрофа.

Как уже неоднократно было сказано Янукович попробовал сыграть с взрослыми
дядями в геополитику, и ожидаемо облажался. Но это уж точно не от того что был
под полным контролем Путена.

Когда еще Украина была под контролем России? При Ющенко? Или при Кучме? Кучма
вообще двери в Кремль ногой открывал, ибо тогда зависимость (именно так!!!)
России от Украины была огромной. СССР совсем недавно распался, огромное
количество критически важных производств остались в бывшей УССР, газопроводов в
обход не было.

Ну и промыли же вам мозги укрТВ, вы вообще очень плохо, как и большинство
украинцев, ориентируетесь в ситуации. Ничего удивительного нет в том что ридна
нэнька в такой опе.

\iusr{Dmitriy Krasnov}
\textbf{Константин Станиславский} 

да, это то, что любят делать выпердыши из России. Сидеть под пустой, фейковой
странице и вонять о промытых мозгах)Как же однообразно и уныло. Ещё напиши
«нэзалежная» как твои пропагандонские личинки типа Скабеевой любят делать и все
- дискурс соблюдён, можешь идти фапать


\iusr{Константин Станиславский}
\textbf{Dmitriy Krasnov} 

очень глупо, почему интересно все украинцы, когда я своим комментом их задеваю
за живое, и макаю в очевиднейшую вещь, и при этом по делу, они явно не способны
ответить. Так вот они в первую очередь лезут ко мне на страничку. Бедный.
Совсем плохо с аргументами, сильно задавил?

Ну да, нет у меня там ничего, друг сделал года три назад, зачем то. Я то сам с
Украины, читаю Страну, а там есть лента соцсетей. И когда есть время почитываю
и захожу. Постоянно мне каких то друзей предлагают, но мне это глубоко
неинтересно.

Ну от того что у меня страница такая, на вполне конкретные замечания, ответить
не способен? Ох братки украинцы, довели страну.

Телевизор кстати не смотрю с 2004-го года. Вот как интернет у меня появился так
и не смотрю, не интересно. И тебе братуха не советую, уж больно сильно он на
тебе действует.

Не расстраивайся. То что не можешь ответить по существу, это для политических
украинцев нормальная вещь. Нет у вас фактов и аргументов. Вот в чем беда.
Бывай.

\iusr{Dmitriy Krasnov}
\textbf{Константин Станиславский} 

ну вот, одно и тоже повторяешь. Оправдываешься как телка. Сам придумал - сам
наговорил и все это под соусом «последнее слово за мной». Хух, ты победил. И не
«с» Украины, а «из»)

«Читаю Страну»

 @igg{fbicon.beaming.face.smiling.eyes} Я не удивлён, что ты читаешь этот ресурс)


\iusr{Константин Станиславский}
\textbf{Dmitriy Krasnov} 

ты прекрасен. Украинец во всей красе. Вот интересно почему французам плевать
что их столицу русские называют Парижем, а не как они Пари. Да просто они знают
что их город, лучший в мире, и как там кто там называет им не интересно. Но
замечу что именно \enquote{на} и \enquote{с} говорил Т. Г. Шевченко. Говорил и писал. Не знал?!

Это просто комплекс. Увы, комплекс неполноценности. Когда иностранцы говорят
Москау, вместо Москва, как все нормальные люди, или Раша, вместо нормального
Россия, у русских не возникает к ним никаких претензий. Ну говорят, и говорят,
что с этим сделать то?

Моя бабушка и мама и все мои родственники так говорили, я так привык братуха, я
не хочу ни кого оскорбить, или показать отношение, ты можешь понять такие
простые вещи? Я привык говорить так, как говорили мои предки.

И если тебе кажется что это оскорбительным (!!!!), то это, извини, твои
проблемы. Завтра ты мне расскажешь что слово Таллин я должен писать с двумя
буквами Н? А, ну да, плохой пример...

Ну вы ребята реально смешные, и очень несчастные, если такие вопросы для вас
очень важны.

Читай Кобзаря братуха, много поймешь.

\iusr{Dmitriy Krasnov}
\textbf{Константин Станиславский} 

ты уже третий раз повторяешь одно и тоже, это признак тупости. Не оправдывайся,
комментатор с рефлексом бабушки у лавочки) Приплёл свою неграмотность к
неполноценности)

\iusr{Константин Станиславский}
\textbf{Dmitriy Krasnov} 

к большому сожалению, ты весьма не интеллектуален, судя по твоим ответам, ибо
кроме Страны, я читаю еще и Униан, и Цензор.нет. Ой, какая неожиданность? Да? В
этом то и состоит твоя проблема, твой мозг слишком девственнен, и слишком
промыт что бы понять существует масса других мнений, и твое не является
абсолютно истинным.

\iusr{Dmitriy Krasnov}
\textbf{Константин Станиславский} да, ты это уже говорил. Молодец, ещё повтори)

\iusr{Константин Станиславский}
\textbf{Dmitriy Krasnov} 

и что же я повторяю в третий раз, совсем бедняжечка зарапортовался. Бывает. Но
заметь, ни на один вопрос, заданный в моем, самом первом посте, ты не ответил.
Ну не пиши глупости пожалуйста, реально жалко выглядишь. Мне нравятся сильные
оппоненты.

\iusr{Dmitriy Krasnov}
\textbf{Константин Станиславский} молодец, сам себя похвалил, отлизал сам себе и будь доволен)

\iusr{Константин Станиславский}
\textbf{Dmitriy Krasnov} 

ладно. Раз уж ну нас вышел такой разговор, ты несешь непонятную чепуху, и я
тебя не осуждаю. Понимаю, тебе очень сложно.

Но! Может быть ты ответишь на самый первый мой пост. Сначала ты не хотел
отвечать, потому что у меня аккаунт какой то не такой. Если бы ты после этого
перестал писать, было бы понятно. Ну там какой то бот украинофобский, фигню
пишет. Но разговор у нас продолжается. После твоих слов об отлизывании, я
предлагаю вернуться в конструктивную область.

Напоминаю. Ты заявил что до 2014-го года, Украина была чуть ли не полуколонией
России. А я привел массу аргументов что это бред сивой кобылы. Так что же ты
все таки ответишь, по поводу этих очевидных фактов.

Да, я знаю тебе тяжело. Да эти факты неопровержимы. Но будь таки мужчиной,
парируй.

Мы оба знаем что это невозможно, и мне тебя даже жаль слегка. Но! Такова судьба
любого политического украинца. Нет аргументов! Нет фактов! Есть только бездна
глупейших поступков украинских власть предержащих. И ты, и всевозможные
пропагандисты с вашего ТВ не способны объяснить все эти бесконечно уничтожающие
вас факты.



\end{itemize} % }


\end{itemize} % }
