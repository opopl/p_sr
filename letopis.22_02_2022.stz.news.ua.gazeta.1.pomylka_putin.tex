% vim: keymap=russian-jcukenwin
%%beginhead 
 
%%file 22_02_2022.stz.news.ua.gazeta.1.pomylka_putin
%%parent 22_02_2022
 
%%url https://gazeta.ua/blog/56612/rosiya-viznala-tak-zvani-ldnr-ce-fatalna-pomilka-putina
 
%%author_id bushanskij_valentin
%%date 
 
%%tags __feb_2022.putin.priznanie,dnr,lnr,putin_vladimir,rossia
%%title Росія визнала так звані ЛДНР. Це фатальна помилка Путіна
 
%%endhead 
 
\subsection{Росія визнала так звані ЛДНР. Це фатальна помилка Путіна}
\label{sec:22_02_2022.stz.news.ua.gazeta.1.pomylka_putin}
 
\Purl{https://gazeta.ua/blog/56612/rosiya-viznala-tak-zvani-ldnr-ce-fatalna-pomilka-putina}
\ifcmt
 author_begin
   author_id bushanskij_valentin
 author_end
\fi

Україна на порозі повномасштабної агресії. Але ми вистоїмо.

Президент РФ підписав указ про визнання ЛДНР. А це означає, що РФ порушує
Мінські домовленості. РФ уже не може вимагати від України імплементації
Мінських домовленостей. Бо, з російського погляду, ЛДНР – суверенні держави.
Імплементації Мінських домовленостей не можуть вимагати від України також
Франція та Німеччина.

Мінські домовленості базуються на тому, що Донецьк і Луганськ – це Україна. Але
на цих територіях (нібито) є \enquote{комбайнери та трактористи}, яких непокоїть статус
російської мови, окремі питання історичної пам'яті та обсяг прав місцевих
громад. За це й ведеться війна. А покласти край їй мають Мінські домовленості –
надання ЛДНР широкої автономії. Остання мала включати незалежність суддівської
та прокурорської вертикалей і легалізацію створених там РФ армійських корпусів.

Імплементація Мінських домовленостей – самогубство України. ЛДНР у складі
України – це нескінченний терор і поширення сепаратизму. Україні вдалось
уникнути цієї фатальної загрози.

Визнавши ЛДНР, Кремль вкотре пішов на порушення суверенітету та територіальної
цілісності України. Відтак Москва наражається на нові санкції. Останні, однак,
ніхто не запроваджуватиме. Зараз Вашингтон і Брюссель намагатимуться осмислити
нову реальність. Але безуспішно. Крім \enquote{глибокого занепокоєння} ми
нічого не почуємо. Але й імплементації Мінських домовленостей від України вже
не вимагатимуть. І це – плюс.

\begin{zznagolos}
Імплементації Мінських домовленостей від України вже не вимагатимуть. І це – плюс
\end{zznagolos}

РФ домагатиметься розширення територій ЛДНР до адміністративних меж Донецької
та Луганської областей. Однак адміністративний поділ України є суто формальним.
Донецька та Луганська області – це не аналог Татарстану чи Башкортостану, які
сформовані за етнічним принципом. Тож усі претензії РФ та маріонеткових ЛДНР
позбавлені жодної бази.

Варто очікувати загострення ситуації на лінії розмежування. Але офіційна
підтримка РФ своїх маріонеток буде відкритим порушенням норм міжнародного
права. Тепер світ буде вимушений визнати РФ країною агресором. Маски спадають.
І це добре.

Путін припустився помилки. Зірвав власну геополітичну гру.

Україна на порозі повномасштабної агресії. Але ми вистоїмо!
