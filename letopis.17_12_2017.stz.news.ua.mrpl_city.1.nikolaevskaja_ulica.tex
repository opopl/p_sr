% vim: keymap=russian-jcukenwin
%%beginhead 
 
%%file 17_12_2017.stz.news.ua.mrpl_city.1.nikolaevskaja_ulica
%%parent 17_12_2017
 
%%url https://mrpl.city/blogs/view/nikolaevskaya-ulitsa
 
%%author_id burov_sergij.mariupol,news.ua.mrpl_city
%%date 
 
%%tags 
%%title Николаевская улица
 
%%endhead 
 
\subsection{Николаевская улица}
\label{sec:17_12_2017.stz.news.ua.mrpl_city.1.nikolaevskaja_ulica}
 
\Purl{https://mrpl.city/blogs/view/nikolaevskaya-ulitsa}
\ifcmt
 author_begin
   author_id burov_sergij.mariupol,news.ua.mrpl_city
 author_end
\fi

Эту улицу назвали в честь левого предела Харлампиевского собора. Предел этот
был посвящен святому Николаю, покровителю рыбаков и моряков. Улица начиналась у
старого рынка, а точнее — старожилы, которым перевалило за семьдесят,  должны
помнить — у мясных лавок. От рынка, мясных лавок и прилегающих к ним строений
не осталось и следа.

\ii{17_12_2017.stz.news.ua.mrpl_city.1.nikolaevskaja_ulica.pic.1}

Поэтому свое путешествие по Николаевской начнем от окрашенного в
грязновато-розовый цвет дома, который  стоит на углу Земской улицы. До войны
здесь был клуб профсоюза металлистов, а после него — тоже клуб, но морской, где
бывалые моряки Иван Ильич Опушнев, Андрей Афанасьевич Торорощенко, Леонид
Андреевич Сосновский, Андрей Ефимович Черцов, Евгений Викторович Павлов (в его
честь названа привокзальная площадь) и другие, прошедшие суровую школу морских
сражений, дерзких кровопролитных десантов и торпедных атак, передавали свои
знания и житейский опыт мальчишкам, которые, только повзрослев, осознали, какую
роль в их воспитании сыграли эти мужественные, добрые люди.

Для большинства ныне живущих мариупольцев, даже очень преклонного возраста,
название Николаевская — относительно новое. Кто постарше, помнит ее как улицу
Ленина. Но с тех пор, когда имя основателя советского государства было
присвоено главному городскому проспекту, - а это произошло в 1960 году, - она
стала именоваться скромно — Донбасской. Чуть выше Торговой улицы среди домов, в
большинстве своем еще дореволюционной застройки, возвышается параллелепипед
здания \enquote{Горгаза}. Его построили на месте снесенных жилых домов,
послевоенной городской типографии и Дома колхозника — полугостиницы,
полуночлежки.  Наискосок от \enquote{Горгаза}, на другой стороне улицы,
находится, по старинным меркам, большой дом (ул. Николаевская, 23). В нем в
первые послевоенные годы размещалась контора артели инвалидов \enquote{Донбасс}
(может, поэтому улицу назвали Донбасской ?). Деятельность артели была
разнообразной: в ее дворе в сараях клепали стальные бочки, собирали
металлические кровати. Сюда же приходили надомницы-вязальщицы трикотажа, чтобы
сдать выполненную работу (носки, полосатые \enquote{моряцкие} нательные
фуфайки, невообразимого фасона пуловеры из хлопчато­бумажной пряжи), получить
скудную зарплату и материал — сигарообразные шпули с навитыми на них нитками,
но самое главное — хлебные карточки. Позже артель прекратила свое
существование; через какое-то время особняк заняло управление
\enquote{Отделстрой}. Это его мастерицы заменили известковую штукатурку
керамической плиткой. Теперь у облагороженного особняка новые хозяева.

\ii{17_12_2017.stz.news.ua.mrpl_city.1.nikolaevskaja_ulica.pic.2}

На нечетной стороне Донбасской улицы, сразу после Харлампиевской (кажется, что
еще недавно она называлась Советской), стоит многоэтажное здание (ул.
Николаевская, 27). Его построили перед самой войной для водолечебницы. Здесь
доктор Курзон и его коллеги лечили больных водой, пропущенной через щебенку из
доменных шлаков, и целебными грязями. Грязи возили на автомобилях с морского
побережья, а шлаки — с завода \enquote{Азовсталь}. Кстати, и сама водолечебница была
построена за счет этого предприятия. Говорили, что бальнеологические процедуры
давали лечебные результаты не хуже, чем на модных кавказских курортах. В сорок
третьем году гитлеровцы сожгли водолечебницу. Восстанавливали ее по частям.
Сначала здесь нашло пристанище Товарищество художников, потом была баня,
наконец, мастерская по изготовлению эмалевых табличек для могильных памятников
и уличных указателей. В конце 90-х годов здесь же находилась Азовская
универсальная товарная биржа, возглавляемая генеральным директором Василием
Николаевичем Александровым. В наши дни  там помещаются, судя по вывескам,
страховое и туристское агентства.

На противоположном углу есть еще одно приметное здание с двумя рядами окон,
сложенное из светло-коричневого кирпича. До революции — это одна из
мариупольских синагог, а после, когда \enquote{выяснилось}, что  Бога нет,  —
производственное помещение портняжной артели \enquote{Вільна праця}, затем детский сад
швейной фабрики. Какое-то время помещение пустовало, а теперь в ней молятся
верующие одной из местных религиозных общин. За несколько шагов до Греческой
улицы по нечетной стороне Николаевской расположен особняк, сложенный из
серо-желтого песчано-глинистого кирпича. (ул. Николаевская, 51).  В нем в
послевоен­ные годы помещалась станция скорой медицинской помощи. Сюда бежали
люди за врачом или фельдшером, когда случалась беда. Именно бежали, потому что
телефоны-автоматы были редкостью, еще большей редкостью считались телефоны
домашние, особенно в кварталах, застроенных одноэтажными домишками. Рядом со
\enquote{скорой} стояли линейки, - экипажи напрочь исчезнувшие с улиц Мариуполя, -
запряженные лошадками. На них дежурные бригады медиков выезжали к больным.

\ii{17_12_2017.stz.news.ua.mrpl_city.1.nikolaevskaja_ulica.pic.3}

На Николаевскую улицу выходит крыло средней школы №1, а до 1919 года —
Мариупольской Мариинской женской гимназии. Это старейшее в нашем городе здание,
специально построенное для школы, которое используется по первоначальному
назначению. Ближе к Таганрогской, то бишь к улице Артема, селились в прошлом
люди с достатком. И дома были у них основа­тельные: с высокими потолками, с
большими окнами. Прочные парадные двери, украшенные резьбой, затейливыми
латунными ручками, осенялись козырьками, которые опирались на кованые узорчатые
кронштейны. Часть этих особняков была национализирована в известный период
нашей истории. Владельцем стал горкомхоз. Без должного ухода и под действием
времени дома стали мало-помалу ветшать. Пришлось отселить жильцов, и теперь
некогда добротные жилища представляют собой жалкое зрелище. Правда, в самое
последнее время некоторые из них куплены предпри­нимателями. Устраивая в них
офисы и магазины, деловые люди пытаются кто как может придать фасадам божеский
вид. Еще одно старинное здание (ул. Николаевская, 63). Когда-то в нем был
центральный телеграф. До тех пор, пожалуй, пока в шестидеся­тые годы не
построили известный сейчас всем Дом связи. Что запомнилось об этом учреждении
связи? Пожалуй, только стол, заляпанный фиолетовыми чернилами, чернильница -
\enquote{невылевайка}, простенькая ручка с вечно царапающим пером.

Почти с того времени, как бывшая улица Ленина превратилась в Донбасскую, может,
чуть раньше, по ней стали ходить трамваи. Четвертый маршрут шел в порт, а
третий — на железнодорожный вокзал. Наверное, и теперь, по старой памяти,
трамваи, подъезжая со стороны рынка к улице Артема, останавливаются перед
перекрестком, словно задумываются: ехать ли им прямо или свернуть налево? А
потом, как бы вспомнив, что рельсы, ведущие в порт и на вокзал, давно сняты,
покорно поворачивают на улицу Артема и, позванивая колокольчиком, катятся вниз
по наезженному пути. Стоит только пересечь улицу Артема, как попадешь совсем в
другой мир. Шум, гам, толчея. Из одних трамваев толпами выходит люд, в другие
заходит... Отчаюги-водители умудряются на своих автомобилях пробраться между
трамвайными вагонами и людьми. Здесь же идет торговля яблоками, сигаретами и
еще бог знает чем.  Чем дальше удаляешься от трамвайной остановки, тем меньше
встречается старинных домов. Кажется, что покидаешь добрый старый Мариуполь и
въезжаешь в чужой город. В какой? Неизвестно. Ведь девятиэтажки такой серии
могут стоять в Макеевке, Сумах, Рыбинске или Стерлитамаке.

\ii{17_12_2017.stz.news.ua.mrpl_city.1.nikolaevskaja_ulica.pic.4}

Путешествие по улице, которая последовательно была Николаевской, потом Ленина,
затем Донбасской и, наконец, вновь обрела первородное имя, заканчивается в
толчее у Центрального рынка. Что за судьба у улицы? Обязательно она должна
упираться своей оконечностью в рынок.
