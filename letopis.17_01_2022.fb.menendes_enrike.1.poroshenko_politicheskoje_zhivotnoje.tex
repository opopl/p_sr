% vim: keymap=russian-jcukenwin
%%beginhead 
 
%%file 17_01_2022.fb.menendes_enrike.1.poroshenko_politicheskoje_zhivotnoje
%%parent 17_01_2022
 
%%url https://www.facebook.com/e.menendes/posts/6842034145838953
 
%%author_id menendes_enrike
%%date 
 
%%tags politika,poroshenko_petr,ukraina
%%title Порошенко — это идеальное политическое животное
 
%%endhead 
 
\subsection{Порошенко — это идеальное политическое животное}
\label{sec:17_01_2022.fb.menendes_enrike.1.poroshenko_politicheskoje_zhivotnoje}
 
\Purl{https://www.facebook.com/e.menendes/posts/6842034145838953}
\ifcmt
 author_begin
   author_id menendes_enrike
 author_end
\fi

Порошенко — это идеальное политическое животное, как метко написал Gennadiy
Druzenko. Я бы добавил, что политическое животное в плохом значении этого
слова. Такая идеальная беспринципность, что даже зависть берёт. Замечу, что это
не просто политический оппортунизм, в западном смысле, а именно животность,
пикантности которой добавляют обстоятельства, в которых разворачивается
действие. Да-да, я имею в виду коррупцию на войне и расколотое украинское
общество, где миллионы людей при Порошенко начали чувствовать себя чужими в
собственной стране. 

В комментариях пользователей в фейсбуке я встретил две важные мысли. Первая
звучит от сторонников Петра Алексеевича и повторяет одну из линий его защиты –
что этот человек в 2014 году спас страну и отразил Путина. Это важный тезис,
потому что у меня есть существенное возражение. А не слишком ли большой ценой
была спасена страна? И была ли вообще реальная угроза её потерять? Как
государство, мы потеряли Крым и значительную часть Донбасса, чего можно было
избежать – я настаиваю на этом. И всё ради того, чтобы на остальной территории,
благодаря удачному использованию мифологии Евромайдана и отражения атаки
Путина,  установилась чья-то личная власть без особой политической конкуренции.
А война против Путина хорошо отражена в ставшей классической фразе «Жму руку.
Обнимаю». Как после этого у кого-то поворачивается язык прикрываться
«спасением» страны и отражением агрессии?

Второй тезис высказывают все без исключения – и сторонники, и противники, -
что, мол Порошенко фигура слишком большого масштаба и ему не грозит тюремное
заключение. В силу ли миллионов сторонников, которые этого не допустят, или
сговора элит, где реального наказания топ-преступников никогда не происходит.
Скажу откровенно: на мой взгляд, по совокупности Порошенко это один из самых
заслуживших тюремного заключения политиков в истории Украины. Но я тоже не
особо верю, что правосудие свершится. Мне совершенно не нравится дело, по
которому ГБР ведёт следствие. Торговля углём не выглядит в глазах общества
преступлением. Те, кто выступает против торговли с ОРДЛО в основном,
поддерживают Порошенко и найдут ему оправдание даже если он средь бела дня
убьёт кого-то на улице. А те, кто не поддерживают блокаду, не видят в этом
проблемы, кроме моральной.

За что по-настоящему стоит судить Порошенко, так это за то, что произошло на
востоке Украины. За пиар и коррупцию на крови военных, за погибших гражданских,
за ненависть и боль, которые на долгие годы поселились в сердцах украинцев. И
это не пустые слова. Даже если правосудие не свершится, что довольно вероятно,
учитывая наши реалии, история всё равно даст свою оценку всем участникам драмы.
И бывшему, и нынешнему, и всем, кто их окружает.

А сейчас, если убрать долой всю мишуру, то перед нам предстанет фигура седого
человека, который боится и очень хочет власти. И готов ради её получения на
всё. Но ему больше не суждено.

\ii{17_01_2022.fb.menendes_enrike.1.poroshenko_politicheskoje_zhivotnoje.cmt}
