% vim: keymap=russian-jcukenwin
%%beginhead 
 
%%file 12_04_2021.youtube.zharkih_ekaterina.istoria_post_jazyk.cmt
%%parent 12_04_2021.youtube.zharkih_ekaterina.istoria_post_jazyk
 
%%url 
 
%%author 
%%author_id 
%%author_url 
 
%%tags 
%%title 
 
%%endhead 
\subsubsection{Коментарі}
\label{sec:12_04_2021.youtube.zharkih_ekaterina.istoria_post_jazyk.cmt}

\begin{itemize}

\item \emph{Petr shumski}
Какая прелесть: умная красавица, да еще и оратор хороший. Успехов тебе, красавица.

\item \emph{Екатерина Жарких}
Благодарю!

\item \emph{Сергей Андрусов}
Очень красивая и смелая умная девушка👍респект из Новосибирска

\item \emph{Александр Волобуев}
Пока будет попытка остаться над схваткой, вы дальше будите ущемляться в правах.
Вы конечно хорошие люди, но живёте в розовых очках. Все это уже проходил

\item \emph{Igor Bond}
Спасибо. Желаю вам успеха!

\item \emph{Екатерина Жарких}
Благодарю!

\item \emph{Серёга HOVER 174 Челябинск}
Умная, красивая, интересная👍👍👍👍 Молодец, привет из Челябинска

\item \emph{Егор 018}
Победили памятники, победим язык и заживём... наконец-то.

\item \emph{Игорь Белов}
Спаси Господи и во всем помоги. Аминь

\item \emph{Sergey Sakevich}

Катя молодец, только сегодня узнал что есть такая журналистка. Не боится эту
украинствующую злобную серость, не прогибается под неё. Всё правильно ты Катя
сказала, именно из-за этих воинствующих злобных "патриотив" и разрушается эта
территория. Крым ушел, Донбасс  ушел, и судя по всему развал на этом не
остановится. Что вполне закономерно, ибо рушится экономика, завязанная на
производстве, которое разрушают и пилят на металл с 1991г, объясняя это
"розбудовой нэзалэжности". Луганск.

\item \emph{Егор 018}
Молодец Катюша, правильно рассуждает.

\item \emph{Tpoлль обыкновенный}

На мой взгляд, цели и мотивы действий некоторых украинцев в ограничениях
использования русского языка и «наездах» на УПЦ во многом лежат в желании
увидеть негативную реакцию как у жителей России, так и украинских граждан,
использующих русский язык, или у прихожан УПЦ. «Сʙᴎдoᴍыe» совершают какие-то
действия, и смотрят на реакцию: если она негативная, то их это радует, и они
продолжают действовать в том же направлении. Иными словами сами «цели», которые
они декларируют, их мало интересуют, им важнее реакция «зpaдʜᴎᴋᴏʙ» (по принципу
«Пусть у соседа корова сдохнет»).

Это похоже на действия бывших супругов при разводе, когда стороны зачастую
действуют не потому что им что-то реально нужно (например имущество, или право
общаться с детьми), а назло друг другу, пытаясь вызвать у другой стороны как
можно больше негативных эмоций, разозлить, «выбесить», как сейчас стали
выражаться. (Кстати, это к вопросу о «стране как большой семье», о которой
недавно Катя говорила Джангирову, чем вызвала множество комментариев с
обвинением в наивности).

К сожалению, эти люди («cʙᴎдoᴍыe») живут только сегодняшним днём, и даже не
пытаются задумываться что будет завтра -- например, какова будет судьба детей
не изучавших в школе русский язык.

\item \emph{Алиса Рыбка}
Вы не тролль. Вы умничка!

\item \emph{Андрей}
Успехов!

\item \emph{Александр Пользователь}

Катя, а почему Вы считаете, что гражданином Украины не может быть русский? А
венгром или,  например,  уйгуром? Или Вы несколько не то хотели сказать?😉

\item \emph{Арт}

Ошибочны тезисы типа "ситуация в инфопространстве не отражает реальности -
таких людей в действительности меньшинство". Потому как инфопространство
отражает не количество, а качество (которое определяется СТОЙКОСТЬЮ и
АКТИВНОСТЬЮ) - что действительно важно. Количество на самом деле вторично: вся
история человечества - с его зарождения и по сей день - это история навязывания
воли меньшинства большинству. Различаются лишь методы достижения этого.
Простой пример: какое значение имеет количество поддерживающих русский язык,
если явное меньшинство (зато стойко-активное) просто не даёт провести
референдум. И не даст - пока не переломит, и нытьём и капаньем, общественное
мнение гарантированно под себя.

И это потихоньку происходит - процесс пошёл, и время работает против нас.

В этой связи показательно, что даже люди что называется на острие активности с
нашей стороны - напр. та же Катя - уже по умолчанию приняли статус вторичности,
неполноценности русского. Это проявляется казалось бы в мелочах типа "меню на
русском НЕ принципиально" - но это такие мелочи, из которых в конечном итоге и
складывается реальность.

Вот так и работают окна Овертона :(

\item \emph{andromeda 27}

Удивительно слышать, как озвученные тобой мысли совпадают с моими)). Ты
говоришь ровно то же самое, что и я. Тут как и с Шарием - сразу чувствуется,
что мы земляки, потому что мыслим одинаково). Это дает повод предположить, что
осталось еще много коренных киевлян, которые тоже думают точно так же, просто
их не слышно. Хотя по моим наблюдением наверное с половина их все-таки попала
под влияние всей этой пропаганды.

Желаю нам успехов в борьбе, и думаю, что все получится, просто для этого нужно
время... Спасибо).

\begin{itemize}
\item \emph{Екатерина Жарких}
Спасибо большое за поддержку! Думаю, что здравомыслящих больше и когда мы объединимся - страна станет лучше)

\item \emph{andromeda 27}

\verb|@Екатерина Жарких|  В этом наша главная задача. Их мало, но они как
шакалы, сбиваются в стаи и атакуют нас по отдельности. Но как только они видят
всего лишь численное превосходство - сразу разбегаются по щелям). А наша
проблема - нет объединяющего фактора. 

Нужны лидеры, которые способны нас объединить. Я пока вижу только одного -
Шария... Ой, двоих - с женой, конечно). Больше, к сожалению, не вижу, они куда
ни посмотри - сплошной Дизлукаш). Может, есть и другие пути объединения, нужно
их искать. 

Тогда для отрезвления власть имущих можно было бы как минимум пойти по пути
страйков, которыми пытались раскачать Беларусь. Я тогда помню смотрел и
завидовал, на самом деле достаточно эффективный метод).

\item \emph{andromeda 27}

\verb|@Екатерина Жарких| Кстати, мне понравилось, как ты поешь социально важные
песни. В Украине много людей, которые в них нуждаются. Они как-то помогают
более достойно преодолевать эти испытания, воодушевляют... Успехов тебе в этом
деле👍. И еще советую послушать киевскую группу "Миелофон у меня", особенно
песню "Плавниками" - она меня тоже очень удивила. Василец в свое время навел).
Такое творчество сейчас просто необходимо. Если вдруг понадобится помощь с
гитарой - обращайся, чем смогу помогу. И привет от ГП "Антонов").

\end{itemize}

\item \emph{Andrew Dronsson}
"Ёпта" - даже это не могут правильно транслитерировать, надо было писать не "epta", а "yopta".

\item \emph{Владимир Шевченко}
Спасибі вам, ЛЮДИ.

\item \emph{Sasha Bah}
Добрая наивная девочка. Надеюсь, выживет.

\item \emph{Екатерина Жарких}
С Вашими молитвами 🙏🏻❤️

\item \emph{Геннадий Сычёв}

Больше всего в "языковом вопросе" поражает тотальное лицемерие. Как сказал один
из депутатов (от ППШ, если не ошибаюсь): "В Украине 99\% понимают русский язык,
и 1\%  делают вид, что не понимают."

Большинство из ратующих за "державну мову" в быту общаются на русском, а
украинского толком не знают! На всех живых эфирах у большинства выступающих на
украинском заметны явные паузы в речи, поскольку поголовно они думают НА
РУССКОМ и в уме переводят на украинский. 

ВСЕ прекрасно знают, что нет никакого "русскоязычного меньшинства", ВСЕ
прекрасно знают, что во всех сферах -  в литературе, в быту, в интернете, в
развлекательном контенте etc русский язык БОЛЕЕ востребован, ВСЕ прекрасно
знают, что называющие русский язык "мовой агрессора"  политики дома и не на
камеру общаются ТОЛЬКО на русском - и все делают вид, что так и надо!

\begin{itemize}

\item \emph{Eugene Baranov}

У славян язык język, език, jezik ... это знаковая система передачи  смыслов и
только у индивидов слово в основном своем значении это мышечный орган.

мова от праславянского молва (слухи, толки), говорить, просить

\item \emph{Геннадий Сычёв}

А еще меня крайне веселит манера переводить на украинский имена собственные .)
"Дмитро ПЄсков", "ВОлодимир Путин", "Тетяна", "Евген" - и так далее. Ну
переводите уж тогда и иностранные - почему только русские?) "Мыхайло Яковлевич"
вместо Майкла Джексона, Мыкола вместо Николаса, Юлiю вместо Джулии, ну и
остальных в том же духе.))

\item \emph{Eugene Baranov}

\verb|@Геннадий Сычёв|  Я не против закрепления языкового обычая для новоиспеченных
граждан НО переименовывать Михаила (командующий ангельским воинством) от
рождения в Мыхайло это конечно рафинированная русофобия

\item \emph{Angeles y Demonos}

перенять мировой опыт: если 2\%-8\% населения говорит на своем языке, то он
признается вторым, третьим государ. языком. Даже кривая украинская статистика
признает, что 24\% населения говорят на русском. миллионы людей в нескольких
поколений русскоязычных строили современную Украину, основали города, и дети их
имеют право на рус.яз. гос. язык. Русские не пришлые с 1991г.

\item \emph{Геннадий Сычёв}
\verb|@Angeles y Demonos|  

Дык, все всё это знают.
Но тем не менее..
"Маемо шо маемо".

\item \emph{Samuel}

\verb|@Angeles y Demonos|  У каждой страны своя ситуация и история,
проецировать опыт страны А на страну Б не совсем верно а порой и
контрпродуктивно, во Франции тоже есть нацменьшинства но официальный язык
везде только один - французский, считайте что Украина=Франция.

24\% населения не выражают никакой потребности в русском языке на региональном
или же официальном уровне, уже была куча соцопросов и только 11-13\% заявили о
необходимость русского как 2-го государственного или же регионального и я даже
предполагаю за кого эти люди голосуют на выборах, вопрос закрыт.

\item \emph{Angeles y Demonos}

\verb|@Samuel|  где ваша статистика что только 11\%-13\% хотят чтобы русский был
государственным языком? сами придумывали и не учитывали мнение русскоязычных
Донбасса?  И даже 11\%-13\% населения, это огромный \%, основываясь на мировом
опыте, чтобы язык был государственным. При чем тут Франция, назовите какое там
нац меньшинство имеет 11\% населения, которое  говорит на не французом языке.
Официальный французский язык в Бельгии, Канаде, хотя там есть еще другие
официальные государственные языки. Будете ломать людей языком, не видать
единой Украины. Запугивайте дальше население.

\item \emph{Samuel}

\verb|@Angeles y Demonos|  Никто никого не запугивает, все живут спокойно. Я
честно скажу, мнение русскоязычного Донбасса для меня не имеет значения после
того как они взялись за оружие и позвали чужое государство в мою страну. Ссылки
не могу оставить ибо ютуб их удаляет, поэтому просто напишу кто проводил
исследования - группа рейтинг, Демократические инициативы, центр Разумкова. 

Наша ситуация схожа с Латвией и Литвой где тоже есть крупные русскоязычные
диаспоры. Вопрос русской культуры это вопрос национальной безопасности из-за
действий России, Франция не угрожает Бельгии и Канаде а у нас висит вполне себе
реальная угроза от РФ и один из способов сделать её экспансию сюда невыгодной
это выдавить её культуру, сделать людей максимально нелояльными к ней. Это
может показаться некрасиво, мы это делаем топорно и с запретами но по другому
мы пока не научились.

\item \emph{Геннадий Сычёв}
 \verb|@Samuel|

\begin{itemize}
\item 1. \enquote{Куча соцопросов} - это ни о чём. Референдум был? Нет? Значит и не говорите за всех.
\item 2. Вопрос не только ли не \enquote{закрыт}, а даже не открывался - какая разница, \enquote{за
кого эти люди голосуют на выборах}? Это имеет значение? Вы уже по этому
признаку делите людей  на тех, чьё мнение важно, и тех, чьё нет?
\item 3. Вишенка на тортик: \enquote{считайте что Украина=Франция.} 😅😅😅
\end{itemize}

Да без базара.)) Будем считать. Только вы сперва СТАНЬТЕ как Франция - в плане
свобод, доходов, уровня жизни, влияния в мире etc etc. И сразу начнем, ок.))

Пардон, совсем упустил еще один важный момент: \enquote{.. во Франции тоже есть
нацменьшинства} - а не подскажете, уважаемый, каким именно \enquote{нацменьшинством}
в государстве УКРАИНА являются русскоговорящие УКРАИНЦЫ?!

\item \emph{Samuel}

\verb|@Геннадий Сычёв|  Я не говорю за всех, я говорю то что могу подкрепить
3мя соцопросами которые провели за последние 2 года. Вопрос был закрыт в 1996
когда приняли конституцию Украины с украинским как единственно государственным
и если судить по опросам то 88\% соглашаются с этим а потому не вижу смысла
мусолить эту тему, как говорится границы открыты, кому не нравится тот может
уехать. 

Идем к уровню Франции не волнуйтесь, Моисей 40 лет из Египта уходил а нам еще
10 лет осталось чтобы окончательно выйти из России 😎


\item \emph{Геннадий Сычёв}

\verb|@Samuel|  

А, вот оно даже как.) "Остаточно прощевай", ну ну.))

Словом, я вижу, тут аргументация бессмысленна. Идёте, значит.. 

Ну что ж, через 10 лет, если вспомню наш диалог, я даже злорадствовать не буду - мне вас жаль.

\item \emph{Виталий Алимпиев}

\verb|@Геннадий Сычёв|  а мне их не жаль рюськомовное г должно платить и каяться раз стали украинцями

\item \emph{Геннадий Сычёв}

\verb|@Виталий Алимпиев|  

Вы, простите, на каком языке и о чём?)) 

Я, конечно, к убогим отношусь с сочувствием, но вас ,право, просто не понимаю.

\item \emph{Виталий Алимпиев}

\verb|@Геннадий Сычёв|  не даёт написать . желания выварачиваться нету ютуб меня понял попробуй и ты

\item \emph{Виталий Алимпиев}

\verb|@Геннадий Сычёв|  не буду советовать пристроить твоё сочувствие меж батонов ты
умный сам додумаешь . и да я алкаш и быдло . бояра валежник дырка в полу это
всё про меня

\item \emph{Виталий Алимпиев}

\verb|@Геннадий Сычёв|  на пробу описание ордынца попробовал всё до последнего проходит . а толерентно про украину но правду нет

\item \emph{Геннадий Сычёв}
\verb|@Виталий Алимпиев| 

Таблеточек попей.
Реально, проблемы у тебя, чувак.
Сочувствую.
Лечись.

\item \emph{Виталий Алимпиев}

\verb|@Геннадий Сычёв|  на  х это не икс

\item \emph{Амелия А.}

\verb|@Eugene Baranov|, а Николу в Мыколу? А Анну в Ганну? Бррр

\item \emph{Eugene Baranov}

\verb|@Samuel|  Что-то я забыл как "вы" отвоевали землю русскую у Польши и Литвы,
территория вернулась бывшему владельцу России, бандитсвующий на этих
территориях сброд еще не страна и не государство, императрица слишком поздно
их умножила на ноль, много они успели крови попить у народа.

\item \emph{Генадий Петрович}

\verb|@Samuel|  люди на выборах проголосовали за порошенко? нет? за зеленского нет? А
Выборы и голосование в крыму, почему -то не правда. Так что на счёт люди
проголосовали и выборы?

\item \emph{Генадий Петрович}

\verb|@Samuel|  Чему Вы научились за 30 лет независимости? Может снизить 6-ю
экономику мира до последней экономики в европе?  Выдавливая русскую культуру,
поднимать культуру крымских татар? Это как? а где для Вас были крымские татары
все 30 лет независимости?

\item \emph{Генадий Петрович}

\verb|@Samuel|  т.е. Вы за 30 лет совсем ничего не сделали, но за 10 мечтаете?
скажите, а что будет когда МВФ потребует вернуть кредиты? А что будет когда
полностью заработает СП-2? А что будет когда закончится контракт в 2025, о
транзите газа через украину? видимо это даст скачёк экономике украины,
волшебным образом. Сравнивая себя с моисем, Вы евреи или украинцы? Если евреи,
то зачем рассуждния про киевскую русь? если украинцы, то что вам пример евреи?

\item \emph{Генадий Петрович}

\verb|@Геннадий Сычёв|  ну реально как дети, 30 лет ни шагу вперёд, наоборот 1000
шагов назад, отбросили себя на полстолетия и за 10! лет намереваются догнать
своё прошлое и достать францию! Хорошая у них там пропаганда.

\item \emph{Samuel}

\verb|@Генадий Петрович|  Эта тема настолько обсосана что только бот или же
российский гражданин который погряз в кисилевшине не может понять что не так с
незаконным Крымским референдумом 2014 года, но окей. 

Выборы в Украине регулярны, следуют законодательству, они проводятся не под
дулами автоматов и на них присутствуют всевозможные наблюдали не из Украины. 

А
теперь давайте сравним с Крымским незаконным референдумом который вы так любите
- российский спецназ согнал не всех депутатов чтобы запустить незаконную
процедуру на вопрос отделения (Напоминаю, территория Украины общая и
принадлежит всем её гражданам а потому такие вопросы решаются на всеобщем
референдуме, на России такой вопрос даже поднять нельзя), международных
наблюдателей не было, нарисованная явка и результаты как у Лукашенко на
последних выборах. 

Все соцопросы проведенные до оккупации Крыма утверждали что желающих
присоединиться к РФ больше 40\% никогда не было а откуда взялась цифра 96\% в
2014 не понятно, попахивает махинациями.

\item \emph{Samuel}

\verb|@Генадий Петрович|  \enquote{т.е. Вы за 30 лет совсем ничего не сделали, но за 10
мечтаете?} Наша элита и не пыталась что-то делать, мы все тут жили на том
богатстве что оставил СССР и мы его к сожалению не пытались приумножить.

Экономические рывки в 2-4 раза для маленьких экономик как наша делаются за
10-20 лет. 

"скажите, а что будет когда МВФ потребует вернуть кредиты?" Мы им деньги
возвращаем каждый год, на это есть статья в бюджете а вообще не вижу проблем
поскольку в ЗВР лежит 28млрд которые покроют траты на экстренные нужды. 

"А что будет когда полностью заработает СП-2? А что будет когда закончится
контракт в 2025, о транзите газа через украину?" Возможно потеряем 1-2 млрд
транзитных денег, больно но не критично. Если РФ не захочет продлять контракт в
2025 то будем покупать ваш же газ но через Европу, тоже не вижу большой
проблемы. 

"видимо это даст скачёк экономике украины, волшебным образом." Я так легко
говору о скачке ибо наша экономика маленька и построена на сырьевом экспорте,
её можно легко увеличить в 2-3 раза за короткий период времени (Турция такой же
трюк сделала за 20 последних лет, Южная Корея сделала скачек из сырьевой
державы в высокотехнологичную за 30 лет), секрет в том чтобы не мешать
инвесторам.

\item \emph{Генадий Петрович}

\verb|@Samuel|  как удобно, неудобные референдумы признать неправильными фальшивыми. 

это пороховщина или зеленщина?

Поэтому у Вас то педофил у власти то фашист, то клоун? То сразу всё вместе? в
чём смысл этих выборов, что бы в стране всё было хуже и хуже жить?

А теперь давайте сравним с нацизмом, где всё что Вам Не нравится признаётся
незаконным.

Российский спецназ? на территории украины? в РФ изобрели мгновенную
телепортацию, может бронежилеты невидимки? где были пограничные войска украины?
где регулярная армия? Что у вас за страна такая, приходит некий спецназ,
захватывает у Вас типа депутатов и Вы как ни в чём не была возмущаетесь.

Вам самим не смешно от Ваших объяснений?

Напоминаю, что к примеру где должен быть крым решают кто-то в киеве, во львове
или в самом крыму?

Чего Вы до сих пор не провели ни одного референдума, например по русскому языку
или тут уже вся украина не решает, только кучка у власти?

Вы разницу между соцопросом и референдуумом понимаете? одно дело когда просто
спрашивают на улице часть граждан. Уверен Вы точно не опросили даже 20\%
проживающих в крыму, на то он и соцопрос, где берётся некая часть населения
выборочно, из разных слоёв и делаются приблизительные выводы. А вот
референдум, это уже изъявление всех.

Вы бы методички с 2014 года сменили, попахивает глупостью.

\item \emph{Генадий Петрович}

\verb|@Samuel|  т.е. Вы 30 лет ничего не пытались, даже более пытались Но в качестве
регресса и за 10 лет мечтаете не просто вернуть, что было но и францию
переплюнуть? Вам бы фантастику сочинять, не научную.

Для ттго, что бы сделдать то самый рывок, надо иметь хотя бы эклономику, а у
Вас её нет, что не разграбили, то продали, что не продали в собственности
европейских компаний. Какие рывки? ну разве только к сортиру. Вы знаете
сколько должны МВФ? и что в сентябре этого года зеленскому надо отчитаться и
выплатить кредит? 

А что СП-2 перекроет существенный доход бюджета?

Можно подробнее когда и что Вы вернули МВФ? Вы только берёте те самые кредиты,
но пока ничего не вернули, именно поэтому МВФ так возмущены и уже очень не
охотно дают следующие.

Ну если пару миллиардов потери, для ВАс не критичны, почему у Вас люди на
грани нищеты живут? Тарифы как космические ракеты взлетают и предела не видно? 

Какие-то пару миллиардов, ещё дадут, ничего страшного.

Конечно будете покупать, уже у европы, но увы воровать и на халяву уже никак не
получится, причём ценник уже будет выставлять сама европа, а с ней Вам будет
сложнее договориться, их же газ через Вашу территорию не идёт. 

Т.е. потери будут и ещё кроме пары миллиардов., причём не гривен.

У ВАс были территории с1991 до 2014, у Вас 30 лет был транзит газа через
страну, у Вас были те самые миллиарды, которыми разбрасывается свядомый бил
гейц и никакого прорыва , скачка и даже ползка экономики не было и тут, станет
денег меньше, прилично меньше и будет скачок экономики? на дровах? Вам реально
самому не смешно писать такую чушь?

Ладно, отставим зравый смысл логику и перейдём к Вашей логике, чем сильнее
писец, ну денег меньше, всё плохо тем быстрее и выше скачок? 

Так чего Вы жалуетесь по поводу крыма? или не отпустите донбасс, и территория
меньше и писец больше, значит скачок будет быстрее и больше или нет?

Теперь поговорим о турции или корее. Вы простите турок, может кореец? Вы
воровать умеете, а те люди работают, что-то делают, каждый свой цент
вкладывают, поэтому у них и прогресс, а что делаете Вы? воруете перепродаётся
разбазариваете и приводите в пример другие страны?

С кем Вы себя сравниваете? Ваш уровень зимбавве, хотя там хоть что-то делают.

за 30 лет незалежности, причём реально сравните корею, им не доставалась 6- я
экономика мира, не так ли, им не дарили научный потенциал, промышленность
авиастроение не так ли? Может это показатель, что у Вас и через 50 лет ничего
не изменится, хотя изменится.. развалится украина на части от обнищания лет
через 5-7 максимум, так что 10, для украины, это сильно долгий срок.

А Вы пока по методичкам теребонькайте, как выкопали чёрное море вышиванками и
сотворили великий борщ, это важнее.

\item \emph{Samuel}

\verb|@Генадий Петрович|  "как удобно, неудобные референдумы признать
неправильными фальшивыми." А чего вы ожидали ? Референдум был проведен с
нарушением нашего законодательства и есть большие сомнения в явке и результате
сего деяния. 

Лучше пускай у нас будет клоун президент нежели КГБист который отравляет
нелояльных граждан на территории других государств. 

"в чём смысл этих выборов, что бы в стране всё было хуже и хуже жить?" Мне
конкретно жить хуже не становится, за других говорить не могу, хотя должен
сказать что растет расслоение между бедными и средним классом. 

"Российский спецназ? на территории украины? в РФ изобрели мгновенную
телепортацию, может бронежилеты невидимки?" Представьте себе, неизвестные люди
в зеленой форме вышли из баз РФ в Крыму, бронемашины оттуда же выехали. 

"где были пограничные войска украины? где регулярная армия?" Сидели на базах и
ждали приказа который не поступил, не нашлось у нас человека который отдал бы
приказ стрелять по россиянам. А вообще намеренная сдача Крыма нашим
правительством, развал нашей армии во времена Януковича это темы достойные
обсуждения в суде. 

"Что у вас за страна такая, приходит некий спецназ, захватывает у Вас типа
депутатов и Вы как ни в чём не была возмущаетесь." Это Украина) 

"Чего Вы до сих пор не провели ни одного референдума, например по русскому
языку или тут уже вся украина не решает, только кучка у власти?" До Зеленского
который подписал закон о референдуме в 2021г, простые граждане не могли
стартовать референдум, так что решала кучка людей у власти который вопрос
русского языка был не важен. 

"А вот референдум, это уже изъявление всех." Референдум это изъявление тех кто
него пришел и конкретно с Крымом по некоторым данным пришло 30\%. Это не
обязательный процесс для каждого как допустим перепись населения где каждого
обязательно опрашивают.

"Вы бы методички с 2014 года сменили, попахивает глупостью." В глаза той
методички не видел, пишу от сердца.

\item \emph{Samuel}

\verb|@Генадий Петрович|  "Вы 30 лет ничего не пытались, даже более пытались Но
в качестве регресса и за 10 лет мечтаете не просто вернуть, что было но и
францию переплюнуть?" ВВП Украины 150 млрд, умножаем на 3 и получаем 450 а
теперь вопрос где тут Французский уровень с их 2.7трлн ? Это вполне реально,
абы наши в правительстве этого хотели. Экономика есть у любой страны кроме
недогосударств по типу Сомали. 

МВФ возмущен что мы не делаем того что обещали когда брали их деньги, для них
пара миллиардов которые они дают Украине это копейки. 

Турки и корейцы были нищими и глупыми в то время как украинцы и русские летали
в космос и шо ? Им это помешало сейчас жить в 2 раза лучше всего пост-ссср ?
Нет.

Вы сами себе противоречите, как сами сказали, мы были 6й экономикой мира а
значит что наши люди умеют работать не хуже турок и корейцев, была наука, были
полеты в космос и промышленность с армией а значит я говорю о вполне возможных
вещях которые были у Украины в недалеком прошлом.


\item \emph{Генадий Петрович}

\verb|@Samuel|  а чего я мог ожидать, от необразованный зашоренных фашистов. Вы правы.

Ещё раз спрашиваю, где были Ваши пограничники войска? где партизаны в данное время?

С чего Вы взяли, что референдум был с нарушением? Потому, что это не
укладывается в концепцию Ваших методичек? у Вас так были всё прекрасно в
стране что никто бы никогда не захотел отделиться, особенно после
государственно переворота.

Или майдан это что-то другое?

Есть большие сомнения, простите, сомнения, домыслы предположения, не относятся
к фактам. А фактов, у Вас похоже нет, но есть сомнения и фантазии.

Мне нравится, лучше пусть клоун, который разворовывает страну, при котором
нищает население.. ЗАКРЫВАЮТСЯ опс, неудобные каналы, убираются неудобные
оппозиционеры.запрещается язык. ой, да тут у нас не просто клоун, а клоун
кгбист.. Неожиданный поворот, Вы видимо этого не знали? забыли в методичку
добавить...

А что делает клоун с не лояльными?

Получается в РФ всего один не лояльный гражданин? и целое государство кгб, не
смогло кго отравить, довести до конца задуманное? 

Деточка, конечно фсб, это не КГБ СССР которое ворочала государствами и решало
кто будет у власти во многих странах, но это не настолько детская организация
из разряда сбу, которая может убрать неугодно и сделать так, что никто в мире
никогда не узнает что было на самом деле. В Вашем понимании, спецслужбы
огромной страны, это тупые дети по примеру СБУ?

Конкретно жить лучше не становится? может подведём результаты?

1991 год- 6- я! экономика мира.. 2021, последняя экономика европы, это как
минимум.. куда уже куже.. Сколько доллар у Вас был 10 лет назад хотя бы и
сколько сейчас? может узнаем не на сколько, а ВО сколько выросли коммунальные
платежи, а за газ?

Может узнаем, сколько людей выехало насовсем из страны, без учёта крыма и
донбасса и сколько уехали и работают только за пределами украины, не принося
доход государству? может ещё хуже? 

крыма нет, донбасса тоже нет, есть гражданская война.. Это по Вашему НЕ хуже?
не хуже чего?

У ВАс не расслоение, у Вас обнищаение.

теперь про спецназ.. Т.е. Вы знали сколько людей на базах, или нет? они значит
без Вашей воли и никто стрелять по ним не стал? отвлекусь от темы.. Вы на
украине кричите, что сталин был в дружбе с гитлером, но что-то при нападении
22 июня никто не ждал приказа сверху, есть приказ защищать свою родину. Вы же
РФ агрессором по жизни считаете, что у Вас за трусливые войска?

Простите, где суд страны над януковичем?  Где судебные разбирательства? где
выясненные виновные?

Это украина.. смешно, может Вам и не нужно быть государством, Вы так легко
отдаёте территории, кстати, янукович Вы выбрали или кгб назначило? может
Выборы не Вы проводили?

Итак это украина.. может Вам вообще не нужно быть государством? зачем, это
украина.. то отдадим, придумаем потом причину заочно осудим правительство
виновато.


\item \emph{Генадий Петрович}

\verb|@Samuel|  "До Зеленского который подписал закон о референдуме в 2021г,
простые граждане не могли стартовать референдум, так что решала кучка людей у
власти который вопрос русского языка был не важен. "

А теперь резко стал важен вопрос русского языка, в стране развал идёт
гражданская война, людям многим нечего жрать, коммуналка в стратосферу, но
нужно решать проблему русского языка, ну да есть ещё борщ и вышиванка.. это всё
поднимет экономику страны, нет, не поссорит даже то население страны, которое
осталось.. это логично.

"Референдум это изъявление тех кто него пришел и конкретно с Крымом по
некоторым данным пришло 30\%. ".. по каким некоторым данным? где те самые
данные? опять данные из разряда сомнений предположений.. 

Вы даже не можете обосновать несостоятельность референдума. Плохо работает сбу.

Вы всё пишите не из сердца, а из методичек, как под копирку, ну хоть бы слово
изменили или предложение по иному построили, так что не врите про сердце.
Кстати рассуждать должен мозг, а не сердце или Вам объяснить разницу между
мыслительным органом и мышцей, насосом для перекачки крови?

Вы реально перестали самостоятельно думать размышлять и критически думать
особенно.

\item \emph{Генадий Петрович}

\verb|@Samuel|  "....ВВП Украины 150 млрд, умножаем на 3 и получаем 450 а теперь
вопрос где тут Французский уровень с их 2.7трлн ? ... " Вы так легко кидаете
миллиардами, легко умножаете на три..  Вы реально взрослый человек? с чего
будет в три раза, у Вас гос долг 1./3 ВВП..

Экономика есть, пока есть государство, но судя по Вашим свободным и
независимым выборам, скоро у Вас её не будет.

".... абы наши в правительстве этого хотели..... " а кто-то из правительства
за 30 лет захотели? 

Как Вы там говорите, лишь бы клоун, пускай крадёт, ворует, разваливает, а в
методичках напишем, что там стоит, умножьте на 3 и будет как во франции.

Какая разница копейки или нет.. Вы что из тех кто не отдаёт долги? только МВФ, это не сосед, он терпеть не будет.

Для Вас пара миллиардов, судя по Вашим сообщениям, это тоже копейки..

Именно поэтому украина в 2021 просит и ждёт от МВф ещё 700 миллионов? Зачем они Вам, эти копейки, можно же раз и на три умножить..

Вот я и говорю украинцы в составе СССР летали в космос, а что сейчас у Вас
летает? фанера над парижем, богдан? Один самолёт и тот только для
показательного выступления президента.. ОДИН!!!!  куда там турции и корее.

Теперь понятно, что у турции и корее прогресс, а у Вас уже 30 лет регресс и
огромный., т.е. если Вы и попытаетесь нагнать, все остальные за это время от
Вас ещё дальше убегут.

Но по итогу выборов и президента, как минимум до конца его срока у Вас будет
такой же стабильный регресс, если клоун, слава богу не кгбист, до этого
времени с потрохами страну не продаст и не разделит.. Это это во много раз
более вероятнее, чем Ваши фантастические умножаем в три раза.. сколько ноль не
умножай, ничего не будет..

Это Вы себе противоречите. Повторю страны РАЗВИВАЮТСЯ, а не деградируют, разница есть?

Украина НЕ развивается, от слова совсем, и не стоит на месте, а падает.. Вы
реально сравниваете, можно из нищеты подняться, если работать, но у ВАс в
крови и поколение выросло воров и взяточников и кого они воспитают? Видимо
тех, кто поднимет страну? Вы реально не смеётесь?

Не Вы работали, не сядомые работали, а вся страна работала, не Вы себе сделали
6-ю экономику мира, а Вам её подарили , оставили.. и что Вы наработали? 

Римская империя когда-то правила половиной мира и где сейчас рим и италия?

Довожу ещё раз, что бы у Вас не было выдуманных противоречий. У Вас было 100,
за 30 лет остался 1, утрирую.. выросло поколение, которое НЕ заработало, не
умеет работать, только воровать.. Куда Вы с ним поднимитесь? с какого
перепуга? Инопланетяне прилетят? Или воры своих детей воспитают Не воровать?

Ещё раз разжую.. нищие собрались, начали работать и разбогатели и дальше
двигаются, богатые начали пробухивать своё богатство, вырастили поколение и не
одно таких же лодырей и воров, довели себя до нищеты, но работать не хотят и
Не будут.. откуда взлёты к турции? Логика понятна? В который раз говорю,
отложите методички, сами начинайте думать, а не то, что Вам на уши наложат.

 
\item \emph{Генадий Петрович}

\verb|@Samuel|  Вы все свои возможности уже профукали и не просто не
останавливаетесь.. а вниз несётесь причём без уменьшения скорости, пока у Вас
воры лицемеры у власти, а Вы думаете, ай.. просто клоун, ничего страшного
выберем другого.. какого? за 30 лет хоть один был? с чего он будет дальше?
пришлют с марса? У Вас людей уже зомбировали, что Не будут они выбирать
нормального.

Прошлое Вам дали и что у Вас осталось от того прошлого? Вы и то прошлое
замарываете как только можно, у Вас кругом враги, вначале Рф, теперь
беларусь..

Вы уже европу не интересует, точнее им уже достало кидать в пустоту миллиарды,
короновирус научил их экономить. Они даже не отстаивали газопровод через
украину, думаете почему? Да они прекрасно знают, что Вы воруете.. 30 лет газ
воруете.. Какие взлёты? какая турция? У Вы даже не предполагали, что Вам газ
перекроют, думали, можно будет воровать вечно, чего париться.. куда РФ
денется.. при этом наделывали под дверь рф и тоже считали, что так будет
вечно. Брали дотации и не думали возвращать ? а разве думали, а чего такой
большой госдолг? кто Вам его навязал? может кгб? 

И вы реально рассуждаете и умножаете на 3? Что можно сказать.. реально раньше
было образование на украине научный потенциал, а теперь даже элементарно
мыслить ну никак не получается.

Сколько Вам нужно что бы остановить падение? что бы были по стране голодные
бунты? и территория осталась одна область и та львовская?  войну гражданскую
вначале остановите, а потом про миллиарды долларов и умножение на 3
фантазируйте


\item \emph{Samuel}

\verb|@Генадий Петрович|  "у Вас гос долг 1./3 ВВП.."  "Какая разница копейки
или нет.. Вы что из тех кто не отдаёт долги? только МВФ, это не сосед, он
терпеть не будет."

Еще раз повторю если туго доходит, про внешние долги я могу не волноваться,
ежегодно из бюджета уходит 5\% на внутренние долги и 6\% на внешние, также у нас
есть 28млрд в ЗВР которыми можно покрыть пару миллиардов если надо будет срочно
отдать МВФ в следующие пару лет, отдавать есть чем. 

"Как Вы там говорите, лишь бы клоун, пускай крадёт, ворует, разваливает, а в
методичках напишем, что там стоит, умножьте на 3 и будет как во франции." Вы
опять приписываете мне то что я не писал, я лишь выразился что наш клоун лучше
бывшего КГБиста который убивает своих граждан ради стабильности своего режима.
Про кражи и развал я ничего не писал, между прочим из-за авантюр вашего КБГиста
в Крыму и на Донбассе вы потеряли триллион ВВП, с пиковых 2.3 млрд в 2013 упали
до 1.2 в 2016 и сейчас еле еле развиваетесь в год по 1.6\% а наша экономика
растет по 3-4\%. 

"Для Вас пара миллиардов, судя по Вашим сообщениям, это тоже копейки.." Я мыслю
в масштабах государства и пара миллиардов в год даже для Украинского
государства это не столь большая сумма чтобы бить тревогу, считайте это
последний гвоздь в гроб кооперации Украины и России. 

"Именно поэтому украина в 2021 просит и ждёт от МВф ещё 700 миллионов? Зачем
они Вам, эти копейки, можно же раз и на три умножить.." Сумма смешная и скорее
всего пойдет стабилизацию макроэкономической ситуации. 

"Но по итогу выборов и президента, как минимум до конца его срока у Вас будет
такой же стабильный регресс, если клоун, слава богу не кгбист, до этого времени
с потрохами страну не продаст и не разделит.. Это это во много раз более
вероятнее, чем Ваши фантастические умножаем в три раза.. сколько ноль не
умножай, ничего не будет.." 

Сейчас регресса не наблюдаю, все значимое что можно было уничтожить и продать
было продано давно, доля нашего государства в экономике ничтожна мала и
составляет 12\%, окончательно мы были на дне в 2014-2015 благодаря действиям РФ
и за 8 лет потихоньку оттолкнулись обратно. 

К примеру у нас средняя ЗП в долларовом эквиваленте в 2013 была 390\$ а теперь
500\$, для сравнения в РФ в 2013 люди в среднем получали 915\$ а сейчас 750\$ и у
кого в итоге регресс :) ?

"Это Вы себе противоречите. Повторю страны РАЗВИВАЮТСЯ, а не деградируют,
разница есть?" Украина развивается насколько это возможно при плохих отношениях
с бывшим первым торговым партнером который отжал туристическую базу и уничтожил
донбасскую промышленность. 

"т.е. если Вы и попытаетесь нагнать, все остальные за это время от Вас ещё
дальше убегут." Ерунда, все страны не могут постоянно развиваться в одном
темпе, рано или поздно что-то случается и происходит откат и стагнация, Россия
и Китай тому примеры.

"Украина НЕ развивается, от слова совсем, и не стоит на месте, а падает.. Вы
реально сравниваете, можно из нищеты подняться, если работать, но у ВАс в крови
и поколение выросло воров и взяточников и кого они воспитают? Видимо тех, кто
поднимет страну? Вы реально не смеётесь?" 

В корне неверно, поколение взяточников и воров это уходящие советское которое
почуствовало свободу и безнаказанность 90-х и нелегальным путем забрало
богатства страны и их главные рожи - Порошенко, Янукович, Кучма, Медведчук,
Коломойский, далее по списку форбс и самый главный Ахметов, все советские
старики. Новое поколение с либеральными и капиталистическими ценностями валит
отсюда работать и жить в Европу не желая разгребать здешние проблемы либо же
уходят в ИТ сферу подальше от государства (200тысяч работают в ИТ и на сегодня
мы 11я страна по ИТ)

"выросло поколение, которое НЕ заработало, не умеет работать, только воровать..
Куда Вы с ним поднимитесь? с какого перепуга? Инопланетяне прилетят? Или воры
своих детей воспитают Не воровать?" Ворует уходящие советское поколение, их
рожи можете увидеть в украинском форбс а молодое поколение либо едет в Европу и
перенимает их рабочую этику и толерантность к взяткам либо идет в ИТ сферу, вы
совсем не понимаете здешней ситуации. Процесс перехода власти от старого
советского поколения к моему 20-30 летнему уже пошел и можно увидеть первые
плоды, дальше будет только лучше.

\item \emph{Генадий Петрович}

\verb|@Samuel|  Видимо до Вас не доходит.. у Вас 1/3 ВВП, это госдолг, Вы о каких-то
5\% говорите, а долго 25! Вы вообще в школе учились, Вы должны 25\% выплачиваете
каждый год 5-8\%, при этом берёте в долг ещё..  Похоже реально на украине все
школы позакрывали.

Осенью 2021 посмотрим, как у Вас есть чем отдавать долг, кстати, если есть чем
отдавать, зачем Вам ещё инвестиции? Вы противоречие сами себе и есть чем долги
отдавать, но до сих пор не отдали и долг растёт и просите ещё у МВФ. Это как
сочетается?

".....Вы опять приписываете мне то что я не писал, я лишь выразился что наш
клоун лучше бывшего КГБиста который убивает своих граждан ради стабильности
своего режима. ...."

Ваш клоун убивает людей на донбасе ради своего режима. Убивает пенсионеров,
ради своего режима.. Дальше продолжать? может напомнить чистку телеканалов,
неудобных?

"....  а наша экономика растет по 3-4\%. ..." расскажите сказки про рост Вашей
экономики, госдолг растёт, инвестиций в 700 миллионов ждёте как манну
небесную.. просите что бы транзит не остановили, тарифы улетают в космос.. Это
всё потому что растёт экономика? Вы бредите?

"....Я мыслю в масштабах государства и пара миллиардов в год даже для
Украинского государства это не столь большая сумма чтобы бить тревогу,
считайте это последний гвоздь в гроб кооперации Украины и России. ..." забавно
как некий недоучка мыслит в масштабах государства, может Вам в президенты? Вон
зеленский на предвыборной компании тоже пообещал такого, там масштабно
промыслил, что не просто ничего не осуществил, так ещё и в большее дно
государство загнал.

Что касается гроба, то у Вас уже гроб вогнал с китаем, теперь будет с Россией.

Вы уверены, что кто-то с Вами реально будет торговать? У Вас по сути ничего
нет, ну кроме дам лёгкого поведения и сборщиков клубники.

\item \emph{Генадий Петрович}

\verb|@Samuel|  прошли шутки за 500

"...Сейчас регресса не наблюдаю, все значимое что можно было уничтожить и
продать было продано давно, доля нашего государства в экономике ничтожна мала
и составляет 12\%, окончательно мы были на дне в 2014-2015 благодаря действиям
РФ и за 8 лет потихоньку оттолкнулись обратно. ..."

Т.е повышение тарифов в разы, это  Вы не наблюдаете? так зеленский и свита тоже
не наблюдают.. тотальное покидание людей страны именно после 2014 года, это
тоже Вы не замечаете? может с каждым годом всё возрастающий госдолг? может
инфляцию с 2014 не замечаете?

 Вы оттолкнулись, но в не в верх, а вниз и правильно можно было подумать, что
 Вы достигли дна, но снизу вам постучали и теперь у вас ещё более глубокое дно.

"...Сумма смешная и скорее всего пойдет стабилизацию макроэкономической
ситуации. ..." так зачем Вы ждёте просите и даже требуете эту смешную сумму?
видимо даже что-то в мелочах не можете сами стабилизировать? Вы реально
прикалываетесь? Страна на грани бедности, но в Ваших рассуждения, Вы на уровне
минимум китая. В который раз повторю. Вы по методичкам говорите, но любите
рассуждать про сердце, а нужно рассуждать мозгом.

"....Ерунда, все страны не могут постоянно развиваться в одном темпе, рано или
поздно что-то случается и происходит откат и стагнация, Россия и Китай тому
примеры..... Вы 30 лет уже так стагнируете и предела пока не достигли. Но это
всё ерунда, так к чему эти крики про крым донбасс, СП-2? уменьшение транзита
газа, мольбы про очередную дотацию МВФ, просьбы отсрочить выплаты по долгу,
если у ВАс так всё замечательно? Рост экономики, поднимаетесь.. Вы в который
раз сами себе противоречите, но методичка из киева, не даёт Вам возможность это
увидеть. Там написано? значит так и есть и не важно что есть на самом деле.

"....В корне неверно, поколение взяточников и воров это уходящие советское
которое почуствовало свободу и безнаказанность 90-х и нелегальным путем забрало
богатства страны и их главные рожи - Порошенко, Янукович, Кучма, Медведчук,
Коломойский, далее по списку форбс и самый главный Ахметов, все советские
старики..... Видимо зеленский, это советский старик? а может безуглая? а может
черновол? У ВАс уже более дыух лет у всоасти не старики и не советские люди и
что в итоге? воровать стали меньше? нет, даже больше, может что-то сделали в
своей стране? тоже нет, опять вам мешает СССР?

".....Новое поколение с либеральными и капиталистическими ценностями валит
отсюда работать и жить в Европу не желая разгребать здешние проблемы либо же
уходят в ИТ сферу подальше от государства.... Хоть и не перевариваю на дух
такое поколение и либералов, но Вы сами написали.. что Валит и подальше от
Вашего государства.. Так что Вы там пели про развитие? про поднятие со дна? про
рост доходов? Ваши доходы возросли не в стране, а у тех, кто из неё выехал и
все Ваши IT приносят доход другим государствам, а не украине.

"....Процесс перехода власти от старого советского поколения к моему 20-30
летнему уже пошел и можно увидеть первые плоды, дальше будет только лучше.....
Ну точно агитка из книжки.. прямо всё по брошюре капитализма.

Так можно увидеть в чём плоды? покажите где, в какой сфере пошёл это самый
процесс? в 2025 отключат транзит, до 25-го будет существенно сокращён транзит,
осенью этого года надо будет дать денег МВФ, который на украине нет, а если не
дадут Вам те самые смешные 700 миллионов, потому что даже в европе уже устали
кормить ваших воров и зеленского в том числе, то в следующую зиму дома будет
топить дровами и костры на улицах будут, для погреться.

\item \emph{Генадий Петрович}

\verb|@Виталий Алимпиев|  скажите, а когда украинцы заплатят за то, что им подарили,
в том числе и русскоговорящие? или они считают, что им кто-то за что-то
должен, а они нормально могут принимать подарки?

\item \emph{Samuel}

\verb| @Генадий Петрович|  В расходной статье бюджета есть строчка
"Обслуживание внешних долговых обязательств" и рядом с ней стоит цифра в 6.25\%,
про 25\% ничего не знаю. 

"Вы противоречие сами себе и есть чем долги отдавать, но до сих пор не отдали и
долг растёт и просите ещё у МВФ. Это как сочетается?" Нужны деньги на покрытие
дефицита бюджета, из ЗВР в бюджет просто так перекинуть нельзя а на сокращение
социальной статьи никто не хочет идти, так шо приходится брать 2-3 млрд в год.

"Ваш клоун убивает людей на донбасе ради своего режима. Убивает пенсионеров,
ради своего режима.. Дальше продолжать? может напомнить чистку телеканалов,
неудобных?" Войну начали до него и он со старта своей каденции делает все
возможное чтобы остановить стрельбу, при этом очевидно не сдавая наших
национальных интересов. Пенсионеров никто не убивает, они везде плохо живут,
наши пенсионеры с пенсией в 100\$ мало чем отличаются от тех же российских с
200\$. 

"может напомнить чистку телеканалов, неудобных?" Не чистка телеканалов а защита
национальной безопасности от российских агентов влияния ! Вы неправильно все
понимаете.

"расскажите сказки про рост Вашей экономики, госдолг растёт, инвестиций в 700
миллионов ждёте как манну небесную.. просите что бы транзит не остановили,
тарифы улетают в космос.. Это всё потому что растёт экономика? Вы бредите?" 

Госдолг растет у США и шо? Конечно просим не останавливать транзит, вот вам
будет приятно если вы потеряете халявные 2 млрд в год ? 

" тарифы улетают в космос" Они улетают потому что на этом зарабатывают
олигархи, у нас есть газ своей добычи которого хватает для потребления людей но
цена у него международная.

"забавно как некий недоучка мыслит в масштабах государства, может Вам в
президенты? Вон зеленский на предвыборной компании тоже пообещал такого, там
масштабно промыслил, что не просто ничего не осуществил, так ещё и в большее
дно государство загнал." Конечно недоучка ! Я же выпускник современного
украинского университета, вы могли не писать столь очевидные вещи а в
президенты к сожалению не могу, юный еще слишком.

"Что касается гроба, то у Вас уже гроб вогнал с китаем, теперь будет с Россией.
Вы уверены, что кто-то с Вами реально будет торговать? У Вас по сути ничего
нет, ну кроме дам лёгкого поведения и сборщиков клубники." С этим проблем не
будет, мы торгуем тем что всем всегда нужно - то что растет в земле и полезные
ископаемые.

\item \emph{Генадий Петрович}

\verb|@Samuel|  "...про 25\% ничего не знаю....." Если Вы про это не знаете,
может просто в методичку не прописали? Хотя это я ошибся, не 25\% а 30...

"...Нужны деньги на покрытие дефицита бюджета, из ЗВР в бюджет просто так
перекинуть нельзя а на сокращение социальной статьи никто не хочет идти, так шо
приходится брать 2-3 млрд в год.... Это очередное оправдание? типа страна такая
успешная, что ей необходимы постоянно деньги? Тем более такая мелочь, ни на что
не влияющая, что ждут её, как манну небесную.

Перекинуть нельзя, а почему своровать можно? это видимо проще?

"......Войну начали до него и он со старта своей каденции делает все возможное
чтобы остановить стрельбу, при этом очевидно не сдавая наших национальных
интересов. ......" Расскажите что именно он делает, что бы война закончилась?
два года уже? ездит на позиции? Пугает всех войной с РФ? Может ведёт переговоры
с представителями ЛНР и ДНР?  Что он свои интересы отстаивает, это понятно. Но
по Вашему и по зеленскому получается, что война в ваших интересах?

\item \emph{Генадий Петрович}

\verb|@Samuel|  "......Войну начали до него и он со старта своей каденции
делает все возможное чтобы остановить стрельбу, при этом очевидно не сдавая
наших национальных интересов. ......" Расскажите что именно он делает, что бы
война закончилась? два года уже? ездит на позиции? Пугает всех войной с РФ?
Может ведёт переговоры с представителями ЛНР и ДНР?  Что он свои интересы
отстаивает, это понятно. Но по Вашему и по зеленскому получается, что война в
ваших интересах?


\item \emph{Генадий Петрович}

\verb|@Samuel|  "....Пенсионеров никто не убивает, они везде плохо живут, наши
пенсионеры с пенсией в 100\$ мало чем отличаются от тех же российских с 200\$.
... Т.е. это в РФ пенсия меньше, чем выплата за газ! без коммуналки, еды
лекарств? Вы что там реально курите?

Вы постоянно распеваете, как у Вас всё хорошо, всё растёт , страна на подъёме,
и тут.. пенсионеры живут плохо, прям как в то самой России, у которой, по
Вашим же методичкам подъёма нет! это как?

Я Вам разжую, без методички, есть, по Вашим словам украина которая на подъёме,
всё супер, развивается, рост ВВП , долги не страшные и есть рф, которая по
Вашим же методичкам, наоборот, разрушается, как минимум и всё там плохо, но
пенсионеры живут и там и у Вас, в развивающей стране плохо?

Это ещё раз спрашиваю как?

"....Не чистка телеканалов а защита национальной безопасности от российских
агентов влияния! Вы неправильно все понимаете.... Как я мог перепутать.. Вы бы
просто написали, это другое нам это можно, это в других странах считается
нарушением прав человека и свободы слова, а если на украине или в европе, это
защита интересов. А в других странах, а зачем им свои интересы защищать.. Я
реально думал, что Вы хоть как-то способны думать.

\item \emph{Samuel}

\verb|@Генадий Петрович|  "Т.е повышение тарифов в разы, это  Вы не наблюдаете?" Нет,
не наблюдаю. Как правило тарифы повышаются вместе с зарплатами, я посмотрел
сколько я тратил на тарифы в 2013 и сравнил с сегодняшней цифрой, разница
минимальная и составляет 27\$. 

"тотальное покидание людей страны именно после 2014 года, это тоже Вы не
замечаете?" Замечаю но в стране идет война с 2014 и восточные европейцы
предлагают более лучшие условия труда и более высокую зарплату. 

"У ВАс уже более дыух лет у всоасти не старики и не советские люди и что в
итоге? воровать стали меньше? нет, даже больше, может что-то сделали в своей
стране? тоже нет, опять вам мешает СССР?" До этого 28 лет советские люди
строили неэффективную государственную систему, новое поколение получило
разграбленную страну с войной в придачу, также никуда не делись олигархи родом
из СССР, это все нельзя изменить всего за 2 года не прибегая к авторитарным
действиям. Черновол ? Серьезно ? Он мыслил по совестки но с украинским душком.

\item \emph{Генадий Петрович}

\verb|@Samuel|  "....Не чистка телеканалов а защита национальной безопасности
от российских агентов влияния ! Вы неправильно все понимаете.... Как я мог
перепутать.. Вы бы просто написали, это другое нам это можно, это в других
странах считается нарушением прав человека и свободы слова, а если на украине
или в европе, это защита интересов. А в других странах, а зачем им свои
интересы защищать.. Я реально думал, что Вы хоть как-то способны думать.----

 ----------Выделяю Вам самые смаковые места отдельно, может хоть что-то
 разбудит Ваш спящий мозг.

\item \emph{Генадий Петрович}

\verb|@Samuel|  "....Нет, не наблюдаю. Как правило тарифы повышаются вместе с
зарплатами, я посмотрел сколько я тратил на тарифы в 2013 и сравнил с
сегодняшней цифрой, разница минимальная и составляет 27\$. ..."

А теперь оф данные:

Год	2013	2018	2019	Подорожало

Электричество	0,26-0,33 грн за кВт/ч	0,9 - 1,68 грн за кВт /ч	 	в пять раз
Квартплата	2,41 грн/м2	5,85 грн/м2	 	в два раза
Газ	0,93 грн/м3	8,54 грн/м3	 	почти в 10 раз
Холодная вода	3,18 грн/м3	19,59 грн/м3	 	в шесть раз
Отопление	2,91 грн/м2	31,55 грн/м2	36,59 грн/м2	в 12 раз
Горячая вода	16 грн м3	74,53 грн/м3	86,45 грн/м2	почти в шесть раз

Один только газ подорожал в 10! раз, а у Вас, видимо свой личный газ.

Напомнить во сколько обесценилась гривна? в 3! раза! Вам бы врать научиться
вначале или Вы думаете, что сбу со своим методичками перекрывает мировую сеть-
интернет?

"...Замечаю но в стране идет война с 2014 и восточные европейцы предлагают
более лучшие условия труда и более высокую зарплату.... " Вам реально
скопировать Ваши изречения из методички, как у Вас всё хорошо, потери
минимальные, страна на подъёме, а люди бегут, это как? Так всё хорошо, что люди
бегут? Вы сами противоречия в своих же словах Не замечаете?

"....До этого 28 лет советские люди строили неэффективную государственную
систему,.. " т.е 6-я экономика мира, какой была украина при СССР, это не
эффективно, а последняя экономика, минимум европы, какой в данное время
является украина, это что-то из разряда эффективности? Вы вообще понимаете
значение слов ? Эффективно, это когда в плюс, а не в минус. Чувствую себя
учителем в детском саде, в группе с детьми с отклонением в умственном
развитии.

Деточка, эффективно, это когда у тебя есть два яблока и потом становится 4, а
не одно или ни одного.

".... новое поколение получило разграбленную страну с войной в придачу, также
никуда не делись олигархи родом из СССР, это все нельзя изменить всего за 2
года не прибегая к авторитарным действиям. Черновол ? Серьезно ? Он мыслил по
совестки но с украинским душком...."  черновол, не он , а ОНА.. из депутатов
слуги народа, Вы видимо своих людей не знаете, такое поколение юное.. Или Вы
не видели рядом безуглую? , это видимо тоже что-то из советского образования,
ой, да её в США сама псаки стажировала...

"....не прибегая к авторитарным действиям... " т.е закрытие запрещения и
прочее, это не авторитарные действия? Ах, да, забыл, это в интересах украины,
плевать на конституцию международные права человека и свободу слова, это в
других странах нельзя, но тут другое.

Перезидета на украине выбирают на 5 лет, если не ошибаюсь, смысл в нём, если
ему пол срока мало, хоть что-то сделать. Зато можно говорить про другие страны
в которых постоянные литеры, они точно понимают, что на перемены нужно время.

Может поспорить, если зеленскому дать и 10 лет, ничего в Вашей стране к лучшему
не измениться, да впрочем и страны уже не будет, что не разворуют, то
распродадут, а территорию по европе разделят, ну видимо всё это в интересах
страны.

\item \emph{Виталий Алимпиев}

\verb|@Генадий Петрович| пятрович вы больны? с какого перепугу я должен знать когда
vыpycь самуил отдаст долги .там по их наративам это только трусы не те что
лёшик шьёт отдают

\item \emph{Генадий Петрович}

\verb|@Samuel|  "......Конечно недоучка ! Я же выпускник современного
украинского университета..." ------------------------ они и видно, что
современного и украинского. Есть методички, та всё написано, думать? нельзя.
Анализировать? ни в коем разе, мозг взорвётся, критическое мышление? а зачем
оно человеку с современным образованием. Вы реально показали насколько низко
сейчас образование на украине, просто на уровне зимбаве, хотя там наверняка
всё же людей учат думать самостоятельно, а не методички читать, у нас всё
хорошо, надо немного времени, да никакого роста коммуналки тарифов за газ в 10
раз нет, это всё происки кремля, то что Вы получаете жировки, это фейк, но
заплатить должны, а мы вам дальше будет петь про процветание страны.

\item \emph{Генадий Петрович}

\verb|@Виталий Алимпиев|  Вы бы написали не левой ногой, через правое ухо...

\item \emph{Генадий Петрович}

\verb|@Виталий Алимпиев|  поставьте хотя бы запятые, если грамотности в словах нет.


\item \emph{Генадий Петрович}

\verb|@Samuel|  "....Госдолг растет у США и шо? Конечно просим не останавливать
транзит, вот вам будет приятно если вы потеряете халявные 2 млрд в год ? ....
Вы же несколько раз повторили, что для Вас и государства это просто тьфу и
мелочи? зачем такой скандал ради мелочи? или Не мелочи?

"...у нас есть газ своей добычи которого хватает для потребления людей но цена
у него международная.... У Вас есть газ? Вы 30 лет об этом знаете, но ничего
не сделали? Вы реально психически адекватные? Теперь понятно почему украина
никогда нр при каком президенте не будет хорошо жить.. воровство в генах.

Зачем своё добывать? можно же своровать.

Что касается  своего газа , которого хватает на нужны, Вам напомнить сколько
было скандалов когда переговоры между украиной и Рф по газу не клеились и как
газ на украине отключали и как просили включить транзит, потому что люди
мёрзнут..

Именно потому что хватает на потребление людей, зеленский говорит, что если
запустят северный потом, то накроется армия украины..

"......Конечно недоучка ! Я же выпускник современного украинского
университета..." ------------------------ они и видно, что современного и
украинского. Есть методички, та всё написано, думать? нельзя. Анализировать? ни
в коем разе, мозг взорвётся, критическое мышление? а зачем оно человеку с
современным образованием. Вы реально показали насколько низко сейчас
образование на украине, просто на уровне зимбаве, хотя там наверняка всё же
людей учат думать самостоятельно, а не методички читать, у нас всё хорошо, надо
немного времени, да никакого роста коммуналки тарифов за газ в 10 раз нет, это
всё происки кремля, то что Вы получаете жировки, это фейк, но заплатить должны,
а мы вам дальше будет петь про процветание страны.

".....С этим проблем не будет, мы торгуем тем что всем всегда нужно - то что
растет в земле и полезные ископаемые...." ---- ну реально по методичке всё. Вы
конкретнее покажите, а потом посмотрим, что из того, что реально могла
импортировать украина, она постоянно и много экспортирует.. Вот например тот же
газ. Вы же сказали, что у Вас есть, его хватает, но обогреваетесь исключительно
российским.. незадача.. кстати, не слышали про массовый закуп сала из рф
украиной? тоже в методички не добавили? Ну зачем такими мелочами мозг
заполнять, лучше порассуждать, что пару миллиардов это ничего страшного, но
скандал устроить, из-за мелочи нужно, даже германии угрожать с СП-2, США в
претензии, что СП-2 не закрыли и всё из-за каких-то жалких пару миллиардов.

\item \emph{Виталий Алимпиев}

\verb|@Генадий Петрович| я тебе один раз ответил нет ответа не буду
усложнять . идите гендон куда я думаю понял. ты тут накрапал на дохтурскую не
калбасу рузумеется

\end{itemize}


\item \emph{darkhour}
если не бороться с русским языком и вообще, то под каким предлогом тогда
выпрашивать кредиты? как там говорят - "хіба хочеш - мусиш"

\begin{itemize}
\item \emph{Екатерина Жарких}
Именно 🤷🏼♀️

\item \emph{Александр Пользователь}
Кредиторам абсолютно равнобедренно на каком языке они получат прибыль. Вы
очевидно не видите разницы между кредитами и грантами.

\item \emph{andromeda 27}
\verb|@Александр Пользователь|  Причем здесь гранты? Это вообще отдельная история.

\item \emph{Александр Пользователь}
\verb|@andromeda 27|  При чём здесь отдельная история?

\item \emph{andromeda 27}
\verb|@Александр Пользователь|  На городі бузина, а в Києві - дядько)))

\item \emph{Александр Пользователь}
\verb|@andromeda 27|  Ученье — свет?

\item \emph{andromeda 27}
\verb|@Александр Пользователь|  Культпросвет

\item \emph{Александр Пользователь}
\verb|@andromeda 27|  Культорг.
\end{itemize}

\item \emph{Александр Fight}

Манипуляция, перевертывание, переиначивание, переверание, лицемерие, лукавство
и вранье - постоянно и кругом -особенно идет от верхушки правления страны и
спускаеться далее по порочному кругу. Екатерина вы как всегда на высоте и
правильно все делаете и говорите и ведете себя. Оставайтесь "человеком" тем
которым вы являетесь не изменяйте себе - спасибо вам.

\item \emph{Виктория Викторова}

Раньше было очень приятно слышать украинский, сейчас раздражает безумно(((а
слышать то как говорят по украински с телевизора-просто стыд, настолько
безграмотной речь украинская не была никогда((((

\begin{itemize}
\item \emph{Екатерина Жарких}
Такое ощущение, что специально коверкают 🤷🏼♀️ упрощают, как у Оруэлла

\item \emph{Виктория Викторова}

\verb|@Екатерина Жарких|  я Оруэлла читала в 91, мне казалось что это ужас
37года, и переболели. Оказазалось что это вечно, сделай людей беднее и в
опасности - и вот оно, свежая охота на ведьм, просто ведьмы теперь говорят
по-русски. И инквизиция, и концлагеря, и борьба с коммунистами в штатах-корень
один(((

\end{itemize}

\item \emph{Washe Polnoe}

Федерализация Украины позволит жителям Закарпатья, Буковины, Галичины, Волыни,
Слабожанщины, Таврии и других, самим решать на каком языке учить детей в школе,
и на каком языке общаться больному с врачем.

\item \emph{Екатерина Жарких}

Вы правы. На практике, правда, это будет *феодализация

\item \emph{Егор 018}

Зелёные человечки - не худший выход для окраины. Спросите у Крыма.

\item \emph{Александр Пользователь}

Где Вы, голубчик, услышали в этом ролике хоть одно упоминание о некой окраине?

\item \emph{Егор 018}

\verb|@Александр Пользователь|  окраина Польши, окраина  России, окраина
здравого смысла (в части русскоязычных "патриотов") и тд. Если нравится, то у
края. Это ещё очевиднее.

\item \emph{Пикейный Жилет}

Не совсем корректно называть "премьерой" ролик, который лежит на Первом
Козацком с 1 июля 2020 г., даже если это премьера на данном канале. А вообще-то
корректнее было бы дать ссылку на оригинал, хотя бы в описании.

\begin{itemize}
\item \emph{Екатерина Жарких}

Спасибо, что следите)

Первый Казацкий дал добро, так что все корректно. 

А тут будут собраны и старые и новые ролики с моим участием.

\item \emph{Пикейный Жилет}

\verb|@Екатерина Жарких|  Старые ролики, конечно, "наполнят" канал, но вряд ли
соберут много просмотров, и картинка получится не очень привлекательная --
люди будут заходить, и видеть, что у большинства роликов 200-300 просмотров,
что заинтересует немногих. Вам нужно стараться делать свои авторские ролики,
стараться приглашать интересных людей (совет банальный, но что делать). 

А ещё (как это ни странно звучит) вам надо бороться за права эксклюзивности,
или хотя бы стараться добиться отсрочки показа на других каналах. Посмотрите:
ролик с Джангировым, размещённый у вас вчера, набрал 7 тыс. просмотров, а
сегодня тот же ролик на канале Д.Г. набрал уже 15 тыс. (думаю наберёт и ещё
пару десятков тыс., как минимум). Ролик записывался не вчера, и если бы Джанго
его выложил ещё через пару дней, то у вас бы прибавилось просмотров -- у
Джангирова есть свой "фан-клуб" (я в нём тоже состою), и многие люди за пару
дней посмотрели бы ролик на вашем канале.

Понятно, что каждый преследует свой интерес -- "в пользу дяди" работать никто
не хочет, но нужно пытаться как-то договариваться, если конечно получится.

\item \emph{Пикейный Жилет}

\verb|@Екатерина Жарких|  В догонку. Ход с показом первой части на "Первом Козацком",
а второй на вашем канале был хорошим, но тактически лучшим было бы показать
первую чать у Джангирова на "Капитале", а вторую здесь. Я не знаю тонкостей
ваших взаимоотношений как с Джангировым, так и с "Первым Козацьким", но с моей
точки зрения как зрителя так выглядело бы лучше.

Правда, (к сожалению) это не избавило бы вас от комментариев с обвинениями в
наивности. В вас, извините за менторские нравоучения, ещё говорит юношеский
максимализм, когда кажется, что люди обязательно должны найти общий язык, если
конечно они разумные. С приобретением жизненного опыта, как правило, человек от
этого избавляется, ежедневно сталкиваясь с тем, что разные люди мыслят
по-разному, воспитаны по-разному, имеют разный темперамент, и в соответствии с
этим, в разных ситуациях по-разному себя ведут. 

Вообще-то это называется "социализацией", при которой люди сталкиваются с тем,
что не все окружающие такие же как они. В нашем (постсоветском) обществе у
социализации, как правило, несколько ступеней -- детский сад, школа, возможно,
какие-то лагеря (в моё время -- пионерские), институт, рабочие коллективы, у
мужчин -- армия (это очень серьёзная адаптация к обществу, и "школой жизни" её
называют не просто так -- в армии человек абсолютно отрывается от всего
привычного, и вынужден выстраивать отношения в большом коллективе "с нуля", без
поддержки кого бы то ни было, безо всяких друзей-родственников, и там человек
"стоит" ровно столько, сколько "стоит" он сам, лично). 

А результатом всех этих социализаций (у кого позже, у кого раньше) и является
жизненный опыт, который избавляет от иллюзий, что "с людьми всегда можно
договориться", и приходит понимание того, что к людям надо хорошо относиться,
но не надо хорошо про них думать. Это называется "цинизм". К сожалению, как-то
так.

\end{itemize}

\item \emph{Eugene Baranov}

По самым заниженным сведениям русских проживало всего от 20\%, в Одессе от 60\%
зачем искать ответы на простой вопрос или русские граждане Украины или общность
плохо образованных людей не знающие родного языка (языка своего рода)

\begin{itemize}

\item \emph{Samuel}

В Одессе никогда не было 60\% русских, максимум 50\% на 1897г а на 2001 украинцев
было 61\% а русских 21\%.

\item \emph{Eugene Baranov}

\verb|@Samuel|  В 1874 году в Киеве русских 45,7\%, малороссов 30,2\% евреев 10,5\%, поляк 6,4\%, немцев 2,1\%. 

Одесса  европейский город Российской империи (малую часть составляли малороссы) 

\obeycr
	Язык Италии златой 
	Звучит по улице веселой, 
	Где ходит гордый славянин, 
	Француз, испанец, армянин, 
	И грек, и молдаван тяжелый, 
	И сын египетской земли 
	Корсар в отставке Морали.
\restorecr

Массовый исход одесситов превратил город в провинциальный украинский облцентр.

Ничего удивительно если русских там больше нет (больше не миллионник) как не
стало красивейшего города.

\item \emph{Samuel}

\verb|@Eugene Baranov|  Как то вы странно перешли с Одессы на Киев но ладно.

Не надо врать и манипулировать, в переписи 1874 года идётся речь о языковых
группах а не о национальном составе и очевидно что в Киеве на то время больше
людей использовало русский из-за того что там всячески уничтожали украинский
язык и запрещали его использование вплоть до постановок в театральной пьесе. 

Массовый исход одесситов это -9 тысяч за 20 лет?) Одесса как и любой другой
крупный город Украины за 30 лет приукрасился, просто город теперь не ваш и если
вы там были то скорее всего вам было банально не комфортно, ментальность
отличается так сказать...

\item \emph{Eugene Baranov}

\verb|@Samuel|  Одесса русский город захваченный нацистами. Перепись какого
года 1897, 1917, 1919 подойдет, речь идет об украинцах как таковых, все
остальное незначительные  детали.

\url{https://zhenziyou.livejournal.com/53951.html}

\item \emph{Eugene Baranov}

\verb|@Samuel|  С логикой вижу не дружите, произошло замещение населения одесситы
убежали, украинцы заселились,  и даже на этом фоне переехавшие в город селяне
не спасли от деградации и спада


\item \emph{Samuel}

\verb|@Eugene Baranov|  Не замещение а естественный процесс, русским не
комфортно в чужом им государстве и они начали уезжать на родину.

\item \emph{Виталий Алимпиев}
\verb|@Samuel|  процесс бывает обратим не так ли? возвращаться в схроны не приятно но придётся

\item \emph{Samuel}
\verb|@Виталий Алимпиев|  Безусловно процесс может быть обращен если Украине не
станет, но это пока из разряда фантастики.

\item \emph{Виталий Алимпиев}
\verb|@Samuel|   не проходит описание обратимого процесса . как ты знаешь там
ваши и ничего не напишешь . удачи . будь спокоен и жди Русских

\end{itemize}

\item \emph{Ivanov Ivan}
Резист...як там кажуть .

\item \emph{R. D.}
А разве 112 и NewsOne ещё есть? Их же Зеле фашисты из британской разведки приказали закрыть.

\item \emph{Vera Nica5515}

Язык - только часть программы создания антиРоссии, успешно реализуемой на
русской земле на русских людях, забывших свои корни... Программы, придуманной и
реализуемой врагами России. 

Скорее бы это квазигосударство "у края" с позорным гимном-реквиемом, гербом -
вилами, флагом от несбывшихся колонизаторов и "терплячим народом', думающим
желудком, исчезло с карты мира! Всем будет счастье.

\item \emph{MAGIRUS}

Родился и вырос в Киеве, за последние 15 лет Город изменился очень сильно, как
итог в сентябре 2020 отчалил в Венгрию, потому как и с работой уже печаль, а
про национализацию уже и промолчу, Киев уже давно не русско язычный  город, а
вы выступаете как внутренняя фронта взывая к национальности ((( жаль, очень
жаль

\item \emph{Александр Волобуев}

Какая разница кого больше, важно кто задаёт повестку. В нацистских концлагерях
тоже все заключённые вроде как равны, но находились те кто готов был
обслуживать печи. Так и у вас происходит. И ситуация с женщиной которая отдала
своего ребенка в украинский класс это доказывает

\item \emph{Виталий Уткин}
\urlFriend{https://www.youtube.com/channel/UCKdk3AiTTgdtwkpRKSi1Fnw}\par
\ifcmt
  ig https://yt3.ggpht.com/ytc/AKedOLRt7QX7yjqVjF5jDK0xuzWC5fLqNLbrYcfbjFNSDd0=s88-c-k-c0x00ffffff-no-rj
  width 0.15
\fi

Помню, в мои детские годы родители вместе со мной отправились на время летних
отпусков в западную Украину. Остановились в деревенском доме, где жили хозяйка
с сыном примерно моего возраста.

На следующий день по приезде тот мальчик подошёл ко мне в саду — знакомиться. А
я в это время стругал какую-то палку.

Он спросил по-украински, глядя на палку: 

— Що ти робиш?

... Я в том возрасте знал только русские слова, поэтому его не понял и попросил повторить.

— Що ти робиш? — повторил вопрос он.

... По-русски это значило всего лишь «Что ты делаешь?». Но он ведь тоже не ведал,
как выразить эту мысль, кроме как по-украински!...

Он повторял свой вопрос несколько раз, всё громче и громче, думая, что от этого
мне будет понятней. Но я не понимал... Тогда он заплакал и побежал жаловаться на
меня своей маме...

Завершилась эта история тем, что наши родители впервые объяснили нам обоим, что
существуют разные языки. И что те, кто говорят не на моём языке, — вовсе не
являются плохими людьми.

... Попадание в разнообразные ситуации, где обитают люди, которые вначале
воспринимаются как «другие», очень полезно, особенно в раннем возрасте, — для
правильного развития души. Ведь тогда противопоставление себе «чужих» и
«других» исчезает. Человек научается их понимать и любить — такими, как они
есть. Все — исходно — «свои»!

... Хотя научиться подразделять людей на разные категории — надо. Но критерии
такой дифференциации — иные. Это — уровни развития интеллекта и духовной
продвинутости.

* * *

Чем примитивней конкретная социальная среда — тем по более мелким признакам
разделяют и объединяют себя люди в своем самоощущении, а отсюда — и большая
конфликтность в такой среде и тем более агрессивно вынуждены вести себя её
лидеры по отношению к инакомыслящим: им же надо удерживать в повиновении и не
растерять своё «стадо»! Ведь такие лидеры «кормятся» в финансовом отношении от
своих одураченных адептов. Собственные «адепты» при этом могут содержаться в
страхе наказания за попытки выйти за пределы мышления!

Обращу внимание, что все принципы таких примитивов объединены одной глубинной
причиной — эгоцентризмом, то есть, любовью к себе, хотением от других (в т.ч. в
ущерб им) — для себя.

Такой «моральный облик» характерен для многих представителей низких ступеней
интеллектуального развития. Он ведёт их к эволюционной деградации.

Не наделённые же интеллектом люди формируют свои нравственные убеждения на
основе подражания или подчинения авторитетам. Но последние часто оказываются
всего лишь высоко энергичными примитивами...

С детского возраста важно закладывать в поведение духовную этику Бога, а она
основана на противоположных принципах. Бог указывает, что на Пути к Нему нам
надо превращаться в подобие Ему. А Он есть Любовь, Мудрость, Сила и
Утончённость.

Первым качеством, которое Он советует нам осваивать, является многоаспектная
любовь — как нежность, ласковость, благодарность, почтение к достойным того,
забота о других, прощение им, предпочтение интересов других — своим интересам,
жизнь ради блага других, самопожертвенность ради них.

Воспитание нравственности надо прививать с детства.

В детях нужно доброжелательно, но твёрдо пресекать все проявления порочных
тенденций, таких как агрессивность, склонность к присвоению чужих вещей, пусть
хоть самых пустяковых. Дети хорошо запоминают, например, такие формулы: \enquote{Иисус
Христос учил, что нельзя делать другому то, чего не желаешь себе!}, или \enquote{Тот,
кто берёт без разрешения чужое, называется вором. А вор — это очень нехороший
человек!} и т.д.

Так побеждается зло.

Поэтому-то — столь важно знать истину!

Именно знания покажут людям, что это в их личных интересах — быть простыми и
добрыми, умерить личные «земные» желания.

Приблизится к Богу, нельзя, без вмещения именно любви ко всем существам и
людям.

И обретя веру-любовь, человек должен начать взращивать эмоции любви к людям, к
Богу, в независимости от рассы и вероисповедания, а не отстаивать каноны своей
организации.

Бог есть Любовь. Он сострадает нам. Он стремится нам помочь, направляя нас всё
время на Путь к Себе, в Свою Обитель, к нашему конечному счастью

Но мы-то ведь — не идём к Нему! Вместо этого мы «грызёмся» за блага «мира
сего», изнуряем себя ненавистью к другим людям...

\begin{itemize}
\item \emph{Александр Пользователь}
Мудрые родители у этих мальчиков. Да ещё и время нашли на передачу этой
житейской мудрости своим чадам.🙂

\item \emph{Генадий Петрович}
\verb|@Юра Лотоцький|  кстати белорусь та же европа, цивилизованная ругает,
даже украина запретила полёты над территорией, раньше чем та самая европа.
\end{itemize}

\item \emph{Александр Волобуев}

Кто нибудь может объяснить что такое украинец это гражданство, этнос, субэтнос,
что-то ещё?

\item \emph{Petyr Baelish}

Двухстульчатая позиция птенцов гнезда медведчукового. Или ты русский
(национальность, не гражданство!), или украинец (то же самое). Национальность
определяется родным языком и культурой, и точка. А все эти "русскоязычные
украинцы"... ну такое. Сами себя в унтерменши записывают, оправдываются перед
нациками ("ой, вы не так поняли, мы не русские, мы всего лишь русскоязычные"),
а потом ещё чему-то удивляются...

\begin{itemize}
\item \emph{Екатерина Жарких}
Я оправдываюсь по вашему?

\item \emph{R. D.}

Вообще-то это с Совесткого Союза пошло такое. В результате имеем что в каждой
республике бывшего СССР есть разные русскоязычные: 

именно русские - то есть русскоязычные и этнические русские, у которых либо
непосредственно есть родня и предки на территории бывшей РСФСР, либо, как верно
в случае Украины, прожившие много поколений на территории того, что было частью
Новороссийской губернии; 

Есть ещё русскоязычные, которые могут быть этнически хоть поляками,
малороссами, молдаванами, и могут даже не иметь родни среди "коренных россиян",
но принадлежат,  в разной степени в зависимости от поколения и образования, к
тому что осталось от "русско-советской" культуры. Таких мы видим в этом видео; 

Есть русскоязычные в бывших республиках, которые траванулись разными
идеологиями типа евроатлантизма и местечкового национализма, ну и соросята
всякие. Они и есть те, кто с пеной у рта доказывает, что нужно запретить язык,
который родной для половины населения, знакомый большинству населения, и
являющийся по сути lingua franca постсоветского региона.

\end{itemize}

\item \emph{Александр Пользователь}

Катя, чтобы не удивляться, а понимать... общайтесь (читайте) с психологами.
Особенно много будет Вам понятно если хоть чуть-чуть почитаете работы по
социальной психологии.

\end{itemize}
