% vim: keymap=russian-jcukenwin
%%beginhead 
 
%%file 05_04_2021.fb.nashe_prykarpattia.1.kvitka_cisyk_68_let
%%parent 05_04_2021
 
%%url https://www.facebook.com/NashePrykarpattia/posts/3729213987201214
 
%%author 
%%author_id 
%%author_url 
 
%%tags 
%%title 
 
%%endhead 

\subsection{Сьогодні Квітці Цісик виповнилось би 68 років}
\label{sec:05_04_2021.fb.nashe_prykarpattia.1.kvitka_cisyk_68_let}
\Purl{https://www.facebook.com/NashePrykarpattia/posts/3729213987201214}

Квітка Цісик народилася у Нью-Йорку. Її батьки емігрували із Західної України після Другої світової війни. Батько Володимир Цісик походив з-під Коломиї, мав музичну освіту - Львівська консерваторія, Прага, Мюнхен, до війни був головним концертмейстером Львівської опери. 
Мама Іванна Лев належала до роду відомих львівських архітекторів: Квітка доводилася правнучкою Василеві Нагірному, будівничому понад 200 храмів Галичини і засновнику десятків товариств та організацій. 
Сім'я Цісиків підтримувала українські традиції: і Квітка, і її старша сестра Марія відвідували по суботах “Школу українознавства” і були активними учасницями “Пласту”. 
У чотири роки батько почав вчити Квітку грати на скрипці. Втім, коли вона вступала до консерваторії у Нью-Йорку, то обрала таки вокал. Квітка володіла рідкісним - колоратурним сопрано, а також технікою так званого “білого голосу”, властивого народній традиції карпатського регіону, коли голос виходить із глибини, кажуть із самої душі.
Ще студенткою консерваторії, аби оплачувати навчання, оскільки батько помер (просто на сцені у нього стався серцевий напад), почала працювати студійною співачкою. Хоч як згадувала її двоюрідна сестра, якось Квітка зізналась, що боїться виступів перед великою аудиторією, тому і обрала роботу в студії.
Квітка взяла псевдонім Kacey - Кейсі - за першими літерами імені та прізвища K.C. - Kvitka Cisyk. 
В середині 1970-х Квітка почала брати активну участь у комерційних проектах на радіо й телебаченні. Вона озвучувала реклами для таких корпорацій як ABC, NBC, CBS, Burger King, McDonald's, American Airlines та Delta Air Lines, Coca-Cola, Pepsi-Cola, виконувала джингли для Chevrolet, Cadillac, Toyota.
У 1981 вона записала для "Ford Motor" рекламну пісню "Have You Driven a Ford Lately?" Відтоді й аж до смерті вона залишалася єдиним голосом цієї автомобільної компанії. Фірма дарувала їй кожну нову модель машини. 
Квітка стала найдорожчою і найзатребованішою виконавицею рекламних джинглів у США. 
У 1977 вона зіграла епізодичну роль подружки нареченої в комедійній мелодрамі "Ти світло мого життя". Однойменна пісня з цього фільму у виконанні Цісик отримала "Оскар" як найкраща звукова доріжка. Через махінацію нагороду отримала інша виконавиця.
Втім Квітка часто говорила, що хоче зробити щось для України. І записала два альбоми українських пісень у 1980 і 1989. Їй акомпанували три десятки найкращих музикантів Нью-Йорка. 
“Я всім говорила, що українська музика дуже гарна, мелодійна, і що український народ співучий. А мені відповідали: якщо так, зіграй що-небудь, - згадувала. - Це дуже сумно, але в нас (українців діаспори) не було жодної платівки на професійному рівні, оскільки це дорога річ і записати її забирає багато часу. Я зрозуміла, що мушу її зробити. І не тільки для своїх приятелів, але і для всіх українців. Адже у мене не було можливості щось зробити для нашого суспільства, а це був би такий подарунок. І я мріяла, що в Україні мене почують”, - згадувала Квітка.
“Коли я вперше привіз її диск 1990-го з Канади до себе в Карпати, – згадував поет і видавець Іван Малкович, – то сусіди, як почули ”Ніч така, Господи, місячна, зоряна”, просили мене переписати їм платівку. Ми ж не знали, що там має бути це ”Господи”. Один чоловік, який жив на найвищій горі, мав гучномовця і в неділю після церкви пускав голос Квітки на всі найближчі гори. Вона співала ”Я піду в далекі гори”.
Не доживши п'ять днів до свого 45-річчя Квітка померла від раку молочної залози. Мати Іванна Лев та старша сестра Квітки - Марія померли від цієї ж хвороби. 
Пісня у виконанні Квітки для Ford Motors https://www.youtube.com/watch?v=KEKX9iBjWh8
Пісня, яка отримала "Оскара"
https://www.youtube.com/watch?v=5tZvb7C2X7Q
Ніч така, Господи, місячна, зоряна
https://www.youtube.com/watch?v=43EENWJJbNU
Я піду в далекі гори
https://www.youtube.com/watch?v=Yl5cx4PYSoY
Де ти тепер
https://www.youtube.com/watch?v=Ri_QYIV5cMQ


\ifcmt
  pic https://scontent-amt2-1.xx.fbcdn.net/v/t1.6435-9/168934328_3729213710534575_1431712590167913515_n.jpg?_nc_cat=101&ccb=1-3&_nc_sid=8bfeb9&_nc_ohc=5FcaIgfbVPYAX-E8j-Z&_nc_ht=scontent-amt2-1.xx&oh=8641dafd5b90af6e63574a06b1e8b154&oe=608F823F
  width 0.4

	pic https://scontent-amt2-1.xx.fbcdn.net/v/t1.6435-9/168801203_3729213780534568_5853564611441216121_n.jpg?_nc_cat=102&ccb=1-3&_nc_sid=8bfeb9&_nc_ohc=czbjxqjylxQAX8GTUJV&_nc_ht=scontent-amt2-1.xx&oh=581ea56388894bee2e5925ef1a9b5df0&oe=6092342E

	pic https://scontent-ams4-1.xx.fbcdn.net/v/t1.6435-9/168840604_3729213843867895_8921739452570676662_n.jpg?_nc_cat=100&ccb=1-3&_nc_sid=8bfeb9&_nc_ohc=5BC3TpBBv-IAX8IGiJ6&_nc_ht=scontent-ams4-1.xx&oh=0bd7ed5f1f76c0082d56e4f9ca856806&oe=608FE0AF

	pic https://scontent-amt2-1.xx.fbcdn.net/v/t1.6435-9/168801203_3729213907201222_4342465594082148204_n.jpg?_nc_cat=109&ccb=1-3&_nc_sid=8bfeb9&_nc_ohc=X-SBD6FAqbYAX8YCgZJ&_nc_ht=scontent-amt2-1.xx&oh=a2e71e6c8bbdf3150644689939d47af0&oe=609232C3
\fi

