% vim: keymap=russian-jcukenwin
%%beginhead 
 
%%file 28_01_2020.fb.institut_prosvity.1.slovo_mrija_starickij.cmt
%%parent 28_01_2020.fb.institut_prosvity.1.slovo_mrija_starickij
 
%%url 
 
%%author 
%%author_id 
%%author_url 
 
%%tags 
%%title 
 
%%endhead 
\subsubsection{Коментарі}

\begin{itemize}
%%%fbauth
%%%fbauth_name
\iusr{Victor Tregubov}
%%%fbauth_url
%%%fbauth_place
%%%fbauth_id
%%%fbauth_front
%%%fbauth_desc
%%%fbauth_www
%%%fbauth_pic
%%%fbauth_pic portrait
%%%fbauth_pic background
%%%fbauth_pic other
%%%fbauth_tags
%%%fbauth_pubs
%%%endfbauth
 

А які лексеми використовувалися в цих значеннях раніше? Маю на увазі - як до
Старицького називали мрію, а до Пчілки переможця?

\begin{itemize}
%%%fbauth
%%%fbauth_name
\iusr{Oleksander Zaborowski}
%%%fbauth_url
%%%fbauth_place
%%%fbauth_id
%%%fbauth_front
%%%fbauth_desc
%%%fbauth_www
%%%fbauth_pic
%%%fbauth_pic portrait
%%%fbauth_pic background
%%%fbauth_pic other
%%%fbauth_tags
%%%fbauth_pubs
%%%endfbauth
 
\textbf{Victor Tregubov} Я теж про це подумав. Ймовірно,що вони існували в розмовній мові народу,а літературна мова дала їм можливість широкого вжитку.

%%%fbauth
%%%fbauth_name
\iusr{Наталья Федоркова}
%%%fbauth_url
%%%fbauth_place
%%%fbauth_id
%%%fbauth_front
%%%fbauth_desc
%%%fbauth_www
%%%fbauth_pic
%%%fbauth_pic portrait
%%%fbauth_pic background
%%%fbauth_pic other
%%%fbauth_tags
%%%fbauth_pubs
%%%endfbauth
 
\textbf{Victor Tregubov} 

В старих книгах про козацтво не зустрінеш слова "перемога", там є тільки
"Побіда", "побідіти". Жодного разу не бачила слова, наприклад, "місто", є
гОрод, у множинні- городИ. Замість "люди" завжди пишуть "людЕ". ......


%%%fbauth
%%%fbauth_name
\iusr{Тарас Матлак}
%%%fbauth_url
%%%fbauth_place
%%%fbauth_id
%%%fbauth_front
%%%fbauth_desc
%%%fbauth_www
%%%fbauth_pic
%%%fbauth_pic portrait
%%%fbauth_pic background
%%%fbauth_pic other
%%%fbauth_tags
%%%fbauth_pubs
%%%endfbauth
 
\textbf{Victor Tregubov} теж зацікавило.

%%%fbauth
%%%fbauth_name
\iusr{Олег Широкий}
%%%fbauth_url
%%%fbauth_place
%%%fbauth_id
%%%fbauth_front
%%%fbauth_desc
%%%fbauth_www
%%%fbauth_pic
%%%fbauth_pic portrait
%%%fbauth_pic background
%%%fbauth_pic other
%%%fbauth_tags
%%%fbauth_pubs
%%%endfbauth
 
Вочевидь "марити", як у польській.

%%%fbauth
%%%fbauth_name
\iusr{Игорь Коломииец}
%%%fbauth_url
%%%fbauth_place
%%%fbauth_id
%%%fbauth_front
%%%fbauth_desc
%%%fbauth_www
%%%fbauth_pic
%%%fbauth_pic portrait
%%%fbauth_pic background
%%%fbauth_pic other
%%%fbauth_tags
%%%fbauth_pubs
%%%endfbauth
 

От не смешите меня, чтоб у народа не было слова обозначающего 'огненный'. Или
чтобы один человек взял, да и вытеснил более древнее слово на совсем за один
присест

%%%fbauth
%%%fbauth_name
\iusr{Наталья Федоркова}
%%%fbauth_url
%%%fbauth_place
%%%fbauth_id
%%%fbauth_front
%%%fbauth_desc
%%%fbauth_www
%%%fbauth_pic
%%%fbauth_pic portrait
%%%fbauth_pic background
%%%fbauth_pic other
%%%fbauth_tags
%%%fbauth_pubs
%%%endfbauth
 
\textbf{Igor Kolomiiets} ...сказало воно на руцццком....

%%%fbauth
%%%fbauth_name
\iusr{Maryna Byshenko}
%%%fbauth_url
%%%fbauth_place
%%%fbauth_id
%%%fbauth_front
%%%fbauth_desc
%%%fbauth_www
%%%fbauth_pic
%%%fbauth_pic portrait
%%%fbauth_pic background
%%%fbauth_pic other
%%%fbauth_tags
%%%fbauth_pubs
%%%endfbauth
 
\textbf{Igor Kolomiiets} ой, пане, Вам ще до того довго, щоб розуміти ти аспекти. Для початку спробуйте писати українською, тоді потроху почнете відкривати для себе цілий світ, а поки Ви будете дивуватися і не розуміти...

%%%fbauth
%%%fbauth_name
\iusr{Игорь Гаврилов}
%%%fbauth_url
%%%fbauth_place
%%%fbauth_id
%%%fbauth_front
%%%fbauth_desc
%%%fbauth_www
%%%fbauth_pic
%%%fbauth_pic portrait
%%%fbauth_pic background
%%%fbauth_pic other
%%%fbauth_tags
%%%fbauth_pubs
%%%endfbauth
 
\textbf{Наталья Федоркова} В старих книгах писали староболгарським текстом... це так, меж iншим ...))

%%%fbauth
%%%fbauth_name
\iusr{Ella Yevtushenko}
%%%fbauth_url
%%%fbauth_place
%%%fbauth_id
%%%fbauth_front
%%%fbauth_desc
%%%fbauth_www
%%%fbauth_pic
%%%fbauth_pic portrait
%%%fbauth_pic background
%%%fbauth_pic other
%%%fbauth_tags
%%%fbauth_pubs
%%%endfbauth
 
\textbf{Igor Kolomiiets} "огненний", "полум'яний", наприклад. ото вже точно "не бійтесь заглядати у словник"))

%%%fbauth
%%%fbauth_name
\iusr{Василина Михайлівна}
%%%fbauth_url
%%%fbauth_place
%%%fbauth_id
%%%fbauth_front
%%%fbauth_desc
%%%fbauth_www
%%%fbauth_pic
%%%fbauth_pic portrait
%%%fbauth_pic background
%%%fbauth_pic other
%%%fbauth_tags
%%%fbauth_pubs
%%%endfbauth
 
\textbf{Igor Kolomiiets} 

думаю, це відбувається протягом довгого часу, з літературних творів слова
потраплять у словники, звідти у мовлення... Напр., колись було слово у нас
підсоння. А тепер клімат. 🙄 Різні чинники впливають на життя слів.


%%%fbauth
%%%fbauth_name
\iusr{Lucy Shnyr}
%%%fbauth_url
%%%fbauth_place
%%%fbauth_id
%%%fbauth_front
%%%fbauth_desc
%%%fbauth_www
%%%fbauth_pic
%%%fbauth_pic portrait
%%%fbauth_pic background
%%%fbauth_pic other
%%%fbauth_tags
%%%fbauth_pubs
%%%endfbauth
 
\textbf{Наталья Федоркова}

👍 В тому-то і справа, що українська містить в собі слова, які зараз можуть
комусь здатись "калькою" з російської! Це величезний скарб і можливість для
креативного підходу \Smiley[1.0][yellow]

Є "гарний", а є "красивий", є "добрий", а є "хороший", є "зміст", а є "смисл" і
т. д. і т. п. Аа, ще є не зовсім літературне, але обожнюване мною "лучче" як
еквівалент "краще" 😉

Тому і "побіда" була легитимною, хоча зараз вже, на жаль, так не скажеш.

Також не впевнена, що названі письменники "створили" вищеназвані слова - радше
популяризували, закріпили.

P. S. А от про "мрію" так ніхто і не сказав - як говорили до того?! Бо ж не
мріяти не могли! 😉

%%%fbauth
%%%fbauth_name
\iusr{Sergiy Michael}
%%%fbauth_url
%%%fbauth_place
%%%fbauth_id
%%%fbauth_front
%%%fbauth_desc
%%%fbauth_www
%%%fbauth_pic
%%%fbauth_pic portrait
%%%fbauth_pic background
%%%fbauth_pic other
%%%fbauth_tags
%%%fbauth_pubs
%%%endfbauth
 
\textbf{Lucy Shnyr} мріта

%%%fbauth
%%%fbauth_name
\iusr{Надія Лукащук}
%%%fbauth_url
%%%fbauth_place
%%%fbauth_id
%%%fbauth_front
%%%fbauth_desc
%%%fbauth_www
%%%fbauth_pic
%%%fbauth_pic portrait
%%%fbauth_pic background
%%%fbauth_pic other
%%%fbauth_tags
%%%fbauth_pubs
%%%endfbauth
 
\textbf{Igor Kolomiiets} Поки що ,,смішить... "і не тільки мене ,, великий знавець української мови" Ігор Коломієць

%%%fbauth
%%%fbauth_name
\iusr{Ігор Сокорчук}
%%%fbauth_url
%%%fbauth_place
%%%fbauth_id
%%%fbauth_front
%%%fbauth_desc
%%%fbauth_www
%%%fbauth_pic
%%%fbauth_pic portrait
%%%fbauth_pic background
%%%fbauth_pic other
%%%fbauth_tags
%%%fbauth_pubs
%%%endfbauth
 
Звитяга.

%%%fbauth
%%%fbauth_name
\iusr{Наталія Процко}
%%%fbauth_url
%%%fbauth_place
%%%fbauth_id
%%%fbauth_front
%%%fbauth_desc
%%%fbauth_www
%%%fbauth_pic
%%%fbauth_pic portrait
%%%fbauth_pic background
%%%fbauth_pic other
%%%fbauth_tags
%%%fbauth_pubs
%%%endfbauth
 

@Наталья Федоркова, а старі книги про козацтво написані церковнослов’янською?
Наскільки я розумію, це не була вже мова народу? Чи була?

Бо в нашому селі (Кіровоградщина) завжди говорили «поїхати в гОрод» - «місто»
ніхто не вживав.

%%%fbauth
%%%fbauth_name
\iusr{Julia Trifonova}
%%%fbauth_url
%%%fbauth_place
%%%fbauth_id
%%%fbauth_front
%%%fbauth_desc
%%%fbauth_www
%%%fbauth_pic
%%%fbauth_pic portrait
%%%fbauth_pic background
%%%fbauth_pic other
%%%fbauth_tags
%%%fbauth_pubs
%%%endfbauth
 
Дурень думкой богатіє... Стус приніс українцям атом, а Бандера гомонацюцюник.. Скаженіли зовсім.

%%%fbauth
%%%fbauth_name
\iusr{Julia Trifonova}
%%%fbauth_url
%%%fbauth_place
%%%fbauth_id
%%%fbauth_front
%%%fbauth_desc
%%%fbauth_www
%%%fbauth_pic
%%%fbauth_pic portrait
%%%fbauth_pic background
%%%fbauth_pic other
%%%fbauth_tags
%%%fbauth_pubs
%%%endfbauth
 
\textbf{Victor Tregubov} ,не було до Старицького в нас мрій, а до Олени і перемог теж...

%%%fbauth
%%%fbauth_name
\iusr{Митя Спирин}
%%%fbauth_url
%%%fbauth_place
%%%fbauth_id
%%%fbauth_front
%%%fbauth_desc
%%%fbauth_www
%%%fbauth_pic
%%%fbauth_pic portrait
%%%fbauth_pic background
%%%fbauth_pic other
%%%fbauth_tags
%%%fbauth_pubs
%%%endfbauth
 
\textbf{Олег Широкий} да, но это больше "бредить"

%%%fbauth
%%%fbauth_name
\iusr{Митя Спирин}
%%%fbauth_url
%%%fbauth_place
%%%fbauth_id
%%%fbauth_front
%%%fbauth_desc
%%%fbauth_www
%%%fbauth_pic
%%%fbauth_pic portrait
%%%fbauth_pic background
%%%fbauth_pic other
%%%fbauth_tags
%%%fbauth_pubs
%%%endfbauth
 
\textbf{Julia Trifonova} тай зараз нэма...

%%%fbauth
%%%fbauth_name
\iusr{Oleck Veremko-Berezhny}
%%%fbauth_url
%%%fbauth_place
%%%fbauth_id
%%%fbauth_front
%%%fbauth_desc
%%%fbauth_www
%%%fbauth_pic
%%%fbauth_pic portrait
%%%fbauth_pic background
%%%fbauth_pic other
%%%fbauth_tags
%%%fbauth_pubs
%%%endfbauth
 
"Побідник" був розповсюджений ще довго, та й ниньки трапляється.

\end{itemize}

%%%fbauth
%%%fbauth_name
\iusr{Yura Tykhyy}
%%%fbauth_url
%%%fbauth_place
%%%fbauth_id
%%%fbauth_front
%%%fbauth_desc
%%%fbauth_www
%%%fbauth_pic
%%%fbauth_pic portrait
%%%fbauth_pic background
%%%fbauth_pic other
%%%fbauth_tags
%%%fbauth_pubs
%%%endfbauth
 

Гм.

Декотрі слова придумували двічі?

Промінь і променистий - це одне і те саме.

Але перше придумала Леся Українка, а друге - її матір?

\begin{itemize}
%%%fbauth
%%%fbauth_name
\iusr{Vyacheslav Toporov}
%%%fbauth_url
%%%fbauth_place
%%%fbauth_id
%%%fbauth_front
%%%fbauth_desc
%%%fbauth_www
%%%fbauth_pic
%%%fbauth_pic portrait
%%%fbauth_pic background
%%%fbauth_pic other
%%%fbauth_tags
%%%fbauth_pubs
%%%endfbauth
 
Більш того, переможець придумала Леся, перемагати - Квітка-Основ’яненко

%%%fbauth
%%%fbauth_name
\iusr{Ella Yevtushenko}
%%%fbauth_url
%%%fbauth_place
%%%fbauth_id
%%%fbauth_front
%%%fbauth_desc
%%%fbauth_www
%%%fbauth_pic
%%%fbauth_pic portrait
%%%fbauth_pic background
%%%fbauth_pic other
%%%fbauth_tags
%%%fbauth_pubs
%%%endfbauth
 

"промінь" і "променистий" - це два однокореневі, але різні слова. якщо Пчілка
взяла слово "промінь" і зробила з нього прикметник, то це так і називається -
створити нове слово

\end{itemize}

%%%fbauth
%%%fbauth_name
\iusr{Andrii Kusii}
%%%fbauth_url
%%%fbauth_place
%%%fbauth_id
%%%fbauth_front
%%%fbauth_desc
%%%fbauth_www
%%%fbauth_pic
%%%fbauth_pic portrait
%%%fbauth_pic background
%%%fbauth_pic other
%%%fbauth_tags
%%%fbauth_pubs
%%%endfbauth
 

Стосовно слова "лелека". Дивним виглядає твердження щодо питомості його в нашій
мові, оскільки в турецькій мові на цю птаху є слово "лейлек". А в арабській
"лякайляк".

\begin{itemize}
%%%fbauth
%%%fbauth_name
\iusr{Наталья Федоркова}
%%%fbauth_url
%%%fbauth_place
%%%fbauth_id
%%%fbauth_front
%%%fbauth_desc
%%%fbauth_www
%%%fbauth_pic
%%%fbauth_pic portrait
%%%fbauth_pic background
%%%fbauth_pic other
%%%fbauth_tags
%%%fbauth_pubs
%%%endfbauth
 
\textbf{Andriy Kusiy} 

Лелека походить від назви стародавнього бога Леля, жоною якого була Лада -
богиня материнства. Тому птах, що приносив діточок, - лелека.


%%%fbauth
%%%fbauth_name
\iusr{Наталья Федоркова}
%%%fbauth_url
%%%fbauth_place
%%%fbauth_id
%%%fbauth_front
%%%fbauth_desc
%%%fbauth_www
%%%fbauth_pic
%%%fbauth_pic portrait
%%%fbauth_pic background
%%%fbauth_pic other
%%%fbauth_tags
%%%fbauth_pubs
%%%endfbauth
 

Доречі, в старих книжках використовують саме "жона", а не "дружина", як зараз.
Дружини були тільки військові, над якими були "СтаршИни".


%%%fbauth
%%%fbauth_name
\iusr{Олександр Малярчук}
%%%fbauth_url
%%%fbauth_place
%%%fbauth_id
%%%fbauth_front
%%%fbauth_desc
%%%fbauth_www
%%%fbauth_pic
%%%fbauth_pic portrait
%%%fbauth_pic background
%%%fbauth_pic other
%%%fbauth_tags
%%%fbauth_pubs
%%%endfbauth
 
\textbf{Наталья Федоркова} 

швидше базовим терміном був "бузько/бусел". В 5-х слов'янських мовах він -
бусел, bocian, boson. Болгари і македонці чомусь гермашське шторк запозичили, в
балканців він - рода, в чехів - чап (чапля).


%%%fbauth
%%%fbauth_name
\iusr{Lilia Bezbah}
%%%fbauth_url
%%%fbauth_place
%%%fbauth_id
%%%fbauth_front
%%%fbauth_desc
%%%fbauth_www
%%%fbauth_pic
%%%fbauth_pic portrait
%%%fbauth_pic background
%%%fbauth_pic other
%%%fbauth_tags
%%%fbauth_pubs
%%%endfbauth
 
\textbf{Олександр Малярчук} 

читала раніше, що , коли птахи відлітали, іх називали лелеками чи журавлями. Бо,
відлітаючи, журилися. Коли верталися- веселиками. Бусел кажуть не по всій
Украіні.

%%%fbauth
%%%fbauth_name
\iusr{Людмила Гаценко}
%%%fbauth_url
%%%fbauth_place
%%%fbauth_id
%%%fbauth_front
%%%fbauth_desc
%%%fbauth_www
%%%fbauth_pic
%%%fbauth_pic portrait
%%%fbauth_pic background
%%%fbauth_pic other
%%%fbauth_tags
%%%fbauth_pubs
%%%endfbauth
 
\textbf{Andriy Kusiy} 

питомі імена якої мови Іван, Марія, Федір, Ганна? Вся справа у фонетичному
сприйнятті різних слів різними народами. Походження цих імен відоме, вони
прийшли до різних народів через християнство, однак адаптовані були згідно до
вимови певних народів. Слова ж, які згадуєте ви, виникали на своїх теренах.


%%%fbauth
%%%fbauth_name
\iusr{Людмила Гаценко}
%%%fbauth_url
%%%fbauth_place
%%%fbauth_id
%%%fbauth_front
%%%fbauth_desc
%%%fbauth_www
%%%fbauth_pic
%%%fbauth_pic portrait
%%%fbauth_pic background
%%%fbauth_pic other
%%%fbauth_tags
%%%fbauth_pubs
%%%endfbauth
 
Послухайте, як звучить пісня цього птаха, а народи, дотично до своєї фонетики, кожен окремо від іншого, дали ймення птасі.

%%%fbauth
%%%fbauth_name
\iusr{Наталья Федоркова}
%%%fbauth_url
%%%fbauth_place
%%%fbauth_id
%%%fbauth_front
%%%fbauth_desc
%%%fbauth_www
%%%fbauth_pic
%%%fbauth_pic portrait
%%%fbauth_pic background
%%%fbauth_pic other
%%%fbauth_tags
%%%fbauth_pubs
%%%endfbauth
 
\textbf{Олександр Малярчук} чому ви орієнтуєтесь на інші народи-мови-традиції? Чому не на українське?

%%%fbauth
%%%fbauth_name
\iusr{Олександр Малярчук}
%%%fbauth_url
%%%fbauth_place
%%%fbauth_id
%%%fbauth_front
%%%fbauth_desc
%%%fbauth_www
%%%fbauth_pic
%%%fbauth_pic portrait
%%%fbauth_pic background
%%%fbauth_pic other
%%%fbauth_tags
%%%fbauth_pubs
%%%endfbauth
 
\textbf{Наталья Федоркова} 

тому що потрібно об'єктивно оцінювати термін, враховуючи й інші мови. Це
конкретне слово - лелека - 100\% запозичене. Турецьке leylek, аварське лакълакъ,
казахське дегелек, азербайджанське leylək, киргизське илегилек і узбекське
loyqa (думаю, що корінь ви впізнаєте) точно вказують, до якої саме мовної сім'ї
належить той конкретний термін. Тоді як те ж саме бузько/бусел/боцьон поширене
серед половини слов'ян (умовно).


%%%fbauth
%%%fbauth_name
\iusr{Наталья Федоркова}
%%%fbauth_url
%%%fbauth_place
%%%fbauth_id
%%%fbauth_front
%%%fbauth_desc
%%%fbauth_www
%%%fbauth_pic
%%%fbauth_pic portrait
%%%fbauth_pic background
%%%fbauth_pic other
%%%fbauth_tags
%%%fbauth_pubs
%%%endfbauth
 
\textbf{Олександр Малярчук} 

Орда вам в поміч... Дивно, як охоче люди відцураються від свого. А ви в курсі,
що санскрит є нашою прамовою, коли наші предки дійшли до Індіі та заснували там
великі міста?Що і досі є багато слів зі спільним коренем?Чи для вас головне
принизити своє?

%%%fbauth
%%%fbauth_name
\iusr{Олександр Малярчук}
%%%fbauth_url
%%%fbauth_place
%%%fbauth_id
%%%fbauth_front
%%%fbauth_desc
%%%fbauth_www
%%%fbauth_pic
%%%fbauth_pic portrait
%%%fbauth_pic background
%%%fbauth_pic other
%%%fbauth_tags
%%%fbauth_pubs
%%%endfbauth
 
\textbf{Наталья Федоркова} 

у вас якась неадекватна реакція. В будь-якій мові є купа слів, запозичених в
сусідів, це норма взагалі-то.

Наші безпосередні предки, до речі, в Індію не ходили і міста там не
засновували. Східне відгалуження індо-європейських племен (індо-перси), умовні
двоюрідні брати (якщо аналогії проводити) наших предків, що вторглись в Індію
приблизно 3600 років тому, прийшли туди як варвари-завойовники, захопивши вже
існуючі міста Харапської цивілізації (дравідів). Плюс-мінус аналогічно після
того варвари-дорійці завоювали ахейську Грецію, а ще пізніше германці Римську
імперію завоювали - принцип тут однаковий.

\end{itemize}

%%%fbauth
%%%fbauth_name
\iusr{Александр Резниченко}
%%%fbauth_url
%%%fbauth_place
%%%fbauth_id
%%%fbauth_front
%%%fbauth_desc
%%%fbauth_www
%%%fbauth_pic
%%%fbauth_pic portrait
%%%fbauth_pic background
%%%fbauth_pic other
%%%fbauth_tags
%%%fbauth_pubs
%%%endfbauth
 

У свій час, писав магістерську дисертацію "Створення неологізмів у творах
українських письменників першої половини хх століття".М. Старицький , О. Пчілка
та інші були творцями багатьох нових слів які ми вживаємо сьогодні.... Інколи, у
звичайних речах ми не помічаємо мистецтва!

\begin{itemize}
%%%fbauth
%%%fbauth_name
\iusr{Віктор Рибачок}
%%%fbauth_url
%%%fbauth_place
%%%fbauth_id
%%%fbauth_front
%%%fbauth_desc
%%%fbauth_www
%%%fbauth_pic
%%%fbauth_pic portrait
%%%fbauth_pic background
%%%fbauth_pic other
%%%fbauth_tags
%%%fbauth_pubs
%%%endfbauth
 
\textbf{Александр Резниченко} а посиланнячко на вашу працю можна, аби почитати.

%%%fbauth
%%%fbauth_name
\iusr{Александр Резниченко}
%%%fbauth_url
%%%fbauth_place
%%%fbauth_id
%%%fbauth_front
%%%fbauth_desc
%%%fbauth_www
%%%fbauth_pic
%%%fbauth_pic portrait
%%%fbauth_pic background
%%%fbauth_pic other
%%%fbauth_tags
%%%fbauth_pubs
%%%endfbauth
 
\textbf{Віктор Рибачок} писав ще у 2000 році. Після захисту, здав роботу на кафедру. Навіть не знаю чи є вона зараз у наявності.

%%%fbauth
%%%fbauth_name
\iusr{Ольга Орещук}
%%%fbauth_url
%%%fbauth_place
%%%fbauth_id
%%%fbauth_front
%%%fbauth_desc
%%%fbauth_www
%%%fbauth_pic
%%%fbauth_pic portrait
%%%fbauth_pic background
%%%fbauth_pic other
%%%fbauth_tags
%%%fbauth_pubs
%%%endfbauth
 
\textbf{Александр Резниченко} Дякую за інформацію. Саме ці патріоти і їх духовні нащадки дбали про красу і розвиток мови

%%%fbauth
%%%fbauth_name
\iusr{Ольга Орещук}
%%%fbauth_url
%%%fbauth_place
%%%fbauth_id
%%%fbauth_front
%%%fbauth_desc
%%%fbauth_www
%%%fbauth_pic
%%%fbauth_pic portrait
%%%fbauth_pic background
%%%fbauth_pic other
%%%fbauth_tags
%%%fbauth_pubs
%%%endfbauth
 
\textbf{Александр Резниченко} з'ясовуйте друкуйте

%%%fbauth
%%%fbauth_name
\iusr{Ольга Орещук}
%%%fbauth_url
%%%fbauth_place
%%%fbauth_id
%%%fbauth_front
%%%fbauth_desc
%%%fbauth_www
%%%fbauth_pic
%%%fbauth_pic portrait
%%%fbauth_pic background
%%%fbauth_pic other
%%%fbauth_tags
%%%fbauth_pubs
%%%endfbauth
 
👏

%%%fbauth
%%%fbauth_name
\iusr{Lucy Shnyr}
%%%fbauth_url
%%%fbauth_place
%%%fbauth_id
%%%fbauth_front
%%%fbauth_desc
%%%fbauth_www
%%%fbauth_pic
%%%fbauth_pic portrait
%%%fbauth_pic background
%%%fbauth_pic other
%%%fbauth_tags
%%%fbauth_pubs
%%%endfbauth
 
\textbf{Oleksandr Riznichenko} ...але, що цікаво, так ніхто і не відповів, як "мрію" називали!! Мало ж бути якесь слово?...

%%%fbauth
%%%fbauth_name
\iusr{Александр Резниченко}
%%%fbauth_url
%%%fbauth_place
%%%fbauth_id
%%%fbauth_front
%%%fbauth_desc
%%%fbauth_www
%%%fbauth_pic
%%%fbauth_pic portrait
%%%fbauth_pic background
%%%fbauth_pic other
%%%fbauth_tags
%%%fbauth_pubs
%%%endfbauth
 
\textbf{Lucy Shnyr}. Дуже цікаво...

%%%fbauth
%%%fbauth_name
\iusr{Александр Резниченко}
%%%fbauth_url
%%%fbauth_place
%%%fbauth_id
%%%fbauth_front
%%%fbauth_desc
%%%fbauth_www
%%%fbauth_pic
%%%fbauth_pic portrait
%%%fbauth_pic background
%%%fbauth_pic other
%%%fbauth_tags
%%%fbauth_pubs
%%%endfbauth
 
\textbf{Lucy Shnyr} 🤔🤔🤔🤔

%%%fbauth
%%%fbauth_name
\iusr{Svitlana Svitlana}
%%%fbauth_url
%%%fbauth_place
%%%fbauth_id
%%%fbauth_front
%%%fbauth_desc
%%%fbauth_www
%%%fbauth_pic
%%%fbauth_pic portrait
%%%fbauth_pic background
%%%fbauth_pic other
%%%fbauth_tags
%%%fbauth_pubs
%%%endfbauth
 
\textbf{Lucy Shnyr} так ніби відповіли - марення... Чи я не так зрозуміла?

%%%fbauth
%%%fbauth_name
\iusr{Олена Даценко}
%%%fbauth_url
%%%fbauth_place
%%%fbauth_id
%%%fbauth_front
%%%fbauth_desc
%%%fbauth_www
%%%fbauth_pic
%%%fbauth_pic portrait
%%%fbauth_pic background
%%%fbauth_pic other
%%%fbauth_tags
%%%fbauth_pubs
%%%endfbauth
 
\textbf{Lucy Shnyr} мріта, писали ж вже вам
\end{itemize}

%%%fbauth
%%%fbauth_name
\iusr{Олексій Панібудьласка}
%%%fbauth_url
%%%fbauth_place
%%%fbauth_id
%%%fbauth_front
%%%fbauth_desc
%%%fbauth_www
%%%fbauth_pic
%%%fbauth_pic portrait
%%%fbauth_pic background
%%%fbauth_pic other
%%%fbauth_tags
%%%fbauth_pubs
%%%endfbauth
 
лелека запозичення з тюркських мов. в сучасній турецький лелека - leylek

\begin{itemize}
%%%fbauth
%%%fbauth_name
\iusr{Роман Посікіра}
%%%fbauth_url
%%%fbauth_place
%%%fbauth_id
%%%fbauth_front
%%%fbauth_desc
%%%fbauth_www
%%%fbauth_pic
%%%fbauth_pic portrait
%%%fbauth_pic background
%%%fbauth_pic other
%%%fbauth_tags
%%%fbauth_pubs
%%%endfbauth
 
\textbf{Олексій Панібудьласка} а може це тюрки запозичили?)

%%%fbauth
%%%fbauth_name
\iusr{Олексій Панібудьласка}
%%%fbauth_url
%%%fbauth_place
%%%fbauth_id
%%%fbauth_front
%%%fbauth_desc
%%%fbauth_www
%%%fbauth_pic
%%%fbauth_pic portrait
%%%fbauth_pic background
%%%fbauth_pic other
%%%fbauth_tags
%%%fbauth_pubs
%%%endfbauth
 
ні. я не побоявся заглянути у словник

%%%fbauth
%%%fbauth_name
\iusr{Игорь Гаврилов}
%%%fbauth_url
%%%fbauth_place
%%%fbauth_id
%%%fbauth_front
%%%fbauth_desc
%%%fbauth_www
%%%fbauth_pic
%%%fbauth_pic portrait
%%%fbauth_pic background
%%%fbauth_pic other
%%%fbauth_tags
%%%fbauth_pubs
%%%endfbauth
 
\textbf{Роман Посікіра} Нi... це ви поплутали - це слово козак запозичили у тюрков!
\end{itemize}

%%%fbauth
%%%fbauth_name
\iusr{Roman Kurach}
%%%fbauth_url
%%%fbauth_place
%%%fbauth_id
%%%fbauth_front
%%%fbauth_desc
%%%fbauth_www
%%%fbauth_pic
%%%fbauth_pic portrait
%%%fbauth_pic background
%%%fbauth_pic other
%%%fbauth_tags
%%%fbauth_pubs
%%%endfbauth
 
От якого слова скоріш за все дійсно нема - так то "непереливки"

%%%fbauth
%%%fbauth_name
\iusr{Юрий Устиновский}
%%%fbauth_url
%%%fbauth_place
%%%fbauth_id
%%%fbauth_front
%%%fbauth_desc
%%%fbauth_www
%%%fbauth_pic
%%%fbauth_pic portrait
%%%fbauth_pic background
%%%fbauth_pic other
%%%fbauth_tags
%%%fbauth_pubs
%%%endfbauth
 

Гамуй, гамувати вживав ще Мілетій Смотрицький у 1616 році більш як за 2
століття до Котляревського, тож корені багатьох "неологізмів" насправді
набагато древніші і є природними давньоукраїнськими словами.


%%%fbauth
%%%fbauth_name
\iusr{Boris Mamin}
%%%fbauth_url
%%%fbauth_place
%%%fbauth_id
%%%fbauth_front
%%%fbauth_desc
%%%fbauth_www
%%%fbauth_pic
%%%fbauth_pic portrait
%%%fbauth_pic background
%%%fbauth_pic other
%%%fbauth_tags
%%%fbauth_pubs
%%%endfbauth
 

Я теж придумав неологізм- СОЦИДЕСЯТІ роки. Маю право. чи не так?...

Щиро бажаю успіхів у вашій праці , шановний Інститут Просвіти. Д🤗уже добре ,
що ви таки НПО!


%%%fbauth
%%%fbauth_name
\iusr{Алла Волкова}
%%%fbauth_url
%%%fbauth_place
%%%fbauth_id
%%%fbauth_front
%%%fbauth_desc
%%%fbauth_www
%%%fbauth_pic
%%%fbauth_pic portrait
%%%fbauth_pic background
%%%fbauth_pic other
%%%fbauth_tags
%%%fbauth_pubs
%%%endfbauth
 
Цікава дискусія. Пізнавальна. Не хотілося б сюди домішувати політику.

\ifcmt
  ig https://scontent-cdg2-1.xx.fbcdn.net/v/t39.1997-6/s168x128/851586_126362104215253_1651254063_n.png?_nc_cat=1&ccb=1-3&_nc_sid=ac3552&_nc_ohc=0zUqry8GqbIAX_gRy3n&_nc_ht=scontent-cdg2-1.xx&oh=c58860b71204a5a5b4378251bda5a561&oe=610BFB45
  width 0.2
\fi

%%%fbauth
%%%fbauth_name
\iusr{Tymofij Lviv}
%%%fbauth_url
%%%fbauth_place
%%%fbauth_id
%%%fbauth_front
%%%fbauth_desc
%%%fbauth_www
%%%fbauth_pic
%%%fbauth_pic portrait
%%%fbauth_pic background
%%%fbauth_pic other
%%%fbauth_tags
%%%fbauth_pubs
%%%endfbauth
 
Цікаво дізнатися, які слова вживали українці для цих значень до того, як ЦІ слова були вигадані ❓

%%%fbauth
%%%fbauth_name
\iusr{Божена Антоняк}
%%%fbauth_url
%%%fbauth_place
%%%fbauth_id
%%%fbauth_front
%%%fbauth_desc
%%%fbauth_www
%%%fbauth_pic
%%%fbauth_pic portrait
%%%fbauth_pic background
%%%fbauth_pic other
%%%fbauth_tags
%%%fbauth_pubs
%%%endfbauth
 
Тільки і napawać i samotność є в польській.

%%%fbauth
%%%fbauth_name
\iusr{Валерій Томашевський}
%%%fbauth_url
%%%fbauth_place
%%%fbauth_id
%%%fbauth_front
%%%fbauth_desc
%%%fbauth_www
%%%fbauth_pic
%%%fbauth_pic portrait
%%%fbauth_pic background
%%%fbauth_pic other
%%%fbauth_tags
%%%fbauth_pubs
%%%endfbauth
 

Типічне образованіє замість того щоб одержати освіту з вас зліпили образ от
ваші висновки як російська пропаганда нічим не підтверджена марнопрофесорство
якесь!!


%%%fbauth
%%%fbauth_name
\iusr{Nick Coll}
%%%fbauth_url
%%%fbauth_place
%%%fbauth_id
%%%fbauth_front
%%%fbauth_desc
%%%fbauth_www
%%%fbauth_pic
%%%fbauth_pic portrait
%%%fbauth_pic background
%%%fbauth_pic other
%%%fbauth_tags
%%%fbauth_pubs
%%%endfbauth
 

Мені цікаво, а як же тоді говорили ДО Старицького для позначення поняття
"мрії"?


\begin{itemize}
%%%fbauth
%%%fbauth_name
\iusr{Любов Прокопець}
%%%fbauth_url
%%%fbauth_place
%%%fbauth_id
%%%fbauth_front
%%%fbauth_desc
%%%fbauth_www
%%%fbauth_pic
%%%fbauth_pic portrait
%%%fbauth_pic background
%%%fbauth_pic other
%%%fbauth_tags
%%%fbauth_pubs
%%%endfbauth
 
\textbf{Nick Coll} мріта
\end{itemize}

%%%fbauth
%%%fbauth_name
\iusr{Lucy Shnyr}
%%%fbauth_url
%%%fbauth_place
%%%fbauth_id
%%%fbauth_front
%%%fbauth_desc
%%%fbauth_www
%%%fbauth_pic
%%%fbauth_pic portrait
%%%fbauth_pic background
%%%fbauth_pic other
%%%fbauth_tags
%%%fbauth_pubs
%%%endfbauth
 

В тому-то і справа, що українська містить в собі слова, які зараз можуть комусь
здатись "калькою" з російської! Це величезний скарб і можливість для
креативного підходу \Smiley[1.0][yellow]

Є "гарний", а є "красивий", є "добрий", а є "хороший", є "зміст", а є "смисл" і
т. д. і т. п. Аа, ще є не зовсім літературне, але обожнюване мною "лучче" як
еквівалент "краще" 😉

Тому і "побіда" була легитимною, хоча зараз вже, на жаль, так не скажеш.

Також не впевнена, що названі письменники "створили" вищеназвані слова - радше
популяризували, закріпили.

P. S. А от про "мрію" так ніхто і не сказав - як говорили до того?! Бо ж не
мріяти не могли! 😉

\begin{itemize}
%%%fbauth
%%%fbauth_name
\iusr{Oksana Hauk Myers}
%%%fbauth_url
%%%fbauth_place
%%%fbauth_id
%%%fbauth_front
%%%fbauth_desc
%%%fbauth_www
%%%fbauth_pic
%%%fbauth_pic portrait
%%%fbauth_pic background
%%%fbauth_pic other
%%%fbauth_tags
%%%fbauth_pubs
%%%endfbauth
 
\textbf{Lucy Shnyr} що ще за «лучче», звучить як сарказм на суржик.

%%%fbauth
%%%fbauth_name
\iusr{Наталія Пилипенко}
%%%fbauth_url
%%%fbauth_place
%%%fbauth_id
%%%fbauth_front
%%%fbauth_desc
%%%fbauth_www
%%%fbauth_pic
%%%fbauth_pic portrait
%%%fbauth_pic background
%%%fbauth_pic other
%%%fbauth_tags
%%%fbauth_pubs
%%%endfbauth
 
\textbf{Oksana Hauk Myers} є так слово. "Лучче було, лучче було б не ходити"

%%%fbauth
%%%fbauth_name
\iusr{Митя Спирин}
%%%fbauth_url
%%%fbauth_place
%%%fbauth_id
%%%fbauth_front
%%%fbauth_desc
%%%fbauth_www
%%%fbauth_pic
%%%fbauth_pic portrait
%%%fbauth_pic background
%%%fbauth_pic other
%%%fbauth_tags
%%%fbauth_pubs
%%%endfbauth
 
\textbf{Lucy Shnyr} а до "мрii" говОрили "мiчта"🤣🤣🤣

%%%fbauth
%%%fbauth_name
\iusr{Ірина Геча}
%%%fbauth_url
%%%fbauth_place
%%%fbauth_id
%%%fbauth_front
%%%fbauth_desc
%%%fbauth_www
%%%fbauth_pic
%%%fbauth_pic portrait
%%%fbauth_pic background
%%%fbauth_pic other
%%%fbauth_tags
%%%fbauth_pubs
%%%endfbauth
 
\textbf{Oksana Hauk Myers} на Полтавщині в Чорнухинському районі в селі Ковалі вперше почула від бабусі ,,лучче''

%%%fbauth
%%%fbauth_name
\iusr{Lucy Shnyr}
%%%fbauth_url
%%%fbauth_place
%%%fbauth_id
%%%fbauth_front
%%%fbauth_desc
%%%fbauth_www
%%%fbauth_pic
%%%fbauth_pic portrait
%%%fbauth_pic background
%%%fbauth_pic other
%%%fbauth_tags
%%%fbauth_pubs
%%%endfbauth
 
\textbf{Dimitri Spirin} 

та ні, на відміну від діалектно-застарілого, але милого "лучче" ота "мічта" -
це вже якось не тойво, як кажуть у нас у Львові... то вже дійсно суржик, мені
здається.

Але на питання моє так ніхто і не відповів!! 😉


%%%fbauth
%%%fbauth_name
\iusr{Митя Спирин}
%%%fbauth_url
%%%fbauth_place
%%%fbauth_id
%%%fbauth_front
%%%fbauth_desc
%%%fbauth_www
%%%fbauth_pic
%%%fbauth_pic portrait
%%%fbauth_pic background
%%%fbauth_pic other
%%%fbauth_tags
%%%fbauth_pubs
%%%endfbauth
 
\textbf{Lucy Shnyr} а менi нравiцца🤣🤣🤣

%%%fbauth
%%%fbauth_name
\iusr{Петро Ілик}
%%%fbauth_url
%%%fbauth_place
%%%fbauth_id
%%%fbauth_front
%%%fbauth_desc
%%%fbauth_www
%%%fbauth_pic
%%%fbauth_pic portrait
%%%fbauth_pic background
%%%fbauth_pic other
%%%fbauth_tags
%%%fbauth_pubs
%%%endfbauth
 
Дивлюсь я на небо та й думку гадаю...
\end{itemize}

%%%fbauth
%%%fbauth_name
\iusr{Сергей Таранец}
%%%fbauth_url
%%%fbauth_place
%%%fbauth_id
%%%fbauth_front
%%%fbauth_desc
%%%fbauth_www
%%%fbauth_pic
%%%fbauth_pic portrait
%%%fbauth_pic background
%%%fbauth_pic other
%%%fbauth_tags
%%%fbauth_pubs
%%%endfbauth
 
А як же, цікаво, раніше називали ті ж самі "перемога" і "промінь"?

%%%fbauth
%%%fbauth_name
\iusr{Vasyl Obolonsky}
%%%fbauth_url
%%%fbauth_place
%%%fbauth_id
%%%fbauth_front
%%%fbauth_desc
%%%fbauth_www
%%%fbauth_pic
%%%fbauth_pic portrait
%%%fbauth_pic background
%%%fbauth_pic other
%%%fbauth_tags
%%%fbauth_pubs
%%%endfbauth
 
Просто з польської назапозичували

%%%fbauth
%%%fbauth_name
\iusr{Olia Olia}
%%%fbauth_url
%%%fbauth_place
%%%fbauth_id
%%%fbauth_front
%%%fbauth_desc
%%%fbauth_www
%%%fbauth_pic
%%%fbauth_pic portrait
%%%fbauth_pic background
%%%fbauth_pic other
%%%fbauth_tags
%%%fbauth_pubs
%%%endfbauth
 
Дуже цікаво

%%%fbauth
%%%fbauth_name
\iusr{Елена Панасюк}
%%%fbauth_url
%%%fbauth_place
%%%fbauth_id
%%%fbauth_front
%%%fbauth_desc
%%%fbauth_www
%%%fbauth_pic
%%%fbauth_pic portrait
%%%fbauth_pic background
%%%fbauth_pic other
%%%fbauth_tags
%%%fbauth_pubs
%%%endfbauth
 
Слово "лелека" есть в турецком языке. Переводится как "аист". Leylek.

%%%fbauth
%%%fbauth_name
\iusr{Irina Pronkina}
%%%fbauth_url
%%%fbauth_place
%%%fbauth_id
%%%fbauth_front
%%%fbauth_desc
%%%fbauth_www
%%%fbauth_pic
%%%fbauth_pic portrait
%%%fbauth_pic background
%%%fbauth_pic other
%%%fbauth_tags
%%%fbauth_pubs
%%%endfbauth
 
Дякую за таку інформацію

%%%fbauth
%%%fbauth_name
\iusr{Оксана Войчак}
%%%fbauth_url
%%%fbauth_place
%%%fbauth_id
%%%fbauth_front
%%%fbauth_desc
%%%fbauth_www
%%%fbauth_pic
%%%fbauth_pic portrait
%%%fbauth_pic background
%%%fbauth_pic other
%%%fbauth_tags
%%%fbauth_pubs
%%%endfbauth
 

Тільки вдуматись в це слово Мрія ,це щось добре, це те що серце б'ється в твоїх
грудях і воно щире і кохане.. Багатослівна наша мова і д'якуючи видатним
письменникам, що одне слово можна замінити другим і мріяти, творити і писати для
людей...


%%%fbauth
%%%fbauth_name
\iusr{Сергій Шум}
%%%fbauth_url
%%%fbauth_place
%%%fbauth_id
%%%fbauth_front
%%%fbauth_desc
%%%fbauth_www
%%%fbauth_pic
%%%fbauth_pic portrait
%%%fbauth_pic background
%%%fbauth_pic other
%%%fbauth_tags
%%%fbauth_pubs
%%%endfbauth
 
Почимчикувати - іноді так кажу😃

%%%fbauth
%%%fbauth_name
\iusr{Bejlo Vereczkey}
%%%fbauth_url
%%%fbauth_place
%%%fbauth_id
%%%fbauth_front
%%%fbauth_desc
%%%fbauth_www
%%%fbauth_pic
%%%fbauth_pic portrait
%%%fbauth_pic background
%%%fbauth_pic other
%%%fbauth_tags
%%%fbauth_pubs
%%%endfbauth
 
Непогано би було в цьому контексті розповісти про Григорія Савича Сковороду. Пардон.

%%%fbauth
%%%fbauth_name
\iusr{Татьяна Ковшик}
%%%fbauth_url
%%%fbauth_place
%%%fbauth_id
%%%fbauth_front
%%%fbauth_desc
%%%fbauth_www
%%%fbauth_pic
%%%fbauth_pic portrait
%%%fbauth_pic background
%%%fbauth_pic other
%%%fbauth_tags
%%%fbauth_pubs
%%%endfbauth
 
Цікаво. Мені здавалось що ці слова в українській мові з покон віків.

\begin{itemize}
%%%fbauth
%%%fbauth_name
\iusr{Отто Фон Штирлиц}
%%%fbauth_url
%%%fbauth_place
%%%fbauth_id
%%%fbauth_front
%%%fbauth_desc
%%%fbauth_www
%%%fbauth_pic
%%%fbauth_pic portrait
%%%fbauth_pic background
%%%fbauth_pic other
%%%fbauth_tags
%%%fbauth_pubs
%%%endfbauth
 
\textbf{Татьяна Ковшик} ,
\href{http://sum.in.ua/s/spokonvikiv}{%
СПОКОНВІКІВ - Словник української мови, Академічний тлумачний словник (1970—1980)%
}
\end{itemize}

%%%fbauth
%%%fbauth_name
\iusr{Ольга Олійник}
%%%fbauth_url
%%%fbauth_place
%%%fbauth_id
%%%fbauth_front
%%%fbauth_desc
%%%fbauth_www
%%%fbauth_pic
%%%fbauth_pic portrait
%%%fbauth_pic background
%%%fbauth_pic other
%%%fbauth_tags
%%%fbauth_pubs
%%%endfbauth
 
Може письменники написали ті слова, які чули від оточуючих, а не вигадали?

%%%fbauth
%%%fbauth_name
\iusr{Егор Подкользин}
%%%fbauth_url
%%%fbauth_place
%%%fbauth_id
%%%fbauth_front
%%%fbauth_desc
%%%fbauth_www
%%%fbauth_pic
%%%fbauth_pic portrait
%%%fbauth_pic background
%%%fbauth_pic other
%%%fbauth_tags
%%%fbauth_pubs
%%%endfbauth
 
Слово «бандура» запозичене з західноєвропейських мов. Там різні щипкрові інструменти називалися й називаються мандола, мандора, бандуррія, бандола.

%%%fbauth
%%%fbauth_name
\iusr{Лариса Ковбасюк}
%%%fbauth_url
%%%fbauth_place
%%%fbauth_id
%%%fbauth_front
%%%fbauth_desc
%%%fbauth_www
%%%fbauth_pic
%%%fbauth_pic portrait
%%%fbauth_pic background
%%%fbauth_pic other
%%%fbauth_tags
%%%fbauth_pubs
%%%endfbauth
 
Те, що Котляревський творив слова - я в курсі, про Лесю Украінку читала, а от внші - відкриття!

%%%fbauth
%%%fbauth_name
\iusr{Мария Коваль}
%%%fbauth_url
%%%fbauth_place
%%%fbauth_id
%%%fbauth_front
%%%fbauth_desc
%%%fbauth_www
%%%fbauth_pic
%%%fbauth_pic portrait
%%%fbauth_pic background
%%%fbauth_pic other
%%%fbauth_tags
%%%fbauth_pubs
%%%endfbauth
 
Спасибі за таку чудову інформацію .Цікаво ,а дітям в школі про це розповідають?

%%%fbauth
%%%fbauth_name
\iusr{Ольха Краевска}
%%%fbauth_url
%%%fbauth_place
%%%fbauth_id
%%%fbauth_front
%%%fbauth_desc
%%%fbauth_www
%%%fbauth_pic
%%%fbauth_pic portrait
%%%fbauth_pic background
%%%fbauth_pic other
%%%fbauth_tags
%%%fbauth_pubs
%%%endfbauth
 
Наколесництво)))

%%%fbauth
%%%fbauth_name
\iusr{אלישע פרוש}
%%%fbauth_url
%%%fbauth_place
%%%fbauth_id
%%%fbauth_front
%%%fbauth_desc
%%%fbauth_www
%%%fbauth_pic
%%%fbauth_pic portrait
%%%fbauth_pic background
%%%fbauth_pic other
%%%fbauth_tags
%%%fbauth_pubs
%%%endfbauth
 
Нечуй-Левицький не є автором лексеми «стосунок» – то є взяте з польської слово <stosunek>. Нащо сим мізки морочити?

%%%fbauth
%%%fbauth_name
\iusr{Roman Melnyk}
%%%fbauth_url
%%%fbauth_place
%%%fbauth_id
%%%fbauth_front
%%%fbauth_desc
%%%fbauth_www
%%%fbauth_pic
%%%fbauth_pic portrait
%%%fbauth_pic background
%%%fbauth_pic other
%%%fbauth_tags
%%%fbauth_pubs
%%%endfbauth
 

Давайте, перед тим як писати нісенітниці, "заглядати у словник". Хоча б у
польський: niesamowity, samotność, dzierżawa, hamować, zawzięcie, mordować się,
samoświadomość, światopogląd, stosunek. лелека, як писав Andriy Kusiy,
тюркського походження. соромно - від ст.-слов. срам.

Вже мовчу, що більшість всіх тих слів є в білоруській. Всі ті слова не
ексклюзивно чи питомо українські і не створені неологізми, а радше загально
слов'янські або адаптовані від польської.

Так, безперечно є маса неологізмів з 19 ст, такі як "мрія", "чинник" ітд. Але
вони свідчать якраз про літературну молодість нашої, до того суто народної,
мови, яка тоді якраз формувалась і потребувала абстрактної лексики. А таку
можна було або взяти з тих вже сформованих мов, що вже побутували на наших
теренах - старослов'янської (але цю нішу вже зайняла російська) чи польської,
або видумати нову. Чим справді чимало класиків і займалися.

Але давайте не перебільшувати з новизною і ексклюзивністю. І не продукувати
міфи і сенсації \Smiley[1.0][yellow]

\begin{itemize}
%%%fbauth
%%%fbauth_name
\iusr{Yulia Pinchakovska}
%%%fbauth_url
%%%fbauth_place
%%%fbauth_id
%%%fbauth_front
%%%fbauth_desc
%%%fbauth_www
%%%fbauth_pic
%%%fbauth_pic portrait
%%%fbauth_pic background
%%%fbauth_pic other
%%%fbauth_tags
%%%fbauth_pubs
%%%endfbauth
 
А ще баскетбол польською і донині koszykówka. Тож кошиківка явно не авторство професора Боберського \Laughey[1.0][white]\Laughey[1.0][white]

%%%fbauth
%%%fbauth_name
\iusr{Roman Melnyk}
%%%fbauth_url
%%%fbauth_place
%%%fbauth_id
%%%fbauth_front
%%%fbauth_desc
%%%fbauth_www
%%%fbauth_pic
%%%fbauth_pic portrait
%%%fbauth_pic background
%%%fbauth_pic other
%%%fbauth_tags
%%%fbauth_pubs
%%%endfbauth
 
\textbf{Yulia Pinchakovska} так, багато неологізмів Боберського - це адаптація з польської (кошиківка, сітківка), але часом досить творча (копаний м'яч = piłka nożna), а часом і дійсно авторське - відбиванка

%%%fbauth
%%%fbauth_name
\iusr{Roman Melnyk}
%%%fbauth_url
%%%fbauth_place
%%%fbauth_id
%%%fbauth_front
%%%fbauth_desc
%%%fbauth_www
%%%fbauth_pic
%%%fbauth_pic portrait
%%%fbauth_pic background
%%%fbauth_pic other
%%%fbauth_tags
%%%fbauth_pubs
%%%endfbauth
 
\textbf{Ольга Павліченко} 

нішу адаптації старослов'янської лексики. Тому нам тепер здається, що наша
стара мова (як її тепер називають, староукраїнська) така схожа на російську.
Насправді вона просто насичена старослов'янською абстрактною лексикою. Але
оскільки її вже прийняла собі російська мова, а ми хотіли від неї конче
відрізнятись, то нам довелось від неї відмовитись і формувати нову.


%%%fbauth
%%%fbauth_name
\iusr{Oleksii Shevchuk}
%%%fbauth_url
%%%fbauth_place
%%%fbauth_id
%%%fbauth_front
%%%fbauth_desc
%%%fbauth_www
%%%fbauth_pic
%%%fbauth_pic portrait
%%%fbauth_pic background
%%%fbauth_pic other
%%%fbauth_tags
%%%fbauth_pubs
%%%endfbauth
 
Ще слово "держава" є у сербській та болгарській мовах.

%%%fbauth
%%%fbauth_name
\iusr{Наталя Кармазин}
%%%fbauth_url
%%%fbauth_place
%%%fbauth_id
%%%fbauth_front
%%%fbauth_desc
%%%fbauth_www
%%%fbauth_pic
%%%fbauth_pic portrait
%%%fbauth_pic background
%%%fbauth_pic other
%%%fbauth_tags
%%%fbauth_pubs
%%%endfbauth
 
\textbf{Roman Melnyk} 

перепрошую але українська не схожа на російську ,російська більше походить від
угро- фінів,також багато слів монгольського походження,такі слова як" завтрак,
ужин і т д ви не знайдете в інших слов'янських мовах, а тих декілька схожих
слів і висловів старослов'янської мови ще не дає право вам писати що українська
мова схожа на російську, ви хоч історію почитайте чи як?


%%%fbauth
%%%fbauth_name
\iusr{Олег Лук'янюк}
%%%fbauth_url
%%%fbauth_place
%%%fbauth_id
%%%fbauth_front
%%%fbauth_desc
%%%fbauth_www
%%%fbauth_pic
%%%fbauth_pic portrait
%%%fbauth_pic background
%%%fbauth_pic other
%%%fbauth_tags
%%%fbauth_pubs
%%%endfbauth
 
\textbf{Roman Melnyk} Почитайте Едварда Кінана. Він це все давно описав

%%%fbauth
%%%fbauth_name
\iusr{Людмила Гаценко}
%%%fbauth_url
%%%fbauth_place
%%%fbauth_id
%%%fbauth_front
%%%fbauth_desc
%%%fbauth_www
%%%fbauth_pic
%%%fbauth_pic portrait
%%%fbauth_pic background
%%%fbauth_pic other
%%%fbauth_tags
%%%fbauth_pubs
%%%endfbauth
 
\textbf{Roman Melnyk} 

справді, давайте не топитися у міфах, не порпатися у стереотипах, а думати,
читати першоджерела. Читайте давні літописи, українські. 

Там ви не побачите польських слів, тим більш "російських" (може, ви не знаєте,
звідкіль взялося, власне, саме це слово, а природу "російських" слів знаєте?
Певно що ні).


%%%fbauth
%%%fbauth_name
\iusr{Roman Melnyk}
%%%fbauth_url
%%%fbauth_place
%%%fbauth_id
%%%fbauth_front
%%%fbauth_desc
%%%fbauth_www
%%%fbauth_pic
%%%fbauth_pic portrait
%%%fbauth_pic background
%%%fbauth_pic other
%%%fbauth_tags
%%%fbauth_pubs
%%%endfbauth
 
\textbf{Наталя Кармазин} 

по-перше, я до історії маю пряме і безпосереднє відношення. не люблю переходити
на особистості, але судячи з написаного, куди ближче, ніж Ви (без образ). тепер
по ділу:

1. російська мова належить до слов'янської мовної групи. до угро-фінської сім'ї
належать фінська, угорська та естонська. тут, думаю, пояснювати не треба. ви
вочевидь плутаєте етнічне і мовне.

2. завтрак (від завтра) і ужин (від уживати) - слова якраз чисто слов'янського
походження, хай й оригінальні російські. тут в принципі теж все очевидно.

3. російська мова значно більше має лексики зі старо(церковно)слов'янської
(тої, що від староболгарської походить), ніж укранська. тому що українці
свідомо почали в 19 ст. нову літературну традицію на чисто народній основі без
давніх церковних домішок.

4. я ніде не писав, що українська схожа на російську. я писав, що часом може
здаєтися, що СТАРОукраїнська (це та сама що старобілоруська) схожа на
російську, адже містить ту лексику, яка в російській залишилась, а в сучасній
українській ні.

тому я дякую за пораду почитати історію - тим і займаюсь. але Вас я прошу хоча
б читати уважно те, на що відповідаєте.

%%%fbauth
%%%fbauth_name
\iusr{Roman Melnyk}
%%%fbauth_url
%%%fbauth_place
%%%fbauth_id
%%%fbauth_front
%%%fbauth_desc
%%%fbauth_www
%%%fbauth_pic
%%%fbauth_pic portrait
%%%fbauth_pic background
%%%fbauth_pic other
%%%fbauth_tags
%%%fbauth_pubs
%%%endfbauth
 
\textbf{Людмила Гаценко} 

я не дуже зрозумів, що Ви маєте на увазі. я нічого не говорив про давні
літописи чи про російські слова. мова йшла про сучасну українську літературну
мову, впливи польської лексики на неї, відмову від старослов'янських слів та
словесну творчість наших класиків. хоча враховуючи, що весь цей процес
відбувався в Російській імперії, то якийсь вплив і російської безумовно мусив
бути.


%%%fbauth
%%%fbauth_name
\iusr{Roman Melnyk}
%%%fbauth_url
%%%fbauth_place
%%%fbauth_id
%%%fbauth_front
%%%fbauth_desc
%%%fbauth_www
%%%fbauth_pic
%%%fbauth_pic portrait
%%%fbauth_pic background
%%%fbauth_pic other
%%%fbauth_tags
%%%fbauth_pubs
%%%endfbauth
 
\textbf{Олег Лук'янюк} 

чув про праці Кінана, хоч і не читав, про російську історію і його версії про
фальсифікацію "Слова о полку Ігоревім" та кількох інших рукописів. хіба він
писав щось про формування української літературної мови?) знаю з цієї теми Юрія
Шевельова.

%%%fbauth
%%%fbauth_name
\iusr{Тарас Загорський}
%%%fbauth_url
%%%fbauth_place
%%%fbauth_id
%%%fbauth_front
%%%fbauth_desc
%%%fbauth_www
%%%fbauth_pic
%%%fbauth_pic portrait
%%%fbauth_pic background
%%%fbauth_pic other
%%%fbauth_tags
%%%fbauth_pubs
%%%endfbauth
 
\textbf{Наталя Кармазин} 

перегинаєте. Російська це якраз і є адаптована
староболгарська-церковнослов'янська. Тому вона більше схожа на південно-слов'
янські мови - сербську чи болгарську.



%%%fbauth
%%%fbauth_name
\iusr{Окса Старко}
%%%fbauth_url
%%%fbauth_place
%%%fbauth_id
%%%fbauth_front
%%%fbauth_desc
%%%fbauth_www
%%%fbauth_pic
%%%fbauth_pic portrait
%%%fbauth_pic background
%%%fbauth_pic other
%%%fbauth_tags
%%%fbauth_pubs
%%%endfbauth
 
\textbf{Roman Melnyk} 

а звідки взялася польська мова???? Поляки настільки старші за нас, чи ми з
космосу вчора прилетіли, чи до того німі були. Дуже багато мудрих розвелося, і
всі з вченими сТУПЕнями! Ще санскрит підівчіть, гуцульску....!

%%%fbauth
%%%fbauth_name
\iusr{Катерина Масалова}
%%%fbauth_url
%%%fbauth_place
%%%fbauth_id
%%%fbauth_front
%%%fbauth_desc
%%%fbauth_www
%%%fbauth_pic
%%%fbauth_pic portrait
%%%fbauth_pic background
%%%fbauth_pic other
%%%fbauth_tags
%%%fbauth_pubs
%%%endfbauth
 

Звідки взялася польська мова взагалі чи запозичення з польської в українській
мові? Якщо йдеться про останнє, то просто слід знати історію України і тоді для
виникнення такого питання у Вас не буде причини! 

Польська шляхта панувалана Західній Україні, володіла українськими землями і
насаджувала серед українських селян свою мову і свою віру! 

Згадайте про Унію і що таке УГКЦ, котра досі поширена в Україні!

%%%fbauth
%%%fbauth_name
\iusr{Roman Melnyk}
%%%fbauth_url
\url{https://www.facebook.com/roman.melnyk.lviv}\par
%%%fbauth_place
%%%fbauth_id
%%%fbauth_front
%%%fbauth_desc
%%%fbauth_www
%%%fbauth_pic
%%%fbauth_pic portrait
%%%fbauth_pic background
%%%fbauth_pic other
%%%fbauth_tags
%%%fbauth_pubs
%%%endfbauth
 
\textbf{Oxana Starko} 

обійдімось без санскриту і взаємних образ. Поясню ще раз: є мова розмовна
народна, а є літературна писемна. Це дві різні категорії.

Так от, розмовна мова - і українська, і польська - зародилася приблизно
однаково в 8-10 ст. Тут вони ровесниці. Натомість літературна польська
розвивається безперервно з 16 ст., в тому числі і на наших землях: перший
польськомовний поет - "галичанин" Миколай Рей. А літературна українська
розвивається безперервно з 19 ст. Тому так, літературна українська молодша за
польську десь на 200-300 років.

Це не означає, що українці до того не мали своєї літературної мови. Мали. Тоді
називали її руською мовою, тепер - староукраїнською/білоруською (бо вона була
спільною). Але ця мова відмерла в 18 ст., поступившись місцем польській на
заході і т.зв. слов'яноруській на сході. Тому свій літературний процес ми в 19
ст. починали з нуля, від народних основ, виходячи з тих реалій які були. І
абстрактну лексику ми могли взяти в тих мов, які її мали і які були живі на
наших землях - себто польської або слов'яноруської. Саме про це я і писав, не
більше і не менше.

% -------------------------------------
\ii{fbauth.melnyk_roman.lvov.ukraina.istorik}
% -------------------------------------

 
Катерина Масалова згрубша маєте рацію, але дозвольте пару зауважень.

- До Польської корони належала не тільки західна, але в 16-17 ст. й практично вся тодішня Україна.

- "Свою віру і мову" насаджувала не шляхта, а церква. Шляхті насправді було
байдуже, якою мовою і до якої церкви ходять її селяни, головне щоб на полі
працювали. Щоб задобрити робочу силу, шляхта могла меценувати і церкви, і
костели. А от католицька церква дійсно була зацікавлена у приверненні нових душ
у своє лоно - це факт.

Але справедливости ради - разом з "окатоличенням" ця церква несла сюди і
освіту. Перші середні і вищі школи на наших землях справді відкрили католики.
Православні, на жаль, за ними не встигали, а до Острозького і Могили не мали
майже нічого. Саме тому католики мали вищий рівень освіти, відтак більше
престижу + лояльність держави - і тому зверхньо ставились до православних.

- Фраза "польська шляхта" - це легка маніпуляція радянського розливу. Шляхта
була різна - і польська, і литовська, і руська. І католицька, і православна.
Вони всі мали рівні права і обов'язки в тій державі. Як і селяни - і ті, і ті
однаково були кріпаками. Подобається це комусь чи ні, але українці/русини того
часу - це не лише козаки і селяни.

Справа в тому, що з 17 ст. руська шляхта почала потрохи, але впевнено
переходити на католицтво і відповідно ставала "польською". Але це не означає,
що вона одвічно була польською. І не вся шляхта таки перейшла - частина
покозачилась, частина пішла в духовенство - до речі, греко-католицьке.

- Тому не знаю, в чому ви звинувачуєте Унію - це якраз була спроба компромісу в
момент загострення ситуації. Унія дозволяла зберегти традицію і водночас
торкнутися безпосередньо західної культури. Після анексії київського
православ'я московським якраз Унійна церква в Галичині краще зберегла і
відродила київську традицію. І якраз її духовенство (яке часто походило зі
старої руської шляхти) починало національний рух в Галичині.

Вибачте, що багато тексту, але ці моменти для мене важливі \Smiley[1.0][yellow]

%%%fbauth
%%%fbauth_name
\iusr{Тарас Загорський}
%%%fbauth_url
%%%fbauth_place
%%%fbauth_id
%%%fbauth_front
%%%fbauth_desc
%%%fbauth_www
%%%fbauth_pic
%%%fbauth_pic portrait
%%%fbauth_pic background
%%%fbauth_pic other
%%%fbauth_tags
%%%fbauth_pubs
%%%endfbauth
 

Всім, хто тут авторитетно говорить про мову, рекомендую пошукати в ютюбі лекції
Констянтина Тищенка. Він як мовознавець краще розкаже про походження, розвиток
мови та інше що цього стосується


%%%fbauth
%%%fbauth_name
\iusr{Nataliya Kolesnikova}
%%%fbauth_url
%%%fbauth_place
%%%fbauth_id
%%%fbauth_front
%%%fbauth_desc
%%%fbauth_www
%%%fbauth_pic
%%%fbauth_pic portrait
%%%fbauth_pic background
%%%fbauth_pic other
%%%fbauth_tags
%%%fbauth_pubs
%%%endfbauth
 
\textbf{Roman Melnyk} дуже цікаво читати ваші коментарі. Дякую, я багато дізналась. \Smiley[1.0][yellow]

%%%fbauth
%%%fbauth_name
\iusr{Валентина Шутяк}
%%%fbauth_url
%%%fbauth_place
%%%fbauth_id
%%%fbauth_front
%%%fbauth_desc
%%%fbauth_www
%%%fbauth_pic
%%%fbauth_pic portrait
%%%fbauth_pic background
%%%fbauth_pic other
%%%fbauth_tags
%%%fbauth_pubs
%%%endfbauth
 
\textbf{Roman Melnyk}


\end{itemize}

%%%fbauth
%%%fbauth_name
\iusr{Andro Gorbacio}
%%%fbauth_url
%%%fbauth_place
%%%fbauth_id
%%%fbauth_front
%%%fbauth_desc
%%%fbauth_www
%%%fbauth_pic
%%%fbauth_pic portrait
%%%fbauth_pic background
%%%fbauth_pic other
%%%fbauth_tags
%%%fbauth_pubs
%%%endfbauth
 
Так люди завжди творять свою мову і щось все таки приживається.

Я наприклад п'ятий рік спостерігаю, як приживеться придумане і запущене мною
слово "путіноїд, путіноїди".

А колись, на початку двотисячних, у звітах по правоохоронній роботі, які
надсилав у центральні органи, почав застосовувати слово "оптимізувати", яке до
цього використовувалося у вузькій технічній сфері. А через декілька років
дивився, як воно то тут то там використовується вищими посадовими особами. І що
характерно, семе у тому значенні, яке було задумано - для окозамилювання.


%%%fbauth
%%%fbauth_name
\iusr{Надя Томюк}
%%%fbauth_url
%%%fbauth_place
%%%fbauth_id
%%%fbauth_front
%%%fbauth_desc
%%%fbauth_www
%%%fbauth_pic
%%%fbauth_pic portrait
%%%fbauth_pic background
%%%fbauth_pic other
%%%fbauth_tags
%%%fbauth_pubs
%%%endfbauth
 
Якось дивно й сумно - до Старицького не було "мрії"....

\begin{itemize}
%%%fbauth
%%%fbauth_name
\iusr{Тетяна Ільків}
%%%fbauth_url
%%%fbauth_place
%%%fbauth_id
%%%fbauth_front
%%%fbauth_desc
%%%fbauth_www
%%%fbauth_pic
%%%fbauth_pic portrait
%%%fbauth_pic background
%%%fbauth_pic other
%%%fbauth_tags
%%%fbauth_pubs
%%%endfbauth
 
Проте це не заважало людям мріяти \Smiley[1.0][yellow]

%%%fbauth
%%%fbauth_name
\iusr{Хулио Моралес}
%%%fbauth_url
%%%fbauth_place
%%%fbauth_id
%%%fbauth_front
%%%fbauth_desc
%%%fbauth_www
%%%fbauth_pic
%%%fbauth_pic portrait
%%%fbauth_pic background
%%%fbauth_pic other
%%%fbauth_tags
%%%fbauth_pubs
%%%endfbauth
 
\textbf{Надя Томюк} мрію збудували на авіазаводі літакобудування

%%%fbauth
%%%fbauth_name
\iusr{Игорь Гаврилов}
%%%fbauth_url
%%%fbauth_place
%%%fbauth_id
%%%fbauth_front
%%%fbauth_desc
%%%fbauth_www
%%%fbauth_pic
%%%fbauth_pic portrait
%%%fbauth_pic background
%%%fbauth_pic other
%%%fbauth_tags
%%%fbauth_pubs
%%%endfbauth
 
\textbf{Надя Томюк} Нiчого дивного - до Старицького i лiтака "Мрiя" не iснувало , i украiнськоi ракети ""Зенiт" ))
\end{itemize}

%%%fbauth
%%%fbauth_name
\iusr{Лариса Косідло}
%%%fbauth_url
%%%fbauth_place
%%%fbauth_id
%%%fbauth_front
%%%fbauth_desc
%%%fbauth_www
%%%fbauth_pic
%%%fbauth_pic portrait
%%%fbauth_pic background
%%%fbauth_pic other
%%%fbauth_tags
%%%fbauth_pubs
%%%endfbauth
 
Цікаво. Пам'ятаю, у Михайла Стельмаха зустрічається понад 40 синонімів слова "ходити". Мабуть, теж взяті з розмовної мови.

%%%fbauth
%%%fbauth_name
\iusr{Кара Грин}
%%%fbauth_url
%%%fbauth_place
%%%fbauth_id
%%%fbauth_front
%%%fbauth_desc
%%%fbauth_www
%%%fbauth_pic
%%%fbauth_pic portrait
%%%fbauth_pic background
%%%fbauth_pic other
%%%fbauth_tags
%%%fbauth_pubs
%%%endfbauth
 
А хто такий Лесь Українка?)

%%%fbauth
%%%fbauth_name
\iusr{Владислав Клочков}
%%%fbauth_url
%%%fbauth_place
%%%fbauth_id
%%%fbauth_front
%%%fbauth_desc
%%%fbauth_www
%%%fbauth_pic
%%%fbauth_pic portrait
%%%fbauth_pic background
%%%fbauth_pic other
%%%fbauth_tags
%%%fbauth_pubs
%%%endfbauth
 
Наційональна Ліга Гаківки створена українцями, що перепливли океан навпомацки й навимашки...

%%%fbauth
%%%fbauth_name
\iusr{Ольга Барановская}
%%%fbauth_url
%%%fbauth_place
%%%fbauth_id
%%%fbauth_front
%%%fbauth_desc
%%%fbauth_www
%%%fbauth_pic
%%%fbauth_pic portrait
%%%fbauth_pic background
%%%fbauth_pic other
%%%fbauth_tags
%%%fbauth_pubs
%%%endfbauth
 
Копаний м'яч та пориванка - це щедевр!

%%%fbauth
%%%fbauth_name
\iusr{Артем Шевелєв}
%%%fbauth_url
%%%fbauth_place
%%%fbauth_id
%%%fbauth_front
%%%fbauth_desc
%%%fbauth_www
%%%fbauth_pic
%%%fbauth_pic portrait
%%%fbauth_pic background
%%%fbauth_pic other
%%%fbauth_tags
%%%fbauth_pubs
%%%endfbauth
 
Моя мрія, це мати всі відповідники лише українською, як мають японці -японською.

\begin{itemize}
%%%fbauth
%%%fbauth_name
\iusr{Halyna Holosovska}
%%%fbauth_url
%%%fbauth_place
%%%fbauth_id
%%%fbauth_front
%%%fbauth_desc
%%%fbauth_www
%%%fbauth_pic
%%%fbauth_pic portrait
%%%fbauth_pic background
%%%fbauth_pic other
%%%fbauth_tags
%%%fbauth_pubs
%%%endfbauth
 
\textbf{Артем Шевелєв}, а поляки - польською. Відсутні треба створювати.
\end{itemize}

%%%fbauth
%%%fbauth_name
\iusr{Митя Спирин}
%%%fbauth_url
%%%fbauth_place
%%%fbauth_id
%%%fbauth_front
%%%fbauth_desc
%%%fbauth_www
%%%fbauth_pic
%%%fbauth_pic portrait
%%%fbauth_pic background
%%%fbauth_pic other
%%%fbauth_tags
%%%fbauth_pubs
%%%endfbauth
 
"Мiстецьтво" они придумали определённо от немецкого "Mist".
Маэстро Лэсь Поддеревьянский это разъясняет в произведении "Пiздэць" про "митцiв".🤩😂😂

%%%fbauth
%%%fbauth_name
\iusr{Оксана Дуда}
%%%fbauth_url
%%%fbauth_place
%%%fbauth_id
%%%fbauth_front
%%%fbauth_desc
%%%fbauth_www
%%%fbauth_pic
%%%fbauth_pic portrait
%%%fbauth_pic background
%%%fbauth_pic other
%%%fbauth_tags
%%%fbauth_pubs
%%%endfbauth
 
Знала лише про промінь) Дуже класна інформація!

%%%fbauth
%%%fbauth_name
\iusr{Oleksii Shevchuk}
%%%fbauth_url
%%%fbauth_place
%%%fbauth_id
%%%fbauth_front
%%%fbauth_desc
%%%fbauth_www
%%%fbauth_pic
%%%fbauth_pic portrait
%%%fbauth_pic background
%%%fbauth_pic other
%%%fbauth_tags
%%%fbauth_pubs
%%%endfbauth
 
Держава не знайду у інших мовах? Ви серйозно? Це ж загально-слов'янізм.

\begin{itemize}
%%%fbauth
%%%fbauth_name
\iusr{Oleksii Shevchuk}
%%%fbauth_url
%%%fbauth_place
%%%fbauth_id
%%%fbauth_front
%%%fbauth_desc
%%%fbauth_www
%%%fbauth_pic
%%%fbauth_pic portrait
%%%fbauth_pic background
%%%fbauth_pic other
%%%fbauth_tags
%%%fbauth_pubs
%%%endfbauth
 
Фортеця? Ахахах. Нічого що у англ. це fortress і це очевидно індоєвропейське слово. \Smiley[1.0][yellow]

%%%fbauth
%%%fbauth_name
\iusr{Богдан Матата}
%%%fbauth_url
%%%fbauth_place
%%%fbauth_id
%%%fbauth_front
%%%fbauth_desc
%%%fbauth_www
%%%fbauth_pic
%%%fbauth_pic portrait
%%%fbauth_pic background
%%%fbauth_pic other
%%%fbauth_tags
%%%fbauth_pubs
%%%endfbauth
 
а слово "розмаїття" придумав Симон Петлюра

а слово "сум'яття" вигадав Винниченко
\end{itemize}

%%%fbauth
%%%fbauth_name
\iusr{Niina Ohrimenko}
%%%fbauth_url
%%%fbauth_place
%%%fbauth_id
%%%fbauth_front
%%%fbauth_desc
%%%fbauth_www
%%%fbauth_pic
%%%fbauth_pic portrait
%%%fbauth_pic background
%%%fbauth_pic other
%%%fbauth_tags
%%%fbauth_pubs
%%%endfbauth

А естонці взяли від нас слово "щирий" - у них не було такого поняття. А тепер є
- "siiras. Один естонець-філолог жив якийсь час Україні.

\begin{itemize}
%%%fbauth
%%%fbauth_name
\iusr{Adrian Harasym}
%%%fbauth_url
%%%fbauth_place
%%%fbauth_id
%%%fbauth_front
%%%fbauth_desc
%%%fbauth_www
%%%fbauth_pic
%%%fbauth_pic portrait
%%%fbauth_pic background
%%%fbauth_pic other
%%%fbauth_tags
%%%fbauth_pubs
%%%endfbauth
 
Це слово давнє індо-європейське, воно існує навіть в англійській (sheer), з тим самим значенням - "чистий".

Ми кажемо "щире золото". А англійці "шір ґолд".

Ґолд, доречі, теж з золотом споріднене. Тільки східно-індоєвропейські мови (так звані "сатум") вживають "с" і "з", коли західні мови ("кентум") вживають відповідно "к" і "ґ".

Якщо знати мовні закони, то мови здаватимуться значно подібнімишими.

Навіть "зелений" і "green" англійське споріднені.

Є просто чергування з-ґ і л-р.

Л-Р, доручі, в дуууже багатьох мовах чергується (лицар-рыцарь).
\end{itemize}

%%%fbauth
%%%fbauth_name
\iusr{Алла Галицька}
%%%fbauth_url
%%%fbauth_place
%%%fbauth_id
%%%fbauth_front
%%%fbauth_desc
%%%fbauth_www
%%%fbauth_pic
%%%fbauth_pic portrait
%%%fbauth_pic background
%%%fbauth_pic other
%%%fbauth_tags
%%%fbauth_pubs
%%%endfbauth
 

До Шевченка ніхто не говорив "фортеця"? Можливо. Але ж "форт" був!? Тобто, це
як вчора була жінка-колега, а сьогодні колежанка? А про чимчикуючих і
замордованих, він перший записав, можливо, але у народному вжитку вони були,
тим паче "вогняні" кулі кидали на форти під час боїв.

Глібов "соромно"ввів як одиницю?

Щось не те.

От "забави й ігри рухові" - з народними термінами замість іншомовних - цікаво і
самобутньо.

\begin{itemize}
%%%fbauth
%%%fbauth_name
\iusr{Олег Марченко}
%%%fbauth_url
%%%fbauth_place
%%%fbauth_id
%%%fbauth_front
%%%fbauth_desc
%%%fbauth_www
%%%fbauth_pic
%%%fbauth_pic portrait
%%%fbauth_pic background
%%%fbauth_pic other
%%%fbauth_tags
%%%fbauth_pubs
%%%endfbauth
 
\textbf{Alla Halytska} "колежанка", принаймні так як я чув на західній Україні, має інше значення, аніж колега жіночого роду \Smiley[1.0][yellow]. Це подруга.

%%%fbauth
%%%fbauth_name
\iusr{Elizaveta Pavlenko}
%%%fbauth_url
%%%fbauth_place
%%%fbauth_id
%%%fbauth_front
%%%fbauth_desc
%%%fbauth_www
%%%fbauth_pic
%%%fbauth_pic portrait
%%%fbauth_pic background
%%%fbauth_pic other
%%%fbauth_tags
%%%fbauth_pubs
%%%endfbauth
 
Олег Марченко/коло лежить.
Коло не стоїть в прямому значенні (хіба як скрутити
Гіллочку у вигляді коло))
Й, напевно, тому ко+лежанка=
колежанка, в значенні подруга.

%%%fbauth
%%%fbauth_name
\iusr{Алла Галицька}
%%%fbauth_url
%%%fbauth_place
%%%fbauth_id
%%%fbauth_front
%%%fbauth_desc
%%%fbauth_www
%%%fbauth_pic
%%%fbauth_pic portrait
%%%fbauth_pic background
%%%fbauth_pic other
%%%fbauth_tags
%%%fbauth_pubs
%%%endfbauth
 
\textbf{Олег Марченко} це у них у значенні за інтересами, але не подруга-рідна душа.

%%%fbauth
%%%fbauth_name
\iusr{Marta Yurkiv}
%%%fbauth_url
%%%fbauth_place
%%%fbauth_id
%%%fbauth_front
%%%fbauth_desc
%%%fbauth_www
%%%fbauth_pic
%%%fbauth_pic portrait
%%%fbauth_pic background
%%%fbauth_pic other
%%%fbauth_tags
%%%fbauth_pubs
%%%endfbauth
 
Колєжанки і колєдзи - це у польській мові подруги і друзі. У нас на Західній Україні кажуть "коліжанка" - це подруга

%%%fbauth
%%%fbauth_name
\iusr{Ольга Мельничук}
%%%fbauth_url
%%%fbauth_place
%%%fbauth_id
%%%fbauth_front
%%%fbauth_desc
%%%fbauth_www
%%%fbauth_pic
%%%fbauth_pic portrait
%%%fbauth_pic background
%%%fbauth_pic other
%%%fbauth_tags
%%%fbauth_pubs
%%%endfbauth
 
І кошикувка також польською баскетбол

%%%fbauth
%%%fbauth_name
\iusr{Inna Huba}
%%%fbauth_url
%%%fbauth_place
%%%fbauth_id
%%%fbauth_front
%%%fbauth_desc
%%%fbauth_www
%%%fbauth_pic
%%%fbauth_pic portrait
%%%fbauth_pic background
%%%fbauth_pic other
%%%fbauth_tags
%%%fbauth_pubs
%%%endfbauth
 
\textbf{Marta Yurkiv}, 

kolega / koleżanka z klasy - це однокласник / однокласниця, kolega / koleżanka
z pracy - це колега по роботі. Kolega / koleżanka - це хтось, з ким довелося
познайомитися при певних обставинах, в тому числі і на роботі, але це не друг /
подруга, це знайомі. Точніше це поняття передає комуняцьке слово "товариш".

Друг / подруга - przyjaciel / przyjaciółka.

%%%fbauth
%%%fbauth_name
\iusr{Ніна Слєпкова}
%%%fbauth_url
%%%fbauth_place
%%%fbauth_id
%%%fbauth_front
%%%fbauth_desc
%%%fbauth_www
%%%fbauth_pic
%%%fbauth_pic portrait
%%%fbauth_pic background
%%%fbauth_pic other
%%%fbauth_tags
%%%fbauth_pubs
%%%endfbauth
 
\textbf{Elizaveta Pavlenko} Народна етимологія як вона є... 🤕

%%%fbauth
%%%fbauth_name
\iusr{Elizaveta Pavlenko}
%%%fbauth_url
%%%fbauth_place
%%%fbauth_id
%%%fbauth_front
%%%fbauth_desc
%%%fbauth_www
%%%fbauth_pic
%%%fbauth_pic portrait
%%%fbauth_pic background
%%%fbauth_pic other
%%%fbauth_tags
%%%fbauth_pubs
%%%endfbauth
 
\textbf{Ніна Слєпкова} 

/Згадалось!

В 80-х роках їздила на весілля

подруги в Смілу,Черкаська область. СВІДОК жениха називався ДРУЖБА.

Свідок нареченої, - ДРУЖКА.

Так там було прийнято, не вами, не мною.

Й на столі стояла заквітчена

ГІЛКА,іноді називал襳ЛьЦЕ!

Мовлення завжди має свій

сюрприз! Народна етимологія,

породила діалектне Мовлення!!

%%%fbauth
%%%fbauth_name
\iusr{Ирина Довгань Дорощак}
%%%fbauth_url
%%%fbauth_place
%%%fbauth_id
%%%fbauth_front
%%%fbauth_desc
%%%fbauth_www
%%%fbauth_pic
%%%fbauth_pic portrait
%%%fbauth_pic background
%%%fbauth_pic other
%%%fbauth_tags
%%%fbauth_pubs
%%%endfbauth
 
\textbf{Elizaveta Pavlenko} 

це слово пишеться і говориться коліжанка, а не колежанка. І є ще в них слово
колєґа, це співробітник. А коліжанка файна, це дуже добра подруга з якою і
філіжанку кави можна випити, і кулемет закопати. Тому коло і лежанка тут
недоречно

%%%fbauth
%%%fbauth_name
\iusr{Вікторія Проць}
%%%fbauth_url
%%%fbauth_place
%%%fbauth_id
%%%fbauth_front
%%%fbauth_desc
%%%fbauth_www
%%%fbauth_pic
%%%fbauth_pic portrait
%%%fbauth_pic background
%%%fbauth_pic other
%%%fbauth_tags
%%%fbauth_pubs
%%%endfbauth
 
\textbf{Elizaveta Pavlenko} 

так і на Львівщині завжди були дружки і дружби.

У класиків почитайте : світелки з дружечками.

%%%fbauth
%%%fbauth_name
\iusr{Inna Huba}
%%%fbauth_url
%%%fbauth_place
%%%fbauth_id
%%%fbauth_front
%%%fbauth_desc
%%%fbauth_www
%%%fbauth_pic
%%%fbauth_pic portrait
%%%fbauth_pic background
%%%fbauth_pic other
%%%fbauth_tags
%%%fbauth_pubs
%%%endfbauth
 
\textbf{Elizaveta Pavlenko}, на Хмельниччині теж дружба і дружка, а оте деревце - різка))

%%%fbauth
%%%fbauth_name
\iusr{Lyubov Chyzhyk}
%%%fbauth_url
%%%fbauth_place
%%%fbauth_id
%%%fbauth_front
%%%fbauth_desc
%%%fbauth_www
%%%fbauth_pic
%%%fbauth_pic portrait
%%%fbauth_pic background
%%%fbauth_pic other
%%%fbauth_tags
%%%fbauth_pubs
%%%endfbauth
 
і на Івано- Франківщині- дружка і дружба

%%%fbauth
%%%fbauth_name
\iusr{Оксана Маркович}
%%%fbauth_url
%%%fbauth_place
%%%fbauth_id
%%%fbauth_front
%%%fbauth_desc
%%%fbauth_www
%%%fbauth_pic
%%%fbauth_pic portrait
%%%fbauth_pic background
%%%fbauth_pic other
%%%fbauth_tags
%%%fbauth_pubs
%%%endfbauth
 

На весіллі у коровай(святковий хліб) ставили ліску(це були гарні не живі квіти)
за звичаєм, дружки молодої торгувалися із хресним(брали викуп)лише тоді
дозволялося розрізати коровай - це Тернопільщина

%%%fbauth
%%%fbauth_name
\iusr{Оксана Маркович}
%%%fbauth_url
%%%fbauth_place
%%%fbauth_id
%%%fbauth_front
%%%fbauth_desc
%%%fbauth_www
%%%fbauth_pic
%%%fbauth_pic portrait
%%%fbauth_pic background
%%%fbauth_pic other
%%%fbauth_tags
%%%fbauth_pubs
%%%endfbauth
 
Дружки і свашки

%%%fbauth
%%%fbauth_name
\iusr{Ruslana Martseniuk}
%%%fbauth_url
%%%fbauth_place
%%%fbauth_id
%%%fbauth_front
%%%fbauth_desc
%%%fbauth_www
%%%fbauth_pic
%%%fbauth_pic portrait
%%%fbauth_pic background
%%%fbauth_pic other
%%%fbauth_tags
%%%fbauth_pubs
%%%endfbauth
 
\textbf{Elizaveta Pavlenko} і на Західній Україні назви ідентичні. Ніхто свідками їх ніколи не називав. Хіба вже тепер, якось по незрозумілій на це моді.

%%%fbauth
%%%fbauth_name
\iusr{Киля Швець}
%%%fbauth_url
%%%fbauth_place
%%%fbauth_id
%%%fbauth_front
%%%fbauth_desc
%%%fbauth_www
%%%fbauth_pic
%%%fbauth_pic portrait
%%%fbauth_pic background
%%%fbauth_pic other
%%%fbauth_tags
%%%fbauth_pubs
%%%endfbauth
 
Шевченків край, Звенигородка - подружки нареченої - дружки, а друзі нареченого - бояри. Старша дружка і старший боярин - найближчі, або ті, кого хотіли познайомити з метою теж одружити.

%%%fbauth
%%%fbauth_name
\iusr{Александр Дудар}
%%%fbauth_url
%%%fbauth_place
%%%fbauth_id
%%%fbauth_front
%%%fbauth_desc
%%%fbauth_www
%%%fbauth_pic
%%%fbauth_pic portrait
%%%fbauth_pic background
%%%fbauth_pic other
%%%fbauth_tags
%%%fbauth_pubs
%%%endfbauth
 
\textbf{Олег Марченко}
Польською \enquote{kolega}, як і \enquote{koleżanka} – значать друга/подругу, але не так близьких, як \enquote{przyjaciele}.

%%%fbauth
%%%fbauth_name
\iusr{Александр Дудар}
%%%fbauth_url
%%%fbauth_place
%%%fbauth_id
%%%fbauth_front
%%%fbauth_desc
%%%fbauth_www
%%%fbauth_pic
%%%fbauth_pic portrait
%%%fbauth_pic background
%%%fbauth_pic other
%%%fbauth_tags
%%%fbauth_pubs
%%%endfbauth
 
\textbf{Ruslana Martseniuk}
Вперше чую.

Ніколи не знав, шо дружбу й дружку можна якось инакше звати...

\end{itemize}

%%%fbauth
%%%fbauth_name
\iusr{Ольга Лисенко}
%%%fbauth_url
%%%fbauth_place
%%%fbauth_id
%%%fbauth_front
%%%fbauth_desc
%%%fbauth_www
%%%fbauth_pic
%%%fbauth_pic portrait
%%%fbauth_pic background
%%%fbauth_pic other
%%%fbauth_tags
%%%fbauth_pubs
%%%endfbauth
 
Дуже цікава інформація

%%%fbauth
%%%fbauth_name
\iusr{Любов Городиська}
%%%fbauth_url
%%%fbauth_place
%%%fbauth_id
%%%fbauth_front
%%%fbauth_desc
%%%fbauth_www
%%%fbauth_pic
%%%fbauth_pic portrait
%%%fbauth_pic background
%%%fbauth_pic other
%%%fbauth_tags
%%%fbauth_pubs
%%%endfbauth
 

Цікава інформація. Більше 20 років викладаю у школі, діти із цікавістю
сприймають такий матеріал. Правда, це програмою профільного навчання.


%%%fbauth
%%%fbauth_name
\iusr{Остап Галаджун}
%%%fbauth_url
%%%fbauth_place
%%%fbauth_id
%%%fbauth_front
%%%fbauth_desc
%%%fbauth_www
%%%fbauth_pic
%%%fbauth_pic portrait
%%%fbauth_pic background
%%%fbauth_pic other
%%%fbauth_tags
%%%fbauth_pubs
%%%endfbauth
 
Дурна стаття, ці слова були раніше

%%%fbauth
%%%fbauth_name
\iusr{Nataliya Hammouda}
%%%fbauth_url
%%%fbauth_place
%%%fbauth_id
%%%fbauth_front
%%%fbauth_desc
%%%fbauth_www
%%%fbauth_pic
%%%fbauth_pic portrait
%%%fbauth_pic background
%%%fbauth_pic other
%%%fbauth_tags
%%%fbauth_pubs
%%%endfbauth
 

Ці події вже надто "віддавніли".

Ви мене "розплакали" - моє

%%%fbauth
%%%fbauth_name
\iusr{Adrian Harasym}
%%%fbauth_url
%%%fbauth_place
%%%fbauth_id
%%%fbauth_front
%%%fbauth_desc
%%%fbauth_www
%%%fbauth_pic
%%%fbauth_pic portrait
%%%fbauth_pic background
%%%fbauth_pic other
%%%fbauth_tags
%%%fbauth_pubs
%%%endfbauth
 

Я, звичайно, не інститут, і філологія тільки моє приватне захоплення (хоббі), але дозволю собі з дечим не погодитися:

Деякі з наведених "щироукраїнських слів" викликають сумніви:

"Бандура" наша явно запозичена від греків:

\url{https://en.m.wikipedia.org/wiki/Pandura}

"Держава" існує як "држава" в сербів та хорватів.

"Лейлек" прилітає і до Албанії.

А "козачок" - це зменшувальне від слова "козак", котре, як відомо, тюркського
походження.

Тож, щоб стверджувати, що "тих слів нема в інших мовах", треба знати інші
мови...

Та й серед слів, котрі буцім-то " винайшли" письменники, теж думаю буде немало
народних слів. Те, що їх не зустрічаємо в літописах та піснях і думах, не
означає, що їх не вживалося в мовленні. Просто слова могли не записуватися до
того.

Та сама "фортеця" давно вживалася до Шевченка. Це слово італійське -
"фортецца", звідти перейшло до польської. Поляки будували фортеці в Україні ще
в 16 столітті. 300 років до Шевченка. Народ точно називав ці споруди тим
польсько-італійським словом, котрим ці споруди називали поляки.

Одним словом, " учітеся, брати мої, думайте, читайте!"

%%%fbauth
%%%fbauth_name
\iusr{Любов Котис}
%%%fbauth_url
%%%fbauth_place
%%%fbauth_id
%%%fbauth_front
%%%fbauth_desc
%%%fbauth_www
%%%fbauth_pic
%%%fbauth_pic portrait
%%%fbauth_pic background
%%%fbauth_pic other
%%%fbauth_tags
%%%fbauth_pubs
%%%endfbauth
 

До слова, для тих невіруючих, що не здатні прийняти, те що люди придумують
слова, мій синок, в три рочки на кожну річ вигадував свою назву, шкода, що не
записувала, збагатила б мову, можливо колись би сперечалися, що таке вже
існувало. Людям властиво думати, придумувати, творити!

\begin{itemize}
%%%fbauth
%%%fbauth_name
\iusr{Віталій Чорнобай-Соколовський}
%%%fbauth_url
%%%fbauth_place
%%%fbauth_id
%%%fbauth_front
%%%fbauth_desc
%%%fbauth_www
%%%fbauth_pic
%%%fbauth_pic portrait
%%%fbauth_pic background
%%%fbauth_pic other
%%%fbauth_tags
%%%fbauth_pubs
%%%endfbauth
 
\textbf{Luba KOtys} І мій. Віник-заміталівка, шукати загублене-виховувати.
\end{itemize}

%%%fbauth
%%%fbauth_name
\iusr{Людмила Чернюк}
%%%fbauth_url
%%%fbauth_place
%%%fbauth_id
%%%fbauth_front
%%%fbauth_desc
%%%fbauth_www
%%%fbauth_pic
%%%fbauth_pic portrait
%%%fbauth_pic background
%%%fbauth_pic other
%%%fbauth_tags
%%%fbauth_pubs
%%%endfbauth
 
Дякую за інформацію. Дуже цікаво.

%%%fbauth
%%%fbauth_name
\iusr{Яна Устимко}
%%%fbauth_url
%%%fbauth_place
%%%fbauth_id
%%%fbauth_front
%%%fbauth_desc
%%%fbauth_www
%%%fbauth_pic
%%%fbauth_pic portrait
%%%fbauth_pic background
%%%fbauth_pic other
%%%fbauth_tags
%%%fbauth_pubs
%%%endfbauth
 

щось то трохи не так. цікаво, знати джерело. автор трохи прибрехав)), слова
"промінь" і "квітник" є у творах Франка, який старший за Л, У. і Нечуя
Левицького. "Нумо ж, дідусю! Хапай за ті поводи, з променю ткані," ( Мійові
елегіїї") "Бо, властиво, о що ходить? Цивілізація – се квітник". Якщо поритися,
можна изнайтий інші байки щодо авторства слів. ). Мордуватися - поширене слово
в Галичині за А. У і Польщі. Сміливий - теж є в польській, а значить і в
галицькій мові було. Взагалі, варто уточнити, бо в часи розділення України
двома імперіями паралельно існували 2 мови - галицька і так звана
великоукраїнська. тому приписувати введення слів, які вживалися в галицькій
мові, яка теж є частиною української, неправильно.


%%%fbauth
%%%fbauth_name
\iusr{Kateryna Minaieva}
%%%fbauth_url
%%%fbauth_place
%%%fbauth_id
%%%fbauth_front
%%%fbauth_desc
%%%fbauth_www
%%%fbauth_pic
%%%fbauth_pic portrait
%%%fbauth_pic background
%%%fbauth_pic other
%%%fbauth_tags
%%%fbauth_pubs
%%%endfbauth
 
А як ввідбуваються такі процеси? Чи одразу неологізм, який придумав письменик_ця, входить в офіційну мову?

\begin{itemize}
%%%fbauth
%%%fbauth_name
\iusr{Людмила Гаценко}
%%%fbauth_url
%%%fbauth_place
%%%fbauth_id
%%%fbauth_front
%%%fbauth_desc
%%%fbauth_www
%%%fbauth_pic
%%%fbauth_pic portrait
%%%fbauth_pic background
%%%fbauth_pic other
%%%fbauth_tags
%%%fbauth_pubs
%%%endfbauth
 
\textbf{Kateryna Minaieva} часто спершу у розмовну, потім до словників.
\end{itemize}

%%%fbauth
%%%fbauth_name
\iusr{Олена Єрьоменко}
%%%fbauth_url
%%%fbauth_place
%%%fbauth_id
%%%fbauth_front
%%%fbauth_desc
%%%fbauth_www
%%%fbauth_pic
%%%fbauth_pic portrait
%%%fbauth_pic background
%%%fbauth_pic other
%%%fbauth_tags
%%%fbauth_pubs
%%%endfbauth
 
Да, цікава інформація і диспут тут у вас серйозний.

\end{itemize}

