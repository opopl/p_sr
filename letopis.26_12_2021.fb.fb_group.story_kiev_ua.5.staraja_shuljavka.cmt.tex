% vim: keymap=russian-jcukenwin
%%beginhead 
 
%%file 26_12_2021.fb.fb_group.story_kiev_ua.5.staraja_shuljavka.cmt
%%parent 26_12_2021.fb.fb_group.story_kiev_ua.5.staraja_shuljavka
 
%%url 
 
%%author_id 
%%date 
 
%%tags 
%%title 
 
%%endhead 
\zzSecCmt

\begin{itemize} % {
\iusr{Natasha Levitskaya}

Спасибо, Георгий! Как всегда, очень интересно о ваших предках! Все ваши
публикации в группе, это такая далёкая история вашей семьи, документы, фото и
главное, ваши поиски и знание этой истории.

Просто респект, как сейчас говорят и уважение вам! Спасибо! @igg{fbicon.hands.applause.yellow} 

\begin{itemize} % {
\iusr{Георгий Майоренко}
\textbf{Natasha Levitskaya} Благодарю. Некоторые фото обнаружились совсем недавно. Лежали в закромах и мама вдруг их нашла! А в них - эпоха!
\end{itemize} % }

\iusr{Людмила Ива}

Огромное спасибо, Георгий! Очень интересные воспоминания! Легко и приятно
читать! @igg{fbicon.rose} 

\begin{itemize} % {
\iusr{Георгий Майоренко}
\textbf{Людмила Ива} Спасибо, с наступающим! Удачи и добра!

\iusr{Людмила Ива}
\textbf{Георгий Майоренко} взаимно! Здоровья и благополучия Вам!)
\end{itemize} % }

\iusr{Надежда Кулиш}

\ifcmt
  ig https://i2.paste.pics/ebe47a0be53cf220cc6bd46c44e9c4c8.png
  @width 0.2
\fi

\begin{itemize} % {
\iusr{Георгий Майоренко}
\textbf{Надежда Кулиш}

\ifcmt
  ig https://scontent-mxp2-1.xx.fbcdn.net/v/t39.1997-6/s168x128/17527812_1652591011433966_4391041969200037888_n.png?_nc_cat=1&ccb=1-5&_nc_sid=ac3552&_nc_ohc=5c8vh81yn3oAX90m1uL&_nc_ht=scontent-mxp2-1.xx&oh=00_AT846U1f9Qe7omF6o7C2gmjQPg1WJPzH_rpqKDPUws5xOA&oe=61DA0558
  @width 0.1
\fi

\begin{itemize} % {
\iusr{Надежда Кулиш}
\textbf{Георгий Майоренко} Училась в школе на Шулявке

\iusr{Георгий Майоренко}
\textbf{Надежда Кулиш} Это в какой? Отец мой учился в той, что рядом с кинотеатром Довженко, а я - до 5 класса в 41.

\iusr{Надежда Кулиш}
\textbf{Георгий Майоренко} В 74, на Смоленской ул. возле КПИ

\iusr{Lyubov Pakholchenko}
\textbf{Надежда Кулиш} Моя мама закончила 74 школу перед самой войной. У меня сохранились ее табеля за все классы. После освобождения Киева в этой школе был госпиталь, потом опять школа.
\end{itemize} % }

\end{itemize} % }

\iusr{Olena Kharkevich}
Очень интересно! Спасибо!

\iusr{Георгий Майоренко}
\textbf{Olena Kharkevich} Спасибо за добрые слова! Успехов и позитива!

\iusr{Iryna Naidonova}
Благодарю!

\iusr{Георгий Майоренко}
\textbf{Iryna Naidonova} Спасибо! Удачи и достатка в новом году!

\iusr{Виктор Задворнов}

В течение 1990-2005-х \enquote{на базе} международного рекламного агентства \enquote{Посредник}
выходила районная газетв \enquote{Шулявка} с еженедельной ТВ-программой. Всегда с
удовольствием редактор Евгений Хорунжий публиковал на страницах газеты
исторические очерки о правобережном Киеве. А уж о Соломенке, Шулявке, Галаганах
- особенно.

\begin{itemize} % {
\iusr{Георгий Майоренко}
\textbf{Виктор Задворнов} Помню эту газету! Это ещё во времена \enquote{Авизо}, \enquote{Экспресс-объявы}. Доинтернетная эпоха!

\iusr{Виктор Задворнов}
\textbf{Георгий Майоренко} газета \enquote{Шулявка} поддерживалась отчасти администрацией Жовтневого района. Очень по душе пришлась из-за дешевизны пенсионерам и жителям Правобережья.

\iusr{Георгий Майоренко}
\textbf{Виктор Задворнов}

\ifcmt
  ig https://i2.paste.pics/f6789ca517f4a844329db36ce7368ef1.png
  @width 0.2
\fi

\iusr{Виктор Задворнов}
\textbf{Георгий Майоренко} И Вам спасибо за воспоминания.

\end{itemize} % }

\iusr{Svitlana Plieshch}

Великолепная история, познавательная, интересная и глубокая. Столько истории в
коротком рассказе. Училась в КПИ, район знакомый, многого не знала. Огромная
благодарность


\iusr{Георгий Майоренко}
\textbf{Svitlana Plieshch} Благодарю за добрые слова. Удачи и добра!

\iusr{Раиса Карчевская}
Большое спасибо, Георгий!
Как всегда очень интересно

\begin{itemize} % {
\iusr{Георгий Майоренко}
\textbf{Раиса Карчевская} Благодарю! Удачи и добра в новом году!

\iusr{Раиса Карчевская}
\textbf{Георгий Майоренко} Спасибо большое - взаимно
\end{itemize} % }

\iusr{Георгий Майоренко}
Благодарю! С наступающим! Удачи в новом году!

\iusr{Раиса Карчевская}
Взаимно. Жду новых публикаций

\begin{itemize} % {
\iusr{Георгий Майоренко}
\textbf{Раиса Карчевская} Спасибо огромное за добрые слова! Новые публикации в новом году!

\iusr{Раиса Карчевская}
\textbf{Георгий Майоренко}

\ifcmt
  ig https://i2.paste.pics/555ed7238379c69093d9f56c9c201872.png
  @width 0.2
\fi

\end{itemize} % }

\iusr{Светлана Мищенко}
Очень интересно, ярко. описывает район, где жили ваши родные и выросли вы.
Эти места мне тоже близки, на территории КПИ вырос мой отец, окончил КПИ.
Дальнейших вам успехов в творчестве

\begin{itemize} % {
\iusr{Георгий Майоренко}
\textbf{Светлана Мищенко} 

Ну так от Шулявки до Львовской (ваши родные места) было рукой подать! Недавно
обнаружил, что бабушку мою крестили в Златоустовской церкви на Евбазе! Нашел
запись в метрической книге. Киев тогда был не такой огромный, как сейчас.

\iusr{Ирина Панченко}
\textbf{Георгий Майоренко} 

Моя бабушка училась в техникуме (или не знаю как училище тогда называлось) на
Львовской площади, так рассказывала, что ходила пешком туда и обратно. (с Шулявки)

\iusr{Георгий Майоренко}
\textbf{Ирина Панченко} 

Приветствую, Ирина! А ваши предки, если я не ошибаюсь, жили до революции на
Борщаговской 27? Янковский Людвиг Францевич?

\end{itemize} % }

\iusr{Татьяна Соловьева}

Помните, у Ильфа и Петрова \enquote{Одноэтажная Америка}? Ваша \enquote{Одноэтажная Шулявка} -
не менее увлекательна!

\begin{itemize} % {
\iusr{Георгий Майоренко}
\textbf{Татьяна Соловьева} 

Ильф и Петров это - супер! У них и фельетоны шикарные. Когда-то попалось их
подписное издание, читал с упоением! Чего стоила сатирическая стенгазета из
банно-прачечного комбината \enquote{Мочалкой по черепу!}. А отец мой так
зачитал \enquote{12 стульев} и \enquote{Золотого теленка}, что книга
превратилась в ветошь! Все родственники хохотали.


\iusr{Галина Компанченко}
\textbf{Георгий Майоренко} Обажала это произведение и я.. А вот \enquote{Однооэтажная Америка}, прочлась мною единажды..

\iusr{Георгий Майоренко}
\textbf{Галина Компанченко} Надо будет перечитать. Классику можно читать бесконечно.
\end{itemize} % }

\iusr{Константин Тюрменко}
Жил на Борщаговской 159. Теперь этого дома нет. Его мой отец построил

\iusr{Caty Si}

Спасибо большое за такие интересные исторические факты моей родной Шулявки!
Часть из вашего рассказа уже мне знакома от моего дедушки, который всю свою
жизнь (примерно с конца 50х) проработал на заводе Петровского и получил от него
квартиру в своих же \enquote{заводских} домах. С середины 70х и по сей день мы с
родителями являемся коренными жителями Шулявки @igg{fbicon.face.smiling.eyes.smiling}.

\begin{itemize} % {
\iusr{Георгий Майоренко}
\textbf{Caty Si} Благодарю за добрые слова! Удачи и здоровья в новом году!

\iusr{Caty Si}
\textbf{Георгий Майоренко} Вам здоровья и благополучия в Новом году и, конечно же, вдохновения на новые интереснейшие истории! @igg{fbicon.face.smiling.eyes.smiling} 
\end{itemize} % }

\iusr{Арт Юрковская}

Очень интересно! И про революцию интересно. А то сейчас получается, что ее
сделала высадка с Марса. Меня всегда интересовал вопрос куда исчезли потомки
революционеров, чекистов, петлюровцев и немецких полицаев? В основном потомки
Прони Прокоповны, князей и раскулаченных делятся воспоминаниями. Первые
проклинают большевиков, которые лишили их поместий и балов, а вторые за то, что
увели единственную корову (вариант - лошадь) в колхоз. А где же все остальные?

\begin{itemize} % {
\iusr{Георгий Майоренко}
\textbf{Арт Юрковская} 

Увы, сейчас не приветствуется объективный взгляд на исторические события. Как в анекдоте.

Прапорщик объявляет солдатам :

-Сейчас все пойдут грузить люмимний!

Один из строя:

- Не люмимний, а алюминий?

-Сказано - люмимний, значит люмимний!

А самый умный пойдет грузить г...но.

\iusr{Ірина Кравець}
\textbf{Арт Юрковская} 

в одном из Ваших откровений Вы сообщили, что предок воевал у Петлюры. Так что, Вы
же пишите о трофейных вещах, как о нечто драгоценном, а по сути о мародёрах. Не
скажите... Разные воспоминания у киевлян. И почему не пишут? Не дано - это
первое Унаследовать надо было. 2) - потомки непишущих ещё и в школе плохо учились.


\iusr{Эдуард Джафаров}
\textbf{Арт Юрковская} , да и после революции если кто и сидел, то только по политической статье. За пошлую уголовщину деды нибожежмой!!)
\end{itemize} % }


\iusr{Людмила Козак}
СПАСИБО!

\iusr{Георгий Майоренко}
\textbf{Людмила Козак}

\ifcmt
  ig https://scontent-mxp2-1.xx.fbcdn.net/v/t39.1997-6/s168x128/17527812_1652591011433966_4391041969200037888_n.png?_nc_cat=1&ccb=1-5&_nc_sid=ac3552&_nc_ohc=5c8vh81yn3oAX90m1uL&_nc_ht=scontent-mxp2-1.xx&oh=00_AT846U1f9Qe7omF6o7C2gmjQPg1WJPzH_rpqKDPUws5xOA&oe=61DA0558
  @width 0.1
\fi


\iusr{Нина Гордийчук}
Спасибо за интересный рассказ.

\iusr{Георгий Майоренко}
\textbf{Нина Гордийчук} Благодарю за добрые слова! С наступающим!

\iusr{Сергій Савелій}

Цікаво. Випало і мені доторкнутися, щоправда, до сучасної Шулявки. У середині
80-х жив у малосімейці на Довженка, відразу за будинком з тилу заходив гуляти в
Пушкінський парк з донькою, а старший син ходив до школи на Ванди Василевської
( нині Богдана Гаврилишина), відразу за кінотеатром Олександра Довженка (
нині,на жаль, колишнього)...

\begin{itemize} % {
\iusr{Георгий Майоренко}
\textbf{Сергій Савелій} Рiднi мiсця! Успiхiв в новому роцi!

\iusr{Сергій Савелій}
\textbf{Георгий Майоренко} Навзаєм!

\iusr{Юрий Панчук}
\textbf{Сергій Савелій} 102а школа на вулиці Шулявській, яка перетинає Богдана Гаврилишина, я теж там навчався ))

\iusr{Сергій Савелій}
\textbf{Yury Panchuk} Запам'ятав класну керівничку сина, така вредненька була, в літах, чи Ганна Лук'янівна, чи Галина Лук'янівна...
\end{itemize} % }

\iusr{Владимир Дубровский}
Я уверен, что в 7 или 8 веке жил человек по прозвищу Шелява, что посадил лес.

\iusr{Георгий Майоренко}
\textbf{Владимир Дубровский} Хорошая версия про Шеляву! А он упоминается в каких-то летописях?

\iusr{Oksana Perederiy}
Мой дедушка - Чепурковский Антон Степанович, был настоятелем Марие-Магдалиновской церкви.

\begin{itemize} % {
\iusr{Георгий Майоренко}
\textbf{Oksana Perederiy} Как интересно! А дедушкин архив не сохранился?

\iusr{Maria Minyaeva}
\textbf{Oksana Perederiy} с моей мамой училась Ира Чепурковская в 74 школе

\iusr{Oksana Perederiy}
\textbf{Maria Minyaeva} Ира Чепурковская - моя сестра. Ей 85 лет. А фамилия мамы?

\iusr{Maria Minyaeva}
\textbf{Oksana Perederiy} Галина Поперенко, ей тоже 85  @igg{fbicon.smile}  Передавайте Ирине привет от моей мамы!

\iusr{Maria Minyaeva}
\textbf{Oksana Perederiy} Если хотите, можем списаться в личке подробнее
\end{itemize} % }

\iusr{Lola Shpilskaya}

Шулявский двор. Фото довоенное, приблизительно 1936 года или чуть позже.
Девочка в шапочке на руках (возможно родственница) это моя мама

\ifcmt
  ig https://scontent-mxp2-1.xx.fbcdn.net/v/t39.30808-6/270016191_4291968530909806_8647595297430959391_n.jpg?_nc_cat=103&ccb=1-5&_nc_sid=dbeb18&_nc_ohc=8k83yG2HL7MAX_D6iRl&_nc_ht=scontent-mxp2-1.xx&oh=00_AT8W4HfF_PjWiMfPiDIMWfrOrbED4lRBETlsx3XRy0tJ7Q&oe=61DA8B46
  @width 0.3
\fi

\begin{itemize} % {
\iusr{Георгий Майоренко}
\textbf{Lola Shpilskaya} Как здорово, что фото сохранилось! Такие довоенные фотографии на вес золота!

\iusr{Lola Shpilskaya}
\textbf{Георгий Майоренко} 

у меня есть фотографии прабабушки с детьми, прадедушки с одним из сыновей. А
все собираюсь рассказать о семье, но что-то не складывается. А Вы очень красиво
рассказали о Шулявке


\iusr{Георгий Майоренко}
\textbf{Lola Shpilskaya} 

Как здорово, что фотографии сохранились! Если опубликуете, будет замечательно.

\iusr{Tanja Macsimenco}

Шулявка-мой родной район. Рядом со 142школой, где я училась, сохранился, как мы
называли, \enquote{поповский дом}. Саму церковь Марии-Магдалины разрушили и на ее месте
построили райком. Вокруг него сохранился поповский сад с грушами, орехами,
кустами смородины. В детстве мы туда лазили через забор. А в поповском доме
сделали Дом пионеров с разными кружками. Я ходила в кукольный кружок. Мы ставили
спектакль \enquote{Волк и семеро козлят}, сами учились делать куклы. Там еще был
фотокружок, кружок кройки и шитья, был струнный оркестр. Прекрасно помню
Борщаговка. На ней и прилегающих переулках жили мои одноклассники. Моя тетя жила
на улице Светлой в частном доме. На Шулявку замечательные парки:Пушкинский(как
много там было развлечений. Летний кинотеатр, летняя эстрада, качели, карусели,
шахматный клуб, бильярдная, танцплощадка) Политехнический парк (какие там
проходили выпускные вечера Политехнического института!), Зоопарк. А кинотеатр
им. Довженка. Там показывали самые новые фильмы - наши и зарубежные. И первые
станции метро тоже начинали отсюда. Политех, Большевик. Да забыла сказать - в мой
родной Серый дом напротив Метро. Из окна смотрела как его строили. Много можно
вспоминать.

\iusr{Maria Minyaeva}
\textbf{Tanja Macsimenco} Моя мама до сих пор называет \enquote{серый дом}

\end{itemize} % }

\iusr{Юлия Шеремет}
Дуже цікаво написано, і з повагою і любов'ю!

\iusr{Георгий Майоренко}
\textbf{Юлия Шеремет} Спасибо, Юлия! С наступающим!

\iusr{Владимир Новицкий}
Спасибо, Вы очень интересно излагает события, любим Вас читать!!

\iusr{Георгий Майоренко}
\textbf{Владимир Новицкий} Благодарю, уважаемый Владимир. Ваше внимание особенно дорого! Здоровья и благополучия в новом году!

\iusr{Светлана Здор}

Боже мой, как интересно. Тем более, так сложилось, что мы живем с некоторых
пор (10лет) на Шулявке, возле Индустриального. И так приятно прочитать живую
историю этого района Киева. Восхищаюсь такими талантливым и преданным людьми.
Спасибо.

\iusr{Георгий Майоренко}
\textbf{Светлана Здор} 

Благодарю. К огромному сожалению, историей Шулявки никто вплотную не занимался.
И там много белых пятен и загадок.


\iusr{Margarita Kaminsky}

До 1988 года работала на Шулявке в 102 школе (за кинотеатром Довженко).

\begin{itemize} % {
\iusr{Георгий Майоренко}
\textbf{Margarita Kaminsky} В этой школе отец учился. А я - в 41.

\iusr{Юрий Панчук}
\textbf{Margarita Kaminsky} Я Вас помню, русский язык и литература, а еще вы работали библиотекарем. Всего вам наилучшего, здоровья и благополучия ))

\iusr{Margarita Kaminsky}
\textbf{Yury Panchuk}
Спасибо, Юра! Всё правильно. И Вам всех благ!
\end{itemize} % }

\iusr{Роза Миколаївна Кириченко}

А мой дедушка Соловей Никита Ефимович до революции тоже служил в этой церкви.
Возможно они служили одноаременно

\begin{itemize} % {
\iusr{Георгий Майоренко}
\textbf{Роза Миколаївна Кириченко} 

О, это известная фамилия на Шулявке! Как начнёшь листать справочники и
метрические книги, часто встречается. Многих детей крестил \enquote{на районе}!

\iusr{Роза Миколаївна Кириченко}
\textbf{Георгий Майоренко} Спасибо. Но я пока ничего не знаю о нем.

\iusr{Георгий Майоренко}
\textbf{Роза Миколаївна Кириченко} Это поправимо.

\iusr{Роза Миколаївна Кириченко}
\textbf{Георгий Майоренко} надеюсь
\end{itemize} % }

\iusr{Роза Миколаївна Кириченко}
Спасибо за очень интересный рассказ.

\begin{itemize} % {
\iusr{Георгий Майоренко}
\textbf{Роза Миколаївна Кириченко} Удачи и благополучия в новом году! Здорово, что ваши жили рядом с нашими в давние времена! И были прихожанинами одной церкви.

\iusr{Роза Миколаївна Кириченко}
\textbf{Георгий Майоренко} возможно и знали друг друга. И Вам всего наилучшего в новом году.

\iusr{Георгий Майоренко}
\textbf{Роза Миколаївна Кириченко} Где-то виделись, так это точно!
\end{itemize} % }

\iusr{Татьяна Косолапова}

Спасибо... воспоминания о Шулявке. Я тоже в 60-десятых годах жила там, в доме
специалистов...

\begin{itemize} % {
\iusr{Георгий Майоренко}
\textbf{Татьяна Косолапова} 

Колоритный был район. Я эту одноэтажную Шулявку помню смутно.
Через пару лет после моего рождения ее снесли и построили новые
дома. А отец мой помнил, когда в речке Лыбеди рыбу ловили, и на
плотах по ней плавали.

\end{itemize} % }

\iusr{Мария Маслова}

Спасибо, очень интересно и всесторонне!

Моя тетя много лет прожила жила на Выборгской, это недалеко от описываемых Вами
мест. И Татьяну Гурьеву помню по порталу «Что хочет автор» и его отделению
«Пишущая Украина».

\iusr{Георгий Майоренко}
\textbf{Мария Маслова} Благодарю! Бывал на Выборгской!

\iusr{Lyubov Pakholchenko}

Мои родственники по линии бабушки - Жаворонковы, жили на ул. Шулявской.
Песчаная, Борщаговская, Полевая, Фабричная - все названия улиц помню с
детства. По рассказам бабушки в Кадетскую рощу ходили на майовку с самоваром. У
бабушки было много подруг на Шулявке. Пушкинский парк, зоопарк, больница
Калинина, парк Политеха - Шулявка родина моя.

\begin{itemize} % {
\iusr{Георгий Майоренко}
\textbf{Lyubov Pakholchenko} Замечательно! Наши люди! Здоровья и удачи в новом году!

\iusr{Lyubov Pakholchenko}
\textbf{Георгий Майоренко} Спасибо! Взаимно поздравляю Вас и всех ваших и наших!!!!!
\end{itemize} % }

\iusr{Людмила Шова}
Очень интересно.
Мои предки тоже с Шулявки.
Бабушка Серафима жила на Борщаговской 100 и работала в КПИ.
А Мама заканчивала 74 школу и потом училась в КПИ.


\end{itemize} % }
