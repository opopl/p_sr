% vim: keymap=russian-jcukenwin
%%beginhead 
 
%%file 28_12_2021.fb.fb_group.story_kiev_ua.2.zolotye_vorota.cmt
%%parent 28_12_2021.fb.fb_group.story_kiev_ua.2.zolotye_vorota
 
%%url 
 
%%author_id 
%%date 
 
%%tags 
%%title 
 
%%endhead 
\zzSecCmt

\begin{itemize} % {
\iusr{Вадим Сандино}

Может перевели неправильно! Может наборщик и печатник напились и неправильно
строки поставили! Может летописец увидел красивую женщину и перо пошло не в ту
сторону! Что говорит Толочко, Бебик и музейники из Софии Киевской... Их там
человек 300 сидит???

\begin{itemize} % {
\iusr{Надежда Владимир Федько}
\textbf{Вадим Сандино} Я сканував оригінал довідника!

\iusr{Вадим Сандино}
\textbf{Надежда Владимир Федько} так я и об оригинале и говорю!

\iusr{Надежда Владимир Федько}
\textbf{Вадим Сандино Куліженко} був достатньо відповідальний видавець.
\end{itemize} % }

\iusr{Вика Нурибекова}

\ifcmt
  ig https://scontent-mxp2-1.xx.fbcdn.net/v/t39.30808-6/270289382_136930995425048_3019540093643024827_n.jpg?_nc_cat=110&ccb=1-5&_nc_sid=dbeb18&_nc_ohc=BDatcAsnhssAX9diAGI&_nc_ht=scontent-mxp2-1.xx&oh=00_AT_JDjBABOahDg902o_cJ2E8MukqTolvW7lhRPYuFQuwyA&oe=61DA7B92
  @width 0.2
\fi

\iusr{Ирина Яблочкова}

Если отталкиваться полностью от текста , то, оказывается \enquote{... Херсонес был
колыбелью РУССКОГО христианства и исскуства}... стр. 11... В свое время я
подарила \enquote{Музею улицы Крещатик} свой экземппляр книги-экскурсии по Киеву более
раннего издания (как мне помнится, 1905 года, могу немного ошибаться. Кстати, в
этот музей я передала и старинный патефон с пластинками. Судьба его дальнейшая
мне не известна). Так вот в этом издании я такого текста не помню....


\iusr{Алла Парадня}

Дякую дуже за Вашу розповідь про це визначне для нас усіх місце. Сприймаю її
якось особливо близько, бо місце це пов язано ще і з нашим дитинством, ми жили
на Золотоворітській вулиці, тобто метрах у 50 від самих воріт. У садику біля
воріт ми гуляли будучи дітьми, там ріс і мій син. А мене малою чомусь так
тягнуло туди, що я збирала камінці зі стін и викладала їх у коробках з-під
конфет на ваті. А одна загадка мучила мене весь час, навіть і досі. Коли у
50-51 роках проводили у Києві газ, вздовж вулиць копали траншеї, поруч з
тротуаром. І от одного разу всі були страшно здивовані, на глибині близько
метра з обох боків ями виявились численні ряди отворів, в деяких були чиїсь
останки. Тобто це були організовані ряди захоронень. Місце це огородили, через
деякий час приїхали вчені, зібрали, що могли, закрили все асфальтом. А я все
шукаю розгадки, що це було. Адже кладовища не могло бути у самому центрі города
перед Золотими воротами, хоронили як заведено за городом. Чи є якесь пояснення
цьому?

\begin{itemize} % {
\iusr{Надежда Владимир Федько}
\textbf{Алла Парадня} НЕ знаю. Моя тематика - військова історія ХХ століття.
\end{itemize} % }

\iusr{Tetyana Kilesso}
Раджу прочитати книжку \enquote{Золоті ворота в Києві} Сергія Олександровича Висоцького.

\iusr{Микола Гончаренко}
Це могли буть поховання ранішого періоду

\iusr{Микола Гончаренко}
Коли місто закінчувалося воротами на Артема

\iusr{Оксана Можаровская}
Дякую!

\iusr{Галина Полякова}
Замечательно! Спасибо!

\iusr{Геннадий Харченко}

На фотографиях Крещатика ХІХ века в далеке видны целые ворота. Стоит дом рядом
с ЦУМом, с огромными арками. В те годы на его месте была столовая для рабочих,
а еще раньше дом какой-то. Я думаю их в ХХ веке разрушили.

\begin{itemize} % {
\iusr{Надежда Владимир Федько}
\textbf{Геннадий Харченко} А фотографії де???

\iusr{Геннадий Харченко}

\ifcmt
  ig https://scontent-mxp2-1.xx.fbcdn.net/v/t39.30808-6/270713617_4876629739067588_2686019656491991206_n.jpg?_nc_cat=108&ccb=1-5&_nc_sid=dbeb18&_nc_ohc=raPU6ZyFU-IAX_M9_OI&_nc_ht=scontent-mxp2-1.xx&oh=00_AT_s2X-yMQkO4oKKTsLfyYLtJuloS0K9P1Me3Fv_ig8osQ&oe=61D8CFF2
  @width 0.4
\fi
\end{itemize} % }

\iusr{Oleksandr Tarashchuk}
Ещё одна история ))


\ifcmt
  tab_begin cols=3,no_fig,resizebox=0.7

     pic https://scontent-mxp2-1.xx.fbcdn.net/v/t39.30808-6/270140666_1027295207850128_5471021491655266162_n.jpg?_nc_cat=104&ccb=1-5&_nc_sid=dbeb18&_nc_ohc=WfJjZYV026AAX-_h-qx&_nc_ht=scontent-mxp2-1.xx&oh=00_AT8oVzoSu5_dXikUxH7PCPntmD9LU4m2rdLeo5GYsnhknA&oe=61D99089

		 pic https://scontent-mxp2-1.xx.fbcdn.net/v/t39.30808-6/270038786_1027295327850116_812589157615660667_n.jpg?_nc_cat=106&ccb=1-5&_nc_sid=dbeb18&_nc_ohc=t2531lCboQcAX_q5-Bi&_nc_ht=scontent-mxp2-1.xx&oh=00_AT-CtmDke3CYLs2bLbPWh27pNNJ4TNKtiMoAzRZ0aeOg1w&oe=61D8E70C

		 pic https://scontent-mxp2-1.xx.fbcdn.net/v/t39.30808-6/270061909_1027295471183435_4909363988174395426_n.jpg?_nc_cat=105&ccb=1-5&_nc_sid=dbeb18&_nc_ohc=f7eXtUNIcZ4AX_-gYFa&_nc_ht=scontent-mxp2-1.xx&oh=00_AT83ThKWDReJcRr_grmSjv-unJCMP8G_TeVtD97W9mkGSg&oe=61D9F193

  tab_end
\fi

\begin{itemize} % {
\iusr{Оксана Салан}
\textbf{Oleksandr Tarashchuk} , от тільки цікаво, хто міг придумати зберігать княже золото (скарбницю?) у воротах міста, а не десь у внутрішніх дворах града княжого, хоромах княжих чи храмах божих ?

\iusr{Oleksandr Tarashchuk}
\textbf{Оксана Салан} , сховали в самому неочікуваному місці ))

\iusr{Алексей Шимановский}
\textbf{Оксана Салан} Пошлину за вход с купцов брали, там же и складывали. Таможенная касса. Время от времени инкассация, конечно. Скорее всего, это имелось в виду.
\end{itemize} % }

\iusr{Ванька Стахов}

Цікавий матеріал. Зайвий доказ того, що на мокшанські болота православ'я
прийшло з Києва а не навпаки.

\begin{itemize} % {
\iusr{Надежда Владимир Федько}
\textbf{Ванька Стахов} Що цікаво в старих ДОРЕВОЛЮЦІЙНИХ путівниках по Києву - Київське князівство було ПРИЄДНАНО до Московського! Тобто, ніякого "воз'єднання! не було!
\end{itemize} % }


\iusr{Надежда Владимир Федько}
З путівника по Києву 1886 року..

\ifcmt
  tab_begin cols=2,no_fig,resizebox=0.5
     pic https://i2.paste.pics/b94e96605d4b3afeeb9b9ef9d94db790.png
		 pic https://i2.paste.pics/dcd7aa06102bfc90eb8e452378465d97.png
  tab_end
\fi

\iusr{Lora Berdichevsky}
Какой Вы молодец! Очень интересно

\iusr{Лариса Давиденко}
Дякую вам за історичну інформацію!

\iusr{Татьяна Шиверская}
Интересно

\iusr{Таня Володарка}
Дуже вдячна. цікаво

\end{itemize} % }
