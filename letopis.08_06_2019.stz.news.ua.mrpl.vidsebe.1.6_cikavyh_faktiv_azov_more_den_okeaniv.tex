% vim: keymap=russian-jcukenwin
%%beginhead 
 
%%file 08_06_2019.stz.news.ua.mrpl.vidsebe.1.6_cikavyh_faktiv_azov_more_den_okeaniv
%%parent 08_06_2019
 
%%url https://vidsebe.media/read/istorii/34079/6-cikavih-faktiv-pro-azovske-more-do-vsesvitnogo-dnya-okeaniv
 
%%author_id news.ua.mrpl.vidsebe,petrushkina_tetjana.mariupol
%%date 
 
%%tags fakty,geografia,mariupol,more,more.azov,okean,priroda
%%title 6 цікавих фактів про Азовське море до Всесвітнього дня океанів
 
%%endhead 
 
\subsection{6 цікавих фактів про Азовське море до Всесвітнього дня океанів}
\label{sec:08_06_2019.stz.news.ua.mrpl.vidsebe.1.6_cikavyh_faktiv_azov_more_den_okeaniv}
 
\Purl{https://vidsebe.media/read/istorii/34079/6-cikavih-faktiv-pro-azovske-more-do-vsesvitnogo-dnya-okeaniv}
\ifcmt
 author_begin
   author_id news.ua.mrpl.vidsebe,petrushkina_tetjana.mariupol
 author_end
\fi

08.06.2019 17:25

\ifcmt
  ig https://i2.paste.pics/c45e14551dd09f529a1bf0cc048def37.png
  @wrap center
  @width 0.8
\fi

\begin{qqquote}
\Large
8 червня відзначається Всесвітній день океанів, а оскільки Азовське море є
частиною Світового океану, ми вирішили поділитися з вами цікавими фактами про
наше море.
\end{qqquote}

\ii{08_06_2019.stz.news.ua.mrpl.vidsebe.1.6_cikavyh_faktiv_azov_more_den_okeaniv.pic.1}

\textbf{1. \href{https://kirillovka.ks.ua/azov-sea}{Азовське море} – внутрішнє море басейну Атлантичного океану.} Так що, коли
наступного разу будете на маріупольському пляжі, вважайте, що ви купаєтесь не в
Меотійському болоті, а в Атлантичному океані, принаймні у віддаленій його
частині, бо Азовське море – найконтинентальніше море всієї планети.

Меотійським болотом наше море називали давні римляни, а от для давніх греків
воно було Меотійським озером. А на рубежі нової ери в Приазов'ї жили племена
меотів і вони називали своє море Темеринда, що означає \enquote{матір морів}. Одне й те
саме море – а такий різний смисл закладено в його назвах.

\textbf{2. В Азовське море впадає 19 річок}. При невеликій площі і малому об'ємі наше
море отримує досить багато річкової води – до 12\% об'єму води моря. Так каже
Wikipedia. Таке співвідношення – найбільше серед усіх інших морів.

Цей факт, а також утруднений водообмін з Чорним морем зробили Азовське море
втричі менш солоним за середні показники Світового океану. Однак протягом XX
століття практично всі більш-менш великі річки, що впадають в Азовське море,
були перегороджені греблями для створення водосховищ. Тож солоність Азовського
моря почала підвищуватися, і це становить велику проблему для морських
мешканців.

\ii{08_06_2019.stz.news.ua.mrpl.vidsebe.1.6_cikavyh_faktiv_azov_more_den_okeaniv.pic.2}

Також мала солоність дозволяє воді легко замерзати взимку, що утруднює
судноплавство, загрожує прибережним будівлям і шибайголовам, які лізуть на
тонку кригу, але забезпечує нас цікавою розвагою і красивими фото після
прогулянок по безпечній товстій кризі.

\ii{08_06_2019.stz.news.ua.mrpl.vidsebe.1.6_cikavyh_faktiv_azov_more_den_okeaniv.pic.3}

\textbf{3. Довжина берегової лінії Азовського моря – трохи менше 1,5 тис. км.}
Але нас, простих маріупольців, більше цікавлять ті частини берегової лінії,
звідки можна з насолодою пірнути у прохолодні хвилі. Отже, в Маріуполі довжина
пляжів складає 16 км. Офіційно є всього 3 пляжі: Лівобережний, Міський і
Піщанка. Але, звісно, місць для купання набагато більше.

Цього року рейтинг пляжів ще ніхто не робив, але найбільш вірогідно, що перше
місце зайняв би міський пляж в тій його частині, де Dream Pirs. Принаймні,
вільні лежаки там знайти важко майже впродовж усього дня. А за перші вихідні
літа, за даними
\href{https://mariupolrada.gov.ua/ru/news/u-vihidni-mariupolski-pljazhi-vidvidalo-blizko-53-tisjach-cholovik}{міськради}
усі маріупольські пляжі відвідали близько 53 тисяч осіб.

\ii{08_06_2019.stz.news.ua.mrpl.vidsebe.1.6_cikavyh_faktiv_azov_more_den_okeaniv.pic.4}

\textbf{4. Колір і прозорість Азовського моря залежать від кількості планктону
у воді.} Його багато, тому вода непрозора і зеленувата.

У 2015 році вчені завершили масштабне дослідження мікроскопічних організмів в
океанах і виявили, що планктон – це не тільки їжа для китів, а й надважлива
частина системи підтримування життя на Землі. Планктон забезпечує виробництво
50\% кисню за допомогою фотосинтезу. Звісно, учені досліджували не азовський
планктон, але хто знає, може наше зеленувате море забезпечує нас киснем так
само, як і великий синій океан.

Щоправда, коли вода занадто прогрівається, кількість фітопланктону збільшується
і тоді море \enquote{цвіте}. Надмірне \enquote{цвітіння} є дуже шкідливим, бо синьо-зелені
водорості живуть недовго і їх гниття відбувається зі споживанням кисню. В
результаті кисню не вистачає рибі і вона гине. А причиною цього є діяльність
людини. Синьо-зелені водорості, через надлишок яких вода \enquote{цвіте}, розмножуються
через забруднення води органічними речовинами, особливо пестицидами, які
змиваються в море з полів.

\ii{08_06_2019.stz.news.ua.mrpl.vidsebe.1.6_cikavyh_faktiv_azov_more_den_okeaniv.pic.5}

\textbf{5. Шкідливе \enquote{цвітіння} води – це проблема, але це тільки
наслідок ще більшої проблеми – забруднення моря.} За даними
\href{http://web.kpi.kharkov.ua/safetyofliving/wp-content/uploads/sites/171/2018/12/eas-2018.pdf}{наукового
дослідження},
\footnote{%
Збірник тез наукових доповідей Х - ї міжнародної науково-методичної конференції %
Міжнародної конференції EAS \enquote{Безпека людини у сучасних умовах} 6–7 грудня 2018 року, %
Україна, Харків, 2018%
} підприємства, що знаходяться в прибережній зоні Азовського моря,
скидають туди щорічно понад 1000 т сполук заліза, 300 т нафтопродуктів, 150 т
фенолів, близько 70 т солей цинку, 200 т сполук міді та свинцю. Щороку
підприємствами Маріуполя скидають в водойми близько 900 млн м куб стічних вод,
з них 87\% припадає на Металургійний комбінат \enquote{Азовсталь} — найбільший
забруднювач Азовського моря в межах Донецької області.

\ii{08_06_2019.stz.news.ua.mrpl.vidsebe.1.6_cikavyh_faktiv_azov_more_den_okeaniv.pic.6}

Всі маріупольці знають шлакову гору біля Лівобережного пляжу. Але не всі
знають, що навколишнє середовище в радіусі 200 км від місця поховання
металургійних відходів (тих самих шлакових відвалів) є забрудненим за
замовчанням.

Ще недавно Азовське море перевершувало за рибопродуктивністю з одиниці
площі в 6,5 разів Каспійське, в 40 разів Чорне, в 160 разів Середземне моря.
Але хімічне забруднення, незбалансованість господарської діяльності,
браконьєрство та інші чинники призвели до екологічної кризи. До речі, осетрових
в Азовському морі майже не лишилося і вилов цієї риби заборонений.

\ii{08_06_2019.stz.news.ua.mrpl.vidsebe.1.6_cikavyh_faktiv_azov_more_den_okeaniv.pic.7}

\textbf{6. Поки що на поверхні Азовського моря не плавають острови зі сміття і
пластику.} А шлунки азовських бичків ніхто, здається, ще не досліджував на
наявність мікропластику, який через рибу потрапляє і до організму людини. Але
це тільки справа часу, поки до наших локальних проблем не доєднаються проблеми
світового масштабу. Тож ми закликаємо усіх наших читачів дбати про навколишнє
середовище і, зокрема, про наше чудове Азовське море.

Адже Всесвітній день океанів запровадили, щоб нагадувати про ключову роль
Світового океану в повсякденному житті, щоб інформувати громадськість про вплив
людини на Світовий океан, щоб захистити його і, звісно, щоб помилуватися його
красою.

\ii{08_06_2019.stz.news.ua.mrpl.vidsebe.1.6_cikavyh_faktiv_azov_more_den_okeaniv.pic.8}

Наостанок для підняття настрою дозвольте розповісти цікавий факт з життя
Світового океану. Є ряд звуків з глибин океанів, які вчені досі не можуть
пояснити. \enquote{Bloop} (англ. рев, завивання) може бути найвідомішим підводним
звуком, виявленим в 1997 році гідрофонами, встановленими Національним
Управлінням Океанічних і атмосферних досліджень (NOAA). Це один з найгучніших
океанських звуків, коли-небудь записаних, та в той же час цей шум відповідає
підводному льодовому потоку - великому розлому айсберга. Вчені не знають
напевно, що або хто видавав цей звук.

\href{https://www.youtube.com/watch?v=r7Vnw9qmrf0}{%
The Bloop | Normal and 16x Speed (HD), AndrewJay's Investigations of the Unknown, youtube, 24.07.2017}

\ii{08_06_2019.stz.news.ua.mrpl.vidsebe.1.6_cikavyh_faktiv_azov_more_den_okeaniv.scr.bloop}

Оскільки Bloop імітують в тому чи іншому вигляді деякі морські тварини, люди
жартома приписали цей шум Ктулху - вигаданому монстру-восьминогу, створеному
письменником-фантастом Говардом Філліпсом Лавкрафтом у 1928 році. Гучність
звуку занадто велика для будь-кого з морських істот, відомих науці, але є і
безліч істот, які ще відкриті.

Фото: Тетяна Петрушкіна, Афіна Хаджинова, brainskewer.com, novosti.in.ua

\#факты \#Азовское море \#Океан
