% vim: keymap=russian-jcukenwin
%%beginhead 
 
%%file 22_05_2021.fb.bilchenko_evgenia.1.sekirka
%%parent 22_05_2021
 
%%url https://www.facebook.com/yevzhik/posts/3913202852048129
 
%%author Бильченко, Евгения
%%author_id bilchenko_evgenia
%%author_url 
 
%%tags lavra,ukraina
%%title БЖ. Секирка
 
%%endhead 
 
\subsection{БЖ. Секирка}
\label{sec:22_05_2021.fb.bilchenko_evgenia.1.sekirka}
\Purl{https://www.facebook.com/yevzhik/posts/3913202852048129}
\ifcmt
 author_begin
   author_id bilchenko_evgenia
 author_end
\fi

\ifcmt
  pic https://scontent-bos3-1.xx.fbcdn.net/v/t1.6435-9/189944833_3913202772048137_7713009696268107056_n.jpg?_nc_cat=109&ccb=1-3&_nc_sid=8bfeb9&_nc_ohc=WyX8YuvqX_IAX_HCOf1&_nc_ht=scontent-bos3-1.xx&oh=a1fa1d9d33e63de47c4f3573f862d1f5&oe=60CDD67A
\fi

Стихи - не молот, не долото, не дятлова дрель: сей киркой
Не раздробить бездушия камня мозгов кастрации.
Печерская Лавра на десять метров выше горы Секирка,
Но та же в ней радость и та же скорбь - голгофность, горластость, карцерность.

Ночью можно друг друга греть, скручиваясь и скучиваясь.
Днём подыхать на семи фрилансах, отбывая, как срок, портфолио.
Страстями Христовыми - мёрзлый май застит и ластит - тучами.
Досказана исповедь.Транки выпиты. Далее - спазмы-фобии.

Чередование двух желаний - дожить до утра и сдохнуть бы.
Друзья, живущие при режиме сносно, всей тьмы острожной
Не понимают, хотя сочувствуют мне и добрые:
Добрые, пуганые, родные, до рвотного осторожные.

При каждом слове со мной, бессознательно самоцензурируясь,
Они причиняют мне боль, ни с каким врагом не сопоставимую.
Моя Секирушка - дом родной, сквозь чёрную клеть зари струя,
Живая Дева, Живой Христос, и вава моя, и viva моя.

Наверное, лучше сидеть в тюрьме, с самим же собой сычуя,
Чуя течение лет, недель, месяцев - до Исхода.
Не думать о модных тряпках. Кафтаны носить и чуни.
Не провоцировать и не дёргаться. Ждать смерти и ждать свободы.

Руси моей Велесовой, Православной, Красной и Белой, Рэперской -
Разве когда-то в душу мою заползал червь сомненья, быть ли ей?
Буквами из "Обители", рельсами, рейсами - креп в горсти
Зажимать над собственным гробом до Воскресного Сверхсобытия.

Дозволили б мне за Ея соборность распять себя, так не дали:
Так рассудил Господь, испытанье послав мне круче.
Я снова ложусь с тобой спать впритык, мне снятся Дали и дали.
Холодно... Холодно... Пневмония... Дизентерия...   Ручки!

Ручки держи - на моем животе, на рёбрах, на лбу, на печени!
Это - уже не озноб, а нечто, похожее на смирение.
Ты - последняя моя печка, первый расстрельный перечень...
Коль уж пространство - неодолимо, значит, уйдем из времени.

22 мая 2021 г.

Фото: Константин Антипин. Секирная гора.
