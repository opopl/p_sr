%%beginhead 
 
%%file 09_09_2023.fb.mariupol.muzej.mkm.1.mariupol_ta_kinematograf
%%parent 09_09_2023
 
%%url https://www.facebook.com/100093184796939/posts/pfbid0VdkxSFVA5y1P5GpxkvkpDTSB8Wk5UyZKzwKDs2VDdMXyHhMEcNADAh8yEjNa8FPal
 
%%author_id mariupol.muzej.mkm
%%date 09_09_2023
 
%%tags 
%%title Маріуполь та кінематограф
 
%%endhead 

\subsection{Маріуполь та кінематограф}
\label{sec:09_09_2023.fb.mariupol.muzej.mkm.1.mariupol_ta_kinematograf}

\Purl{https://www.facebook.com/100093184796939/posts/pfbid0VdkxSFVA5y1P5GpxkvkpDTSB8Wk5UyZKzwKDs2VDdMXyHhMEcNADAh8yEjNa8FPal}
\ifcmt
 author_begin
   author_id mariupol.muzej.mkm
 author_end
\fi

🎬 Маріуполь та кінематограф. 

Саме про це хочеться говорити сьогодні, у другу суботу вересня, в день
українського кіно.🎉

✅️ З нашим містом пов'язане виробництво як документальних так і художніх
фільмів. Деякі стрічки свого часу були мега популярними та бунтівними, бо
висвітлювали гострі проблеми суспільства.

Але мова піде не про фільм \enquote{Маленька Віра}, що наробив багато галасу
наприкінці радянської епохи, який знімався в Маріуполі і про який, можливо,
зараз пригадали містяни. Сьогодні хочемо звернути увагу на маріупольське
аматорське кіно, створенню якого не могла завадити навіть відсутність
можливості придбати професійну відеокамеру в середині минулого століття.

📽У 1955 році маріуполець Георгій Сергійович Котельников зібрав власноруч
унікальний кінопроекційний апарат, яким можна було записувати \enquote{домашнє} кіно на
35-міліметрову фотоплівку. Камера виглядала як невеличка скринька з
полірованого дерева та металу, з двома ручками на корпусі й конусоподібним
видошукачем.

Вже тоді Георгій Сергійович захоплювався літературою про кіноприладдя, роботу
кінорежисерів та кінооператорів. Інженер за фахом, у березні 1975 року він
заснував та протягом 40 років очолював радіо-конструкторський гурток при
міському будинку піонерів. А у 1950-60-і роки активно займався кінозйомкою,
формував любительський фото- та відеоархів. Зі своєю камерою Георгій
Котельников майже не розлучався, знімав під час туристичних подорожей по
Кавказу, Алтаю, Тянь-Шаню, Байкалу. Долучився він і до роботи аматорських
кіностудій міста.

📍У 1960-70-х роках аматорські кіногуртки та кіностудії активно розвивались
майже на всіх великих  підприємствах м. Жданова. Кінохроніки зафіксували життя
заводу ім. Ілліча, заводу \enquote{Азовсталь}, Жданівського заводу важкого
машинобудування, Жданівського коксохімічного заводу, Азовського морського
пароплавства та ін.

Заводські кіностудії створювали та очолювали активісти-кіно\hyp{}любителі, які
пройшли школу аматорської кіностудії Жданівського металургійного інституту –
\enquote{ЖМИ-фильм}, створеної у березні 1959 р. За це Сергій Давидович Буров у 1967
році назвав кіностудію інституту кінофакультетом. Крізь цю школу пройшли Сергій
Буров, Михайло Резченко, Григорій Фідельман та ін.

📹📺В місті випускався щоквартальний кіножурнал \enquote{Приазовский экран}, сюжети
якого монтували з матеріалів, наданих кіноаматорськими студіями. Георгій
Котельников брав участь у написанні сценаріїв та зйомках кількох сюжетів.

Кінострічки того періоду були здебільшого документальними, але відображали не
лише індустріальні досягнення, а й важливі події в житті міста, такі як
спортивні досягнення чи відкриття меморіального музею А. О. Жданова.

🙌Як же склалася доля саморобної кінокамери Георгія Котельникова? Вона стала
одним з найяскравіших експонатів виставки \enquote{Суспільство з кіноапаратом}, що
демонструвалась восени 2021 року у Львові в приміщенні Центру міської історії.
На 2022 рік планувалося експонування кіноапарату в Довженко-Центрі в Києві, а
далі Георгій Сергійович мріяв про виставку в Маріупольському краєзнавчому
музеї, з яким мав дружні стосунки. 

😓Проте, повномасштабна війна завадила здійсненню цих пла\hyp{}нів, наразі унікальний
експонат поки що зберігається у львівському Центрі міської історії.

Переглянути уривок кіножурналу \enquote{Приазовский экран} та уривки деяких
робіт міських кіноаматорів можна на Маріупольському парку пам'яті: 

\url{https://www.mariupolmemorypark.space/library/kinoamatorky-mariupolia}

\#mkmmariupol \#mariupol \#маріупольськиймузей \#маріуполь \#відродимомаріупольськиймузей \#кінематограф

\ifcmt
  ig https://i2.paste.pics/eb11adcb4f41298206c32bb83acc4f98.png
  @wrap center
  @width 0.9
\fi
