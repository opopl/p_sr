% vim: keymap=russian-jcukenwin
%%beginhead 
 
%%file 15_05_2018.fb.lesev_igor.1.vudu
%%parent 15_05_2018
 
%%url https://www.facebook.com/permalink.php?story_fbid=1920133071351120&id=100000633379839
 
%%author_id lesev_igor
%%date 
 
%%tags narod,obschestvo,strana,ukraina
%%title Такие себе вуду, приковылявшие кто из Шумера, а кто из русских былин
 
%%endhead 
 
\subsection{Такие себе вуду, приковылявшие кто из Шумера, а кто из русских былин}
\label{sec:15_05_2018.fb.lesev_igor.1.vudu}
 
\Purl{https://www.facebook.com/permalink.php?story_fbid=1920133071351120&id=100000633379839}
\ifcmt
 author_begin
   author_id lesev_igor
 author_end
\fi

Удивительна способность жителей одной древне-европейской страны жить чужими
жизнями, победами и поражениями. Такие себе вуду, приковылявшие кто из Шумера,
а кто из русских былин и так и забывшие уплыть на Гаити или куда-то еще после
1991 года, в котором все дружно поставили крест на прошлом и типа начали жизнь
с нового листа.

\ifcmt
  ig https://scontent-frt3-1.xx.fbcdn.net/v/t1.6435-9/32667386_1920132881351139_2102108817461346304_n.jpg?_nc_cat=104&ccb=1-5&_nc_sid=730e14&_nc_ohc=DLnbpmn5jhUAX9rpZOH&_nc_ht=scontent-frt3-1.xx&oh=32eaf21a0a6ee2d213bdf1713c874501&oe=61B9A43F
  @width 0.4
  %@wrap \parpic[r]
  @wrap \InsertBoxR{0}
\fi

Итак, мост. Отправить гаранта в пикантном расположении духа в этот день в село,
чтобы параллельно открыть даже лень уточнять у гугла какой именно сарай – это,
сука, круто. Разогнать легально работающее и аффилированное с РУ СМИ –
традиционное днище, но к этому то как раз мы и привыкли. А чтобы не отвыкали,
ну еще раз напомнили. Короче, вроде бы обосрались, показали себя во всей красе,
ну и лады. Но нет же. Давай включим «экспертное середовище» и расскажем, как
мост непременно наипнется. Тектонические плиты там не такие, дно чрезвычайно
илистое, сами немцы при Адольфе Алоизовиче не сумели… И ведь эти дурни все это
озвучивают бесплатно, «обманываться рады». А потом еще и обижаются, когда им
напоминают об их умственных и моральных качествах. Там – мост, а тут – хрю-хрю
на скотном дворе.

Но есть еще вторые персонажи в чудесной стране. Они то не считают себя
потомками шумер, за что последние их презрительно обзывают «ватой». Это все
Оболенские, Волконские, наследники Суворова и рыцари фразы «а мы же говорили».
Сегодня у них праздник – большая строительная перемога. «Наши» возвели мост. С
эстетической точки зрения можно, конечно, от этого масштабного сооружения
получить толику восхищения. Инженерная штуковина по сложности ничем не
уступающая Эресунскому мосту между Данией и Швецией или более знаменитому
тоннелю под Ла-Маншем. Вот только мост строили не совсем «наши». Наши ехали
вместе с Ницой из Киева в Запорожье и молча смотрели, как сумасшедшая баба
унижает целый поезд. Еще наши смотрели на Омеляна в Днепре, который в очередной
раз посмеялся с дурачков и рассказал о гиперлупе. Вот это все наши.

И вот теперь стоит мост ТАМ, а мы со всем этим живем ЗДЕСЬ. Гордимся и
ненавидим. Ну чем не люди третьего тысячелетия? До нашей эры, естессна.

\ii{15_05_2018.fb.lesev_igor.1.vudu.cmt}
