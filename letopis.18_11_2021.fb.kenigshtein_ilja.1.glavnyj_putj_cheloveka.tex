% vim: keymap=russian-jcukenwin
%%beginhead 
 
%%file 18_11_2021.fb.kenigshtein_ilja.1.glavnyj_putj_cheloveka
%%parent 18_11_2021
 
%%url https://www.facebook.com/kenigshtein/posts/4463958600323886
 
%%author_id kenigshtein_ilja
%%date 
 
%%tags chelovek,garmonia,gory,nepal
%%title Главный путь человека - достичь гармонии с самим собой
 
%%endhead 
 
\subsection{Главный путь человека - достичь гармонии с самим собой}
\label{sec:18_11_2021.fb.kenigshtein_ilja.1.glavnyj_putj_cheloveka}
 
\Purl{https://www.facebook.com/kenigshtein/posts/4463958600323886}
\ifcmt
 author_begin
   author_id kenigshtein_ilja
 author_end
\fi

Я не пишу о политике, а стараюсь концентрироваться на качестве жизни, бизнесе,
инновациях и креативных сферах. Я такой какой есть - со своими взлетами и
падениями, ошибками и верными решениями, удачами и неудачами. Я никого не учу,
не выдумываю истории из жизни и не строю себе ореол идеального человека,
успешного во всем. Таких сейчас вокруг хватает и без меня.

\ifcmt
  ig https://scontent-frx5-1.xx.fbcdn.net/v/t39.30808-6/257900054_4463817663671313_3685595016215224041_n.jpg?_nc_cat=110&ccb=1-5&_nc_sid=730e14&_nc_ohc=EOj6UGQhH6UAX9g_QZx&_nc_ht=scontent-frx5-1.xx&oh=ce6aca55d3dffb3984bb64df2892d163&oe=619C23AC
  @width 0.4
  %@wrap \parpic[r]
  @wrap \InsertBoxR{0}
\fi

Но сидит среднестатистический обыватель на диване и мотает ленту фейсбука (или
как он там сейчас называется). А в ней - министр пишет о своих заслугах,
депутат о своих победах, миллионер о своих миллионах и селебрити о своих
поклонниках. Все публикуют собственные истории успеха. И кажется что всё это -
рядом, на расстоянии вытянутой руки, одного комментария. Все на одной скамейке,
в одной песочнице. 

Искаженная картина подменяет настоящую. Проблема искажённого восприятия
событий, происходящих в мире - это серьезная современная болезнь всего
человечества, и не только наша. Современная сеть дала возможность создавать
виртуальные социальные кластеры, делящие людей по сферам интересов, а не по
уровню доходов - как это было раньше. Люди сбиваются в группы с близкими по
духу и потребностями. Постоянная коммуникация с близкими по духу рождает ложное
ощущение динамики окружающего мира, создавая иллюзию единства в многообразии. 

Но в реальности жизнь состоит из скучных событий, в которых ничего особенного
не происходит. Однако, при помощи соцсетей картинка сильно искажается. Люди
видят только то, что впечатляет, воспринимая мир с большим перевесом негатива и
сенсационного позитива. 

На самом деле это не так. Наверное, сейчас люди одиноки как никогда. В мире
нарастает социальное неравенство, причем не только между людьми, но и между
целыми странами. И это неравенство достигло такого уровня, что одни тратят
миллионы по щелчку пальца, а другие теряют способность зарабатывать на жизнь. В
определенном смысле, люди сейчас заново переживают перераспределение по
социальным нишам. 

И от этого у многих едет крыша. Потому что им кажется, что мир плоский. С одной
стороны - одна за другой проплывают истории успеха. С другой стороны, в
реальной жизни, всё складывается далеко не так удачно. И уж тем более ярко. В
сознании человека возникает и растет поток разочарования. За которым следует
коллективный шейминг, ненависть и агрессия. 

Заметили сколько агрессии вокруг в последнее время? 

Наш мир сильно изменился за последние годы. Один монах в Непале сказал, что
главный путь человека - достичь гармонии с самим собой. Пока её нет - он в
пути. Так было, есть и будет ещё сотни лет. Гармония - это не нечто мифическое
и неосязаемое. Просто у каждого она своя и путь у каждого тоже свой. 

Мне, например, тоже часто не хватает гармонии. Хотя я нахожу её в своей семье -
с женой и дочкой. Я обожаю когда у нас в команде Creative States всё
получается, но за всем этим стоит колоссальное количество ежедневной скучной
рутины. И ещё, конечно же, я нахожу гармонию в горах, на которые я поднимаюсь. 

Сижу порой в палатке базового лагеря, читаю новости, среди которых высосанные
из пальца байки и истории успеха made in Ukraine. Понимаю какой у нас в
обществе надлом, какая коллективная травма, когда люди не хотят быть самими
собой. И мне и грустно и легко одновременно. Потому что я знаю, что как только
подниму глаза - я увижу перед собой склоны белого исполина при свете солнца,
который тут стоял, стоит и будет стоять вечно. 

Моя история - это история пути.

\ii{18_11_2021.fb.kenigshtein_ilja.1.glavnyj_putj_cheloveka.cmt}
