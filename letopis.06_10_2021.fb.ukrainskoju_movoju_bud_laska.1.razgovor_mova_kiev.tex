% vim: keymap=russian-jcukenwin
%%beginhead 
 
%%file 06_10_2021.fb.ukrainskoju_movoju_bud_laska.1.razgovor_mova_kiev
%%parent 06_10_2021
 
%%url https://www.facebook.com/permalink.php?story_fbid=1879084038949741&id=100005444881046
 
%%author_id ukrainskoju_movoju_bud_laska
%%date 
 
%%tags identichnost',jazyk,kiev,mova,nacionalnost,obschestvo,ukraina,ukraincy
%%title Вчорашній вечір врізався в голову і не дає спокою…
 
%%endhead 
 
\subsection{Вчорашній вечір врізався в голову і не дає спокою…}
\label{sec:06_10_2021.fb.ukrainskoju_movoju_bud_laska.1.razgovor_mova_kiev}
 
\Purl{https://www.facebook.com/permalink.php?story_fbid=1879084038949741&id=100005444881046}
\ifcmt
 author_begin
   author_id ukrainskoju_movoju_bud_laska
 author_end
\fi

\begin{center}
	\begin{fminipage}{0.9\textwidth}
\Large\bfseries\color{orange}
Українська мова – це НЕ засіб спілкування. 

Українська мова – це утвердження і панування українців на українській землі. Це
утвердження України. 

Доки українці в Україні дискриміновані – доти українці НЕ утвердять, НЕ
збудують міцної України. 
	\end{fminipage}
\end{center}

Вчорашній вечір врізався в голову і не дає спокою…

Юний двоюрідний брат моєї дружини - гордість родини. Добре навчається, має
приємний і поміркований характер, здобуває призові місця на спортивних
змаганнях та є одним з небагатьох, хто в родині киян балакає українською. Мене,
україномовну білу ворону, це тішило завжди, бо хлопчина не піддається на
тотальне змосковщення столиці. Не піддавався…

Телефоную і замість знайомого говору чую: «Прівєт. Куди єхать. А гдє це? Возлє
чево?»

Для мене стало все зрозуміло… Я навіть не питав причини, бо ці причини «чому»
типові: російськомовні друзі, російськомовні вчителі, які тільки на уроці
застосовують державну, російськомовний гурток з англійської та спортивна секція
з «карате»… Невже йому стало соромно говорити рідною мовою своєї матері в
рідній країні?! Та чи рідна ця мова йому, коли всі інші говорять інакше…

«А какая разніца? Он імєєт права абщацца на том язикє, на катором єму удобна!» 

Як це втомлює…

\ifcmt
  pic https://scontent-mia3-2.xx.fbcdn.net/v/t1.6435-9/244083655_1879082302283248_9025893877346744165_n.jpg?_nc_cat=107&ccb=1-5&_nc_sid=8bfeb9&_nc_ohc=-6UKWzjrS9kAX8hfGWd&_nc_ht=scontent-mia3-2.xx&oh=36062b2e2fe611a42d3d5956cc1317f1&oe=61810D24
  @width 0.8
  %@wrap \parpic[r]
\fi

Українська мова – це НЕ засіб спілкування. 

Українська мова – це утвердження і панування українців на українській землі. Це
утвердження України. 

Доки українці в Україні дискриміновані – доти українці НЕ утвердять, НЕ
збудують міцної України. 

Слабка країна – ми бідні. Якщо молодь продовжує це ігнорувати, то будемо
квазідержавою ще довго.

Після такої ситуації розумієш, що нині українська — не для всіх. Вона тільки
для сильних. Тих, хто прагне кращого життя в Україні. Якісного, гідного,
щасливого, заможного, українського. Для мене це як бути обраним та вільним.
Бути кращим за безлику масу.

У нас надзвичайно багато розумних людей! Але ще більше людей без світогляду,
без переконань, цінностей та ХАРАКТЕРУ. На похиле дерево кози скачуть! Не
будьте похилим деревом! Характер потрібно формувати! Мова. Почніть утвердження
з неї. 

Це найлегше з утверджень. Для цього не потрібні високий інтелект, великі гроші
чи особливі вмілості.


\begin{center}
	\begin{fminipage}{0.9\textwidth}
\large\bfseries\color{blue}
\underline{\color{orange} Andriy Izvitskyi}\par
Як я це розумію... Теж маю купу знайомих, які все життя прожили в селі, а
кілька років в Києві - і вже кацап. Це слабкі душею люди, їм, по суті, всеодно,
якою мовою говорити і хто над ними цар. Так, саме цар чи пан, бо вільно вони
жити не вміють, треба дороговказ і нагайку. І таких ще дууже багато в Україні,
навіть, я б сказав, більшість. Тому еволюція буде довгою...

\underline{\color{orange} Kate Rina => Andriy Izvitskyi}\par 
якщо держава не візьме свою нагайку..  і не зажене це узькоязике стадо в
україномовний хлів. Без мови, без роботи. Без мови - без громадянства. Тільки
так!
	\end{fminipage}
\end{center}

\ii{06_10_2021.fb.ukrainskoju_movoju_bud_laska.1.razgovor_mova_kiev.cmt}
