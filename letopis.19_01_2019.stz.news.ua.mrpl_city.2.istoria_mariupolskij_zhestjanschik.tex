% vim: keymap=russian-jcukenwin
%%beginhead 
 
%%file 19_01_2019.stz.news.ua.mrpl_city.2.istoria_mariupolskij_zhestjanschik
%%parent 19_01_2019
 
%%url https://mrpl.city/blogs/view/istoriya-mariupolskij-zhestyanshhik
 
%%author_id burov_sergij.mariupol,news.ua.mrpl_city
%%date 
 
%%tags 
%%title История: мариупольский жестянщик
 
%%endhead 
 
\subsection{История: мариупольский жестянщик}
\label{sec:19_01_2019.stz.news.ua.mrpl_city.2.istoria_mariupolskij_zhestjanschik}
 
\Purl{https://mrpl.city/blogs/view/istoriya-mariupolskij-zhestyanshhik}
\ifcmt
 author_begin
   author_id burov_sergij.mariupol,news.ua.mrpl_city
 author_end
\fi

\ii{19_01_2019.stz.news.ua.mrpl_city.2.istoria_mariupolskij_zhestjanschik.pic.1}

В послевоенные годы летом в перенаселенных мариупольских дворах старой части
города, мощенных грубыми плитами известняка, между которыми там и сям стелился
тысячелистник, дворах, где заботливые хозяйки в крошечных цветниках под окнами
своих жилищ выращивали бархатистые чернобривцы, майоры с разноцветными, словно
присыпанными мельчайшей пылью, жесткими лепестками и нежными розовыми,
сиреневыми и белыми петуньями, примерно раз в месяц раздавалось:

\emph{- Лужу, паяю, ведра починяю!}

Этот призыв-реклама, звучащий сначала отдаленно, повторяясь, постепенно
приближался. Со временем появлялся и сам источник рекламы - мужичонка с сумкой
из-под противогаза на боку и со стальным квадратного сечения \enquote{аршином},
водруженным на плечо, который был продет сквозь рулон свернутых жестяных листов
разных размеров.

\textbf{Читайте также:} 

\href{https://mrpl.city/news/view/devushka-iz-donbassa-stala-samym-kreativnym-konduktorom-stolitsy-foto}{%
Девушка из Донбасса стала самым креативным кондуктором столицы, Яна Іванова, mrpl.city, 18.01.2019}

К этому моменту обладательницы прохудившейся утвари уже его ждали. Самая
предусмотрительная из них выносила из недр своей квартиры табуретку. Именно
табуретку, а не иную мебель для сидения, например, стул. Мастер, проверив
устойчивость табуретки, тщательно укладывал на нее \enquote{аршин} и садился на него
верхом. Противогазная сумка занимала свое место у правой ноги.

Получив от заказчицы таз или чайник, нуждающийся в ремонте, жестянщик обращал
этот предмет к солнцу, высматривая просвечивающиеся отверстия, чтобы
определиться с видом ремонта. К этому моменту все малолетнее население двора
собиралось вокруг священнодействующего мастера. Мальчишки с неослабевающим
вниманием следили за каждым его движением.

Наконец выносился приговор: дно на тазике или ведре подлежит замене. Тазик или
ведро водружались на свободный конец \enquote{аршина}, из сумки извлекались молоток и
зубило. С помощью этих инструментов негодное дно отделялось, край утвари
обстукивался молотком и обрезался ножницами. Наступал момент выбора. Выбора
листа для нового донышка. Рулон из листов разворачивался, и из него жестянщик
доставал два-три заранее заготовленных кружка жести, по его мнению, наиболее
подходящих. Не расправляя их, жестянщик прикладывал кружки один за другим к
предмету, подлежащему ремонту. В конце концов, подбиралась нужная заготовка.
Только теперь ее можно было ровнять деревянной киянкой.

\textbf{Читайте также:} 

\href{https://mrpl.city/news/view/shoumen-iz-mariupolya-sfotografirovalsya-s-yajtsom-rekordsmenom}{%
Шоумен из Мариуполя сфотографировался с \enquote{яйцом-рекордсменом}, Анастасія Селітріннікова, mrpl.city, 15.01.2019}

Дальнейшие движения мастера более походили на действия циркового фокусника.
Ведро так и летало вокруг \enquote{аршина}. Кажется, в одно мгновение отделялось лишнее
от заготовки, загибались края на ведре и новом донышке, образуя прочный замок.
Последняя ремонтная операция - замазывание замка специальной мазью, хранимой
мастером в жестянке из-под довоенного зубного порошка \enquote{Экстра}. Приняв гонорар
от владелицы восстановленной посуды, жестянщик приступал к исполнению заказа
следующей обитательницы двора...

С той поры прошло не менее семи десятилетий. Что-то давно не встречаются
путешествующие из двора во двор жестянщики, время от времени оглашающие округу
скороговоркой: \enquote{Лужу, паяю, ведра починяю}. Они сделали свое дело: сами выжили
и помогли, как могли, выживать сотням бедствующих семей, а таких в Мариуполе
послевоенных лет было ой как много. Потому-то и заслуживают жестянщики памяти.

\textbf{Читайте также:} 

\href{https://mrpl.city/news/view/iskusstvo-varit-kofe-ili-otkroveniya-mariupolskih-barista-foto-plusvideo}{%
Искусство варить кофе, или Откровения мариупольских бариста, Яна Іванова, mrpl.city, 17.01.2019}

P. S. Известный художник Лель Николаевич Кузьминков, царствие ему небесное,
прочитав строки о жестянщике, посетовал:

\emph{- Жаль, что текст перегружен описанием работы жестянщика.}

Конечно, Лель Николаевич был прав. Но, с другой стороны, как объяснить
современным читателям, привыкшим при малейшей поломке какой-нибудь посудины
сразу же выбрасывать ее на помойку, зачем ходил по мариупольским дворам
мужичонка, выкрикивая:

\emph{- Лужу, паяю, ведра починяю!}
