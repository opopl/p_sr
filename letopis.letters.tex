% vim: keymap=russian-jcukenwin
%%beginhead 
 
%%file letters
%%parent body
 
%%url 
 
%%author_id 
%%date 
 
%%tags 
%%title 
 
%%endhead 

\ii{letters.27_01_2023}
\ii{letters.06_02_2023}
\ii{letters.14_02_2023}
\ii{letters.18_02_2023}

%https://www.facebook.com/permalink.php?story_fbid=pfbid0sbMsAukawJAnZeA8fdWBvWhfxh4pusncRbwfHWN2sQYRxNnjbGn4FkQ5fDmWm4ajl&id=100087641497337

Надіє, дуже важлива книга! Мені, якщо чесно, як людині, що на щастя не бачила
наочно всього цього пекла, хоча я багато читав і читаю постів про Маріуполь, і
я відчуваю ваш біль, який так, відчуваєш, навіть якщо всього цього не бачив,
важко коментувати, оскільки подумки питаєш себе, що і як варто казати людям, що
все це пережили. Але... я мушу зізнатись в такому. Ще до війни я бачив вже
приблизно все те, що було в Маріуполі... у фільмі. Польському фільмі. Фільм
називається Місто 44 (Город 44). Страшний фільм насправді, і я в той момент
зовсім не думав, що всі ті жахіття, що були показані у тому фільмі, колись
будуть повторені вже в Україні. І цей фільм про трагедію Варшавського
повстання. Так, трагедія такого ж масштабу...  Хоча інша країна, інша історична
епоха, інші конкретні обставини. Так, у 1944 році, коли Варшава була під
німецькою окупацією, поляки підняли повстання проти нацистів, яке було в
результаті надзвичайно жорстоко придушене. Надзвичайно жорстоко. Місто було в
вогні, зруйновано, близько 100 000 поляків було в результаті вбито. Поляки
дотепер пам'ятають про це. Але... Варшава... якщо ходиш вулицями цього міста -
красивого, гарного міста... не відчуваєш, що тут колись коїлось... Так
заведено, що Міста мають внутрішню, так, внутрішню велику Силу... яка все
подолає і переможе! І так! В кінці фільму - якщо знайдете час колись подивитись
- дуже зворушливо. По-перше... (якщо подивититесь, побачите самі... - все
розказувати наперед напевне не треба... )  І по друге, що стосується власне
Міста Варшава. Кінець фільму... Палаюче знищене місто зникає... І постає
Варшава наших днів... І оцей перетворення із палаючого міста у сучасне мирне
усміхнене європейське місто колись мене надзвичайно зворушив, аж до сліз... І
так само Маріуполь. Місто залікує свої рани, так само як і маріупольці. І
колись, через десятиліття те, що відбувалось, залишиться лише у книжках та у
відео та фільмах, та пам'яті народу. Але приклад поляків дуже показовий. Вже
скільки років пройшло, а вони зняли цей фільм зовсім недавно, хоча стільки
років вже пройшло! Але вони пам'ятають, так, і дотепер, і будуть пам'ятати
завжди! 

от ви кажете, що зоряні війни, то є нудно... знаєте. А я почну здалека... з
Природи. Ось природа... виходите на вулицю... вдихаєте свіже повітря...
радієте... Сонечко світить... чи не задумувались ви про те, як все прекрасно,
взаємопов'язано між собою. Сонце світить - дощ ллє. Земля всмоктує в себе
вологу, і дає життєдайні сили, щоби прекрасна квіточка - скажімо, десь у Києві
на Співочому Полі - або у бабусі в селі -  проросла. Потім з'являються люди,
висаджують квіти, створюють парки, як от парк Кейкенхов в Голландії, або ж
дитячий парк Веселка в Маріуполі - справжня міні Голандія у прекрасному місті
біля Моря... Так от. Сонце світить, вітер віє... дощ ллє... Саме по собі Сонце
- то є величезна розжарена куля, на якій немає ніякого життя, там панує лише
Вогонь і Смерть... Так, Величезний Страшний Вогненний Океан під назвою Сонце,
розміром із тисячу таких планет як Земля... Але... світло від Сонця досягає
Землі, і дає в результаті життя прекрасній тендітній квіточці... Одне - велике,
неймовірно велике, дуже гаряче і смертоносне - дає в результаті життя чомусь
дуже малому, тендітному, крихкому... Так... В Природі, у Всесвіті все
взаємопов'язане... мале пов'язане з великими... мертве - з живим, гаряче - з
холодним... А як можна назвати те, что все це уособлює собою? Одні кажуть -
Матір-Природа, інші кажуть - Бог, Творець. А в цьому фільмі центральним
лейтмотивом проходить ідея Сили... Сили, що все об'єдную між собою... Дуже
філософський фільм насправді. І показовий діалог з передостанньої, восьмої серії
фільму (Останні Джедаї)... 

Таймкод - 49-10

Люк - тянись до неї своєю душею...  що ти бачиш?
Рей - острів... Життя... Смерть та тлін... початок нового Життя... тепло...
холод... мир... насилля... 
Люк - а що їх об'єднує?
Рей - баланс... і енергія... Сила...
Люк - а в тобі самій...
Рей - і в мені самій... та ж сама Сила... 
Люк - це і є твій урок... Сила не є власністю джедаїв...
ідея, що зі смертю джедая вмирає Світло, є хибною... ти відчуваєш це!? 

Ну от. У Всесвіті все пов'язано. Як колись писав англійський поет-романтик Вільям Блейк, 

Бачити світ у зерні піску, Небо — у квітці синій, В своїй долоні безкрайність усю І вічність в одній годині.

А Сила - то є назва тому, що все об'єднує
в одне ціле, і далекі галактики, і розжарене Сонце, і тендітну квіточку десь у бабусі на городі.
Ось у цьому власне і є основна ідея фільму, ідея Сили, так, ідея Сили.

%https://t.me/maripol_hope/1291

Дуже влучно сказано про останню крапку, що поставить Життя! Бо дійсно, так і
буде! Життя переможе Смерть. І щоби це сталось насправді... потрібно...
навчитись смерть перемагати у собі, в душі, в серці... перемагати відчай,
розпач, відчуття безнадії... Розумію, що для маріупольців це важко зараз... Я -
киянин... дивлюсь все це зі своєї сторони... Але... знаєте. Я от десь недавно
прочитав у одного автора - він каже про нескінченний постійний потік фото  в
телеграмі про зруйнований Маріуполь. Я скажу так... Це звичайно дійсно так.
Але... Інтернет - така штука... що тут ти насправді сам вибираєш, що ти можеш
дивитись... Я, коли заходжу в телеграм або фейсбук, звичайно, також бачу багато
фото або історій поганих про Маріуполь. Але... також! Нескінченний, дійсно
реально нескінченний потік радісних, гарних фото, історій про Маріуполь на
різних ресурсах та в абсолютно різних ракурсах! З яких на нас дивиться саме
Життя! Так, саме Життя, бурхливе невгамовне Життя, а не Смерть! Більше 200
років Життя, акумульованого у цих прекрасних та різноманітних фото чудового
міста біля моря, так - 200 років - бо фото, відео, історії, враження - це
акумуляція, сумарний ефект величезної Духовної Сили Міста, що стоїть біля Моря
вже більше двох століть! у порівнянні з цим роком смерті... Що ж переможе
врешті решт? 200 років Життя, чи рік (два-три) занепаду та смерті? Я думаю,
Життя! Так, Життя! Українське Життя в Українському Маріуполі! 💛💙💛💙

% 23.02.2023
% https://t.me/maripol_hope/1307

Доброго ранку! 

Щодо книги - то є велика подія. Обов'язково прочитайте, книжка сама по собі вас
не вб'є!  Так, можливо, буде боляче читати, але це лише книжка. Її завжди можна
відкласти в сторону, почитати пізніше, повернутись, перегорнути сторінки,
закрити, подумати, уявити, згадати...  потім знову повернутись до тих чи інших
рядків... Книжка - то є велика духовна річ, яка дійсно здатна лікувати душевні
рани! Але... у мене тут є такі думки. Щодо повернення у звільнений Маріуполь.
Скажу таке, це напевне не дуже вам сподобається, що я скажу. Зауважу наперед,
що я ніколи не був у Маріуполі, і тому... як це парадоксально не звучить, він
для мене такий самий живий - на фото, відео, як і до війни. Відсторонена
позиція, так сказати, тут допомагає це усвідомлювати.  Так, я співчуваю
маріупольцям душею і серцем, але я також знаю - що справжній Маріуполь живий,
хоча він і залишився таким лише у пам'яті та у фото та відео. А щодо звільнення
Маріуполя. Ви мрієте про повернення, уявляєте як все буде, як ЗСУ звільнять
місто... от можна уявити собі, ...  найоптимістичний варіант... російські
війска розбиті вщент, путін вмер або його вбито, військова перемога України
здобута, росія підписує принизливий для себе мир, навіть погоджується на
репарації, і виводить свої війська взагалі... Ви купляєте квиток... радісно
їдете на поїзді Київ-Маріуполь, наприклад... і знаєте. тут на вас чекає
смертельна небезпека. Абсолютно і без жартів.  А чому... Тому що ви приїдете в
напів-мертве місто, в місто-примару. ЗСУ може звільнити місто фізично, але
звільнити місто в душі і серці - то є справа персональна кожного, справа
духовна, інтимна, тут ніякий хаймарс не допоможе. Так от... Повернетесь... В
місто, яке стало Кладовищем, і в місто, яке є лише Тінню Маріуполя. В місто,
яке зараз насичене невидимою страшною духовною отрутою... І це є страшна
небезпека для душі, для серця... Для тих, хто вживу знав його зовсім іншим...
Оскільки це може просто вбити на місці... як тільки ви приїдете туди,
пройдетесь вулицями, дійдете до тих місць, які колись робили вас щасливими... і
усвідомите на повну і всерйоз, що повернення до того минулого, яким ви його
пам'ятаєте, вже ніколи не буде... В таких випадках людина може просто вмерти на
місці від атомного вибуху у серці та душі, або зійти з розуму і назавжди...
Так, ваше найрідніше місто може вас вбити, якщо ви душевно, духовно не станете
як слід сильними, готовими до цього повернення !  Так що...  Звільнення,
повернення Маріуполя у душі, у серці повинне передувати справжньому поверненню
у реальності.  І чим більше спогадів, чим більше всього, що вас поєднує із
Маріуполем, тим страшніша ця небезпека, і тим більша ймовірність, що без
відповідної розумової і душевної підготовки, духовної роботи, мертве отруєне
місто вас просто вб'є на місці, як удар блискавки в дім без громовідводу.

%Дякую.. 

%Я теж про це думаю! Додому хочется, але.. Я вирішила після деокупації поїхати
%на тиждень, подивитися, придивитися ЯК мені, дитині .. Ми зможемо там жити!? А
%який контингент навколо буде? Не відразу повертатися. Можливо там буде жахлива
%енергетика💔😣 Але як то кажуть ми не перші .. багато міст відновили після війн
%і зараз це квітучі міста!


Доброго дня! Щодо контингенту, відповідь на коментар вище... як там буде... не
в цьому річ, абсолютно не в цьому. А в Ваших персональних відносинах, так,
інтимних душевних відносинах із Вашим Містом... Розумієте. Неважливо насправді,
який там буде контингент на момент вашого приїзду у звільнене місто. Головним
тут є ваша зустріч із Містом один-на-один.  І так... щодо контингенту.
Припустимо, що... ось...  Маріуполь звільнено, більше того, всі зрадники та
колаборанти або заарештовані, або просто повтікали...  Зібрався цілий потяг
друзів, маріупольців, всі щасливі, радісні, нарешті ми їдемо додому!!! Поїзд
прибуває на розгромлений вокзал...  перші гнітючі думки... руйнування усюди,
незважаючи на те, що місто вже наше, і гордо майорить вже наш найгарніший у
світі прапор над будівлею мерії... потім, ви всі розходитесь... або гуляєте
разом...  спогади, минулі жахи оживають... ось тут було оце, ось тут було це.
Тут у дворі поховані мої друзі, там заживо згоріла бабуся... Тут був будинок із
муралом...  нема вже ні будинку, ні муралу, ні тих людей, що там жили....
Розумієте...  І Місто починає вас душити, всю вашу прекрасну компанію.
Безжалісно душити, безжалісно бомбардувати вашу душу, ваше серце. Починає
душити всіх вас поодинці, жорстко, безжалісно розбиває вашу єдність... Береться
за кожного, і починає роздирати ваші душі, ваші серця на частини... Мертве,
зруйноване - хоча і звільнене місто, але ще наскрізь отруєне місто... береться
за кожного маріупольця, і починає творити над ним (нею) свій Страшний Суд...
Спогадів, з'єднаних із жорсткою реальністю, все більше і більше, як смертоносна
сніжна лавина в Карпатах зимою, вони сходять на вас, зносячи все на своєму
шляху... вони починають душити вас звідусіль...  Так!  Пекло, через яке ви
колись пройшли, і з якого ви вибрались живими, і яке зараз залишилось лише у
книжці Надії та ваших спогадах та фото і відео в інтернеті, повертається
вживу... В результаті три можливі варіанти - або ви берете квиток і наспіх
тікаєте з міста щодуху, щосили, щоби туди більше ніколи не повернутись... або
ж...  Місто просто вбиває вас насмерть...  через інфаркт, інсульт, просто
вбиває на місці... або ж.. Ваша Внутрішня Сила, так, Ваша Духовня Броня - дає
відсіч демонам темряви і відчаю... (бажаний варіант) Ви виходите переможцем
врешті решт... І Місто відпускає свої сталеві єжові рукавиці від вас, і каже
вам - залишайся тут, друже. Залишайся тут, я, Українське Місто Маріуполь,
найгарніше місто біля моря у світі, буду з тобою скільки ти забажаєш. Залишайся
тут, вкладуй свою Душу, своє Серце в мене, в Місто Маріуполь, і я тобі віддячу
сповна! Будуть тобі і фонтани, і новий парк Веселка, і новий ДрамТеатр, і
радісні веселі новорічні трамваї, і просто трамваї, і тролейбуси, і може навіть
метро! все буде, не сумнівайся! Ось так я це все бачу...

""Я згодна Тому і кажу треба поїхати, а так ми просто фантазуємо" ... -  а що
ви будете робити з тим, що цим мертвим містом гуляють тисячі і десятки тисяч
невинних загублених душ, душ людей, померлих жахливою смертю, що день і ніч
шукають умиротворення і спокою, але не знаходять його? Що ви будете робити,
коли ці невинно вбиті - а їх тисячі, десятки тисяч прийдуть до вас в гості,
заполонять всю вашу душу, почнуть розривати її на частини... абсолютно не
питаючи вашого дозволу? Розумієте, тут немає жартів. Все це дуже-дуже серйозно.
Місто не впустить до себе назад тих, хто не готовий до цього. Я взагалі вважаю,
зі своєї сторони, що на даний момент (можливо це звучить дуже зухвало для вас,
але я просто кажу зараз, що думаю) лише декілька людей із усіх маріупольців
переживуть своє повернення до Маріуполя, витримають свою першу зустріч із
мертвим отруєним Містом. Наприклад, це пані Надія і пані Наталія. А чому. Пані
Надія написала книжку, вклала величезну кількість сил, душі і серця. Пані
Наталя так само - дуже сильна жінка, стільки пережила, вона витримає, вона
стільки зусиль вкладає в проєкт голоси мирних, стільки зусиль докладає, щоби
зберегти всі ці історії. Так! Книжка та пости, проєкт, душа і серце - вкладені
в усе це - це і є та духовна броня, яка дасть цим конкретним людям достатній
захист проти смертоносних обіймів мертвого Міста... А щодо інших
маріупольців... я не знаю. Я бачу по коментарям, що навпаки, багато людей
тримають свій біль всередині... Мої судження тут лише по тому, що я бачу як
стороння, але небайдужа людина... Справжнє повернення у Маріуполь - це ваш
духовний Еверест. Якщо ви не подолаєте спочатку цю гору всередині себе, тобто
не подолаєте, хоча б частково свій біль, свої страхи, свої спогади, не
збалансуєте більш-менш свою душу... поки ви на безпечній відстані від Міста, то
справжня зустріч із зруйнованим мертвим Містом-Кладовищем буде для вас просто
фатальною.

Доброе утро! Ох! Надеюсь я еще не надоел всем тут со своими мыслями! Знаете...
сейчас как раз годовщина... не спится совсем... итак... вы хотите приехать
туда, наверное припасть к тому, что осталось еще от дома... посидеть,
поплакать... правильно я понял? чтобы вот так сидишь день у дома и плачешь... а
потом как бы слезы наверное должны пройти, вы думаете, и станет легче?
знаете... если бы Ваш Город был цел... если бы живая ткань Города как Живого
Организма не была бы так жестоко разрушена... если бы дело было только лишь в
одном Вашем доме... ну вот скажем... как в Днепре... ракета попала в Ваш дом.
Предположим, что вы жили когда-то в том доме, а потом - уехали. Бывает так по
жизни... А вы далеко... Дом разрушен верно, погибли Ваши друзья, знакомые,
частичка Вашего детства. Вы с ужасом об этом узнали. Все - дома Вашего детства
нет. Но! Все остальные дома целы, Город цел. Набережная цела, офисные здания,
магазины, университеты, школы, все работает. И вы приезжаете издалеко... дом
уже сносят... А вы стоите недалеко, в Днепре... и прощаетесь... Туда-сюда снуют
люди... Ездят машины. Город живой. Конкретно ваш дом умер - но Город живой. В
этом случае это бы получилось. Вы бы приехали - поплакали. И попрощались бы.
Понимаете. Потому что живая энергия Города, который живет по прежнему, помогла
бы Вам пережить все это. А здесь... все не так. Дело то не только в том, где
жили Вы. Дело также в соседнем доме. Дело также в доме Натальи Дедовой или
Надежды Сухоруковой, дело также в убитых тысячах, дело в сотнях варварски
уничтоженных домов дело в душах, которых никто не похоронил, не помянул как
следует.  Дело в том, что тот Город, который должен быть живой, и который вы
помните, по факту мертв. Понимаете, не просто конкретно ваш дом уничтожили, а
уничтожен ваш живой Город. И это колоссальная негативная психическая и душевная
энергия, которая будет убивать наповал тех, кто не позаботится зараннее о том,
как противостоять этому. Как вирус, знаете, убивает, если не вколоть себе
вакцину... Да... Прощаться, выплакивать нужно не только ваш дом, а вообще весь
тот Город, в котором вы жили! Вы говорите... приедете, поплачете... А Вы не
думали, что, приехав... вы сядете или упадете как камень... чтобы больше
никогда не встать? ... Понимаете... был живой Город. А сейчас его место заняла
его Тень... И вот эта Тень... да, Тень Города... она ждет вас, да, там на месте
ждет вас... чтобы, когда вы приедете... забрать к себе навсегда, чтобы сожрать
вас целиком... Именно там... Вот знаете... бойтесь вот этого - именно там на
том клочке, потому что этот клочок может стаь для вас могилой по настоящему...
Ну ладно. Насчет выплакать. Выплакивать нужно начинать прямо сейчас! А лучший
способ наверное... Прочтете книгу Надежды - а потом... ну возьмите и напишите
тоже книжку... и про ваш дом... и про ваш живой довоенный Город, про то, чем вы
занимались, чем жили, про ваших друзей. И так далее!  Вам же есть что
вспомнить, не так ли? Так мало сейчас книжек про Мариуполь!  Начинайте
вспоминать, что с вами было в том, живом Городе!  Вспоминайте, вспоминайте,
записывайте, записывайте, как следует распишитесь. Выплачьте свой Город через
Слово!

%09:41:13 25-02-23
Доброго ранку! Так, звісно добре поспати... Але ось ранок, час вставати!  Що я
хотів сказати щодо війни та взагалі. От Ви недавно написали, що Ваша мама не
вірить у ракетну атаку і читає книжку Надії. Але спочатку... відкрив я ленту,
тут одна людина пише - можливо Ви цей допис теж вже бачили -

" ... Захоплююся мамою. В річницю війни вона, знаходячись за сотні кілометрів від
дому, в незнайомій країні, незнайомій квартирі, з невідомістю що далі і як
далі, вже планує, як вона повернеться на роботу в дитячий садок Маріуполя щоб
допомагати відновити, щоб знову висадити квіти, щоб знову оживити рідні стіни,
рідні вулиці. Людей не оживити тільки.  А ще вона боїться того, що «оживити» не
вийде, тих, хто сліпий, глухий зараз, хто не зрозумів досі. Але ця її Надія в
мене викликає захоплення. І це в той час, як в будинку її мами, теж, до речі,
Надії, російські солдати сплять, їдять, може дивляться наші родинні фото,
топчуть ногами килими, візерунки яких я в дитинстві обводила пальчиком щоб
швидше заснути ... "

І також, щодо Вашої мами - Ви написали - тільки Ваша мама не вірить у ракетну
атаку. Я так скажу.  Розумієте. Ви вірите у ракетну атаку, і не можете заснути,
все чекаєте. І нічого не стається. Дивно, правда!  А я не вірю у ракетну атаку,
і спокійно вчора гуляв Містом Києвом. До речі, навіть заїхав на Петрівку (),
купив собі книжок!  Розумієте... в ракетну атаку вірити не треба, вірити можна
в Бога. Вірити можна в дітей, в себе, в свою Країну, в Перемогу України!
Насправді, кожного дня може бути атака, і зараз, коли я пишу, і завтра, і через
тиждень.  Не треба тут чіплятись до дат. А щодо Києва та дому. Ви втратили свій
дім у Маріуполі, були змушені переїхати, Київ став зараз Вашим новим домом.
Звичайно, Вас тягне додому в Маріуполь, Ви день і ніч читаєте новини про
Маріуполь... але... Скажіть...  Наскільки Ви добре знаєте Київ, так,
тисячолітній Вічний Город Київ? Чи були Ви наприклад у Софійському Соборі або ж
чи знаєте Ви про Бабин Яр, місце страшної трагедії масштабу Маріуполя? Чи
спускались Ви коли небудь із свічкою у печери Лаври, або ж чи роздивлялись коли
небудь фрески на стінах Володимирського Собору? Поки Ви тут, чи знайшли Ви час
прошло піти і просто посидіти подивитись на Дніпро, подумати, ось так... Просто
йдеш дивишся на Дніпро і про все на світі забуваєш...  Чи маєте Ви вдячність
Києву, так, Києву як Місту, що дав Вам прихисток? Розумієте... У Києві зараз
мир. Так, взагалі то йде війна.  Але в Києві зараз мир. І Київ...  все, що
зараз відбувається. Вже бачив за свою історії багато разів.  Його багато раз
спалювали... тут коїлись страшні людські трагедії... падіння 1240 року,
громадянська війна, окупація, Бабин Яр, Чорнобиль, розстріл на Грушевського...
Місто все це пам'ятає. Але, знаєте. Київ таке місто, що воно відкриває свої
двері всім, хто приїздить до нього. Це дуже щедре, неймовірно щедре Місто. І
воно дає можливість жити у ньому людям, яким в принципі абсолютно нецікавий ні
сам Київ, ні його історія. Але... Матір Городам Руським немовби каже... друже,
жити тут скільки хочеш, роби що хочеш... але все ж таки, я маю надію...  що
коли небудь ти відволікнешся від своїх справ, і поки ти живеш тут, ти почнеш
відкривати для себе Місто... повір, друже, воно того варте! Якось так...

%08:14:38 25-02-23
а еще они говорят друг друг, когда прощаются - дуй! дуй дуй! (doei ) когда я в
свое время такое слышал, у меня всегда была ассоциация с "дуй отсюдова" )) до
свидания же звучит tot ziens (тот зинс) кстати, интересно, что дуй - это
производное от глагола doen -то есть, делать, работать. То есть, буквально, как
бы, прощаясь, голландцы друг друг желают работать дальше. Неудивительно,
кстати, потому что голландцы - очень трудолюбивая нация, закаленная постоянным
противоборством с суровой природой, а также соседством с гораздо более мощными
и сильными странами-соседями рядом (Германия, Франция). Удивительная страна
Голландия! Такая маленькая, но такая сильная, мощная, ухоженная страна! У
них... знаете, есть пословица... Бог создал Землю... а голландцы Голландию.
Кстати, Вы никогда не задумывались, что любой язык несет как бы структуру
мировозрения людей, народа, который на нем говорит. Вот например, Франция. Язык
красивый, изысканный, там даже прямое недвусмысленное приглашение девушке
переспать звучит красиво и так по французски! А вот Голландия. Простой язык.
Звучит грубо, не очень красиво на ухо. Практичный простой язык без излишеств.
Точно так же как и голландцы. Простые прямые открытые и очень практичные люди.
Английский - туманный Альбион. Вежливость, сдержанность, учтивость, уважение к
индивидуальности и личности... загадочность, намеки... коварство, лживость,
лицемерие... Слова пишутся и произносятся с тысячами исключений, слова как бы
сохраняют по возможности свою индивидуальность... Грамматика то простая, а вот
слова сами по себе - их произношение и написание - ну уж никак не хотят вести
себя по одному правилу! Прям как люди! И хорошие, и плохие стороны английского
национального характера, все это отображается в языке... Ну и немецкий. Строгий
четкий язык, более строгий с точки зрения грамматики, чем украинский или
русский. А ведь немцы так известны своей упорядоченностью, педантичностью...
Как то я в свое время изучал немного немецкий. Потом понял... трудное это дело,
хотя и интересно. Все эти приставки, которые улетают в конец... Как бы знаете,
воспитывается вежливость к собеседнику, терпение... потому что пока человек не
закончит свою речь, не станет окончательно понятно, что же он именно имеет в
виду (улетевшая в конец приставка  ab/auf/an может поменять весь смысл... ) Вот
так. Ну а возвращаясь к Голландии, ниже фото из цветочного парка Кейкенкоф )


%14:45:09 25-02-23
%Natalya Dedova
%Kiev Fortress доброго ранку! Про маму моєї колеги - читала.

%По Києву - гуляла. Тільки що зі свічкою - ні.

%Щодо ударів. Знаєте, що таке - історична пам'ять? А знаєте, що таке три тижні
%щоденного безперервного вбивства та виживання? Я та мій син досі присідаємо,
%коли сусід зверху щось кидає. І для мене сирени - це не просто щось там за
%вікном. І попередження про ракетний удар - не я його вигадала. І я б не вірила
%в ракетну атаку, якби з 2014 не жила в стані війни. Ось в чому справа.

%А щодо Києва. Так, прихистив. І це прекрасно. І за це - дякуємо. Але. Я живу в
%чужій квартирі, сплю на чужому дивані. Мій син спить взагалі на кухні. Адже не
%хоче спати зі мною. І сплачую я за квартиру майже всю свою зарплату. І працюю в
%нашій родині - тільки я. 24/7.

%А щодо Бога. Після масового знищення людей, зокрема маріупольців, я не розумію,
%де в той час був Бог. І чому карлик досі живий. А мої близькі - ні. Якось я
%порахувала. Дуже приблизно. Мінімум 40 людей із мого оточення - загинули.

%А про Бабин Яр - знаю. Влітку 2021 ми останнього разу приїздили всією родиною
%до Києва. І ходили саме туди. І я читала кожну табличку. І вдивлялася в кожне
%обличчя. І думала, не дай Бог..... А воно - ось як вийшло.


Доброго дня! Дякую! Знаєте... я ризикую, що я пишу надто багато слів, ризикую
Вас втомити та прогнівати... Але я все ж таки допишу. Бог... він одному дає
одне, іншому - інше. Не нам Його судити, не нам міряти Його діяння. Він нас
створив по Своїй Подобі та Образу, і Він може кожного з нас знищити будь-якої
миті. Кожного з нас, і божевільного карлика, і Вас, і мене. Де Він був під час
того, що відбувалось у Маріуполі... Я відповім так, щодо своїм уявленням про
Нього. Він був у тому самому місці, коли стався Бабин Яр, коли було знищено
більше 100 000 киян, євреїв, комуністів та безпартійних. І де стався розстріл
на Майдані, Він також був недалеко. І Він був там, де було придушене Варшавське
повстання 1944 року, яке по трагічності має такий розмір, як і Маріуполь, і по
якому вже в наші дні був знятий фільм Місто 44. Він був у тому самому місці,
коли розстрілювали Розстріляне Відродження, і Він також був у тому самому
місці, де... Ви і та Ваш син врятувались, а Ваш чоловік та ваші друзі та просто
сусіди - на жаль, загинув. І є поруч з нами, усюди, і вдень і вночі, і буде
завжди, во Віки Віків... І знаєте. Сьогодні день народження Лесі Українки. Вона
з десяти років хворіла на туберкульоз, все життя страждала від цього.  Ось
такий у неї є вірш.. Ви його напевне знаєте. Без надії таки сподіваюсь. Contra
spem spero

Гетьте, думи, ви хмари осінні!
То ж тепера весна золота!
Чи то так у жалю, в голосінні
Проминуть молодії літа?

Ні, я хочу крізь сльози сміятись,
Серед лиха співати пісні,
Без надії таки сподіватись,
Жити хочу! Геть, думи сумні!

Я на вбогім сумнім перелозі
Буду сіять барвисті квітки,
Буду сіять квітки на морозі,
Буду лить на них сльози гіркі.

І від сліз тих гарячих розтане
Та кора льодовая, міцна,
Може, квіти зійдуть - і настане
Ще й для мене весела весна. 

Я на гору круту крем'яную
Буду камінь важкий підіймать
І, несучи вагу ту страшную,
Буду пісню веселу співать.

В довгу, темную нічку невидну
Не стулю ні на хвильку очей -
Все шукатиму зірку провідну,
Ясну владарку темних ночей. **
Так! я буду крізь сльози сміятись,
Серед лиха співати пісні,
Без надії таки сподіватись,
Буду жити! Геть, думи сумні!

% 18:09:21 25-02-23
% #Мариуполь #Надежда Chat
Добрый день! Важные темы затронули... Чудесные фото прекрасного удивительного
Города! А насчет темы... психологии того, во что превратились россияне.
Знаете... вот иногда хочется найти какое то ключевое слово... которое бы как то
емко передавало суть... Матрица. Когда был снят такой фильм, Матрица. Про то,
что наш мир как бы на самом деле Матрица... Здесь, конечно, всего лишь
аналогия, воображаемое понятие, но суть довольно близка... Орки создали себе
свою Матрицу, да, мощное информационное поле - Смысловое Когнитивное Поле,
которое их целиком поглотило, и которое управляет их мыслями, поведением.
Смыслы, элементы этой Матрицы - это Слова, Предложения, Конструкции, ну.. типа
хохлы-продались-америке, россия-великая-страна, путин-президент-мира,
нацисты-бандеровцы ляляля, и понимаете. Эта Матрица, Виртуальный Мир - это...
Раковая Опухоль Интернета... Они... создали эту Матрицу, а теперь Матрица
сожрала их мозги целиком, и живет своей жизнью, питаясь их мозгами. Знаете,
такой невидимый ментальный вирус, ментальный информационный вирус массового
поражения... У меня в этом смысле был в прошлом году довольно интенсивный опыт
изучения поведения российских зомби в соцсети одноклассники. Специфика той сети
в том, что там ты вплотную сталкиваешься лоб в лоб с этими зомби, там можно
писать что хочешь и как хочешь, и в этой соцсети - так она устроена виртуальное
общение наиболее приближено по динамике к реальному человеческому общению.
Понимаете... Я уже так данных насобирал наверное на диссертацию... И в конце
концов видишь, насколько же безумно смертоносным может быть Интернет...
насколько он многократно усиливает влияние пропаганды, ложных бредовых идей...
Да! Скажу я... состояние российского общества в плане пропаганды - это промывка
мозгов похлеще, посильнее будет, чем во времена Геббельса! А почему это так -
отдельная большая тема... Вот, чтобы начинать все это понимать в таком
основательном ключе, есть книжки... ну например такие...

%20:32:26 25-02-23
Спасибо! Ох! А касательно зомби, конкретно насколько они безумны. Они реально
безумны, и это можно показать на примере... Вот реальный пример из моих данных
из ок. Щас туда зашел, порылся в своих постах там, достал. Тут сначала
небольшое пояснение. Я под эту сеть специально понаписывал текстов - штук 60
наверное, которые бы били по этой самой Матрице в целом. Как можно сильнее били
по зомби. То есть, насыщенный сильный текст, зараннее подготовленный, который
бьет по ключевым точкам российской пропаганды в мозгу этих безумцев, бьет
конкретно, персонально. И ты приходишь в эту сеть, и начинаешь в чатиках
методично их бить, бить и еще раз бить. Вплотную берешься и начинаешь их
трусить Словом, просто выносишь им мозги, целенаправленно и систематично. И
тут... оказывается, что вместо живых людей с Волей и Разумом... это зомби,
просто какие то жалкие мешки с дерьмом в голове вместо мозгов и разума.  И эти
зомби... когда их так атакуешь... они пытаются защищаться... и одним из методов
самозащиты у них было переделка моих текстов под свои смыслы. Ну то есть - если
в тексте россия виновата - они заменяли на Украину, российские нацисты - меняли
на "усраинский (хохляцкий) нацист" - меняли что виноват, кто прав.  И
отправляли назад переделанный текст. Но... на выходе у них реально получался
уже бред! Понимаете, бред, даже с точки зрения российской пропаганады! Безумие
и шизофрения, так сказать, выделенные в концентрированной форме! Ну... а ниже
фото - мой изначальный текст и безумная переделка зомби.

% https://ok.ru/profile/586956477134/statuses/155969245918670
Леся Українка 💛 💙 💛 💙 💛 💙 💛 💙        

Слава Україні, друзі!!! Вибачте, я в останній час заглибився в інші справи, і тут перестав так активно писати. Вибачте, що не всім відповідаю вчасно... Вчора було річниця дуже сумних подій, які докорінно змінили все наше життя... І війна все йде і йде... Ну а сьогодні - так, сьогодні... хочу нагадати, що у цей день, 25 лютого 1871, народилася Леся Українка - наша геніальна поетеса.

Ось трохи нагадаю нижче...

Леся Українка (справжнє ім’я — Лариса Косач) народилася 25 лютого 1871 р. у місті Новоград-Волинському. Вона була дочкою поетки та письменниці Олени Пчілки та племінницею філософа, політика та письменника Михайла Драгоманова, який значною мірою вплинув на формування її світогляду. До 13 років Леся мешкала на Волині — у Луцьку, Ковелі та у батьківському маєтку в с. Колодяжному, пізніше — у Гадячі, а від 1884 р. — у Києві. Тут вона долучилася до літературної групи «Плеяди» та до «Літературно-артистичного товариства».

Від 10 років Леся хворіла на туберкульоз кісток, тому навчалася вдома, проте часто виїздила на консультації та лікування до Криму, Болгарії, Італії, Австрії, Німеччини, Литви, Швейцарії, Єгипту, Греції, Польщі, Грузії. Саме у цих мандрівках вона написала багато своїх віршів і навіть цілі поетичні цикли, як-от: «Подорож до моря» (1888), «Кримські спогади» (1893), «Кримські відгуки» (1897), «Весна в Єгипті» (1910), «З подорожньої книжки» (1911).

Леся навчилася читати у 4 роки, у 5 років почала компонувати невеликі музичні п’єси, у 9 років написала свій перший вірш «Надія» і у 13 років уперше опублікувала свій вірш «Конвалія» (у львівському журналі «Зоря»). У 14 років вона видала свою першу книгу — збірку перекладів українською мовою оповідань Миколи Гоголя, у 19 років для своїх молодших сестер написала підручник «Стародавня історія східних народів», а свою першу збірку віршів «На крилах пісень» видала у 22-річному віці.

... так! ну а нижче одна з моїх улюблених поезій Лесі Українки - насправді, відома річ дуже! - але тим не менш! 

Contra Spem Spero!

Гетьте, думи, ви хмари осінні!
То ж тепера весна золота!
Чи то так у жалю, в голосінні
Проминуть молодії літа?

Ні, я хочу крізь сльози сміятись,
Серед лиха співати пісні,
Без надії таки сподіватись,
Жити хочу! Геть, думи сумні!

Я на вбогім сумнім перелозі
Буду сіять барвисті квітки,
Буду сіять квітки на морозі,
Буду лить на них сльози гіркі.

 І від сліз тих гарячих розтане
Та кора льодовая, міцна,
Може, квіти зійдуть — і настане
Ще й для мене весела весна.

Я на гору круту крем'яную
Буду камінь важкий підіймать
І, несучи вагу ту страшную,
Буду пісню веселу співать.

В довгу, темную нічку невидну
Не стулю ні на хвильку очей —
Все шукатиму зірку провідну,
Ясну владарку темних ночей.

Так! я буду крізь сльози сміятись,
Серед лиха співати пісні,
Без надії таки сподіватись,
Буду жити! Геть, думи сумні!

Так що... якщо у вас погано на душі... не сумуйте! Читайте твори, поезії Лесі, і це вам безумовно підніме настрій і окрилить душу в такі тяжкі і буремні для усіх нас часи! Все буде добре! Все буде Україна! Слава Україні! 💛 💙 💛 💙 💛 💙 💛 💙        
