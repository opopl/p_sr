% vim: keymap=russian-jcukenwin
%%beginhead 
 
%%file 03_11_2021.fb.nikonov_sergej.1.solncepek_film
%%parent 03_11_2021
 
%%url https://www.facebook.com/alexelsevier/posts/1618039501874692
 
%%author_id nikonov_sergej
%%date 
 
%%tags donbass,film.solncepek.donbass,kino,rossia,ukraina,vojna
%%title Посмотрел российский фильм Солнцепек
 
%%endhead 
 
\subsection{Посмотрел российский фильм Солнцепек}
\label{sec:03_11_2021.fb.nikonov_sergej.1.solncepek_film}
 
\Purl{https://www.facebook.com/alexelsevier/posts/1618039501874692}
\ifcmt
 author_begin
   author_id nikonov_sergej
 author_end
\fi

Посмотрел российский фильм Солнцепек. О событиях в Луганской области в 2014 г.
Если желаете оставить комментарий, то лучше посмотреть фильм. Ибо часто бывает,
что пишут "За Россию" или "Україна понад усе". 

\href{https://rezka.ag/films/drama/41224-solncepek-2021.html}{%
Солнцепек, rezka.ag%
}

И начинается опасная склока.
Назад к теме Судя по фразе "одна страна" ЛНР ещё не было. Два моих слова: "Ука
война". Иногда общеславянские вульгаризмы очень точно характеризуют понятия.
Далее идет мой взгляд на вещи. В фильме показана российская точка зрения. Но
подлость и мерзость войны раскрыта полностью во всем ужасе. Показано, каким
грузом ложится она на судьбы и души людские, как она проявляет сущность
человеческую. Это не фильм для развлечения , это певцам про батька одного,
покажите. Спивают они... Угу...В душе может и патриоты, а по факту... Это
тяжелый фильм, беспощадный, без шаблонов. Вся мерзость войны вывернута наружу.
И характеры человеческие показаны в их взаимодействии и противопоставлении.
Тут многое увидите. И диалоги офицера повстанцев, прошедшего Афган, с русским
парнем из правого сектора и нормальним чесним українським юнаком, що пішов
битися за Батьківщину, та побачив смерть на власні очі і повернувся додому. До
родини. И семью врача, которая погибает, кроме удочеренной девочки. Ангелочка.
В первом случае это две России и две позиции: офицерская традиционная и
молодежная либеральная. Эта молодая вырусь для меня один из самых негативных
персонажей. Хуже только некоторые правосеки, их командиры и уголовники. Он не
нужен никому и остается один. В третьем случае бывший афганец. Он мог бы взять
в руки ружье и убивать. И мотив уже был и видел, как убивали в селе. Но он стал
врачом. ВРАЧОМ И ОТЦОМ. Тем, кто каждый день под обстрелом выезжает на вызов,
чтобы помочь и делать все для сбережения семьи. НО ВОЙНА- УКА. И его семья
погибает от обстрела. Одно дитя малолетнее и остается. Есть там и наш
украинский командир. Тоже бывший афганец, отпустивший замученного добробатами
парня. Настолько обезумевшего от ран и издевательств, что считает себя причиной
войны. И именно этот куратор обучает наших замечательных парней, которые идут в
бой необстрелянными и погибают с честью. Но от  кого? От русских или бывших
советских офицеров. У которых по фильму своя честь. Командир застрелился. При
том, что в миру у него не сложилось. Война - его стихия. Напоследок. Самые
тяжелые сцены для меня - это массовая гибель мирных жителей под Луганской ОГА,
другом обстреле, гибель наших украинских молодых ребят в боях добробатов и
танкового подразделения и  в самолете, а также от рук бандитов. Усіх хлопців,
що тільки починали  жити. Ука война. Тем более, что опять разжигают мрази.
Нельзя добавлять С по мирным правилам. И МИРА НАШЕМУ ДОМУ. НАШЕЙ УКРАИНЕ. Я
ГРОМАДЯНИН УКРАЇНИ ТОМУ МЕНІ ПОТРІБНА ЄДИНА ДЕРЖАВА І ТОМУ Я СКОРБЛЮ ЗА УСІХ,
ХТО ЗАГИНУВ В ЦІЙ БІЙНІ.  
