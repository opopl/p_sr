% vim: keymap=russian-jcukenwin
%%beginhead 
 
%%file 20_10_2020.news.ru.lenta.1.sovok_nebesnii
%%parent 20_10_2020
%%tags ussr,lenta.ru,russia
 
%%endhead 

\subsection{Совок небесный}

Россияне мечтают возродить СССР. Кому выгоден миф о коммунистическом рае?

«Нерушимый» Советский Союз распался почти три десятилетия назад. Россияне
хорошо помнят времена закрытых границ, пустых полок и дефицитной мебели из
стран Восточного блока. Но, несмотря на весь негатив, жители страны добрым
словом вспоминают советскую эпоху и хотят в нее вернуться. Это подтверждают
социологические исследования. Кто и почему ностальгирует по рухнувшей стране?
Кто заставил россиян поверить в миф о величии Советского Союза и полюбить
Совок? А главное — кому выгодна легенда об ушедшей прекрасной и сытной жизни? В
рамках спецпроекта «Мифы о России» «Лента.ру» отвечает на эти вопросы вместе с
историками и социологами.

\subsubsection{«Фантастический райский мир»}

Российские социологи отмечают странный парадокс: о желании вернуться в недавнее
прошлое все чаще говорят совсем молодые россияне — им еще не исполнилось 30
лет, они родились уже в новой России, выросли при Путине, привыкли к гаджетам,
«Тиктоку» и «Тиндеру», а жить хотят в эпоху Советского Союза.

Социологи констатируют, что молодые люди оказались крайне восприимчивы к мифу
об идеальной стране. Им греют душу рассказы о счастливом детстве, самом вкусном
мороженом по 13 копеек, пионерских лагерях, магазине «Балатон», старых
«Волгах», пляжах Крыма, кавказских курортах и народной дружбе, а главное —
социальных гарантиях: дешевой еде, стабильной зарплате, достойной пенсии и
бесплатных квартирах.

\ifcmt
pic https://icdn.lenta.ru/images/2020/10/02/18/20201002182416360/preview_10f5c78e2142e905395d0af567162752.png
\fi

То, что «развитой социализм», помимо всех вышеперечисленных благ, имел и
серьезные негативные стороны, из национальной памяти уже как-то стерлось.

Как, например, необходимость ни свет ни заря ехать на другой конец города,
потому что там продают молоко из бочки, или бежать сломя голову в магазин, где
выкинули (выпустили в продажу) обычные женские колготки. Наконец, знаменитая
загадка советских времен: «Длинная, зеленая, пахнет колбасой». Вряд ли многие
сегодняшние молодые люди с ходу скажут, что правильный ответ — «электричка из
Москвы». Ведь часто жители регионов действительно ездили в столицу за такими
обычными и общедоступными сейчас продуктами, как колбасные изделия, кофе и
апельсины из Марокко.

Как подчеркивает историк Николай Сванидзе, современный молодой человек, который
говорит о том, что хотел бы жить во времена Советского Союза, попросту не
отдает себе отчета в том, с какими бытовыми трудностями, кажущимися сейчас
невероятными, приходилось тогда сталкиваться большинству советских граждан.

\ifcmt
pic https://icdn.lenta.ru/images/2020/10/01/14/20201001143313275/preview_3472c111a9f2c3de6187a0e7b0ec9d66.png
\fi

«Это как — нет в магазине мяса? Значит, в соседнем есть. Как это — нет десяти
сортов колбасы или сыра? Значит, в соседнем есть. Поэтому когда им говорят, что
было все поровну, была справедливость, они хотят этот фантастический райский
мир. Вот и все», — объясняет историк.

До умов современной молодежи информация о минусах «социалистического образа
жизни» попросту не доходит.

Это подтверждает и социолог Денис Волков, неоднократно убеждавшийся в этом при
работе с фокус-группами.

Молодые люди говорят, как хорошо было жить при Брежневе. А когда им возражаешь
— отвечают, что им мама рассказывала

Денис Волков социолог

75 из 100 россиян называют советскую эпоху лучшим временем в истории страны, а
не согласились с этим утверждением только 18 процентов опрошенных. При этом
вернуться на путь Советского Союза согласны только 28 из 100 граждан страны.

Забегая вперед, отметим, что социологи находят вполне логичное объяснение столь
значительному несоответствию числа убежденных в «лучших временах» СССР и
желающих снова встать на путь социализма (75 против 28 процентов).

\ifcmt
pic https://zoomdecorate.rambler.ru/elhfzoawo/MTR3ZmZyLjFrem41QHsiZGF0YSI6eyJBY3Rpb24iOiJQcm94eSIsIlJlZmZlcmVyIjoiaHR0cHM6Ly9sZW50YS5ydS9hcnRpY2xlcy8yMDIwLzEwLzIwL215dGgyLyIsIlByb3RvY29sIjoiaHR0cHM6IiwiSG9zdCI6ImxlbnRhLnJ1IiwiTG?qdbd=lua1R5cGUiOiJpbWFnZS8qIn0sImxpbmsiOiJodHRwczovL2ljZG4ubGVudGEucnUvaW1hZ2VzLzIwMjAv&rfoukt=MTAvMDEvMTQvMjAyMDEwMDExNDMzMzc4NTIvcHJldmlld19iMTM1ZDczNDQxMzk4NTI5ODQ0NjQyZDBkND&xrkl=BhNGUzMy5wbmcifQ%3D%3D
\fi
 
Интересный опрос \footnote{https://fom.ru/TSennosti/12875} на близкую тему
провел в 2016 году Фонд «Общественное мнение».

Респондентам задавали вопрос, хотели бы они родиться в другое время, и
предлагали выбрать понравившуюся историческую эпоху. В итоге большинство
предпочло остаться в современности (72 процента). Однако, недовольны своим
временем оказалась почти пятая часть опрошенных (19 процентов). Любопытно, что
среди молодых людей 18-22 лет эти цифры составили 67 и 24 процента
соответственно. То есть молодежь оказалась более подвержена желанию сменить
эпоху.

Примерно треть из тех, кто хотел бы жить в другое время (6 процентов из 19) в
качестве предпочтительной для них эпохи назвали Советский Союз 20-70-х годов ХХ
века. Желающих перенестись в 80-90-е годы оказалось тоже относительно много,
хотя и втрое меньше (2 процента из 19). То есть в общей сложности вернуться в
недавнее прошлое хотела почти половина из тех, кому настоящее по каким-то
причинам не нравится (8 процентов из 19). И это при том, что респонденты были
вольны выбрать совершенно любой исторический период и страну или вообще
отправиться в будущее.

\ifcmt
pic https://icdn.lenta.ru/images/2020/10/02/17/20201002173105389/original_2a9675884edcd0fc1cb7f3526594b560.png
\fi

Ностальгию часто приписывают россиянам как типичную национальную черту. И
действительно — тема тоски по ушедшим временам богато представлена, например, в
классической русской литературе.

\ifcmt
pic https://icdn.lenta.ru/images/2020/10/01/14/20201001143302152/preview_e7b71fd386ad6476f92bef08246a29ce.png
\fi

С тем, что это «очень по-русски», соглашается, в частности, и автор книги
«Страна утраченной эмпатии. Как советское прошлое влияет на российское
настоящее», социальный психолог Алексей Рощин.

«Это такая русская традиция. Вспомнить хотя бы Антона Павловича Чехова и его
знаменитые пьесы. Например, "Дядя Ваня". "Мы отдохнем, мы еще увидим небо в
алмазах..."», — отметил он в беседе с «Лентой.ру».

Член совета Вольного исторического общества Никита Соколов находит объяснение в
восприятии исторического прошлого и настоящего. «Люди, конечно, живут не в
объективной реальности, а в своих представлениях о ней. Точно так же и в
отношении истории: люди живут не в объективном знании о прошлых событиях, а в
том, как они сформулированы в общественной повестке и что считается правильным
в этом году», — считает историк.

\ifcmt
pic https://icdn.lenta.ru/images/2020/10/06/17/20201006170821167/preview_f5c9854797c36a7c186a7ba3d684cfd1.jpg
\fi

\subsubsection{Добрые дедушки}

Переоценка отношения к недавно ушедшим эпохам неизменно сопровождается
изменением отношения к лидерам тех лет. Те, кто жил в годы застоя или в 90-е,
хорошо помнят, как много ходило шуток и анекдотов о Леониде Ильиче Брежневе и
Борисе Николаевиче Ельцине, многие из которых уже порядком подзабылись. Не
меньше звучало и критики, порой весьма жесткой (особенно в адрес Ельцина).
Причем если Брежневу доставалось за отсутствие реформ, то Ельцина кляли ровно
за противоположное — времена шоковой терапии у нас до сих пор многие поминают
недобрым словом.

Это всегда так происходит. Всегда меняется представление о людях после их ухода
с течением времени. Когда они есть, о них рассказывают анекдоты, обсуждают
какие-то их слабости. А человек, уже ушедший в историю, бронзовеет в сознании
людей.

Николай Сванидзе историк

Вместе с тем Брежнева, например, нередко называют лучшим из советских
правителей.

\ifcmt
pic https://icdn.lenta.ru/images/2020/10/01/14/20201001143320084/preview_de065f44f0469547adc1234d29f5fc52.png
\fi

Образ Брежнева уже в значительной степени «забронзовел» в сознании россиян —
хотя бы потому, что с его смерти прошло уже почти 40 лет. При этом, судя по
всему, на его популярность в народе совершенно не влияет отсутствие зримых
успехов в годы его правления: войну он не выиграл, Гагарина в космос не
запустил.

Всероссийский центр изучения общественного мнения (ВЦИОМ) выяснил, что в эпоху
правления генерального секретаря ЦК КПСС Леонида Брежнева, которую многие
привыкли называть периодом застоя, хотели бы жить сразу 37 процентов россиян.
Это всего на три процента меньше, чем тех, кто предпочел бы остаться в
современности.

\ifcmt
pic https://icdn.lenta.ru/images/2020/10/01/14/20201001143344190/preview_2ad2b8e8d7530e62509aac1da8ba64f6.png
\fi

И Алексей Рощин находит этому довольно логичное объяснение.

Брежнев наиболее соответствует этому состоянию общества, которое мы сейчас
имеем. Здесь он обретает черты такого Санта-Клауса, советского Санта-Клауса. В
чем-то он аналог того, как американцы понимают своего Рузвельта. Такой
советский Рузвельт

Алексей Рощин, социальный психолог

Возможно, более ощутимая трансформация в массовом сознании ждет в будущем и
образ Бориса Ельцина. Сложно сказать, в Санта-Клауса ли он превратится или в
Деда Мороза, однако, скорее всего, для этого потребуются запрос со стороны
общества и усилия отвечающих за это социальных институтов, считает Денис
Волков.

«Это зависит от многих показателей, в том числе от того, кто будет работать с
памятью о том времени, какие институты будут эту память не просто хранить, а
передавать, перерабатывать, изучать», — пояснил социолог.

\subsection{Напуганное поколение в «угаре перестройки»}

Большое влияние на мировоззрение детей оказывает мнение родителей, дедушек и
бабушек. Поэтому эксперты считают, что ответ на вопрос, почему молодежь верит в
мифы о Совке, нужно искать в событиях недавнего прошлого — прошлого взрослых.
Речь прежде всего о перестройке и последующем распаде страны, которые произвели
очень сильное впечатление на современников.

В частности, Алексей Рощин называет это явление феноменом «напуганного
поколения» и связывает массовое желание вернуться в прошлое именно с ним.
Эксперт отмечает, что на закате существования Советского Союза настроения людей
кардинально отличались от того, что мы видим сейчас.

В конце 80-х — начале 90-х был так называемый «угар перестройки». Конечно,
настроения тогда и сейчас отличались разительно. Это, можно сказать, два разных
общества. Хотя с одним корнем, естественно.

Алексей Рощин, социальный психолог

По его словам, именно конец советской эпохи знаменовался общей уверенностью в
том, что виной всем проблемам — прогнившее государство, прогнивший строй,
который сдерживает некий невероятный потенциал советских людей. Это повсеместно
распространенное мнение хорошо оттенялось пустыми полками магазинов по всей
стране, которые были зримым свидетельством той самой немощи советской власти.
Известный факт, что на рубеже 1980-1990-х масштабы товарного дефицита достигли
невиданных доселе масштабов, которые поражали даже привыкших к перманентному
«кризису предложения» советских людей. С прилавков пропало буквально все, в том
числе и совсем обычные вещи, такие как носки и сигареты.

«Люди ощущали себя Ильями Муромцами, которые 33 года просидели на печах, и за
это время, как в русской народной былине, не иссохли, не впали в анемию, как
должно было быть по медицинским представлениям, а наоборот — накопили
невероятную силушку богатырскую, которая их распирает. Они были готовы свернуть
горы и добиться невероятных успехов буквально во всем», — рассказал Рощин.

\ifcmt
pic https://icdn.lenta.ru/images/2020/10/01/14/20201001143329540/preview_c4d9a6d33133e2a1ab9610731cb8e1fc.png
\fi

Многим перестройка принесла новые надежды. Однако затем произошли распад страны
и болезненные преобразования в экономике, которые вернули всех с небес на
землю. «Новый кризис, развал Советского Союза — это нанесло травму большинству
россиян. Именно распад страны, распад экономики, распад общего пространства», —
соглашается Денис Волков.

Люди с энтузиазмом бросились в рыночную стихию, но достаточно быстро осознали,
что это трудно, а главное — опасно. Советского человека, привыкшего к спокойной
размеренной жизни в эпоху «развитого социализма», такое новое для него
перманентное состояние опасности буквально шокировало. Причем, считает Рощин,
память о том шоке до сих пор так и не ушла.

Все помнят ощущение облома. Причем именно своего облома. Грубо говоря, нация
считает, что она оказалась не на высоте. Это ощущение настолько болезненное,
это настолько неприятно и даже стыдно признавать, что в нашем коллективном
бессознательном это все вытесняется вообще. Вытесняется и отрицается.

Алексей Рощин, социальный психолог

\subsubsection{Хотят назад, но не совсем}

Однако, как оказалось, со стремлением вернуться в советскую эпоху все не так
просто. Исследователи неизменно фиксируют, что желающих вернуться во времена
СССР становится куда меньше, если задавать людям уточняющие вопросы.

«Все-таки большинство людей, признавая какие-то положительные моменты в
советском прошлом, особенно экономические — что все были равны, что была
уверенность в завтрашнем дне, что была зарплата, — не хотят туда возвращаться.
Если мы спрашиваем, хотите ли вы жить в Советском Союзе, большинство скажут:
нет, жить не хотим, но было хорошо», — констатирует Волков.

\ifcmt
pic https://icdn.lenta.ru/images/2020/10/01/14/20201001143238379/preview_5ab247411d095f71a79d24b298907249.png
\fi

Информационный вакуум, отсутствие возможности выехать за рубеж, тупая
пропагандистская долбежка с никому не нужными ритуальными партсобраниями —
стоит напомнить о них респондентам, как образ «Совка небесного» сразу меркнет.

Этот «Совок небесный» предстает как образ. Это такой Совок, очищенный от
негатива. С одной стороны, гарантированная работа, зарплата и квартира, а с
другой — чтобы никаких очередей, дефицита и даже власти партии.

Алексей Рощин, социальный психолог

«В данном случае это такой нарисованный образ, где противоречия не преодолены
за счет какой-то общественной идеи, а просто игнорируются. То есть образ в
форме сказки», — пояснил Рощин, добавив, что это также признак некоторой
опустошенности и надломленности постсоветского общества.

Николай Сванидзе тоже полагает, что в наше время многие воспринимают Советский
Союз как некий «фантастический райский мир», где все было поровну и была
справедливость. При этом он уверен, что «никто не хочет вспоминать про
коммунальные квартиры с одним туалетом на 30 человек».

\ifcmt
pic https://icdn.lenta.ru/images/2020/10/01/14/20201001143244126/preview_17eedbfca6aba7c5645e189fab035778.png
\fi

«Вот этот расширенный сталинский миф касается всего Советского Союза. Что это
была страна всеобщего равенства, всеобщего братства», — считает историк.

Человеческая память такова, что запоминается прежде всего хорошее. Социологи
отмечают, что респонденты, отвечая на общие вопросы о какой-либо эпохе —
совершенно неважно, о советских годах речь или о 90-х, — неизменно первым делом
называют ее положительные стороны. И лишь потом, после уточняющих вопросов,
говорят о негативных сторонах жизни в те времена. Это накладывает свой
отпечаток на то, что принято называть ностальгией по прошлому.

\ifcmt
pic https://icdn.lenta.ru/images/2020/10/02/17/20201002173105389/original_2a9675884edcd0fc1cb7f3526594b560.png
\fi

Похожие процессы сейчас происходят в массовом сознании в отношении 90-х годов,
которые ознаменовались стремительными переменами в обществе и экономике по
сравнению с предшествующей советской эпохой. Несмотря на резкое падение уровня
жизни значительного числа наших соотечественников, многие из них в последние
годы все чаще видят в той эпохе положительные стороны. Как правило, они
являются зеркальным отражением минусов эпохи предшествующей, то есть советской:
в магазинах появилось изобилие, исчез идеологический диктат коммунистической
партии, исчезли пристальный надзор и контроль за повседневной жизнью людей, на
порядок уменьшилось число разного рода проверок со стороны контролирующих
органов. Социолог Денис Волков отмечает, что сейчас действительно идет процесс
некоторой переоценки тех лет.

«Все-таки любое время несет как хорошее, так и плохое, и пусть в массовой
культуре и политическом лексиконе за 90-ми закрепилось название "лихих", говоря
о них, прежде всего назовут хорошее, но если спросить о плохом — тоже все
назовут, так же как и с советским периодом», — отметил он.

В то же время эксперты подчеркивают, что о советской эпохе и о 90-х годах
ностальгируют разные люди, даже если речь идет о молодежи моложе 30 лет,
которая ни того, ни другого времени не помнит. Причем эта тоска может
становиться все сильнее.

«В 90-х была полная свобода, был ветер весны, в воздухе витало освобождение от
всех пут, которые были при советской власти. Наполнились прилавки магазинов.
Впервые появились магазины, полные товаров. И я, конечно, понимаю, что это
может вызывать ностальгию», — резюмирует Николай Сванидзе.

Игорь Дмитров
