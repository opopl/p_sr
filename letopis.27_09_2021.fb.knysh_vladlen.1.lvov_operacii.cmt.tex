% vim: keymap=russian-jcukenwin
%%beginhead 
 
%%file 27_09_2021.fb.knysh_vladlen.1.lvov_operacii.cmt
%%parent 27_09_2021.fb.knysh_vladlen.1.lvov_operacii
 
%%url 
 
%%author_id 
%%date 
 
%%tags 
%%title 
 
%%endhead 
\subsubsection{Коментарі}

\begin{itemize} % {
\iusr{Оксана Лой}
Возможно это не последняя операция для парня, может там условно через 20 лет надо новое ..

\begin{itemize} % {
\iusr{Владлен Кныш}
\textbf{Оксана Лой} ну...наверное хотелось бы раз и все

\iusr{Тетяна Кириченко}
\textbf{Оксана Лой} насколько мне известно, трансплантаций ,,навсегда,, к сожалению не бывает. В среднем 10 лет. С приемом специальных препаратов и под медицинским наблюдением регулярным. Рокфеллеру, например, сердце пересаживали 7! раз
\end{itemize} % }

\iusr{Станислав Махотин}

у нас ситуация с трансплантологией поставленна так что выбирать особо не
пиходится, ранее многие ездили в РБ для таких операций, но сейчас я так понимаю
на этом фоне есть проблемы

\begin{itemize} % {
\iusr{Владлен Кныш}
\textbf{Станислав Махотин} ну, догадываюсь, что выбирать не приходиться -)))))))

\iusr{Станислав Махотин}
\textbf{Владлен Кныш} 

на самом деле довольно серьезная проблема. чтоб решить ситуацию нужен закон о
безусловном согласии на донорство, но этот закон может нормально работать
только в государстве где нормально работают гос институты, чего у нас нет. Так
что пока замкнутый круг и множество людей умирают которых можно было бы спасти


\iusr{Владлен Кныш}
\textbf{Станислав Махотин} опять страна виновата...)

\iusr{Станислав Махотин}
\textbf{Владлен Кныш} не страна, государство, политики ну и конечно мы, граждане которые их выбираем
\end{itemize} % }

\end{itemize} % }
