% vim: keymap=russian-jcukenwin
%%beginhead 
 
%%file 22_11_2020.news.lnr.lug_info.lugansk_inform_center.1.professor_atojan
%%parent 22_11_2020
 
%%url http://lug-info.com/comments/one/professor-arsentii-atoyan-my-soobschestvo-ischuschikh-a-ne-nashedshikh-806
 
%%author 
%%author_id lugansk_inform_center
%%author_url 
 
%%tags 
%%title Профессор Арсентий Атоян: "Мы - сообщество ищущих, а не нашедших"
 
%%endhead 
 
\subsection{Профессор Арсентий Атоян: \enquote{Мы - сообщество ищущих, а не нашедших}}
\label{sec:22_11_2020.news.lnr.lug_info.lugansk_inform_center.1.professor_atojan}
\Purl{http://lug-info.com/comments/one/professor-arsentii-atoyan-my-soobschestvo-ischuschikh-a-ne-nashedshikh-806}
\ifcmt
	author_begin
   author_id lugansk_inform_center
	author_end
\fi

\ifcmt
  pic http://img.lug-info.com/cache/d/9/atoyan.jpg/w620h420.jpg
  width 0.6
	caption Профессор Арсентий Атоян: "Мы - сообщество ищущих, а не нашедших"
\fi

\index[names.rus]{Атоян, Арсений!член совета Философского монтеневского общества, ЛНР, 22.11.2020}

О недавно вышедшей книге культуролога Нины Ищенко "Монтень в Луганске"\Furl{http://lug-info.com/comments/one/kulturolog-nina-ischenko-kniga-monten-v-luganske-novyi-shag-na-puti-poiska-novykh-form-802}
ЛуганскИнформЦентру рассказывает доктор философских наук, профессор кафедры
психологии и конфликтологии Луганского государственного университета имени
Владимира Даля, член совета Философского монтеневского общества (ФМО) Арсентий
Атоян.

\subsubsection{ПЯТЫЙ СБОРНИК}

Пятый Монтеневский сборник посвящен тридцатилетию самого феномена ФМО, но,
как у нас принято, и тем проблемам, что волнуют выступавших или
намеревающихся это сделать в ближайший год. Иными словами, в сборнике
участвуют те, кто хочет и с тем, с чем хочет. Никому из желающих у нас
выступить за тридцать лет не было отказано, а это почти две сотни людей!
Причем некоторые выступали не только не единожды, но и десятки раз.
Неполный список докладов по годам мы поместили во втором сборнике
"Четверть века с философией".

Это может вызвать недоумение. Давать слово всем – значит и тем, кто против
истины. Разумеется. Мы - сообщество ищущих, а не нашедших. Негуманным
предложениям давался отпор, но не кулаками, а аргументами. Помнится
инцидент с украинским националистом. Гнев на георгиевскую ленту: "Все вы
колорады…" и тут же монтеневка, сидевшая слева от националиста, достает из
дамской сумочки георгиевский бант и вешает себе на грудь… Нельзя сказать,
что мы эклектики, ибо рамки гуманистических ценностей остаются
незыблемыми, однако, мы прежде всего – разномышленники. Этим наше общество
отлично от множества других существующих или давно распавшихся
интеллектуальных сообществ.

\subsubsection{КАК ВСЕ НАЧИНАЛОСЬ?}

Общество создано осенью 1990 года. Первое собрание прошло 15 ноября в Доме
политпросвещения – тогда истфаке педагогического института, теперь дом
Фемиды напротив Генеральной прокуратуры. Часть луганской интеллигенции в
предчувствии поражения перестройки, не сговариваясь, стихийно заняли
скептическую позицию к тому, что должно уйти, и к тому, что должно прийти.
Запугивание перспективой гражданской войны только входило в господствующий
дискурс. Сегодня трудно сказать, была ли позиция скептиков единственно
верной. Но из ситуации каждый выбирался сам. Разномыслие создавало
многовекторность суждений. Сегодня часть луганских скептиков оказалась за
линией соприкосновения и даже пытается организовать ФМО в Северодонецке.
Однако все, что я знаю об этом, - первое собрание в Северодонецке
закончилось шашлыками, чего мы себе за 30 лет так и не позволили.

\subsubsection{10 МИНУТ НА РЕПЛИКУ}

Собрания проходили по разным формам проведения, пока не выработалась
привычная. 40 минут доклада, три круга вопросов. Порой очень длинных,
доходило до анекдотичного вопроса размером в доклад, многое вспоминается с
улыбкой, остановились в формате до десяти минут желающему на реплику.
Среди первого поколения присутствующих на собраниях были известные в
Луганске люди: Александр Еременко, Константин Деревянко, Илья Кононов,
Константин Зарубицкий, Сергей Прасолов, Петр Нестеров, Василий Кузнецов,
Олег Соловьев, Юлия Молчанова, Владислав Карабулин и многие другие.

\subsubsection{ПЕРЕМЕНА ПРОБЛЕМНОГО ПОЛЯ}

В последующие годы особенно важную роль в перемене проблемного поля, его
приближении к реальным, не диссертабельным темам за счет расширения тематикой
самой жизни или ее отражений в культуре сыграли Сергей Бойчук, Игорь
Шупчинский, Александр Ермашев, Анатолий Зеленько, Виталий Даренский, Василий
Попов, Алексей Блюминов, Владимир Сабадуха, Виктор Даниленко, Руслан Егоров,
Владимир Рудоквас и многие другие.

\subsubsection{ТРИ ЧАСТИ ЦЕЛОГО}

Книга разбита на три части, название которых даны в подзаголовке "Монтень
в Луганске. Фокус осознания, спектр возможностей, периферическое видение".

Часть первая "Фокус осознания" посвящена самому монтеневскому обществу,
его истории и современному состоянию. В статье Нины Ищенко анализируется
путь и основные вехи, дается представление как об инфантильном, так и
зрелом содержании тематики обсуждений. Статья Виталия Даренского проводит
параллели метода майевтики и практики наших собраний, находит нечто общее
между интеллектуальными практиками философов в разных частях мира и в
разные эпохи.

Часть вторая "Спектр возможностей" посвящена докладам, пока только
предполагающимся к прочтению, среди коих особенно интересен доклад Валентины
Патерыкиной о Кафке с разбором его произведений через лабиринт его собственной
жизни, что довольно свежо читается и неплохо прозвучит в критически настроенной
аудитории. Статья Константина Деревянко является очень острым комментарием к
Парижской декларации европейских интеллектуалов, различению ложной и истинной
Европы, тупику стремления украинской нации и культуры в ложную Европу. Статья
Нины Ищенко "Принципы социологии воображения в книге Елены Заславской
"Донбасский имажинэр"", где анализируется сборник современной луганской
поэтессы в контексте философского учения Жильбера Дюрана, известного
французского антрополога, развивавшего теорию коллективных архетипов.
Разбираемые стихи Заславской также помещены в этом разделе. Статья Андрея
Кондаурова – маленькое и существенное предупреждение о пророческом характере
жертв Теночтитлана.

Вообще латиноамериканская тематика – не редкий гость в нашем обществе, как
индейская, индийская, китайская и вообще экзотическая. Иногда с ней выступают
люди, побывавшие в Индии, Японии, Таиланде, Турции и других странах. Думаю, нет
смысла пересказывать представленные статьи, отсылаю к сайту "Одуванчик".

Часть третья "Периферическое видение" представляет отклики на все сборники.
Среди рецензентов – творческие люди, известные в литературном мире России,
Украины и Луганска. В этом разделе помещены отклик Вука Задунайского, Натальи
Матвеевой, Александра Еременко, Александра Сигиды-младшего, критика и секретаря
Союза писателей ЛНР Андрея Чернова, четырежды чемпионки мира по тайскому боксу,
факира, йогини, учившейся в Бомбее Ольги Бодрухиной, а также монтеневцев,
проживающих ныне в других городах – Евгения Гнатенко и других. Большинство из
них помнится своими докладами на ФМО и, надеемся, еще примет участие в наших
сборниках, а возможно, и в собраниях.

\subsubsection{ДОРОЖИТЬ ЛЮДЬМИ}

Вообще у нас принято говорить, что в обществе побывали люди самые разные – от
киевской профессуры до городских сумасшедших. Есть просто уникальные люди с
интересным замыслом жизни или судьбой странствий – Александр Сгонников и
представленный в сборнике своей развернутой репликой Георгий Елпашев. ФМО
дорожит людьми, и это притягивает. География наших поклонников включает Питер.
Наш первый спонсор Николай Щербаков – предприниматель из этого города.

\subsubsection{ПАМЯТИ КАРБАНЯ}

В сборнике есть сведения обо всех авторах и перечень работ нашего товарища,
ушедшего в мир иной весной этого года, когда пятый сборник готовился к печати,
Владимира Яковлевича Карбаня, которому мы обязаны продолжением неформального
общения монтеневцев на протяжении многих лет, едва ли не с года основания… Его
отзыв "Из-под глыб" соседствует с отзывом Александра Еременко "Поверх барьеров"
– они были друзьями – дает представление об отличии позиций двух крыльев ФМО,
разделенных войной.  Кстати, именно Карбань сделал ФМО известным в
интернет-сообществе, помещая там наши анонсы и другие материалы.

Карбань был примером независимого интеллектуала, который, будучи в курсе всего,
что творится в культуре, а он был не только сотрудником художественного музея,
квалифицировано разбирался в философии, архитектуре, скульптуре – чего только
стоят его заметки о двух скульптурах Владимира Даля – живописи, поэзии, музыке,
литературе и других искусствах, но и человеком, чья эрудиция составляла не
главное, а вспомогательное средство распространения доброты, отзывчивости,
внимания к молодым талантам. У него в доме часто бывали гости из Москвы и
других городов России, у него можно было узнать о любых новинках в области
культуры, его оценки текущих и исторических событий вызывали не ожесточенные
споры ввиду их взвешенности. За тысячи километров он дружил по скайпу с
известными людьми в культурном мире, часто гостеприимно принимал журналистов,
художников, музееведов, к нему сходились ниточки самых разных людей, носителей
различных мировоззрений, что было ценным для ФМО. Он действительно не делал
докладов, но горячо участвовал в их обсуждении и создавал атмосферу легкого
настроения при кризисных ситуациях, из которых было выйти сложнее без его
заинтересованной и примирительной манеры ведения дискуссии. В этом он остался
образцом. Словом, мы потеряли друга, и наше желание посвятить ему сборник
естественно, ибо и сам Монтень отрицал в "Опытах" ритуалы скорби, но учил
поминать людей их добрыми делами.

\subsubsection{БЫТЬ ПОЛЕЗНЫМ}

В наших планах привлечение большего числа выступающих, качественные доклады
новых лиц. После долгих лет в педуниверситете и Далевском университете мы
опробуем новую площадку – Горьковскую библиотеку, где каждую среду, минус
каникулы студентов, на которых мы также имеем виды, мы встречаемся в половине
третьего. Что касается дальнейшего продвижения наших сборников, то нам кажется
перспективной тема антропологии восточнославянских народов. Акцент на общность
судьбы и государствообразования, что представляется одним из путей собирания
людей, духа и земель на завтрашнее торжество и пиршество славянства и входящих
в Русский мир народов. И что представляется альтернативой нынешнему унылому
самостийничеству.

Хочется поискать и путей мира – усталость давит людей, мешает жить… Вместе мы
сила, в отдельности – недоразумение. Но тематика еще только прорабатывается.
Хочется быть не ангажированными той или иной стороной, но полезными Народной
Республике и философскому сообществу. Хотя Монтень и говорил, что каждый должен
сам себе быть обществом, но и единение разномыслящих обществ необходимо в мире,
где ломается пространство взаимодействия, которое складывалось веками. Нужно
быть готовыми к неожиданностям. Мысль не всемогуща, но проникает дальше лба и
носа.  Приобщайтесь к монтеневскому кругу! 

ЛуганскИнформЦентр — 22 ноября — Луганск

