% vim: keymap=russian-jcukenwin
%%beginhead 
 
%%file 03_08_2019.stz.news.ua.golos_ukrainy.1.kapitan_jahty_mrie_pro_novu_mandrivku
%%parent 03_08_2019
 
%%url http://www.golos.com.ua/article/320111
 
%%author_id news.ua.golos_ukrainy
%%date 
 
%%tags 
%%title Капітан яхти, що здійснив першу українську навколосвітню подорож, мріє про нову мандрівку
 
%%endhead 
 
\subsection{Капітан яхти, що здійснив першу українську навколосвітню подорож, мріє про нову мандрівку}
\label{sec:03_08_2019.stz.news.ua.golos_ukrainy.1.kapitan_jahty_mrie_pro_novu_mandrivku}
 
\Purl{http://www.golos.com.ua/article/320111}
\ifcmt
 author_begin
   author_id news.ua.golos_ukrainy
 author_end
\fi

\ifcmt
  ig https://i2.paste.pics/c9f5cb70d46dda3debd64dd1ffafce6b.png
  @wrap center
  @width 0.9
\fi

\begin{center}
  \em\color{blue}\bfseries\Large
	Три океани, 13 морів і 30 тисяч миль навколо світу пройшов під синьо-жовтим
				прапором за чотири роки київський екіпаж яхти \enquote{Лелітка}.
\end{center}

\ii{03_08_2019.stz.news.ua.golos_ukrainy.1.kapitan_jahty_mrie_pro_novu_mandrivku.pic.1}

\textbf{Складався він з двох людей — капітана, кандидата технічних наук Валерія Дісановича Петущака} (на знімку)

\ii{03_08_2019.stz.news.ua.golos_ukrainy.1.kapitan_jahty_mrie_pro_novu_mandrivku.pic.2}

\textbf{та його дружини, теж науковця, Наталії Леонідівни Македон. У далеку незвідану
путь яхтсмени вирушили 25 років тому, 31 липня 1994-го. В дорогу покликали не
лише моря і океани, а й бажання відкрити світові Україну, що понад три століття
була колонією Росії й асоціювалася тільки з Російською імперією.}

На далекі мандрівки Валерій Петущак захворів ще у дитинстві. Солоний океанський
бриз в'ївся в душу, коли він почав читати книжки Жюля Верна. Пригодницькі
романи живили мрії про далекі мандрівки в незвідані країни роками. Вперше
Дісанович, як його називають найближчі друзі, спробував здійснити навколосвітку
у 1990 році. Але коли з екіпажем він вийшов в океан на яхті \enquote{Дніпро},
виявилося, що його друзі-яхтсмени з наддніпрянського міста лиш хотіли знайти,
як вони казали, \enquote{нормальну} країну і там осісти. У капітана були інші думки і
мрії... Тож, дійшовши до Нової Зеландії, він повернувся додому один — без
екіпажу і яхти. Як людина високих моральних і життєвих принципів ніколи не
дозволяв собі себе жаліти. Навпаки, пообіцяв: \enquote{Клянусь, обійду навколо світу на
власній яхті під українським прапором!} І так щиросердечно, що вже через
тиждень доля повністю змінила його життя. Його друг запропонував зайнятися
бізнесом і за кілька років Валерій Петущак зміг купити власну яхту.

Назвати її допомогла пані Наталя. Дружина капітана любить читати геніальну Лесю
Українку і у її творах віднайшла такі рядки: \enquote{Ми тоді знайдемо з папороті
квітку/ З неба зірвем зірку, золоту лелітку...}

Так і з'явилась ідея назвати яхту \enquote{Леліткою}. Під час мандрів і пана Валерія, і
пані Наталю мандрівники з різних країн запитували, що означає ця назва. Ті
відповідали, що це відображення світла зірки в океані. Іноземцям це надзвичайно
подобалося. А вже коли мандрівники повернулися з подорожі, то з книжки
Братка-Кутинського \enquote{Феномен України} дізналися, що Лелітки — це дочки Лелі,
богині води, богині Всесвіту, їхнім батьком був Полель — бог сонця. Тобто,
Лелітки — це втілення жовто-блакитного, сонця і води.

Під час подорожі екіпаж \enquote{Лелітки} чекало багато небезпечних пригод — зустріч з
сомалійськими піратами, боротьба з 12-бальними штормами, акулами і крокодилами.

Вижити допомогла віра в добро і океан. Під час подорожі на яхті \enquote{Дніпро}, коли
екіпаж вийшов в екваторіальні води Тихого океану, де сонце стоїть майже
вертикально і гріє так, що можна розплавитись, а температура в каюті 50
градусів, вода за кормою 35, друзі Петущака почали жалітися на океан, на те, що
солона вода виїдає очі й шкіру. Для них це була ворожа стихія, а для Дісановича
— рідна.

\enquote{Коли зранку виходиш на палубу, дивишся на схід сонця, бачиш безмежну площу
океану і відчуваєш, що це все твоє, — розумієш, що щасливий}, — каже він. Тоді
біля Японських островів команда \enquote{Дніпра} потрапила у тайфун. Швидкість вітру
була понад 200 км на годину, а хвилі досягали 15 метрів. Яхта двічі
переверталась так, що щогла йшла під воду. «Коли ти бачиш, як на тебе котиться
цей гребінь, сотні тонн води, коли бачиш люту безкінечну енергію океану,
розумієш, що він може знищити будь-яке судно», — каже капітан. Врятував екіпаж
і яхту Дісанович, який годинами чергував біля стерна — він ніколи не відчував
страху перед стихією.

Українські мандрівники, заходячи на \enquote{Лелітці} в іноземні яхтклуби для
поповнення провіантом, паливом та відпочинку, розповідали яхтсменам з усього
світу про Україну. Її історію, витоки, про трипільську цивілізацію, одну з
найдавніших на усіх континентах, традиції. Усіх вражала тотожність українських
і полінезійських орнаментів та міфів. Один з них розповідає про те, як з
далекої країни, прабатьківщини полінезійців, де взимку можна ходити по воді,
яку навесні оживляє білий птах з чорною ознакою, предки полінезійців пішли в
Індію. А потім, збудувавши катамарани і напнувши вітрила, переселилися на
райські острови.

Під час подорожі Дісанович задумав написати історію своїх мандрів. Склалася
вона на основі щоденників, які вела Наталія Македон.

Книжка під назвою \enquote{Солоний гопак} побачила світ у 2005 році і швидко стала
бестселером. Згодом, завдяки братам Капрановим (теж, до речі, завзятим
яхтсменам), у видавництві \enquote{Зелений пес} вийшли спогади \enquote{Ходіння за три океани}.
Цю книжку перевидано у Польщі...

Любов до морів і океанів легендарний капітан передав своєму сину Сергію та
внукам Олександру і Валерію, які теж стали досвідченими мореходами. Онук
Валерій Петущак-молодший здійснив мрію свого діда — пройшовся навколо
Антарктиди. Він вже двічі побував, як помічник капітана яхти, з туристичною
експедицією біля її берегів, де відвідав станцію \enquote{Академік Вернадський}.

Не полишає мрій про нові мандри і Дісанович. Щотижня під вітрилами він з
друзями виходить на Дніпро, а в короткі години перепочинку пише наукові статті
та нову книжку.

\textbf{Георгій ЛУК'ЯНЧУК, Світлана ЧОРНА.}

\textbf{Фото Георгія ЛУК'ЯНЧУКА.}
