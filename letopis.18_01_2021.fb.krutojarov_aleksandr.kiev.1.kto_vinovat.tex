% vim: keymap=russian-jcukenwin
%%beginhead 
 
%%file 18_01_2021.fb.krutojarov_aleksandr.kiev.1.kto_vinovat
%%parent 18_01_2021
 
%%url https://www.facebook.com/permalink.php?story_fbid=1110034316101650&id=100012852751261
 
%%author_id krutojarov_aleksandr.kiev
%%date 
 
%%tags chelovechnost,chelovek,obschestvo,strana,ukraina
%%title Кто виноват?
 
%%endhead 
 
\subsection{Кто виноват?}
\label{sec:18_01_2021.fb.krutojarov_aleksandr.kiev.1.kto_vinovat}
 
\Purl{https://www.facebook.com/permalink.php?story_fbid=1110034316101650&id=100012852751261}
\ifcmt
 author_begin
   author_id krutojarov_aleksandr.kiev
 author_end
\fi

Кто виноват?

В том, как мы живем, не виноват никто. Все дело в нас, поскольку каждый
повинуется своему «Я»: простые люди и магнаты, члены кабинета министров и
обычные клерки. И никакие законодательные меры здесь не помогут, пока мы не
изменим себя. Конечно, можно принимать всевозможные законы, но кто будет их
исполнять?

Другими словами, дело тут не за властью, а за народом, то есть, за всеми нами.
Ведь ни одно правительство неспособно действовать вопреки реальному и
действенному волеизъявлению людей. И потому использует ту силу, которая
наиболее востребована в сегодняшнем мире – силу удовлетворения человеческих
потребностей, поскольку (чтобы мы не говорили по этому поводу) в данный момент
они максимально эгоистичны.

А теперь представим, что каким-то чудом к власти пришли политики, желающие
реально улучшить наше благосостояние. Причем всех и каждого, без каких-либо
исключений.

Так что, многие с этим согласятся по существу? Когда от голословных призывов
придется переходить к делу и начинать заботиться о других, помогать в чем-то, в
том числе за счет собственных льгот и привилегий? Но в таком случае можно
лишиться самого основного наслаждения, а именно: вкуса к жизни за счет утраты
ощущения своего «превосходства» перед остальными людьми, в любом аспекте
социальной и общественной жизни. И кому тогда надо такое равенство?

Вот и выходит, что проблема здесь не в деньгах, не в богатстве, не в почестях
или славе, а во взаимоотношениях между людьми. Поэтому без самовоспитания, без
внутренней революции внутри себя тут не обойтись. Это значит, что не надо
«брать Бастилию» или «штурмовать Зимний дворец» – не поможет. Вместо этого надо
разобраться с собственной эгоистической природой, со своим варварством внутри.

А уличные беспорядки только ускоряют процесс, и не более. Сами по себе, они не
принесут никому ни малейшей пользы. Все сведется лишь к тому, что жизнь будет
дорожать, а проблемы нарастать повышенными темпами. Самое интересное, что этот
парадокс уже виден во многих странах мира, когда самые благоразумные и
реалистические планы преодоления кризиса ведут к обратным результатам: богатые
зарабатывают еще больше, и разрыв в обществе между различными слоями населения
только увеличивается. Более того, бедные платят еще более высокие налоги, а
цены продолжат свой рост. И это притом, что в мировом сообществе многие
государства всё искреннее пытаются помочь друг другу. Но, как говорится,
«система» уже не желает нас слушаться. До тех пор, пока мы не изменим свое
внутреннее мироощущение.
