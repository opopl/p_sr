% vim: keymap=russian-jcukenwin
%%beginhead 
 
%%file 20_01_2018.stz.news.ua.mrpl_city.1.zimnie_zabavy
%%parent 20_01_2018
 
%%url https://mrpl.city/blogs/view/zimnie-zabavy
 
%%author_id burov_sergij.mariupol,news.ua.mrpl_city
%%date 
 
%%tags 
%%title Зимние забавы
 
%%endhead 
 
\subsection{Зимние забавы}
\label{sec:20_01_2018.stz.news.ua.mrpl_city.1.zimnie_zabavy}
 
\Purl{https://mrpl.city/blogs/view/zimnie-zabavy}
\ifcmt
 author_begin
   author_id burov_sergij.mariupol,news.ua.mrpl_city
 author_end
\fi

В те времена, на которые припало детство людей, ныне названных \enquote{детьми войны},
зимы в Мариуполе были, не в пример нынешним, снежными. Признаками устоявшейся
снежной погоды были цепи, закрепленные на задних колесах грузовиков, лошаденки,
запряженные в сани вместо привычных телег, а также немалой высоты сугробы,
нагроможденные между тротуаром и проезжей частью улицы. Они лежали до оттепели,
покрываясь грязными брызгами от проезжающих автомашин. О снегоуборочной технике
тогда слыхом не слыхивали, и единственными орудиями борьбы с \enquote{осадками в виде
снега} были фанерная лопата да лом с расплюснутой оконечностью – это для того,
чтобы наледи дробить. Квартальные (была такая общественная должность в каждом
квартале), дворники пристально следили за тем, чтобы владельцы частных домов
(по их мнению, жители города второго сорта), своевременно прочищали тротуары
вдоль своих домовладений и посыпали их золой – благо этого добра хватало,
печи-то топились углем. Так что для взрослых зима с ее снегопадами привносила в
их быт дополнительные заботы и неудобства.

Для ребятни же обилие снега, особенно первого, было радостью, поскольку сулило
множество развлечений и забав. Первые из них – снежки, которыми обменивалась
дворовая ребятня. Тут и прохожему могло попасть. Когда баталия надоедала,
ребята, вываленные в снегу с головы до пят, брались за сооружение снежной бабы.
Для этого скатывали из холодящего руки белого искрящегося ковра три шара.
Ставили их друг на друга, а сверху водружали дырявое ведро. Несколько кусочков
золы изображали глаза и зубы, а обломок ветки был носом. Затем срочно из
чуланов, сараев и чердаков доставались санки, коньки и лыжи. Малышей в
нахлобученных до самых глаз шапках, укутанных в шали, возили на саночках мамы,
старшие братья и сестры, и не столько для удовольствия, сколько для того, чтобы
отвезти чадо в детский сад или на присмотр к родственникам, пока мама будет на
работе.

\enquote{Санки}, \enquote{салазки}, как хотите, так их и называйте. Малышня съезжала на них с
пригорков во дворах. А сорванцы постарше отправлялись покататься на улицы с
крутыми спусками. Для юных аборигенов Торговой улицы это был прежде всего
относительно короткий участок дороги, ведущий на Малофонтанную улицу.
Излюбленным местом для катания был квартал улицы Розы Люксембург (Земской) от
улицы Карла Либкнехта (Митрополитской) до кирпичной стены, огораживающей
территорию бывшей электростанции и старейшей насосной станции городского
водопровода. Здесь более ловкие саночники умудрялись продлить удовольствие,
свернув на кривой отрезок улицы, выходивший к улице Фонтанной. Бывало, что
отправлялись на Советскую улицу. \enquote{Стартовали} от улицы Карла Либкнехта, мчались
сломя голову до Комсомольской (Евпаторийской) улицы, до самого тупика. Для
самых отчаянных \enquote{трассой} служил Гамперовский спуск. Кое-кто из наших
впечатлительных современников вправе спросить: это же так опасно - кататься на
проезжей части улицы! Там же машины! Дело в том, что в описываемый отрезок
истории нашего города на тех улицах, где мариупольская ребятня соревновалась,
кто быстрее достигнет конечного пункта, ни автотранспорта, ни транспорта
гужевого, т.е. лошадок, запряженных в сани, практически не было. Вероятно,
шоферов и возниц смущала крутизна покрытых наледью дорог.

Счастливы были те ребята, которым коньки достались в наследство от старших
братьев и родителей или были куплены на барахолке по случаю. Это были
\enquote{пионерки} с остроконечным передом довоенного производства, \enquote{снегурки} с
загнутыми в бублик носами – спортивное приспособление, изобретенное еще в
дореволюционные годы. И \enquote{нурмисы} - ими выписывали вензеля на льду еще дедушки
и бабушки на мариупольском катке, который устраивался на замерзшем озере
Домаха. Изначально все эти коньки, кроме \enquote{снегурок}, должны были крепиться
заклепками к подошвам специальных ботинок. \enquote{Снегурки} же фиксировались на обуви
с помощью так называемых жабок. 

Но у кого в послевоенные годы были ботинки для коньков? Зимней обувью тогда
были бурки - стеганые матерчатые сапоги, описание которых заняло бы много
места, и очень редко - валенки. Понятно, что ни к тем, ни к другим коньки не
приклепаешь и тем более не прикрепишь \enquote{жабками}. Ребятня нашла выход из
положения – стала привязывать коньки к буркам и валенкам веревками, не ведая
того, что именно так поступали голландцы. Говорят, именно они изобрели
устройство для катания на льду, только вместо веревок использовали кожаные
ремни. Мальчишки, которым стальные коньки не достались в наследство или не были
куплены, делали их сами. К деревяшке особым образом прикреплялись заточенные
стальные полоски из расклепанной и тщательно заточенной толстой проволоки. В
деревяшке раскаленным прутом прожигались отверстия. Это для того, чтобы
пропустить веревки крепления. Конечно, на таких коньках заниматься фигурным
катанием было невозможно, но прокатиться по обледенелой дорожке с ветерком еще
как получалось.

В начале 50-х годов в магазине \enquote{Динамо} (он находился в небольшом помещении
рядом с Городским отделом милиции на улице 1-го Мая, позже он переехал в дом №
45 на проспекте Ленина) в продаже появились коньки, прозванные \enquote{дутышами}.
Такое название они получили за стальную трубку, к которой было прикреплено
лезвие, и конусовидные стойки пластин для приклепывания конька к ботинку. Они
предназначались для игры в хоккей с шайбой, который только-только начал
культивироваться в Советском Союзе. Чуть позже в том же магазине \enquote{Динамо} можно
было приобрести и ботинки для коньков.

Где показывали свое мастерство юные мариупольские конькобежцы? Да на каждом
пятачке во дворах, на улицах, где застывал лед. На всех поверхностях с
утоптанным снежным покровом. На льду замерзшего моря, рискуя провалиться в
полынью. Но особым шиком все же было зацепиться крючком за задний борт
какой-нибудь полуторки и прокатиться по дороге, высекая лезвием коньков искры
из проглядывающих сквозь ледяную корку булыжников. 

Запомнились три катка, которые заливали, пожалуй, на рубеже сороковых и
пятидесятых годов. Один был в Центральном сквере, второй - в Городском саду, а
третий - на стадионе \enquote{Спартак}. Там, конечно, были самые лучшие условия
для катания. К тому же из динамика, установленного на центральной трибуне,
лилась веселая музыка. Сейчас трудно вспомнить, действовали все эти катки
одновременно или каждый из них в разные зимы. Справедливости ради, нужно
упомянуть здесь и тех мальчишек, и девчонок, которые занимались буерным спортом
на водной станции Городского дома пионеров. Руководил ими демобилизованный
военный  моряк и спортсмен Леонид Андреевич Сосновский. Для непосвященных
сообщим, что буер – \enquote{небольшая платформа, установленная на специальные коньки и
оснащенная мачтой и парусом}. Чтобы управлять буером, мчащимся с попутным
ветром по льду, нужно было обладать и мужеством, и сноровкой. Да так, чтобы, не
дай бог, не приблизиться к открытой воде канала, пробитого ледоколом для судов,
участников зимней навигации.

Что закрепилось в памяти о зимних забавах? Обледенелые, промокшие насквозь
бурки, одеревеневшие пальцы на руках от десятков слепленных снежков, цыпки на
запястьях, заживающие только к середине апреля, разбег и плюханье на санки,
восторг от бешеного спуска с горы, онемевшие ступни, стянутые веревками
крепления коньков. И радость от стремительного движения, морозца, окрашивающего
щеки и нос в ярко-красный колер, от яркой голубизны февральского неба,
искрящегося только что выпавшего снега. Юность и отрочество - время бесшабашных
поступков. Вспоминается один день из зимних каникул, проведенный на катке
стадиона. Был прекрасный лед, коньки хорошо скользили. Рядом было много
знакомых ребят и девчат. Одни плавно скользили по льду кругами, то обгоняя друг
друга, то меняя направление движения, другие, таких было не так уж много,
пытались выполнить кое-какие элементы фигурного катания. О том, что пора домой,
вспомнилось только после того, как начали ныть голеностопы. До сих пор
вспоминается переход на коньках, с нестерпимой болью, от стадиона у Городского
сада до Торговой улицы. Но все-таки дошел. После этого несколько дней болели
ноги.
