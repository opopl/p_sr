% vim: keymap=russian-jcukenwin
%%beginhead 
 
%%file 27_09_2021.fb.blazhnova_rekuha_julia.1.mysli_posle_boja_usik.cmt
%%parent 27_09_2021.fb.blazhnova_rekuha_julia.1.mysli_posle_boja_usik
 
%%url 
 
%%author_id 
%%date 
 
%%tags 
%%title 
 
%%endhead 
\subsubsection{Коментарі}

\begin{itemize} % {
\iusr{Шарапиденова Дана}
Живи Украина! @igg{fbicon.trophy} 

\iusr{Андрей Коваленко}
 @igg{fbicon.hands.applause.yellow} 

\iusr{Людмила Конвисарова}
Ежедневный труд простых пацанов, которые догребли до вершины Олимпа - лучше не скажешь!

\iusr{Анна Ермолаева}
 @igg{fbicon.100.percent} 

\iusr{Анна Жарова}
Как хорошо сказано

\iusr{Наталія Василівна Криворученко}

\ifcmt
  ig https://scontent-yyz1-1.xx.fbcdn.net/v/t39.1997-6/s168x128/17633073_1652591054767295_6333333619058147328_n.png?_nc_cat=1&ccb=1-5&_nc_sid=ac3552&_nc_ohc=9l4KPN4DSSMAX8sHGKl&_nc_ht=scontent-yyz1-1.xx&oh=cc48d6277aea6c7675afdf5b09c0a6d1&oe=61586D3C
  @width 0.1
\fi

\iusr{Olga Belousova}
Замечательный пост. Одновременно практически безоценочный, но расставляющий акценты. Спасибо.

\iusr{Жандос Уап}
Пост @igg{fbicon.thumb.up.yellow}{repeat=3} . На сколько могуч русский язык!

\iusr{Olga Brown}

Не зовсім так. Англійці так само реагують на політичні, расові і релігійні
погляди людей у всіх сферах суспільства. Просто в них не прийнято обговорювати
це в компанії щоби не зіпсувати людям настрій. Це нажаль не нова тема коли
талановиті люди є носіями сумнівних поглядів. Найвідоміший приклад Вагнера,
музика якого була заборонена в Ізраїлі. Усика я вітаю з перемогою, поглядів
його я не поділяю .

\begin{itemize} % {
\iusr{Oleg Zabielin}
Вагнер был нацистом? )) Или вы другие его взгляды считаете сомнительными?

\iusr{Olga Brown}
\textbf{Oleg Zabielin} почитайте о нем. Там все есть... Период Гитлера, его статья...

\iusr{Oleg Zabielin}
Я прошу прощения, он умер ещё до рождения Гитлера. О чем вы?

\iusr{Olga Brown}
\textbf{Oleg Zabielin} він написав свою відому статтю про те що євреям не дано творити мистецтво. Гітлер вважав його найважливішим німецьким композитором, він був заборонений в Ізраїлі і здається першим хто наважився виконувати його музику в Ізраїлі був Баренбойм.
\end{itemize} % }

\iusr{Елена Маленкова}

Юля, спасибо за этот пост.

\iusr{Жандос Уап}
Для Кличко большой привет из станции Жангиз-Тобе.Казахстан

\iusr{Иван Бавыкин}
Золотые слова!

\iusr{Ульяна Лысак}
до мурашок...

\iusr{Евгения Кучма}

Британцам просто уже много лет спокойнее, чем нам, они свое с соседями
отвоевали. А когда-то ведь и фамилию королевской семьи пришлось новую
изобретать, и Баттенбергов на Маунтбаттенов менять. Было бы наверное и нам
легче жить вот так по-современному, не заморачиваться этим, но это не только
выбор Украины, к сожалению. Немного как про жертву насилия говорить - а чего ты
грустная такая, и по подворотням боишься ходить.

\begin{itemize} % {
\iusr{Ольга Смаль}
\textbf{Евгения Кучма} вот писала то же самое практически. Вообще разные контексты.

\iusr{Евгения Кучма}
\textbf{Ольга Смаль} да, нас к сожалению, нельзя сравнивать с пост-империями, нужно с пост-колониальными странами. В той же Индии, думаю, часто "не те" взгляды вызывали проблемы, и сейчас тоже. На Западе этого часто не понимают - "как вы можете ощущать то же, вы же белые".

\iusr{Ольга Смаль}
\textbf{Евгения Кучма} да я бы вообще скорее сравнивала нас не с пост-колониальными, а скорее с постсоветскими странами просто. И да, мы одна из немногих из этих стран (не считая прибалтийских), где в принципе живы политические процессы, регулярно меняются президенты и есть активная часть неаморфного гражданского актива, которая готова даже на жертвы ради свободы, свободных выборов и тд.
\end{itemize} % }

% -------------------------------------
\ii{fbauth.smalj_olga.ukraina}
% -------------------------------------

Вообще разные исторические отрезки становления у государств, определения своей
идентичности + настоящая, кровавая, действующая война с Россией откладывают
отпечаток. Люди в Великобритании могут позволить себе роскошь быть изи и ни о
чем эдаком не думать. В Украине — нет. Тут постоянно гибнут военные, у каждого
есть знакомый раненный, погибший, а агрессор РФ ведет с 2014 года жесткую
гибридную войну (в том числе информационную), и любое слово любого публичного
украинца может быть использовано не в пользу Украины. Нас буквально пытаются
стереть, растворить в «русском мире», регулярно рассказывают что мы —
недогосударство, у нас недорусский язык и прочую ахинею, давят политически,
экономически и по закону противодействия мы втройне упираемся. Поэтому градус
накала страстей, агрессии по сопряженным темам в разы выше: все элементарно
пропитано горечью о потерях. И автоматом ответственность на публичных людях
лежит тройная — такие времена. Усик — великолепный спортсмен, спасибо ему за
победу.

\begin{itemize} % {

% -------------------------------------
\ii{fbauth.ivanov_vladimir.kovrov.vladimir_obl.rossia}
% -------------------------------------

\textbf{Ольга Смаль} из 14 тысяч погибших в конфликте на востоке Украины только 4 тысячи украинских военных, остальные жители непризнанных республик

\iusr{Ольга Смаль}
\textbf{Владимир Иванов} 

не надо мне рассказывать кто где погиб и что там происходит: у меня в Донецке
дядя и двоюродный брат с женой. Я могу вам сама рассказать как российское
дерьмо вроде Гиркина, Гиви и прочей швали и отбросов на российские деньги
повалило в Украину и тыкало рос флаги на наши администрации. Как мои друзья
видели колонны военной техники из Рф. Как другие друзья находились при штурме
картогрграфического центра в Крыму, где уроды из Росии убили нашего военного. И
как Рф до этого много лет на деньги рос налогоплательщиков кормила Януковича и
говеную Партию регионов, состоящую на 80\% из донбасских бандюков и накачивала
Восток и юг своими деньгами, полит консультантами, своими смыслами и
раскачивала по разным осям примерно как и Севастополь. Украина отстаивает свою
территориальную целостность: ее поставили в положение, когда она вынуждена это
делать. Напомню еще вам сколько погибло мирных жителей в Чечне, когда россиюшка
отчего-то не дала ей самоопределиться (отчего же?): только в Первой чеченской
войне по данным правозащитных организаций погибло 30-40 тыс мирного населения.

\iusr{Владимир Иванов}
\textbf{Ольга Смаль} перечислите сотни и тысячи российских военных, которых вы обменяли, одних украинцев на других меняете все восемь лет

\iusr{Ольга Смаль}
\textbf{Владимир Иванов} так что по убитым мирным чеченским жителям и свободной республике Ичкерия?

\iusr{Владимир Иванов}
\textbf{Ольга Смаль} 

это то мирное, которое взрывало Буйнакск, или может те миряне которые детишек с
родителями захватывали в Беслане, или те мирные, обвязанные поясами шахидов,
взявшие в заложники посетителей мюзикла "Норд-ост" на дубровке в Москве, а
может те бородачи мирные, что вторглись в Дагестан в 1999? Спонсировались
саудитами и США, а воевал весь сброд, те арабы-наемники мирные или дудаевцы? В
глаза мне смотри, это Донбасс приехал в вам в Киев, Фраковск, Львов на танках,
может это те мародеры из добробатов и ВСУ, которые отправляли награбленные веши
"Новой поштой"? В глаза мне смотри!

\iusr{Kata Turner}
\textbf{Olga} підтримую вас, сил вам  @igg{fbicon.hands.pray}  а на іншого коментатора не звертайте увагу. То громадянин расєї, якому звичайно краще видно через рашатв події в нашій країні.

\iusr{Ольга Смаль}
\textbf{Kata Turner} та я його забанила, вже не бачу:).

\end{itemize} % }

\iusr{Катерина Шиянова}
Отличный пост, спасибо  @igg{fbicon.hands.applause.yellow} 

\iusr{Людмила Кондакова}
Юля, спасибо.

\iusr{Мирослава Павловская}

У UK свои проблемы. Нельзя объективно сравнить Британию и Украину. Если
сравнивать, то наша история ближе к истории Ирландии. Ну, и про маленькие
города: Шевченко вырос в Киеве на Оболони, Кличко выросли в Киеве.

А победа Усика на ринге в Лондоне - это победа украинца, это - победа Украины.

\begin{itemize} % {
\iusr{Yuliya Blazhnova-Rekukha}
\textbf{Miroslava Pavlovskaya}

\ifcmt
  ig https://scontent-yyz1-1.xx.fbcdn.net/v/t1.6435-9/243286973_2933849746853800_6113287000771658697_n.jpg?_nc_cat=101&_nc_rgb565=1&ccb=1-5&_nc_sid=dbeb18&_nc_ohc=Jut9SZnWM3IAX8JUU5g&_nc_ht=scontent-yyz1-1.xx&oh=5adf4789c58255fad07e633260606a63&oe=6179FF2F
  @width 0.3
\fi

\iusr{Мирослава Павловская}
\textbf{Yuliya Blazhnova-Rekukha} , слухайте, народився, це - не жив. Не пишіть казок і не вводьте в оману людей

\iusr{Ольга Смаль}
\textbf{Yuliya Blazhnova-Rekukha} можно при скрине чуть больше захватить:)

\ifcmt
  ig https://scontent-yyz1-1.xx.fbcdn.net/v/t1.6435-9/243260606_10159044244481939_2843035842572056997_n.jpg?_nc_cat=101&_nc_rgb565=1&ccb=1-5&_nc_sid=dbeb18&_nc_ohc=nsOPK-zedh8AX9mvsz7&_nc_ht=scontent-yyz1-1.xx&oh=1db5e921c0dfddeaae698760aca07f6e&oe=61769F33
  @width 0.3
\fi

\iusr{Yuliya Blazhnova-Rekukha}
\textbf{Olga Glushchenko Smal} це все міняє і Андрій Шевченко нездара?)

\iusr{Мирослава Павловская}
\textbf{Yuliya Blazhnova-Rekukha} , це Ви щось собі нафантазували. Я поважаю Андрія Шевченка і горджусь його перемогами та досягненнями для моєї Батьківщини.

\iusr{Yuliya Blazhnova-Rekukha}
\textbf{Miroslava Pavlovskaya} попередній коментар не до вас. Ви мені не цікава

\iusr{Ольга Смаль}
\textbf{Yuliya Blazhnova-Rekukha} 

я думала, ви журналіст, а не маніпулятор.  @igg{fbicon.smile}  вам мирослава
вказала на фактологічну неточність (яка малює портрет, має давити на емоції,
жаліти маленького Андрійка, який «попри усе» з маленького містечка (столиця
України Київ, приблизно 3 мільйони мешканців) йшов до своєї футбольної мрії). А
ви у відоповідь: Так що? Шевченко нездара?:) Маніпуляція чистої води. Звісно,
Андрій —найталановитіша людина, яка багато працювала і йшла до успіху ціною
здоров’я.

У тексті посту взагалі у цілому порівнюються реалії соціуму сталої розвинутої
держави з багаторічною історією, колишньої очільниці імперії і з іншого боку
молодої країни, що розвивається — з 30-річною історією державності, яка саме
зараз проходить процеси самоідентифікації, відокремлення від культурного,
політичного і інших полей Росії; держави, що знаходиться під тиском країни, яка
сама була імперією і зараз не хоче «відпускати», а навпаки займається процесом
«собирания земель русски», як вони самі вважають. (Почитайте останні статті
Путіна, там чітко усе сказано). 

Для них Україна — це частина Росії. Вони вже проковтнули Біларусь, а ми ще
спираємося і як заноза у сідниці. Цей опір не може не нести наслідків у
суспілстві. Великобританія і Україна — взагалі зараз різні світи, різні
суспільні явища, різні культурологічні, політичні проблеми і процеси, виклики.
Ви порівнююте непорівнювальне. 

І повторю: в нас натурально йде війна з конкретною державою. А ви така ніби не
розумієте: «чего это украинцы такие агрессивные? Цепляются вот это вот,
какие-то вопросы задают, недовольны чего-то? С чего б это? Можно ведь просто
чилить и радоваться жизни!  Более позитивными быть надо!»просто ніби не
помічаючи усього цього болю і бекграунду і ігноруючи контекст, у якому усе це
відбувається. 

Можливо із Лондона це якось інакше усе відчувається, але коли допомогаєш довго
волонтерам, коли хтось зі знайомих служив у ВСУ, а у самої двоюрідний брат і
дядя у Донецьку і ти наслишана про «підвали», дивишся на усе зовсім інакше. це
тільки поверхово у країні поза ОРДЛО мирно, зримі і незримі нитки війни усюди.
У військовому госпіталі у Києви, де робили операцію моєму татові пару років
тому ти це бачиш наживо: молоді/немолоді ранені хлопці «звідти» нагадують. 

І українці мають право на свою злість, агресивну реакцію до усього, що
пов‘язано з державою-агресором, РФ. А публічні особи мають бути більш
обережними у висловах і діях під час війни, тому що усе це гостро, усе це
болить. Коли усе закінчиться має пройти ще багато років щоб відболіло, а градус
емоцій знизився.  Тоді у когось виникне моральне право запитувати аналогічні
питання і щось там порівнювати.

\end{itemize} % }

\iusr{Iryna Dmytrova}

Юля, пост Ваш прекрасный, справедливый, душевный. Вы не только талантливая
журналист, но и очень чуткая, чувственная, эмоциональная женщина, как я
воспринимаю Вас по Вашим репортажам.

Мы часто в разговорах о жизни у них и у нас жалуемся на Украину, например:«а у
нас пенсионеры не могут даже себя прокормить на пенсию, не то, что в круиз
поехать..» и т.п.

Но это же МЫ! Это У НАС! Что же мы живем и все жалуемся? И никак не выберемся
из постсоветской нищеты духа? А, может, мы на пути к нашей мечте?

\iusr{Ирина Свирипа}
Юля, вы профессионал! За фото отдельное спасибо!

\iusr{Sasha Kot}
Должно пройти время. И Украина будет жить спокойно. Когда никто не будет посягать на независимость и границы.

\iusr{Марина Кищенко}
Бесподобно написано! Да, эти ребята сделали себя сами и именно с их именами ассоциируют Украину.

\begin{itemize} % {
\iusr{Татьяна Федорченко}
\textbf{Марина Кищенко} они все результат работы Тренера @igg{fbicon.heart.exclamation}, от мала до велика @igg{fbicon.hands.raising} 

\iusr{Марина Кищенко}
\textbf{Татьяна Федорченко} Заслуги тренера никто не умоляет, просто из миллионов мальчишек именно они стали символами современной Украины.
\end{itemize} % }

\iusr{Vasil Kalynchuk}
Усик молодець. Претензії на адресу України не зрозумілі.

\iusr{Александр Гладкий}
Троє найвідоміших у Європі українців

\iusr{Liliya Tutaj}
Юля, абсолютно с вами согласна. Я прослезилась читая ваш пост. Фото - просто гениальное.

\iusr{Valentina Valentina}
 @igg{fbicon.hands.applause.yellow}{repeat=5} 

\iusr{Sergey Sheltushev}
І Лондона відніше

\iusr{Игорь Иванюкович}
Такой бред любит писать откровенная вата. Которым какая разница.

\begin{itemize} % {
\iusr{Yuliya Blazhnova-Rekukha}
\textbf{Igor Ivanyukovich} не тот случай

\iusr{Олександр Пшенік}
\textbf{Yuliya Blazhnova-Rekukha} саме той

\iusr{Ирина Бордюг}
\textbf{Игорь Иванюкович} как вы надоели кастрюлеголовые,,лжепатриоты.сам то для прославления страны ,что сделал?

\iusr{Игорь Иванюкович}
\textbf{Ирина Бордюг} , Мои заслуги знают многие, но дело не в этом, Вы сторонник русского мира и этим всё сказано.
\end{itemize} % }

\iusr{Sergey Sheltushev}
Картина "Троє колаборантив"

\iusr{Vitalii Vitalii}
Особливо в Лондоні поважають Вільяма Воллеса

\iusr{Брати Грім}

Цих людей ідентифікуюють з Україною не тільки по їх спортивних досягненнях і
нагородах, але й по їх вчинках, словах і діях. Вони невільно стали публічними
особами зі всіма наслідками. Цей допис був би доречний після нашої Перемоги,
розвалу Вставалосколєнії і масових арештів агентів кгб по всьому світу. Одним
словом - на сьогоднішній день це допис вати про вату.

\iusr{Олександр Коваленко}
Зараз розплачусь. А як там обнімашки із силовиками країни-агресора під час війни?

\iusr{Татьяна Федорченко}
Не только в Историю Украины, но и в международные Залы Славы  @igg{fbicon.love.letter} 

\iusr{Дмитрий Загуменный}
 @igg{fbicon.hands.applause.yellow}{repeat=3} 

\iusr{Serhiy Mazur}
Очень глубоко и правильно. Спасибо

\iusr{Саша Лепесій}

\ifcmt
  ig https://scontent-yyz1-1.xx.fbcdn.net/v/t39.1997-6/s168x128/17640308_1652591141433953_2515677274297073664_n.png?_nc_cat=1&ccb=1-5&_nc_sid=ac3552&_nc_ohc=_gvTY4f3ofcAX-LalWi&_nc_ht=scontent-yyz1-1.xx&oh=617a107bffb8b8760c825eba26388f81&oe=6158A22E
  @width 0.1
\fi

\iusr{Валерій Преподобний}

Було б дуже добре, щоб ті двоє що зліва на фото не пішли в політику, як той, що
справа на фото, бо Україна цього не переживе.

\begin{itemize} % {
\iusr{Petro Chernii}
\textbf{Валерій Преподобний} та нормальний мер/політик, кращий за відсотків 80 українських політиків

\iusr{Валерій Преподобний}
\textbf{Petro Chernii} ну якщо порівнювати з льонею космосом, то напевно і кращий. @igg{fbicon.face.smiling.eyes.smiling} 
\end{itemize} % }

\iusr{Олександр Пшенік}

Британці не ведуть війну з агресором, війська, церкву та народ якого прославляє
Джошуа. Нахєр хай іде той тупорилий Усик

\begin{itemize} % {
\iusr{Александр Гичко}
\textbf{Олександр Пшенік} больной?

\iusr{Олександр Пшенік}
\textbf{Александр Гичко} невиліковно

\iusr{Александр Гичко}
\textbf{Олександр Пшенік} заметно

\iusr{Олександр Пшенік}
\textbf{Александр Гичко} як одразу помітно й те, що ти тупорило ватний

\ifcmt
  ig https://scontent-yyz1-1.xx.fbcdn.net/v/t39.30808-6/243194400_663140071329286_1817719314775148003_n.jpg?_nc_cat=105&_nc_rgb565=1&ccb=1-5&_nc_sid=dbeb18&_nc_ohc=nfxxYvHIFc0AX91zORP&_nc_ht=scontent-yyz1-1.xx&oh=1081050869be36d604154951d89c8501&oe=61583C0D
  @width 0.3
\fi

\iusr{Александр Гичко}
\textbf{Олександр Пшенік} ну по тебе видно, что ты нацист и бЭндэровец.

\iusr{Олександр Пшенік}
\textbf{Александр Гичко} так, я Бандерівець, і пишаюсь цим. А нацист - саме ти і такі як ти

\end{itemize} % }

\iusr{Lesya El}

Люди всегда добрее в стране, где нет коррупции. Честные правила для всех,
налоги идут на стариков, больницы, детей-сирот... А у нас Каждая власть
покрывает коррупцию, сами имеем и вам даём... А старых и слабых - в пропасть,
нагрузка для бюджета. Аморальность всегда вызывает злость, и никогда не
приносит счастья

\iusr{Сергей Гордиенко}
Забавно, жалко что неправда и манипуляция) но попытка была)

\begin{itemize} % {
\iusr{Yuliya Blazhnova-Rekukha}
\textbf{Sergii Gordiienko} конкретно, что неправда и в чём манипуляция?
\end{itemize} % }

\iusr{Petro Chernii}

Який гарний якісний пропогандистсько-маніпулятивний допис!

Звучить як "Такакаяразніца, якщо спортсмен успішний і золото взяв!?"


\end{itemize} % }
