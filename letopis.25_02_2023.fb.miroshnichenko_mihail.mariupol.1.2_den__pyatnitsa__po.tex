%%beginhead 
 
%%file 25_02_2023.fb.miroshnichenko_mihail.mariupol.1.2_den__pyatnitsa__po
%%parent 25_02_2023
 
%%url https://www.facebook.com/mms.couch.ips/posts/pfbid02xVy98Q4g9hKYFFprNcZQdcmidTw8LFUtkeBazm3Mmue7BdXnchFjZskcr8cvn166l
 
%%author_id miroshnichenko_mihail.mariupol
%%date 25_02_2023
 
%%tags mariupol,mariupol.war,dnevnik
%%title 2 день. Пятница. Под Мариуполем с севера-востока и востока приходили сведения об обстреле пригорода
 
%%endhead 

\subsection{2 день. Пятница. Под Мариуполем с севера-востока и востока приходили сведения об обстреле пригорода}
\label{sec:25_02_2023.fb.miroshnichenko_mihail.mariupol.1.2_den__pyatnitsa__po}

\Purl{https://www.facebook.com/mms.couch.ips/posts/pfbid02xVy98Q4g9hKYFFprNcZQdcmidTw8LFUtkeBazm3Mmue7BdXnchFjZskcr8cvn166l}
\ifcmt
 author_begin
   author_id miroshnichenko_mihail.mariupol
 author_end
\fi

▶️ 2 день. Пятница. 

Под Мариуполем с севера-востока и востока приходили сведения об обстреле
пригорода.

Люди начали собираться и выезжать в центр города.

Первым делом население города и пригорода Мариуполя собиралось в очереди к
банкоматам, кассам и заправкам.

Вода стала на вес золота, мука, каши и сахар, стали продуктами номер один.

Прийдя домой, я проанализировал какое место в квартире может подойти для
безопасного нахождения при обстреле города, так как моя квартира была
расположена на 7 этаже с прекрасным видом на парк  ( самолёт) и выезд из города
в сторону Бердянска и Запорожья, это западная сторона я решил занять комнату с
выходом в южную сторону.

Одна из комнат, это была детская в которой ещё днём ранее спал мой сын,
находилась она ближе к лифтовой шахте и была окружена несущими стенами, окна
выходили на  соседний дом в южном направлении.

Сначала я спал в кровати своего сына, так как активных боевых действий ещё не
было.

Город превратился в муравейник, многие были в панике и сметали всё с прилавков
магазинов и стояли в очередях на заправку по несколько часов в надежде
заправится.

Пройдя по квартире я начал искать чем бы обезопасить своё убежище, закрыть окно
в детской, для защиты себя  и своего четвероногого друга Джека, если ударной
волной выбьет стекло.

Для защиты окна я разобрал в гостиной стол и использовал столешницу как щит.

Ночь прошла тихо. 

👉 Продолжение следует...

«Наша главная 

задача - не заглядывать в туманную даль будущего, а действовать сейчас, в
направлении, которое нам видно».

%\ii{25_02_2023.fb.miroshnichenko_mihail.mariupol.1.2_den__pyatnitsa__po.cmt}
