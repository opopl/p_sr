% vim: keymap=russian-jcukenwin
%%beginhead 
 
%%file 05_05_2018.stz.news.ua.mrpl_city.1.mariupolskij_tramvaj
%%parent 05_05_2018
 
%%url https://mrpl.city/blogs/view/mariupolskij-tramvaj
 
%%author_id burov_sergij.mariupol,news.ua.mrpl_city
%%date 
 
%%tags 
%%title Мариупольский трамвай
 
%%endhead 
 
\subsection{Мариупольский трамвай}
\label{sec:05_05_2018.stz.news.ua.mrpl_city.1.mariupolskij_tramvaj}
 
\Purl{https://mrpl.city/blogs/view/mariupolskij-tramvaj}
\ifcmt
 author_begin
   author_id burov_sergij.mariupol,news.ua.mrpl_city
 author_end
\fi

\ii{05_05_2018.stz.news.ua.mrpl_city.1.mariupolskij_tramvaj.pic.1}

Молодцы исследователи Мариупольского краеведческого музея и любознательные
работники трамвайно-троллей\hyp{}бусного управления. Они собрали материалы по истории
городского электротранспорта, а теперь еще выложили в интернете для всеобщего
обозрения. Но всякая история не только учит, но и побуждает к размышлениям.
Итак, трамвай на электротяге в Париже появился в 1895 году, в Лондоне - в 1901
году, в Москве - в 1899 году, в Киеве – в 1892 году. Кстати, киевский
электрический трамвай был первым в Российской империи.

В Мариуполе транспортная проблема многие годы особенно не волновала. Город был
мал - и в церковь, на базар, в гости и на работу ходили пешком. Имелись у нас,
конечно, не столько прогрессивные, сколько состоятельные люди, которые держали
лошадей и экипажи. На этих экипажах они сами и их жены, и их тещи могли поехать
на базар за покупками, отправиться в гости в близлежащее село. Мало ли куда
можно поехать, если есть свой транспорт? Дотошный читатель напомнит, что у нас
в Мариуполе были легковые извозчики. Да, были, но удовольствие проехаться на
линейке, фаэтоне или ландо было не из дешевых.

Транспортная проблема в нашем городе впервые обратила на себя внимание с
началом строительства в 1886 г., а затем и эксплуатации морского торгового
порта, а также разрастающегося припортового поселка. А через относительно
короткое время - и со строительством металлургических заводов и поселков при
них близ станции Сартана в конце XIX века. Судя по отчетам, мариупольская
городская власть не сидела сложа руки. В одном из исторических источников
зафиксировано, что 29 ноября 1899 года выступил секретарь комитета по
устройству в городе электроосвещения и трамвайного движения Н.С. Караманов и
предложил несколько линий будущего трамвая. Вот они: \emph{Александровская площадь (в
наши дни – площадь Театральная) – заводы \enquote{Никополь} и
\enquote{Провиданс}; Александровская площадь – Городской сад; Александровская
площадь – морской порт}. Предложение г-на Караманова, как известно, не было
реализовано. Мировая война, Октябрьская революция, Гражданская война,
разрушенные заводы и фабрики, восстановление, коллективизация. До трамваев ли
было?

Только в 1931 году было принято решение о создании в нашем городе трамвая.
Технический проект этого сооружения был разработан крупным отечественным ученым
и инженером \textbf{Павлом Петровичем Копняевым}. Были субботники и воскресники, было
привлечение материальных и людских ресурсов предприятий. Итогом всеобщих усилий
было завершение строительства одноколейной трамвайной линии от гавани до улицы
Франко (ныне – проспект Металлургов). Это историческое - без всякой иронии –
событие свершилось 1 мая 1933 года и было отмечено митингом. 

\textbf{Читайте также:} 

%https://mrpl.city/news/view/tramvayu-mariupolya-ispolnilos-85-let-foto
%https://archive.org/details/01_05_2018.jana_ivanova.mrpl_city.tramvaju_mariupolja_ispolnilos_85_let

\href{https://archive.org/details/01_05_2018.jana_ivanova.mrpl_city.tramvaju_mariupolja_ispolnilos_85_let}{%
Трамваю Мариуполя исполнилось 85 лет, Яна Іванова, mrpl.city, 01.05.2018}

До начала Второй мировой войны в Мариуполе трамвайным сообщением были связаны
центр города с заводом им. Ильича, железнодорожным вокзалом, портом и Правым
берегом. В основном это двухколейные пути. Но были и одноколейные. Например, на
линии Центр - порт по улице Котовского (Первая Слободка) из-за недостаточной
ширины улицы для прокладки двухколейной линии. 8 октября 1941 года Мариуполь
был захвачен гитлеровцами. Через некоторое время по приказу оккупационных
властей трамвайное движение было восстановлено. В некоторых источниках
указывается, что в 1942 году немцы демонтировали рельсы одного из двух
параллельных трамвайных путей. У кого сейчас спросишь, как на самом деле было?
Мариуполь был освобожден от немецко-фашистских оккупантов 10 сентября 1943 г.
Более половины города было сожжено. Был разрушен трамвайный парк, и не сразу
трамваи пошли по городу. Например, от сквера на завод имени Ильича трамваи
поехали только 10 января 1945 года. Две знаменательные даты по теме. 5 октября
1952 года завершилось строительство трамвайных линий на Левый берег, в начале
ноября 1964 года был введен в строй путепровод через Кальмиус и железнодорожные
пути МПС. Стоит ли скрывать, что текст, представленный выше, подготовлен из
сведений, почерпнутых из интернета.

\ii{05_05_2018.stz.news.ua.mrpl_city.1.mariupolskij_tramvaj.pic.2}

А теперь осколки собственных воспоминаний о мариупольских трамваях. В 40 - 50
годы прошлого века в довоенных вагонах двери во время движения частенько не
закрывались. Скорости трамваев были небольшие. Поэтому иногда можно было
наблюдать, как какой-нибудь парень мчится за красно-желтым вагоном, настигает
его, хватается за поручни и вскакивает на подножку. На остановках пассажиры
входили в вагон через заднюю дверь. Вход через переднюю дверь был разрешен
только инвалидам, беременным женщинам, женщинам с младенцами на руках,
милиционерам... Кондуктор восседала у задней двери, вошедшие люди покупали билет
и продвигались по вагону в сторону водительского отсека. Кондуктора можно было
узнать по черной сумке особого покроя, ремень этой сумки был наброшен на шею, а
сама сумка висела у ее хозяйки на животе. В часы пик вагоны были облеплены
пассажирами со всех сторон, некоторые умудрялись залезть на крышу. Сиденья и их
спинки в трамваях были набраны из деревянных лакированных брусков.

Сценка из прошлого. На Слободке у третьего разъезда останавливается трамвай, он
ждет встречного. Народ вываливает из вагона. Коренастый здоровяк в модной по
тем временам пестрой рубашке купил у старушки семечки и тут же начинал смачно
их щелкать. Два парня, облаченные во флотскую форму, поспешили в находящийся
тут же магазин. Через пару-тройку минут они выходят из него умиротворенные. В
каждой руке у них по бутылке \enquote{Жигулевского}. Пожилая женщина рассказывает
что-то своей знакомой, то и дело поправляя сползающую с головы белую косынку.
Тут издалека забренчал колокольчик приближающегося трамвая. Люди заспешили в
вагон. А вот и встречный трамвай. Вагоновожатые совершили процедуру передачи
жезла. Звонок. И один трамвай, позванивая колокольчиком, поехал в Центр города,
а другой отправился в порт.
