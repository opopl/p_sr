% vim: keymap=russian-jcukenwin
%%beginhead 
 
%%file topics.mova_jazyk.cit
%%parent topics.mova_jazyk
 
%%url 
 
%%author_id 
%%date 
 
%%tags 
%%title 
 
%%endhead 

%%%cit
%%%cit_head
%%%cit_pic
%%%cit_text
Особливість українських реалій в тому, що саме МОВА тривалий час заступила
собою відсутність Української \emph{Держави}. Питання мови було питанням влади і
панування. Не комунікації. Саме звідси сотні указів заборон української мови з
боку \emph{держав} окупантів: Московії, Польщі, Румунії, Угорщини, що через фізичне і
моральне знищення українців намагалися панувати на споконвіку наших етнічних
землях.  В основі нашої розвідки – сучасний терор Москви проти української мови
– та її жертви і Герої водночас. Не було б тут російської мови як наслідку
історичної окупації України Московією – не було б сьогодні війни, бо Путін не
мав би кого захищати. Він каже про це сам: "Наголошу, що ми були змушені
захищати російськомовне населення на Донбасі, змушені були відреагувати на
прагнення людей, що живуть в Криму, повернутися до складу РФ" (Путін, 12 жовтня
2016) [3].  Тобто словосполука "російськомовне населення" – це "частина воєнної
доктрини Путіна" [1]. У нас НЕМА російськомовного населення. Є етнічні росіяни
– нащадки окупантів, сьогоднішні окупанти і з'яничарені українці – жертви
московського терору.  Що маємо сьогодні на окупованих теренах?
%%%cit_comment
%%%cit_title
\citTitle{МОВА СЬОГОДНІ – ЦІНА ЖИТТЯ}, 
Ірина Фаріон, blogs.pravda.com.ua, 09.11.2021
%%%endcit
