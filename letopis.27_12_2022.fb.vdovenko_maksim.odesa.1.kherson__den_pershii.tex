%%beginhead 
 
%%file 27_12_2022.fb.vdovenko_maksim.odesa.1.kherson__den_pershii
%%parent 27_12_2022
 
%%url https://www.facebook.com/snarkfog/posts/pfbid04V849kNLNDQ8NzfpqpE1uzcZoKtiNCY4wPe6D2Uz6ebKYMy9bPnCrN5KMufv2U9Dl
 
%%author_id vdovenko_maksim.odesa
%%date 27_12_2022
 
%%tags herson
%%title Херсон. День перший
 
%%endhead 

\subsection{Херсон. День перший}
\label{sec:27_12_2022.fb.vdovenko_maksim.odesa.1.kherson__den_pershii}

\Purl{https://www.facebook.com/snarkfog/posts/pfbid04V849kNLNDQ8NzfpqpE1uzcZoKtiNCY4wPe6D2Uz6ebKYMy9bPnCrN5KMufv2U9Dl}
\ifcmt
 author_begin
   author_id vdovenko_maksim.odesa
 author_end
\fi

Херсон. День перший.

Прямуємо у Херсон у складі команди волонтерів. Це мій перший подібний виїзд
фактично до зони бойових дій. Знання про потенційну небезпеку та можливі
труднощі суто теоретичні. Наші завдання (загалом) - роздача гуманітарки та
допомога в організації добровільної евакуації населення.

Їдемо до Антонівки біля того самого мосту. На лівому березі сидять орки та
ведуть постійні обстріли. Повз мосту треба пролітати дуже швидко і зайвий раз
не світитися.

Допомагаємо розвантажувати гуманітарку. Раптом приліт у приватний будинок.
Загинула собака, почалася пожежа. Люди не постраждали. Влучання чимось не
сильним. Можливо, з міномету. Допомагаємо заливати пожежу звичайною водою з під
крану, викликаємо пожежників, які так і не приїхали. Може вони приїхали
пізніше, ми не чекали. Локалізували вогонь, надихалися димом і поїхали по інших
справах. \enquote{Непоганий} початок.
