% vim: keymap=russian-jcukenwin
%%beginhead 
 
%%file 06_10_2021.fb.bilchenko_evgenia.1.poezia_cvetajevskoje
%%parent 06_10_2021
 
%%url https://www.facebook.com/yevzhik/posts/4317905534911190
 
%%author_id bilchenko_evgenia
%%date 
 
%%tags bilchenko_evgenia,cvetajeva_marina,poezia
%%title БЖ. Цветаевское
 
%%endhead 
 
\subsection{БЖ. Цветаевское}
\label{sec:06_10_2021.fb.bilchenko_evgenia.1.poezia_cvetajevskoje}
 
\Purl{https://www.facebook.com/yevzhik/posts/4317905534911190}
\ifcmt
 author_begin
   author_id bilchenko_evgenia
 author_end
\fi

\ifcmt
  ig https://scontent-mia3-1.xx.fbcdn.net/v/t39.30808-6/244612232_4317905664911177_695914850337225887_n.jpg?_nc_cat=108&_nc_rgb565=1&ccb=1-5&_nc_sid=8bfeb9&_nc_ohc=IDupFnDNoUIAX8mGpn3&_nc_ht=scontent-mia3-1.xx&oh=52e6e0c41249b94252cabb2cbe08e109&oe=6163EBC7
  @width 0.4
  %@wrap \parpic[r]
  @wrap \InsertBoxR{0}
\fi

БЖ. Цветаевское
Так змея меж двумя камнями сдирает кожу.
Так река, погибая в море, себя итожит.
Так последний глагол становится первым самым
Существительным из роддома и храма - "мама".
Так кончаются сигареты, обеты, связи.
Списки лиц в телефоне, крупы, плита на газе.
Недосказанных слов абсурдная несуразность:
В ровность сходит так никому-ничему неравность.
Так из окон, из драных рам, из гнилых прокладок
Городские шумы, похожие на оладок,
Тех, дедовских, шуршанье в дом проникают робко.
Так считаются дни, лекарства, кресты, коробки.
Так рождает стих - анапестом, жёлтым оком
Классицизма. И так анализ сдают на онко.
Так боишься соседа в скважине хрип замочной.
Так не холодно и не жарко от яств заморских.
Так берешь и по корню лезвием - режешь, режешь.
Корень сгнил, ибо дождь: упрям, кособок, безбрежен.
Так сырой коробок подбрасываешь с досады.
Ни полцарства и ни коня - ничего не надо.
Так поэзию в горле душишь, поскольку слово
Было только у Бога: дальше уже не ново.
Так, из гусениц метя в бабочки, умирает
То, что, ад искупив, - не в силах дойти до рая.
6 октября 2021 г.

\begin{itemize} % {
\iusr{Анастасия Рыжова}
Мне сначала показалось, что это фото пирога:))

\iusr{Гамельнский Крысолов}
Как рождаются стихи такие? Для меня загадка.
Не заказные, а, душевно резкие, с рифмою без оглядки.
Кто диктует их, шепчет тихо, вкладывая поэтам в уши?
Тот, ли, который, зарю поднимая, разжигает Души?
\end{itemize} % }
