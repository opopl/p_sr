% vim: keymap=russian-jcukenwin
%%beginhead 
 
%%file 17_02_2022.stz.news.ua.fraza.1.kak_zapadnyje_partnery_opuskajut_ukrainu.1.ekonomika_posledstvia
%%parent 17_02_2022.stz.news.ua.fraza.1.kak_zapadnyje_partnery_opuskajut_ukrainu
 
%%url 
 
%%author_id 
%%date 
 
%%tags 
%%title 
 
%%endhead 

\subsubsection{Экономические последствия информационного психоза}
\label{sec:17_02_2022.stz.news.ua.fraza.1.kak_zapadnyje_partnery_opuskajut_ukrainu.1.ekonomika_posledstvia}

Да только радоваться особо нечему. Раздутый Западом психоз вокруг «нападения»
Путина нанес сильнейший удар по экономике Украины, которая и без того на ладан
дышит. Несостоявшаяся война не нужна ни Европе, ни тем более Украине, ни даже
Путину. Она была нужна Америке и Британии для решения своих международных и
внутренних проблем, более того — в «несостоявшемся» виде эта война принесла им
определенную экономическую и политическую выгоду.

Эта фейковая война, точнее истерия вокруг нее, принесла значительные
экономические выгоды России в виде подскочившей выше 1000 долларов на спотовом
рынке цены на газ, что тянет вверх и цены по долгосрочным контрактам, а также в
виде возросшей почти до 100 долларов за баррель цены на нефть. В результате,
только по самым скромным подсчетам российская казна получит дополнительные
десятки миллиардов долларов. Так «борьба с Россией», устроенная
англо-американцами, обернулась для Путина неожиданным, но весьма ощутимым
дополнительным профитом.

От этой истерии проиграла Европа. Хотя бы по причине роста цен на углеводород.
Впрочем, это проблемы Европы, которая суетится между Америкой и Москвой.

Но самое главное, что больше всех, как всегда, пострадала Украина и ее
экономика. И здесь не праздновать надо мифическую победу в несостоявшейся войне
по команде Зе-власти, а эту самую Зе-власть вместе с папередниками посадить на
нары, а лучше на кол, за то, что они втянули страну в межимпериалистические
разборки.

Патриотическая клоунада и пиар на несостоявшейся победе в несостоявшейся войне,
устроенные Зеленским в виде очередного «дня єднання», призваны отвлечь внимание
от очевидного факта: экономика и каждый житель Украины понесли сильнейший урон,
последствия которого будут сказываться еще очень долго.

По некоторым оценкам, \textbf{истерия вокруг вторжения, запущенная нашими «друзьями» из
Вашингтона и Лондона обошлась украинской экономике в 13 млрд. долларов.}

С такими «друзьями» не нужен Путин с его агрессией...

В качестве компенсации за огромные экономические потери, Штаты предложили
Украине «целый» 1 миллиард долларов под гарантии США, на очень «своеобразных»
условиях, о чем далее.

