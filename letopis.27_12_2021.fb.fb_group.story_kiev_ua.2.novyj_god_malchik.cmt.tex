% vim: keymap=russian-jcukenwin
%%beginhead 
 
%%file 27_12_2021.fb.fb_group.story_kiev_ua.2.novyj_god_malchik.cmt
%%parent 27_12_2021.fb.fb_group.story_kiev_ua.2.novyj_god_malchik
 
%%url 
 
%%author_id 
%%date 
 
%%tags 
%%title 
 
%%endhead 
\zzSecCmt

\begin{itemize} % {
\iusr{Альбина Орел}
Помню, что мама сама мне шила пачку снежинки и белые тапочки к утреннику)))

\iusr{Наталия Луц}
\textbf{Альбина Орел} мне тоже

\iusr{Наталия Луц}
Мне тоже

\iusr{Lora Berdichevsky}
\textbf{Наталия Луц} и мне @igg{fbicon.face.grinning.smiling.eyes} 

\iusr{Катя Бекренева}
Маленький мальчик символизировал Новый год), к концу года он превращался в Деда Мороза)

\begin{itemize} % {
\iusr{Maksym Oleynikov}
\textbf{Катя Бекренева} 

Такого не пам'ятаю ні тоді, ні пізніше. Було таке, що Дід Мороз із Снігуркою
забавляють дітей, аж тут звляється хлопчик - Новий рік, і це символізувало
очікуваний прихід нового року, і Дід Мороз прощався.... Мабуть, різноманітні
були сценарії.

\end{itemize} % }

\iusr{Ирина Иванченко}
Приємні ваші спогади, до душі, дякую.

Ніколи не користувалася \enquote{казенними}, у дитинстві новорічні костюми шили
самостійно, завжди вигадували щось оригінальне, не як у всіх.


\iusr{Татьяна Кривицкая}

Гарний петрушка. Трохи пізніше Схоже і в мене було. Я була сніжинкою, а
\enquote{чешки} мамуля пошила за ніч, з білоі тканини, плаття прикрасили з мішури
сніжинками. Я була найщасливіша і найкраща сніжинка! А, саме головне, в наступному
році був Мир і молоді батьки.

\iusr{Анна Доманская}

Новорічні спогади з дитинства самі теплі і чудові. Саморобні новорічні костюми і
прикраси на ялинку, все було казкове в дитинстві.

\iusr{Elena Marijchuk}

А мне мама тапочки - балетки ВЯЗАЛА КРЮЧКОМ. Получались САМЫЕ КРАСИВЫЕ.

\iusr{Inna Radzinsky}

Дякую за приємні спогади. Мені того ж 1959 року в дитячому садку на вулиці Анрі
Барбюса довелося бути Снігуронькою. Незважаючи на те, що я була бодай
найменшого зросту в групі. Мене вповноважили на цю посаду тому що я була здатна
запам'ятати і проголосити досить довгу новорічну промову. Біло-блакитний
кожушок і такий самий капелюшок були оздоблені ватою з безліччю блискучих
плямок з потовчених кольорових кульок. Десять хвилин акторської слави в
невимовно красивому одязі - один з найяскравіших спогадів дитинства.

\begin{itemize} % {
\iusr{Maksym Oleynikov}
\textbf{Inna Radzinsky} 

Інночко, ти й у дорослому віці була аж ніяк не дилдою....
@igg{fbicon.face.smiling.eyes.smiling} 

\iusr{Наталя Будзинська}
А мене ось так відягнули в садку в 1958

\ifcmt
  ig https://scontent-frt3-1.xx.fbcdn.net/v/t39.30808-6/270317983_4523673071075751_8336366262052736821_n.jpg?_nc_cat=104&ccb=1-5&_nc_sid=dbeb18&_nc_ohc=eiBh2RBsqqwAX-QsWH-&_nc_ht=scontent-frt3-1.xx&oh=00_AT8C6q4AXmaypwDKHyRWpP1U5mBppYUcSe3mWNmtBnpjYQ&oe=61D3C636
  @width 0.3
\fi

\begin{itemize} % {
\iusr{Maksym Oleynikov}
\textbf{Наталя Будзинська} О!!! Звідки така розкіш? З якої скрині?

\iusr{Наталя Будзинська}
\textbf{Maksym Oleynikov} 

Садок був відомчий, від заводу - ДВРЗ. Нас вдягали й годували дуже добре, я
цього не пам'ятаю, бо мені на цьому фото ще не виповнилося 3-років. Мама
розповідала, завод виготовляв і ремонтував вагони з усієї України, і , мабуть,
Союзу. До речі, кілька років тому зробили фільм про дільничного з ДВРЗ,
прототипом якого був справжній дільничий, гроза всіх хуліганів на ДВРЗ -
прізвище Маценко. От з його дочкою я була в одній групі в цьому самому садку


\iusr{Olena Korenets Nebuchadnezzar}
\textbf{Nataly Budzynska} Яка ж гарнесенька!

\iusr{Inna Radzinsky}
\textbf{Наталя Будзинська} \textbf{Maksym Oleynikov} 

Цікаво - мій садок теж був відомчий, від Київського хлібозаводу. Теж дуже добре
годували, від чого я плакала бо не любила їсти (тоді) @igg{fbicon.face.wink.tongue} 

\end{itemize} % }

\end{itemize} % }

\iusr{Alexey Novozhylov}

Колись в дитинстві пам'ятаю як дуже злякався якогось родича, перевдягнутого в
дєда мороза, здається то був 67 або 68 рік. Подумалось чомусь, як українці
Канади або Америки тоді Різдво святкували, колядки співали, Св. Миколай, Санта
Клаус, цікаво мабуть...  @igg{fbicon.face.smiling.eyes.smiling}{repeat=2} 

\iusr{Тамара Мельничук}
Але все було стабільно при дефіцитах всього

\iusr{Єлизавета Бойко}
\textbf{Tamara Melnichuk} Дійсно,що стабільно не було нічого.

\iusr{Владимир Мазур}

В послевоенные годы такие костюмы петрушки шили из марли, колпаки делали из
ватмана и украшали обвёртками от конфет. Все были счастливы

\iusr{Тамара Ар}
Сейчас очень похожие детские костюмы есть в детском мире, но по приличной цене

\iusr{Аліса Забой}

1956 рік - перший в історії СРСР міжнародний фестиваль молоді та студентів!
Небачене і неймовірне для післясталінської доби...

\begin{itemize} % {
\iusr{Анна Дубницкая}
\textbf{Alisa Zaboy} 1957 рік.

\iusr{Аліса Забой}
\textbf{Анна Дубницкая} дякую, але в пам'яті відбився 1956 й... Може двадцятий з'їзд?

\iusr{Аліса Забой}
\textbf{Анна Дубницкая} але, враження і відчуття від того не забути...

\iusr{Анна Дубницкая}
\textbf{Alisa Zaboy} Так, ХХ з'їзд КПРС.

\iusr{Аліса Забой}
\textbf{Анна Дубницкая} Вже перевірила... Для дітей-школярів - фестиваль, для батьків - з'їзд...
\end{itemize} % }

\iusr{Людмила Терещенко}
Все девочки снежинки, мальчики зайцы)))

\iusr{Natasha Kovalchuk}

Новогодний костюм @igg{fbicon.snowflake} из марли, до сих пор хранится у Мамы! Вот уже 44
года @igg{fbicon.man.facepalming}  Помню, как мама его шила, а Папа вырезал самые красивые
снежинки.

\ifcmt
  ig https://scontent-frx5-1.xx.fbcdn.net/v/t39.30808-6/270204999_10220224308108390_3115706692024135972_n.jpg?_nc_cat=100&ccb=1-5&_nc_sid=dbeb18&_nc_ohc=_YQdK8odr1YAX961-kQ&_nc_ht=scontent-frx5-1.xx&oh=00_AT9EVIw2rXSC45hE_YTjJbvVc3f9JnzpraX2iHPLrp-xwQ&oe=61D3DD1B
  @width 0.3
\fi

\begin{itemize} % {
\iusr{Наталья Чекотун}
\textbf{Natasha Kovalchuk} А у меня был даже не из марли,а из широкого бинта. Юбка в несколько ярусов. Накрохмалено,канечно
\end{itemize} % }

\iusr{Нина Бондаренко}

От і я пам'ятаю ті білі тапки, що мама шила для щороку під Новий рік. А ще
пам'ятаю, віршик:

Открываем календарь -

Начинается - январь...

А я з натхненням кричала: \enquote{...Комендант... Комендант..}, тому що Дідом Морозом
одягли коменданта гуртожитку. Садочки були відомчі, всі жили у відомчих (
заводських) будинках, всі знали один одного...


\iusr{Наталья Борчанинова}

Ось що є в моєму архіві))) Це Запрошення моє і моєі сестри ( м. Біла Церква)

Зустріч НР 1968!!!

\ifcmt
  ig https://scontent-frx5-1.xx.fbcdn.net/v/t39.30808-6/269977182_1549087795458726_7137437500624395470_n.jpg?_nc_cat=105&ccb=1-5&_nc_sid=dbeb18&_nc_ohc=7IYsy7t_5XoAX8ZE2On&_nc_ht=scontent-frx5-1.xx&oh=00_AT-oRq2zJ0ed1v_R5CH_LwE1_c4rUHLwL07rXVdkAbuI9g&oe=61D530E2
  @width 0.4
\fi

\iusr{Валентина Мірошниченко}
НЕ завжди хлопчик. У 1964 році в Жовтневому палаці НР була я.  @igg{fbicon.laugh.rolling.floor} 

\begin{itemize} % {
\iusr{Maksym Oleynikov}
\textbf{Валентина Мірошниченко} 

Чудово! Але на теренах СРСР від 50-х і фактично до 80-х це був хлопчик, бо рік
- чоловічого роду, тож на сцені могла бути і дівчинка, але вдягнена саме так,
як хлопчик-Новий рік... @igg{fbicon.face.savoring.food} 


\iusr{Валентина Мірошниченко}
\textbf{Maksym Oleynikov} Так.
\end{itemize} % }

\iusr{Larisa Dmitrakova}
Нищета..

\begin{itemize} % {
\iusr{Maksym Oleynikov}
\textbf{Larisa Dmitrakova} Добре, що Ви це розумієте. Бо у нас тьма-тьмуща тих, хто сприймає ті часи як щось незрівнянно краще, аніж сьогодення...

\iusr{Mike Kaufman-Portnikov}
\textbf{Maksym Oleynikov} пропоную вважати це психіатричним захворюванням. Це просто жах.

\iusr{Раиса Сухарева}
\textbf{Maksym Oleynikov} ото ж тьма -тьмуща.

\iusr{Юлия Александровна}
\textbf{Maksym Oleynikov} 

для многих - это всеря детской и юношеской беззаботности, где небыло проблем,
родители живы, праздники всем миром праздновались небыло по телику злобы, а в
людях зависти, потому что все были бедны)

Вспоминают счастливые для себя времена, вспоминибт счастливых себя)

\end{itemize} % }

\iusr{Larisa Dmitrakova}

\ifcmt
  ig https://scontent-frt3-2.xx.fbcdn.net/v/t39.30808-6/270102197_1021946631723143_8662830794233552072_n.jpg?_nc_cat=103&ccb=1-5&_nc_sid=dbeb18&_nc_ohc=SEy_o48Nb3oAX9VHH2u&_nc_ht=scontent-frt3-2.xx&oh=00_AT_ouPq0FW4A-SGHhXbf5JTSTvO-C1OiLMOt9g9Vv4EjGA&oe=61D487D1
  @width 0.4
\fi

\iusr{Ніна Оксьон}
\textbf{Larisa Dmitrakova} ... А пропаганда яка! З дитинства...

\iusr{Нина Алексеева}
\textbf{Larisa Dmitrakova} Це ви осуджуєте чи заздрите? Не робіть висновки, все попереду...

\iusr{Лариса Осіпенко}
\textbf{Larisa Dmitrakova} Це ви про себе! Злидні душі?! Я так вас зрозуміла!!!

\iusr{Наталия Ковалева}
Такой хорошенький!!!

\iusr{Наталия Ковалева}

\ifcmt
  ig https://i2.paste.pics/ee7f0f509e3494299d9644d4fe4b489a.png
  @width 0.2
\fi

\iusr{Валентина Дуб}

В 60 роки мені мама обшивала звичайні тапочки білою тканиною для костюма
сніжинка... як я люблю спогади про дитинство...

\iusr{Татьяна Жалнина}

Дети были счастливы, родители работали, у всех была уверенность в будущем,
образование, медицина бесплатно, квартиры получали от государственной, в Крым
ездили летом, войны не было, что кому то не нравится? Сейчас тряпок полно, а
нет веселья и уверенности, да и государства нет

\begin{itemize} % {
\iusr{Михайло Наместник}
\textbf{Татьяна Жалнина} ай, бедная, бедная Пеппилотта! Когда был Дед Мороз все было совсем по-другому

\iusr{Оксана Макаренко}
\textbf{Tatyana Zhalnina} Здуріти можна. Мені вас жаль, жіночко.

\iusr{Люся Киевская}
\textbf{Татьяна Жалнина} старые песни о главном ! ...

\iusr{Елена Кирина}
\textbf{Татьяна Жалнина} 

Вы медицину бесплатную какую видели?! Войны это когда не было? Не было
советских солдат в Корее Вьетнаме Египте Алжире Афгане? не былот грузов 200? не
было всеобщего дефицита унижения человека и очередей? Вы где жили и когда?

\begin{itemize} % {
\iusr{Ростислав Чентемиров}
\textbf{Елена Кирина}, 

возможно, дама работала в торговле. Или администратором в гостинице, или в
каком-нибудь райкоме-горкоме.


\iusr{Елена Кирина}
\textbf{Ростислав Чентемиров} возможно.

\iusr{Татьяна Жалнина}
\textbf{Ростислав Чентемиров} 

кстати, почему-то, в почете были рабочие профессии, шахтёры, строители и т. д. у
них были самые высокие зарплаты, сейчас это заметно по пенсиям, самые высокие
пенсии у шахтёров

\iusr{Татьяна Жалнина}
\textbf{Ростислав Чентемиров} 

шахтёры были в почёте, самая большая зарплата, сейчас хорошие пенсии, но
конечно, сейчас, самые высокие зарплаты, пенсии и все льготы не у рабочих, у
депутатов ВР, барыги, которые налоги не платят преуспевают

\iusr{Евгений Бабенко}
\textbf{Tatyana Zhalnina} від розміру суддівської пенсії вам стане зле

\iusr{Наталия Лузан-Исупова}
\textbf{Татьяна Жалнина} Вам бы в те годы в селе родится и жить, потом бы про зарплаты и пенсии вспоминать тут.

\iusr{Игорь Бабич}
\textbf{Евгений Бабенко} а у них не пенсии..
Вы посмотрите правильное название..  @igg{fbicon.cry} 
\end{itemize} % }

\iusr{Gary Sorokin}
\textbf{Tatyana Zhalnina} а у меня никогда ничего этого не было ,

\iusr{Елена Грицай}
\textbf{Tatyana Zhalnina} 

нищета, многоразовые шприцы в вашей медицине, Крым я увидела в первый раз,
когда сама зарабатывать стала, чтобы получить квартиру, надо было нехило
извернуться с выпиской и пропиской всех родственников до 7 колена. Война была.
Афганистан тот же. Или думаете, что раз не на нашей территории, то и наши не
воюют?

Слава богам, мое советское детство осталось позади

\iusr{Gary Sorokin}
\textbf{Olena Gritsay} как насчёт одноразовых перчаток, которые стирали

\iusr{Эмилия Супрун}
\textbf{Татьяна Жалнина} 

все-все квартиры получали? И коммуналок не было? Войны не было, а афганцы - это
ряженые? Медицина бесплатная была, да? Вот прям без взяток и операции делали, и
в санатории путевки получали?

Вы или в маразме, или тролль, или родились после 90х в семье номенклатуры,
засравшей ваш мозг сказками.


\iusr{Ростислав Чентемиров}
\textbf{Татьяна Жалнина}, 

не знаю, как в Украине, а в России самые высокие пенсии у госслужащих любого
уровня - 50-75\% от среднего заработка. А госслужащих - это не только депутаты,
это миллионы чиновников и даже клерков разных ведомств.

\begin{itemize} % {
\iusr{Татьяна Жалнина}
\textbf{Ростислав Чентемиров} к сожалению, кто не работает, тот ест ,рабочие специальности ,сейчас не в почете

\iusr{Svetlana Kozlova}
\textbf{Ростислав Чентемиров} у нас гос пенсии сильно срезали

\iusr{Ростислав Чентемиров}
\textbf{Svetlana Kozlova}, вы не путаете с пенсиями госслужащих? Госпенсии и пенсии госслужащих - это две огромные разницы.

\iusr{Svetlana Kozlova}
\textbf{Ростислав Чентемиров} я о госслужащих.
\end{itemize} % }

\iusr{Людмила Таркан}
\textbf{Татьяна Жалнина} 

нічого безкоштовного не було! За медицину та освіту народ розраховувався своєю
мізерною зарплатою. Люди десятками років чекали свою чергу на отримання житла,
машини і т.п. В санаторії і дома відпочинку їздили одиниці і по «великому
блату». За кордон взагалі не випускали!

\begin{itemize} % {
\iusr{Gary Sorokin}
\textbf{Людмила Таркан} я так и не дождался, тесть на очереди стоял
\end{itemize} % }

\iusr{Наталия Добик}
\textbf{Tatyana Zhalnina} 

квартити отримували???? Моя мама поки не вИходила, можна сказать не вибила
квартиру, ніхто і не подумав за багатодітну сім'ю

\iusr{Лариса Соколова}
\textbf{Наталия Добик} 

А Вот моя мама получила на троих детей квартиру на Чоколовке. А потом через
много лет детям сиротам дали небольшую квартиру


\iusr{Ирина Собко}
\textbf{Tatyana Zhalnina} 

Помню я эти прекрасные времена. Бить жену и детей-это было нормой. Если жена с
мужем разводилась, то все все равно жили в одной квартире, потому что продать
нельзя, купить нельзя, на очереди стой лет 10-15 и все это время муж продолжал
молотить и жену и детей. Чудесные времена. Все свободное время в очередях,
среди постоянного хамства.

\begin{itemize} % {
\iusr{Татьяна Петрачек}
\textbf{Ирина Собко} а сейчас, что мужья не бьют жён и всё остальное что вы перечислили

\iusr{Татьяна Жалнина}
\textbf{Ирина Собко} 

сейчас мужья часто альфонсы, очереди в поликлиниках, например, сейчас также
есть, в магазинах всего много, но качества нет, поэтому дети алергики


\iusr{Ирина Собко}
\textbf{Tatyana Zhalnina} 

а раньше ж не альфонсы были))

Не смешите меня. И раньше аллергиков было куча, просто меньше на это внимания
обращали.

\iusr{Ирина Собко}
\textbf{Татьяна Петрачек} Сейчас в разы меньше. Да и у женщин больше возможностей развестись и жить отдельно.
\end{itemize} % }

\iusr{Тетяна Ліходєдова}
\textbf{Татьяна Жалнина} БОТ, ЗГИНЬ!(

\iusr{Tanya Gabzovska}
\textbf{Tatyana} одна мандарина на рік - прекрасний подарунок, головне стабільність і впевненість

\begin{itemize} % {
\iusr{Светлана Янковая}
\textbf{Tanya Gabzovska}
Ось і я пам'ятаю, що мандаринку бачила тільки в новорічному подарунку( які щасливі спогади)

\iusr{Tanya Gabzovska}
\textbf{Светлана} я ше пам’ятаю сухі банани і зелені... і вже в дорослому віці дізналась їх справжній смак  @igg{fbicon.face.confused} 
\end{itemize} % }

\end{itemize} % }

\iusr{Властелин Кота Мяу}
Не долго хватило козлов на мир и счастье

\begin{itemize} % {
\iusr{Сергей Оборин}
\textbf{Властелин Кота Мяу} А ты там жил, в этом мире? Если нет, то и молчи себе в тряпочку.

\iusr{Властелин Кота Мяу}
\textbf{Сергей Оборин} нет, блять. Только ты жил, дурень старый

\iusr{Андрей Воробей}
\textbf{Властелин Кота Мяу} Зоофил. Не Вам судить. Только помечтать. Да и то фантазии не хватает.
\end{itemize} % }

\iusr{София Печенова}

\ifcmt
  ig https://scontent-frx5-2.xx.fbcdn.net/v/t39.1997-6/s168x128/70051830_2383482495080781_6192462346766516224_n.png?_nc_cat=1&ccb=1-5&_nc_sid=ac3552&_nc_ohc=lHvxZEoj-jQAX_79Yks&_nc_ht=scontent-frx5-2.xx&oh=00_AT9mes8MX8h0IX3Z0i6a1jT0uAeFQCKE2QM6KUaKLIw1fQ&oe=61D534E7
  @width 0.1
\fi

\iusr{Лидия Васильевна}

Есть много хорошего и полезного и в прошлые годы советского союза и в нынешней
современной жизни и есть люди хаявшие или прошлую или нынешнюю жизнь..., НО ОТ
БРЮЗЖАНИЯ НИ ДО НИ СЕЙЧАС ЛУЧШЕ НЕ ЖИВУТ.... НЕ УМЕЮТ РАДОВАТЬСЯ ЖИЗНИ!!!

\iusr{Олена Медведева- Прицкер}

С Новым годом уважаемый Максим ! Невероятно, но у меня в детстве была такая
открытка: мальчик - 1956! Когда-то была традиция посылать новогодние открытки
по почте. Заранее красивые и разные открытки покупались в киосках союзпечати,
писали поздравления и отправляли родственникам и друзьям.

\iusr{Maksym Oleynikov}
\textbf{Олена Медведева- Прицкер}

\ifcmt
  ig https://scontent-frx5-1.xx.fbcdn.net/v/t39.30808-6/270937226_979621815960854_4266684486591070105_n.jpg?_nc_cat=111&ccb=1-5&_nc_sid=dbeb18&_nc_ohc=k4J2Q_z9Pk0AX8BVDl3&_nc_ht=scontent-frx5-1.xx&oh=00_AT_D-YQvaXzs25zJGWd16_Pbt69abnXMr6npjA0vkJw25g&oe=61D3837E
  @width 0.3
\fi


\end{itemize} % }
