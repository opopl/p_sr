% vim: keymap=russian-jcukenwin
%%beginhead 
 
%%file 05_01_2022.yz.zapiski_komivojadzhera.1.russkie_vechnaja_mobilizacia
%%parent 05_01_2022
 
%%url https://zen.yandex.ru/media/id/5e26fee55d636200acbd2f4f/uchenye-russkie-jivut-v-situacii-vechnoi-mobilizacii-6171820053209c033c5bd9f3
 
%%author_id yz.zapiski_komivojadzhera
%%date 
 
%%tags istoria,mentalitet,rossia,russkie
%%title Ученые: русские живут в ситуации вечной мобилизации
 
%%endhead 
 
\subsection{Ученые: русские живут в ситуации вечной мобилизации}
\label{sec:05_01_2022.yz.zapiski_komivojadzhera.1.russkie_vechnaja_mobilizacia}
 
\Purl{https://zen.yandex.ru/media/id/5e26fee55d636200acbd2f4f/uchenye-russkie-jivut-v-situacii-vechnoi-mobilizacii-6171820053209c033c5bd9f3}
\ifcmt
 author_begin
   author_id yz.zapiski_komivojadzhera
 author_end
\fi

Почему в России не могло сложиться демократическое государство? Российская
государственность формируется тогда, когда особенно резко проявляет себя Малый
ледниковый период – это время глобального похолодания XIV – XIX веков (наиболее
холодный период по средним годовым температурам за прошедшие 2 тысячи лет. То
есть урожаи всё меньше по климатическим причинам.

\ii{05_01_2022.yz.zapiski_komivojadzhera.1.russkie_vechnaja_mobilizacia.pic.1}

Но это ещё и время, когда в зоне рискованного земледелия (а вся территория
северо-западной Руси, где и разворачивается борьба за государственность)
хозяйственная деятельность подвергается могучему воздействию Степи. То есть
складываются 3 отрицательных фактора, гасящие любую попытку демократизации:
климат, природа, враждебное окружение.

Можно долго доказывать, пришло ли к нам ордынское иго или возник особый
симбиоз, но вот цифры возникают страшные, считает А.А. Горский: «По
археологическим данным, из 74 русских городов XII-XIII вв., известных по
раскопкам, 49 (2/3!) были разорены Батыем, из них 14 городов (т.е. около 20\% от
их общего числа) уже не поднялись из пепла, а 15 (еще 20\%) постепенно
превратились в села».

По подсчетам А.А. Горского, в середине – второй половине XIII в. прекратили
свое существование 83\% укрепленных поселений Киевской земли (из них позже
возродились лишь 22\%), 81\% – Переяславской земли (возродились 14\%), 76\% –
Галицко-Волынских (возродились 31\%). Сведения взяты из Горский А.А. Русь: от
славянского расселения до Московского царства. М.: Языки славянской культуры,
2004. С. 202.

Да, Западная Европа тоже знала опустошительные войны, например, печально
известные Столетняя и Тридцатилетняя. Но! Любой набег степняков решительно
отличается от феодальных междоусобиц. Король, барон, граф, нападая, не
собирался уничтожать население – история про курицу и её золотые яйца понятна.
Никто не объявлял твой город «злым», не собирался вырезать всех, кто «выше
колеса арбы». Главное было собрать контрибуцию.

А степняки разоряли местность дотла, уничтожали всех, до кого дотягивалась
сабля, уцелевших уводили. Им здесь не жить, не их ландшафт, не их система
хозяйствования.

Такое смертельное соседство создавало особую жизнь – под постоянной саблей.
Выдержать это могло только централизованное государство, живущее в состоянии
постоянной мобилизационной готовности, чтобы дать отпор налётчикам, и это
продолжалось долго, до 1685 г. Московское государство платило регулярную дань
Крыму, а последний набег крымских татар на Украину был в в 1769 г., только
Екатерина ликвидировала это бандитское гнездо.

Ни одно демократическое государство не выдержит подобного сверхнапряжения сил –
России на роду написано было жить, регулярно отбиваясь и от Востока, и от
Запада, опираясь на крепкую самодержавную власть.

Так может быть, наш путь намечен историей?
