% vim: keymap=russian-jcukenwin
%%beginhead 
 
%%file 04_12_2021.fb.krjukova_svetlana.1.avtor_poisk.cmt
%%parent 04_12_2021.fb.krjukova_svetlana.1.avtor_poisk
 
%%url 
 
%%author_id 
%%date 
 
%%tags 
%%title 
 
%%endhead 
\subsubsection{Коментарі}

\begin{itemize} % {
\iusr{Svetlana Kryukova}
Этот пост нужно было написать ради комментов)

\begin{itemize} % {
\iusr{Александр Шибанов}
\textbf{Svetlana Kryukova} Я думаю и в личке весело @igg{fbicon.face.tears.of.joy} 

\iusr{Владимир Гугняк}
Но отчислять стране обязаны - доходы первого звена.  @igg{fbicon.face.tongue} 

\iusr{Serhiy Motorny}
Власть превосходит всех в сатире. Справляется сама!
Но отчислять стране - обязаны,-
за суперстимуляцию и супермотивацию.

\iusr{Serhiy Motorny}
Спокоен и настойчив.

\iusr{Алена Вербицкая}
\textbf{Svetlana Kryukova} у вас склонность к мазохизму @igg{fbicon.wink} 

\iusr{Oleh Butenkov}
какая-то тоска по петросяну просквозила в каменте

\iusr{Serhiy Motorny}
Спасибо, Мастер.

\iusr{Ирина Малина}
\textbf{Svetlana Kryukova} точно! гормональный всплеск на дофаминово- сератониновой странице ! ;)) @igg{fbicon.hand.ok}  @igg{fbicon.hands.applause.yellow} 
\end{itemize} % }

\iusr{Людмила Чикалова}

Однако! А женщины интересного возраста опять на обочине???! Дискриминация по
Зе. \enquote{Вы нам очень дороги, но посидите дома в намордниках.}

\begin{itemize} % {
\iusr{Svetlana Kryukova}
\textbf{Людмила Чикалова} про возраст это вы сказалии

\iusr{Людмила Чикалова}
\textbf{Svetlana Kryukova} ну, так я же все равно не парень!
\end{itemize} % }

\iusr{Станислав Шкурат}
Жаль. Я хотел порекомендовать Марьяну Безуглую. @igg{fbicon.face.relieved} 

\begin{itemize} % {
\iusr{София Гревцова}
\textbf{Станислав Шкурат}  @igg{fbicon.face.happy.two.hands}  @igg{fbicon.face.tears.of.joy}{repeat=3} 


\iusr{Svetlana Kryukova}
\textbf{Станислав Шкурат}  @igg{fbicon.face.tears.of.joy} 

\iusr{Дмитрий Гунов}
\textbf{Станислав Шкурат} цэ маніпуляція  @igg{fbicon.cat.tears.of.joy}{repeat=3} 


\iusr{Максим Чумаченко}
\textbf{Станислав Шкурат} марьяне писать не нужно. она уникум... ее просто к микрофону пустить и все.. аншлаг

\iusr{Valeri Valeri}
\textbf{Станислав Шкурат} точно!
\end{itemize} % }

\iusr{Макс Бужанский}

Я бы рекомендовал Данилова.

\begin{itemize} % {
\iusr{Денис Гороховский}
\textbf{Макс Бужанский} а он писать умеет?

\iusr{Макс Бужанский}
\textbf{Денис Гороховский}, нет, но привит.

\iusr{Денис Гороховский}
\textbf{Макс Бужанский} это несомненно плюс.

\iusr{Макс Бужанский}
\textbf{Денис Гороховский} , можно перекрасить, если есть претензии к цветовой гамме.

\iusr{Денис Гороховский}
\textbf{Макс Бужанский} я б все-таки ему бирку на ухо повесил, чтоб заткнуть оппонентов. Ну колокольчик на шею, но это уже опционально

\iusr{Макс Бужанский}
\textbf{Денис Гороховский} , нет, эта модель не апгрейдится.
Как кнопочный телефон, звонить можно, а скачать что то- нет.

\iusr{Svetlana Kryukova}
\textbf{Макс Бужанский}  @igg{fbicon.face.tears.of.joy} 

\iusr{Макс Бужанский}
\textbf{Svetlana Kryukova} , конкретней, берете, или на тиндер выставлять?)

\iusr{Леша Сокол}
\textbf{Макс Бужанский} 

а кАк могли господин такие дАнЫловы оказаться потрошенообразные и многие прочие
ему подобные у вас ???, конфликт на Востоке страны продолжаться потрошенковский
, бегать тут паскудам всяким и зиговать Ветеранам Великой Отечественной в лицо,
каналы закрывать шо вас же поддерживали бо вы гнили потрошенковской
уничтоживший своим переворотом некогда мирную и экономически благополучную при
Янеке страну - вироком стать пообещали а сами все их ппеступления страшные
продолжаете в их числе и ценовой каждодневный беспредел и тарифный гЭноцыд а
сейчас еще и шмурдяковый !!!тАк кто же вы после этого ???- пусть каждый сам
даст вам название ...

\iusr{Vitalii Yuldashevich}
\textbf{Макс Бужанский} лучше Тищенко Мыкола  @igg{fbicon.laugh.rolling.floor} 

\iusr{Serg Kikabidze}
\textbf{Макс Бужанский} мы все это заметили)))))

\iusr{Ян Валетов}
\textbf{Макс Бужанский} это будет сатира с ветеринарным оттенком. Я думаю, басни о животных.

\end{itemize} % }

\iusr{Marianna Morgenstern}

Вспомнилось...

"Чересчур много юмора. Понятно, что юмор, ирония помогают изъясняться
доходчиво. Возможно, когда надо избавиться от надоевших фигур, юмор полезен, но
если мы не научимся говорить серьезно, то так и останемся лакеями, которые,
стоя в передней, обсуждают господ. Или детьми — дети считают юмор, особенно
рифмованный, высшим проявлением ума. А нам надо обрести смысл общественной
жизни, так что ирония, анекдот, смех, да еще с матерком — неправильный путь".

\begin{itemize} % {
\iusr{Игорь Гайдученко}
\textbf{Marianna Morgenstern} Как-то так...  @igg{fbicon.face.smirking} 

\ifcmt
  ig https://scontent-frt3-2.xx.fbcdn.net/v/t39.30808-6/262195194_4821415031242131_6309436122559376162_n.jpg?_nc_cat=103&ccb=1-5&_nc_sid=dbeb18&_nc_ohc=h-acIJagJY0AX-UE2Zc&_nc_oc=AQkfwGzOl2g6REtHekOlc9Xk_WXJnsMXj7I5vtjq5guoryFl1cqnHSlQTx1vYrkzDLs&_nc_ht=scontent-frt3-2.xx&oh=8febb18c432d45eed601d5ea4af0b57e&oe=61B2AB10
  @width 0.4
\fi

\iusr{Marianna Morgenstern}
\textbf{Игорь Гайдученко} 

Читайте книги, господин Гайдученко, а не наклейки на пивных бутылках.

А думающим людям напомню. Рудольф Герцог, в своем исследовании о юморе в
Третьем Рейхе, дает понять, что авторитарные режимы свергает не смех, а гнев.
Политические анекдоты во многом – проявление апатии, даже покорности:
смеющемуся человеку трудно разозлиться. И как может засвидетельствовать любой,
кто встречался с настоящими политическими борцами, эти люди, как правило,
начисто лишены чувства юмора.

Поэтому анекдоты, которые простые немцы рассказывали о своих правителях,
парадоксальным образом играли на руку режиму. Политические шутки народа порой
не только не подрывают правящий режим, но и способствуют его укреплению.

\end{itemize} % }



\end{itemize} % }
