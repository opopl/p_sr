% vim: keymap=russian-jcukenwin
%%beginhead 
 
%%file slova.lugansk
%%parent slova
 
%%url 
 
%%author 
%%author_id 
%%author_url 
 
%%tags 
%%title 
 
%%endhead 
\chapter{Луганск}

– Ну и вопросы у Вас... Тут ведь как – когда тебе рассказывают, что ты дурак,
ты сам начинаешь в это верить. Пропаганда работает на все 100 процентов. Тут
ведь как считают? Во всем виноват Майдан. В \emph{Луганск} перся «Правый сектор», в
Станице прямо эшелоны останавливали. Но Вы же понимаете, все с телевизора...
Украина говорит, что Россия виновата. Россия – наоборот. Гонят друг друга и
результата нет. У меня большинство друзей уехали... Я этого не понимаю. Я живу в
своем доме – это главное. Да, теперь в кошельке русские деньги. Но как
говорится ведь, деньги не пахнут. Лишь бы они были,
\textbf{«Никто нам не вернет украденные годы»: как в \emph{Луганске} вспоминают захват города},
Донбас.Реалії, Записки з окупації, radiosvoboda.org, 02.06.2021

А я вирішив іти на війну, коли подивився сюжет, який мене вразив. Він був про
те, як у \emph{Луганській} області місцеві жителі пікетують якусь військову частину. І
показують срочника в радянській касці і в якомусь засраному броніку, з
автоматом. А потім його якийсь алкоголік витягує з воріт і здирає з нього каску
зі словами: \enquote{Снимай это говно}. Я подумав тоді, що, бля, ми б на Майдані за
каску, бронік, автомат вбили б, бо нічого з цього у нас не було. А тут якийсь
лиган, якого можна просто лящами прогнати, у пацана все забирає. Тоді я
зрозумів, що, мабуть, Збройним силам треба трохи допомогти, 
\textbf{Доброволець Олександр Воробєй: \enquote{Війна не змінює людину, вона
просто розкриває її сутність. Якщо людина була гандоном, то після фронту вона
стане махровим п\#дарасом}},
Віка Ясинська, censor.net.ua, 02.06.2021



