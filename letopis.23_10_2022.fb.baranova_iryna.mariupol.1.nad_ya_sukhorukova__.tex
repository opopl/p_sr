%%beginhead 
 
%%file 23_10_2022.fb.baranova_iryna.mariupol.1.nad_ya_sukhorukova__
%%parent 23_10_2022
 
%%url https://www.facebook.com/irina.baranova.142687/posts/pfbid0GeWqUNEkwdt3yHcskkq5YB6neordy5UC7CddoFPxFfvfhrGU8kAy4VjjcNqjFNral
 
%%author_id baranova_iryna.mariupol,suhorukova_nadia.mariupol
%%date 23_10_2022
 
%%tags mariupol
%%title Надія Сухорукова - Небо становилось близким
 
%%endhead 

\subsection{Надія Сухорукова - Небо становилось близким}
\label{sec:23_10_2022.fb.baranova_iryna.mariupol.1.nad_ya_sukhorukova__}
 
\Purl{https://www.facebook.com/irina.baranova.142687/posts/pfbid0GeWqUNEkwdt3yHcskkq5YB6neordy5UC7CddoFPxFfvfhrGU8kAy4VjjcNqjFNral}
\ifcmt
 author_begin
   author_id baranova_iryna.mariupol,suhorukova_nadia.mariupol
 author_end
\fi

Надія Сухорукова  \#Мариуполь \#Надежда

Небо становилось близким. 

Оно опускалось на землю,  как лифт на первый этаж. 

Ложилось прямо на дома и застывало.  

Пыталось закрыть Мариуполь собой. 

Нам казалось, Бог должен слышать. 

Нашу  отчаянную молитву. 

Кто-то  произносил эти слова обречённо:  полушепотом, полустоном. 

Кто-то  с надеждой.  

Подняв  глаза к серым подвальным трубам. 

Моя  семилетняя   племянница Варя молилась скороговоркой. 

Смешно переставляя буквы и меняя звуки. 

Раньше мы с мамой хохотали над ее произношением. 

Теперь слушали молча  и верили. 

Именно  она дошепчется до Бога. 

Он поймет  ее. Услышит смешные шипящие, из-за выпавших зубов. 

И прекратит войну. 

Голубоглазая,  светловолосая,  худенькая девочка, уверенная в  Боженьке,
обязательно переживет этот ад.  

Бог  не оставит ее в черном городе под рашистским бомбами. 

Весь Мариуполь  просил Его  посмотреть, что творят на земле нелюди. 

Каждый раз когда летели российские самолёты, била наземная артиллерия или
корабельные пушки, танки направляли дула в подъезды и окна - город отвечал
молитвой. 

Она неслась вверх:  

из дрожащих квартир,

ледяных подвалов, 

разнесенных вдребезги улиц, 

пахнущих кровью операционных, 

движущихся,  между минами, машин скорой,

вспыхнувших как свечки зданий. 

Мой город в эти дни был самым намоленным местом. 

Молились даже те, кто раньше  говорил: "Не верю. Нет смысла"

Мы тревожили всевышнего каждый час. 

И днём и ночью.

Рашистские самолёты  превращали город сначала в руины, потом в обломки, потом в
мусор. 

Мы звали Бога на помощь. 

Мы  рассказывали о себе, жаловались на убийц, ждали реакции.

Снаряды выбивали этажи в домах. Разрывали людей на части. 

Выгрызали дыры в земле. 

Взрывная волна душила, кидала на стены, отбрасывала и трепала все, что было на
ее пути. 

Бог не реагировал. 

Я обижалась на него. 

Не понимала: почему не накажет сразу? 

Почему не работает закон бумеранга? 

А вчера я смотрела в небо Вильгемсхафена. 

Оно было таким же, как в Мариуполе,  до войны. 

Тихое звёздное  небо. 

Я подумала: может быть в Мариуполе  нас было не слышно из-за грохота снарядов?
