% vim: keymap=russian-jcukenwin
%%beginhead 
 
%%file 26_01_2022.fb.sebastjanovich_pavel.1.kogo_dostalo.cmt
%%parent 26_01_2022.fb.sebastjanovich_pavel.1.kogo_dostalo
 
%%url 
 
%%author_id 
%%date 
 
%%tags 
%%title 
 
%%endhead 
\zzSecCmt

\begin{itemize} % {
\iusr{Леонид Шехтер}
В добрый путь!

\iusr{Олександр Жимінюк}

На каждых выборах очень много прилипал это те которые хотят любой ценой к
корыту. Не кто уже не верит лозунгом. Перевернуть все это с головы на ноги
могут только радикально настроенные люди к нынешней власти. Другого варианта не
будет. У всех чиновников самая главная ценность, это деньги. Вся институция
повалена, и поэтому нужны профессионалы дела, а не языка. Все кто это на себе
уже почувствовал, это малый, и средний бизнес. Это те люди которые в нашей
стране платят только налоги, и штрафы и при этом не видят не кокой перспективы
в этой стране.

\iusr{Игорь Ковальчук}
Электоральный Майдан ?  @igg{fbicon.smile} 
Точно, Зе и его дримтимы это ж всё предыдущая власть - Порошенки  @igg{fbicon.smile} 
П/с

Янукович ушел через уличный Майдан и призыв ввести российские войска, Порошенко
в результате самых демократичных выборов (электоральный Майдан). Вы точно
уверены, что \enquote{новые Порошенки} уйдут как те кого вы им приписываете ?

\iusr{Yurii Romanov}
Это все понятно, но когда Петя должен подключится к процессу?

\iusr{Олена Полулях}
В будь якрму випадку акціі пІд ВР на руку РФ

\begin{itemize} % {
\iusr{Святослав Дубина}
\textbf{Олена Полулях}, а хто є вигодоотримувачем свавілля чиновників і некомпетентності влади?

\iusr{Сергей Чепалда}
\textbf{Олена Полулях} Напоминает времена пороха.- делай с народом что хочешь, а кто против, тот рука кремля

\iusr{Татьяна Сердечная}
\textbf{Олена Полулях}, то бишь склонить голову в поклоне и не трепыхаться?

\iusr{Владислав Колодницький}
\textbf{Олена Полулях} вам к психиатру
\end{itemize} % }

\iusr{Світлана Костюк}

Вчера после длительного оплачиваемого отпуска, изнеженные на экзотических
островах депутаты, не вышли к тем, кто работает и платит налоги круглый год без
отпусков. Это было показательное шоу зажравшейся власти.

\begin{itemize} % {
\iusr{Yurii Romanov}
\textbf{Світлана Костюк} 

а шо там есть власть, понабирали хренотень какую то безвольную, еще и
неприкосновенности лишили. Вышли бы они или не вышли, что они решают?

\iusr{Микола Чепурний}
\textbf{Yurii Romanov} 

Если они не решают, следовательно не там протестная акция проведена. И не
экономические требования нужно выдвигать, а политические..., типа: \enquote{Распустить
хренотень ...неприкосновенную и изнеженную!} А для этого НЕОБХОДИМО иметь
ПАРТИЮ, но не \enquote{попа Гапона}

\end{itemize} % }

\iusr{Сергей Чепалда}

Есть такая партия - это партия за радикальные либеральные эконом реформы 5.10.
Балашов много доказывал что этих двух налогов хватит. Не знаю

\begin{itemize} % {
\iusr{Юрій Макарський}
\textbf{Serhiy Chepalda} Ви чули, що він ніс коли вправаджувався владою \enquote{базар} грунтів? отож

\iusr{Alexey Ponomarov}
\textbf{Юрій Макарський} слышали. что то плохое предлагал?

\iusr{Юрій Макарський}
\textbf{Alexey Ponomarov} 

дивлячись, які цілі у влади при запровадженні обігу грунтів - чи це базар, чи це
засіб для створення мільйонів рабочих місць.

\iusr{Юрій Макарський}
\textbf{Alexey Ponomarov} 

а Балашов топив за чистий базар грунтів, який веде до закріплення феодалізму на
землі і аргентинізацію з знищенням сіл.

\iusr{Alexey Ponomarov}
\textbf{Юрій Макарський} 

это повысит стоимость земли, феодализм был с мараторием,когда за копейки можно
орендовать 10000000 гектар, попробуй купить за 30/га столько же

\iusr{Юрій Макарський}
\textbf{Alexey Ponomarov} 

до чого тут мараторій? він свою функцію худо бєдно виконав, а в нас ще є
можливість створити країну самозайнятих куркулів (не дивлячись на сегрегуючий
самозайнятих селян з землі закон 1914 чи прийнятий на користь саме орендаторів
\enquote{базар} грунтів), а не базар грунтів для холдінгів (а за ними ТНК) і
рант'є-міщан, які до \enquote{трудової} діяльності на землі ніяким боком, а лише виводять
ренту в міста (в кращому випадку, а в більшості зовсім в економіку закордоном).

\iusr{Юрій Макарський}
\textbf{Alexey Ponomarov} 

як підніме вартість моїх гектарів 1, в кращому випадку 2 конкуренти? От як би
було так як в цівілізованих світі, що за кожну сотку рій сусідів покупців
відносно малоземельних, можно було б за якесь підвищення гектару заїкатись  @igg{fbicon.frown} 

\iusr{Микола Чепурний}
\textbf{Юрій Макарський} Ось саме тому потрібно мати ПОЛІТИЧНУ національну партію, а не економічну приватну балашовську !

\iusr{Юрій Макарський}
Як поховала гебня Рух, то й не стало таких партій, а чи будуть?
\end{itemize} % }

\iusr{Валерий Метелкин}

Головне коли цього дракона вб'ють не стати новим драконом, був вчора там, дуже
злий(((

\iusr{Сергей Чепалда}

На долгосрочную. Политику Гетьманцева и его самого нужно делать известным и
токсичным. Тогда есть шанс что после очередных выборов что-то изменится

\iusr{Юра Сокол}
ЛЮТО!!!! ПЛЮСУЮ!!!!

\iusr{Татьяна Новикова}
\textbf{Павел Себастьянович} Любая власть делает с народом только то, что ей позволяют

\iusr{Теодор Дяків}

Ще один Реанімаційний пакет реформ ... - \enquote{модернизация}, тобто підмальовування,
підрихтовування московсько-радянської політичної системи та відповідної їй
економіки в Україні.

І, звичайно ж, жодного слова про першочергову потребу політичних реформ ...

\begin{itemize} % {
\iusr{Павел Себастьянович}
\textbf{Теодор Дяків} а что вы называете политическими реформами?

\iusr{Теодор Дяків}
\textbf{Pavel Sebastianovich} 

Політичні реформи - послідовне заміщення успадкованої від СРСР
московсько-радянської політичної системи українсько-європейською. Конкретніше -
зміна характеру, змісту, сутності ключових інститутів держави - місцевого
самоврядування, виборчої, судової і правоохоронної систем - їх призначення,
завдань, функцій, методів роботи.. Тобто, завданням політично притомної частини
суспільного активу є, вважаю, популяризація і організація послідовного переходу
від імітації державою Республіки до реальної Республіки.

В умовах теперішнього характеру політичних інститутів держави в Україні ніякого
економічного розвитку не може бути.

\iusr{Теодор Дяків}
\textbf{Микола Чепурний} Поляки Республіку побудували, і українці рано чи пізно побудують.

\iusr{Микола Чепурний}
\textbf{Теодор Дяків} 

В Польщі США не влаштували ТВД для війни з Кремлем і влили купу долярів. А нам
зробили війну і сварки з усіма сусідами заради збільшення впливу на РФ, та ще й
МВФ зробив наших правнуків боржниками. Різницю хіба не видно ?

\iusr{Теодор Дяків}
\textbf{Микола Чепурний} Різниця в політичній свідомості поляків і українців.

\iusr{Микола Чепурний}
\textbf{Теодор Дяків} 

Є така! Але на початку \enquote{старту} найважливішою різницею була наявність
патріотичної свідомості у Валенси і шкурних інтересів у Кучми

\iusr{Теодор Дяків}
\textbf{Микола Чепурний} 

Не варто зводити успіхи чи проблеми багатомільйонних країн до однієї людини -
польське суспільство ніколи не вибрало б \enquote{двічі несудимого} Януковича чи
теперішнього блазня ...

\iusr{Микола Чепурний}
\textbf{Теодор Дяків} 

\enquote{Не варто} нехтувати \enquote{впливом особистості на хід історії}! Як би польське
суспільство було розграбовано, аморалізовано і розділено так, як це зробив
куЧмо із \enquote{синами} юЩенком, Яником та поТрохом, то не відомо які блазні і як
прийшли б до влади у них. Дуже важко \enquote{зводити до спільного знаменника}
історично + територіально + ментально + олігархічно

відмінні країни=народи

\iusr{Теодор Дяків}
\textbf{Микола Чепурний} 

Історично, ментально і територіально українці і поляки є дуже близькими, а в
минулому були навіть рівноправними частинами однієї держави. А от політично ці
народи сьогодні дуже різні : - поляки почали відриватися від Москви при першій
же нагоді у 1989-90 роках, а Україна - тільки у кінці 2013 року.

\end{itemize} % }

\iusr{Веніамін Тимошенко}

Системна помилка, яка ніколи не призведе до успіху.

Всі активісти-реформатори говорять і говорять про якусь міфічну політсилу, в
якій будуть бездоганні люди, які одразу наведуть лад в країні.

Не наведуть, тому що в партії буде пару тисяч членів і хай ще навіть 200 в
парламенті, а на місцях сотні тисяч вороватої мразі, яка буде саботувати всі
реформи і продовжувати красти. А якщо у вас не вдасться зайти в парламент, бо
оскотинені знову проголосують серцем за своїх старих вороватих кумирів? Що,
чергові 5 років грабунку та глуму і мрії про міфічну політсилу?

Воювати з диржавою потрібно просто зараз, точніше ще вчора, точніше - ще
позавчора.

Працівникам потрібно зупиняти в законних страйках всю залізницю, аеропорти,
автотранспорт, всю торгівлю, все обслуговування, а бізнесу припиняти платити
податки.

І все, це смердюче шапіто зупиниться і почне слухати ВИМОГИ, а не камлання
лоховатих терпіл. Через конкретні щохвилинні збитки та зупинку інфраструктури,
а не молитви та стони безпомічних філософів.

Ах да, всі мають озброїтись, щоб навіть не було думок спробувати розігнати все
силою.

Тут і зараз, а не колись можливо.

\begin{itemize} % {
\iusr{Андрей Гарнат}
\textbf{Веніамін Тимошенко}

А як ти бачиш наслідки, які держава може законно організувати будь підприємству
після «зупиняти платити податки»?

\iusr{Веніамін Тимошенко}
\textbf{Андрей Гарнат} 

страйк - це не шабаш з прапорцями, а зупинка роботи. Якщо немає роботи - немає
доходів і відповідно податків. Зарплат теж немає, значить немає ПДФО і ЄСВ.

\iusr{Константин Орда}
\textbf{Веніамін Тимошенко}, ФОП на єдиному податку зобов'язаний його платити незалежно від отриманих доходів. Як страйкувати?

\iusr{Андрей Гарнат}
\textbf{Веніамін Тимошенко}

Ну вот ты некоторое время начни не получать зарплату, и на что ты будешь жить?

Или если ты руководитель предприятия, возьми останови работу предприятия и
перестань выплачивать зарплату сотрудникам, и посмотришь, что начнём
происходить, так как твои предпринимательские проблемы сотрудникам не
интересны, и так на любом предприятии, независимо от его размера.

\iusr{Веніамін Тимошенко}
\textbf{Андрей Гарнат} 

страйк починають працівники, а не керівник, щоб якось триматись треба страйкові
фонди. Для ефекту достатньо тижня простоя всієї країни, а скиглити і мріяти
можна ще 30+ років.

\iusr{Веніамін Тимошенко}
\textbf{Константин Орда} 

ви можете організовано призупинити діяльність. Ну і масова непокора в кінці
кінців, не платіть і все, але всі.

А зборища під Радою варто припиняти, це безглузде витрачання часу та ресурсів.

\iusr{Веніамін Тимошенко}
\textbf{Андрій Гарнат} 

а на ці твої слова я знову тобі нагадаю свою фразу: \enquote{підприємство - це місце
для синергії керівника та його працівників, це команда}, натомість більшість
підприємців вважають свій персонал власністю чи навіть сировиною.

Результат на табло.

\end{itemize} % }

\iusr{Сергей Кононенко}

Підтримую. А вот эти рожи, которые объединились ради \enquote{спасения} страны и не
вылазящие со Свободы слова - порошенко, яценюк, аваков и т.д - уже вызывают
своим умничанием и \enquote{патриотизмом} рвотный рефлекс.

\begin{itemize} % {
\iusr{людмила карпович}
\textbf{Сергей Кононенко} а Зеленский вызывает высшее наслаждение?

\iusr{Viktoria Sklyarova}
\textbf{людмила карпович} нет, рвотный тоже

\iusr{Сергей Кононенко}
\textbf{людмила Карпович} По-Вашему, если кто-то не за петю, он автоматически за зеленского?

\iusr{людмила карпович}
\textbf{Сергей Кононенко} а что были ещё варианты?
\end{itemize} % }

\iusr{Леся Ворона}
це добре, але не розумію російську, оскільки живу в Україні. Це ви путіна так розмазали?

\iusr{Дарія Матвіїв-Веклин}
\textbf{Lesya Vorona} так \enquote{какая разніца} - це бальзам для України. Твердолобі.

\iusr{Петро Гарматюк}

Невже і цей раз хтось клюне на \enquote{нову} партію, на \enquote{нових}
професіоналів ?!

Мені смішно це читати.

І смішно, і сумно.

Зовнішній воріг - Росія звозить війська і зброю до наших кордонів, весь світ
переймається проблемою як упередити наміри Путіна, як зупинити фашиста, щоб не
напав на Україну.

Внутрішні вороги - петі, юлі, віті, ренати та інші згрупувалися, щоб роздерти
нашу Україну з середини, розсварити нас, розділити, натравити одне на одного.

Президент зі своєю командою робить майже неможливі речі, знаходячи підтримку
Європи, та Америку, Україна отримує допомогу, про яку і мріяти раніше не могли.
До нас прибуває зброя, фінанси, наші дружні країни готові прислати своїх
добровольців. Росії дають знати чим для неї закінчиться не просто
широкомаштабне вторгнення, а спроба переступити наш кордон російському солдату.

Всі українці, весь світ живе в напрузі, живе в очікуванні що ж буде дальше.

ОКРІМ ОДНИХ !!! \enquote{УКРАЇНА 30 +}

У цих одна задача - скористатися надважкою ситуацією і дорватися до влади. Вони
готові і Росії допомогти, лиш аби до корита, лиш аби красти.

ЯК НА МЕНЕ, ТО ВСЯ ЦЯ ШУШЕРА НЕ ВАРТА ВІЇДЕНОГО ЯЙЦЯ.

Судячи з того, що навіть з російської на українську не переклали - дуже спішать
відрапортувати перед Московією.

ГНАТИ В ШИЮ ЦИХ НОВОСПЕЧЕНИХ \enquote{ПРОХВЕСІОНАЛІВ}.

\begin{itemize} % {
\iusr{Веніамін Тимошенко}
\textbf{Петро Гарматюк} 

ви той приклад, кого потрібно вже селити в резерваціях з позбавленням права
голосу, саме ви довели країну до такого ублюдочного стану.

\iusr{Boris Kurchenko}
Що там з САП?

\iusr{Viktoria Sklyarova}
\textbf{Петро Гарматюк} полікуватися б,вам...

\iusr{Святослав Дубина}
\textbf{Петро Гарматюк}, вам треба в контррозвідці працювати! Кремлівських агентів на раз викриваєте  @igg{fbicon.face.tears.of.joy} 

\iusr{Леонид Шехтер}
\textbf{Петро Гарматюк} , 

в армии не сразу, а недавно сделаны титанические сдвиги. Это плюс, и браво!
Тут, как и во внешней политике, горжусь своим президентом ! Но! Почему то же я
не могу сказать о политике закручивания гаек гетьманцевым, например?!!!

\end{itemize} % }

\iusr{Александр Глущенко}

НЕТ такой партии. НЕТ такого лидера. Общество заполитизировано, и искусственно
доведено до нищеты.

\begin{itemize} % {
\iusr{Теодор Дяків}
\textbf{Oleksandr Glushchenko} 

На жаль, в суспільстві, на екранах ТБ немає дійсно політичних дискусій - ніхто
ж реально не аналізує сучасну політичну систему в Україні, не пропонує чогось
(та хоч чого-небудь!) сутнісно іншого. 30 років спостерігаються теревені, які
десятою дорогою обходять успадковану Україною від СРСР московсько-радянську
політичну систему - причину всіх українських проблем.

\end{itemize} % }

\iusr{Тарас Плахтій}

українофобські проекти, що просувають двомовність, а фактично - витіснення
української мови з ужитку, не мають жодних вагомих перспектив в Україні)

\begin{itemize} % {
\iusr{Андрей Гарнат}
\textbf{Тарас Плахтій}
це цивілізаційні проекти, і лише вони мають перспективу

\iusr{Тарас Плахтій}
\textbf{Андрій Гарнат} це \textbf{Сергій Дацюк} так думає - це його концепція)) Він помиляється

\iusr{Павел Себастьянович}
\enquote{українофобські проекти} ещё один прокурор

\iusr{Тарас Плахтій}
\textbf{Pavel Sebastianovich} власне Ваша позиція у цьому питанні, яку Ви поширюєте і відстоюєте, дала підстави зробити такій висновок)

\iusr{Павел Себастьянович}
\textbf{Тарас Плахтій} да, гравитация существует, как и русскоязычные украинцы. У вас другая вселенная? Ну так не читайте меня, читайте Фарион

\iusr{Тарас Плахтій}
\textbf{Pavel Sebastianovich} звичайно вони існують - я ж цього не заперечував. Я писав про відсутність перспектив в Україні у політичної сили, яка просуває двомовність, а фактично - витіснення української мови з ужитку)

\iusr{Павел Себастьянович}
\textbf{Тарас Плахтій} идиоты думают, что русскоязычные существуют, потому что кто-то их просуває. Также они думают, что двомовність существует, потому что её кто-то просуває. Ну и гравитация само собой только благодаря просуванню.

\iusr{Святослав Дубина}
\textbf{Тарас Плахтій}, 

ніхто українську мову не витісняє і не утискає. Люди спілкуються, як їм зручно.
Проблема існує лише у вашій уяві. Своє відношення до «Руського міра»
російськомовні висловили своїми життями на фронті. Але ви можете качати і далі
цю тему і розколювати суспільство на наші і не наші. Тільки, тут постає
питання, хто працює на Кремль?

\iusr{Тарас Плахтій}
\textbf{Pavel Sebastianovich} 

варто уважніше читати написане))) у моєму коментарі написано, що Ви просуваєте
двомовність при мовчазній згоді інших членів (крім Кулика))

\iusr{Тарас Плахтій}
\textbf{Святослав Дубина} 

\enquote{ніхто українську мову не витісняє і не утискає} - взагалі тут дискусія про
позицію майбутньої партії з мовного питання - я написав, що партія, яка
просуває двомовність, не має перспективи в Україні.

\iusr{Святослав Дубина}
\textbf{Тарас Плахтій}, 

мабуть. Але ми не просуваємо двомовність. Особисто я розмовляє трьома мовами. І
орієнтуюся на свого співрозмовниками. Якщо мені він цікавий, я розмовляю тією
мовою, що зручно йому. Якщо ні, розмовляю так, щоб він скоріше відчепилася від
мене. Тактика  @igg{fbicon.shrug} 

\iusr{Тарас Плахтій}
\textbf{Святослав Дубина} 

А Ви знайомилися з позицією Павла з мовного питання? На неї відреаував публічно
лише Віталій) До речі, Ви вдруге намагаєтеся звести підняту мною проблематику
до особистого ставлення до тих, хто на якій мові спілкується - мене в принципі
таке не цікавить)))


\iusr{Святослав Дубина}
\textbf{Тарас Плахтій}, шукайте свою аудиторію.

\iusr{Тарас Плахтій}
\textbf{Святослав Дубина} Ви будете сміятися, але ваша група і є моєю цільовою аудиторією))))

\iusr{Святослав Дубина}
\textbf{Тарас Плахтій}, ну то шукайте інші аргументи для неї. Ми мовним питанням визначились. Vitalii Kulyk з Павел Себастьянович також вийшли на спільну позицію.

\iusr{Vitalii Kulyk}
\textbf{Святослав Дубина} я уже це Тарасу говорив

\iusr{Тарас Плахтій}
\textbf{Святослав Дубина} якщо мені не зустрічався текст позиції, то його, скоріш за все, не існує)))

\iusr{Святослав Дубина}
\textbf{Vitalii Kulyk}, опублікуй на нашій сторінці руху нашу позицію щодо мовного питання. Щоб не виникало питань

\iusr{Vitalii Kulyk}
\textbf{Святослав Дубина} ввечері виставлю

\iusr{Микола Чепурний}
\textbf{Павел Себастьянович} в настояшее время, как показывает опыт военного вторжения Кремля, именно \enquote{еще один прокурор} !!!

\iusr{Микола Чепурний}
\textbf{Павел Себастьянович} 

идиоты не думают ! ...по определению. Но умники, типа Путина, пользуются тем,
что умники в бывших республиках не думают о вреде своих идиотов. \enquote{Более
сложного вопроса, чем национальный (\enquote{языковый}) в мире не существует !}

\iusr{Микола Чепурний}
\textbf{Святослав Дубина} 

шукати свою аудиторїю не важко, а от знайти в своїй аудиторіі порозуміння -
важко... От саме тому і нема емерджентності=синергії!

\iusr{Святослав Дубина}
\textbf{Микола Чепурний}, 

шукаємо. Не легко. Дуже нелегко. Вважаю, це самий великий виклик для
української інтелегенції. Але, якщо ллють воду на моїх однодумців. Та й ще
несправедливо. Тут я дуже рішучий. Вибачте, якщо іноді це виглядає образливо.

\iusr{Микола Чепурний}
\textbf{Святослав Дубина} 

Непорозуміння - \enquote{самий великий виклик для ...} усіх, хто не хоче знати основну
причину того. А для інтеленції тим більше! Бо саме вона має зрозуміти що
емерджентність (~ сінергія) з'являється лише за умови оприлюднення усіх
реальних (свідомих і підсвідомих =таємних) інтересів КОЖНОГО учасника
суСПІЛЬНОГО договору. Емерджентність - є феноменом лише при наявності
одновекторної спрямованності вивільнення енергіі мас при відсутності у неї
нез'ясованих ІНТЕРЕСІВ. Кожен спорадично проявлений таємний інтерес призводить
до КОНФЛІКТУ інтересів. Ще раз - ВСІ інтереси мають бути попередньо повністю
з'ясовано і погоджено шляхи їх досягнення.

\end{itemize} % }

\iusr{Владислав Колодницький}
Как присоединиться?

\iusr{Taras Sirko}

Учитывая, что все \enquote{выборы} являются имитацией, по сценариям правящей верхушки,
то никто другой, \enquote{через выборы} во власть не пролезет. Пока что, это замкнутый
круг.

\iusr{Маша Иванова}

Написано всё правильно. Все всегда пишут и обещают всё, что хочет услышать
народ. А в итоге маємо те що маємо.  Слово «рух» в названии партии вызывает
неприятные ассоциации ....

\iusr{Павел Себастьянович}
\textbf{Masha Ivanova} это не название партии, это просто движ

\iusr{Тимур Гольдран}
Вы правы!!! Именно мы его и создали...
Всё начинается с нас!

\iusr{Сергей Николаевич}

Любая пришедшая к власти политическая сила (новая, старые уже подготовлены) после
\enquote{поездки в Трускавец} изменит курс на 180 градусов...Не знаю, чем их там
кормили...

\iusr{Larysa Kiselyova}

Для такої політсили потрібні політики, а в нас їх немає. Все те, що є - не
політики, а заробітчани на політиці. Є експерти, економісти, фінансісти, тобто
ті, хто розуміють, як це повинно працювати, але тих чи того, хто готовий
зорганізувати та повести за собою суспільство, з волею до втілення цих рішень
не для себе, коханого та свої орди, а для країни та людей - ніт, немає.
Принаймні поки що.

\iusr{Павел Себастьянович}
\textbf{Larysa Kiselyova} будем становиться

\iusr{Лариса Остапчук}
С пенсией ви преуведичили. 1936 грн.

\iusr{Микола Чепурний}

Не мы создали такое \enquote{государство}, а мы позволили бандитам создать ОПГ. И если
по Закону государственный переворот запрещен, то по новому закону о
территориальной защите - бандитскую ОПГ НУЖНО разгонять

\begin{itemize} % {
\iusr{Павел Себастьянович}
\textbf{Микола Чепурний} чтобы были новые законы, в парламенте должны сидеть новые люди

\iusr{Микола Чепурний}
\textbf{Павел Себастьянович} чтобы в парламенте РАБОТАЛИ новые люди, нужно менять Систему

\iusr{Павел Себастьянович}
\textbf{Микола Чепурний} чтобы менять систему, нужно стать системой

\iusr{Микола Чепурний}
\textbf{Павел Себастьянович} Нужно и ОЧЕНЬ !
\end{itemize} % }

\iusr{Alexandr Orlov}
Когда чинуша будет бояться быть чиновником - тогда будет законопослушная страна!

\iusr{Lily Chaichenko}
Они должны нести ответственность свои предвыборные обещания, иначе никакого смысла в выборах нет.

\iusr{Зиновій Бокійчук}

Готовий підписатися під кожною думкою автора, тим більше, що ці думки ми з ним
узгодили під час нашої недавньої зустрічі. Прошу теж з розумінням поставитися
до мови, якою написаний допис. Автор не є етнічним українцем, але його дух,
розуміння, і вболівання за долю України я ціную вище ніж деяких з поміж тих,
котрі ходять у вишиванках і розмовляють українською. В той же час, я вважаю, що
застосування російської, як другої мови в Україні, безглузде і безперспективне.

Рух SaveФОП був приречений на невдачу з моменту активізації спротиву стосовно
РРО. Я намагався донести це до пана Сергій Доротич. Чому - це тема окрема.
Скажу лише, що ХІV розділ Податкового кодексу про спрощену систему, який
підприємці так завзято відстоюють, був написаний моїм колегою Володимир
Щербина. В співавтори до нього потім записалося багато, котрі цей розділ в
значній мірі лише спохабили. Сам же автор помер у віці 58 років від того, що в
нього не було достатньо коштів для лікування. Якби кожен ФОП, який користується
цим законом заплатив йому за його роботу, не пожертвував, а саме заплатив за
роботу 1 гривню в рік, він приніс би Україні ще багато користі, бо був
компетентним. Сьогодні ФОПи витрачають на порядки більше, але не компетентні і
не професійні лідери ведуть їх самі не знають куди.

Войовничість спротиву мало що дає. Трансформація економіки держави це не
спринт, це марафон, де під час бігу потрібно вірно розраховувати свої сиди і
приймати до уваги всі фактори. Це командна робота і такою командою є політична
партія. Вона уже твориться і в певному часі буде представлена загалу. Первиною
задачею цієї партій є створення умов для виправлення суспільної моралі і етики.
Якщо хочете - це відповідь на закид Веніамін Тимошенко.

\begin{itemize} % {
\iusr{Павел Себастьянович}
\textbf{Зиновій Бокійчук} про мову в тексте нет ни слова  @igg{fbicon.smile} 

\iusr{Зиновій Бокійчук}
\textbf{Pavel Sebastianovich} 

те що є в тексті патріоти в коментарях не прямо ставлять під сумнів на підставі
мови.

\iusr{Святослав Дубина}
\textbf{Зиновій Бокійчук}, 

а хто такий патріот? Той хто у вишиванці, українською мовою сіє розбрат, чи той
хто російською мовою виходить на протест проти свавілля влади, чи ціною свого
життя боронить країну від путінської агресії?

\iusr{Зиновій Бокійчук}
\textbf{Святослав Дубина} Це запитання з явною відповіддю. Але дякую.
\end{itemize} % }

\iusr{Вячеслав Хоменко}

Подобається Ваш ідейний підхід до справ у державі та ваша команда
\enquote{України 30+}@igg{fbicon.index.pointing.up}

\iusr{Yartsev Anatoliy}

Чиновников много, нет возможности людям развивать свой бизнес спокойно, без
налоговой, судебной, банковской и т.д. вакханалии, увы, в Украине.

\iusr{Евгений Мартынюк}

Павел, с большой симпатией отношусь к вашей позиции и позиции ваших однодумцев по движению 30+.

А возможно и будущей политической силе, я так надеюсь!
Вы часто взываете к новому общественному договору, единым правилам и законам для всех.
И есть к вам и вашим однодумцам один очень простой вопрос, но в тоже время очень сложный.
На нем погорели все наши политические партии, многие политики и слуги тоже пошли по наклонной не сумев ответить на данные вызовы.
Вопрос вытекает из обычной, банальной и бытовой ситуации, в которой может оказаться любой из вас, не дай бог конечно.
ГИПОТЕТИЧЕСКАЯ СИТУАЦИЯ!
Ваш родственник, близкий, хороший знакомый случайно или не случайно попадает в ДТП, где по его вине гибнет или травмируется человек.
Или в бытовой, случайной драке также по его вине гибнет человек.
Вы при власти!!! (предположим)
Родственники, друзья, кумовья смотрят на вас с надеждой "порешать бы надо, хороший человек"
ВОПРОС!
Вы тихо незаметно подключите все знакомства и возможности "порешать" или возьмете на себя свой "крест", не смотря на негатив, проклятия и недовольства окружения?!
На самом деле мне не важно, ответите вы мне или ваши соратники, напишите ли ответ. Ответ будет очевиден у всех - конечно мы за закон.
Самое главное - задать себе этот простенький вопрос, идя во власть или общественные движения с социальными инициативами!
ГОТОВ ЛИ Я НЕСТИ СВОЙ КРЕСТ?!!!
Р.S. Для тех что спросит, а ты сам готов.
При новом общественном договоре и равных правилах - ДА.

\iusr{Ирина Спицкая}
Пенсия 1935 минимальная.

\iusr{Ольга Юрьевна Русина}

Для них проблемы народа не интересно, так как живут в своем мире, где деньги
раздают в туалете, где премии в 10 раз больше зарплаты, где никто не спрашивает
откуда у него квартира появилась, машина, как он получил землю в 2 гектара. Там
свои законы. Кто для них ФОП, мышь навозная, со своими несчастными копейками,
тут миллионы. Они имея активов на миллионы долларов не стесняются брать
финансовую помощь. Я часто вижу в декларации депутатов и чиновников большого
ранга. Думали, что Зеля свой, придет, что-то поменяет. Ничего. Так и не убрали
недоторканность, ничего не изменилось. Опять пришли те же, просто поменялись
креслами в кабинете. Ни каких новых лиц, все те же абармоты, ненасытные рожи.

\iusr{Сергей Кононенко}
Партия 30+, крупный калибр для гопников во главе государства, достаточно 5,45 или 7,62

\iusr{Igor Golodnitsky}

Я очень извиняюсь, но риторика вышла какой-то левой по своей сути.

Вы собирались добиться именно этого эффекта?

\iusr{Елена Волосович}

Ну, вы перечислили тезисно Новый Курс, который разрабатывался совместно с ЮВТ,
более того у неё есть уже партия, которой более 20 лет и есть рейтинг, как у
неё, так и у партии. Конечно можно создать все с нуля, но... когда вы достигните
ее рейтинга, например, сколько пройдёт лет? Батькивщина парламентская партия и
у неё есть ежедневный выход на украинское ТВ. У вас только ютуб-канал. Или
примыкать к кому-то рейтинговому со сходной идеологией, либо зарабатывать и
вкладываться уже в следующее поколение, вашей жизни на 5\%-ый рейтинг, чтобы
хотяб попасть в раду... не хватит, а уж о большем и говорить нечего. Удачи.


\iusr{Марина Лебеденко}
Назвіть, будь ласка, фронтмена \enquote{Рух +30}

\iusr{Микола Лаврушко}
Система 5.10 Геннадія Балашова - единственный выход из нищеты!

\iusr{Андрей Бока}

депутаты, как и весь остальной госаппарат не намерены выполнять Конституцию и
действовать на благо общества.

===============

Ребята, ну когда закончится эта наивность?

Госаппарат, то есть государство - это инструмент диктатуры господствующего
класса.

Если кому-то не нравится (почему?) термин \enquote{класс}, можно использовать
термин \enquote{часть общества}.

В свете этого Конституция, Закон, а тем более благо общества - это понятия
строго несущественные.

\iusr{Дмитрий Задорожный}

Правильно товарищи! Дальше так нельзя! Мы с 18го года просили Порошенка
вырастить нам нового, другого лидера, за которого будут голосовать порохоботы,
платить Порошенко, а действовать тот свеже выращенный новый лидер будет в наших
интересах! Но, подлый Порошенко так нам никого и не вырастил! Потом мы
голосовали за любых первых попавшихся, только бы не Порошенко, но и они (те
Кто-угодны) тоже оказались выращенными не за наши деньги и не для ублажения
нас! Как подло это со стороны Ктоугоднов! Как трудно быть преданными теми, кто
нам ничем не обязан и рассматривает нас исключительно как лошковатый ресурс! Но
мы сильные! По сему давайте теперь еще за кого-нибудь нового мазу потянем! Тут
же ж главное, что б лица были или новые или прогрессивные, и что б финансировал
партию кто угодно только не мы, а работала она что б исключительно на наши
интересы! На этот раз Павел не ошибется же! Поддержим!...... Там правда все
сплошь лица, которые рассказывали, что надо Зелю выбирать - хуже не будет и
всех их уже видели и слышали не раз. Но мало ли. Вдруг на этот раз благодаря
новому названию все получится гораздо лучше и хуже действительно не будет. Вот
знакомьтесь, если еще не знакомы:
\url{https://www.facebook.com/watch/?v=335845658354782}


\end{itemize} % }
