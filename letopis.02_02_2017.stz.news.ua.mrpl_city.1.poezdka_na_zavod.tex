% vim: keymap=russian-jcukenwin
%%beginhead 
 
%%file 02_02_2017.stz.news.ua.mrpl_city.1.poezdka_na_zavod
%%parent 02_02_2017
 
%%url https://mrpl.city/blogs/view/poezdka-na-zavod
 
%%author_id burov_sergij.mariupol,news.ua.mrpl_city
%%date 
 
%%tags 
%%title Поездка на завод
 
%%endhead 
 
\subsection{Поездка на завод}
\label{sec:02_02_2017.stz.news.ua.mrpl_city.1.poezdka_na_zavod}
 
\Purl{https://mrpl.city/blogs/view/poezdka-na-zavod}
\ifcmt
 author_begin
   author_id burov_sergij.mariupol,news.ua.mrpl_city
 author_end
\fi

\ii{02_02_2017.stz.news.ua.mrpl_city.1.poezdka_na_zavod.pic.1}

В самом конце ΧΙΧ века у Мариуполя появились соседи. За какие-то два года близ
станции Сартана Екатерининской железной дороги были построены два
металлургических завода, казармы для рабочих и коттеджи для начальства.
Поначалу особых контактов между мариупольцами и обитателями казарм не было. У
них были свои заботы, а у людей, пришедших из близлежащих губерний в поисках
заработка на заводы, – свои. Кроме того, до появления Белого и Красного мостов
горожане и обитатели заводских поселков были разделены водной преградой. Это
был еще полноводный Кальчик, особенно непроходимый во время весенних паводков.
Однако постепенно появилось взаимное проникновение. Кто-то из горожан нашел
работу на заводах, кто-то из них устроил там свое торговое заведение, да и
заводчане нет-нет да наезжали на городской базар, чтобы разжиться рыбой, в те
достопамятные времена в изобилии водившейся в Азовском море...

Увлекшись экскурсом в историю, как-то забыта тема очерка, обозначенная в
заголовке. Прежде  всего, надо сказать, что у аборигенов города у моря бытовало
понятие \enquote{поехать на заводы}. Подразумевалось – побывать там, где дымили трубами
\enquote{Никополь} и \enquote{Провиданс}. Главным и единственным видом мариупольского
городского транспорта в описываемую эпоху были извозчики.  Адрес-календарь
\enquote{Весь Мариуполь и его уезд} за 1910 год дает возможность узнать стоимость
поездки \enquote{на заводы}. Легковые извозчики делились на два разряда. 29 мая 1908
года мариупольская городская дума утвердила таксу для них.

К первому разряду относились извозчики, имевшие \enquote{одобренные городской Управой
экипажи лучшего типа на резиновом ходу, хороших парных лошадей, хорошую сбрую и
кучерскую одежду}.  Ко второму разряду относились все остальные. Были еще и
одноконные извозчики. Согласно тарифам поездка на завод Общества \enquote{Русский
Провиданс} обходилась седокам в 1 рубль 25 копеек на перворазрядном экипаже, 1
рубль - на второразрядном, а на одноконном – 60 копеек. Визит на завод
Никопольско-Мариупольского общества стоил 1 рубль 50 копеек, 1 рубль 20 копеек
и 70 копеек соответственно. Заметим, что неквалифицированный рабочий завода
\enquote{Русский Провиданс} за рабочий день получал один рубль. Его квалифицированные
коллеги  оценивались в два, а некоторые – в три раза  больше...

1 мая 1933 года в нашем городе были пущены трамваи. Маршруту \enquote{Гавань Шмидта –
завод имени Ильича} был присвоен номер 1. Для него был  построен деревянный
мост через речку Кальчик с одной колеей. Трамвай от конечной остановки
поднимался по улице Апатова, у базара поворачивал на проспект Республики
(теперь – пр. Мира, с 1960 г. – пр. Ленина), возле сквера был у него поворот на
улицу Артема, а как далее пролегал его путь, нет смысла рассказывать, поскольку
он нашим современникам известен.  Конечным пунктом было кольцо, устроенное на
проспекте Ильича в метрах ста за улицей Левченко. Трамвайная линия
первоначально была одноколейной.

Дабы не мудрствовать лукаво, приведем цитату из Википедии: \enquote{Трамваи в Мариуполе
ходили по улицам вплоть до оккупации Мариуполя немецко-фашистскими войсками 8
октября 1941 года, затем во время оккупации - до середины 1942 года. Немецкая
администрация сохранила трамвайное движение для бесперебойного снабжения
рабочими немногих действующих цехов завода имени Ильича. Однако во всех местах
движение стало однопутным, вторые рельсы были сняты и отправлены в Германию на
переплавку}. Гитлеровцы, покидая город, полностью разрушили трамвайные линии,
депо и подстанции. Только 10 января 1945 года удалось восстановить движение
трамваев на маршруте №1. Трамваи не справлялись с перевозкой пассажиров,
особенно в часы пик. Они ходили с 5 часов утра до 9 вечера, после этого на
линиях оставались только дежурные вагоны.

Еще до пуска трамваев было организовано движение рабочего поезда от станции
Жабовка, которая находилась около гавани Шмидта, до Рабочей площадки в
Ильичевском районе. Поезд состоял из небольшого паровозика и нескольких
двухосных товарных вагонов, наскоро оборудованных лавками для пассажиров.
Где-то в начале 50-х годов теперь уже прошлого столетия товарняки заменили
трофейными румынскими пассажирскими вагонами. Воспоминания юности о поездке в
таком вагоне. Мягкие сиденья, привычных для  нас верхних полок нет, над
сиденьями пристроены сетки для шляпы и зонтика. Свисток паровоза - поезд
трогался. Мимо пробегали нагорная часть старого города, домишки Верхних Аджах и
бараки Правого берега.  Затем в вагоне становилось темно, это состав проходил
через туннель, устроенный под насыпью железной дороги МПС. Затем был мост, и
через непродолжительное время достигал конечного пункта...

Некоторые предприятия, чтобы облегчить доставку людей на работу, применяли для
этого грузовые автомобили. В кузове делали лавки из толстых досок, а сверху -
укрытие из фанеры, которое в народе называли  \enquote{будкой}. Не редкостью на улицах
были линейки – одноконные экипажи для начальства средней руки. К слову, на
таких же линейках выезжали на вызовы бригады скорой медицинской помощи...

Вернемся снова к трамваю. Конец 40-х годов. Что можно было увидеть  во время
поездки по маршруту №1? Почти сплошной строй  сгоревших домов на проспекте
Республики и улице Артема, разъезд перед мостом через Кальчик, ожидание
встречного трамвая, разминовка и неспешное продвижение по одноколейному участку
маршрута. Столбы для подвески трамвайных токоведущих проводов в виде буквы \enquote{Т}.
Слева все пространство, занятое теперь парком имени Н. А. Гурова и
Экстрим-парком, засажено фруктовыми деревьями. Справа – бескрайние поля
подсолнуха и кукурузы. Примерно там, где в наши дни улица Покрышкина пересекает
проспект Металлургов, - еще один разъезд.

Первые строения, вернее, остатки того, что сохранилось после бомбежек,
появлялось у теперешней улицы Блажевича. Вот воспоминания ветерана войны
И.С.Смыка: \enquote{У перекрестка  с улицей Блажевича стояла школа №42. А там, где
находится теперь гостиница \enquote{Дружба}, перед войной были построены два
многоэтажных дома. В них во время оккупации располагались казармы немецких
летчиков. Школу и казармы эти разбомбила советская авиация}. А далее снова поля
до самой улицы Ильича, почетное наименование \enquote{проспект}  эта городская
магистраль получила значительно позже. Поворот, слева - 41-я школа, вдоль
заводского забора -  бараки, с противоположной стороны -  маленькие домишки,
сбегающие в балку. Косо пересекает улицу железнодорожный путь, связывающий
площадки \enquote{А} и \enquote{Б} завода имени Ильича,  фабрика-кухня, снова бараки. Наконец,
трамвайное кольцо и Ильчевский рынок, в который упиралась улица, образуя тупик.
Приехали.
