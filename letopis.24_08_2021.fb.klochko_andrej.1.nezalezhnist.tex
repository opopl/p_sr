% vim: keymap=russian-jcukenwin
%%beginhead 
 
%%file 24_08_2021.fb.klochko_andrej.1.nezalezhnist
%%parent 24_08_2021
 
%%url https://www.facebook.com/vyborkherson/posts/3097905350440164
 
%%author_id klochko_andrej
%%date 
 
%%tags nezalezhnist,ukraina
%%title Независимость - Украина
 
%%endhead 
 
\subsection{Независимость - Украина}
\label{sec:24_08_2021.fb.klochko_andrej.1.nezalezhnist}
 
\Purl{https://www.facebook.com/vyborkherson/posts/3097905350440164}
\ifcmt
 author_begin
   author_id klochko_andrej
 author_end
\fi

Ровно 30 лет назад, в силу провозглашения Акта о независимости Украины,
осуществились желания для большинства народа Украины – самостоятельно и
суверенно строить процветающее будущее, уже собственной страны. Тогда, каждый
украинец был уверен в собственной надежде, что с таким мощным экономическим
наследством, с огромным потенциалом  в промышленности, в науке, в сельском
хозяйстве и т.д.  С таким успешным  сухопутным географическим расположением с
выходами в Черное и Азовское моря, наличием богатейших природных ресурсов на
своей территории, имея многонациональный и многомиллионный состав населения
своего государства. Украина просто обязана стать одной из успешной и сильной
державы, среди независимых государств мира. Тогда, именно такие чаяния
испытывал наш народ.

\ifcmt
  pic https://scontent-frx5-1.xx.fbcdn.net/v/t1.6435-9/240568379_3097904920440207_6280734566424165345_n.jpg?_nc_cat=110&_nc_rgb565=1&ccb=1-5&_nc_sid=730e14&_nc_ohc=I6NDH0NHkisAX-jB0xK&_nc_ht=scontent-frx5-1.xx&oh=1cf768e5e2491f71e50cb775a9051a0b&oe=615A1ECE
  width 0.7
\fi

Однако, свой государственный юбилей Украина отмечает в условиях больших
потерь: демографических, экономических, социальных, территориальных и наличия
политического кризиса. 

Государство, которое было  одним  из богатейших из числа  СНГ, за столь,
короткий период, превратилось, из-за недальновидной проводимой властью
политикой,  в  страну, где люди поставлены на грань выживания, где системно
уничтожается медицина и образования, где высокая трудовая эмиграция по
причине  отсутствия рабочих мест внутри  страны, где не соблюдается
Конституция, где делят людей по языковому, этническому и национальному
признаку.  Когда, по ключевым вопросам жизнедеятельности страны,  вместо
всенародного референдума, власть авторитарно принимает решения под диктовку
внешних кураторов, открыто действуя в ущерб собственного народа. Где на
стратегические предприятия, назначаются иностранцы, после которых остаются
сплошные убытки и развал целых отраслей и многое другое.

Поэтому, можно ответить однозначно, что не о такой независимости мечтал народ
Украины 30 лет назад. И нам, украинцам, прежде всего, самим себе стоит
ответить: возможно ли быть независимым государством, без благополучия
собственных граждан,  без экономической независимости,  без гарантий на
медицинскую и без гарантий на социальную помощь  и т.д.?

Исходя из реалии нашей жизни в День Независимости, хочу пожелать:
взаимопонимания, взаимовыручки, добра и мира. Всего того, что служит для
консолидации общества. Ведь только объединившись, мы способны сделать
страну мирной и  успешной, сохранить территориальную целостность Украины,
избавиться от вредительства  внешнего управления,  обеспечить национальное
единство и реальную Независимость страны.

Всем мира и добра!

\begin{itemize}
%%%fbauth
%%%fbauth_name
\iusr{сергей синицкий}
%%%fbauth_url
%%%fbauth_place
%%%fbauth_id
%%%fbauth_front
%%%fbauth_desc
%%%fbauth_www
%%%fbauth_pic
%%%fbauth_pic portrait
%%%fbauth_pic background
%%%fbauth_pic other
%%%fbauth_tags
%%%fbauth_pubs
%%%endfbauth
 

Какое ты имеешь отношение к Украине??? Ваш праздник- День России. Из за таких
пидаров опзжешников, регионалов, комунистов и другое нечести в стране пиздец.
Потому что страна 30 лет независимости была зависима от Москвы. Потому что у
власти и в силовых структурах были граждане мордора или их 6. Потому что ваши
деятели разворовали страну и до сих пор пытаются её утопить.
\end{itemize}

