%%beginhead 
 
%%file 19_12_2018.fb.fb_group.mariupol.biblioteka.korolenka.1.chitayuchi_stusa
%%parent 19_12_2018
 
%%url https://www.facebook.com/groups/1476321979131170/posts/2001880426575320
 
%%author_id fb_group.mariupol.biblioteka.korolenka,kibkalo_natalia.mariupol.biblioteka.korolenko
%%date 19_12_2018
 
%%tags mariupol,biblioteka,stus_vasyl,poezia,literatura,film
%%title Читаючи Стуса
 
%%endhead 

\subsection{Читаючи Стуса}
\label{sec:19_12_2018.fb.fb_group.mariupol.biblioteka.korolenka.1.chitayuchi_stusa}
 
\Purl{https://www.facebook.com/groups/1476321979131170/posts/2001880426575320}
\ifcmt
 author_begin
   author_id fb_group.mariupol.biblioteka.korolenka,kibkalo_natalia.mariupol.biblioteka.korolenko
 author_end
\fi

Читаючи Стуса 

19 грудня в Центральна міська бібліотека ім. В.Г. Короленка м. Маріуполь
презентований документальний фільм «Читаючи Стуса». Фільм створений
Маріупольським телебаченням до 80-річчя відомого українського поета і
громадянина Василя Стуса, що широко відзначалось в січні цього року. Режисером
виступила журналістка Лариса Бондаренко, відео монтаж - Володимир Фартушняк.  У
фільмі  «Читаючи Стуса» звучать  пісні самодіяльного композитора і виконавця
Григорія Кабанцева, написані на стусівські вірші. «Ці пісні я проніс крізь своє
серце, вони буквально вистраждані мною», - каже Григорій Григорович.

Запрошені старшокласники  із ЗОШ № 65 та № 1 після перегляду відеофільму
прочитали декілька віршів, також від молоді пролунав заклик бути сильними
духом, відстоювати свої думки, як Василь Стус.
