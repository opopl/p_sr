% vim: keymap=russian-jcukenwin
%%beginhead 
 
%%file 20_10_2022.stz.news.ua.donbas24.1.vidnovlennja_knyzhkovogo_fondu_mrpl_biblioteka
%%parent 20_10_2022
 
%%url https://donbas24.news/news/vidnovlennya-knizkovogo-fondu-mariupolya-koli-vidkrijetsya-nova-biblioteka
 
%%author_id demidko_olga.mariupol,news.ua.donbas24
%%date 
 
%%tags 
%%title Відновлення книжкового фонду Маріуполя — коли відкриється нова бібліотека
 
%%endhead 
 
\subsection{Відновлення книжкового фонду Маріуполя — коли відкриється нова бібліотека}
\label{sec:20_10_2022.stz.news.ua.donbas24.1.vidnovlennja_knyzhkovogo_fondu_mrpl_biblioteka}
 
\Purl{https://donbas24.news/news/vidnovlennya-knizkovogo-fondu-mariupolya-koli-vidkrijetsya-nova-biblioteka}
\ifcmt
 author_begin
   author_id demidko_olga.mariupol,news.ua.donbas24
 author_end
\fi

\ii{20_10_2022.stz.news.ua.donbas24.1.vidnovlennja_knyzhkovogo_fondu_mrpl_biblioteka.pic.front}
%\ii{20_10_2022.stz.news.ua.donbas24.1.vidnovlennja_knyzhkovogo_fondu_mrpl_biblioteka.txt}

\begin{center}
  \em\bfseries\Large
Центральна бібліотека ім. В. Г. Короленка у Маріуполі, приміщення якої
зруйновано, працює над відновленням книжкового фонду
\end{center}

До 24 лютого у Маріуполі функціонувало 15 публічних бібліотек, 5 із них були
дитячими. Зараз майже всі \href{https://donbas24.news/news/mariupolska-biblioteka-vidkrivajetsya-u-dnipri-i-zbiraje-novii-fond-fond}{\emph{бібліотеки зруйновані}}.%
\footnote{Маріупольська бібліотека відкривається у Дніпрі і збирає новий фонд, Яна Іванова, donbas24.news, 24.08.2022, \par%
\url{https://donbas24.news/news/mariupolska-biblioteka-vidkrivajetsya-u-dnipri-i-zbiraje-novii-fond-fond}%
}
Наразі маріупольська
бібліотека відновила діяльність у Центральній бібліотеці м. Дніпро.

\begin{leftbar}
\emph{\enquote{На жаль, нам не вдалося евакуювати книги з Маріуполя. Проте ми вже
відновлюємо наш бібліотечний фонд, який майже повністю було
знищено.Частину книг мешканці міста забирали для розпалювання багаття,
адже треба було готувати їжу. Також розграбована вся техніка і меблі}},
— підкреслила керівниця Комунальної установи \enquote{Центральна міська
публічна бібліотека Маріупольської міської ради з філіями} та
директорка центральної бібліотеки ім. Короленка \href{https://www.facebook.com/viktoria.lisogor}{Вікторія Лісогор}.
\end{leftbar}

\textbf{Читайте також:} \emph{Українським дітям — українська книга: як скористатися соціальною програмою}%
\footnote{Українським дітям — українська книга: як скористатися соціальною програмою, Алевтина Швецова, donbas24.news, 20.09.2022, \par%
\url{https://donbas24.news/news/ukrayinskim-dityam-ukrayinska-kniga-yak-skoristatisya-socialnoyu-programoyu}%
}

\subsubsection{Бібліотека перетворилася на прихисток}

Приміщення Центральної бібліотеки імені Короленка у Маріуполі слугувало
прихистком як для співробітників бібліотеки з їхніми дітьми, так і для
маріупольців, які втратили домівки. Для себе Вікторія однозначно вирішила, що
залишатиметься в Маріуполі.

24 лютого Вікторія приїхала до бібліотеки, тоді ж вона роздала всі трудові
книжки співробітникам. Інженер з охорони праці попросилася ночувати зі своєю
донькою в бібліотеці, адже вона жила на 8 поверсі і їй було страшно через
постійні вибухи залишатися у власній квартирі. Квартира Вікторії знаходилася на
7 поверсі, вікна виходили на \enquote{Азовсталь}. Саме тому вона теж вирішила, що
будівля бібліотеки може стати надійним прихистком, тому з донькою переїхала до
бібліотеки.

Понад два тижні у бібліотеці жило близько 50 осіб. Зокрема, 5 березня до
приміщення приїхали поліцейські, які повідомили, що бібліотека стане місцем
їхнього дислокування. 6 березня вони привезли до бібліотеки свої сім'ї.

\ii{20_10_2022.stz.news.ua.donbas24.1.vidnovlennja_knyzhkovogo_fondu_mrpl_biblioteka.pic.1}

7 березня, коли будинки на пр. Миру № 43 та 45 потрапили під масові обстріли,
до бібліотеки приєдналися мешканці цих будівель. Поступово маріупольці зайняли
все приміщення адміністративного корпусу бібліотеки. Завдяки поліцейським до 12
березня Вікторія отримувала деяку інформацію про обстановку в місті.

\textbf{Читайте також:} \emph{Освіта в умовах війни: українським учням та вчителям доступні електронні підручники}%
\footnote{Освіта в умовах війни: українським учням та вчителям доступні електронні підручники, Сорокіна Наталія, donbas24.news, 06.09.2022, \par%
\url{https://donbas24.news/news/osvita-v-umovax-viini-ukrayinskim-ucnyam-ta-vcitelyam-dostupni-elektronni-pidrucniki}%
}

\begin{leftbar}
\emph{\enquote{Поки ми перебували в бібліотеці, нам вдалося сховати в надійному місці
найбільш цінні екземпляри}}, — зазначила директорка бібліотеки.
\end{leftbar}

\subsubsection{Як було знищено приміщення бібліотеки?}

\ii{20_10_2022.stz.news.ua.donbas24.1.vidnovlennja_knyzhkovogo_fondu_mrpl_biblioteka.pic.2}
\ii{20_10_2022.stz.news.ua.donbas24.1.vidnovlennja_knyzhkovogo_fondu_mrpl_biblioteka.pic.3}

16 березня почалося масоване бомбардування Маріуполя. Напередодні 15 березня
Вікторії вдалося додзвонитися до чоловіка, який виїхав ще 24 лютого. Зранку 16
березня вона прибігла до своєї квартири і зібрала деякі речі. Того ж дня її,
доньку та декількох співробітників вивезли до Мангуша. На той час в бібліотеці
залишилося всього 8 осіб.

\begin{leftbar}
\emph{\enquote{Все, що мені вдалося забрати — це документи на бібліотеку і печатка. Трудові
книжки я всім роздала ще 24 лютого. А вивозити книги, на жаль,
можливості не було}}, — зазначила Вікторія.
\end{leftbar}

Добиратися було дуже складно, навколо все бомбили. Вже у Мангуші жінці
повідомили, що драмтеатр зруйнований.

\ii{20_10_2022.stz.news.ua.donbas24.1.vidnovlennja_knyzhkovogo_fondu_mrpl_biblioteka.pic.4}

В ніч на 19 березня бібліотеку розбомбили. Можливо, це сталося через наводку
місцевих жителів, адже в бібліотеці перебували поліцейські. Слід зазначити, що
в цій будівлі з 1905 року працював банк. Це була \emph{\textbf{перша банківська будівля у
Приазов'ї та на південному сході України.}} Саме через цей банк проходили
фінансові операції маріупольських металургійних заводів \enquote{Нікополь} та
\enquote{Провіданс}, будівництво міських приватних та громадських будівель. Знищення
цінних пам'яток культури, архітектури та історії є злочином, скоєним
російськими військовими, які спрямовані проти культурної спадщини Маріуполя і
загалом України.

\textbf{Читайте також:} \emph{Захисники врятували з Лисичанська українські книжки та передали бібліотеці}%
\footnote{Захисники врятували з Лисичанська українські книжки та передали бібліотеці, Яна Іванова, donbas24.news, 10.08.2022, \par%
\url{https://donbas24.news/news/zaxisniki-vryatuvali-z-lisicanska-ukrayinski-knizki-ta-peredali-yix-biblioteci-foto}%
}

\begin{leftbar}
\emph{\enquote{Прикро визнавати, але від будівлі майже нічого не лишилося. Тільки у
підвальних приміщеннях деякі книжки, але і там багато чого
розграбували}}, — підкреслила директорка установи.
\end{leftbar}

\ii{20_10_2022.stz.news.ua.donbas24.1.vidnovlennja_knyzhkovogo_fondu_mrpl_biblioteka.pic.5}

\subsubsection{Як відновлюватиметься книжковий фонд Маріуполя?}

Вікторія Лісогор змогла виїхати до Запоріжжя лише 26 березня. Наразі вона
перебуває у Дніпрі. Маріупольчанка дуже важко перенесла знищення рідного міста
та місця, з яким у неї було пов'язано чимало важливих ідей та подальших планів.
Зараз жінка працює з психологом, щоб розібратися в багатьох моментах і рухатися
далі. Але головне, на що вона спрямувала всі сили — це відновлення книжкового
фонду Маріуполя. Втім дуже болючим залишається кадрове питання. \emph{60\%
працівників} маріупольських бібліотек лишилися в окупованому місті. Більшість з
них стали колаборантами на чолі з \emph{\textbf{Юлією Коряченко}} — провідною бібліотекаркою
Маріупольської історичної бібліотеки ім. Грушевського, яка до війни плела
жовто-блакитні бісери і відрізнялися патріотичною позицією. Наразі вона
наголошує, що ніколи не приховувала своєї проросійської позиції. \emph{15 працівників}
виїхали до рф.

\ii{20_10_2022.stz.news.ua.donbas24.1.vidnovlennja_knyzhkovogo_fondu_mrpl_biblioteka.pic.6_8}

\begin{leftbar}
\emph{\enquote{Я не знаю, як далі співпрацювати з людьми, які зробили такий вибір. Звісно,
для мене це неможливо. Не розумію, чому після всього, що сталося в
Маріуполя, після знищення нашого дому, вони почали працювати під владою
окупантів та ще й виїхала до країни-агресорки. Для мене все це
неприпустимо!}}, — наголосила Вікторія Лісогор.
\end{leftbar}

\textbf{Читайте також:} \emph{\enquote{Маріуполь монументальний}: у Львові надрукують книгу про втрачене мистецтво}%
\footnote{\enquote{Маріуполь монументальний}: у Львові надрукують книгу про втрачене мистецтво, Яна Іванова, donbas24.news, 31.08.2022, \par%
\url{https://donbas24.news/news/mariupol-monumentalnii-u-lvovi-nadrukuyut-knigu-pro-vtracene-mistectvo}%
}

\emph{10 працівників бібліотек} виїхали до України і \emph{13 осіб} — до Європи. Саме ці
співробітники на чолі з Вікторією Лісогор почали збирати електронні версії книг
як авторів Приазов'я, так і творців інших куточків України. Водночас фонд
бібліотеки суттєво поповнився завдяки допомозі від мешканців Дніпра,
благодійних організації та бібліотек з інших міст України. Вікторія вже
отримала 54 посилки. Знану допомогу надали Інститут української книги,
Благодійний фонд \enquote{Бібліотечна країна}, відділ культури міста Чернігова та
Дніпровська бібліотека. Зараз фонд бібліотеки налічує \emph{\textbf{близько 1000 книжок}}.
Планується, що найближчим часом маріупольська бібліотека розташовуватиметься у
новому центрі \enquote{Я-Маріуполь} в Дніпрі, який відкриється у двоповерховому (більше
за площею) приміщенні.

Кожен українець може допомогти маріупольській бібліотеці імені Короленка
відновити книжковий фонд, відправивши книгу або декілька. Перевага надається
україномовним виданням.

\textbf{Контакти для відправки:} \emph{Вікторія Олександрівна Лісогор, +38 (096) 323−29−35, м.
Дніпро, відділення Нової Пошти № 162.} Важливо вказати, що оплата відбудеться за
рахунок отримувача. Адже Нова пошта вирішила підтримати бібліотеку Маріуполя і
надала велику кількість бонусів для оплати за отримані книжки.

Раніше Донбас24 розповідав, які \emph{книги авторів Донеччини}%
\footnote{Топ-5 книг авторів Донеччини, які варто прочитати всім українцям, Ольга Демідко, donbas24.news, 16.09.2022, \par%
\url{https://donbas24.news/news/top-5-knig-avtoriv-doneccini-yaki-varto-procitati-vsim-ukrayincyam}%
}
будуть цікаві кожному українцю.

Найсвіжіші новини та найактуальнішу інформацію про Донецьку й Луганську області
також читайте в нашому телеграм-каналі Донбас24.

ФОТО: з особистого архіву Вікторії Лісогор.

\ii{insert.author.demidko_olga}
