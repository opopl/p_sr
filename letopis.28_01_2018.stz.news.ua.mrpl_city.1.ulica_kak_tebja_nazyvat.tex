% vim: keymap=russian-jcukenwin
%%beginhead 
 
%%file 28_01_2018.stz.news.ua.mrpl_city.1.ulica_kak_tebja_nazyvat
%%parent 28_01_2018
 
%%url https://mrpl.city/blogs/view/ulitsa-kak-tebya-nazyvat
 
%%author_id burov_sergij.mariupol,news.ua.mrpl_city
%%date 
 
%%tags 
%%title Улица, как тебя называть?
 
%%endhead 
 
\subsection{Улица, как тебя называть?}
\label{sec:28_01_2018.stz.news.ua.mrpl_city.1.ulica_kak_tebja_nazyvat}
 
\Purl{https://mrpl.city/blogs/view/ulitsa-kak-tebya-nazyvat}
\ifcmt
 author_begin
   author_id burov_sergij.mariupol,news.ua.mrpl_city
 author_end
\fi

\ii{28_01_2018.stz.news.ua.mrpl_city.1.ulica_kak_tebja_nazyvat.pic.1}

Есть такая наука – топонимика. Ученые, которые ею занимаются, изучают
географические названия, смысловое значение, их изменения и т.п. Так вот, для
этих ученых наш благословенный Мариуполь может стать неисчерпаемым источником
для исследований. Действительно, само название города менялось три раза. Город
Павловск - с 1778 г., Мариуполь - с 1780 г., Жданов - с 1948 г. и с 1989 г.
снова Мариуполь. А улицы? У них названий почти век вообще не было. Да кому эти
названия были нужны? Все и так знали, кто где живет, где базар, где какая лавка
и чем в ней торгуют. И неудивительно. Например, в городе с пригородными селами,
которые впоследствии с ним слились, проживало в 1795 году 2623 жителя обоего
пола.

\ii{28_01_2018.stz.news.ua.mrpl_city.1.ulica_kak_tebja_nazyvat.pic.2}

Но с середины XIX века в город стал прибывать народ со стороны, и стало ясно,
что без наименования улиц не обойтись. И вот, 28 сентября 1876 года улицы
получили собственные имена: продольные - Малая Садовая, Большая Садовая,
Итальянская, Георгиевская, Екатеринин­ская, Николаевская, Митрополитская,
Фонтанная, Евпаторийская, Кафайская, Готфейская, Карасевская, Бахчисарайская,
поперечные - Торговая, Харлампиевская, Греческая, Марии-Магдалиновская,
Та­ганрогская, Больничная, Константиновская. Названия улицам давали со смыслом.
Торговая – потому, что на ней было много торговых заведений, Больничная –
поскольку на ней находилась больница, Евпаторийская – на ней жили  выходцы из
Гезлева (Евпатории), Митрополитская – там была усадьба митрополита Игнатия и
так далее.

После Октябрьской революции, говоря казенным языком, элементы инфраструктуры
Мариуполя начали переименовываться. Причем некоторые из них многократно.
Пожалуй, \enquote{чемпионом} среди них можно назвать улицу Георгиевскую. В первые годы
советской власти ей присвоили имя Троцкого. Но оказалось, что Троцкий не
столько вождь революции, сколько злейший ее враг. Так появилось нейтральное –
\enquote{1 Мая}. 8 октября 1941 года наш город оккупировали гитлеровцы. Прислуживающая
им городская управа заменила \enquote{1 Мая} на \enquote{Богдана Хмельницкого}. 31 декабря 1991
года, когда  мариупольским улицам вернули исторические названия, улица обрела
первородное имя – Георгиевская.

Переименовывали несколько раз и главную улицу города. В народе ее называли
\enquote{Большой}, хотя ни в одном внушающем доверие документе такое
наименование вы не найдете. Она действительно была большой по меркам скромного
уездного городка – самая широкая из всех и самая протяженная. Официальное ее
название было Екатерининская. Это \enquote{девичье} имя сохранялось до самой
революции, вернее, до тех пор, когда укрепилась в наших краях советская власть.
Вдруг улица Екатерининская превратилась в проспект Республики. При этом ни один
дом, даже ни один камень великолепной брусчатки не изменился. Кстати,
говаривали, что гранитное покрытие проезжей части этой городской магистрали
делала та самая артель каменотесов, что мостила Красную площадь в Москве. В
1941 году оккупантам или продажной городской управе показалось, что
\enquote{Республика} - слишком революционно звучит. Так (к счастью, на весьма
короткое время), главная городская магистраль стала проспектом  Театральным.
Правда, таблички с наименованием не поменяли. То ли недосуг было, то ли
\enquote{хозяева} денег на их изготовление не дали. Проспект под
\enquote{революционным} названием просуществовал до 22 апреля 1960 года, когда
в честь 90-летия вождя революции ему дали имя Ленина. К тому времени уже была
улица Ленина, теперешняя Николаевская. Что делать? Да просто \enquote{Ленина}
поменяли на \enquote{Донбасскую}. Теперь это проспект Мира.

Приведем здесь парадоксальный пример мариупольского городского \enquote{имятворчества}.
В сентябре 1924 года исполнилось 60 лет с момента создания I Интернационала.
Это выдающееся событие мариупольские власти решили отметить переименованием
одной из улиц. И переименовали улицу Торговую в улицу... III Интернационала. Не
удивляйтесь, здесь нет ошибки, именно III Интернационала, в честь годовщины
Интернационала Первого.

Есть на картах нашего города разных времен феномен составных названий улиц.
Причем появился он с дореволюционных времен. Таганрогская улица. Она начиналась
у сквера и круто спускалась вниз, к Бахчисарайской, то есть к нынешнему
бульвару Шевченко. Странного здесь ничего нет, часть современной улицы Куинджи
- от сквера до Малой Садовой (теперешней Семенишина) - в прежние времена
называлась Больничной. Таганрогская была ограничена двумя православными
храмами: Марии-Магдалиновской церковью, что возвышалась в центре
Александровского сквера, и Рождества Богородицы (в народе ее называли
Карасевской) церковью. Улица Марии-Магдалиновская начиналась у Малой Садовой,
достигала Екатерининской у Госбанка, а далее уже начиналась Греческая улица.
Когда возвращали исторические названия мариупольских улиц, Марии-Магдалиновская
канула в Лету. Среди переименовывателей не нашелся человек, который бы знал,
что таковая существовала.

Эта заметка, конечно, не исчерпывает тему наименований и переименований.
Единственно, что можно сказать: исторический опыт нашей малой родины
показывает, что от перемены названия проспектов, площадей, улиц и даже
переулков вид фасадов их домов, состояние мостовой и тротуаров совершенно не
изменяется.
