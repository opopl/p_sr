% vim: keymap=russian-jcukenwin
%%beginhead 
 
%%file 10_12_2021.fb.fb_group.story_kiev_ua.1.istoria_kiev
%%parent 10_12_2021
 
%%url https://www.facebook.com/groups/story.kiev.ua/posts/1815999311930195
 
%%author_id fb_group.story_kiev_ua,vakulenko_tatjana.kiev
%%date 
 
%%tags kiev
%%title История произошла в Киеве
 
%%endhead 
 
\subsection{История произошла в Киеве}
\label{sec:10_12_2021.fb.fb_group.story_kiev_ua.1.istoria_kiev}
 
\Purl{https://www.facebook.com/groups/story.kiev.ua/posts/1815999311930195}
\ifcmt
 author_begin
   author_id fb_group.story_kiev_ua,vakulenko_tatjana.kiev
 author_end
\fi

История произошла в Киеве.

Здесь нет описаний Киева, но возможно будет интересно?

Суровые новогодние будни.

Сидим с котом вяжем носки на заказ, никого не трогаем, даже примусы не
починяем. На дворе нынче вечер 31 декабря 2020 года. Елка нарядная огнями
переливается.

\ii{10_12_2021.fb.fb_group.story_kiev_ua.1.istoria_kiev.pic.1}

Я уже и песню неприличную в \enquote{тик-ток} спела (матерные страдания назыается), и
блинчиков нажарила, и в сториз в разных костюмах (бабыягином и снегуркином) меня
красивую выставила...

Обидно что работу новогоднюю отменили! Я ж за месяц от других заказов
отказалась - уж очень предложение вкусное было! Артисты знают-новогодняя ночь весь
январь кормит.

А тут еще и год сами знаете какой... Не сильно-то и зовут Снегуркой работать на
праздники. И так сплошной маскарад...

Кароч, сижу я как невеста без места, дурочка-снегурочка с помытой шеей... Если кто
не понял - Снегурочка я! Пожизненно.

Как начала с пяти лет снегурить, так и остановиться не могу. В другие-то дни я
всё больше маленькую разбойницу играла, ну вы помните-из \enquote{Снежной королевы}. А уж
нлвогодний главный утренник-только Снегурочку! Потом постарше как из детства по
возрасту вышла, Бабу-Ягу стала играть. В другие дни. В новогоднюю же ночь только
Снегурочку и точка. Так жизнь и пролетела весело и незаметно. Уж и возраст
критический, надо бы о Зимушке-Зиме подумывать, а с меня всё Снегурочку требуют.

Вот только в этом году видимо не потребовалось. Ну и лан, нам с котом и дома
неплохо.

Еще и телефон звонит. Голос вкрадчивый партнера моего вечного Деда Мороза

-Ты что счас делаешь?

Что может делать нормальная безработная сорокадвухлетняя Снегурочка 31 декабря
в 7 часов вечера на ночь глядя?

-Блинчики ем от огорчения и носки вяжу!

И тут мой радостный Дед Мороз кричит

-Выезжай немедленно, срочная работа нарисовалась!

Кароч, какой-то форс-мажор, кто-то не приехал что ли, надо спасать ситуацию. Хватаю
костюм, бегу-переворачиваюсь и волосы назад!

Приехала Под тра-ля-ля в каком-то коридоре репетируем наше па-де-де, идем
гримироваться, одеваться... И тут я понимаю, что костюм Бабы-Яги я взяла, а
Снегурочку НЕТ!!!

Ну баба-то, типо ягодка, и смысл в том, что праздник начинает Баба-Яга, которая
против всех и наступления Нового года в особенности. А в новогоднюю полночь она
же превращается в Снегурочку, танцует с Дедом Морозом торжественно-нежное
па-де-де в сверкающих белоснежных костюмах... Салют, все дела... Ну вы поняли.

Так вот костюм Снегурочкм остался дома! Ы-ы-ы...

На минуточку, девять вечера; через час в крайнем-полтора надо начать
программу, блин!

Видели бы вы это забег с Теремков на Дорогожичи посредством метро! Потомушо по
пробкам предновогоднего Киева на такси за это время фиг доберешься!

Представляете себе как смотрели пассажиры на странную расхристанную краснолицую
женщину, которая ссыпалась по эскалатору и металась по вагону, видимо пытаясь
ускорить процесс... И вот это вот всё!

Спасибо Деду Морозу, который попридержал программу, мужественно под водочку
дооолго вызывая Снегурочку.. Накидались все изрядно за её здоровье, а появилась
без четверти таки Баба-Яга! Но всем уже было весело, да и Яга попалась
юморная-Яги бывают и добрыми, чтоб вы знали.

А ы полночь-ТАДАМММ-после салюта, Яга превратилась в Снегурочку! И было радостное
белоснежно-нежное па-де-де, и восторги, и комплименты; и новогодняя ночь
покатилась своим чередом.

Но это  уже совсем другая история...

Татьяна Вакуленко

\ii{10_12_2021.fb.fb_group.story_kiev_ua.1.istoria_kiev.cmt}
