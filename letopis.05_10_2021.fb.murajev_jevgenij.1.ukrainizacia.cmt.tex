% vim: keymap=russian-jcukenwin
%%beginhead 
 
%%file 05_10_2021.fb.murajev_jevgenij.1.ukrainizacia.cmt
%%parent 05_10_2021.fb.murajev_jevgenij.1.ukrainizacia
 
%%url 
 
%%author_id 
%%date 
 
%%tags 
%%title 
 
%%endhead 
\subsubsection{Коментарі}

\begin{itemize} % {
\iusr{Елена Елена}

В главном не согласна. Гос языков должно быть несколько!!! Если в регионе более
15\% говорит на языке - он должен быть региональным. Если в государстве, в более
чем 15\% регионов принят язык - он должен стать государственным. Только так.

Украинизация, как и прочая ...языкозация- это современный способ манипуляции в
интересах определённых политических групп.

А вот действия различных общин по развитию своего языка и культуры- похвальны,
как и действия государства, направленные надружбу народов, проживающих в нем.

Именно с таких слов, как ваши начинался геноцид русских в Украине, уж извините
за прямоту.

\begin{itemize} % {
\iusr{Елена Елена}
\textbf{Евгений Мураев} интересно Ваше мнение по поводу нескольких гос языков

\iusr{Евгений Мураев}
\textbf{Елена Елена} 

шутку в адрес абсолютно русскоязычного человека оценил). Наверное я один из
немногих, кого не заставили отказаться от родного языка. Политики от простых
граждан, если они хотят решить проблему, а не спекулировать на чувствах
избирателей, должны предлагать то, что реально сделать, а не «пипл хавает». Нам
30 лет обещали два государственных, а по факту мы получили полную дерусификацию
и поражение в правах русскоязычных. Хотите наступить на грабли ещё раз- вперёд!
Это не сделали до 2014-го, а без Крыма и Донбасса сделать не реально (-7 млн
голосов). За сказками к Рабиновичу, ко мне за реалиями. Региональные языки
решат проблему на данный момент времени. После изменения территориального
устройства страны и реинтеграции территорий можно будет говорить о большем. И
только тогда, остальное- пустые иллюзии, пограничные с глупостью.

\iusr{Anatoliy Skhidnyak}
\textbf{Yevgeniy Murayev}, как Вы определяете РОДной язык?

\iusr{Жанна Попова}
\textbf{Евгений Мураев}
А на каком языке вы видите высшее образование после введения регионального языка?

\iusr{Елена Елена}
\textbf{Евгений Мураев} 

спасибо, рационально. Ещё не забываем тот факт, что дерусификация в принципе
невозможна. По крайней мере до тех пор, пока вокруг нас русскоязычный рынок с
огромными возможностями. То есть независимо от желания различных кременей,
дефакто будут существовать региональные языки - русский так точно.

Конечно, это всего-лишь один из аспектов политики государства в целом.

Умный политик безусловно будет говорить с электоратом на языке этого самого
электората, а не наоборот.

А пока мы с руководством страны говорим на разных языках светлого будущего не
будет.


\iusr{Николай Раимов}
\textbf{Елена Елена} 

"Поезд ушел" с уходом Крыма и Донбасса. Но ведь в этом появился и огромный "+":
он стал уже...........................французским
дворянским.......................

\iusr{Елена Елена}
\textbf{Николай Раимов} таки да)

\iusr{Николай Раимов}
\textbf{Евгений Мураев} Это называют "трезвый политик"! А "огурец на ладошке рубать", поедете в Запорожье?

\end{itemize} % }

\iusr{Ирина Попеску}

Так он за украинизацию, но более приемлемую? Ещё один повод не голосовать за
Мурайку...

\begin{itemize} % {
\iusr{Жанна Попова}
\textbf{Ирина Попеску} я бы с вашей фамилией не рисковала коверкать чужие !!!!

\iusr{Ольга Подщанская}
\textbf{Ирина Попеску} . Только дети в детском саду и начальной школе коверкают фамилии. Не надо писать ерунду, если вы не знаете человека.

\iusr{Ирина Попеску}
\textbf{Ольга Подщанская} ну мы незнакомы с ним @igg{fbicon.face.tears.of.joy} , анализирую и слушаю то что он говорит. И только. Чего наезжать...

\iusr{Ольга Подщанская}
\textbf{Ирина Попеску} . Вы из детсада!

\iusr{Жанна Попова}
\textbf{Ольга Подщанская}
Нет, она с помойки!!

\iusr{Ольга Подщанская}
\textbf{Ирина Попеску} . Не насилуют на кухне. Но школы русские закрыли полностью, предмета «русский язык» нет, а русская литература изучается в мизерном объёме как зарубежная литература. А на кухне- пожалуйста!

\iusr{Ольга Подщанская}
\textbf{Ирина Попеску} . Вы просто дрянь, Пеписка!

\iusr{Ирина Попеску}
\textbf{Ольга Подщанская} хороший аргумент!

\iusr{Ирина Попеску}
\textbf{Ольга Подщанская} западный регион многоязычен, если вы не знали. Ладно...

\iusr{Ирина Попеску}
Ждём когда МураЕв предоставит русскому языку госстатус и откроет школы с руским языком преподавания на востоке. Ждём не дождёмся. @igg{fbicon.wink}  @igg{fbicon.beaming.face.smiling.eyes} 

\iusr{Евгений Мураев}
\textbf{Ирина Попеску} хорошо, что Вашу фамилию и смысла кривлять нет) как и дискутировать с человеком, чей уровень развития на уровне детского сада.

\iusr{Генадій Полторацький}
\textbf{Ирина Попеску} дура вы явная!

\iusr{Anton Omelchenko}
\textbf{Ирина Попеску} ИРА иди в попеску

\iusr{Оля Землянкина}
\textbf{Ирина Попеску} , чем сильнее и ярче личность, тем больше псов на неё лают...
От бессилия и зависти...

\iusr{Василь Грищенко}
\textbf{Евгений Мураев} не в бровь, а в глаз.

\iusr{Василий Иванов}
\textbf{Ирина Попеску} что вы плетёте. Вы вообще откуда и кто вам дал право оскорблять людей, кривлять фамилии. Может Вас надо отправить в Приднестровье? Там и выражайте свою точку зрения.

\iusr{Ирина Николаенко}
\textbf{Евгений Мураев} не обращайте внимания, всем мил не будешь.

\iusr{Лариса Палагина}
\textbf{Евгений Мураев} от слова полоскали

\iusr{Лариса Палагина}
\textbf{Ирина Попеску} в польше попеска иное значение чем фамилия.

\iusr{Алена Алена}
\textbf{Ирина Попеску} ты знаешь с твоей фамилии есть с чего посмеяться. Зачем ты оскорбляеш человека он этого незаслужывает. Евгений с добром к вам, а видимо нужно с кнутом. Тогда будите понимать.

\iusr{Татьяна Литвин}
\textbf{Ирина Попеску} Вы свій рівень знань навіть не порівнюйте з Евгеном. Не коментуйте, бо дуже смішно

\iusr{Григорий Щербаков}
\textbf{Евгений Мураев} вот нехотел говорить, что она дура.

\iusr{Артём Ккк}
\textbf{Евгений Мураев} мда, вычеркиваем

\iusr{Тамара Макулова}
\textbf{Ирина Попеску} Высказывание свое мнение, но зачем оскорблять, что говорит о вашей невоспитанности

\iusr{Натали Потапова}
\textbf{Евгений Мураев} на уровне ракушки)))

\iusr{Ольга Медведкова}
\textbf{Евгений Мураев} Евгений, не тратьте свое драгоценное время (откуда оно у вас вообще берется)на ответы всяким попескам, хотя, очень смешно)))

\iusr{Дмитро Ширченко}
\textbf{Евгений Мураев} Так как Вы доходчиво всё объясняете то трудно Вас не понять. И главное всё в точку и справедливо.

\iusr{Александр Козка}
\textbf{Ирина Попеску} А что он не так сказал?

\iusr{Ольга Каваценко}
\textbf{Евгений Мураев} Попеску оно есть Попеску..Что с оно взять??? Только пожалеть или ...поржать.

\end{itemize} % }

\iusr{Анатолий Дещенко}

Хто їх переслідує, що несеш хуйню, скажи що ти засланий Росією казачок.

\begin{itemize} % {
\iusr{Евгений Мураев}
\textbf{Анатолий Дещенко} а имбецилу что не говори, он все равно не поймёт.

\iusr{Анатолий Дещенко}
\textbf{Евгений Мураев} 

ты первый раз соврал, когда в президенты болатировался, партию скинул, и сам снял
кандитатуру, а люди тебе поверили, надо было идти до конца, а теперь твои,,
гастроли ,,руського мира ,кто поведеться, а говорить ты умеешь, только для
имбецил, народ тебя раскусил, в тебя безнес, зачем тебе политика, делай людям добро
и тебя увидят, а рассказывать, какой Зеленский, мы все знаем, но это не
Порошенко, делай выводы, займись делом.

\iusr{Светлана Малик}
\textbf{Анатолий Дещенко} 

а когда это было, что-то не припомню? Насколько я помню в президенты
баллотировался Вилкул, а Мураев был в его команде. Я в первом туре и голосовала
за Вилкула, потому что он обещал сделать Мураева премьером.

\iusr{Евгений Деревянко}
\textbf{Анатолий Дещенко} рагуляка,чого лаєшся?
Крім тебе тут порядні українці спілкуються.

\iusr{Анатолий Дещенко}
\textbf{Светлана Малик} 

Мураев скинул свою партию, и снял свою кандидатуру в пользу этого Вилкула.

\iusr{Светлана Малик}
\textbf{Анатолий Дещенко} 

ему не президентом, а премьером надо быть! Только толковый премьер может
навести в стране порядок, а президент только летает из страны в страну и
торгует лицом. Хотя после нынешней власти я не знаю сколько времени нужно
наводить порядок...

\iusr{Dima German}
\textbf{Анатолий Дещенко} Поменяй аватарку .

\iusr{Алена Алена}
\textbf{Dima German} 

а зачем ты человека оскорбляеш он незаслужывает такого. Это не означает то, что
тебе нравиться баррель. Вот так то Герман будь по осторожнее.


\iusr{Dima German}
\textbf{Алена Алена} не заслуживает пишется раздельно и какой баррель ? Я никого и не думал оскорблять .

\end{itemize} % }

\iusr{Игорь Турчик}

Любовь к языку можно привить только любовью!
Запретами на родную речь, дискриминационными мовными законами можно вызвать обратный эффект - ненависть.
Может именно этого они и добиваются!??

\begin{itemize} % {
\iusr{Татьяна Дягилева}
\textbf{Игорь Турчик} я так понимаю, что эти меры ориентированы на детей и молодежь, а те реагируют так, как и предполагалось и впитывают все постулаты современной идеологии с языком, вышиванками, ненавистью ко всему русскому
\end{itemize} % }

\iusr{Николай Абрамов}

Украина многонациональная страна ,ее граждане не делятся по сортам ,
вероисповеданию , религиозным убеждениям , мы один народ. Тут наша земля 

\begin{itemize} % {
\iusr{Лощенов Владимир}
\textbf{Николай Абрамов} В том то и дело, что делится. Не делился б и проблем было б на порядок меньше.

\iusr{Александр Мартыщенко}
\textbf{Николай Абрамов} к сожалению после Майдана, людей делят на правильных и неправильных, одних можно убивать без суда и следствия, другие могут убивать и не быть подсудными! Запад выписал лицензию на убийства!

\iusr{Николай Абрамов}
Из нас хотят сделать вечноскорбящих баранов .

\iusr{Евгений Кожевников}
\textbf{Николай Абрамов} уже не ваша. Парламент 259 голосами поддержал законопроект № 2178-10 об обороте земель сельхозназначения в целом. Это еще в марте.
\end{itemize} % }

\iusr{Marina Glyva}

А я считаю, что государственным языком должен быть и русский тоже! В первую
очередь русский, а вторым государственным - украинский. Это будет справедливо,
ведь русский язык возник на территории современной Украины на шесть столетий
раньше украинского и он является родным языком для половины населения страны.
Украинский сделали государственным потому, что у страны должен быть свой,
отличный от других народов и стран, язык, только мало кто знает, что само слово
"мова" - польское, как и украинские слова:"вежа", "червоний", "брама" и еще 500
других слов. Плохо, что в Украине люди свою историю забыли. Бог создал землю
для всех людей, независимо от национальности и языка, а люди уже придумали
создавать государства и государственные языки, хорошо бы было, если бы Украина
была страной для людей и люди имели право говорить, учиться и работать на
родном русском языке, а не наоборот: люди - рабы государства и в ущерб своим
правам и своей родной культуре должны подчиняться диктатуре чиновников.

\begin{itemize} % {
\iusr{Serge Kozy}
6 столетий назад русский был совсем другой, был древне русский.. с другими словами и буквами

\iusr{Marina Glyva}
\textbf{Serge Kozy} Язык - живой, как и все живое в этом мире, он тоже развивается и меняется. Русский язык 19 века отличается от русского языка 21 века, есть устаревшие слова и неологизмы.

\iusr{Aleksandr Khmelnytskyi}
\textbf{Serge Kozy} ты про ходу с самим собой))) она же ведь не знает поди историю... Если пишет такой бред

\iusr{Надежда Сурай}
\textbf{Serge Kozy}. А украинский какой был шесть столетий назад?

\iusr{Валентина Гаркава}
\textbf{Надежда Сурай} польский

\iusr{Serge Kozy}
Такого не было, у украинского буквы свои появились в 19 веке..

\iusr{Михаил Максимов}
\textbf{Serge Kozy} современный литературный украинский - от Котляревского, современный литературный русский - от Пушкина.
Посмотрите кто писал раньше  @igg{fbicon.smile} 
Про уровень грамотности в Гетьманщине и Московском царстве - тоже поищите. И про то где раньше университеты-академии открывали.

\iusr{Виктория Козюберда}
\textbf{Marina Glyva} Как вы правильно сказали, конечно же русский и только,на этом точка. А второй конечно же украинский, потому как, потому...

\iusr{Виктория Козюберда}

Много писать прийдёться, и тем более что укр язык, здесь на востоке страны особо
то и главный, в основном русский в городах, а в деревнях укр.

\iusr{Наталья Нуждина}
\textbf{Надежда Сурай}
Шесть столетий назад украинского языка вообще не существовало.
А современный украинский смесь многих языков, и польский и венгерский и ...

\iusr{Serge Kozy}
\textbf{Михаил Максимов} что вы вдоль, да около ходите? где поискать, в укр википедии ? Спасибо, не надо. :)))

\end{itemize} % }

\iusr{Инна Колягина}

Если бы искусственно не раздували, то никакой проблемы и нет. И украинский язык
нужно развивать - дети в школе учат и дома начинаешь понемногу разговаривать,
есть качественый перевод фильмов. И русский язык великий и красивый, и зря, что
запретили в школе, теперь ребенок русский текст с трудом читает, путается.
Любая культура - это прежде всего культура, а не месть языку за то, что
агрессор на нем разговаривает. Но заставлять меня говорить так, как кому-то
угодно - это перебор, никакого украинского не захочется. С такой логикой, так
почему у нас гарант извините носит имя агрессора? Пусть переименуется. Так до
маразма дойдем!

\begin{itemize} % {
\iusr{Жанна Попова}
\textbf{Инна Колягина} а зачем вкладывать деньги в развитие укр языка? Где он вам пригодится? Для чего нужен примитивный, хуторской говор вашим детям?

\iusr{Надежда Сурай}
\textbf{Жанна Попова}. Молоток, Жанна, поддерживаю!

\iusr{Инна Колягина}
\textbf{Жанна Попова} Зачем провоцировать конфликт на почве культуры и языка? Это национальные ценности. Любую культуру нужно уважать и не уподобляться националистам. Развитие всегда лучше деградации

\iusr{Elena Surgina}
\textbf{Жанна Попова} Настоящий укр язык очень красивый, но только тогда, когда в нем нет примеси молдавских, румынских, польских и прочих слов. И учить его надо, только без насилия. Иначе пропадает всякое желание, скорее наоборот...

\iusr{Жанна Попова}
\textbf{Инна Колягина}
На уровне фольклора, пожалуйста. Пойте, танцуйте сколько угодно. А в жизни зачем? Чтобы работать кассиром в Сильпо? Или в чем его ценность?

\iusr{Жанна Попова}
\textbf{Elena Surgina}
Кто сказал, что этот сельский говор красивый и для чего его нужно учить??? Какая от него польза??? Чтобы читать Ницой в оригинале???

\iusr{Elena Surgina}
\textbf{Жанна Попова} Чтобы читать Шевченко в оригинале... Мне вас жаль...

\iusr{Жанна Попова}
\textbf{Elena Surgina}
Себя пожалейте и детей своих. Шевченко всю прозу писал на русском. Да и вы почему-то не на соловьиной пишете?? Она же такая красивая!!!

\iusr{Elena Surgina}
\textbf{Жанна Попова} не писала тому, що ви писали російською і я вам теж відповіла російською. І мене жаліти не треба, тому що я вільно володію і російською, і українською мовами, і не тільки... Миру та добра вам!!!

\iusr{Елена Снегирёва}
\textbf{Жанна Попова} 

Жанна, в Днепропетровске, практически всё население говорит на русском языке.
Поверите ли Вы в то, что во всех сёлах и пригородах вокруг мегаполиса, народ
говорит на украинском? А, ведь это так и есть. И это тоже сложилось
исторически. Прикажете не уважать этих замечательных людей? Умных, добрых,
красивых, щедрых! И делить нас на сорта? Это не Западная Украина, это вот,
рядом, соседи. О чём Вы вообще?

\iusr{Жанна Попова}
\textbf{Елена Снегирёва}

Что значит не уважать? Они говорят на суржике, так и мне нужно перейти на
суржик, чтобы сёла не обидеть ? Или я им запрещаю так разговаривать? Что вы,
вообще, несете?? Вы не понимаете проблему или притворяетесь???

\end{itemize} % }

\iusr{Лариса Дорганова}

на неподконтрольной Украине территории Донбасса, в Макеевке, в 6-м классе
украинский язык один(!) урок в неделю

\begin{itemize} % {
\iusr{Alla Poliakova}
\textbf{Лариса Дорганова} Позор! В Донецке и Луганске - есть укр. яз. в школе, а в Украине - ни ОДНОГО урока русского языка / в неделю, или месяц , или год. Предлагают попросить церковь организовать воскресную школу. ,хи,-хи. - но ведь и для этого нужны учебники, а их НЕТ,!

\iusr{Алекс Буравлёв}
\textbf{Лариса Дорганова} много...

\iusr{Igor Markov}
\textbf{Лариса Дорганова} много. Он ни к чему здесь.

\iusr{Лариса Дорганова}
\textbf{Igor Markov} а английский?

\iusr{Igor Markov}
\textbf{Лариса Дорганова} есть

\iusr{Лариса Дорганова}
\textbf{Igor Markov} А он вам нужен? Зачем?

\iusr{Igor Markov}
\textbf{Лариса Дорганова} а вы серьезно считаете, что украинский и английский - языки одного порядка?

\iusr{Лариса Дорганова}
\textbf{Igor Markov} я считаю, что таким образом ваши власти ограничивают выезд молодёжи на учёбу в Украину, а в ЕС или США можно выезжать на учёбу?

\iusr{Igor Markov}
\textbf{Лариса Дорганова} а причём тут уроки мовы в школах и выезд на учёбу???

\iusr{Лариса Дорганова}
\textbf{Igor Markov} 

потому что из вашего гетто надо бежать хоть куда-нибудь, будущего у вас нет, РФ
вас никогда не признает и не присоединит, в Украину вы сами не хотите, поэтому
вашей молодёжи надо выезжать на учёбу, определяться с профессией и своим
будущим, мои родственники с детьми 4 и 6 класс будут уезжать как только умрут
пожилые родители

\iusr{Igor Markov}
\textbf{Лариса Дорганова} без ваших советов разберемся, как нам жить.

\iusr{Igor Markov}
\textbf{Лариса Дорганова} ваши родственники могут делать, что хотят.

\iusr{Лариса Дорганова}
\textbf{Igor Markov} картошка то почём нонче в Донецке?

\iusr{Igor Markov}
\textbf{Лариса Дорганова} какое это имеет отношение к вопросу о языке? И вы считаете цену на картошку реальным поводом к радикальным решениям? Кастрюля не жмет?

\end{itemize} % }

\iusr{Владимир Клюшниченко}
Согласен с вами я тоже так говорю и против насилия

\iusr{Светлана Цоновская}

Украинский язык знаю с детства, Мама была преподавателем украинского, но
сегодня, как и у Евгения нет желания говорить на украинском языке,хоча в свій
час я інколи навіть думала украінською мовою. Любое насилие порождает протест.
Когда начинались события в Крыму, мои знакомые там объясняли своё желание
перейти в Россию тем, что боялись , что их заставят разговаривать на украинском
языке. Я пыталась доказать обратное, теперь я вижу, что они оказались правы

\begin{itemize} % {
\iusr{Gena Bak}
Не порть украинский ! Крапкы та зупынкы !

\iusr{Татьяна Бабийчук}
\textbf{Светлана Цоновская} Я сама с Черновицкой области. Мама читала
украинский в школе. Дома разговаривали на украинском. Но когда насильственный
«переход» ввели. И когда фарион, дрозд, ницой и иные пламенные патриоты , стали
каждому русскоязычному торжественно вручать чемоданы...Я перешла на русский
язык. И любой, орущий «мова» у меня вызывает подозрение и «оскому». Но... даже
тех, кого знаю в лицо, кричащих с пеной у рта «мова», я не могу отнести к
умному, разумному, думающему человеку. IQ явно не высок. Кукувейтики, патриоты
чемоданов, пройдите тест Айзенка, а уж после открывайте свой пенный рот.

\iusr{Наташа Иванова}
\textbf{Татьяна Бабийчук} аналогично. к мове раньше было нейтральное отношение. теперь мандраж. бесит их тупое "мы жывэмо в украйини тому вси мають говорыты украйинською". с какого хе ра???
\end{itemize} % }

\iusr{Нина Ворушило}

Вот пока, вы будете считать, что на Украине государственным может быть только
украинский, на выборах вы не получите голоса, а страна дальше будет
распадаться.

\begin{itemize} % {
\iusr{Ирина Попеску}
\textbf{Нина Ворушило} приятно слышать такое
\end{itemize} % }

\iusr{Ирина Лагутина}

Всё перечисленное уже было в Украинской ССР. И УКРАИНСКАЯ киностудия Довженко,
и театры украинской (и отдельно русской) драмы и двуязычие в школах... Сначала
всё разрушили с обретением "незалежностi" "самостiйностi" , уничтожили ложью
базовое - культуру личностную, намеренно уничтожая русское, насаждая
насильственную украинизацию, а теперь не знают, как избавиться от собственно
рукотворных проблем. Ничего не изменится, если не будут восстановлены равные
права граждан. В ряде стран 2, а то и 3 гос языка и все уживаются без проблем,
а Украине резко понадобилось выпендриться на смех всему миру. Рванёт это
непременно очередными неприятными событиями...

\iusr{Любовь Лещенко}
Какая разница, на украинском, русском, французском, просто будьте людьми. От этого проблем не станет меньше и заводы не откроются.

\iusr{Юлия Яцук}
На Украине должно быть 2 государственных языка.

\begin{itemize} % {
\iusr{Anatoliy Skhidnyak}
\textbf{Юлия Яцук}, а че не 8?

\iusr{Жанна Попова}
\textbf{Anatoliy Skhidnyak}
А чтобы таких, как вы не было.

\iusr{Anatoliy Skhidnyak}
\textbf{Жанна Попова}, так Вам "два языка", как у мутанта или " чтобы таких, как вы не было"?
\end{itemize} % }

\iusr{Роман Грабчак}
мураев красиво расказует а толку нет

\begin{itemize} % {
\iusr{Татьяна Литвин}
\textbf{Роман Грабчак} Будет, когда он станет у власти.

\iusr{Роман Грабчак}
\textbf{Татьяна Литвин} посмотрим
\end{itemize} % }

\iusr{Инна Вишня}

С какой радости те кто говорит по русски, должны говорить на украинском,сейчас
школы на украинском языке, теперь они не знаю ни русский язык ни украинский
язык, потому что, кому-то так захотелось.


\iusr{Ирина Роман Дрогобецька}
Мураєв найкращий

\iusr{Таша Таша}

Никто не думает о культуре и развитии народа Все больше вгоняя людей в нищету
Когда люди начинают думать только о том как выжить И выплескивают свою злость
отчаянья друг на друга Подогреваемую манипуляциями , которые разрушают все ,
ещё больше

\iusr{Раиса Ковалик}
с детства говорила украинским сейчас не навижу, политики постарались.

\iusr{Елена Клименко}

Евгений, добрый день! У Вас, большой потенциал и будущее! Много
единомышленников, среди народа. Вас мы знаем, следим за Вами @igg{fbicon.wink} , Но, меня
интересует Ваша команда, по фамилиям,практика доказала что один в поле не воин,
на политической арене должна быть группа профессионалов во всех областях!

\begin{itemize} % {
\iusr{Татьяна Литвин}
\textbf{Елена Клименко} Думаю у него будет сильная команда. Он это не ЗЕ.

\iusr{Елена Клименко}
\textbf{Татьяна Литвин} возможно, хотелось бы с ней познакомиться.
\end{itemize} % }

\iusr{Лисса Биллионс}
Круто всё сказал, согласна со всем  @igg{fbicon.thumb.up.yellow}{repeat=4} 

\iusr{Ivan Trifonov}

Борьба с русским языком в Украине, это не борьба с Россией, а борьба со своими
русскими гражданами Украины, это их притеснение и лишение права на родной
русский язык!

\iusr{Евгений Кожевников}
почему пост не на латинице?  @igg{fbicon.smile} 

\begin{itemize} % {
\iusr{Татьяна Литвин}
\textbf{Евгений Кожевников} С вас пример взяли, вы ведь тоже вопрос задаёте не на латинице!  @igg{fbicon.face.grinning.squinting} 

\iusr{Евгений Кожевников}
\textbf{Татьяна Литвин} на латинице и мовой было бы веселее, Таня!  @igg{fbicon.smile} 
\end{itemize} % }

\iusr{Сергей Рябуха}

Старательно выискивалось все, что может разделить народ - язык, вера, прошлое,
политические взгляды... С верой не очень, даже "томос" не сильно помог.
Политика? Тоже не очень - националисты не сильны в аргументации и диалоге. Вот
подраться, оскорбить, спеть - это их стихия. С прошлым тоже проблема - в память
не залезешь, ее не снесешь, как памятник. А язык - штука публичная, с ним
попроще - запретил да и всего делов. Дураки найдутся - поддержат...


\iusr{Светлана Корзинова}
БРАВО!!! БРАВО!!! БРАВО, Е. Мураев!!!

\iusr{Алексей Гришкин}
Два языка должны быть державними, остальные-на уровне официальных

\iusr{Vera Vera}
Голосовала всегда за Мураева, теперь буду думать

\begin{itemize} % {
\iusr{Татьяна Литвин}
\textbf{Vera Vera} и напрасно, хотя это ваше право.
\end{itemize} % }

\iusr{Надежда Сурай}
Пусть мне кто-то обьяснит, почему нельзя взять пример с Канады или Швейцарии. Я о языках.

\begin{itemize} % {
\iusr{Anatoliy Skhidnyak}
\textbf{Надежда Сурай}, для того чтобы взять пример "с Канады" надо уничтожить или загнать коренное население в резервации, а с Швейцарии - надо чтобы был швейцарский язык и соответствующий этнос. Какой вариант вам Вам больше нравится?

\iusr{Надежда Сурай}
\textbf{Anatoliy Skhidnyak}. Из предлагаемой вами хрени и чуши - никакой.

\iusr{Anatoliy Skhidnyak}
\textbf{Надежда Сурай}, 

разве индейцы в резервациях, это чушь? Это реальность для "последнего из
могикан" Ф. Купера. Не говоря уже о Швейцарии, для того чтобы брать из нее
пример нужно что бы, имеющие деньги, согласились хранить их в Украине и никакой
хрени типа: "Я о языках". Похоже Вам этого не понять.

\iusr{Wladimir Zherdev}
Неужели не понятно, что русский - язык России! И это основная причина!

\iusr{Anatoliy Skhidnyak}
\textbf{Wladimir Zherdev}, "не понятно", "основная причина" чего?

\iusr{Wladimir Zherdev}
\textbf{Anatoliy Skhidnyak} ненависти определённых украинцев

\iusr{Anatoliy Skhidnyak}
\textbf{Wladimir Zherdev}, к кому?

\iusr{Wladimir Zherdev}
Толя, от€би\$ь
\end{itemize} % }

\iusr{Дима Семенов}
Хвороба мозга!!! Нет никакого вируса, есть, например, химтрейлы. И никто об этом не говорит

\iusr{Николай Баряхтар}
Мураев очень прав!!!

\begin{itemize} % {
\iusr{Liubov Datsenko}
\textbf{Николай Баряхтар} в чем??

\iusr{Николай Баряхтар}
\textbf{Liubov Datsenko} Практически во всём!!!
\end{itemize} % }

\iusr{Александр Крамарь}
100\% Евгений, поддерживаю!!

\iusr{Диана Баюанская}
Евчений Вы образец культуры,ума и красоты в политике Украины.

\iusr{Михаил Ерогов}
Спасибо.

\iusr{Сергей Карасев}

Вот именно миллионы русскоязычных граждан, а вы говорите статус
регионального.... Как же так... Как минимум и русский язык должен быть
государственным. Крым вам как пример, где три государственных языка.

\begin{itemize} % {
\iusr{Anatoliy Skhidnyak}
\textbf{Сергей Карасев}, как это русскоязычных? Это после пересадки органов? Кто донор?

\iusr{Жанна Попова}
\textbf{Anatoliy Skhidnyak}
Кто вас выпустил из психушки? Вам ещё рано к людям.

\iusr{Anatoliy Skhidnyak}
\textbf{Жанна Попова}, если ваш язык, как часть организма, от сверчка до человека, и средство общения от песни сверчка до "ленгвич " - мовы человека, обозначается одним и тем же знаком, то вам еще не только "рано к людям", в виду развития вашего средства общения, вам рано даже в "психушку", только под печку петь дуэтом на языке.

\iusr{Жанна Попова}
\textbf{Anatoliy Skhidnyak}
Я , таки, не ошиблась. Читайте выше.

\iusr{Anatoliy Skhidnyak}
\textbf{Жанна Попова}, "не ешьте на ночь интеллигентов", БЕСсонница замучает, обусловлена сомнениями

\iusr{Жанна Попова}
\textbf{Anatoliy Skhidnyak}
Я так понимаю, что умом вы никогда не отличались, а как надели вышивайку, так совсем мозги набекрень. Жаль вас.

\iusr{Anatoliy Skhidnyak}
\textbf{Жанна Попова}, что Вы еще "понимаете" по теме?

\iusr{Anatoliy Skhidnyak}
\textbf{Жанна Попова}, "я так понимаю", что Вам к лицу больше паранджа, а не "вышиваЙка"
\end{itemize} % }

\iusr{Виталия Козина}
30 лет одно и тоже! Каждая политическая сила усаживается на тему языка и начинаются спекуляции для привлечения электората.

\iusr{Аркадий Бородкин}

Было бы правильнее употреблять словосочетание "говорить (читать, писать)
по-русски" (по-украински, по-английски), или "разговаривать на русском языке"
(на украинском, на английском и т.д.).

\iusr{Владимир Клюшниченко}

А вообще проблема языка это тупо дутая проблема . Как и та что Путин нападет им
просто надо раскол общества и держать его в страхе приследуя при этом свои
меркантильные интересы . Ведь заняться экономикой страны подъемом производства
и создания конкурентноспособной продукции это сложно и не для них . Ведь
формула очень проста оборота производство товар деньги плюс налоги - каждая
заработная гривна тратится в стране при этом с нее идет налог с предприятий с
заработной платы с покупки и продажи потребительских товаров чем больше оборот
тем более богаче страну и более зажиточнее народ . Но мы с поразительно
скоростью и неведомой силой теряем свои рынки в мире уступая их другим даром и
говорим о успешной политике превращаясь в колонию и свалку мира. Пора уже знать
потерянные ниши на рынке пустовать не будут их обязательно займут и потом
уединиться обратно будет практически нереально . После победы над фашизмом
война незакончилась она продолжается это гибридная война за рынки и влияние в
мире и гонка вооружения для того чтобы усилить свое влияние и диктовать свои
условия на рынках только и всего


\iusr{Ксения Костенко}

Я- русскоязычна. Простите меня, мои"правильные" сограждане, но десятки лет я не
слышала украинской речи и привыкла думать на русском языке. Да, да! В Украине
есть русскоязычные регионы. Мне очень жаль, что это , новость для вас.

\iusr{Aleksander Jacek}

В Польше регионы, в которых проживают национальные меньшинства, имеют полное
право использовать собственный язык. Например, Кашубия на севере Польши. В
Польше проживает около 200 000 Кашубов. Местное кашубское радио финансируется
государством. Государство поддерживает национальные меньшинства, финансирует
местные культурные организации, издательства на местных языках и многое другое.
Хотя Кашубы воевали с поляками в составе немецкой армии во время Второй мировой
войны. Как и Силезцы, живущие на юге Польши.

\ifcmt
  ig https://scontent-mia3-1.xx.fbcdn.net/v/t39.30808-6/244451810_4450478828323586_6237667843753897743_n.jpg?_nc_cat=108&_nc_rgb565=1&ccb=1-5&_nc_sid=dbeb18&_nc_ohc=gIfob98gyuAAX8Db-4V&_nc_ht=scontent-mia3-1.xx&oh=a44be7b7a2defe99c13818fe6a386f48&oe=6161EF7A
  @width 0.4
\fi

\begin{itemize} % {
\iusr{Евгений Кожевников}
\textbf{Aleksander Jacek} в Татарстане тоже самое. Вот указатель в метро. Уличные тоже на двух языках

\ifcmt
  ig https://scontent-mia3-2.xx.fbcdn.net/v/t39.30808-6/244179413_1989226351239808_2651607875254318762_n.jpg?_nc_cat=109&_nc_rgb565=1&ccb=1-5&_nc_sid=dbeb18&_nc_ohc=CIiqFDEQUuoAX968Pou&_nc_ht=scontent-mia3-2.xx&oh=85e1a4db8255461a0a8099ef4781a240&oe=6161F29B
  @width 0.4
\fi

\iusr{Евгений Кожевников}

\ifcmt
  ig https://scontent-mia3-2.xx.fbcdn.net/v/t39.30808-6/243499264_1989226824573094_1841608925435078962_n.jpg?_nc_cat=102&_nc_rgb565=1&ccb=1-5&_nc_sid=dbeb18&_nc_ohc=fLAs7ok5_j4AX_O7wks&_nc_ht=scontent-mia3-2.xx&oh=906ea4d3e63e474648ac3be4721c930a&oe=616384CB
  @width 0.4
\fi

\iusr{Евгений Кожевников}
и никто не париться

\iusr{Евгений Кожевников}
но самое интересное, в 90-х в Татарстане был опыт использования латиницы, но не зашло.

\ifcmt
  ig https://scontent-mia3-2.xx.fbcdn.net/v/t39.30808-6/244207370_1989229204572856_165304003595932415_n.jpg?_nc_cat=102&_nc_rgb565=1&ccb=1-5&_nc_sid=dbeb18&_nc_ohc=XmvMs6I2scQAX9AI6lP&_nc_ht=scontent-mia3-2.xx&oh=a91106cac690fa3fb7850b10d78f775d&oe=61632B82
  @width 0.4
\fi

\end{itemize} % }

\iusr{Artem Zinchenko}
Согласен на все100\%! Русский сейчас - это язык протеста!

\begin{itemize} % {
\iusr{Олег Онищенко}
\textbf{Artem Zinchenko} любое давление вызывает сопротивление. Все русскоязычные знали, знают, либо выучили українську мову. Но этого мало, теперь мову на латиницу переведут, но и этого будет мало, позаимствуют арабскую вязь и т.д. И процесс бесконечности изучения мовы и ее проблемы обеспечен .Просто нужен предлог, дабы держать народ в напряжении и грабить, грабить, грабить под видимостью реформ.
\end{itemize} % }

\iusr{Александр Мартыщенко}

Думаю, что скоро произойдёт раскол в обществе. Как только власть не сможет
обеспечить своих граждан газом, электроэнергией, бензином и продуктами питания
вспыхнет бунт, который будет жестким и скорым на расправу! И будет как в
считалке-кто не спрятался тот и будет отвечать!

\begin{itemize} % {
\iusr{Vera Kozar}
\textbf{Александр Мартыщенко} ...к сожалению у нас очень трусливый и не солидарный народ, думаю , что навряд ли будут бунты.Опять будут терпеть все!!
\end{itemize} % }

\iusr{Nataliya Kaluzhanina}
Спасибо большое вам.

\iusr{Алекс Щеглов}
Простите, просто минутка юмора, которого так не хватает сейчас) Евгений, оказывается, рыбак ещё)) А я не поехал...

\iusr{Екатерина Райлян}
Как всегда грамотно, четко, доступно. Уважаю.

\iusr{Виталий Вав-Таг}
А он так кому то из комментирующих и ответил. А может это не его аккаунт?

\begin{itemize} % {
\iusr{Liubov Datsenko}
\textbf{Виталий Вав-Таг} Ответил, почитайте комментарии.
\end{itemize} % }

\iusr{Андрей Дариенко}
Очень умно.

\iusr{Олег Олег}
Блин ,так же в Крыму говорили уж... Лет так 25... Я так примерно... И шо, и стена...
Не понимаю.

\iusr{Галина Зубова}
Самый умный, адекватный политик на Украине. Вот кто должен вести народ за собой

\iusr{Любов Гайдейчук}
Дуже гарно, чітко, аргументовано відповів

\iusr{Сергей Ельчищев}

Говорит всегда четко, по делу, правильно а главное профиссионально! Но всё
трудней и трудней верить политикам.) Как будет в деле? Неужели опять как у
попередников ?

\iusr{Victoria Chursina}
Два гос. языка. Иначе проблема не решится никогда.

\iusr{Марина Дмитриева}
Мне сейчас очень комфортно жить в Донецке, т.к. не заставляют лелеять на мове)

\iusr{Антон Васин}
Евгений как всегда всё чётко  @igg{fbicon.100.percent} 

\iusr{Владимир Сверба}
Эту мову скоро многие просто возненавидят.


\end{itemize} % }
