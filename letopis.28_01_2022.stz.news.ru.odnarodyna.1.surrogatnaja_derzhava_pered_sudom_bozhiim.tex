% vim: keymap=russian-jcukenwin
%%beginhead 
 
%%file 28_01_2022.stz.news.ru.odnarodyna.1.surrogatnaja_derzhava_pered_sudom_bozhiim
%%parent 28_01_2022
 
%%url https://odnarodyna.org/article/surrogatnaya-derzhava-pered-sudom-bozhiim
 
%%author_id saenko_andrej
%%date 
 
%%tags ukraina
%%title Суррогатная держава перед судом Божиим
 
%%endhead 
 
\subsection{Суррогатная держава перед судом Божиим}
\label{sec:28_01_2022.stz.news.ru.odnarodyna.1.surrogatnaja_derzhava_pered_sudom_bozhiim}
 
\Purl{https://odnarodyna.org/article/surrogatnaya-derzhava-pered-sudom-bozhiim}
\ifcmt
 author_begin
   author_id saenko_andrej
 author_end
\fi

С тех пор как в Киеве произошла так называемая «революція гідності», меня
постоянно призывают гордиться. Иногда прямо. Иногда намекают, мол, если не
гордишься, если тебе не нравится, как мы тут всё обустроили, тогда сам знаешь:
чемодан, вокзал...

Но так уж выходит, что мне всё чаще приходится не гордиться, а стыдиться того,
что происходит в моей стране, в моём городе. А как человек, знающий Бога и
верящий Его слову, я не могу не понимать: за беззаконие, происходящее на нашей
земле рано или поздно нам всем придётся отвечать перед Господом. И ответ может
быть весьма горьким. 

\ii{28_01_2022.stz.news.ru.odnarodyna.1.surrogatnaja_derzhava_pered_sudom_bozhiim.pic.1}

К чему это я? А вот к чему.

На днях депутаты Верховной Рады предложили официально продавать услуги
суррогатных матерей иностранцам. Подан проект закона «Об использовании
вспомогательных репродуктивных технологий и суррогатном материнстве». Инициатор
– глава парламентского комитета по здоровью нации Михаил Радуцкий.

Иными словами, кроме известной с древнейших времён торговли телом, осуждённой
Церковью Христовой как грех перед Богом, появился новый вид подобной торговли.

В связи с новой, скажем прямо, дьявольской затеей, конечно, возникнут
юридические, психические и другие проблемы. И об этом уже пишут. Но мне хочется
сегодня говорить о том, как извращённое явление суррогатного материнства влияет
на бессмертную душу человека. И что думает Русская (а значит, и Украинская)
церковь о данном деянии.

Прежде чем мы вспомним слова уважаемых иерархов и священнослужителей, хочу
напомнить: каждое верующее сердце воспринимает рождение женщиной нового
человека как её участие в осуществлении Божьего замысла. И потому относится к
рождению, к вечной связи матери и ребёнка с величайшим благоговением. Как к
тайне Божьего Промысла. Всякая мать, родившая дитя (настоящая или суррогатная)
будет стоять рядом со своим ребёнком на Суде Всевышнего. Великий пример
материнства церковь видит в Пресвятой Богородице. В Ней раскрывается высочайшее
достоинство женщины и её материнского призвания.

Как же уродливо рядом с этим выглядит духовное падение, которое навязывается
ныне украинскому обществу его законодательной властью.

Недавний скандал с рождёнными для иностранцев младенцами, которых, как живой
товар, из-за карантина не смогли вывезти с Украины, как всегда горячо и точно
оценил митрополит Запорожский и Мелитопольский Лука:

\begin{zznagolos}
«Сейчас очень много говорят о детях, которые были рождены украинскими
суррогатными матерями и которые в связи с карантином \enquote{застряли} в Украине. Но
никто не говорит о проблеме суррогатного материнства вообще. В нашей стране
политики продали всё, что можно было продать. Продали свою совесть, нашу землю,
страну. Ну, а людям остаётся продавать только то, что у них осталось. А своего
у них осталось немного – это они сами».	
\end{zznagolos}

Нет никакого сомнения в том, что по своей сути суррогатное материнство является
аморальным, антихристианским и бесчеловечным явлением. Следовательно,
преступлением перед Богом.

В связи со всей этой суррогатной дрянью проявилось и особое (даже не знаю, как
точно определить) отношение украинской власти к своему народу. Какое-то
холодное бездушие моральных уродов, не знающих жалости, когда в своей
собственной стране человек становится узником большого концлагеря, где с ним
могут сделать всё что угодно.

Вот что ещё пишет владыка Лука: «Фашистские палачи в концлагерях с немецкой
педантичностью извлекали пользу со всего того "человеческого материала",
который находился в их распоряжении. Над людьми проводили медицинские опыты и
эксперименты, тела мёртвых использовали для изготовления мыла, кожаных
переплетов и т. п. Не превращается ли наша страна в такой вот концлагерь, где
захватчики-колонисты выжимают из украинского народа максимальную выгоду для
себя?»

Страшный вопрос и в том, куда, в какие семьи попадает ребёнок рождённый
«суррогатной украинкой». Обычно мы отгоняем тяжёлые предположения. А делать
этого не стоит. Дитя, возможно, с кем-то из нас связанное узами кровного
родства, может попасть в руки извращенцев, которые выдают себя за «семью».
Известны случаи сексуального насилия над младенцами. И даже если этого не
случится, ребёнок остаётся навсегда травмированным нравственно и
психологически. С другой стороны, суррогатная мать (будучи всё-таки не
биологической машиной, а человеком, творением Божиим) может со временем
почувствовать боль и тоску по ребёнку, с которым её разлучили.

Да, мы живём в мире, где сознательно размываются понятия «грех», «зло»,
«преступление». Человек растерян. Однако у него есть, куда обратиться в поисках
правды. И это место –  Церковь Христова.

В 2000 году на юбилейном Архиерейском соборе был принят важнейший документ –
«Основы социальной концепции Русской Православной Церкви». Очень советую его
изучить. В нём, кстати, даны разъяснения по многим вопросам, в которых до сих
пор плутают даже люди верующие. Так вот, в документе Церковь высказалась и о
суррогатном материнстве. Прошу простить мне длинное цитирование, но сказанное
церковью представляется важным и незыблемым:

\begin{zzquote}
«Церковь освящает брак, уподобляя его духовному союзу Христа и Его Церкви.
Плодом супружеской любви становятся дети, рождение и воспитание которых, по
православному учению, является одной из главных целей брака. Серьёзной
проблемой, с которой нередко сталкиваются семьи, является бесплодие одного или
обоих супругов. Церковь сочувствует бездетным супругам, благословляя им
молиться о даровании потомства, обращаться к врачам для лечения бесплодия, а
также усыновлять (удочерять) детей. Допустимым средством медицинской помощи
бездетным супругам Церковь считает искусственное оплодотворение половыми
клетками мужа, если это не сопровождается уничтожением оплодотворённых
яйцеклеток, поскольку оно не нарушает целостности брачного союза, не отличается
принципиальным образом от естественного зачатия и происходит в контексте
супружеских отношений. Что же касается практики так называемого «суррогатного
материнства», то она однозначно осуждается Церковью. «Суррогатное материнство»,
то есть вынашивание оплодотворенной яйцеклетки женщиной, которая после родов
возвращает ребёнка «заказчикам», противоестественно и морально недопустимо даже
в тех случаях, когда осуществляется на некоммерческой основе».	
\end{zzquote}

Всё сказано предельно ясно. Кроме того, ещё как минимум два последствия
узаконенного преступление войдёт в нашу жизнь. Происходит своего рода
десакрализация. Человек в глазах общества меняет свою природу. Он больше не
создание Бога, но обычный «биологический материал». И второе: когда человек
становится объектом купли-продажи, ценность его жизни, его неприкосновенности
сводится к нулю. С ним, как уже говорилось, можно творить всё что угодно.

Отдельный вопрос: можно ли крестить ребёнка, рождённого суррогатной матерью.
Ответ церкви и здесь вполне определён (хотя есть случаи, когда по факту
правящий архиерей может принимать исключительное решение). В целом же ситуация
такова.

Если существует угроза жизни ребёнка, его крестят без всяких условий. Если же с
просьбой покрестить такого младенца обращаются взрослые (родители, крёстные)
крещение допустимо только в том случае, если обратившиеся признаю̀т явление
суррогатного рождения грехом перед Богом и раскаиваются в нём. Если признания
нет, то это просьба людей, отвергающих Бога и Церковь. В таком случае нужно
подождать, пока ребёнок вырастет и сам примет решение о своём вхождении в
Церковь. Разумеется, при этом факт его вынашивания и рождения ни в коем случае
не будет препятствием для совершения таинства.

В заключение приведу слова Святейшего Патриарха Кирилла, которые он написал
Михаилу Мишустину в письме с просьбой поддержать законопроект о запрете в
России суррогатного материнства для иностранных граждан: «Не считаю, что запрет
суррогатного материнства для иностранцев, которое незаконно в большинстве
развитых и многих развивающихся странах, каким-либо образом ущемит право
иностранных граждан на получение медицинской помощи на территории России.
Подобная \enquote{помощь} в настоящий момент оказывается вопреки цели сбережения народа
России, провозглашенной Президентом России В. В. Путиным \enquote{высшим национальным
приоритетом}».

Я далёк от мысли, что сбережение нашего народа хоть как-то заботит украинскую
власть. Больше того, иногда Украина представляется мне несчастной суррогатной
матерью, которой бездушно пользуются её «зарубежные партнёры». А потом, вырвав
из неё самое дорогое и ценное, эти негодяи оставят её пропадать.

И, честно говоря, было бы мне сегодня очень грустно, если бы не моя надежда на
Бога, милость которого я знаю, чувствую каждый день и на которую только и
уповаю.
