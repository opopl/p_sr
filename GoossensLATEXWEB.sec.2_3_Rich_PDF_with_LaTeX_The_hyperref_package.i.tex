
\isubsec{2_3_Rich_PDF_with_LaTeX_The_hyperref_package}{Rich PDF with\ \LaTeX: The hyperref package }

The hyperref package by Sebastian Rahtz \footnote{With considerable help
over the years from many contributors, notably David Carlisle. } derives
from, and builds on, the work of the Hyper\TeX\  project (see Appendix
B.1 on page 403 and []). It extends the functionality of all the \LaTeX\
cross-referencing commands (including the table of contents,
bibliographies, and so on) to produce \verb|\special| commands that a
driver can turn into hypertext links; it also provides new commands to
allow the user to write ad hoc hypertext links, including those to
external documents and URLs. 
 
The package supports a variety of DVI drivers; they use either the
Hyper\TeX\  \verb|\special| commands or, if designed to produce only
PDF, literal PostScript \verb|\special| commands or \pdfTEX-specific
primitives. The commands are defined in configuration files for
different drivers, selected by package options. The following drivers
are supported: 

\begin{itemize}
  \item \textbf{hypertex} For DVI processors conforming to the Hyper\TeX\  guidelines (that is, 
    \verb|xdvi|, \verb|dvips| with the \verb|-z| option, OzTeX, and Textures); 
  \item \textbf{dvips} Writes \verb|\special| commands producing literal PostScript, tailored for 
    dvips. 
  \item \textbf{dvipsone} Writes \verb|\special| commands producing literal PostScript, tailored for 
    dvipsone. 
  \item \textbf{pdftex} Writes commands for Han The Thanh's \ \TeX\  variant which produces PDF 
    directly (see \refsec{2_4_Generating_PDF_directly_from_TeX}). 
  \item \textbf{dvipdfm} Writes \verb|\special| commands for Mark Wicks's DVI to PDF driver 
    dvipdfm. 
  \item \textbf{dviwindo} Writes \verb|\special| commands which Y\&Y's Windows previewer interprets as hypertext jumps within the previewer. 
  \item \textbf{vtex} Writes \verb|\special| commands which MicroPress' HTML and PDF-producing 
    \ \TeX\  variants interpret as hypertext jumps. 
\end{itemize}

Output from \verb|dvips| or \verb|dvipsone| \footnote{
Both these drivers support partial font downloading; it is advisable to turn it off when preparing 
PostScript for Acrobat Distiller, since this has its own system of making font subsets. 
} must be processed using Acrobat Distiller to 
obtain a PDF file. The result is generally prefererable to that produced using the 
\verb|hypertex| driver and subsequent processing with the command \verb|dvips -z|. The 
advantage of a DVI file written using the Hyper\TeX\  \verb|\special| commands is that it 
can also be used with hypertext viewers like \verb|xdvi|. 

\isubsubsec{2_3_1_Implicit_behavior_of_hyperref}{Implicit behavior of hyperref}

The package can be used more or less with any normal \LaTeX\  document
by requesting it in the document preamble. You must make sure it is the
last of the loaded packages to give it a fighting chance of not being
overwritten since its job is to redefine many \LaTeX\  commands.
Hopefully you will find that all cross-references work correctly as
hypertext, unless the implicit option is set to false, in which case
only explicit hyperlink commands will be processed. Options control the
appearance of links and give extra control over PDF output. 

\reffig{2-6} shows the result of processing our test file (see Appendix A.1), with 
hyperref defaults, to PDF; \reffig{2-7} shows the same file displayed using \verb|xdvi|. 
 
%%page page_59                                                  <<<---3
%%page page_60                                                  <<<---3
 
Two commonly used package options are 

\begin{itemize}
  \item \verb|colorlinks|, which colors the text of links instead of putting boxes around 
    them (see \reffig{2-8}, which uses gray scales instead of color); and 
  \item \verb|backref|, which inserts extra ``back'' links into the bibliography for each entry. 
    Figures \reffig{2-9} and \reffig{2-10} show what happens with this
    option set on and off. Note: The \verb|backref| and
    \verb|pagebackref| options can work properly only if there is a
    blank line after each \verb|\bibitem| (as there is if it is created
    by BIB\TeX\ ). 
\end{itemize}


%%page page_61                                                  <<<---3

\isubsubsec{2_3_2_Configuring_hyperref}{Configuring hyperref}

All user-configurable aspects of hyperref are set using a single
``key-value'' scheme 
(using the \verb|keyval| package with the key \verb|Hyp|). The options can be set either in the 
optional argument to the \verb|\usepackage| command or with the command: 

\begin{fminipage}{5in}
\verb|\hypersetup{|\emph{keyvalue pairs}\verb|}| 
\end{fminipage}

Note that optional argument of the package command uses an experimental
extension to \LaTeX's syntax. \LaTeX\ imposes some restrictions on the
detailed content of package options and so this method may not always
work; in general, options that involve only letters, digits, and
punctuation will be safe. 

In addition, when the package is loaded, a file \verb|hyperref.cfg| is read if it can 
be found; this is a convenient place to set options on a sitewide basis. Thus the 
behavior of a particular file could be controlled by: 

\begin{itemize}
  \item A sitewide \verb|hyperref.cfg| setting up the look of links, adding backreferencing, 
    and setting a PDF display default: 

    \begin{verbatim}
    \hypersetup{backref, 
    pdfpagemode=FullScreen, 
    colorlinks=true} 
    \end{verbatim}

  \item A global option in the file that is passed down to \verb|hyperref|: 

    \begin{verbatim}
    \documentclass[dvips]{article} 
    \usepackage{hyperref} 
    \end{verbatim}

  \item File-specific options in the \verb|\usepackage| commands that override the ones set 
    in \verb|hyperref.cfg|: 

    \begin{verbatim}
    \usepackage[pdftitle={A Perfect Day},colorlinks=false]{hyperref} 
    \end{verbatim}
\end{itemize}

Details of all the package options are given in \refsec{2_3_8_Catalog_of_package_options} on page 62. For 
many options you do not need to give a value because they default to the value 
true if used. These are the ones classed as Boolean. The values true and false 
can always be specified, however. 

\ipar{General_options}{General options}

A back-end driver can be chosen using one of the options listed in \reftab{2-4} on 
 62. If no driver is specified, the package defaults to loading the \verb|hypertex| 
driver. Most drivers provide the expected behavior without further ado; if you use 
dviwindo, however, you may need to redefine the following command: 

\begin{verbatim}
\wwwbrowser 
\end{verbatim}

This command tells the \verb|dviwindo| driver what program to launch;
the default is \verb|c:\netscape\netscape.exe|. Thus users of Internet
Explorer might add something like the following to \verb|hyperref.cfg|: 

\begin{verbatim}

\renewcommand{wwwbrowser}{C:\string\Program\space 
  Files\string\Plus!\string\Microsoft\space 
  Internet\string\iexplore.exe} 

\end{verbatim}

%%page page_62                                                  <<<---3

A number of general options, listed in \reftab{2-5} on page 62, apply to all drivers. 
The importance of the \verb|breaklinks| option is demonstrated in \reffig{2-11}; this 
example was processed using \pdfTEX, which has allowed the URL to break, and 
has made each part into separate links. The other drivers are unable to manage this 
trick, and the URL would have to be left protruding into the margin. 

\reftab{2-6} on page 63 lists options that also apply to all drivers but
provide extended functionality. There are various options to specify the
color of text in links. All color names must be defined before use,
following the normal system of the standard \LaTeX\ \verb|color|
package. Users must also realize that the color of colored links is part
of the text; if you color URLs green and then print the page from
Acrobat, for example, the text will be printed in green (that is, a gray
scale on a black-and-white printer). 

\ipar{Using_the_xr_package_with_hyperref}{Using the xr package with hyperref}

A collection of interacting files can be created automatically, using
the \verb|xr| package. However, since either \verb|dvi| or \verb|pdf| versions of the results
may be used, the \verb|hyperref| package does not necessarily know which files
to refer to. Consider the following file: 


\begin{verbatim}
  \documentclass{article} 
  \usepackage{xr} 
  \usepackage{hyperref} 
  \externaldocument{other} 
  \begin{document} 
  See section \ref{facts} in the other file 
  \end{document} 
\end{verbatim}

The label \verb|facts| is defined in the file \verb|other.tex|; when the
current file is processed, it reads other \verb|.aux| and makes all of
its labels available for \verb|\ref| in the current file (this is the
job of the \verb|xr| package). Because we have loaded \verb|hyperref|,
the command \verb|\ref{facts}| creates a hypertext link referring to the
other file. But does it ask for \verb|other.dvi| (which is what we want
if we are using \verb|dviwindo|, for example), or \verb|other.pdf|
(which Acrobat can open directly)? The \verb|hyperref| package does its
best to guess correctly, but on occasion you may wish to override it by
specifying the file suffix with the \verb|extension| option. 

\ipar{Options_specifically_for_making_PDF_files}{Options specifically for making PDF files}

When the target is a PDF file, there are many options to configure the
output; these are listed in \reftab{2-7} on page 63. 

 
%%page page_63                                                  <<<---3
\begin{itemize}
  \item \textbf{Setting link views} Setting the view for links in Acrobat can be complicated; 
    unlike many other hypertext systems, Acrobat associates a magnification or zoom 
    value with every link. \reftab{2-1} on the following page shows the possibilities, that is, 
    a set of keys with a variable number of parameters. Unfortunately it is often rather 
    hard for a hyperref user to work out what values to set for these parameters; they 
    have to be expressed in the PDF default coordinate space, which is not necessarily 
    the same as \ \TeX\  is thinking in. The good news is that \pdfTEX tries to work out 
    sensible values for you, supplying default parameters for the commonly used keys, 
    \verb|XYZ| and \verb|FitBH|; the bad news is that drivers using the pdfmark system do not supply 
    defaults. So, if you say 

    \begin{verbatim}
    \usepackage[dvips, pdfview=FitBH]{hyperref} 
    \end{verbatim}

    you would normally get a catastrophic result, since \verb|FitBH|
    \emph{must} be followed by a number. To make life a little easier in
    practice, \verb|hyperref| supplies a value of -32768 as a parameter \footnote{
    This meaningless value forces Acrobat Distiller to set a usually
    sensible default for some keys.}
    to the view command, if none is explicitly given; this value is ignored
    by the \pdfTEX\  driver. 

    The \emph{default} is always \verb|XYZ| followed by an appropriate value
    for the driver, that is, the magnification does not change when a link
    is followed. A typical change would be to set 

    \begin{verbatim}
    pdfview=FitBH 
    \end{verbatim}

    so that links jump to a view that fills the window with something rational, the width 
    of the text area on the current page. 

  \item \textbf{Coloring links} The color of link borders in Acrobat can
    be specified \emph{only} as three numbers in the range O..1, giving an
    RGB color. You cannot use the colors defined in \ \TeX. These colors
    do \emph{not} form part of the text and will not show when printed. 

  \item \textbf{Setting the window display} The options relating to the window display are 
    demonstrated in \reffig{2-12}; this example was created with 

    \begin{verbatim}

    \usepackage[ 
    pdftoolbar=fa1se, 
    pdfmenubar=fa1se, 
    pdfwindowui=fa1se, 
    pdffitwindow=true, 
    pdfpagelayout=TwoCo1umnLeft 
    ]{hyperref} 

    \end{verbatim}

    %%page page_64                                                  <<<---3

    Because the toolbar and menu bar have been removed, any interaction
    for the user must be provided by the document itself (including
    basics like a \verb|Quit| button); the command \verb|\Acrobatmenu|
    (see Section 2.3.4 on page 47) comes in very useful here.  The menu
    and toolbar can, in fact, be restored manually using Ctrl-Shift-M
    and Ctrl-Shift-B key sequences, respectively. 

\end{itemize}

%%page page_65                                                  <<<---3

\ipar{Acrobat_bookmark_commands}{Acrobat bookmark commands}

The bookmark commands need further explanation. They are stored in a
file called \emph{jobname}\verb|.out|, and you can postprocess this file to remove 
\LaTeX\ codes if needed.  The bookmark text is \emph{not processed} by \LaTeX,
so any markup is passed through literally. In addition, bookmarks must
be written in Adobe's PDFDocEncoding. \reffig{2-13} shows the effect of
the bookmarksopen option and also the limitations of PDFDocEncoding,
since math cannot be displayed. To aid any editing you need to do, the
out file is not rewritten by \LaTeX\  on the next pass if it is edited
to contain the line \verb|\let\WriteBookmarks\relax|. The \verb|hyperref| package does try its best to 
convert internal encoding for European accented characters to PDFDocEncoding. 

\noindent\textbf{PDF document information fields} The options listed in \reftab{2-8} on page 65 
allow you to put text in PDF's information fields. \reffig{2-14} shows the display of 
document information in Acrobat; the document was created with the following 
command in the document preamble: 

\begin{verbatim}

\usepackage[ 
  pdfauthor={Maria Physicist}, 
  pdftitle={Simu1ation of Energy Loss Straggling}, 
  pdfcreator={pdfTeX}, 
  pdfsubject={Energy Loss}, 
  pdfkeywords={physics,energy} 
]{hyperref} 

\end{verbatim}
 
%%page page_66                                                  <<<---3
%%page page_67                                                  <<<---3
 
\isubsubsec{2_3_3_Additional_user_macros_for_hyperlinks}{Additional user macros for hyperlinks}

If you need to make references to URLs or write explicit point-to-point links, the 
following set of user macros is provided. Note that it is possible to differentiate links 
and anchors by category, but the feature is not seriously exploited in \verb|hyperref|. 

\begin{verbatim}
\hyperbaseurl{url} 
\end{verbatim}

A base \emph{url}, prepended to other specified URLs to make it easier to write portable 
documents, is established. 

\begin{verbatim}
\href{url}{text} 
\end{verbatim}

The text is made into a hyperlink to the \verb|url|; this must be a full URL (relative to 
the base URL, if that is defined). The special characters \verb|#| and \verb|~| do not need to be 
escaped in any way. For example, 

\begin{verbatim}
\href{http://www.tug.org/~rahtz/nonsense.htm1#fun}{Some fun} 
\end{verbatim}

\begin{fminipage}{3in}
  \verb| \hyperimage{| \emph{image url} \verb|} | 
\end{fminipage}

The image referenced by the \emph{image url} is inserted. 

\begin{verbatim}
\hyperdef{category}{name}{text} 
\end{verbatim}

A target area of the document (the \emph{text}) is marked and given the name \emph{category.name}. 

\begin{verbatim}
\hyperref{url}{category}{name}{text} 
\end{verbatim}

The text is made into a link to \emph{url\#category.name}. 

\begin{verbatim}
\hyperref[label]{text} 
\end{verbatim}

The text is made into a link to a point established with a normal \LaTeX\ \verb|\label| 
command with the symbolic name \emph{label} (see the following description of \verb|\ref*| for 
a use for this syntax.) 

\begin{verbatim}
\hyperlink{name}{text} 
\hypertarget{name}{text} 
\end{verbatim}

A simple internal link is created with \verb|\hypertarget|, two parameters of an anchor 
\emph{name}, and anchor \emph{text}. The \verb|\hyperlink| command has two arguments: the name of 
a hypertext object defined somewhere by \verb|\hypertarget|, and the text used as the 
link on the page. 

In HTML parlance, the \verb|\hyperlink| command inserts a notional \verb|#| in front of 
each link, making it relative to the current document; \verb|\href| expects a full URL. 

%%page page_68                                                  <<<---3

\begin{verbatim}
\autoref {label} 
\end{verbatim}

This is a replacement for the normal \verb|\ref| command that puts a
contextual tag in front of the reference. The difference is shown in
\reffig{2-15}, where the first section link was made using
\verb|\autoref{...}| and the second using \verb|\ref{...}|. The former
has the word ``section'' as part of the link, whereas the latter has
just the number. The behavior of the former is often more friendly for
users than that of the latter. 

The tag is worked out from the context of the original \verb|\label|
command by hyperref using the macros listed in \reftab{2-2}. The macros can
be redefined in documents using \verb|\renewcommand|; note that some of
these macros are already defined in the standard document classes. The
mixture of lowercase and uppercase initial letters is deliberate and
corresponds to the author's practice.  Sometimes you might want to make
a link text all by yourself and do not want \verb|\ref| and
\verb|\pageref| to form links. For this purpose, there are two variant
commands: 

%%page page_69                                                  <<<---3

\begin{verbatim}
\ref*{label} 
\pageref*{label} 
\end{verbatim}

A typical use would be to write 

\begin{verbatim}
\hyperref[other]{that nice section (\ref*{other}) we read before} 
\end{verbatim}

where we want \verb|\ref*{other}| to generate the right number but not to form a link. 
We will do this ourselves with \verb|\hyperref|. 

\isubsubsec{2_3_4_Acrobat-specific_commands}{Acrobat-specific commands}

If you want to access the menu options of Acrobat Reader or Acrobat Exchange, 
the following command is provided in the appropriate drivers: 

\begin{verbatim}
\Acrobatmenu{menuoption}{text} 
\end{verbatim}

The text is used to create a button that activates the appropriate
menuoption. \reftab{2-3} \footnote{ This table was laboriously derived by
Thomas Merz, who was experimenting with Acrobat Exchange, and published
in Merz (1998), since the names are not listed in Adobe documentation. } 
lists the \emph{menuoption} names you can use. A
comparison of this list with the menus in Acrobat will show what they
do. Obviously some are appropriate only to Exchange. 

As an example, let us add a menu bar in the footer of our document, using the 
\verb|fancyhdr| package: 

\begin{verbatim}

\usepackage{fancyhdr} 
\usepackage[colorlinks]{hyperref} 
\pagestyle{fancy} 

\cfoot{\NavigationBar} 
\newcommand{\NavigationBar}{% 
    \Acrobatmenu{PrevPage}{Previous}~ 
    \Acrobatmenu{NextPage}{Next}~ 
    \Acrobatmenu{FirstPage}{First}~ 
    \Acrobatmenu{LastPage}{Last}~ 
    \Acrobatmenu{GoBack}{Back}~ 
    \Acrobatmenu{Quit}{Quit}% 
} 

\end{verbatim}

The effect is shown in the following picture:
 
%%page page_70                                                  <<<---3
%%page page_71                                                  <<<---3
 
The text can, of course, be any arbitrary \LaTeX\  piece of typesetting.
The following variation uses symbols from the ZapfDingbats font (loaded
with the \verb|pifont| package) for the same set of menu options: 

\begin{verbatim}

\usepackage{pifont} 
\usepackage{graphics} 
\newcommand{\NavigationBar}{{\Large 
  \Acrobatmenu{PrevPage}{\ref1ectbox{\ding{227}}} 
  \Acrobatmenu{NextPage}{\ding{227}} 
  \Acrobatmenu{FirstPage}{\ref1ectbox{\ding{224}}} 
  \Acrobatmenu{LastPage}{\ding{224}} 
  \Acrobatmenu{GoBack}{\ref1ectbox{\ding{249}}} 
  \Acrobatmenu{Quit}{\ding{54}}% 
}} 

\end{verbatim}

with the effect as follows: 

Instead of a Dingbat, we could also have used a picture, with code like 

\begin{verbatim}
\Acrobatmenu{Back}{\includegraphics{backpic}} 
\end{verbatim}

\ipar{2_3_5_Special_support_for_other_packages}{Special support for other packages}

\verb|hyperref| tries to cooperate with as many other package as possible, but
this laudable aim is sometimes impractical. Causes of conflict are 

\begin{itemize}

  \item Packages that manipulate the bibliographical mechanisms. Peter Williams's 
    \verb|harvard| package is supported. However, the recommended package is Patrick 
    Daly's \verb|natbib| package which has specific \verb|hyperref| hooks to allow reliable interaction. This package covers a very wide variety of layouts and citation styles, 
    all of which will work with \verb|hyperref|. 

  \item Packages that typeset the contents of the \verb|\label| and
    \verb|\ref| macros, for example \verb|showkeys|. The \verb|hyperref| package
    redefines all of these commands, unless the \verb|implicit=false| option is
    used; then these packages will not work properly. 

  \item Packages that do anything serious with the index. 

\end{itemize}

The \verb|hyperref| package is distributed with variants on two useful packages
designed to work especially well with it. These are\verb| xr| and \verb|minitoc|, which
support cross-document links using  \LaTeX's normal \verb|\label/\ref| mechanism
and per-chapter tables of contents, respectively. 

%%page page_72                                                  <<<---3
 
\isubsubsec{2_3_6_Creating_PDF_and_HTML_forms}{Creating PDF and HTML forms}

It is fast becoming commonplace (and even necessary) to convert paper forms to 
an electronic equivalent. Many Web pages now use HTML forms to collect data 
in very complicated ways, but it is not widely realized that PDF contains all the 
same functionality. In this section, we look at the support in \verb|hyperre|f for creating 
full-fledged PDF (and HTML) forms. 

Those interested in forms should keep three things in mind: 

\begin{enumerate}
  \item  Fill-in forms are not the only use for form objects in PDF. D. P. Story 
    [ACROTEX] and Hans Hagen [CONTEXT] use them for building sophisticated interactive applications and advanced navigation. \reffig{2-16} shows Hans 
    Hagen's calculator, developed in \ \TeX\  (his CON\TeX T package) with embedded 
    JavaScript and graphics drawn using METAPOST and delivered as PDF. 
  \item  The current powerful forms are a relatively recent addition to PDF; to use 
    them, you need Acrobat 3.01 or later and the Forms 3.5 add-ons. 
  \item Few PDF-generating applications, apart from \ \TeX, really support markupbased creation of PDF forms, and most documentation deals with creating them 
    manually using Acrobat Exchange. It is also a late addition to the \verb|hyperref| 
    package, although the interface may need to change. \emph{Note}: only the pdftex, 
    dvips, and \verb|tex4ht| drivers support forms. 
\end{enumerate}

The excellent book by Thomas Merz is required background reading for
understanding how PDF handles forms (see Merz (1998), Chapters 7 and
10), and Story's Web site [ACROTEX] has both well-designed examples and
a detailed tutorial on using \verb|pdfmark| (see \refsec{2_2_Generating_PDF_from_TeX} on page 27)
to create forms. You should also keep in mind that much underlying
functionality requires use of JavaScript, which you will need to learn.
With the Forms plug-in Adobe supplies a manual called \emph{Acrobat
Forms JavaScript Olject Specification}; Netscape's \emph{JavaScript
Reference Manua}l is also helpful. PDF forms have huge potential, and
\verb|hyperref| only scratches the surface of what they can do.  

The form support in \verb|hyperref|
is designed to mimic that in HTML. To this end, it requires you to put
all your form fields inside a \verb|Form| environment; only one is allowed per
file. 

\begin{verbatim}

\begin{Form}[parameters] 
fields 
\end{Form} 

\end{verbatim}

%%page page_73                                                  <<<---3

The \emph{parameters} are a set of key-value pairs, as listed in \reftab{2-9} on page 65. 

Four types of form fields are supported: 

\begin{enumerate}
  \item  Text fields, that allow free entry of text; 
  \item  Checkboxes, that allow a box to be selected or deselected; 
  \item  Choice fields, that allow the user to choose one of a range of possibilities; and 
  \item  Push buttons, that instigate some action. 
\end{enumerate}

In addition, there are special \verb|Submit| and \verb|Reset| fields. 

The following six macros are used to prepare fields: 

\begin{verbatim}

\TextField[options]{label}
\CheckBox[options]{label}
\ChoiceMenu[options]{label}{cboices}
\PushButton[options]{label}
\Submit[options]{label}
\Reset[options]{label}

\end{verbatim}

%%page page_74                                                  <<<---3

Note that at the top level there is no distinction drawn between 

\begin{itemize}
  \item simple choice menus, where all possibilities are listed; 
  \item pop-up choice menu, where the default is shown, and the rest appear only when 
    the field is selected; 
  \item combo menus, where a list of possibilities is given, but the user can type in a 
    new value; \footnote{This type does not appear to be allowed in HTML.} and
  \item radio fields, where one of a list of possibilities can be checked. 
\end{itemize}

There is a large set of options (see \reftab{2-10} on page 65) that affect what the form 
fields do. 

Making a field involves supplying a textual label, possibly a list of choices, (optionally) a default, and an initial selection. The way a list of choices is presented 
depends on which options are used; the list is simply separated by commas. For 
each item, it is possible to specify a visible string separately from the value actually 
returned, if this choice is made, by supplying two strings separated by an = sign for 
the choice. The first part is what is shown; the second part is the value returned. 

Each of the three main field types (text, checkbox, and choice) consists of two 
parts-the \emph{label} and the \emph{field} itself. The position of the label in relation to the field 
is determined by three macros that you can redefine: 

\begin{verbatim}
\LayoutTextField{label}{field} 
\LayoutChoiceField{label}{field} 
\LayoutCheckboxField{label}{field} 
\end{verbatim}

These macros default to \verb|#1 #2|, that is, the label is set to the left of the field. A 
typical redefinition might be 

\begin{verbatim}
\renewcommand{\LayoutTextField} [2] {\makebox [2in] {#1}#2} 
\end{verbatim}

which would set all labels within a fixed-width box, 2 inches wide. 

What is actually created as the typeset area for the field is determined by 

\begin{verbatim}

\MakeRadioField{width}{height} 
\MakeCheckField{width}{height} 
\MakeTextField{width}{height} 
\MakeChoiceField{width}{height} 
\MakeButtonField{text} 

\end{verbatim}

These macros default to making the field a rectangle \emph{width} wide
and \emph{height} high, with the field contents placed in the center.
The \emph{width} and \emph{height} default to the size of the field
contents but can be overridden with options (see \reftab{2-10} on 65). The
exception is the macro for button fields, which defaults to \verb|#1|;
it is used for push buttons and for the special \verb|\Submit| and
\verb|\Reset| macros. 

You might also need to redefine the following macros; these are used to
work out sizes when no other information is available: 

%%page page_75                                                  <<<---3
 
\begin{center}
  \begin{tabular}{cc}
  
  \emph{Macro}                      & \emph{Default}           \\
  \hline
  
  \verb|\DefaultHeightofSubmit|     & \verb|12pt |             \\
  \verb|\DefaultWidthofSubmit|      & \verb|2cm |              \\
  \verb|\DefaultHeightofReset|      & \verb|12pt |             \\
  \verb|\DefaultWidthofReset|       & \verb|2cm |              \\
  \verb|\DefaultHeightofCheckBox|   & \verb|0.8\baselineskip | \\
  \verb|\DefaultWidthofCheckBox|    & \verb|0.8\baselineskip | \\
  \verb|\DefaultHeightofChoiceMenu| & \verb|0.8\baselineskip | \\
  \verb|\DefaultWidthofChoiceMenu|  & \verb|0.8\baselineskip | \\
  \verb|\DefaultHeightofText|       & \verb|\baselineskip |    \\
  \verb|\DefaultWidthofText|        & \verb|3cm |              \\
  
  \hline
  
  \end{tabular}
\end{center}

%%page page_76                                                  <<<---3

Note that all colors must be expressed as RGB triples in the range 0..1
(for example, color=0 0 0.5). In general, these options simply provide
an interface to the relevant PDF code which is fully documented in Bienz
et al. (1996). Familiarity with that document is vital if you plan to go
beyond the defaults and simple variations. 

Now that we have all the tools described, what about a simple example?
\reffig{2-17} shows a typical form, demonstrating almost all of the
different field types.  Let us look in detail at the code that produced
this form. First, in the document preamble, we load hyperref and then
start a Form environment with an appropriate URL (simply mail it). 

\begin{verbatim}

	\documentclass{article} 
	\usepackage [bookmarks=false]{hyperref} 
	
	\setlength{\parindent}{Opt}\setlength{\parskip}{10pt} 
	\begin{document} 
	\begin{Form}[action=mailto:srahtz,method=post] 

\end{verbatim}

The following entries show the different field types: 

\begin{enumerate}

  \item Text field. Here we supply a value for the width; by default it would be the size 
    of the default value: 

    \begin{verbatim}
    \TextField[width=3in,name=xname,va1ue={Bi1bo Baggins}]{Fu11 name: } 
    \end{verbatim}

  \item Text field, multiline. The text color and box color are changed and a dashed 
    border is drawn around the field: 

    \begin{verbatim}
    \TextField[multiline,width=1in,name=address,borderstyle=D, 
    color=1 1 1,backgroundcolor=O O .5, 
    value={Bag End, The Hill, Hobbiton}]{Address: } 
    \end{verbatim}

  \item Choice. By default, the height of the field would be enough to hold all the 
    choices, but we limit it to showing three at a time: 

    \begin{verbatim}
    \ChoiceMenu[default=Home,menulength=3,width=2in,name=travel,default=Beorn] 
    {Favorite part of your travels:} 
    {Tro11s,Misty Mountains,Beorn,Mirkwood,E1ves,Laketown,% 
    Smaug,The Battle} 
    \end{verbatim}

  \item Checkboxes. Only one box is checked at startup: 

    \begin{verbatim}
    Have you still got your: 
    \CheckBox[]{Sword} 
    \CheckBox[name=coat]{Mithril coat} 
    \CheckBox[name=ring,checked]{\textbf{Ring!}} 
    \end{verbatim}

    %%page page_77                                                  <<<---3

    \item Choice, radio style. Note here the supplying of different values shown from 
      those returned by the checked box: 

      \begin{verbatim}
      \ChoiceMenu[radio,default=Again,name=next, 
      borderwidth=3,bordercolor=O 1 0]{Do you want to:} 
      {Do it all again=Again, 
      Pretend it never happened=Forget, 
      Write a book about it=Write} 
      \end{verbatim}

    \item Text field, password style. When text is entered, it is shown as asterisks: 

      \begin{verbatim}
      \TextField[password,name=made]{who made the ring? } 
      \end{verbatim}


    \item Choice, combo style. This field type allows us either to select one of the provided options or to type in a new one: 

      \begin{verbatim}

      \ChoiceMenu[combo,default=Bofur,name=whatdwarf] 
      {Select funniest name, or add one} 
      {Bofur,Thorin,Gollum,Smaug,Gandalf} 

      \end{verbatim}

      This shows the state when the field is not active: 

      And this shows the list popping up: 

    \item PushButton. This type activates some JavaScript code when the field is clicked: 

      \begin{verbatim}
      \PushButton[name=xxx,onclick={app.beep(0)}]{Make a horrid beep} 
      \end{verbatim}

    \item Send $\wedge$ Clear fields. These submit the contents of the form, and clear all the 
      fields, respectively: 

      \begin{verbatim}
      \Submit{Send} \Reset{C1ear} 
      \end{verbatim}

  \end{enumerate}

Finally, complete the \verb|Form| environment and end the document. 

\begin{verbatim}

\end{Form} 
\end{document} 

\end{verbatim}
 
%%page page_78                                                  <<<---3

The same \LaTeX\  file can also be used to produce an almost identical HTML 
form (\reffig{2-18}) by specifying the \verb|tex4ht| option when loading the \verb|hyperref| 
package. Chapter 4 discusses how to run tex4ht to get the HTML file. 

Now that we have a form ready to interact with, how can we get at the data? 
Merz (1998), Chapter 10, deals well with this issue, and in this book we can only 
summarize the possibilities. There are essentially two ways to process the form data: 

\begin{enumerate}
  \item  If the PDF form is viewed inside a Web browser and a suitable URL is provided, 
    clicking on the \verb|Send| button will post the data to the URL. 
  \item  There is an Acrobat menu option (File $\rightarrow$ Export
    $\rightarrow$ Form Data) that prompts for the name of an FDF file in
    which to save the data. 
\end{enumerate}

The FDF file format is described in detail in Bienz et al. (1996),
Appendix H; and Adobe makes available a toolkit for writing programs to
process it. It is a simplified form of PDF, and the following example
(from our simple form) shows the straightforward data structures:  
 
%%page page_79                                                  <<<---3

\begin{verbatim}
 
%FDF-1.2 
1 0 obj 
<< 
/FDF << /Fields [ 
  << /V (Bag End, The Hill, Hobbiton)/T (address)>> 
  << /V /Off /T (coat)>> 
  << /V /nextl /T (next)>> 
  << /V /Yes /T (ring)>> 
  << /V /Yes /T (Sword)>> 
  << /V (Mirkwood)/T (travel)>> 
  << /V_(Kili)/T (whatdwarf)>> 
  << /V (Bilbo Baggins)/T (xname)>> 
] 
>> 
endobj 
trailer 
<< /Root 1 O R >> 
%%EOF 

\end{verbatim}

\ipar{2_3_6_1_Validating_form_fields}{Validating form fields}

It is possible to write JavaScript code to perform sophisticated
validation of the contents of form fields (Merz (1998), pages 141-144,
shows how to check an ISBN code, for instance), but Acrobat provides a
range of built-in JavaScript functions for simple checking. These can be
accessed by giving the function name and arguments with the keystroke,
format, validate, and calculate options. Warning: These functions are
undocumented and unsupported by Adobe! It is possible that they may no
longer be available in later versions of Acrobat. 

A summary of the available JavaScript functions follows; you should enter the 
code exactly as it appears here, for example 

\begin{verbatim}
\TextField[name=ndwarves, 
validate={AFRange_Va1idate\string\(true, 3, true, 10\string\);}] 
{How many dwarves came along: } 
\end{verbatim}

Note the distinction between keystroke functions, which determine what
the user is allowed to type, and format functions, which determine how
it is displayed.  Usually, both will be supplied for a given field. 

\noindent\textbf{Ensure field content is entered and formatted as a percentage or
a number} 

\begin{verbatim}

  AFPercent_Keystroke\string\(places, 0\string\); 
  AFPercent_Format\string\(value, O\string\); 
  AFNumber_Keystroke\string\(places, O, 0, 0, "", true\string\); 
  AFNumber_Format\string\(places, 0, 0, 0, "", true\string\); 

\end{verbatim}

where \emph{places} is the number of decimal places. 

%==========80==========<<<---2
 
%%page page_80                                                  <<<---3
 
\noindent\textbf{Ensure field content is entered and formatted as a date} 

\begin{verbatim}
AFDate_Keystroke\string\ ( type\string\) ; 
AFDate_Format\string\ ( type\string\) ; 
\end{verbatim}

where \emph{type} is a number from the following table: 

\begin{center}
  \begin{tabular}{cccccc}
  0 & 1/3      & 5 & 3-Jan-81   & 10 & Jan 3, 1981     \\
  1 & 1/3/81   & 6 & 03-Jan-81  & 11 & January 3, 1981 \\
  2 & 01/03/81 & 7 & 81-01-03   & 12 & 1/3/81  2:30pm  \\
  3 & 01/81    & 8 & Jan-81     & 13 & 1/3/81  14:30   \\
  4 & 3-Jan    & 9 & January-81                        \\
  \end{tabular}
\end{center}


\noindent\textbf{Ensure field content is entered and formatted as a time}

\begin{verbatim}

AFTime_Keystroke\string\( type\string\) ; 
AFTime_Format\string\ ( typ e\string\) ; 

\end{verbatim}

where type is a number from the following table: 

\begin{center}
  \begin{tabular}{cc}
  0 & 14:30     \\
  1 & 2.30 pm   \\
  2 & 14:30:15  \\
  3 & 2:30:15pm \\
  \end{tabular}
\end{center}

Format entry in a specific way 

\begin{verbatim}
AFSpecial_Format\string\( type\string\) ; 
\end{verbatim}

where type is a number from the following table: 

\begin{center}
  \begin{tabular}{cc}
  0 & Zip Code               \\
  1 & Zip Code + 4           \\
  2 & Phone Number           \\
  3 & Social Security Number \\
  \end{tabular}
\end{center}

\textbf{Validate numbers to a given range} 

\begin{verbatim}
AFRange_Validate\string\(true, minimum, true, maa:imum\string\); 
\end{verbatim}

\textbf{Specify that this field is derived from others} 

\begin{verbatim}
AFSimple_Calculate\string\("function", "List of field names"\string\); 
\end{verbatim}

The possible values for \emph{function} are SUM, PRODUCT, AVERAGE, MINIMUM, and
MAXIMUM; the \emph{list of fields} must be separated by commas. 

%%page page_81                                                  <<<---3
 
\isubsubsec{2_3_7_Designing_PDF_documents_for_the_screen}{Designing PDF documents for the screen}

For many people, simply having a PDF version of a printed document, with active 
links, is useful enough. Others, however, have started to consider the use of PDF 
for documents whose only existence is on the computer screen. 

In Section 2.3.4 on page 47 we saw how we can access all the menu options of 
Acrobat from within \LaTeX; this allows us to build a complete user interface using 
all the power of \ \TeX. 

Let us consider some of the ways in which we can design a document just
for the screen. See \reffig{2-19}, \reffig{2-20} and \reffig{2-21}. 

\begin{itemize}
  \item We set up a landscape page design, which has the same aspect ratio
    as the computer screen; we choose 6 in $\times 4$\ in; page margins
    are adjusted to suit. Remember that \LaTeX\  gives one-inch margins by
    default; we need to crop the page so that the margins are of minimal
    width. The \LaTeX\  oneside option must be used, as there is no longer
    any concept of odd or even pages. When the text width is small, you
    should consider using the \verb|\raggedright| setting; 
  \item We set the font family to one suited for screen reading; we choose Lucida 
    Bright; 
  \item Since there is no reason to fill pages, we set section headings to start a new 
    page;
  \item We use color; section headings are set in blue, hypertext links are colored, table 
    rows are shaded, etc.; 
  \item The \verb|\ref| commands are replaced by \verb|\autoref| (or by \verb|\hyperref| if a more 
    customized link is needed); this makes the colored links have a more visible 
    context; 
  \item  Should you wish to disable the Acrobat menu and toolbar, a navigation and 
    function bar can be put at the bottom of each page; 
  \item To provide a visible pointer to the current page's location within the article, a 
    progress gauge is placed at the bottom of each page: this works by comparing 
    the current page number with the number of the last page, and constructing a 
    colored bar of appropriate length. 
\end{itemize}

We would also need to work out a strategy for marginal notes and footnotes, since 
these will look unnatural in a screen display. 

\reffig{2-22} shows another possible design; this one has a navigation
panel on the right-hand side of every page, including a summary table of
contents. C. V. Radhakrishnan's \verb|pdfscreen| package (building on
\verb|hyperref|) implements this scheme.  It has options to generate
sidebar or footer menus and commands to specify logos and addresses to
appear on every page. 

%%page page_82                                                  <<<---3
%%page page_83                                                  <<<---3
%%page page_84                                                  <<<---3
 
\isubsubsec{2_3_8_Catalog_of_package_options}{Catalog of package options}
 
%%page page_85                                                  <<<---3
%%page page_86                                                  <<<---3
%%page page_87                                                  <<<---3
 
%%TODO current
 

The \emph{Key(s)} dictate how the effect appears, for example \verb|pdfpagetransition={Blinds /Dm /V}| 

\begin{itemize}
  \item \verb|/Di| (Direction) The direction of movement, in degrees (counterclockwise). Values are generally in 90° steps. 
  \item \verb|/Dm| (Dimension) If a choice between horizontal or vertical is allowed, value is \verb|/H| (horizontal) or \verb|/V| (vertical). 
  \item \verb|/M| (Motion) If an effect can be from the center out or from the edges in, value is \verb|/I| (in) or \verb|/O| (out). 
\end{itemize}
