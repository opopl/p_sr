% vim: keymap=russian-jcukenwin
%%beginhead 
 
%%file 27_08_2021.fb.zharkih_denis.1.savenkova
%%parent 27_08_2021
 
%%url https://www.facebook.com/permalink.php?story_fbid=3064378483775548&id=100006102787780
 
%%author Жарких, Денис
%%author_id zharkih_denis
%%author_url 
 
%%tags donbass,kultura,literatura,lnr,savenkova_faina,tvorchestvo,vojna
%%title Маленький человек с большим сердцем
 
%%endhead 
 
\subsection{Маленький человек с большим сердцем}
\label{sec:27_08_2021.fb.zharkih_denis.1.savenkova}
 
\Purl{https://www.facebook.com/permalink.php?story_fbid=3064378483775548&id=100006102787780}
\ifcmt
 author_begin
   author_id zharkih_denis
 author_end
\fi

Маленький человек с большим сердцем. 

О Фаине Савенковой говорят разное, больше, конечно, хорошее, но я читал в
отзывах и скептические отклики. Смысл скептиков сводится к одному - не может
маленькая девочка 11-12 лет так писать, значит, за нее это делает кто-то
взрослый. И тогда, делают вывод скептики, феномен Фаины просто надутый пузырь,
который работает на определенную конъюнктуру. 

\ifcmt
  pic https://scontent-cdt1-1.xx.fbcdn.net/v/t1.6435-9/240655468_3064357870444276_5402503717570568879_n.jpg?_nc_cat=109&ccb=1-5&_nc_sid=730e14&_nc_ohc=mT5AvjUsx3cAX-XjTXx&_nc_ht=scontent-cdt1-1.xx&oh=af25a418bcc2892bbcd84d0b89475ee1&oe=615116CE
  width 0.4
	fig_env wrapfigure
\fi

Ну, что сказать... я сам попадал в такие ситуации не раз. Мне часто говорили,
что тексты мне не принадлежат, а успех тех или иных проектов это работа
совершенно других людей. Уже в 2002 году один маститый политтехнолог заявил,
чтобы я не примазывался к проекту, который "создали в подвалах ФСБ". Я уж и не
знал радоваться ли мне или горевать. С одной стороны, оценка, вроде как
высокая, не я создавал, а группа товарищей.  из ФСБ. С другой стороны, почему
даже офицерам ФСБ можно заниматься творчеством, а мне нет? Вот сроду с
офицерами ФСБ не был знаком, и в подвалы к ним не заглядывал. Я даже больше
скажу, являюсь одним из немногих украинских журналистов, который не знает, что
думает Путин, и когда он нападет на Украину. Короче, бездарь еще тот. 

Ну, начнем с того, что Фаина часто пишет в соавторстве. Возьмем рассказ
"Учителя и ученики". Вот указано, что он написан в соавторстве с неким
Александром Конторовичем.  Посмотрел - писатель фантаст, годится Фаине в
дедушки. Если делать бренд, то зачем вписывать имя маститого фантаста, да еще и
перед Фаиной?  Неувязочка...

Но только очень недалекий человек не поймет, что тексты Савенковой писал
ребенок. Опаленный войной, рано повзрослевший, но именно ребенок. А взрослые,
если и были, то правили текст. Но тексты и взрослых писателей правят, что в
литературе естественно. Писатель без редактора не бывает, просто редактор часто
не виден. В любой сфере есть тот, кто играет на сцене, и тот, кто остается за
кадром. 

Давайте разберем уже выбранный нами рассказ "Учителя и ученики". Не знаю какой
рассказ был до правок, но бросаются в глаза две вещи - он совершенно не
конъюнктурен, и уводит читателя в мир фантазий подальше от реальностей войны.
Если это не художественное восприятие ребенка, то что это такое? Маститый
писатель никогда бы так не написал. Конъюнктура требует "Киборгов"/"Солнцепека"
и много чего другого, только не этого вот. 

Другой вопрос, что "Учителя и ученики" более искренни, более человечны, более
мудры. Но мудры они мудростью ребенка, который пережил войну, который сохранил
чистую душу и осудил войну, а не поддался военной романтике. Сюжет рассказа
простой, а не закрученный, как у маститых, хорошо оплачиваемых авторов. Среди
жителей полосы военного соприкосновения появляются особые коты "барсики",
которые защищают детей от невзгод войны - мин, обрушений домов, обстрелов. Вы
мне скажите, мог ли это придумать взрослый? Или все же девочка, закрывающая
собой кошку во время обстрела? Необязательно, чтобы она лично ее закрывала, но
обязательно чтобы во время обстрела она переживала за кошку искренне, глубоко и
трагично. 

Взрослый думает о другом, на то он и взрослый. Для него трагедия кошки не
трагедия, а сама кошка не может быть защитой. Но для девочки может быть защита
в виде волшебной кошки. И эта кошка будет защищать собой только доброго и
благородного ребенка. 

То, что рассказ подвергался взрослой редактуре очевидно. Иначе как объяснить,
что "барсики" по рассказу это души погибших солдат. Но почему только солдат, а
еще не детей, стариков и женщин? Разве они не способны защищать детей, если
брать логику авторов? Тут определенная героизация. Бьюсь об заклад, приложили
руку взрослые, хотя со свечкой не стоял, могу и ошибиться. Но дело не в том,
что там принадлежит Фаине, а что взрослым. Если что-то написано в соавторстве,
то принадлежит всем соавторам. 

Вопрос в восприятии войны именно ребенком. И это взрослые убрать/переделать не
смогли, да и незачем это было. Ведь за воображением в рассказе проявляется
острая боль за страну, за детские жизни, за тех, кто войну продолжает и
наваривает на ней деньги. И эта боль не на показ, а глубоко внутри, сильная,
жгучая, честная и искренняя. Ведь в рассказе не только волшебные кошки
переживают за детей, но и дети переживают за кошек. Рассказ вообще не о кошках,
а о сострадании ко всему живому на войне, к несправедливости смерти и борьбе за
жизнь. Борьбе, в первую очередь внутри себя, борьбе против равнодушия и
потребительства, против страха и жестокости, против цинизма и шкурничества. 

Кстати, про шкурничество. Рассказ не на продажу, коммерческой и
пропагандистской нагрузки в нем нет. Там одна боль и вера в маленьких людей,
которые, защищая кошек,  защищают себя. На вере в людей не заработаешь. Но дети
этого не знают, они продолжают верить. Не все дети, только лучшие из них,
способные чувствовать чужую боль, как свою. Даже боль бездомной кошки...
