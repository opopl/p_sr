% vim: keymap=russian-jcukenwin
%%beginhead 
 
%%file 03_02_2018.stz.news.ua.mrpl_city.1.zvuki_proshlogo
%%parent 03_02_2018
 
%%url https://mrpl.city/blogs/view/zvuki-proshlogo
 
%%author_id burov_sergij.mariupol,news.ua.mrpl_city
%%date 
 
%%tags 
%%title Звуки прошлого
 
%%endhead 
 
\subsection{Звуки прошлого}
\label{sec:03_02_2018.stz.news.ua.mrpl_city.1.zvuki_proshlogo}
 
\Purl{https://mrpl.city/blogs/view/zvuki-proshlogo}
\ifcmt
 author_begin
   author_id burov_sergij.mariupol,news.ua.mrpl_city
 author_end
\fi

Звуки, не записанные на восковые валики Томаса Эдисона, на граммофонные
пластинки начала ХХ века из экзотического материала – шеллака, впоследствии
замененного на винил, звуки, зафиксированные магнитофонной пленкой и, тем
более, не современными лазерными дисками. А звуки, казалось бы, совершенно
незначащие, которые нет-нет да и всплывают из глубин памяти, пробуждая
воспоминания о давно ушедших в небытие людей и событий.

\ii{03_02_2018.stz.news.ua.mrpl_city.1.zvuki_proshlogo.pic.1}

Ночь на излете. Долгий нервный стук в окно до дребезжания стекла. Это
дворничиха черенком своей метлы напоминает, что нужно включить лампочку над
табличкой с номером дома. Скрипнула панцирная сетка кровати. Бабушка кряхтит,
слышно, как она ногами пытается нащупать свои шлепанцы. Наконец, шаркая ими,
идет к выключателю, закрепленному у крайнего окна нашей комнаты, к тому самому,
откуда пришел разбудивший звук. Щелчок выключателя. Шепот. Бабушка шепчет:
\enquote{Отче наш, иже еси...} перед иконой. Под ее молитву погружаешься в
дремоту.

Захрипел репродуктор. Стало быть, до шести остались считанные минуты. По
привычке военной поры, когда строго предписывалось подобного рода аппараты не
выключать на случай воздушной тревоги, репродуктор всегда соединен с городской
радиотрансляционной сетью. Шелест шин автомобилей по брусчатке Красной площади,
звуки клаксонов, - пройдет  еще несколько десятков лет, прежде чем на улицах  и
площадях советских городов будет запрещена подача звуковых сигналов
автотранспортом. Бой курантов на Спасской башне, ударами большого колокола
отсчитывается шесть часов. Хор и оркестр исполняет Гимн...  Торжественный голос
Юрия Левитана: \enquote{Московское время шесть часов. Доброе утро, товарищи!
Слушайте последние известия...} Затем следуют \enquote{Пионерская зорька},
\enquote{Утренняя гимнастика} с Константином Родионовым, другие передачи,
названия которых памятны только поколению, относящемуся к так называемым
\enquote{детям войны}.

Гудок паровоздуходувной станции завода \enquote{Азовсталь},  оповещающий рабочих первой
смены, что уже половина седьмого и пора идти на работу. Второй гудок в семь –
начало смены. Еще утренние звуки. Четкий строевой шаг сотен ног – военнопленных
немцев ведут на восстановление заводов, металлургического и коксохимического.
Ни одного сбоя в этом марше, кажется, на века отработанный темп. Время от
времени вклиниваются семенящие шаги одного из немногих сопровождающих колонну
конвоиров. Стук колодок – рабочих ботинок, состоящих из вырезанной из
полудюймовой доски подошвы и верха из грубого брезента. Такой обувью снабжали
парней и девчат из близких и дальних сел и деревень, завербованных на
восстановительные работы...

Кашляющий треск изношенных моторов видавших виды полуторок и ЗиСов,
самодовольное равномерное гудение полученных по ленд-лизу американских
\enquote{Студебекеров} и \enquote{Доджей}. Хлесткие удары кнута, матерок возницы, жалобное
ржание лошади. Прямо перед нашим домом середина крутого подъема Торговой улицы
от Фонтанной до Карла Либкнехта. Шуршание обрезиненных колес, смешивающееся с
прозванием бубенчиков на сбруе. На линейке, должно быть, везут какого-нибудь
строительного начальника на объект. А может, и не начальника, а врача и
фельдшера \enquote{скорой помощи}. Да, в те достопамятные времена медики на помощь к
больному ездили на линейках. Потому-то при лечебных учреждениях тогдашнего
Мариуполя обязательно были конюшни, а приобретением лошадей занимались лично
главные врачи. Душераздирающий рев мотора. Это едет по своим делам на трофейном
мотоцикле с коляской человек, облаченный с головы до ног в кожу. На нем летный
шлем, огромные очки авиатора, закрывающие половину лица, куртка, застегнутая на
все пуговицы, плотно обхватила полный торс, краги и ботинки на толстенных
подошвах...

Перебранка подле продовольственного магазина, что напротив. Очередь за хлебом.
Ее занимают задолго до рассвета. Не только себе, но родственникам и близким, и
соседкам. Но наступает момент истины: разбор кто за кем стоит: 

- Чего  лезешь без очереди?

- Как \enquote{без очереди}?

- Я лично за этой дамой занимала.

- Нюра! Да не за ней я тебе сказала стоять. Вон за той бабкой становись.

- Какая я тебе бабка? Сама старуха беззубая!

- А ты с немцами гуляла!

- Что? Да я в эвакуации была, пухла от голода в Нижнем Тагиле.

- Жаль, что не сдохла! 

Страсти накаляются. К ссорящимся присоединяются другие женщины. Тут возглас:
\enquote{Хлеб привезли!} Все голоса объединяются в общий гул...

Детские шаги. Много детских шагов. Ученицы первой женской неполной средней
школы идут на занятия. Они учатся в первую смену. Когда у них закончатся уроки,
их места (разномастные столы, парты, сохранившиеся с довоенных времен и
сделанные уже в послевоенные годы для своих чад родителями, занимавшими
руководящие посты, - такова была школьная мебель) займут мальчики из третьей
мужской неполной средней школы. А пока самые прилежные из них готовят домашние
задания, прочие же занимаются делами по своему усмотрению. Шелестящие звуки
давно несмазанных шариковых подшипников, щелчок, еще щелчок – подшипники
преодолевают стыки плит тротуара. Это мальчишки устроили гонки на самокатах,
сделанных собственными руками из двух дощечек и двух подшипников, раздобытых
неведомо где. Старт у них  - на углу Торговой и улицы Карла Либкнехта. Оттуда
они очертя голову несутся до Фонтанной, где мостовая из грубо отесанных камней,
уложенных кое-как, тормозит их стремительное движение...

Решения задач по арифметике, выполнение упражнений по русскому и украинскому
языкам, заучивание стихотворений и правил, запоминание немецких слов, очередных
параграфов по истории и другим предметам происходит под аккомпанемент
радиопередач из Москвы: \enquote{Угадайки} и  \enquote{Клуба знаменитых капитанов},
радиоспектакля по роману Вениамина Каверина \enquote{Два капитана} и концерта по
заявкам  радиослушателей с голосами Бунчикова и Нечаева, Клавдии Шульженко,
Максима Михайлова, Сергея Лемешева, Ивана Козловского...

В перерывах включаются Украинское республиканское радио, сталинская областная
радиостанция РВ-26, местное радио сообщает городские новости, объявления и
постановления горисполкома. В одиннадцать часов звучит: \enquote{Передаем
новости для областных, городских и районных газет}. Женщина-диктор хорошо
поставленным голосом монотонно, с паузами, чеканя каждое слово, диктует
сообщения ТАСС, передовые статьи и официальные сообщения.

Настенные часы в темном резном футляре с начертанными витиеватыми буквами на
белой эмали маятника словами \enquote{Павелъ Бурэ} давно и безнадежно стоят.
Ориентирами в определении времени служит сетка радиопередач, которая не
меняется годами, и, конечно же, гудки \enquote{Азовстали}:  в половине седьмого
и в семь для первой смены, в половине третьего и три пополудни - для второй, в
половине одиннадцатого и одиннадцать вечера - для ночной. Ведь часы в тот
период истории нашей страны были редкостью, до массового выпуска впоследствии
популярных наручных часов \enquote{Победа} у отечественной промышленности еще
не дошли руки. А привезенные демобилизованными воинами немецкие
\enquote{штамповки} были редки, да и дороги.

Иногда летними вечерами дополнением к музыкальным передачам, которые дарил
репродуктор, становились концерты, устраиваемые долговязым белобрысым парнем.
Он жил через дорогу от нашего дома. Этот местный меломан распахивал окно,
устанавливал на подоконнике немецкую радиолу и часами крутил круглые куски
рентгеновской пленки с изображениями различных поврежденных частей
человеческого скелета. На этих заменителях фабричных пластинок умельцы
самодельными рекордерами наносили звуковую дорожку. Репертуар был невелик -
песни и романсы в исполнении Петра Лещенко, Вадима Козина, Лидии Руслановой,
как тогда говорили, запрещенных певцов, но зато часто за один вечер многократно
повторяющийся. К счастью, никто не донес на обладателя пластинок \enquote{на костях}. А
донесли бы \enquote{куда следует}, попал бы он туда, где \enquote{мотали сроки} и Козин, и
Русланова.

А в это время в соседнем дворе тишину нарушали голоса малышни, они играли в
\enquote{жмурки}. Сначала - считалка: \enquote{ Царь, царевич, король, королевич, сапожник,
портной, а ты – кто такой?} Минута безмолвия, топот босых ног – прячутся, и
тонкий голосок, того или той, на которого  пал жребий: \enquote{Раз, два, три. Вот мои
шаги, кто не заховался, я не виноват!}. После этой тирады на смеси русского
языка с элементом местного суржика начинались поиски спрятавшихся участников
игры. А потом были шум и гам, беготня, споры и сердитые окрики бабушек, чей
сон, присущий пожилым людям с наступлением сумерек, был нарушен. В половине
одиннадцатого – гудок \enquote{Азовстали}. Детям пора ложиться спать...

Звуки прошлого, при воспоминании которых щемит сердце.
