% vim: keymap=russian-jcukenwin
%%beginhead 
 
%%file 18_02_2021.fb.fb_group.story_kiev_ua.2.puscha_vodica_tramvaj_marshrut
%%parent 18_02_2021
 
%%url https://www.facebook.com/groups/story.kiev.ua/posts/1600876860109109
 
%%author_id fb_group.story_kiev_ua,kuzmenko_petr
%%date 
 
%%tags gorod,kiev,kiev.puscha_vodica,tramvaj
%%title Пуща-Водица - трамвайный маршрут
 
%%endhead 
 
\subsection{Пуща-Водица - трамвайный маршрут}
\label{sec:18_02_2021.fb.fb_group.story_kiev_ua.2.puscha_vodica_tramvaj_marshrut}
 
\Purl{https://www.facebook.com/groups/story.kiev.ua/posts/1600876860109109}
\ifcmt
 author_begin
   author_id fb_group.story_kiev_ua,kuzmenko_petr
 author_end
\fi

\ii{18_02_2021.fb.fb_group.story_kiev_ua.2.puscha_vodica_tramvaj_marshrut.pic.1}

КИЕВЛЯНЕ! Сейчас, находясь далеко от родного Города, очень прошу Вас написать
хоть пару слов о Ваших эмоциях, когда Вы смотрите на это фото. О нашей Пуще,
Подоле ( ибо Пуща - Водица Подольский район до сих пор, как мне известно), о
старейшем и наиболее продолжительном трамвайном маршруте Киева и Ваших эмоциях
и воспоминаниях. Благодарю Киевские истории за частичку Дома. @igg{fbicon.heart.eyes} 

\ii{18_02_2021.fb.fb_group.story_kiev_ua.2.puscha_vodica_tramvaj_marshrut.cmt}
