% vim: keymap=russian-jcukenwin
%%beginhead 
 
%%file 16_05_2023.stz.news.ua.donbas24.1.zi_sljozamy_na_ochah_zustrich_objednala_mrplciv.txt
%%parent 16_05_2023.stz.news.ua.donbas24.1.zi_sljozamy_na_ochah_zustrich_objednala_mrplciv
 
%%url 
 
%%author_id 
%%date 
 
%%tags 
%%title 
 
%%endhead 

Ольга Демідко (Маріуполь)
16_05_2023.olga_demidko.donbas24.zi_sljozamy_na_ochah_zustrich_objednala_mrplciv
Маріуполь,Україна,Мариуполь,Украина,Mariupol,Ukraine,Азовсталь,Валерія Суботіна,Зустріч,Київ,Киев,Kyiv,Kiev,date.16_05_2023

Зі сльозами на очах — зустріч, що об'єднала маріупольців (ВІДЕО)

Маріупольці у Києві подякували захисниці свого міста

13 травня у Червоній залі Будинку кіно відбулася творча зустріч з поетесою,
журналісткою та захисницею Маріуполя Валеріює Суботіною (Карпиленко) на
позивний Нава. Цей захід зібрав не тільки друзів, знайомих, посестер і
побратимів Валерії, він об'єднав маріупольців, які були в Києві чи приїхали зі
Львова, Івано-Франківська, Дніпра, щоб подякувати мужній і сталевій дівчині,
яка не так давно вийшла з полону.

Читайте також: Історія сталевої Нави — захисниця Маріуполя розповіла про
весілля на Азовсталі та російський полон (ВІДЕО)

Друзі Валерії виходили на сцену, аби поділитися життєвими історіями, які їх
поєднують з військовослужбовицею, зачитували її вірші зі збірки «Квіти і
зброя». Також присутні присвячували жінці власні поезії, натхненням для яких
стала вражаюча сила Нави та кохання подружжя Суботіних. Всіх присутніх до
глибини душі вразив музичний відеокліп «Кохання сталь». Пісню, присвячену
драматичній історії кохання, виконала народна артистка України Марія Бурмака.

Маріупольці приїхали з рідних міст України, щоб зустрітися з Валерією та
особисто подякувати їй за самовідданість, неабияку мужність і захист. На цьому
заході вони зустрілись з багатьма співмістянами і були дуже розчулені такою
зустріччю.  

«Я переповнена щастям від зустрічі з такими рідними обличчями. Наче знову в
Маріуполі. І головне, я зустрілася і обнялася зі Сталевою Незламною Навою!», —
підкреслила маріупольчанка Вікторія Константіновська.

«Безмежно щаслива від того, що ми всі побачились! Я приїжджала до Києва
спеціально на цю зустріч! Вже вдома, на Львівщині, а думками і далі з кожним з
вас!», — зазначила поетеса Надія Умриш.

«Чудовий вечір. Дуже хвилюючий. Дякую, що не було усяких там „посадових осіб“.
Дуже душевно», — зауважила маріупольчанка Ольга Гапоненко.

Читайте також: У Києві відбувся захід на підтримку військовополоненої з
Азовсталі (ВІДЕО)

Крім цього, під час заходу виступило багато українських митців. Зокрема, пісні
виконав актор-військовослужбовець Дмитро Лінартович, композиції на твори
Григорія Сковороди та Василя Стуса — лірник і бандурист Тарас Компаніченко, а
Марія Бурмака заспівала пісні «Поцілуй мене на прощання», «Повернись живим» та
«Перемога».

Водночас відомі захисники України, актори, поети і письменники, які не
відвідали захід, прочитали вірші онлайн. Серед них: Катерина Поліщук (Пташка),
Сергій Жадан, Римма Зюбина і Павло Вишебаба. До речі, останній був
однокурсником Валерії.

Читайте також: Надихаючі історії про непересічних жінок Приазов’я

Валерія прочитала три нових вірші та поділилася своїми планами з усіма
присутніми, наголосивши, що вона продовжуватиме писати.

«Планую працювати над прозою, тому що обов’язково потрібно написати про
пресслужбу "Азовсталі". Те, про що я можу написати зсередини, то, чим я
займалася, я була з тими хлопцями поруч постійно. І звичайно, що треба писати
про полон: про те, як ми не зламалися», — наголосила Валерія Суботіна.

Нагадаємо, раніше Донбас24 розповідав, що юні маріупольці виступили в Києві з
концертною програмою.

Ще більше новин та найактуальніша інформація про Донецьку та Луганську області
в нашому телеграм-каналі Донбас24.

ФОТО: Юрія Вереса
