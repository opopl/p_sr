% vim: keymap=russian-jcukenwin
%%beginhead 
 
%%file 27_11_2020.fb.group.rrr.2.mama
%%parent 27_11_2020
 
%%url https://www.facebook.com/groups/582412455686590/permalink/761921641069003/
 
%%author 
%%author_id leonova_lilia
%%author_url 
 
%%tags 
%%title 
 
%%endhead 

\subsection{Мама - старенькая девочка}
\label{sec:27_11_2020.fb.group.rrr.2.mama}
\Purl{https://www.facebook.com/groups/582412455686590/permalink/761921641069003/}

\ifcmt
  author_begin
   author_id leonova_lilia
  author_end
\fi

Мама - старенькая девочка.

Забрала к себе жить маму. Навсегда. Ничего заранее не решая, просто одним днём,
с одним пакетом, в пакете у неё --- колготки, тапочки с надписью «Лучшей бабушке
на свете" (подарок моих детей), халат с рубашкой и почему-то наволочка, мама
вещи собирала сама.

Теперь у меня дома живёт старенькая девочка лет пяти с белоснежной гулькой на
голове. Она ходит по коридору, мелко шаркая теплыми тапочками, останавливается
у порога и высоко поднимает ноги, переступая невидимые препятствия, улыбается,
радуется собаке, играется с ней, стала слышать невидимых людей и рассказывает
мне новости от них, всё время как-будто чего-то стесняется, много спит...
Аккуратно кусает шоколадку (я все время ей кладу в комнату шоколад), запивает
чаем, придерживая чашку двумя руками --- руки слабеют, одна рука дрожит.  Боится
потерять с тонкой руки обручальное кольцо, всё время его проверяет. Я вдруг
увидела, какая она старенькая и беспомощная. Она просто в какой-то момент
отпустила себя, расслабилась и перестала играть во взрослую.  И доверила
полностью, во всех мелочах свою жизнь мне. Теперь самое главное для неё --- когда
я дома, она так облегчённо выдыхает, когда я прихожу...

И я опять теперь варю супы и каши, как моим детям в детстве, опять на столе
появилась вазочка с печеньем и конфетами...

Что я чувствую? Сначала --- недоумение. Она была всегда такой активной,
самостоятельной,  все три года после папиной смерти хотела жить одна,
противилась переезду.  И я её понимаю --- впервые в жизни, в свои восемьдесят
пять мама делала то, что хочет сама...

Но гиподинамия и этот треклятый вирус сделали своё пагубное дело, столько
месяцев без выхода,  в четырёх стенах,  без нормального общения,  психика
рухнула.  Сейчас я чувствую к этой моей хрупкой дорогой Вселенной только любовь
и нежность. Все прошлые ссоры и недопонимания остались в прошлом, мы как будто
идём вместе по дороге, и я хочу, чтобы эта дорога была для неё счастливой --- с
любимой дочкой,  в тепле и комфорте, с пирожками и котлетами, и разными
вкусностями. У меня есть теперь дома дочка восьмидесяти пяти лет, и я
счастлива, что Бог дал мне возможность делать её  счастливой, а её дальнейшую
жизнь --- спокойной, без душевных терзаний и катаклизмов... 

Мама, спасибо, что ты --- у меня! Со мной рядом.  Будь, пожалуйста, подольше…
