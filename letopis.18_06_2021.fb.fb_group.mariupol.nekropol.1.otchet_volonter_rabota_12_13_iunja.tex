%%beginhead 
 
%%file 18_06_2021.fb.fb_group.mariupol.nekropol.1.otchet_volonter_rabota_12_13_iunja
%%parent 18_06_2021
 
%%url https://www.facebook.com/groups/278185963354519/posts/540591917113921
 
%%author_id fb_group.mariupol.nekropol,arximisto
%%date 18_06_2021
 
%%tags 
%%title Отчет о волонтерской работе в Некрополе 12-13 июня 2021
 
%%endhead 

\subsection{Отчет о волонтерской работе в Некрополе 12-13 июня 2021}
\label{sec:18_06_2021.fb.fb_group.mariupol.nekropol.1.otchet_volonter_rabota_12_13_iunja}
 
\Purl{https://www.facebook.com/groups/278185963354519/posts/540591917113921}
\ifcmt
 author_begin
   author_id fb_group.mariupol.nekropol,arximisto
 author_end
\fi

\vspace{0.5cm}
\textbf{Отчет о волонтерской работе в Некрополе 12-13 июня 2021}

Восстановление памятника мариупольских поляков и истории рода Чентуковых,
посадка киевских каштанов – краткие итоги волонтерской экспедиции...

\textbf{Открытия и находки}

На прошедших выходных мы восстановили очередной памятник мариупольских поляков.
И попытались вернуть из забвения их имена. На сегодня в Некрополе уже
обнаружено около десятка памятников польской общины.

Предварительная транскрипция надписи:

S P\par
Hipolo\par
Ur. 1900 z. 1901\par
Czesto\par
Ur. 1902 z. 1903\par
Trzeshicey\par

Судя по всему, речь идет о детях Hipolo и Czesto (?), которые не прожили и
годика. Мы не уверены в транскрипции фамилии и ничего не знаем об истории
людей.

Идентификация мариупольских поляков в Некрополе осложняется тем, что
метрические книги римо-католической церкви Мариуполя не были оцифрованы,
находятся на территории т.н. ДНР и поэтому трудно доступны.

Более успешной оказалась расчистка надписей на двух старинных плитах возле
усыпальницы Найденовых. Они принадлежат Марине и Аврааму Чентуковым, из
знаменитого греческого рода Мариуполя (см. историю \href{https://archive.org/details/15_06_2021.fb.andrij_marusov.mrpl_nekropol.chentukovy}{%
\emph{\enquote{Чентуковы: У них гостили
Айвазовский и император Александр II}}, Андрій Марусов, 15.06.2021}
\footnote{\url{https://www.facebook.com/groups/278185963354519/permalink/539096630596783}}
\footnote{Internet Archive: \url{https://archive.org/details/15_06_2021.fb.andrij_marusov.mrpl_nekropol.chentukovy}}
).

Как ни пытались, на древнем участке Хараджаевых-Гофов мы не смогли поднять и
исследовать старинный памятник. Нужен домкрат и несколько крепких парней.

\textbf{Благоустройство}

Из-за ливней в Некрополе повсюду заросли травы и бурьяна по пояс. Мы собрали
бензокосу \enquote{Буковина}, подаренную меценатами в прошлом году, но не
успели приступить к работе.

Зато мы посадили саженцы киевских каштанов вокруг белого памятного Креста!

И – абсолютно все саженцы дубов и платанов, высаженные волонтерами в этом году,
принялись и чувствуют себя отлично! 🙂

\textbf{Использование пожертвований}

Всего мы потратили 457 грн. для этой волонтерской экспедиции (260 грн. – два
каштана, 140 грн. – бензин для бензокосы, 57 грн. – мешки для мусора и
веревка).

\textbf{Наши планы на выходные 19-21 июня 2021 г.}

С помощью бензокосы мы планируем очистить от бурьянов древний участок,
пространство вокруг могилы с прахом Виктора Арнаутова, а также исследовать
памятники вдоль древней \textbackslash~ Платановой аллеи, возле польских памятников и т.п.

Впрочем, реализация этих планов зависит от участия мариупольцев. Ждем всех
неравнодушных к истории и культуре Мариуполя в 11 утра каждый выходной возле
белого памятного Креста в центре Старого кладбища!

Огромная благодарность участникам волонтерской экспедиции Александру и Наталия Шпотаковская, Андрею Марусову.

Мы также признательны Yaroslav Fedorovskyi за доставку каштанов из Киева в Мариуполь!

До встречи, друзья!

\#mariupol\_necropolis\_report
