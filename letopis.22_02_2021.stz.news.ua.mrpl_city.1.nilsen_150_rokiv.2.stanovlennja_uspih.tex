% vim: keymap=russian-jcukenwin
%%beginhead 
 
%%file 22_02_2021.stz.news.ua.mrpl_city.1.nilsen_150_rokiv.2.stanovlennja_uspih
%%parent 22_02_2021.stz.news.ua.mrpl_city.1.nilsen_150_rokiv
 
%%url 
 
%%author_id 
%%date 
 
%%tags 
%%title 
 
%%endhead 

\subsubsection{Становлення і заслужений успіх}

\ii{22_02_2021.stz.news.ua.mrpl_city.1.nilsen_150_rokiv.pic.2}

Спочатку Нільсен почав працювати у місті Рибінськ (тоді Ярославська губернія),
де він спроєктував та вибудував міський водогін. До речі, водонапірна вежа
Рибінська є старшою сестрою нашої вежі. Вона вийшла більш суворою і зовні
скромнішою, ніж маріуполь\hyp{}ська – яскрава і святкова. Збереглася книга В. О.
Нільсена \emph{\enquote{Опис і розрахунок пристрою водогону в місті Рибінськ з додатком
атласу креслень}} 1901 року, яка свідчить, що архітектор дійсно є автором
Рибінської водонапірної вежі. Успіхи молодого і амбітного інженера були
помітні. Саме тоді він і був запрошений маріуполь\hyp{}ською управою на посаду
міського архітектора. Звісно, для 29-рчного чоловіка – це був непоганий шанс
побудувати кар'єру, тому він одразу ж погодився. Прибув до Маріуполя в 1901
році. На той час маріупольці потерпали від нестачі чистої, питної води,
особливо в спекотне літо. Фонтани, які діяли багато років більше не могли
задовольняти потреби п'ятдесятитисячного населення. Виникла нагальна
необхідність бу\hyp{}дівництва водогону в місті.

Однак не так просто було розпочати роботу, адже відсутність електрики не давала
можливості пустити в роботу водопровід, ще й коштів не вистачало. Для
затвердження проєкт надіслали до Міні\hyp{}стерства і дозвіл на будівництво дав
голова Ради міністрів і за сумісництвом міністр внутрішніх справ Російської
імперії \emph{\textbf{Столипін Петро Аркадійович}}. Так кошти на всі роботи з будівництва
водогону були отримані.

Цікаво, що ще до завершення робіт Віктор Олександрович завоював повагу найбільш
заможних і суспільно активних громадян міста, адже в 1907 році він був обраний
міським гласним (сьогодні – депутат) в міську думу. Можливо, зіграли роль гарна
освіта та старанність і сумлінність, притаманні всім німцям.

Архітектор В. О. Нільсен добре опанував справу і виконав важке завдання.
Завдяки новому міському водогону, водопостачання отримали дві третини мешканців
міста (одна третина використовувала колодязі). До міського архітектора зросла
довіра і прийшов матеріальний успіх, він мав майна на 600 рублів. Водонапірна
вежа створена в формах еклектики, тобто поєднує елементи різних стилів, що
знайшли своє втілення в романських і готичних мотивах віконних прорізів і в
системі прикрас фасаду вежі. Сьогодні вежа не має практичного значення, але
вона стала архітектурною візитівкою, символом Маріуполя, що прикрашає
листівки, запрошення, календарі. Вона об'єднує всіх маріупольців, незважаючи на
світогляд, віросповідання, політичні симпатії, ідеологічні переконання і
національні відмінності.

Завдяки головному архітектору міста також були побудовані \emph{Церква Костянтина та
Єлени, будинок з левами, власний будинок, школа при заводі \enquote{Нікополь}} та інші.

\ii{22_02_2021.stz.news.ua.mrpl_city.1.nilsen_150_rokiv.pic.3}

Але найулюбленішою будівлею для Віктора Олександровича стала будівля \emph{жіночого
єпархіального училища} (\emph{старий корпус металургійного інституту}). Виграний
молодим архітектором в 1907 році конкурс, в якому брали участь відомі
архітектори Катеринославської губернії, був його великою перемогою. Стара
частина металургійного інституту і сьогодні привертає увагу виразним силуетом і
оригінальним архітектурним рішенням.

Безперечно, всі будівлі, зведені талановитим Нільсеном, значно збагатили
Маріуполь. Дуже шкода, що ані час, ані люди більшість з цих будов не змогли
зберегти.

