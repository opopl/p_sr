% vim: keymap=russian-jcukenwin
%%beginhead 
 
%%file 29_10_2021.stz.news.ua.mrpl_city.1.mystecki_chellendzi_karantin
%%parent 29_10_2021
 
%%url https://mrpl.city/blogs/view/mistetski-chelendzhi-pid-chas-karantinu
 
%%author_id demidko_olga.mariupol,news.ua.mrpl_city
%%date 
 
%%tags 
%%title Мистецькі челенджі під час карантину
 
%%endhead 
 
\subsection{Мистецькі челенджі під час карантину}
\label{sec:29_10_2021.stz.news.ua.mrpl_city.1.mystecki_chellendzi_karantin}
 
\Purl{https://mrpl.city/blogs/view/mistetski-chelendzhi-pid-chas-karantinu}
\ifcmt
 author_begin
   author_id demidko_olga.mariupol,news.ua.mrpl_city
 author_end
\fi

\ii{29_10_2021.stz.news.ua.mrpl_city.1.mystecki_chellendzi_karantin.pic.1}

Черговий карантин змушує шукати нові ініціативи, які точно не дадуть
нудьгувати. І дійсно під час вимушених карантинних заходів чи самоізоляції
звільняється купа часу. Саме тому виникають онлайнові флешмоби чи креативні
мистецькі челенджі, які запускають у популярних соціальних мережах. Вони дуже
допомагають відволіктися і проявити фантазію. Хочу поділитися власним досвідом.
У 2020 році на сторінці \emph{Студії культуротворчого перфомансу (М-студія)} в
Instagram я запропонувала взяти участь в мистецькому челенджі, який запустив
музей Гетті в Лос-Анжелесі. Насправді цей челендж став для багатьох українців
найулюбленішим, адже кожен охочий міг примірити на себе образ героя відомого
шедевра мистецтва. Для цього було необхідно обрати картину одного відомого
художника та відтворити її за допомогою речей, які є вдома. Знімки розміщали із
хештегом\par\noindent\#tussenkunstenquarantaine.

\ii{29_10_2021.stz.news.ua.mrpl_city.1.mystecki_chellendzi_karantin.pic.2}

Найбільш активно брали участь студенти спеціальності культурологія
Маріупольського державного університету, адже вони докладно вивчають всі
періоди образотворчого мистецтва і знайомі з багатьма світовими шедеврами.
Особисте мене найбільше вразило досконале відтворення студенткою 4 курсу
спеціальності \enquote{Культурологія} МДУ \emph{Кариною Філатовою} автопортретів відомої
мексиканської художниці Фріди Кало.

\ii{29_10_2021.stz.news.ua.mrpl_city.1.mystecki_chellendzi_karantin.pic.3}

Дуже яскравим стало відтворення і картини Пабло Пікассо \enquote{Портрет Дори Маар},
яке представила також студентка 4 курсу спеціальності \enquote{Культурологія} \emph{Злата
Малиновська.}

А \emph{Мартищенко Галина} створила унікальні образи завдяки участі власних дітей. Так
були відтворені \enquote{Селянська дівчинка} Платонова Харитона, \enquote{Дівчина з перловою
сережкою} – одна з найвідоміших картин нідерландського художника Яна Вермера,
портрет Агостіно Паллавічіні Антоніса Ван Дейк а та картина Молодих Марії \enquote{В
саду}.

\ii{29_10_2021.stz.news.ua.mrpl_city.1.mystecki_chellendzi_karantin.pic.4}

Цікавим і доволі актуальним стало відтворення фрески \enquote{Створення Адама}
Мікеланджело Буонарроті на стелі Сикстинської капели (Ватикан) студенткою
\emph{Афанасьєвою Наталею}. Дівчина змогла завдяки найбільш потрібним речам в
період пандемії відобразити найактуальніші проблеми сьогодення.

Всього у цьому челенджі взяли участь 30 креативних маріупольців.

Тиждень тому я запустила ще один мистецький челендж, на який знову відгукнулися
студенти спеціальності \enquote{Культурологія}. Це челлендж
(\#Qurantinechallenge), який створив Віденський музей Леопольда в своєму
Instagram. За принципом він дуже схожий на вірусний \#DollyPartonChallenge, в
якому потрібно було викласти колаж з фотографій для чотирьох різних соцмереж:
LinkedIn, Facebook, Instagram і Tinder. Але цього разу треба зробити колаж з
чотирьох всесвітньовідомих картин (це можуть бути і скульптури), які передають
настрій чотирьох днів карантину.

Студентка \emph{Ольга Юр} і працівниця представила свій колаж до челенджу з наступним
описом:

\begin{itemize} % {
\item День 1. Шок – реакція на інформацію про початок карантину (1 картина
\enquote{Крик} (1893) Едвара Мунка);

\item День 2. Нудьга – коли не знаєш чим зайнятися вдома (2 картина \enquote{Молода італійка} (1900) Поля Сезанна);

\item День 3. Замислення – збиралася з думками, сидячи в парку (3 картина
\enquote{Дівчина освітлена сонцем} (1888) Валентина Сєрова);

\item День 4. Натхнення – вирішила зайнятися саморозвитком (4 картина
\enquote{Дівчина за читанням} (1770) Жана-Оноре Фрагонара)
\end{itemize} % }

Те, що вийшло в інших учасників челенджу можна подивитися на сторінці М-студії
в Instagram (\url{https://www.instagram.com/mst_udio}). Пропоную всім
долучитися до цих мистецьких челенджів, які дозволять не тільки провести цікаво
час, а й розширити власний кругозір та поглибити свої знання про світове
образотворче мистецтво.
