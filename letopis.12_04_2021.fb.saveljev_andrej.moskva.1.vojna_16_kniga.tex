% vim: keymap=russian-jcukenwin
%%beginhead 
 
%%file 12_04_2021.fb.saveljev_andrej.moskva.1.vojna_16_kniga
%%parent 12_04_2021
 
%%url https://www.facebook.com/Vandal97/posts/1401432663554228
 
%%author_id saveljev_andrej.moskva
%%date 
 
%%tags donbass,kniga,rusvesna,vojna
%%title 7 лет Славянской эпопее. Как это было расскажет книга «Война в 16»
 
%%endhead 
 
\subsection{7 лет Славянской эпопее. Как это было расскажет книга «Война в 16»}
\label{sec:12_04_2021.fb.saveljev_andrej.moskva.1.vojna_16_kniga}
 
\Purl{https://www.facebook.com/Vandal97/posts/1401432663554228}
\ifcmt
 author_begin
   author_id saveljev_andrej.moskva
 author_end
\fi

7 лет Славянской эпопее. Как это было расскажет книга «Война в 16»

Сегодня, 12 апреля, в преддверии очередного обострения на Донбассе, мы не
должны забывать, что, возможно, придется вновь встать на защиту наших людей и
территории, а также о том, как ровно 7 лет назад было положено начало Русской
весне. Многие заявляют о другой дате, но точка невозврата произошла именно в
этот день, когда группа из 52 бойцов заняла город Славянск и Краматорск.

\ifcmt
  ig https://external-frt3-2.xx.fbcdn.net/safe_image.php?d=AQEEAevG95nDxoAP&w=500&h=261&url=http%3A%2F%2Fsun9-38.userapi.com%2Fimpg%2FaIrshSaERQZX0UKu8JUs8SX5M0Sn5BsFySx1ug%2FUUuTp-qxB38.jpg%3Fsize%3D554x369%26quality%3D96%26sign%3D912b785ff47950adc1b1723220b5f8fd%26type%3Dalbum&cfs=1&ext=jpg&_nc_oe=6f1b4&_nc_sid=06c271&ccb=3-5&_nc_hash=AQH7OepPOIc4Ua33
  @width 0.4
  %@wrap \parpic[r]
  @wrap \InsertBoxR{0}
\fi

О том, как это было расскажет моя книга «Война в 16. Из кадетов в «диверсанты».
Книга посвящена моему боевому пути из Киева и Крыма в пылающий Славянск в 2014
году.

Захват стратегически важных объектов в Славянске и Краматорске

Сняв гражданку и сложив её в пакеты, мы вышли на улицу и сразу же нацепили по
полкилограмма грязи на каждый ботинок. На дорогах стояла весенняя распутица,
шёл мокрый снег.

Ненужные вещи мы оставили в машинах, а сами пошли к грузовику, в котором сидел
снайпер Глаз и раздавал оружие.

Что всё серьёзно, я понял ещё давно, но когда я держал в руках автомат — свой,
личный, я почувствовал неотвратимость нашего дела и прилив захватывающего
ощущения коллективной боевой готовности, готовности на любое развитие ситуации:
на бой, на смерть и стрельбу на поражение…

Быстро справившись с получением вооружения и приведением группы в полную боевую
готовность, командование нас разделило на небольшие подразделения, и мы
выдвинулись в неизвестность. У выносливых бойцов помимо своих вещей, висело на
шее по несколько автоматов, а в руках по баклажке с патронами (баклажки
использовались вместо цинков для удобства переноса).

Мне не достался второй автомат, но зато свой тяжёлый походный рюкзак на меня
взвалил Борода (опытный боец из нашей группы, ветеран двух чеченских кампаний).
Сам он взял патроны и мой лёгкий рюкзак, а мне, как молодому, сплавил свой
огромный баул. Хоть он и сам нёс немалый вес, но мне казалось, что всё же я
тащу больше. Помимо около 50 кг снаряги и боекомплекта, приходилось тащить за
собой куски мокрой земли, которая прилипала к ногам при каждом шаге.

Наша группа шла полевой дорогой, размокшей от снега и дождя, за командиром,
который следовал за проводником из людей Павла Губарева — он один знал путь.
Вдали виднелось зарево какого-то города или посёлка, но до него было далеко, а
вокруг нас никакого искусственного освещения. Шли долго, быстрым шагом, редко
делая привалы. Но спустя какое-то время часть подразделения сбилась с маршрута
и ушла в другом направлении. Пока её искали, мы пережидали, лёжа в траве.
Опасность была в том, что в любой момент на нас мог выйти украинский патруль. В
тот момент, не побоявшись ответственности, командование этими бойцами принял на
себя Моторола, успешно выведя их на правильный маршрут.

Долго лежать — холодно, хотелось скорее опять идти вперёд. Перед этим
марш-броском мы несколько ночей нормально не спали, поэтому теперь на привалах
клонило в сон. Но спать было равносильно самоубийству — в любом месте могли
нарваться на украинский блокпост.

Наш небольшой отряд формировался из бойцов всех возрастов. И рядом со мной шли
те, кому было и за 40. Молодые еле передвигали ноги, волоча за собой в половину
своего веса рюкзаки и сумки, а старшим во время многокилометрового похода
приходилось ещё тяжелее. Всем казалось, что их силы на пределе, и они вот-вот
упадут без чувств, но ноги сами шагали, а вдали виднелась долгожданная трасса,
на которой нас должны были ждать. Это всё, что мы знали.

И вот мы в 100 метрах от...

Читать далее: 

\url{https://vk.com/@vandalslav-k-pyatiletiu-oborony-slavyanska-sensacionnye-otkroveniya-she}

Сайт книги: \url{https://war16.ru/}

Трейлер книги: \url{https://youtu.be/Fu-nzHcxbMo}

