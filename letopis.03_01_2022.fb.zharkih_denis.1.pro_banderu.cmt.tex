% vim: keymap=russian-jcukenwin
%%beginhead 
 
%%file 03_01_2022.fb.zharkih_denis.1.pro_banderu.cmt
%%parent 03_01_2022.fb.zharkih_denis.1.pro_banderu
 
%%url 
 
%%author_id 
%%date 
 
%%tags 
%%title 
 
%%endhead 
\zzSecCmt

\begin{itemize} % {
\iusr{Raisa Savchenko}
Перепгнут. Перекрасятся.
Развернуться. Поклонятся.
Изогнутся. А аромат какой будет!?!?

\iusr{Сергей Нефедьев}

Ну, не скажите, Денис, не скажите! В.И.Ленин написал громадное количество
статей и писем. Посмотрите на полное собрание его сочинений, сколько томов! А
что написал С.Бандера? С наследием В.И.Ленина работать проще: один план ГОЭЛРО
можно было разбирать пол-семестра. А что делать с Бандерой? Здесь нужна
творческая работа - додумать самому, что же хотел \enquote{построить Степан}? Почему у
него не получилось что-то построить? Кто ему помешал? Нет! Что не говорите, а с
Бандерой гораздо сложнее работать идеологам. Наверное потому у них такие
зарплаты...

\begin{itemize} % {
\iusr{Сергей Никонов}
\textbf{Сергей Нефедьев} , 

это для государственников проще работать с Лениным. Им нужны идеи
государственного строительства, которые не даст Бандера А для тех, кто просто
хочет управлять толпой нужны Гитлер, Бандера или идеолог охоты на каких-то птиц
в Китае. Мол, перебьем каких-то людей или животных и заживем хорошо. Бандера в
основном делал то, что разрешали нацисты. Они же ничего особого не строили.
Разве-что для нужд Германии. У меня свои версии насчет зарплат идеологам: мало
отпетой сволочи и абсолютно беспринципных а грязной и лживой агитационной
работы много,.2. Они работают не только над идеологией. 3. Распиливают
средства. 4. Стрессовая у них работа. Посудите сами: \enquote{проклятые} Шарии, Путин
со своими сложными идеологиями, \enquote{пятая колонна} из честных, но умных, народ все
более трудно внушаемый пошел.


\iusr{Сергей Нефедьев}
\textbf{Сергей Никонов} Ну, я же с сарказмом написал...

\iusr{Сергей Нефедьев}
\textbf{Сергей Никонов} 

Да, окромя всего, что вы перечислили, нужно же ещё и убедить охлос, что им
живётся сегодня лучше, чем когда-либо. Многие верят...
\url{https://youtu.be/JnhA22gWu60}

\end{itemize} % }

\iusr{Юрий Александров}

Бандеровцам всё таки нужно место где бы они жили и пользовались своей
идеологией, надо место куда бы их можно было бы высылать

\begin{itemize} % {
\iusr{Сергей Вартанян}
\textbf{Юрий Александров} КАК ВЫ СМОТРИТЕ НА БИРОБИДЖАН

\iusr{Igor Bog}
\textbf{Сергей Вартанян} Польша ближе и территориально и идеологически... И работа уже есть!

\iusr{Сергей Вартанян}
\textbf{Igor Bog} какой то вы не гуманный к полякам

\iusr{Igor Bog}
\textbf{Сергей Вартанян} почему? Там уже полно этого добра прижилось, просто и семьи подтянутся!

\iusr{Сергей Вартанян}
\textbf{Igor Bog} повезло полякам с исторической памятью не то что у галичан

\iusr{Igor Bog}
\textbf{Сергей Вартанян} ... Да там примерно то же самое! Историческая память исключительно соответствующая актуальной повестке...

\iusr{Сергей Вартанян}
\textbf{Igor Bog} против москалей, я правильно понял

\iusr{Igor Bog}
\textbf{Сергей Вартанян} ну на данный момент да, ещё и коммунистов! А вообще-то они всех ненавидят!

\iusr{Сергей Вартанян}
\textbf{Igor Bog} все им должны

\iusr{Igor Bog}
\textbf{Сергей Вартанян} улавливаете основное сходство идеологии?  @igg{fbicon.wink} 

\iusr{Сергей Вартанян}
\textbf{Igor Bog} вспомнил ЭТО 

\url{https://www.youtube.com/watch?v=YCpeU6_2gAw}

\iusr{Юрий Александров}
\textbf{Сергей Вартанян} нет, только не туда

\end{itemize} % }

\iusr{Сергей Вартанян}

меня удивил наш историк коммунистической партии в институте, который после
развала СССР стал ярым почитателем УПА

\iusr{Антон Колесников}
\textbf{Сергей Вартанян} 

Среди моих бывших преподавателей и коллег подобных персонажей оказалось более
чем. Конформисты-флюгеры и просто политические проститутки.

\iusr{Сергей Вартанян}
\textbf{Антон Колесников} именно, что политические проститутки - всем дают кто заплатит

\iusr{Даниил Богатырёв}

Да, Вы правы. Сейчас поймал себя на мысли, что из всех советских философов
по-настоящему значимым и выдающимся могу назвать, разве что, Александра
Зиновьева. Да и тот диссидент. Разумеется, в советские годы творил и
по-настоящему выдающийся философ Алексей Лосев. Но он, строго говоря, был не
советским, а русским философом, так как продолжал традиции русской религиозной
философии, а университет закончил ещё до революции - в 1915 году. Вот и
получается, что довлеющая над философским образованием в СССР марксистская
идеология, в её догматизированном виде, отнюдь не способствовала раскрытию
настоящих талантов, вместо них выдвигая на первые роли посредственных
конформистов.


\iusr{Сергей Вартанян}
\textbf{Даниил Богатырёв} четко в десятку - вместо них выдвигая на первые роли посредственных конформистов - приспособленцев-ренегатов

\iusr{Евгения Савицкая}

Подождете? Увы. Поздно уже. Время ушло. И если не случится воссоединения двух
братских народов, как кому то хочется, развал страны неизбежен.

К слову сказать, братским я бы взяла в кавычки, п.ч мало кто уже так считает. И
по праву - после известных событий с 14 года по и по сей день.

\begin{itemize} % {
\iusr{Igor Bog}
\textbf{Евгения Савицкая} За семь лет оголтелой пропаганды, можно не только братьев, а и родителей ненавидеть приучить!  @igg{fbicon.wink} 

\iusr{Евгения Савицкая}
\textbf{Igor Bog}
Вы правы.

\iusr{Петр Вдович}
\textbf{Евгения Савицкая} я тоже считаю что точка бифуркации уже пройдена.
Объединение страны невозможно.

\iusr{Евгения Савицкая}
\textbf{Петр Вдович}
А это и не случиться. Невозможно. Слишком много беды....

\end{itemize} % }

\emph{Олександр Михельсон}
  · 
как-то сильно этот пост сводится к примитивизации, котоаря (якобы) шибко не
нравится автору. впрочем, ведь в тексте справедливо замечено, что поклонение
идолам - оно всегда одинаковое, независмо от цвета идолов.

 · 1 нед.
Vogel Paul
Украинская наци философия очень нудная и не то что вторичная, а пятиричная, если можно так сказать. Я вообще без понятия, что там привлекает людей? Философия "ауе" и то более романтична и обаятельна. А зло как известно обязано быть обаятельным.
 · 1 нед.
Прошкин Виктор
Хороший текст!
 · 1 нед.
Алексей Папушников
Приспособленцы без морали и стержня всегда были, во все времена - им по барабану кому орать "Слава" ,они всегда выбирают победителя и продают , когда чувствуют нового , сильнее прежнего.А держится всегда на простых людях со стержнем , которые не изменяют своим моральным, человеческим ценностям. Это они на своих плечах переносят все тяготы страшного времени, остаются всегда верны своим убеждениям. Те кто афиширует свою смену убеждений- типичный приспособленец.

\end{itemize} % }
