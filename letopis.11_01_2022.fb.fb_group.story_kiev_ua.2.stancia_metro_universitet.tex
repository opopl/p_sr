% vim: keymap=russian-jcukenwin
%%beginhead 
 
%%file 11_01_2022.fb.fb_group.story_kiev_ua.2.stancia_metro_universitet
%%parent 11_01_2022
 
%%url https://www.facebook.com/groups/story.kiev.ua/posts/1837755269754599
 
%%author_id fb_group.story_kiev_ua,venge_aleksandr.kiev
%%date 
 
%%tags gorkij_maksim.pisatel,gorod,kiev,metro,mramor
%%title Станция метро "Университет"
 
%%endhead 
 
\subsection{Станция метро \enquote{Университет}}
\label{sec:11_01_2022.fb.fb_group.story_kiev_ua.2.stancia_metro_universitet}
 
\Purl{https://www.facebook.com/groups/story.kiev.ua/posts/1837755269754599}
\ifcmt
 author_begin
   author_id fb_group.story_kiev_ua,venge_aleksandr.kiev
 author_end
\fi

Одна из киевских историй такова. Сначала было морское дно. На него многие
миллионы лет оседала органика, которая со временем сама стала морским дном и
сегодня называется известняком. Затем эти отложения претерпели так называемый
метаморфизм в результате перемещения земной коры и превратились в мрамор.

После море отступило. И однажды появились Украина, Закарпатье, Киев. В 60-х
обработанный мрамор, привезли в Киев и оформили им в числе прочих и станцию
метро \enquote{Университет}. 

\ii{11_01_2022.fb.fb_group.story_kiev_ua.2.stancia_metro_universitet.pic.1}

Вот и выходит, что время от времени мы спускаемся на как бы стилизованное
морское дно, которому от 100 до 400 миллионов лет отроду. Замечу, что учёные
различают несколько запечатленных в мраморе \enquote{зверушек}. Но доступными для
обозрения всех без исключения людей являются аммониты - головоногие моллюски.
Предшественники сегодняшних наутилусов. 

В ряде статей тот аммонит, что под бюстом Алексея Пешкова, называют самым
большим, но это не так. Есть в несколько раз больше. Он на снимке запечатлен
мной давно. Не догадался тогда поместить рядом какой-то известный предмет для
сравнения.

Теперь мои попутные субъективные впечатления. Станция метро \enquote{Университет} самая
пустынная из всех в городе. И это создает определенный шарм и настроение.
Станция метро \enquote{Университет} самая изысканная в архитектурном смысле среди
станций первой очереди метро. И последнее. Первый пятачок в жизни автора этого
материала в Киеве был опущен на следующий день после открытия станции. Проход
через турникет и спуск на эскалаторе прошли успешно. Конечно, после тщательного
инструктажа мамы. 

На снимках самый известный аммонит под бюстом писателя и самый большой - не
помню где. Последнее обстоятельство - мотив для самостоятельного поиска.

\ii{11_01_2022.fb.fb_group.story_kiev_ua.2.stancia_metro_universitet.cmt}
