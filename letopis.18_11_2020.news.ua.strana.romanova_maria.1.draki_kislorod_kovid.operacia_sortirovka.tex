% vim: keymap=russian-jcukenwin
%%beginhead 
 
%%file 18_11_2020.news.ua.strana.romanova_maria.1.draki_kislorod_kovid.operacia_sortirovka
%%parent 18_11_2020.news.ua.strana.romanova_maria.1.draki_kislorod_kovid
 
%%url 
 
%%author 
%%author_id 
%%author_url 
 
%%tags 
%%title 
 
%%endhead 
\subsubsection{Операция \enquote{Сортировка}. В чем причина трагедий?}

На минувших выходных в Украине стартовал карантин выходного дня. После чего
власти припугнули украинцев тотальным локдауном. Условие его введение, по
словам главы комитеты Рады по здравоохранению Михаила Радуцкого, - если будет
запущен протокол сортировки больных коронавирусом. \Furl{https://strana.ua/news/300689-radutskij-nazval-uslovie-dlja-vvedenija-zhestkoho-karantina-v-ukraine.html}

Простыми словами сортировка означает следующее: врачи будут выбирать, каких
пациентов лечить и давать им кислород, а каких - оставлять умирать. В
общемировой практике в таких ситуациях медики склоняются к больным средней
тяжести. Поскольку шансы спасти их намного выше, чем тяжелых. 

Украинские власти описывали этот сценарий как самый плохой и гипотетический.
Но, судя по перечисленным выше новостям из регионов Украины, негласно протокол
сортировки уже внедрен. 

Вернемся к самой идее введения карантина выходного дня. Тогда новую меру
комментировали лидеры мнений, в числе прочих был и киевский предприниматель
Алексей Давиденко, бизнес которого связан с медициной. 

Давиденко активно выступал за полный локдаун. Его единственным аргументом была
инсайдерская информация об отсутствии кислорода в больницах.

\begin{leftbar}
	\begingroup
		\color{orange}
\obeycr
\enquote{...Пример 1.
На прошлой неделе в одной из столичных больниц отданных под Коронавирус запаса кислорода оставалось на 3 часа. Под угрозой были жизни сотен пациентов.

Пример 2.
В Областных центрах уже есть тотальные проблемы с койками в реанимациях. Существует живая очередь, чтобы лечь в реанимацию.

Пример 3.
В крупных областных центрах, таких, например, как Полтава, люди уже стоят в очереди, чтобы подышать кислородом. А пока ждут свою очередь дышат в открытое окно. Вы себе вообще можете представить, что творится в районных больницах, если эти три примера - это уже реальность крупных городов вчерашнего дня.

Вы ещё не поняли.
Нет у них плана.
Нет у них золотой рыбки.
Нет у них запасного варианта

Вывод.

Завтра может быть таким, что наши муки сомнения про тотальный локдаун будут
выглядеть блекло на фоне тысяч закрытых гробов...} - написал Алексей Давиденко
в Фейсбуке. 
\restorecr
	\endgroup
\end{leftbar}

\ifcmt
pic https://strana.ua/img/forall/u/10/91/%D0%A1%D0%BD%D0%B8%D0%BC%D0%BE%D0%BA_%D1%8D%D0%BA%D1%80%D0%B0%D0%BD%D0%B0_2020-11-18_%D0%B2_17.36_.58_.png
\fi

О проблемах с доступом к кислороду \enquote{Стране} рассказывала и врач-инфекционист
Ольга Кобевко. 

\enquote{Кислорода нет. У нас - опорная больница для приема больных с Черновицкой
области, и есть всего четыре кислородных точки и масса сломанных
концентраторов, которые перегреваются и не работают. Люди синеют, задыхаются, а
помочь ты им не можешь, весь кислород роздан, и подключать новых
госпитализированных попросту некуда. Дошло до того, что люди приходят со своими
баллонами - достают через знакомых или вообще приносят промышленный. Нет
кислородной разводки, которую вполне реально было сделать за семь месяцев
эпидемии, не хватает концентраторов, которые тоже можно было бы закупить (30-50
тысяч гривен за штуку - Ред.)}, - сказала Ольга Кобевко.

Похожую ситуацию описала СМИ одесский волонтер Екатерина Ножевникова.

\enquote{В сентябре пациенты реально дрались за кислород, выдирая маску друг у друга.
Была большая проблема, на всех не хватало концентраторов. Власть 8 месяцев
про***ла, извините}, - сказала волонтер.  
