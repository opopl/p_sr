% vim: keymap=russian-jcukenwin
%%beginhead 
 
%%file 27_03_2021.fb.filatova_elena.1.iraklij_toidze
%%parent 27_03_2021
 
%%url https://www.facebook.com/permalink.php?story_fbid=271720091101474&id=100047904543554
 
%%author 
%%author_id filatova_elena
%%author_url 
 
%%tags gruzia,hudozhnik,isskustvo,istoria,mirvojna2,plakat,plakat.rodina_mat_zovet,sssr,toidze_iraklij,vojna,vov
%%title Ираклий Моисеевич Тоидзе
 
%%endhead 
 
\subsection{Ираклий Моисеевич Тоидзе}
\label{sec:27_03_2021.fb.filatova_elena.1.iraklij_toidze}
\Purl{https://www.facebook.com/permalink.php?story_fbid=271720091101474&id=100047904543554}
\ifcmt
 author_begin
   author_id filatova_elena
 author_end
\fi

27 (14 по старому стилю) марта 1902 года родился Ираклий Моисеевич Тоидзе -
Грузинский советский живописец и график. Народный художник Грузинской ССР.
Заслуженный деятель искусств РСФСР. Лауреат четырёх Сталинских премий.

Родился в Тифлисе, первоначальное художественное образование получил у своего
отца, художника и архитектора М. И. Тоидзе, с детства много рисовал,
иллюстрировал произведения грузинских писателей, много работал акварелью с
натуры. В семнадцать лет вступил в общество грузинских художников (1919). В
1920-е годы пробовал себя в портретном жанре, в основном изображал деятелей
культуры, с 1922 – начало работы в «Окнах ГрузКавРОСТА», в 1929 году вошел в
Ассоциацию художников Грузии – САРМА. Преподавал в Народной художественной
студии отца (1922–1930).

\ifcmt
  pic https://scontent-mia3-1.xx.fbcdn.net/v/t1.6435-0/p526x296/165373263_271716261101857_6599258193134302982_n.jpg?_nc_cat=104&ccb=1-3&_nc_sid=730e14&_nc_ohc=VdBOk7UyEdMAX8E_0I9&_nc_ht=scontent-mia3-1.xx&tp=6&oh=7eb0c8f09e3264fee9d6ca40516880b7&oe=60D0D68E
\fi

В 1930 году окончил Тбилисскую Академию художеств, создавал тематические
картины из жизни Грузии. Ранние живописные произведения Тоидзе («Лампочка
Ильича» 1927, Музей искусства народов Востока, Москва) сыграли значительную
роль в утверждении советской темы в грузинском бытовом жанре. Ираклий Тоидзе –
автор картины «Молодой Сталин читает поэму Ш. Руставели «Витязь в тигровой
шкуре»».

В 1931 году Ираклий Тоидзе переезжает в Москву, где начинает работать в области
плаката («Империалисты готовят новую войну против страны советов» 1934; «Под
знаменем Ленина, под руководством Сталина – вперед, за победу коммунизма!»
1936; «Приятно и радостно знать, за что бились наши люди и как они добились
всемирно-исторической победы» 1937) и книжной иллюстрации.

Героико-драматической силой образов отличаются его иллюстрации к поэме Шота
Руставели «Витязь в барсовой шкуре» (тушь, кисть, перо, 1937), иллюстрации к
произведениям грузинских поэтов (книга «История Грузии», масло, бумага, 1950),
но наибольшую известность художник получил в 1940-х годах.

В годы войны его эмоциональные плакаты пользовались невероятной популярностью и
выходили огромными тиражами: «Под знаменем Ленина – вперёд на Запад!» (1941),
«Отстоим Кавказ» (1942), «Клянусь победить врага!» (1943), «Все силы тыла на
помощь фронту!» (1943), «За Родину-мать!» (1943), «Героям, салют Родины!»
(1943), «Сталин ведет нас к победе!» (1943), «Мы заставим немецких преступников
держать ответ за всех из злодеяния! И. Сталин» (1944), «Освободим Европу от
цепей фашистского рабства!» (1945), «Все силы народа – делу полной победы над
фашизмом!» (1945). В 1943 году вышла почтовая карточка с изображением плаката
«Знамя труда – знамя победы!».

Всемирную известность получил плакат 1941 года – «Родина-мать зовёт!». Ираклий
Тоидзе придумал его, когда в мастерскую вбежала жена со словами: «Война
началась!». Уже через несколько минут художник сделал первый набросок. Спустя
неделю после начала войны появился этот знаменитый плакат, изданный миллионными
тиражами на всех языках народов СССР. В нём художник талантливо представил
исполненный романтики обобщенный образ Отчизны. Основная сила воздействия этого
плаката заключена в психологическом содержании самого образа – в выражении
взволнованного лица простой русской женщины, в её призывающем жесте.

В книге Виктора Суворова «День «М»» приводятся предположения о том, что этот
плакат был создан и размножен ещё до войны. Разослан большим тиражом в
секретных пакетах по военным комиссариатам в декабре 1940 года с указанием
вскрыть в день «М». Ни каких документально подтверждённых доказательств автором
не приводится. Некоторые устные свидетельства говорят, что плакат появился на
улицах городов уже 22 июня 1941 года, что якобы доказывает, что заготовлены и
отпечатаны плакаты были значительным тиражом заранее. Не существует и
фотодокументов доказывающих факт раннего выхода в печать плаката «Родина-мать
зовёт!». К подобным предположениям, можно относиться как к художественной
фантазии, не имеющей конкретного исторического подтверждения. За то совершенно
точно известно, что самый ранний из сигнальных экземпляров, хранящихся в
Российской государственной библиотеке, датирован 4 июля 1941 года. Известно,
что этот плакат Тоидзе стал одним из самых тиражируемых военных листов, нет
сомнения, что даже спустя 70 лет, нашлись бы доказательства в пользу этой
теории, рассчитанной на искажение и дискредитацию российской истории.

«Родина зовет своих сыновей. В такое страшное время, когда напали на нас,
хотели нас уничтожить, этот плакат, конечно, сработал тогда. Не только тогда,
он и сейчас работает», – говорит племянник Ираклия Тоидзе.

Работы Тоидзе отличают крепкий уверенный рисунок, ясная выверенная композиция,
колорит работ всегда сдержан и рационален, часто используется красный цвет,
который в сочетании с белым и чёрным придаёт его работам повышенный драматизм.
В послевоенных плакатах художник не изменяет своей манере, продолжает работать
в русле социалистического реализма, иногда прибегая к многоцветной палитре:
«Вперед к новым победам социалистического строительства!» (1946), «Озаряет
сталинская ласка будущее нашей детворы!» (1947), «Под знаменем Ленина, под
водительством Сталина, вперед к победе коммунизма!» (1949), «Спорт в СССР –
спорт миллионов!» (1956), «Во имя мира» (1959), «Мир!» (1960), «Наше знамя –
Ленинизм!» (1965), «Всегда с партией!» (1970).

Иллюстрации, тиражные и оригинальные плакаты И. М. Тоидзе хранятся в фондах
Государственной Третьяковской галереи (ГТГ), Российской государственной
библиотеки, в Государственном Русском музее (ГРМ), Государственном музее
политической истории России, Центральном музее Вооружённых Сил, Центральном
музее Великой Отечественной войны, Тобольском музее-заповеднике, частных
коллекциях России, Великобритании, Германии, Франции, США.
