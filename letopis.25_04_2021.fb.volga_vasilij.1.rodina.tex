% vim: keymap=russian-jcukenwin
%%beginhead 
 
%%file 25_04_2021.fb.volga_vasilij.1.rodina
%%parent 25_04_2021
 
%%url https://www.facebook.com/Vasiliy.volga/posts/2881574685493278
 
%%author 
%%author_id 
%%author_url 
 
%%tags 
%%title 
 
%%endhead 
\subsection{Знаете ли вы?}
\Purl{https://www.facebook.com/Vasiliy.volga/posts/2881574685493278}

Знаете ли вы, как это тяжело остаться без Родины? Остаться вроде бы в том же
городе, возле того же любимого тобой Днепра, возле той же школы в которую ты
ходил и где тебя учили твою Родину любить, и в которую ты влюбился всем своим
естеством, и всё сердце свое ей отдал, а потом вдруг, её не стало, как любимого
человека, больше чем любимого человека, ибо не стало твоего сердца, которое
вырвали и топчут у тебя на глазах сапогами и шпильками, а ты все не умираешь,
ты все смотришь на это и чувствуешь каждую боль, от каждого удара, от каждой
шпильки, и ничего сделать невозможно. 

Знаете ли вы, как это тяжело?

Я знаю. Я иногда напиваюсь и назло соседям включаю песни моей Родины так, чтобы
все слышали. Включаю не в доме, закрывшись в кабинете, а в саду, на заднем
дворе, включаю громко, сколько может дать громкости мой старый магнитофон, и
слезы катятся градом, и душа разрывается на маленькие кусочки от «Смуглянки»,
от «Дня Победы», от «Журавлей». 

Мне говорят, что наша Родина – это Советский Союз и его уже нет. Так нет же.
Наша Родина – это больше. Это тысяча лет Русской Православной Земли. Это не
только Матросов, Гастелло и Жуков – это Дмитрий Донской, Александр Невский,
Суворов, Кутузов, Пушкин, Лермонтов, Гоголь, Достоевский и еще многие миллионы,
да уже несколько миллиардов тех наших предков, которые и создали то, что весь
мир постичь не может уже тысячу лет, и что называется русской душой. 

Русский человек – это главное даже в Советском Союзе, который был отчаянной
попыткой построить Царство Божие Русское на земле. Советский Союз - это одно
только мгновение в нашей истории и в нас. Мы больше, гораздо больше одного
мгновения, мы вечные. 

И Родина наша – это как воздух, который ты не замечаешь, пока ты им дышишь,
пока тебя не лишают его, это и есть наша суть. Это и есть мы – наша Родина. И
когда нас лишают Родины, мы начинаем задыхаться.

Не знаю, доживу ли я, но бороться я буду за Родину столько, сколько будет сил.
Ибо ничего больше здесь, на земле, нет. Ибо «хорошо и приятно братьям жить
вместе». (Пс.132)

Всем сердцем обнимаю вас, братья.

