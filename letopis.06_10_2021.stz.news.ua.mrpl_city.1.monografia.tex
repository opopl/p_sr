% vim: keymap=russian-jcukenwin
%%beginhead 
 
%%file 06_10_2021.stz.news.ua.mrpl_city.1.monografia
%%parent 06_10_2021
 
%%url https://mrpl.city/blogs/view/monografiya-prisvyachena-teatralnomu-zhittyu-pivnichnogo-priazovya
 
%%author_id demidko_olga.mariupol,news.ua.mrpl_city
%%date 
 
%%tags 
%%title Монографія, присвячена театральному життю Північного Приазов'я
 
%%endhead 
 
\subsection{Монографія, присвячена театральному життю Північного Приазов'я}
\label{sec:06_10_2021.stz.news.ua.mrpl_city.1.monografia}
 
\Purl{https://mrpl.city/blogs/view/monografiya-prisvyachena-teatralnomu-zhittyu-pivnichnogo-priazovya}
\ifcmt
 author_begin
   author_id demidko_olga.mariupol,news.ua.mrpl_city
 author_end
\fi

\ii{06_10_2021.stz.news.ua.mrpl_city.1.monografia.pic.1}

7 жовтня відбулася презентація мого видання \emph{\enquote{Театральне життя Північного
Приазов'я (середина XIX – XX ст.)}}. Звісно, хочеться розповісти про нього
більше.

Зізнаюся, я дуже рада представити широкому загалу другу свою монографію, що
висвітлює історію театрального мистецтва Північного Приазов'я. Але це видання є
більш фундаментальним, адже воно створено після захисту моєї дисертації і
головні положення наукової роботи увійшли в представлену книгу.

\emph{Цією монографією я засвідчую глибоку повагу всім професійним  аматорським
театральним колективам, які зіграли вагому роль у формуванні та становленні
театрального мистецтва Північного Приазов'я.}

Одразу ж хочу висловити щиру подяку своєму науковому керівнику \emph{\textbf{Юлії Сергіївні
Сабадаш}} за постійну підтримку та натхнення, рецензентам за об'єктивну та
неупереджену оцінку монографії. Мені запропонували комусь присвятити це
видання, і я, довго не вагаючись, одразу вирішила, що воно буде присвячено моїй
мамі, \emph{\textbf{Демідко Ірині Борисівні}}, яка незважаючи на власну хворобу і життєві
труднощі, завжди підтримувала мене і наполягала, щоб я закінчила своє наукове
дослідження. Насправді її колосальна віра в мене допомагала не опускати руки і
йти до кінця. Були періоди, коли ще під час написання дисертації я думала, що
нічого не вийде. Дійсно займатися наукою в наш час не так легко і зовсім
недешево. Але є люди, завдяки яким з'являються додаткові сили і бажання не
зупинятися. Для мене такою Людиною є моя мама. Окремо хочу щиро подякувати
народній артистці України \emph{\textbf{Світлані Іванівні Отченашенко}}, яка поділилася зі мною
особистим архівом, а також завжди давала дуже корисні поради та доценту кафедри
культурології Маріупольського державного університету \emph{\textbf{Юзефу Мойсейовичу
Нікольченку}}, завдяки якому ще під час моїх студентських років я вирішила
пов'язати своє життя з наукою.

\ii{06_10_2021.stz.news.ua.mrpl_city.1.monografia.pic.2}

Тепер перейду безпосередньо до монографії. Насправді протягом своєї історії
театральні заклади регіону, незважаючи на свою провінційність, здійснювали
важливі соціальні функції, зокрема, сприяли згуртуванню членів спільноти,
консолідації зусиль задля культурного розвитку громади, формуванню моральних
цінностей та естетичних поглядів. Театральна культура Північного Приазов'я є
дійсно унікальною і самобутньою і на сторінках своєї книги я намагаюся це
довести.

Джерельну базу моєї монографії становить широке коло документів і матеріалів із
фондів центральних, обласних, поточних архівів, бібліотек, особистих архівів,
фотодокументів, матеріалів інтерв'ю, періодичних видань.

Більша частина матеріалів зберігається у фондах центральних державних архівів
(ЦДАВО України, ЦДАГО України), місцевих архівів і музеїв (Держархіви
Автономної республіки Крим, Запорізької області, Одеської області), поточних
архівів (Донецький академічний обласний драматичний театр (м. Маріуполь), ПК
Молодіжний), фондах Музею театрального, музичного та кіномистецтва України,
Маріупольського краєзнавчого музею, Бердянського\par\noindent краєзнавчого музею. У
сукупності було вивчено 107 справ. \emph{У науковий обіг вперше введено 102 архівних
джерела.}

Водночас завдяки дослідженню вперше в науковий обіг введено 308 світлин, з них
286 опубліковані в моїй першій монографії \enquote{Ілюстрована історія театральної
культури Маріуполя} (2017 р.), а 22 – в \emph{новому виданні}.

У книзі висвітлюється діяльність 15 професійних театральних закладів Бердянська
та Маріуполя. Серед них Зимовий театр в м. Бердянськ, Зимовий театр в м.
Маріуполь, Державний грецький театр в м. Маріуполь (Грецький
робітничо-селянський театр), Державний драматичний театр ім. 13-річчя жовтня,
Маріупольський музично-драматичний театр ім. Т. Шевченка, Маріупольський театр
ляльок, Бердянський театр драми і комедії,  Бердянський театр оперети,
Державний театр російської драми (м. Маріуполь), Донецький обласний драматичний
російський театр (м. Жданов) (з 1989 р. – м. – Маріуполь) тощо.

Серед аматорських мені вдалося висвітлити творчий шлях 15 колективів. Зокрема,
Маріупольського музично-драматичного товариства, Бердянського аматорського
музично-драматичного гуртка, Народного театру ПК \enquote{Азовсталь}, Українського
драматичного гуртка при ПК \enquote{Азовсталь}, Іллічівського російського драматичного
гуртка клубу ім. Карла Маркса, Народного театру оперети в Бердянську,
Драматичного колективу заводу \enquote{Дормаш}, Народного театру Бердянського міського
Будинку культури; театру-клубу \enquote{Діалог} Палацу культури будівельників тресту
\enquote{Азовстальбуд}; театр сатири Палацу культури \enquote{Іскра}.

Слід зазначити, що найменш дослідженими насьогодні залишається розвиток
професійного й аматорського театрів на теренах Північного Приазов'я протягом
Національно-визвольної боротьби українського народу, утвердження радянської
влади та німецької окупації. Дослідження цих періодів позначено ідеологічною
заангажованістю, тому їхнє об'єктивне вивчення в моєму дослідженні, на мій
погляд, сприятиме відновленню повноти історичної пам'яті.

\ii{06_10_2021.stz.news.ua.mrpl_city.1.monografia.pic.3}

На сторінках монографії вперше запропонована періодизація театрального життя
приазовського краю. У початковий період діяльності театральних закладів регіону
(\emph{1847–1899-ті рр.}) головними особливостями діяльності театральних закладів
регіону стали: визначальна роль антрепренерів, різноманітний репертуар,
постійна наявність у регіоні гастролюючих труп, розвиток аматорського руху та
активна благодійна діяльність акторів. На початку XX ст. у Маріуполі та
Бердянську вистави можна було побачити на декількох сценах, що сприяло розвитку
не тільки професійного театру, а й діяльності аматорських колективів і приїзду
видатних гастролерів. Досліджено специфіку творчого становлення театрального
життя приазовського краю в контексті суспільних трансформацій \emph{1920–1930-х
років}. Створення та діяльність першого в СРСР грецького театру посприяло
пожвавленню театрального життя регіону. Впродовж п'яти років з часу свого
заснування і до трагічного знищення цей колектив був \enquote{трибуною громадської
думки}, пропагуючи у своїх виставах самобутню культуру греків Приазов'я.
Непересічне значення для розвитку театрального життя Північного Приазов'я мала
діяльність в роки окупації театрів у Бердянську та Маріуполі. Протягом другої
половини \emph{40-х років XX століття} театральні заклади регіону відчули на собі
ідеологічний тиск, контроль влади та розвивалися під гаслом соціалістичного
реалізму. Впродовж 1960-х років у регіоні постав і розпочав роботу Обласний
драматичний російський театр м. Маріуполя, що посприяло подальшому піднесенню і
підйому творчої активності митців та пожвавленню театрального життя Північного
Приазов'я. Особливе місце посідають 1970–1980-ті роки, які стали періодом
творчого піднесення в житті маріупольського театру.

Важливими в монографії є іменний покажчик (понад 500 прізвищ видатних
театральних діячів що спряли розвитку театральної культури Приазов'я,
драматурги, театрознавці та інші) та покажчик театральних закладів Північного
Приазов'я (25 театральних колективів, що діяли  досліджений період).

\ii{06_10_2021.stz.news.ua.mrpl_city.1.monografia.pic.4}

Сподіваюся, що дослідження особливостей розвитку театрального життя Північного
Приазов'я від середини XIX–XX століть сприятиме формуванню більш повної картини
історико-культурно\hyp{}го життя нашої держави і, зокрема Північного Приазов'я,
оновленню матеріалів навчальних курсів з історії театру, історії української
культури, а також стануть у пригоді історикам, культурологам, краєзнавцям,
театрознавцям та всім, хто цікавиться історією і культурою України. До цієї
публікації додаю і електронну версію монографії. Маю надію, що вона знайде
свого читача.
