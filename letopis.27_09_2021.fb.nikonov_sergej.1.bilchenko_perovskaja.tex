% vim: keymap=russian-jcukenwin
%%beginhead 
 
%%file 27_09_2021.fb.nikonov_sergej.1.bilchenko_perovskaja
%%parent 27_09_2021
 
%%url https://www.facebook.com/alexelsevier/posts/1591002737911702
 
%%author_id nikonov_sergej,bilchenko_evgenia
%%date 
 
%%tags bilchenko_evgenia,poezia,pravoslavie,vera
%%title БЖ. Перовская
 
%%endhead 
 
\subsection{БЖ. Перовская}
\label{sec:27_09_2021.fb.nikonov_sergej.1.bilchenko_perovskaja}
 
\Purl{https://www.facebook.com/alexelsevier/posts/1591002737911702}
\ifcmt
 author_begin
   author_id nikonov_sergej,bilchenko_evgenia
 author_end
\fi

\ifcmt
  ig https://scontent-yyz1-1.xx.fbcdn.net/v/t1.6435-9/243273899_1591001427911833_5431823397126774206_n.jpg?_nc_cat=106&ccb=1-5&_nc_sid=730e14&_nc_ohc=XbMFS8QjBlQAX9sAblQ&_nc_ht=scontent-yyz1-1.xx&oh=8a0acd64190ed92529cb7d6a129e6323&oe=6177E354
  @width 0.4
  %@wrap \parpic[r]
  @wrap \InsertBoxR{0}
\fi

Снова произведение Евгении Бильченко. А я снова репощу ее текст. Не мне судить о нем.  Итак.
БЖ. Перовская
Девушка с белым платком ходит почти, как Фрида,
По Грибоедовскому каналу Голгофы своей обиды
И всякий раз подаёт платок не народовольцу, Ромулу
Нового Города солнца, а батюшке, Александру Второму.
Если бы перформансистка Софья, если бы
Знала, что спасена была именем Достоевского,
Ей бы легче было управиться с этим платком из ада,
С "Дзержинским районным моргом" во времена блокады,
С хранилищем для картошки в ВОВ голодные годы,
В девяностые, когда сытым монстром стала её свобода.
Софья смогла бы, смогла бы: точное правило есть такое:
Попросить Булгакова попросить царя попросить ей притон покоя
У Того, чей Храм на крови и крове, на Одессе и на Донбассе,
Но Софья ходит, как Pokémon Go, - привидением в общей массе
Таких же героев из поп-киношек про единорога-гаубицу,
А крест не лежит, не стоит, не тонет, не рушится - Воздвигается.
27 сентября 2021 г.
