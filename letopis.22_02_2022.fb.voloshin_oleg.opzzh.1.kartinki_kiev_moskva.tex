% vim: keymap=russian-jcukenwin
%%beginhead 
 
%%file 22_02_2022.fb.voloshin_oleg.opzzh.1.kartinki_kiev_moskva
%%parent 22_02_2022
 
%%url https://www.facebook.com/oleg.voloshin.7165/posts/5236930263006659
 
%%author_id voloshin_oleg.opzzh
%%date 
 
%%tags istoria,kiev,moskva,sravnenie
%%title Нравятся мне эти картинки с соборами Киева X-XI веков и лесами близ реки Москва
 
%%endhead 
 
\subsection{Нравятся мне эти картинки с соборами Киева X-XI веков и лесами близ реки Москва}
\label{sec:22_02_2022.fb.voloshin_oleg.opzzh.1.kartinki_kiev_moskva}
 
\Purl{https://www.facebook.com/oleg.voloshin.7165/posts/5236930263006659}
\ifcmt
 author_begin
   author_id voloshin_oleg.opzzh
 author_end
\fi

Нравятся мне эти картинки с соборами Киева X-XI веков и лесами близ реки
Москва. Можно такой же лес на месте Вашингтона или на Манхеттене того периода
нарисовать. Тоже достоверно будет. А в Париже тогда пару десятков тысяч человек
жили и в лужах свиньи на улицах лежали. История вещь интересная. Но живем мы в
2022 году. 

И в этом 2022 в Киеве сегодня на улицах с обшарпанными фасадами домов,
переполненными мусорными баками, ржавыми маршрутками и ямами на дорогах, как
после бомбежки, каждый сильный дождь на Эспланадой и в ещё ряде мест создаёт
реальное болото. А Москва выглядит чище и ухоженней почти всех европейских
столиц. 

Вавилон на территории Ирака был центром великой цивилизации, когда славяне ещё
собирательством занимались. Расскажите это иракцам, которые через белорусские
леса пробираются в Польшу.

\begin{itemize} % {
\iusr{Виктор Чесноков}

Поддерживаю. Я всегда говорил, что сравнение с Москвой абсолютно некорректное.
Москва никогда не позиционировала себя как древний город. Давайте сравнивать
Киев с Суздалем или Великим Новгородом, как уже было сказано свыше.

А вообще, я лично не был в Москве лет 15. Сказать ничего не могу. Но вот если
начнем сравнивать Киев с Петербургом, о котором я могу судить, то будут очень
неприятные для нашего родного города выводы.

\iusr{Oleg Voloshin}
\textbf{Виктор Чесноков} при том, что Питер тоже не образец. Вон Шнуров на эту тему песню даже написал ))

\iusr{Andre Pigulevsky}

Согласен, но разве мы начинаем экскурс в историю? Не Путин ли возвращает все
назад, игнорируя сегодняшнюю реальность.

Я не против, пускай Москва будет чистая и ухоженная, пускай только к нам в
Украину не лезут.

\begin{itemize} % {
\iusr{Oleg Voloshin}
\textbf{Andre Pigulevsky} 

у меня больше вопрос: почему Киев в таком состоянии? не лучше ли было
«українцям об’єднатися та втерти носа москалям» в этом вопросе?! А то вот снова
добровольческие подразделения в фб собирают пару тысяч долл на рации и тд. Вот
и весь патриотизм нашего бизнеса. Посты писать многие умеют.

\iusr{Andre Pigulevsky}
\textbf{Oleg Voloshin} ну кто пишет, а кто помогает ) если русские сюда сунутся, ощутят на своей шкурке )
\end{itemize} % }

\iusr{Сергей Киселев}

Вообще-то Москва это новая столица Северо-Западной Руси.

Там есть гораздо более древние города. Ростов, Владимир, Суздаль, Переяславль,
Ярославль и т. д.

Я уж не говорю о Великом Новгороде, где Князь Владимир родился

\end{itemize} % }
