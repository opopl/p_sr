% vim: keymap=russian-jcukenwin
%%beginhead 
 
%%file 13_12_2020.news.ru.ukraina_ru.volkov_pavel.1.ob_ukrainskoi_nezavisimosti
%%parent 13_12_2020
 
%%url https://ukraina.ru/exclusive/20201213/1029884239.html
 
%%author Волков, Павел
%%author_id volkov_pavel
%%author_url 
 
%%tags ukraina,independence
%%title Об украинской независимости, или Как не обмануться в мире обмана
 
%%endhead 
 
\subsection{Об украинской независимости, или Как не обмануться в мире обмана}
\label{sec:13_12_2020.news.ru.ukraina_ru.volkov_pavel.1.ob_ukrainskoi_nezavisimosti}
\Purl{https://ukraina.ru/exclusive/20201213/1029884239.html}
\ifcmt
	author_begin
   author_id volkov_pavel
	author_end
\fi

\ifcmt
pic https://cdn1.img.ukraina.ru/images/102866/54/1028665478.jpg
\fi

\textbf{Некоторые в счастливом экстазе ежегодно празднуют день независимости Украины,
другие воспринимают этот день как личную и общественную трагедию, лишившую их
очень многого}

Некоторые хотят сохранить независимость, другие были бы не против
присоединиться в той или иной форме к крупным межгосударственным объединениям
вроде ЕС или Евразийского союза. Некоторые убеждены, что «революция
достоинства» сделала Украину по-настоящему независимой, другие — что майдан
привел ее в состояние полуколонии или даже колонии.

Все это мнения, и таких мнений может быть еще миллион. Чтобы от мнения перейти
к знанию и абсолютно ответственно установить, радоваться нам или печалиться от
нынешнего состояния Украины, необходимо разобраться с тем, что такое
государство. Не какие виды государства бывают, а что это такое сущностно, то
есть дать определение.

\subsubsection{Приснившееся государство}

Такие, занимавшиеся в разное время этим вопросом философы, как Платон и Гегель,
считали, что государство — это материальное воплощение абсолютной идеи. То или
иное государство возникает из идеи государства. А идея откуда берется? Из
ниоткуда, она вечная. Данный подход называется объективным идеализмом или
религией без бога. Объективным потому, что признается реальность и научная
познаваемость нашего мира, идеализмом потому, что наш мир — ухудшенная копия
другого, настоящего, вечного мира. Как там, так и здесь, но хуже.

Отсюда идея некоторых политических групп о том, что раз на небе монархия, а не
республика, значит, и на земле должна быть монархия. Такое доказательство из
заведомо недоказуемой предпосылки называется софистикой. Короче говоря, если
вместо абсолютной идеи поставить бога, то получится всем нам прекрасно
известная религия. Это не плохо и не хорошо, на то она и вера, чтобы верить. В
науке же все надо доказывать, поэтому объективный идеализм для понимания
сущности государства нам с вами никак не походит.

Конечно, объяснять явление государства наличием в иных мирах вечной идеи этого
государства все равно, что объяснять закон всемирного тяготения тем, что
предмет падает, если его отпустить. А почему падает предмет? Потому что закон
всемирного тяготения. Объяснили мы что-то? Нет. Это тавтология. Но есть и
гораздо более экстравагантные взгляды на разбираемый нами вопрос.

Некоторые современные философы, называющие себя постмодернистами или
постструктуралистами, говорят, что никакого государства нет.

Но вот не надо сейчас лезть в свой паспорт и сверяться, стоит ли там до сих пор
отметка о гражданстве, и думать о том, как сложится ваша жизнь, если паспорт
заберут.

Например, правоохранительные органы того государства, которого нет. Не надо
этого всего, философы, наверное, уж поумнее нас с вами будут, раз издают умные
книжки. Такой взгляд на окружающую действительность может показаться шуткой, и
тем не менее теория неокочевников, которые бегают по всей планете, уже не зная,
кто они такие, размытие идентичностей, глобализация, цифровизация,
информационное общество, транснациональные корпорации вместо государств, теория
игр, теория коммуникаций и т.д. — все эти теории в конечном итоге говорят одно:
не надо бороться за улучшение жизни, поскольку и бороться-то не с кем.
Объективной реальности нет, все игра, все коммуникация аватарок в «Фейсбуке»,
где за изображением 14-летней девочки прячется старый лысый и толстый мужик или
вообще никто не прячется. Бог его знает, может, это уже искусственный интеллект
Скайнет там с нами общается. Как узнать? Никак.

Мир познается с помощью наших органов чувств, а мы не можем быть уверены,
передают ли они нам то, что есть на самом деле или что-то другое. Поэтому я
знаю, что я есть, а есть ли государство — без понятия. Мало того, я даже не
знаю, есть ли вы или вы мне просто снитесь, как материальный мир снится Будде.
Такой тип мировоззрения называется субъективным идеализмом. Логически ничего
доказать такому субъективисту нельзя. Вы скажете ему, что цифровое общество не
существует, поскольку все необходимое нам для жизни производилось и
производится промышленностью (даже компьютеры и флешки с информацией), а он вам
ответит, что не существует вас — вы ему приснились.

Такой подход еще менее научен, чем идеализм объективный, поскольку утверждает,
что в природе и обществе нет никаких закономерностей (или они неизвестны, и
узнать их нельзя). Все постоянно течет и постоянно меняется, не сохраняясь в
объективных законах. Но даже если отбросить шутки, придется констатировать, что
субъективно идеалистические методы совершенно не походят для того, чтобы
разобраться с сущностью государства, и ясно почему. Потому что для выяснения
сущности чего-либо, надо для начала точно знать, что это что-то есть. Изучать
то, чего нет, невозможно.

Если есть идеализм, значит, должен быть и материализм. Их, кстати, тоже много
видов. Например, материализм вульгарный. Ярчайшие его представители, философы
\textbf{Бюхнер}, \textbf{Фогт и Молешотт}, говорили, что мысли в голове появляются так же, как
желудочный сок в желудке.

Вульгарный материализм про государство нам может сказать только то, что это
государство есть, констатировать факт его объективного наличия и все. Много ли
нам это дало? Кажется, что не очень. Но вообще-то кое-что дало. Во-первых, хотя
бы то, что государство есть, значит, его можно изучать. Во-вторых, земное
государство не привязано к государству небесному, которое мы, конечно, никак
видоизменять не можем, а значит, в земном государстве можно проводить любые
преобразования как для улучшения его, так и для ухудшения. А когда-нибудь
человечество сможет и вообще жить без государств. Не так, как постмодерные
неокочевники — те живут без государства только в своих головах, — а просто с
другой формой человеческой самоорганизации.

\subsubsection{В себе или не в себе?}

Итак, мы узнали о государстве довольно много, но так и не поняли, что это
такое. Не поняли потому, что понять — значит выразить в понятиях, то есть дать
определение. Знаете определение определения?

Качество, которое есть в себе, в простом нечто и сущностно находится в единстве
с другим моментом этого нечто, с в-нем-бытием, есть определение. Немного
непонятно, правда ведь? Но это нормально, наука всегда такая. Вряд ли всем
будет очень понятно, если сейчас изобразить здесь, например, формулу Тейлора из
матанализа. Чтобы разобраться, нужна подготовка. В философии все немного проще,
чем в математике, поскольку она оперирует языком, которым мы все ежедневно
пользуемся.

Так как понять определение определения? Качество — что здесь неясного? У
каждого из нас есть разные качества — ум, смелость, верность. Иногда, правда, и
глупость с трусостью и вероломством встречаются, но хорошие люди такие качества
стараются в себе побороть. Так вот, определение — это не любое качество, а
такое, которое «в себе». Очень странно звучит, не так ли? «В себе» значит
равенство с собой. Вы в себе или не в себе? Я, например, когда пишу эту статью,
точно в себе, в нормальном для себя состоянии. Вот если у меня вдруг случится
истерика, то буду не в себе, а потом снова приду в себя и продолжу писать
статью. Поэтому для определения нам нужно взять такое качество, которое
сохраняется в любых условиях, даже в таких, когда я не я и сам на себя не
похож. Такое случается, ведь даже умный человек иногда совершает глупые
поступки.

Согласитесь, что вы ни в жизни не перепутаете стол с жирафом, даже если первый
будет покрашен в желтый с пятнышками, то есть обладать одним из качеств жирафа.

А почему? Потому что сущностным, то есть таким, которое сохраняется всегда,
качеством стола и жирафа цвет не является. Стол останется столом, будь он
коричневым, черным или зеленым, будь он на четырех ножках или на трех, будь он
деревянным или пластиковым. Жираф тоже не превратится в антилопу, если его
облить краской соответствующего цвета, и не станет страусом, если ему отрубить
две ноги из четырех. Он останется жирафом-инвалидом, но все равно именно
жирафом. Что значит качество, которое сохраняется в простом нечто? Значит, что
ваше качество в вас, а не во мне, качество жирафа в жирафе, а не в крокодиле.
Не надо судить других по себе, у них не такие качества, как у вас.

«В единстве с другим моментом этого нечто, с в-нем-бытием», то есть в единстве
с таким вашим качеством, которое превращает вас в вашу противоположность. Если
вы умный человек, то ваша противоположность — глупость. Но она же в вас, а не в
ком-то другом, если она ваша. Правда, раз вы умный, то успешно с ней боретесь,
и ваш ум как качество проявляется вовне. Например, в написании умных статей. А
если ваш ум ни в чем не проявляется, так вы и не умный вовсе. Ну или я, а то
про вас как-то неудобно такое говорить. Поскольку дворцы культуры у нас в
основном закрыты или переделаны под офисы, а различные кружки стоят дорого,
очень часто бывает, что у родителей нет средств, чтобы отдать детей на танцы
или рисование, или еще куда-то. Тогда талант ребенка не развивается и вовне
никак не проявляется. То есть мы догадываемся, что он талантливый, но не более
того. Талант не становится определением юного гражданина.

Итак, определение — это такое прочное и устойчивое качество, которое
проявляется вовне. Найти искомое качество можно только исторически, увидев в
разных, противоречивых формах государства что-то такое, что сохраняется в
веках. Указанный подход называется диалектическим материализмом, приложенным к
истории, или историческим материализмом.

\subsubsection{Демократическая диктатура}

\video{https://youtu.be/ZO08VPVm9MQ}

Вот говорят, что демократия — это изобретение современной западной мысли. Якобы
в России всегда «тоталитаризм» (кстати, total — целое, то есть у нас целостное,
а не мозаичное мировоззрение), а на западе полная свобода. Полная — в смысле
тоже тотальная. Получается и там тоталитаризм, только другой.

Иногда, правда, вспоминают, что демократия была еще в Древней Греции. Правда,
там имело место рабовладение, а в современных Европе и США вроде как такого уже
нет. Понятно, что ничего не понятно.

А вообще-то демократия была всегда. В античном рабовладельческом государстве
демократически принимали решения рабовладельцы. Рабам ничего решать нельзя,
поскольку они не люди, а говорящие орудия. Тот факт, что рабовладельцев было
сильно меньше, чем рабов, демократию никак не отменяло. Казалось бы, в Риме, по
крайней мере в имперский период, о демократии и речи быть не могло. Но нет,
сенат существовал и при императорах, а если император делал что-то, что не
нравилось высшей знати, для него это заканчивалось плохо.

Были периоды, когда гвардия ставила новых императоров чуть ли не ежегодно. Так
что принимать решения без оглядки на интересы правящего класса не могли ни
Цезарь, ни Август. Пришедшие на территорию Рима варвары принесли еще больше
демократии, вождистской, племенной, раннефеодальной.

Сколько бы у скандинавов ни было конунгов и ярлов, а ключевые вопросы решались
на регулярных тингах. У нас такая демократия разлагающегося родоплеменного
строя выражалась в традиционных вече, а князь выполнял роль военного вождя во
время войны. Все же хорошо помнят, как новгородцы несколько раз призывали
\textbf{Александра Невского}, а потом прогоняли его. И княжеская дружина в самом городе
не располагалась — ну ее, это же просто опасно для горожан. Военные сделали
дело, спасибо, до свидания.

Удивительно, но демократии было полно даже при монархии, вплоть до абсолютизма.
В наифеодальнейшем из всех феодальных государств — Испанской империи — при
«кровавом тиране» Филиппе II работали Кортесы, сословные представительства. При
нашей версии Филиппа II — «кровавом тиране» Иване Грозном — были созданы
земские соборы — те же сословные представительства, благодаря которым Россия
пережила смутное время. Правда, демократия при феодализме такая же
специфическая, как и при рабовладении. Сословное представительство-то есть, но
кто там заправляет, даже если все сословия представлены равным образом?
Конечно, феодалы, ведь в их руках главное средство производства Средневековья —
земля. А значит, в их руках деньги, оружие и самые передовые идеи своего
времени.

Не менее специфична демократия и в последующем буржуазном периоде. Несмотря на
то что именно такой тип демократии нам сегодня представляют как «истинный»,
сущностно он от предыдущих типов ничем не отличается. Сословных
представительств нет, выборы всеобщие и равные, но это равенство неравных. Раз
капитал — основа данного общества, то и избираются в представительские органы
те, у кого капитала больше. Вы видели когда-нибудь, чтобы шахтер становился
депутатом парламента? А президентом? Бывший шахтер, который сейчас работает,
скажем, топ-менеджером у угольного олигарха, может. А вот действующий, который
прямо сейчас спускается в шахту, нет. Откуда он возьмет деньги на предвыборную
кампанию, залог, подкуп членов избиркома и т.д.? Так что под обществом равных
возможностей имеются в виду равные возможности для тех, у кого есть средства. В
античности у рабовладельцев тоже равные возможности были. Правда, у кого больше
рабов, тот и равнее.

В последнее время либерально настроенная публика любит выступать «за честные
выборы», но сколь бы честными они ни были, при рабовладении управлять
государством будут рабовладельцы, при феодализме — феодалы, а при капитализме —
капиталисты. Видимо, либеральная публика, говоря о честных выборах, имеет в
виду, что к власти должны прийти капиталисты, которых обслуживает именно эта
публика, а не какие-то другие. При социализме тоже демократия — в виде съезда
рабочих советов или диктатуры рабочего класса. Почему такой строй называют
тоталитарным? Наверное, потому, что он в интересах всего (total) общества, а не
отдельных его групп.

Еще очень любят говорить о народовластии и народном государстве. Но это такое
же пустое выражение, как честные выборы. Если государство есть, то оно
классовое, а не народное, если народное, то уже не государство, а следующий
этап общественного развития, речь о котором мы сейчас не ведем.

\subsubsection{Зависимая независимость}

Именно отсюда становится понятным, что такое государство при всех своих
многообразных видах сущностно. Это всегда диктатура, то есть власть, не
ограниченная законами. Кто у власти, тот законы и пишет, если они перестают
устраивать — переписывают. А кто не у власти, вынуждены их выполнять под
угрозой государственного насилия — штрафов, тюрьмы, смертной казни. Поскольку
диктатуры одного человека, даже царя и императора, никогда не бывает (всегда
демократия), речь идет о диктатуре группы людей, которая держит власть
благодаря имеющимся у нее возможностям, прежде всего, конечно, экономическим.

Если вернуться к выведенному нами определению определения, то непреходящее,
постоянное качество государства — это диктатура. Данное качество не меняется с
изменением разнообразных государственных форм и проявляется вовне в виде
законов, которые принуждают исполнять. Поэтому определение государства —
диктатура правящего класса.

Вопрос же независимости — это всегда вопрос того, может ли действовать правящий
класс той или иной страны сообразно своим интересам, или он вынужден принимать
решения в интересах правящего класса другого государства. Нужны ли особенные
комментарии по поводу самостоятельности решений правительства Украины после
пленок Деркача, МВФовских реформ и эпопеи с увольнением главы НАБУ Сытника (ни
одно министерство не может его уволить, поскольку не нанимало, и даже
неизвестно, к какой ветви украинской власти относится НАБУ)?

На этом можно было бы и завершить, но перед думающим гражданином раньше или
позже обязательно встает этический вопрос — хорошо или плохо, когда правящий
класс твоего государства независим? Зная наших правителей, вы отказались бы от
такого типа государственной зависимости, как, скажем, подчинение решениям ЕСПЧ?

Ведь национальные суды у нас как раз независимые, то есть подчиняются
национальному правящему классу и принимают решения в его пользу. Получается,
что зависимость национального правительства от ЕСПЧ делает немного более
независимым народ данного государства. А независимость правящего класса
совершенно не означает, что народу шикарно живется. Такая вот диалектика.
Правда, нам с голубых экранов рассказывают, что жить после майдана стало лучше,
чем до, но мы заглядываем в кошелек и видим на практике несоответствие
сообщаемого с объективной реальностью. Практика — критерий истины. В этом ответ
постмодернистам, которые считают, что истину постичь нельзя. Оказывается,
можно. И в этом различие между сущностью и кажимостью. «Если на клетке слона
прочтёшь надпись «буйвол», не верь глазам своим», — объяснял нам Козьма
Прутков, как воспринимать информацию большинства украинских телеканалов.

На все непростые политические, исторические, культурологические,
обществоведческие вопросы можно достаточно легко найти однозначные ответы, если
вооружиться правильным методом — не односторонним, а таким, который
рассматривает явления во всей их противоречивой сложности, к чему мы и
призываем читателя.

