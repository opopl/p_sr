% vim: keymap=russian-jcukenwin
%%beginhead 
 
%%file slova.rodina
%%parent slova
 
%%url 
 
%%author 
%%author_id 
%%author_url 
 
%%tags 
%%title 
 
%%endhead 
\chapter{Родина}

КАК не нужно торговать \emph{Родиной}...  Урок Армении. Нас это тоже
касается... Спросите сколько ПОЛЯКИ сдоили с ЕС за аналогичные вещи ))) 10
июня, на заседании правительства Армении было одобрено соглашение с Европейским
Союзом о финансовой поддержке республики в размере трех миллионов евро за отказ
Армении от использования названия \enquote{коньяк} при маркировке алкогольной
продукции.  Требование об отказе от названия \enquote{коньяк} предусмотрено Соглашением
о всеобъемлющем и расширенном партнерстве, которое было подписано еще в ноябре
2017 года. В силу соглашение вступило с 1 марта 2021 года.  Согласно этому
документу Армения должна отказаться и от называния \enquote{шампанское}.  Сообщаю... По
минимуму такое решение стоит 3 МИЛЛИАРДА... но никак не миллиона ))),

%%%cit
%%%cit_pic
\ifcmt
	pic https://avatars.mds.yandex.net/get-zen_doc/5098316/pub_60c3819603088a3a1a36d8ea_60c38e2be72f9332f7e8d619/scale_1200
  caption К середине XVII века почти весь Киев лежал в заброшенных руинах. Рисунок Абрагама ван Вестерфельда, посетившего Киев в 1651 году
\fi
%%%cit_text
А южные русские за несколько столетий не приобрели совсем ничего. Их угнетали
монголы, потом поляки с литовцами, то есть совершенно чуждые народы, которые
кроме поборов ничего больше с захваченных ими земель получать не хотели. Потому
и развалилось сельское хозяйство на Украине, и не возникло никаких промышленных
центров. Малороссы могли только воевать под знаменами кого угодно, и то только
большими массами, без проявления какой-либо индивидуальности. И когда их братья
великороссы принялись за восстановление своей былой \emph{родины}, они столкнулись с
искренним непониманием своих бывших братьев
%%%cit_title
\citTitle{Почему великороссы стали такими сильными, а малороссы остались такими слабыми?}, 
Исторический Понедельник, zen.yandex.ru, 11.06.2021
%%%endcit

