% vim: keymap=russian-jcukenwin

%%beginhead 
 
%%file 15_09_2021.fb.fb_group.story_kiev_ua.1.kiev_derevnja_cvetochnica
%%parent 15_09_2021
 
%%url https://www.facebook.com/groups/story.kiev.ua/posts/1753326301530830/
 
%%author_id fb_group.story_kiev_ua,stepanov_farid
%%date 
 
%%tags gorod,kiev
%%title «КИЕВСКАЯ ЦВЕТОЧНИЦА» Или: Киев – большая деревня
 
%%endhead 
 

\subsection{«КИЕВСКАЯ ЦВЕТОЧНИЦА» Или: Киев – большая деревня}

\label{sec:15_09_2021.fb.fb_group.story_kiev_ua.1.kiev_derevnja_cvetochnica}

 

\Purl{https://www.facebook.com/groups/story.kiev.ua/posts/1753326301530830/}

\ifcmt
 author_begin
   author_id fb_group.story_kiev_ua,stepanov_farid
 author_end
\fi


«КИЕВСКАЯ ЦВЕТОЧНИЦА»

Или: Киев – большая деревня

Даже не знаю, когда первый раз услышал выражение: «Киев – большая деревня».
Возможно, ещё от бабушки. Потом ещё много раз слышал и от других людей. И
каждый раз убеждался: как же оно правдиво.

Если кто не знает, люблю подражать полотнам великих художников и фотоработам
классиков фотографии. Это – не плагиат. Мне они нужны для вдохновения. Каждую
свободную минуту посвящаю чтению о мастерах и изучению их творчества. Часто,
когда гуляю Городом или выезжаю на природу, глаз, подсознательно, ловит
«знакомые» виды. Не всегда сразу могу вспомнить: где и у кого мог видеть
подобное. И только уже дома услужливое сознание достаёт нужную информацию из
чертогов памяти. Можно сказать, что самостоятельно осваиваю классическое
художественное образование.

Прошу прощение за долгое вступление. Оно необходимо для объяснения истории,
которой хочу с вами поделиться.

Метро «Харьковская». Небольшой рынок: помидоры, перец, огурцы, разные фрукты и
ягоды. И среди всего этого съедобного изобилия ярким островком выделяется
девушка, продающая цветы. «Киевская цветочница» - определяю для себя сразу. И
не могу вспомнить у кого из художников уже её видел. Делаю фото, стараюсь,
чтобы всё было естественно: не люблю постановочные фотографии. И только сделав
пару снимков вспоминаю: Николай Корнилиевич Пимоненко. Известный киевлянин и
талантливый украинский художник. Именно о нём в «альбоме Фишера» сказано, что в
родном ему Киеве, который он изображал на своих полотнах, он жил всегда. И
подчёркнуто – всегда и постоянно.

\ifcmt
  ig https://scontent-lhr8-2.xx.fbcdn.net/v/t39.30808-6/241765311_645126129795869_3281619987900366543_n.jpg?_nc_cat=105&_nc_rgb565=1&ccb=1-5&_nc_sid=825194&_nc_ohc=qnkmTp83ALQAX-XWngc&_nc_ht=scontent-lhr8-2.xx&oh=93d420bd5f6151b09a7c44e9d523cc5d&oe=6146E44B
  @width 0.4
  %@wrap \parpic[r]
  @wrap \InsertBoxR{0}
\fi

Николай Корнилиевич родился на Приорке – одном из самых живописных пригородов
Киева. Здесь же делал первые шаги в живописи: разъезжая с отцом, владельцем
небольшой иконописной мастерской, по окрестным городкам и деревням, расписывая
церкви. Юный Миколка мечтал о большой живописи. Но денег на образование у семьи
не было.

«С малых лет обнаруживал большую любовь к искусству и постоянно рисовал...
Деревня произвела чарующее впечатление на ребенка и оставила неизгладимый след
в его душе, заставив впоследствии сделаться художником Малороссии, с ее жгучим
солнцем и типичными фигурами малороссиян»©

Из предисловия к первому большому Альбому, приуроченному к 25-ти летию
творчества Пимоненко.

Вот тут я и вспомнил о: «Киев – большая деревня».

Как только начинаешь изучать одну киевскую историю, как, незаметно,
погружаешься в ещё одну, а за ней – в третью. И нет этому конца. Просто
поражаюсь, как в Киеве всё переплетено и связано: люди, события, истории.

Вот и жизнь Николая Пимоненко неотделима от историй многих известных личностей,
сыгравших в жизни Города важную роль.

Николай Иванович Мурашко – ещё один известный художник и основатель Киевской
школы рисования похлопотал перед киевским меценатом Иваном Терещенко о
финансировании обучения Пимоненко. И Терещенко, пораженный талантом Пимоненко,
поставил только одно условие: «Первая хорошая картина - моя!»

«Да у тебя большие художественные способности, приходи учиться ко мне в школу,
ты сделаешься не богомазом, а настоящим художником»©

Николай Мурашко

В годы учёбы в С.-Петербургской академии художеств Пимоненко учился вместе с
Михаилом Врубелем – ещё одним гениальным художником, оставившем след в истории
Города.

После возвращения в Киев Николай Пимоненко, по рекомендации Мурашко, работал
преподавателем в его Школе рисования. А его уроки посещали юная поэтесса Леся
Украинка и Иван Ижакевич – будущий классик украинской живописи.

По картинам Пимоненко можно изучать старый Киев. Он очень тщательно относился к
деталям. Были у художника и забавные истории. Однажды друзья пожурили его: не
хорошо скрывать, что подрабатываешь рекламой. Пимоненко не сразу понял о чём
идёт речь. Тогда ему показали … бутылку водки «Спотыкач» от некого … Николая
Шустова. На этикетке которой была размещена репродукция его картины «До дому».
Только через суд удалось запретить использовать картину и добиться компенсации.

Но вернёмся к «Киевской цветочнице». На углу Крещатика и Прорезной издавна
торговали цветами. И старожилы Города легко узнают на картине знакомые вывески
и здания, расположенные недалеко от Думской площади. Пимоненко много изображал
родной ему город: дома, улицы, храмы, горожан. Думаю, что будь он нашим
современником, быть ему – известным стрит фотографом. На самом деле, в
творчестве Николая Корнилиевича есть три картины «Цветочница». Написанные в
разные годы. И по ним можно наблюдать изменение мировосприятия художника.

Среди искусствоведов до сих пор не стихают споры: кто позировал автору для
образов цветочницы. Версий много. Cherchez la femme (фр.) – ищите женщину.
Самая загадочная и интригующая, связана с будущим академиком Николаем
Дмитриевичем Стражеско, который долгие годы работал и жил в Киеве и похоронен
на Лукьяновском кладбище. И Владимиром Кирилловичем Винниченко. Известным
писателем и политическим деятелем. Многие годы связанным с Киевом. «А где же
женщина?» – спросите вы. Не буду распространять слухи. Но еще многие годы после
смерти Пимоненко одна известная дама приносила белые лилии на его могилу на
Лукьяновском кладбище. А самая ранняя версия «Киевской цветочницы», именно с
корзиной белых лилий, была приобретена Стражеско и долгие годы украшала его
дом.

И в завершение.

Булгаковеды уверены, что в «Собачьем сердце» образ доктора Борменталя, ещё один
киевлянин - Михаил Афанасьевич Булгаков, писал с Николая Дмитриевича Стражеско.
А профессора Преображенского – с Василия Парменовича Образцова. Великого
медика-диагноста, тестя … одного из наших сегодняшних героев Николая Стражеско.
Но это уже совсем другая история.

И скажите после этого, что: «Киев - не большая деревня».

\ii{15_09_2021.fb.fb_group.story_kiev_ua.1.kiev_derevnja_cvetochnica.cmt}
