% vim: keymap=russian-jcukenwin
%%beginhead 
 
%%file 10_08_2023.stz.news.ua.mrpl.0629.1.ljudmyla_zavalej.qt.7
%%parent 10_08_2023.stz.news.ua.mrpl.0629.1.ljudmyla_zavalej
 
%%url 
 
%%author_id 
%%date 
 
%%tags 
%%title 
 
%%endhead 

\begin{quote}
\em\enquote{Ми просто жили. Намагались не думати ні про що. Не планували свій день. Ми
відходили від всього, що довелося пережити, передивитись у Маріуполі. В певний
момент я зрозуміла, якщо не повернусь до справи, збожеволію. Тому весь цей
простір – це результат мого власного посттравматичного синдрому, точніше,
спроба його позбавитися.

Ми отримали грант від німецької організації \enquote{AMICA}, яка підтримує жінок, що
потрапили у біду. За ці гроші ми з дівчатами з Маріуполя, часткою команди
Халабуди, яка тут знаходиться – Галею Балабановою, Ірою Кондратенко та іншими,
нашими маріупольськими волонтерами змогли орендувати приміщення та зробити його
\enquote{своїм} настільки, наскільки це можливо. Обирали його колегіально. Юля Паєвська
(Тайра), яка є частиною нашої команди та нашого простору, сказала, що під час
Революції Гідності у цьому дворі на вулиці Музейній, 8б,  були сутички з Омоном
і наші відстояли урядові квартали. Тож для мене це місце виявилось символічним
таким.

Ми його перефарбували, підлаштували під свої потреби і 1 червня розпочали
роботу. Тепер тут є коворкінг, невеличка зала для зустрічей та презентацій,
зручна кухня. Є такий простір суто військовий, із зібраними воєнними трофеями,
що залишив ворог вже тут, на Київщині. Зараз у нас тут просто своєрідний штаб
утворився. Тут військові можуть залишитися на ночівлю, якщо їдуть транзитом,
тут збираються волонтери. Тут ми проводимо творчі презентації.

Зараз таких культурних заходів небагато, але з вересня ми плануємо розпочати
цикл важливих дискусій щодо візії повернення у Маріуполь. У нас є думка, що вже
пора. Я цей проєкт називаю \enquote{Візія повернення}. Я ж не міська рада. Я хочу про
це поговорити. Я хочу про це поговорити з нашими}, 
\end{quote}
- каже Міла.
