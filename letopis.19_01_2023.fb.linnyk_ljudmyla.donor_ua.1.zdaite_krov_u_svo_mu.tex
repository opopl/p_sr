%%beginhead 
 
%%file 19_01_2023.fb.linnyk_ljudmyla.donor_ua.1.zdaite_krov_u_svo_mu
%%parent 19_01_2023
 
%%url https://www.facebook.com/liudmyla.linnyk/posts/pfbid02GYWJD4Gb6JuasdGzZngqMXCtuScvv38LQ8BHpA2Kp6kv8yvGM8QqVw176KwheLJGl
 
%%author_id linnyk_ljudmyla.donor_ua
%%date 19_01_2023
 
%%tags donor
%%title Здайте кров у своєму місті, бо вона - може воювати
 
%%endhead 

\subsection{Здайте кров у своєму місті, бо вона - може воювати}
\label{sec:19_01_2023.fb.linnyk_ljudmyla.donor_ua.1.zdaite_krov_u_svo_mu}

\Purl{https://www.facebook.com/liudmyla.linnyk/posts/pfbid02GYWJD4Gb6JuasdGzZngqMXCtuScvv38LQ8BHpA2Kp6kv8yvGM8QqVw176KwheLJGl}
\ifcmt
 author_begin
   author_id linnyk_ljudmyla.donor_ua
 author_end
\fi

Я тут цей. Щойно з відпустки. Два тижні, вперше за майже три роки. Що було до
того - я не пам'ятаю. А ні. Пам'ятаю. Було у 2018-му. Щось там. Море було. З
чотирма дітьми, двоє з яких - практично немовлята. Цікава була відпустка. Я б
сказала - незабутня. І десь там було три дні у Стамбулі. А потім було чортішо. 

Коротше, цей. Я тут щойно з відпустки. Ага, під час війни. У відпустку поїхала.
В Італію. У справжню відпустку. Моя команда зробила все, щоб ця відпустка була
справжньою. Щоб я не відписувала на листи, не приймала рішень, не брала
відповідальності. 

Ходила тихими тосканськими містечками. Їздила в дивні кастельйони, чімітеро і
к'єзи (замки, цвинтарі і церкви) - якнайдалі від популярних туристичних
маршрутів, все, як я люблю. Їла пасту, піцу і пила червоне вино з водою (двічі
по три столові ложки на склянку води, я не вживаю алкоголь скількись там
невідомо-скільки років. Але хіба це алкоголь?). Гладила навіжених котів. Водила
маленького блакитного фіата-панду серпантинами і тунелями (ненавиджу тунелі,
обожнюю серпантини, як виявилось). Не розмовляла з людьми, бо я - "Нон о капіто
нєнте" (Насправді - ні, але ви їм не розказуйте). Це так прекрасно - не
розмовляти з людьми. 

Вперше за багато часу мені було так гірко і так добре водночас. 

Вперше від 24 лютого бачила інверсійний слід літака в небі (вже знов не бачу,
бо вони не літають в Україні). 

Вперше від 24 лютого спокійно виходила на вулицю вночі (вже не виходжу, бо в
нас комендантська година). 

Вперше від 24 лютого вимикала на ніч телефон (вже не вимикаю, бо а раптом... ну
ви самі розумієте). 

Вперше від 24 лютого не почувалася винною (вже минулося).

Вперше від 24 лютого хотіла чогось особисто для себе (вже забула, чого саме).    

Вперше від 24 лютого не думала про донати - донорської крові крові і
благодійних грошей. (Вже знов думаю, бо і того, і другого знов нема. Нє, не
мало є, а саме нема.)

Вперше в житті каталась на автодромі - такі машинки, знаєте, що їздять на треку
і смішно буцяються. Вперше за невідомо скільки безтурботно сміялася. Вперше за
багато часу нормально спала. Вперше не їла в Італії морозива, але пила багато
кави. Вперше не хотіла і водночас хотіла повертатися. 

Як завжди, у мене немає жодного довбаного селфі для лайків і привернення уваги.

Як завжди, це просто рефлексія і спогад мені для мене, і трішки  belles
lettres.

А, і цей. Ну, раз я вже все одно повернулася з відпустки. 

Здайте кров у своєму місті, бо вона - може воювати (реєстрація осьтамечки, в
першому коментарі). 

Так. Ваша кров. Вона - може.
