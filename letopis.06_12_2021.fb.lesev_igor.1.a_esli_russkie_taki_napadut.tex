% vim: keymap=russian-jcukenwin
%%beginhead 
 
%%file 06_12_2021.fb.lesev_igor.1.a_esli_russkie_taki_napadut
%%parent 06_12_2021
 
%%url https://www.facebook.com/permalink.php?story_fbid=4850594128304985&id=100000633379839
 
%%author_id lesev_igor
%%date 
 
%%tags napadenie,rossia,ugroza,ukraina
%%title А если русские таки нападут?
 
%%endhead 
 
\subsection{А если русские таки нападут?}
\label{sec:06_12_2021.fb.lesev_igor.1.a_esli_russkie_taki_napadut}
 
\Purl{https://www.facebook.com/permalink.php?story_fbid=4850594128304985&id=100000633379839}
\ifcmt
 author_begin
   author_id lesev_igor
 author_end
\fi

А если русские таки нападут?

Сразу скажу, что на момент написания материала любое гипотетическое
широкомасштабное вторжение России на Украину считаю дебильным. Но помните ту
историю с американской радиопостановкой Уэллса «Вторжение миров»? Сотни тысяч
людей в Штатах свихнулись и по-настоящему подумали, что марсиане высадились на
Земле.

\ifcmt
  ig https://scontent-frx5-1.xx.fbcdn.net/v/t39.30808-6/264651274_4850593434971721_376394617247419575_n.jpg?_nc_cat=110&ccb=1-5&_nc_sid=730e14&_nc_ohc=krRclKJeQ3UAX-iZHHO&_nc_ht=scontent-frx5-1.xx&oh=be7a5515a1a6c1a074684043ad4452f6&oe=61B39683
  @width 0.4
  %@wrap \parpic[r]
  @wrap \InsertBoxR{0}
\fi

Вот давайте представим, что русские марсиане тоже в январе-феврале 22 года таки
решат атаковать самую чудесную страну на планете. Итак, для начала все-таки
нужен мотив. Рассуждения в стиле «потому что Путин – маньяк» или «Россия
исторически считает эти земли своими» не катят даже для сценария конотопского
театра юного зрителя.

Мотив для подобного должен быть пятикратно железобетонным. Вот, скажем, у
каждого в доме есть ножи. Но мы же не режем ими людей, которые нам даже очень
сильно не нравятся. И не потому, что мы какие-то особые гуманисты и думаем, что
если прохожего пырнуть ножом, то ему станет больно. Мы банально просчитываем
последствия для самих себя.

Вторжение России в Украину тоже не останется без последствий. При этом
последствия испытают все – и Россия, и Украина (что естественно), и все без
исключения ключевые страны мира.

Первыми, что естественно и очевидно, можно поставить под угрозу для России –
это торговые отношения со странами Запада. А эти отношения по-прежнему остаются
для России огромными.

Вот я смотрю список торговых партнеров России за прошлый 20 год. Первое место
Китай (104 ярда экспорта/импорта), третье место Белоруссия (30 ярдов). Тут «в
случае чего» все останется без надрывных изменений, разве что китайское
направление еще больше начнет расти. Но смотрим дальнейший список – Германия
(почти 42 ярда), Нидерланды (28), Британия (26), Штаты (да-да, это шестой по
важности торговый партнер России с большой динамикой роста, а вы что тут
думали? – почти 24 ярда), Турция (почти 21). В совокупности эти страны дают
большую цифру, чем Китай и Белоруссия – 141 против 134. А ведь ниже там идут
«всякие» Бельгии и Италии, которые этот разрыв в пользу коллективного Запада
еще больше увеличивают.

Таким образом, большое вторжение России ставит под угрозу большую торговлю
России со странами Запада. При этом нужно понимать, что для коллективного
Запада полностью заморозить отношения с Россией будут гораздо менее болезненны,
чем для самой России. И это выплывает из банальной математики. Условная
Германия от такого разрыва теряет 5\% внешнего товарооборота, а Россия +/- 50\%.
Сами теперь думайте, кому будет больнее.

Другое последствие – это все-таки угроза Большой войны. Она вроде бы и
минимальна, потому что это очень быстро перерастает в перспективу ядерной
войны. Согласитесь, звучит футуристично, когда после захвата Россией Шостки
американцы отдают приказ бомбануть Воронеж, получая перспективу получить тут же
что-то очень неприятное по Сиэтлу. Я почти уверен, что все закончится «очень
глубоким сожалением» и международной платформой «Шостка – це Украина».

Но все же, тут ведь вопрос еще и репутации. Американцы облажались в Ираке,
вынужденно признали наличие Асада в Сирии, и просто жидко обделались в Афгане.
И это все с претензией на роль мирового гегемона. Все это в совокупности дает
осторожные предположения, что за Украину им придется вынужденно расписываться
как-то иначе. А вот как «иначе», похоже и сами американцы не знают.

И все же, какой тогда мотив может быть у России для вторжения в Украину? Все
что было написано выше, как раз говорит, что такого мотива нет. И все же я
такой мотив знаю. Это – Крым.

Крым, как ни странно, самое слабое звено России на украинском направлении. Да,
мы можем трактовать, при этом более чем обоснованно, события февраля 14 года в
Киеве как государственный переворот. При этом он давно уже легализован всем
миром, включая и саму Россию. Все последующие события, включая поддержку
сепаратистов на Донбассе, имеют славную мировую традицию. Одни только
американцы после окончания Второй мировой подобным занимались и занимаются по
всему миру на промышленной основе. И русские здесь не делают чего-то нового или
необычного. Банальная защита своих национальных интересов, ничего более.

Но Крым – это все-таки не создание нового Северного Кипра или Абхазии. Тут, как
ни крути, а отторжение территории от одного государства к другому. И от того,
что в самому Крыму подавляющее большинство такое отторжение приветствуют, в
рамках международного права это все равно остается отторжением и захватом.

И именно Крым, а не ДНР и ЛНР создают невозможность любого конструктивного
российско-украинского диалога. Политсилы, которые выступают за такой диалог,
объективно маргинализированы. Ну потому что, когда условный Бойко или Мураев
говорит «давайте налаживать отношения», они всегда деликатно выносят за скобки
вопрос Крыма. А если это в скобках, тогда и присоединение части государства А к
государству Б тоже в скобках легализовывается. Отсюда и обвинения в
коллаборации, и отсюда же вся последующая русофобия.

Иными словами, Крым всю остальную Украину делает АнтиРоссией. Этому
способствует много факторов, в том числе и международный. Но политэлита даже с
изначально умеренными взглядами этому предпочитает не сопротивляться (сравните
Зеленского образца 19 года и сегодняшнего) и на раз-два бандеризируется. И вот
чтобы сделать Украину другой – прямое вторжение России может быть тем самым
мотивом.

Цели гипотетического вторжения России расписывать, пожалуй, не буду. Все-таки
живу в шизофренической стране, где могут приписать «пропаганду вторжения
страны-агрессора». Отмечу только, что для многих тут обитающих эти цели стали
бы гораздо более приемлемы, нежели то, что они имеют сейчас. Но базовая цель –
это создание пророссийской власти с последующей легализацией Крыма за Россией.

И да, коллективный Запад может по умолчанию и дальше считать Крым украинским.
Но здесь уже столько исторических аналогий, что даже сухой список перечисления
будет гораздо больше этой статьи. Самый свежий пример – «президент»
Тихановская. Если ты не можешь на реальной земле решать ничего, то и твоя
ценность чуть больше, чем ничего. Могу привести «польское правительство в
изгнании», которое успокоилось только после избрания Леха Валенсы президентом.
Вот только Львов чей – польский или советский/украинский?

И последнее. А каким будет сопротивление Украины возможному ру-вторжению?
Во-первых, а кто его знает, как оно будет на самом деле. А во-вторых, как по
мне, простым людям плюс/минус пох. Какое сопротивление украинской армии вы
наблюдаете в Мариуполе или Северодонецке? А какое сопротивление лднр-овцам (или
как у нас их называют «российские оккупационные войска») вы видите в Донецке и
Луганске?

Когда у тебя офшорный президент, несколько миллионов человек названы
«некоренными», еще больше поделены на сорта по языку и мировоззрению, и даже
те, кто в своей идеологической стае все равно занимаются здесь таким же
выживанием – вот когда это все вместе – мне кажется, что будь сюда вторжение
России, Польши или хоть Китая, особого ужаса это не вызовет. Это как
колониальные войны где-то в тропической Африке середины XIX века. Белые
британцы стреляют в белых французов и за всем этим с интересом наблюдают черные
аборигены. Только где во всем этом интересы местных?

Но давайте финализировать. Нет никаких разумных предпосылок для ру-вторжения
этой и не только этой зимой. Хотя бы потому, что Украина для России пусть и
имеет во внешней политике огромное значение, но это не пуп земли. Да, русские
пересматривают итоги своего падения образца 1991 года. В целом, итоги «холодной
войны» сейчас проходят этап ревизии.

Но все это не повод куда-то вторгнуться только на основании, что у тебя есть
армия, а значит надо ее куда-то присунуть. Нет, тупых в этом плане было очень
много. Например, незадолго до своего распада Сомали напало на Эфиопию,
попытавшись присобачить себе обширную провинцию Огаден. Но подавились, и где
теперь то Сомали? Только на карте мира одним цветом и остались. Впрочем,
Эфиопия по итогу тоже единой не сохранилась.

А самым тупым я все-таки считаю Саддама Хусейна. В 1980-м с какого-то хера
начал войну с Ираном, 8 лет мусолили, сотни тысяч трупов и развал экономики, и
по итогу пришли к тому, с чего все началось – мир без изменения границ. Потом
влез в Кувейт и получил уже по щам от международной коалиции с последующей
изоляцией страны. А дальше вы и так знаете – оккупация страны американцами,
петля на шее и уже фактический распад самого Ирака, где по-прежнему натуральное
адище.

Вторжение – это всегда обоснование. Если обоснования нет, а мировые СМИ
шаблонно по-прежнему о нем пишут, то это называется джинса.

\url{https://t.me/Lesev_Igor}

\ii{06_12_2021.fb.lesev_igor.1.a_esli_russkie_taki_napadut.cmt}
