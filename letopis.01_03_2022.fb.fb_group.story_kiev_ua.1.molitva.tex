% vim: keymap=russian-jcukenwin
%%beginhead 
 
%%file 01_03_2022.fb.fb_group.story_kiev_ua.1.molitva
%%parent 01_03_2022
 
%%url https://www.facebook.com/groups/story.kiev.ua/posts/1872054559658003
 
%%author_id fb_group.story_kiev_ua,kaganovich_anatolij
%%date 
 
%%tags 
%%title Молитва
 
%%endhead 
 
\subsection{Молитва}
\label{sec:01_03_2022.fb.fb_group.story_kiev_ua.1.molitva}
 
\Purl{https://www.facebook.com/groups/story.kiev.ua/posts/1872054559658003}
\ifcmt
 author_begin
   author_id fb_group.story_kiev_ua,kaganovich_anatolij
 author_end
\fi

В январе, я вспоминал распятый Грозный и слезно просил Всевышнего не делать из
меня пророка. Меня не услышали...

Вчера Яхиэль Фишзон написал эту молитву. Не услышать ее - невозможно.

И еще - бывших киевлян не бывает...

Молитва

\raggedcolumns
\begin{multicols}{2} % {
\setlength{\parindent}{0pt}

\obeycr
Насквозь продуваема сиреной,
по три раза в час, лежит страна:
в Крым уткнулись голые колена,
влажная полесская спина
коркою покрылась вдоль границы,
седоват Карпатской гривы цвет.
\smallskip
Сердце твоё - Киев, как зарницы,
разрывают сполохи ракет.
\enquote{Выживи!} - мне большего не надо
\enquote{Выстой!} и ступай в бескрайний мир,
\smallskip
мир, где загляделся до упаду
сам в себя влюблённый Синевир,
в день, где распрямляется до хруста
позвоночник славного Днепра,
где ресничка выпавшая Хуста
главное занятие с утра.
\smallskip
Но, сейчас, о крови, не водице,
речь идёт, вернее гул стоит,
пышет гарь и дым не шевелится
над любым, кто может быть убит.
\smallskip
Оттого не сплю. Лежу ночами,
вслушиваюсь в темень, жду - когда
маякнут Одесса и Очаков,
мол \enquote{сегодня тихо, ерунда
комендантский час - такая малость}.
\smallskip
А попозже Харьков шепотком
жаловаться станет на усталость,
танки обливая кипятком
огненным. Забытая столица
слобожанской линии вдова,
моей жизни мятая страница
Харьков, ты пылаешь, как дрова.
\smallskip
Невозможно выстоять на кромке
памяти, глядя в ночной провал -
в Броварах воронка на воронке,
там, где я когда-то целовал
волосы шампунем и Шопеном
пахнувшие, ночь идя домой
вдоль путей затем. Там пахнет креном
падающей мины и войной.
\smallskip
Горе мне дожить и быть на тризне
по местам веселья и любви?
Господи, твердыня нашей жизни,
приговор ужасный отзови;
\smallskip
умоляю - мира дай! И Киев
сладкий, как балабухский цукат,
дай увидеть. И рассвет над Стрыем
и с Приморской лестницы закат!
\smallskip
Степь Кременчуга, скульптуры Львова,
Закарпатья виноградный край -
сохрани! Спаси страну от злого!
А врагов, Всевышний, покарай!
\restorecr

\end{multicols} % }

\ii{01_03_2022.fb.fb_group.story_kiev_ua.1.molitva.cmt}
