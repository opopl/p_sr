% vim: keymap=russian-jcukenwin
%%beginhead 
 
%%file 06_02_2022.fb.baumejster_andrej.kiev.filosof.1.chelovek_degradacia
%%parent 06_02_2022
 
%%url https://www.facebook.com/andriibaumeister/posts/4739330912855097
 
%%author_id baumejster_andrej.kiev.filosof
%%date 
 
%%tags chelovek,degradacia,tiktok
%%title Человеческая деградация
 
%%endhead 
 
\subsection{Человеческая деградация}
\label{sec:06_02_2022.fb.baumejster_andrej.kiev.filosof.1.chelovek_degradacia}
 
\Purl{https://www.facebook.com/andriibaumeister/posts/4739330912855097}
\ifcmt
 author_begin
   author_id baumejster_andrej.kiev.filosof
 author_end
\fi

Я наблюдаю явные симптомы человеческой деградации. Для многих эти симптомы
незаметны. Более того, если я начинаю говорить об этих опасных тенденциях (о
разрушительной роли социальных сетей, о триумфе массового человека, о
саморазрушении элит, о разрушении базовых культурных навыков), мне со
снисходительной улыбкой говорят: вы преувеличиваете. Вы просто цепляетесь за
старое. Те же социальные сети, все эти новые технологии коммуникации - это
просто инструменты. Все зависит от того, как мы их используем. 

\href{https://www.youtube.com/watch?v=VP7xBTn3nNs}{%
TikTok и Co: симптомы человеческой деградации, Andrii Baumeister, youtube, 06.02.2022%
}

\ifcmt
  ig https://i2.paste.pics/8bd98d908442dc60f38e4ef0a2eb9dfd.png
  @wrap center
  @width 0.8
\fi

ТикТок опасен? Не смешите. Привыкайте говорить или показывать главное за
минуту. Завязывайте уже с вашими длинными и нудными рассуждениями. Давайте
кратко и четко.  Те, кто так говорят, думают, что они современны. Что они тем
самым понравятся \enquote{молодежи}. Что новые поколения \enquote{просто
другие}, не лучше и не хуже предыдущих. Но я спрашиваю: при чем здесь
поколения? И при чем здесь возраст? 

Пока идут острые споры о ценностях, идеях, о демократии и авторитаризме, о
свободной торговле и новом протекционизме, к нам подкрадывается опасность,
превосходящая по масштабу все наши опасения. Это опасность разрушения базового
уровня человечности, уничтожения базовых способностей \enquote{быть человеком}.
Это реальная гуманитарная катастрофа. 

Древний вопрос \enquote{что такое человек} сегодня актуален как никогда. Боюсь,
что скоро ответить на этот  вопрос будет уже невозможно. Или почти
невозможно...
