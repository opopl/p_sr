% vim: keymap=russian-jcukenwin
%%beginhead 
 
%%file topics.vojna.poezia.other
%%parent topics.vojna.poezia
 
%%url 
 
%%author_id 
%%date 
 
%%tags 
%%title 
 
%%endhead 

Пьяные крики и запах сортира -
Внешние признаки "русского мира".
Цепь золотая на шее банкира -
Связь поколений "русского мира".
Русский язык стал родным для башкира -
Это политика "русского мира".
У депутата седьмая квартира -
Вот справедливость "русского мира".
Крест золотой на груди рэкетира -
Это духовность "русского мира".
Из негодяя сделать кумира -
Идеология "русского мира".
Ложь ежедневная телеэфира -
Это правдивость "русского мира".
Двадцать копеек пачка пломбира -
Память фантомная "русского мира".
В морду удар от отца-командира -
Вот "педагогика" "русского мира".
Насмерть кувалдой забить дезертира -
Это гуманность "русского мира".
Сытый смешок на устах конвоира -
Это реальность "русского мира".
Вор, матершинник, подлец и задира -
Это сторонник "русского мира".
Кнут, кандалы, яд, петля и секира -
Издревле скрепы "русского мира".
Смерть вместо масла, война вместо мира -
Это основа "русского мира".
Братья, друзья, земляки-украинцы!
Разве нужны нам такие "гостинцы"!
Войны, теракты, "русские марши".
Нам бы от мерзости этой подальше!

Вона кохала. Він кохав.
Купались в щасті, наче діти.
Вона не знала й він не знав:
Недовго суджено радіти.
Вона кохала. Він кохав.
Єднались душами й тілами.
Вона не знала й він не знав,
Що розійдуться між світами.
Вона кохала. Він кохав.
Тримав, мов квітку у долоні.
До серця палко пригортав
В любові чистої полоні.
Вона кохала. Він кохав.
У мріях бавили дитяток.
Вона не знала й він не знав,
Що їх кінець там, де початок.
Вона кохала. Він кохав.
Війна ввірвалась, кров'ю вмита.
І ворог землю шматував -
Пішов у бій посеред літа.
Вона кохала. Він кохав.
До болю слізно проводжала.
Назад вернутись обіцяв -
Вона безмежно довіряла.
Вона кохала. Він кохав.
На зустріч мали сподівання.
Вона не знала й він не знав,
Що бачать очі ці востаннє.
Вона кохала. Він кохав.
Дзвінки між боєм прилітали.
І кожен день, що догорав,
Вони в надії проводжали.
Вона чекала. Він чекав.
Земля покрилась листопадом.
З них вогог душі випускав -
Їх батальйон накрило градом.
Вона чекала. Він стогнав,
Останній подих випускавши.
Вона молилась. Він вмирав,
Ім'я кохане прошептавши.
Вона ридала. Він вже спав.
Спішив з війни до неї нині.
Вернувся. Як і обіцяв.
Лиш не живий. У домовині....
Ірина Голуб - Красуляк

У небі, в полі, у гаю. Де йдуть без зброї на броню, Де кажуть «на@уй» кораблю.
Ти прогниєш у цій землі.  На моїй Визвольній війні. Розквітне сонях навесні —
«могутня» армія русні. За Харків, Марік, за Ірпінь. За прірву надлюдських
терпінь… Наступних десять поколінь у пеклі скнітимеш. Амінь. Ми Чорнобаївка. Ти
мрець безжально злитий нанівець.  Нам уривається терпець. Ти здохнеш, сцуко. Це
кінець.
