% vim: keymap=russian-jcukenwin
%%beginhead 
 
%%file 06_11_2021.fb.filatov_boris.1.strana_sovetov
%%parent 06_11_2021
 
%%url https://www.facebook.com/permalink.php?story_fbid=4509585642456631&id=100002157183088
 
%%author_id filatov_boris
%%date 
 
%%tags filatov_boris,obschestvo,sovet,strana,ukraina
%%title Страна Советов
 
%%endhead 
 
\subsection{Страна Советов}
\label{sec:06_11_2021.fb.filatov_boris.1.strana_sovetov}
 
\Purl{https://www.facebook.com/permalink.php?story_fbid=4509585642456631&id=100002157183088}
\ifcmt
 author_begin
   author_id filatov_boris
 author_end
\fi

\obeycr
Страна Советов.
...
К чему я так и не смог привыкнуть за время своего мэрства, так это к бесконечным советам.
Безусловно, это прекрасно, что народ у нас стал политически сознательным, активно участвующим в жизни территориальной громады и требующим общественного контроля над деятельностью чиновников.
Но я не об этом. 
...
А о социальном феномене, который выражается в том, что десятки ( если не сотни ) тысяч людей в нашей стране почему-то решили, что они точно знают, что нужно делать другим.
При чём не только в профессиональной деятельности, но и в личной жизни и даже на персональной странице в соцсетях. 
И поэтому, эти обладатели сакральных знаний решили бесцеремонно лезть в чужую жизнь.
Не буду далеко ходить, а приведу пару примеров из буквально вчерашней переписки.
...
Пишет незнакомая дама. 
Очень вежливо, связанно, с хорошей речью.
Настоятельно рекомендует мне исправить фамилию покойного Бориса Ивановича Холода с «Холодов» на «Холод». Якобы, я ошибся и выглядит это нехорошо.
Я тупо смотрю в собственный телефон и не могу понять, как я мог перепутать фамилию человека, который был товарищем моего покойного отца и которого я помню с детства.
Может описка?
Дама настаивает, говорит, что она преподаватель французского языка и обладатель какой-то «Пальмовой ветви». 
Я продолжаю тупо смотреть в телефон и не понимаю о чем речь. Захожу в настройки «История правок» и вижу, что фамилия «Холод» изначально была написана правильно.
Но дама всё равно настаивает, что я в начале ошибся, а лишь потом фамилию исправил. И я вдруг понимаю, что с головой всё-таки не в порядке не у меня, а у дамы...
...
Пишет знакомый гражданин. 
Вернее, не гражданин, а старый товарищ, который «надцать» лет назад уехал жить в Японию и обрёл там нирвану.
Обретя нирвану, он решил, что теперь я должен все бросить и начать оцифровывать областные архивы. 
Мы же бывшие историки и поэтому даже на госслужбе должны думать о науке.
Как объяснить старому товарищу, обретшему нирвану, что я не могу оцифровывать областные архивы только потому, что они не принадлежат территориальной громаде Города, а являются чужой собственностью - я не знаю.
Он переживает за отечественную науку. 
Из Японии.
И поэтому нарезает мне оттуда задачи.
Мне же просто нех@й делать. 
Поэтому я должен все бросить и начать  оцифровывать областные архивы. 
Хорошо хоть не государственные.
...
А потом пишет уже незнакомый гражданин.
Что я должен выключить отопление. 
Что на улице +19, цена на газ запредельная и платежки неподъёмные будут.
Как будто это не я ещё десять дней назад орал на каждом углу и из каждого утюга: «Люди дорогие, потерпите немного, оденьте носочки/свитерки».
А в ответ все бегали и кричали: «Падла, включи отопление! Мы мёрзнем !»
Как объяснить незнакомому гражданину, пассионарию и умнице, борющемуся у себя в ФБ за свободную Беларусь и требовавшего отставки Авакова, что ОНО, БЛ@ТЬ, ТАК НЕ РАБОТАЕТ?!!
Что централизованное отопление - это система.
Гребанная СИСТЕМА, которую нельзя каждый раз ВКЛ/ВЫКЛ, как этого хочется прогрессивным трудящимся.
Что ее в раз сезон запускают/заполняют/развоздушивают/подают давление/ убирают порывы и пр.пр. и также эту систему ОДИН раз в конце сезона отключают.
Что у мэра нет в кабинете рубильника управления отоплением в Городе размером в 400 с лишним квадратных километров. 
Больше по площади, чем Харьков и Одесса.
...
Я б не нервничал так сильно, если бы неравнодушный незнакомый гражданин был один.
Но таких граждан nomen illis legio, что в переводе с латыни означает «овердох@я».
Им каждый отопительный сезон рассказываешь одни и те же элементарные вещи, а они тебя не слышат, но в ответ дают умные советы.
...
Но я опять не об этом.
А о том, что я никому не даю советы. 
От слова «вообще».
Безусловно, как человек с двумя университетскими образованиями, доктор наук, заслуженный журналист Украины, корпоративный адвокат, бывший парламентарий, госуправленец и вообще энциклопедист, двухметровый блондин с голубыми глазами и очень скромный человек, я имею полное моральное право это делать.
Но не делаю.
Хотя могу. 
Имею и опыт, и знания поучать всех: от Президента с министрами до подчинённых мне дворников.
Но не поучаю.
И я не советую ни докторам, ни автомеханикам, ни военным, ни маникюршам, ни оперуполномоченным, ни рубщикам мяса на базаре, что они должны делать.
А также не рекомендую всем вышеперечисленным товарищам с кем им жить, с кем дружить, с кем спать и что писать у себя в фейсбуках.
...
Я просто предпочитаю работать.
И, наверное, поэтому, за что бы я не брался в этой жизни, у меня всё получалось...
...
А, во.
Кстати.
Вакцинируйтесь. 
Это не совет. Это - приказ.
:)))
\restorecr

\ii{06_11_2021.fb.filatov_boris.1.strana_sovetov.cmt}
