% vim: keymap=russian-jcukenwin
%%beginhead 
 
%%file 22_11_2021.fb.uljanov_anatolij.1.maidan.2
%%parent 22_11_2021.fb.uljanov_anatolij.1.maidan
 
%%url 
 
%%author_id 
%%date 
 
%%tags 
%%title 
 
%%endhead 
\subsubsection{2}

Оба \sem{майдана} являются для меня олигархическими переворотами, поднявшими со дна
истории наиболее реакционные элементы украинского общества. Сегодня из этого
исторического ила состоят элиты нашей страны. Задавая её повестку, они несут
ответственность за материальное положение украинских граждан. 

Дышать без кредитной иглы Украина, к сожалению, не может. Это ставит её в
\sbfu{зависимость от кормящей руки} внешних геополитических формаций, их воли и
интересов. Как будущие поколения украинцев будут отдавать долги, которые им
оставляют в наследство элиты (пост)Майдана, учитывая состояние украинской науки
и образования, разбазаривание земель и превращение индустриальной страны в
банановую республику с сервильно-сырьевой экономикой? Её элиты расплачиваются
со своими спонсорами украинским суверенитетом, обрекая нас на будущее
колониального должника, и называя жизнь в долг «незалежностью».

Дети элит получают образование за рубежом, и либо остаются там в качестве
\sem{привилегированной обслуги}, либо возвращаются домой, в его правящий класс, чтобы
обслуживать империю как её \sem{колониальные администраторы}, смотрящие за логистикой
вывода ресурсов в имперские центры. Для них уготована сытая жизнь. Для
большинства же украинцев – судьба дешёвой рабочей силы. 

Чтобы её оправдать, существует идеология, чей «справедливый рыночный бог»
объясняет положение человека его качеством, мол, каждому даётся по заслугам:
если ты богат, значит ты талантлив, работал, и \sem{заслужил}. Что касается растущего
неравенства и обнищания большинства... оно объясняется тем, что большинство – это
ленивое тупое быдло, которым должны управлять \sem{достойные, лучшие, более
качественные люди}. То есть, паны. Из них мы и выбираем себе власть, называя всё
это «демократией».
