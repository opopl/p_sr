%%beginhead 
 
%%file 23_05_2018.fb.fb_group.mariupol.biblioteka.korolenka.1.prezentats_ya_novo__
%%parent 23_05_2018
 
%%url https://www.facebook.com/groups/1476321979131170/posts/1686867464743286
 
%%author_id fb_group.mariupol.biblioteka.korolenka,shulga_inna.mariupol
%%date 23_05_2018
 
%%tags mariupol,literatura,poezia,prezentacia,kniga,kultura
%%title Презентація нової книги поета Миколи Новосьолова "Нота надежды"
 
%%endhead 

\subsection{Презентація нової книги поета Миколи Новосьолова \enquote{Нота надежды}}
\label{sec:23_05_2018.fb.fb_group.mariupol.biblioteka.korolenka.1.prezentats_ya_novo__}
 
\Purl{https://www.facebook.com/groups/1476321979131170/posts/1686867464743286}
\ifcmt
 author_begin
   author_id fb_group.mariupol.biblioteka.korolenka,shulga_inna.mariupol
 author_end
\fi

Міський літературний музей центральної бібліотеки ім. В. Г. Короленка продовжує
представляти громаді книги маріупольських авторів, які надійшли для участі у
відкритому міському конкурсі «Краща книга року-2018» в рамках Дев'ятого
міського фестивалю «Книга і преса Маріуполя». Нагадаємо, що протягом 2018 року
вже відбулися презентації поетичних збірок Лідії Белозерової, Анастасії Папуш
та Віри Столярчук.

19 травня в бібліотеці відбулася чергова небуденна подія в літературному
просторі Маріуполя: шанувальникам витонченої словесності презентували нову
книгу поета Мико-ли Новосьолова «Нота надежды». Під час презентації учасникам
нагадали, що в лютому виповнилося 20 років від часу прийняття М.М. Новосьолова
до лав Національної Спілки письменників України. Маріупольські книголюби добре
знайомі з творчістю поета і барда, лауреата Донецької обласної премії ім. В.
Шутова, «Маріупольця  2001 року» в номінації «Літератор», лауреата кількох
фестивалів авторської пісні, в т.ч. ІХ Міжнародного фестивалю «Пісня над
морем-2007».

І все ж таки своєю новою, восьмою, книгою Микола Миколайович знову здивував і
захопив читачів. В «Ноту надежды» включено небагато віршів – близько сорока –
та літературознавча проза «Мысли вслух». Автор з'явився перед слухачами
філософом, який як камертон відгукується на драматичні події сучасного світу,
тонким ліриком, поетичні ряд-ки якого генерують глибинний оптимізм.

Як зауважила ведуча презентації головний методист ЦБ ім. В.Г. Короленка Ольга
Подтинна, епіграфом до «Ноты надежды» могли б стати слова Ліни Костенко:

Твої страждання – особиста справа,

Твоє мистецтво – радощі для всіх.

Своєрідним відлунням цих рядків геніальної української поетеси є фраза Миколи
Новосьолова:

Пусть стихи – точно слезы из глаз –

Облегчают скорбящие души.

Під час презентації в авторському виконанні пролунали вірші з нової книги та
пісні. Слід зауважити, що Микола Миколайович володіє феноменальною пам'яттю: на
прохання учасників він легко декламував вірші, які були написані понад 25 років
тому. Сюрпризом для гостей став виступ поета і барда Валерія Пихтіна, який
виконав декілька своїх пісень.

Завершилася презентація відповідями Миколи Новосьолова на численні питання
допитливої публіки, теплими словами вдячності автору за можливість
доторкнутися до талановитої поезії, душевну атмосферу зустрічі та автографами
на книгах.
