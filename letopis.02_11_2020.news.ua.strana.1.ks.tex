% vim: keymap=russian-jcukenwin
%%beginhead 
 
%%file 02_11_2020.news.ua.strana.1.ks
%%parent 02_11_2020
 
%%url https://strana.ua/news/298504-konstitutsionnyj-sud-chto-proiskhodit-s-eho-razhonom.html
%%author maria romanova
%%tags ks,constitution,ukraine
%%title Зеленский пугает кровью, Разумков идет против Зеленского, фактор Госдепа. Что сейчас происходит вокруг КСУ
 
%%endhead 

\subsection{Зеленский пугает кровью, Разумков идет против Зеленского, фактор Госдепа. Что сейчас происходит вокруг КСУ}
\label{sec:02_11_2020.news.ua.strana.1.ks}

\Purl{https://strana.ua/news/298504-konstitutsionnyj-sud-chto-proiskhodit-s-eho-razhonom.html}
\Pauthor{Романова, Мария}

\ifcmt
img_begin 
	url https://strana.ua/img/article/2985/4_main.jpeg
	caption Конституционный суд Украины. Фото пресс-службы ведомства 
	width 0.7
img_end
\fi

В Украине продолжается борьба вокруг Конституционного суда, который на минувшей
неделе нанес удар по антикоррупционной вертикали.

Суд отменил 18 статей закона "О предотвращении коррупции" и перечеркнул львиную
долю полномочий НАПК - ведомства, проверяющего декларации чиновников и
депутатов. А заодно отменил уголовную ответственность за нарушения в
декларациях. 

Это был мощный удар в основание антикоррупционной архитектуры: именно
Нацагентство задумано как инициатор дел за незаконное обогащение. А посадить по
этой статье можно любого чиновника и депутата. Что, конечно, является мощнейшим
рычагом для контроля всей украинской власти.

Такой контроль пытаются осуществлять американцы, курировавшие создание НАПК,
НАБУ, САП и ВАКС. 

Поэтому решение Конституционного суда вызвало поистине международный скандал. В
игру активно включился Зеленский, который потребовал от Рады распустить КСУ и
отменить решение по антикоррупции. В Раду уже внесен законопроект - но всем
очевидно, что он неконституционен: решения КСУ окончательны и отменить простым
законом их нельзя.

И тем более нельзя решением Рады уволить судей КСУ - по Конституции это может
сделать только сам суд 2/3 голосов.

Однако у Зеленского продолжают проталкивать этот законопроект, заявляя, что в
этом его поддерживает Запад (хотя оттуда идут противоречивые сигналы). 

И главный сейчас вопрос - найдут голоса или нет. Еще в пятницу большинство
фракций уже отказались его поддерживать. Сильные колебания идут и в "Слуге
народа". 

Параллельно идут обсуждения других - более законных - способов "сбить" работу
Конституционного суда.

Разбирались в последних перипетиях скандала вокруг КСУ. 

\subsubsection{Что предлагают "соросята"}

Сегодня один из близких к американскому посольству активистов Виталий Шабунин
опубликовал идеи, как "разобраться" с Конституционным судом. 

Он в целом похвалил идею Зеленского неконституционным способом уволить всех
судей и отменить их решение по НАПК. И заявил, что она позволит решить проблему
быстро.

Но в качестве альтернативы предложил еще один вариант - изменить закон о
Конституционном суде и повысить в нем кворум для принятия решений - до 15 или
18 человек. 

Учитывая, что такого кворума там никогда не будет, то работа КС окажется
полностью парализованной. 

Правда, в своем посте сам же Шабунин пишет, что судьи могут признать этот закон
неконституционным и продолжать работать дальше. Поэтому он подводит к мысли,
что нужно голосовать за законопроект Зеленского, хоть он и антиконституционен.

\ifcmt
pic https://strana.ua/img/forall/u/0/92/image_2020-11-02_114031.png
\fi

Телеграм-канал "Политика Страны" пишет еще об одной угрозе варианта с кворумом.

"Если эта норма все-таки заработает, то будут парализованы любые изменения в
Конституцию, которые должны проходить предварительную проверку в КСУ", -
говорится в сообщении. Что в будущем заблокирует многие необходимые властям
решения - например, по децентрализации. 

\subsubsection{Игра Разумкова}

Сегодня вечером должно состояться собрание фракции "Слуги народа". Туда должен
приехать президент Зеленский. Главная тема - убеждение депутатов проголосовать
за его законопроект.

Накануне президент выступил с резким заявлением против Конституционного суда,
назвав его "судом кенгуру", а его судей "чертями". То есть, дал понять, что
настаивает на их увольнении.

В то же время далеко не все "слуги народа" готовы проголосовать за этот явно
антиконституционный документ.

По данным "Страны", пошли в отказ сразу десятки депутатов. И голосов не
хватает.

Параллельно свою игру начал спикер Дмитрий Разумков.  

Он продвигает альтернативный законопроект, которым просто восстанавливаются
отменённые КСУ статьи антикоррупционных законов. Без увольнения судей.

\url{https://t.me/stranaua/9053}

И как раз за него голоса могут найтись.

Но пока неизвестно, как отреагирует президент на активность спикера, который,
по сути, рубит на корню идею Зеленского.

Президент уже дал понять в предельно жесткой форме, что настаивает на принятии
своего законопроекта. Он разослал "слугам народа" угрожающее аудиопослание, в
котором пугает кровопролитием.

"Странное поведение судей Конституционного суда в считанные часы поставило
страну на грань катастрофы. Или страна снова будет втянута в кровавый хаос, или
государство снова как система прозрачных правил и договоренностей прекратит
свое существование. Или мы наконец станем действительно мощной политической
силой и скажем свое слово - "нет". "Нет лжи. Время требует от нас правильных
эмоций и смелости ... Сегодня надо показать свое единство, показать, что мы
партия, которая умеет драться за себя, за свое имя, за свою политику. Неужели
вы не видите, что другие партии, которые еще вчера должны были уйти в небытие,
стоят в стороне и смеются. Может докажем, что мы намного сильнее?" --- сказал
Зеленский.

При этом решение по Конституционному суду, утверждает президент, обязательно
должно быть принято.

"Это даже не просьба. Никаких просьб, когда уничтожают ценности. Проживем и без
просьб".

"История не судит победителей. И если я решу, что путь разрешения кризиса
возможен только через предложенный закон - надо обязательно поддержать", -
заявил президент. 

Посмотрим, что ответят "слуги народа". Вероятно, ситуация станет понятнее после
сегодняшнего заседания фракции, которое намечено на 19-00.

\subsubsection{Письма из Европы}

Не очень обнадеживающие для президента сигналы идут и из-за рубежа. Если
американское посольство и Госдеп по факту заняли позицию молчаливого одобрения
"мочилова" Конституционного суда, то у структур европейских кардинально иная
позиция. 

В субботу на имя спикера Разумкова поступило обращение сразу двух европейских
структур, которые выступили резко против "зеленского" законопроекта. Речь о
Венецианской комиссии и GRECO - Группы государств против коррупции. 

В документе говорится, что принудительное прекращение полномочий судей
Конституционного суда является вопиющим нарушением Основного закона и принципа
разделения властей.

В организациях согласились с важностью обеспечения доверия к Конституционному
суду. Однако против того, чтобы этот орган делали зависимым от Рады. 

"Доверие не означает отсутствие независимости. Напротив, Конституционный суд
может пользоваться доверием других государственных институтов и общественности
только в том случае, если он действительно независимый", - отмечается в письме.

Организации призывают украинскую власть рассмотреть неблагоприятные, глубокие и
долгосрочные последствия для Украины от поспешного решения об увольнении судей
Конституционного суда.

"Мы призываем вас изучить альтернативные способы обеспечения того, чтобы борьба
с коррупцией в соответствии с международными стандартами оставалась приоритетом
для вашей страны", - говорится в письме.

\ifcmt
pic https://strana.ua/img/forall/u/0/92/%D0%BF%D0%B8%D1%81%D1%8C%D0%BC%D0%BE1.png
\fi

У Зеленского на это ответили, что разгон КСУ "является единственным возможным
выходом из сложившейся ситуации". 

"Я считаю, что КСУ своим решением фактически перечеркнул свою легитимность.
Вряд ли этот состав КСУ может адекватно реагировать на оценку тех законов,
которые ему инициированы. Поэтому я считаю, что законопроект абсолютно
адекватен", - заявил представитель президента в Конституционном суде Федор
Вениславский. 

"В данной ситуации чисто правовой выход сложно найти. Должно быть
политико-правовое решение. Именно такое решение на заседании СНБО предложил
президент, то есть законопроект, который внесен в ВР. А Рада должна принять
соответствующее решение", - добавил он.

Но, повторимся, пока в самой фракции "слуг народа" единства по данному поводу
нет.

\subsubsection{Второй фронт КСУ: украинизация}

Одним из фоновых, но важнейших факторов развернувшейся борьбы проходит тема
отмены закона о тотальной украинизации. Его сейчас рассматривают в КСУ. Причем
Заседание запланировано уже на вторник и пройдет в закрытом режиме. 

Ранее глава НАПК Новиков прогнозировал, что КСУ признает его неконституционным.
Что, мол, укладывается в общую парадигму "зрады", которую развернул суд. 

Но есть и альтернативная версия.

Сотрудничающий с Офисом президента экс-нардеп Сергей Лещенко (он сейчас
превратился чуть ли не в главного лоббиста принятия антиконституционного закона
об увольнении судей КСУ) заявил, что у главы Конституционного суда Тупицкого
есть якобы хитрый план --- признать конституционным закон о языках, чем дать
основание Порошенко и Ко не голосовать за законопроект Зеленского. Ну, и вообще
таким образом усыпить бдительность "патриотической" общественности.

"Конечно, верить на слово Лещенко нельзя.

Но если у Тупицкого действительно есть такой план, то он в корне порочный.

"Патриотическую общественность" этим никак не усыпишь. Она подобный подарок не
оценит. А вот поддержку со стороны антинационалистических сил такое решение
может обнулить.

И тогда падение Тупицкого будет вопросом времени.

Вот только бежать ему после этого будет уже некуда.

Так что Тупицкому следует серьезно подумать, прежде чем запускать этот якобы
"хитрый план" с законом о языке", - анализирует ситуацию Телеграм-канал
"Политика Страны".

\url{https://t.me/stranaua/9027}

Борьба между Зеленским и Порошенко за доверие Госдепа

Мотивы действий Зеленского и их последствия анализирует главный редактор
"Страны" Игорь Гужва.

\begin{itemize}
	
\item 1. Решение Зеленского вопреки Конституции уволить судей КСУ и отменить их
				решение --- это следствие его борьбы с Порошенко и Ко за поддержку
				Госдепа США.  Порошенко и его лобби предпринимают немалые усилия, дабы
				убедить американцев в том, что Зеленский и его окружение во главе с
				Ермаком - это русские шпионы, которые хотят оторвать Украину от Запада
				и ввести ее в зону влияния России.

В копилку этой версии идет и скандал с вагнеровцами, и пленки Деркача с
разговорами Порошенко и Байдена, и попытки снять главу НАБУ Сытника и, конечно
же, решение КСУ по декларациям. Порошенко убеждает американцев, что за всем
этим стоят Зеленский и Ермак. И у Зеленского в такой активности порошенковцев
видят большую для себя угрозу. Особенно с учетом возможной победы Байдена на
выборах.

Поэтому и было принято решение дать ассиметричный ответ --- внести откровенно
неконституционный законопроект по КСУ. И таким образом показать Госдепу, что
Зе-команда не только не имеет отношения к решениям суда, но и готова ради
верности "антикоррупционным реформам" и "евроатлантическому курсу" растоптать
даже Конституцию.

Параллельно Зе-команда и ее лоббисты доказывают американцам, что за решением
КСУ стоит не Зеленский, а Порошенко, который вошел в союз с врагами США –
Коломойским и Медведчуком. То есть это Порошенко русский шпион, а не Зеленский.

\item 2. Зеленский и Ко пытаются также решить вопрос с Конституционным судом,
				который вышел из-под контроля Банковой и ныне управляется Коломойским и
				Порошенко.  Следует сказать, что в таком положении Зеленскому нужно
				винить только самого себя. В прошлом году Зе-команда предала прежнего
				главу Конституционного суда Станислава Шевчука. Шевчук еще до первого
				тура выборов фактически поддержал Зеленского, заявив в интервью
				"Стране", что не приведет к присяге президента, который победит путем
				фальсификаций (понятно, что намек был на Порошенко).  После чего против
				Шевчука был организован заговор и его уволили судьи КСУ в мае 2019 года
				накануне инаугурации. Команда Зеленского не только не препятствовала
				этому, но и устами Рябошапки одобрила переворот. А затем, когда осенью
				того же года Окружной суд восстановил Шевчука в должности, Банковая
				палец о палец не ударила, чтобы помочь ему вернуться на свою должность.
				И теперь --- закономерная расплата за предательство. Сейчас проблему Офис
				президента пытается решать за счет неконституционного законопроекта об
				увольнении всех судей КСУ. Но даже если за него и проголосуют, то это
				не означает, что новый состав суда окажется под контролем Банковой, так
				как Запад постарается максимально наполнить его своими людьми. Поэтому,
				скорее всего, будет борьба, которая заблокирует на долгое время
				формирование Конституционного суда. Хотя для Офиса президента, в
				принципе, эта ситуация будет все равно лучше, чем сейчас.

\item 3. Под шумок скандала с КСУ Банковая попытается усыпить бдительность
				американцев и убедить их в необходимости перезагрузки руководства НАБУ
				и САП.  То есть заменить Сытника в НАБУ и поставить во главе САП кадра
				от Зеленского.  Впрочем, пока вероятность того, что Запад это
				согласует, выглядит небольшой.

\item 4. Один из главных мотивов действий Офиса президента в истории с КСУ –
				показать, кто "главный на районе". Это особенно актуально после
				неудачного выступления "Слуги народа" на выборах. То есть на Банковой
				хотят показать, что, несмотря ни на что, это они в стране власть и
				больше никто. И могут продавливать даже полностью антиконституционные
				законы. Мотив понятный. Но вот только велик риск провала закона (против
				него выступает, хотя пока и не публично, спикер Разумков, а также очень
				многие депутаты от "Слуги народа"). А если законопроект провалится, то
				это будет очень сильным имиджевым ударом по президенту и может резко
				ускорить центробежные тенденции во фракции, окончательно оформив в ней
				альтернативный центр притяжения в лице спикера. Тем более что есть
				немало сил, которые готовы процесс этот простимулировать. Не говоря уже
				о том, что недобитый Конституционный суд тогда точно будет люто мстить
				Банковой и жизни ей вообще не даст, пачками отменяя законы. В общем,
				ставка для президента в этой игре крайне велика. В случае проигрыша
				последствия будут тяжелыми.

\item 5. История с Конституционным судом может стать переломной в отношениях
				Коломойского и Зеленского. Несмотря на то что Игорь Валерьевич уже
				давно не является "олигархом номер 1" при дворе Зе, он сохранял немалое
				влияние и особые отношения с президентом, что и позволяло ему до сих
				пор решать многие вопросы.  Но продавливание им в КСУ вопреки воле
				Зеленского решений по декларированию было воспринято на Банковой крайне
				болезненно. Там считают, что Коломойский вошел в сговор с Порошенко и
				фактически встал на путь войны с Зеленским. А потому отношения прежними
				уже точно не будут. Помимо прочего это приведет к росту влияния в
				окружении Зеленского Рината Ахметова, как естественного союзника
				Банковой в борьбе с Коломойским. Тем более что Ахметов и Зеленский
				взаимно дополняют друга друга. Зеленский, в отличие от Януковича, не
				строит свою бизнес-империю, грозясь потеснить Ахметова. И, в отличие от
				Порошенко, не пытается отжать у него "Нефтегазодобычу". А Ахметов не
				стремится идти в президенты и никого из своих друзей в президенты не
				двигает. У Ахметова много денег, крупнейший телеканал, но в
				политическом смысле он сейчас карлик --- у него нет ни своей фракции, ни
				сколько-нибудь значимой депутатской группы. У Зеленского власть есть
				(по крайней мере, пока), депутатов больше двух сотен, но денег мало и
				своих телеканалов нет. И тут Ахметов может сильно помочь. В конце
				концов, в вопросе убеждения нардепов голосовать за законопроект по КСУ
				финансовая составляющая будет играть не последнюю роль. Кстати, по
				слухам, "слугам народа" увеличили зарплаты в конвертах. Причем не на
				тысячу-две баксов, а сразу раза в 2-3. И для многих из них это куда
				более важная новость, чем результат партии на местных выборах.
\end{itemize}

\ifcmt
pic https://strana.ua/img/forall/u/0/92/%D0%B3%D1%83%D0%B6%D0%B2%D0%B01(13).png
\fi

