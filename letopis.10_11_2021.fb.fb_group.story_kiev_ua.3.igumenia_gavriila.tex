% vim: keymap=russian-jcukenwin
%%beginhead 
 
%%file 10_11_2021.fb.fb_group.story_kiev_ua.3.igumenia_gavriila
%%parent 10_11_2021
 
%%url https://www.facebook.com/groups/story.kiev.ua/posts/1794395227423937
 
%%author_id fb_group.story_kiev_ua,jeremenko_evgenia
%%date 
 
%%tags cerkov,gorod,kiev,pravoslavie,smert,ukraina,vera
%%title Настоятельница Львовского Свято-Преображенского монастыря игумения ГАВРИИЛА
 
%%endhead 
 
\subsection{Настоятельница Львовского Свято-Преображенского монастыря игумения ГАВРИИЛА}
\label{sec:10_11_2021.fb.fb_group.story_kiev_ua.3.igumenia_gavriila}
 
\Purl{https://www.facebook.com/groups/story.kiev.ua/posts/1794395227423937}
\ifcmt
 author_begin
   author_id fb_group.story_kiev_ua,jeremenko_evgenia
 author_end
\fi

Дорогие одногруппники, пусть эта история не покажется вам совсем не киевской.
Завершение прекрасной жизни происходило в нашем городе, много важных событий в
течение её тоже. А завтра Киев навсегда останется земной родиной для человека,
очень любившего его, знавшего историю города, любившего киевлян. Вообще всех
людей. 

Писалось вчера вечером.

9 ноября 2021

Очень грустная, хоть и ожидаемая весть. Сегодня, 9 ноября 2021,  ОТОШЛА КО
ГОСПОДУ настоятельница Львовского Свято-Преображенского монастыря игумения
ГАВРИИЛА (Галина Михайловна Далецкая, в девичестве Смирнова).

\ifcmt
  tab_begin cols=4

     pic https://scontent-lga3-2.xx.fbcdn.net/v/t39.30808-6/255660960_1555150738157611_2965994137185504266_n.jpg?_nc_cat=101&ccb=1-5&_nc_sid=b9115d&_nc_ohc=nLwBOlP1YZwAX82Tp-c&_nc_ht=scontent-lga3-2.xx&oh=932d0442acd1bc6fdc0250be15000647&oe=619240FA
		 @caption 18 июня, мы возле храма Мвт. Федосия Черниговского, Святошино. Матушка приехала на консультацию. И с того времени уже был киевский период её многострадальной жизни

     pic https://scontent-lga3-2.xx.fbcdn.net/v/t39.30808-6/255492993_1555150844824267_7001379284173581748_n.jpg?_nc_cat=108&ccb=1-5&_nc_sid=b9115d&_nc_ohc=d6Jy3I6aQ48AX_NLihB&_nc_ht=scontent-lga3-2.xx&oh=87914d123d2da2142272a8e30510e64e&oe=61911A72
		 @caption Конец июля, Пантелеимоновский монастырь. Первый выход после операции

		 pic https://scontent-lga3-2.xx.fbcdn.net/v/t39.30808-6/255713928_1555150918157593_6872458679428073174_n.jpg?_nc_cat=107&ccb=1-5&_nc_sid=b9115d&_nc_ohc=VWW57kBX-2AAX-aThQN&tn=lCYVFeHcTIAFcAzi&_nc_ht=scontent-lga3-2.xx&oh=9e235f61bbfa56f21423fe9c1ab72986&oe=6192EFCF
		 @caption Игумения Гавриила

  	 pic https://scontent-lga3-2.xx.fbcdn.net/v/t39.30808-6/255318141_1555157718156913_1610268377661530077_n.jpg?_nc_cat=103&ccb=1-5&_nc_sid=b9115d&_nc_ohc=ertlE76VGM8AX-oaK1h&_nc_ht=scontent-lga3-2.xx&oh=068a57a385bf9607ffb33f5cc79a68b1&oe=6191941B

  tab_end
\fi

Грустная ‒ это вообще ничего не сказать. Радуюсь, что матушка после невероятных
страданий последних четырёх месяцев уже отрешилась от тела и, верю, что путь её
прямо в Царство Небесное. Но по-человечески это такая боль, такое
невосполнимое, такое горькое расставание, что и верится с трудом.

\ifcmt
  ig https://scontent-lga3-2.xx.fbcdn.net/v/t39.30808-6/255176009_1555151001490918_2844470733772395973_n.jpg?_nc_cat=107&ccb=1-5&_nc_sid=b9115d&_nc_ohc=eo85iWi6-z8AX9NY5FV&tn=lCYVFeHcTIAFcAzi&_nc_ht=scontent-lga3-2.xx&oh=96d08f7e0472f8afaacdf6c50e9a58cb&oe=619272D7
  @width 0.3
  @wrap \InsertBoxR{0}
\fi

Всегда жизнерадостная, оживлённая ‒ даже в болезни, даже сразу после операции.
Всем интересующаяся и принимающая в сердце каждое твоё движение. Богато
одарённый от Бога музыкант и тонко чувствующий слово глубоко верующий человек.
Много пострадавшая в мирской жизни, потерявшая малолетнего сына, мужа.
Выстрадавшая свой путь в православие через греко-католичество. Жившая последние
почти 20 лет в немыслимых для нас условиях монастыря, расположившегося в
2-комнатной квартире: одна комната ‒ храм Божий, 2-я ‒ "келейный корпус", где,
разделённые занавесками, подвизаются 7 сестёр. Центр Львова, ни метра своей
земли. Богослужения по полному дневному кругу. 

Матушка очень болела. В мае перенесла ковид. Давно страдала хроническими
заболеваниями. Ноги кровоточили от колен, на этих ногах она выстаивала длинные
монастырские службы, была непревзойдённым регентом и прекрасным уставщиком,
готовым делиться своими знаниями и умениями с каждым, кто проявлял интерес к
богослужению. Она жила службой Богу, бесконечно верила Ему и доверяла каждый
свой шаг.    

В начале июля её прооперировали по поводу онкологии, удалив несколько органов,
были метастазы. Операция длилась около 10 часов. И вывели колостому, которая
причиняла не просто неудобства, но и мучения. После первой химии отказал
мочеточник, её оперировали ещё дважды, вывели нефростому. Она шутила: "Теперь я
богатая женщина, у меня на одном боку две сумки!" После второй химии стало
плохо. Две недели жутких мучений, без наркотиков. Но ещё вчера, когда стало
чуть лучше, она просила привезти ей нитки, чтобы связать чётки кому-то из
врачей...

Каждый день её киевских (матушку постригали в Феофании, там она находилась в
перерыве между больницами, там и отошла, окружённая заботой и участием игумении
Елисаветы и сестёр) месяцев я получала от неё весточку: смешную или просто
красивую открытку с пожеланием доброго дня, иногда она желала мне доброй ночи
по-французски (мы обе любим этот язык). На одной из присланных ею  красивых
открыток была надпись: "Никогда не знаешь, что наступит завтра: новый день или
новая жизнь". Завтра утром для матушки будет уже не новый день. 

 Очень часто, когда у неё были силы держать в руках планшет и говорить, я
 слышала её прекрасный, ласковый голос. Она вникала во все мелочи и проблемы
 моей жизни, помнила о моих родных и знакомых, всегда молилась, сочувствовала и
 сорадовалась, всегда наказывала "беречь мамочку" и рассказывала с огромной
 любовью о своей родительской семье. Беспокоилась, если в течение дня я не
 выходила на связь. Всегда ждала и с радостью читала нашу храмовую газету
 "Колокол", поддерживала меня в её издании. А сколько было переговорено об
 особенностях церковного устава, о любимом знаменном распеве. Она была не
 просто знатоком и ревнителем, она жила этим. Да мало ли о чём говорилось...
 Вот ноты, набранные и распечатанные её ручками и любовно соединённые в
 Валаамский сборник. Вот поздравления с Пасхой, с Рождеством, написанные на ею
 оформленных листах. Вот чётки ‒ короткие и подлиннее, связанные ею. А
 удивительно мягкая интонация благословения: «БÓГ благословит!» Милые штрихи,
 из которых складывается жизнь, любовное к ней отношение.

 Невероятное доверие матушки Богу буквально до смерти было поразительным. Когда
 врачи сказали, что ей нужно немедленно оперироваться, матушка всерьёз
 засомневалась, были у неё на то свои основания. Но, будучи ответственной за
 монастырь и привыкшая жить по послушанию, пошла за благословением к
 Блаженнейшему. И услышала: "Спросите ещё раз у врачей и поступайте, как они
 скажут". Ослушаться она не могла, исключительно за послушание пошла на
 операцию. За послушание прошла 2 курса химиотерапии. За послушание до
 последнего радовалась жизни, будучи на 100% зависимой от других. 

"Врачи смотрят на меня как на безнадёжную,‒ это когда выяснилось, что уже 4
стадия,‒но ПОСЛЕДНЕЕ СЛОВО ЗА ГОСПОДОМ!" И Господь давал силы терпеть и жить.

Ухаживала за матушкой инокиня их монастыря Марфа, перед которой я готова
скло-нить голову не до земли, а ещё ниже за то терпение и безграничную любовь,
которые она, будучи сама не очень здоровым человеком, проявляла по отношению к
своей игумении ежесекундно. 

Ровно 70 лет прожила матушка Гавриила среди нас, подарив окружающим мегатонны
любви, обогатив наш общий мир своим сердечным теплом, радостью, высочайшим
профессионализмом как педагог (преподавала в педучилище фортепиано) и
прекрасный музыкант, как церковный регент, уставщик, деликатнейший руководитель
‒ настоятельница. Во Львове для меня это было самое тёплое место, где ты был
безоговорочно своим, где тебя принимали со всеми особенностями, просто как
образ Божий, что в каждом человеке. Незабываемые акафисты перед Иверской иконой
по воскресеньям после литургии, задушевные беседы за трапезой. Мудрые советы в
сложных житейских ситуациях без всякого пафоса: "Я же не старица! Мне просто
кажется, что не нужно торопиться в принятии решения!" 

Необременительность даже в самые невыносимые для неё моменты и при просьбах о
помощи. Как-то так она могла попросить о помощи или рассказать об ужасном,
происходящем с нею, что не было никогда безысходности в тоне, той
безысходности, что придавила бы другого человека. 

Матушка удивительно умела щадить человека, без ложной нежности. Ограждала, в
самом лучшем смысле жалела. Чужих для неё не было. Впервые увидев кого-то, она
уже принимала его в свою жизнь. Могла и решительно сказать, если нужно было, но
никогда не обидно. И сразу же посочувствовать человеку, пожалеть его,
оступившегося.  

Последним её подарком мне было то, что последние 10 дней, написав два слова:
"Очень плохо", она попросила передать, чтобы её не беспокоили, что она сама,
окрепнув, выйдет на связь. И мне верилось, что окрепнет, ведь "последнее слово
за Господом". Может быть хотела остаться в памяти радостной. 

А позавчера попросила передать, что очень меня любит. В ответ я засмеялась: "Я
знаю! Конечно, я знаю, что она меня любит!". Иначе и быть-то не могло. За
столько лет знакомства я ничего другого от неё ведь не чувствовала. Но
примешалось и ещё одно, отразилось в таком дерзком ответе.

Летом, когда мы с сёстрами по храму навещали матушку после операции в
Пантелеимоновском монастыре в Феофании,  она много рассказывала о своей жизни.
И как-то рассказала о своей преподавательнице французского языка во Львовской
консерватории. 

Это была красивая, осанистая женщина, отсидевшая 10 лет в лагерях. Её родители
эмигрировали из ставшей советской России, она родилась во Франции. После войны
СССР объявил, что уехавшие могут вернуться. И её родители засобирались домой.
Несовершеннолетняя дочь (ей было 18, а во Франции совершеннолетие - 21 год) не
хотела ехать в незнакомую страну. Но родители настояли. Вернулись. Голод,
продуктов практически нет. Через какое-то время по возвращении девушка, идя по
улице, увидела в витрине яблоки, вошла и попросила взвесить ей сколько-то.
"Яблок нет! Это только на витрине!" - ответили ей. Она не поверила, возмутилась
обманом. И на следующий день уже была на допросе - кто-то донёс, что она
недовольна советским строем. 

Матушка рассказывала, что была в этой женщине, вернувшейся после 10 лет
лагерей, какая-то отрешённость, но педагогом она оказалась талантливейшим. И
её, не открывавшуюся людям, студенты полюбили. "Сколько раз я собиралась зайти
к ней, просто сказать, что я её люблю! Но то времени не было, то денег, чтобы
купить что-нибудь приятное, то ещё что-то... И вдруг я узнала, что она умерла.
А я не успела сказать, что мы все её любим, что я её люблю. С тех пор я боюсь
не успеть сказать людям, которых люблю, об этом".

Матушка успела передать мне эти слова, но я, глупая, не смогла сообразить и
принять, что она прощается. Душа отказывалась признать, что час её пришёл. Не
сказала, как много она для меня значит, не попросила прощения перед разлукой.

Так странно сейчас написать "Царство Небесное", душа противится. Но ‒ Царство
Небесное, дорогая матушка Гавриила, вечный покой и благодарная память за всё
доброе, чем Вы так щедро дарили. 

В четверг тело ляжет туда, куда и желалось: рядом со схиигумением Вероникой -
первой настоятельницей львовского монастыря и ближайшей духовной наставницей и
сестрой. "Меня вот сюда положат, рядом с матушкой",‒ говорила она, придя на
могилу мат. Вероники в Феофании. "А если занято будет, то хоть в ножках у неё".
Место с декабря 2014 года не занято.

Смиренно прошу всех, у кого будет на то произволение сердечное, по возможности
помолиться о новопреставленной игумении ГАВРИИЛЕ. А её милый лик на этих
фотографиях пусть будет нам в благословение на достойную жизнь.
