% vim: keymap=russian-jcukenwin
%%beginhead 
 
%%file 07_11_2019.fb.sotnik_katerina.1.bulgakov_parad_kiev
%%parent 07_11_2019
 
%%url https://www.facebook.com/permalink.php?story_fbid=539232633568829&id=100024462908393
 
%%author_id sotnik_katerina
%%date 
 
%%tags 1918,belaja_gvardia_roman,bulgakov_mihail,istoria,kiev
%%title БУЛГАКОВ, ДЕКАБРЬСКИЙ ПАРАД В КИЕВЕ И МЕТАМОРФОЗЫ СУДЬБЫ КОЗЫРЯ-ЗИРКЫ
 
%%endhead 
 
\subsection{БУЛГАКОВ, ДЕКАБРЬСКИЙ ПАРАД В КИЕВЕ И МЕТАМОРФОЗЫ СУДЬБЫ КОЗЫРЯ-ЗИРКЫ}
\label{sec:07_11_2019.fb.sotnik_katerina.1.bulgakov_parad_kiev}
 
\Purl{https://www.facebook.com/permalink.php?story_fbid=539232633568829&id=100024462908393}
\ifcmt
 author_begin
   author_id sotnik_katerina
 author_end
\fi

БУЛГАКОВ, ДЕКАБРЬСКИЙ ПАРАД В КИЕВЕ И МЕТАМОРФОЗЫ СУДЬБЫ КОЗЫРЯ-ЗИРКЫ 

Михаил Булгаков описывает в \enquote{Белой гвардии} военный парад 19 декабря 1918, в
честь взятия Киева войсками Директории:

– «В синих жупанах, в смушковых, лихо заломленных шапках с синими верхами шли
галичане. Два двухцветных прапора, наклоненных меж обнаженными шашками, плыли
следом за густым трубным оркестром, а за прапорами, мерно давя хрустальный
снег, молодецки гремели ряды, одетые в добротное, хоть немецкое сукно...

\begin{multicols}{2} % {
\setlength{\parindent}{0pt}

\ii{07_11_2019.fb.sotnik_katerina.1.bulgakov_parad_kiev.pic.1}
\ii{07_11_2019.fb.sotnik_katerina.1.bulgakov_parad_kiev.pic.1.cmt}

\ii{07_11_2019.fb.sotnik_katerina.1.bulgakov_parad_kiev.pic.2}
\ii{07_11_2019.fb.sotnik_katerina.1.bulgakov_parad_kiev.pic.2.cmt}

\ii{07_11_2019.fb.sotnik_katerina.1.bulgakov_parad_kiev.pic.3}
\ii{07_11_2019.fb.sotnik_katerina.1.bulgakov_parad_kiev.pic.3.cmt}

\ii{07_11_2019.fb.sotnik_katerina.1.bulgakov_parad_kiev.pic.4}
\ii{07_11_2019.fb.sotnik_katerina.1.bulgakov_parad_kiev.pic.4.cmt}

\ii{07_11_2019.fb.sotnik_katerina.1.bulgakov_parad_kiev.pic.5}
\ii{07_11_2019.fb.sotnik_katerina.1.bulgakov_parad_kiev.pic.5.cmt}

\ii{07_11_2019.fb.sotnik_katerina.1.bulgakov_parad_kiev.pic.6}
\ii{07_11_2019.fb.sotnik_katerina.1.bulgakov_parad_kiev.pic.6.cmt}

\ii{07_11_2019.fb.sotnik_katerina.1.bulgakov_parad_kiev.pic.7}

\ii{07_11_2019.fb.sotnik_katerina.1.bulgakov_parad_kiev.pic.8}
\ii{07_11_2019.fb.sotnik_katerina.1.bulgakov_parad_kiev.pic.8.cmt}

\ii{07_11_2019.fb.sotnik_katerina.1.bulgakov_parad_kiev.pic.9}
\ii{07_11_2019.fb.sotnik_katerina.1.bulgakov_parad_kiev.pic.9.cmt}
\end{multicols} % }
 
Шли курени гайдамаков, пеших, курень за куренем, и, высоко танцуя в
просветах батальонов, ехали в седлах бравые полковые, куренные и ротные
командиры. Удалые марши, победные, ревущие, выли золотом в цветной реке. За
пешим строем, облегченной рысью, мелко прыгая в седлах, покатили конные
полки. 

Ослепительно резнули глаза восхищенного народа мятые, заломленные папахи с
синими, зелеными и красными шлыками с золотыми кисточками. Пики прыгали, как
иглы, надетые петлями на правые руки. Весело гремящие бунчуки метались среди
конного строя, и рвались вперед от трубного воя кони командиров и
трубачей... 

Трепля простреленным желто-блакитным знаменем, гремя гармоникой, прокатил
полк черного, остроусого, на громадной лошади, полковника Козыря–Лешко. Был
полковник мрачен и косил глазом и хлестал по крупу жеребца плетью.

За Козырем пришел лихой, никем не битый черноморский конный курень имени
гетмана Мазепы. Имя славного гетмана, едва не погубившего императора Петра
под Полтавой, золотистыми буквами сверкало на голубом шелке. 

Народ тучей обмывал серые и желтые стены домов, народ выпирал и лез на тумбы,
мальчишки карабкались на фонари и сидели на перекладинах, торчали на крышах,
свистали, кричали:

ура... ура... – Слава! Слава! – кричали с тротуаров... – Ото казалы банды...
Вот тебе и банды. Ура! – Слава! Слава Петлюри! Слава нашему Батьку!».


– Кто же такой – этот боевой полковник Козырь-Лешко, чьи войска были
авангардом в осаде и взятии Киева и который был мрачен на параде, ибо
"побили най-турсовы залпы в туманное утро на Брест-Литовской стреле лучшие
Козырины взводы, и шел полк рысью и выкатывал на площадь сжавшийся,
поредевший строй" ?

– Его прообразом является Алексей Козырь-Зирка.

«Во время Антигетманского восстания в составе Осадного корпуса Евгения
Коновальца, бравшего Киев, был лишь один конный полк – 1-й конно- партизанский
сечевых стрельцов атамана Алексея Козыря-Зирки. 

Козырь-Зирка и стал прообразом для создания литературного героя Козыря-Лешко.
Откуда его знал Михаил Булгаков, до сих пор сказать сложно, поскольку фамилия
этого атамана в прессе не появлялась. Тем не менее, знакомство с атаманом у
будущего писателя было, поскольку некоторые подробности его боевой жизни
Булгаков перенес в свою "Белую гвардию".

Биография атамана Алексея Козыря-Зирки достаточно сложная и запутанная. В
Овруче, где атаман долгое время был царьком, поговаривали, что Козырь-Зирка —
это вовсе не фамилия, что он некий граф из Белой Церкви, или же — беглый
галицкий каторжник, руки которого действительно были испещрены татуировками.

Совершенно другим рисует нам Козыря-Зирку Михаил Булгаков: "Всю свою жизнь до
1914 года Козырь был сельским учителем. В четырнадцатом году попал на войну в
драгунский полк и к 1917 году был произведен в офицеры. А рассвет
четырнадцатого декабря восемнадцатого года под оконцем застал Козыря
полковником петлюровской армии, и никто в мире (и менее всего сам Козырь) не
мог бы сказать, как это случилось. 

А произошло это потому, что война для него, Козыря, была призванием, а
учительство лишь долгой и крупной ошибкой. Так, впрочем, чаще всего и
бывает в нашей жизни. Целых лет двадцать человек занимается каким-нибудь
делом, например, читает римское право, а на двадцать первом — вдруг
оказывается, что римское право ни при чем, что он даже не понимает его и
не любит, а на самом деле он тонкий садовод и горит любовью к цветам.
Происходит это, надо полагать, от несовершенства нашего социального строя,
при котором люди сплошь и рядом попадают на свое место только к концу
жизни. Козырь попал к сорока пяти годам. А до тех пор был плохим учителем,
жестоким и скучным".

Честно говоря, мы не знаем, был ли когда-нибудь Козырь-Зирка сельским учителем.
Странным кажется и возраст, предопределенный Михаилом Булгаковым — 45 лет.
Опять таки, о реальном возрасте Козыря-Зирки писатель мог и не знать, как,
впрочем, не знаем этого и мы. 

Вполне возможно, что Михаил Булгаков провел аналогию между "полковником"
1-го Синежупанного полка Пащенко (которого писателю пришлось очень хорошо
узнать) и командиром 1-го конно-партизанского полка Козырь-Зиркой. 

Для полноты образа М. Булгаков действительно мог приписать по примеру
Пащенко Козырь-Зирке сельское учительство. В целом же образ Козыря-Лешко,
созданный Михаилом Булгаковым, совпадает с реально существовавшим
Козырь-Зиркой.

А что мы знаем о Козырь-Зирке? Алексей Козырь-Зирка происходил из богатой
казацкой семьи с Екатеринославщины. Был падок на деньги и женские юбки. 

В 1914-м году был мобилизован в российскую армию, служил в драгунском полку,
откуда каким-то неведомым образом попал в Туземную конную дивизию,
сформированную из представителей кавказских народов. 

Всю войну Козырь-Зирка провел на коне, был унтер-офицером, а в 1917 году
получил первое офицерское звание прапорщика. Осенью 1917 года Козырь-Зирка,
как потомственный казак-запорожец, перевелся в 3-ю кавалерийскую дивизию, в
которой создавался украинский конный полк. 

Будучи блестящим оратором, прапорщик Козырь-Зирка быстро приобрел
сторонников в солдатской среде и в скором времени был избран на должность
командира украинского конного полка. Но этот полк просуществовал недолго —
уже в декабре 1917 года большевики разгромили его в районе Екатеринослава.
Самому Козырь-Зирке пришлось спасаться бегством. 

Будущий атаман прибыл в Киев, где принимал активное участие в боях между
большевиками и украинскими частями в начале февраля 1918 года (второй и
третьей властями). 

Вскоре А. Козырь-Зирка поступил в конный полк имени К.Гордиенко войск
Центральной Рады, с которым участвовал в весеннем наступлении украинских и
немецких подразделений против большевиков на Полтавщине, в Причерноморье и
Крыму. 

Во время правления Скоропадского Козырь-Зирка примкнул к восставшим
Киевщины, а после разгрома Таращанско- Звенигородского восстания скрывался
среди сечевиков. Вместе с ними из Белой Церкви он выступил в поход против
Скоропадского, по приказу Коновальца сформировал конный полк, ставший
кавалерией Сечевой дивизии. 

Во время осады Киева Козырь-Зирка по собственно инициативе несколько раз
нарушал перемирие между УНР и немцами, совершая набеги на киевские
предместья. После занятия Киева полк Козыря-Зирки был отправлен в район
Овруча и Коростеня, где атаман "прославился" своими еврейскими погромами.
Полк атамана был расформирован, а сам Козырь-Зирка чуть не попал под суд. 

«До лета 1919 года Козырь-Зирка с небольшим отрядом болтался по тылам
украинской армии, пока его чуть не схватила петлюровская контрразведка.
Алексей Козырь-Зирка бежал к красным. По некоторым свидетельствам, у них
он етап служить в… ЧК. Как сложилась дальнейшая судьба бывшего атамана, мы
не знаем. Открытым остается вопрос и о знакомстве Козыря-Зирки с
Булгаковым...».

(Я.Тинченко. Белая гвардия Михаила Булгакова).

– «Были десятки тысяч людей, вернувшихся с войны и умеющих стрелять…

— А выучили сами же офицеры по приказанию начальства!

Сотни тысяч винтовок, закопанных в землю, упрятанных в клунях и коморах и не
сданных, несмотря на скорые на руку военно-полевые немецкие суды, порки
шомполами и стрельбу шрапнелями, миллионы патронов в той же земле и
трехдюймовые орудия в каждой каждой пятой деревне и пулеметы в каждой второй,
во всяком городишке склады снарядов, цейхгаузы с шинелями и папахами.

И в этих же городишках народные учителя, фельдшера, однодворцы, украинские
семинаристы, волею судеб ставшие прапорщиками, здоровенные сыны пчеловодов,
штабс-капитаны с украинскими фамилиями… все говорят на украинском языке, все
любят Украину волшебную, воображаемую, без панов, без офицеров-москалей, — и
тысячи бывших пленных украинцев, вернувшихся из Галиции.

Это в довесочек к десяткам тысяч мужичков?.. О-го-го!

Вот это было…»

(М.Булгаков. Белая гвардия).

Що ж нині у підсумку відомо про прототипами персонажа М.Булгакова ?

Олекса Козир-Зірка — був командиром полку Дієвої армії УНР.

Походив з козацького роду та селян Катеринославської губернії. 

Під час Першої світової війни служив однорічником у Кавказькій Туземній
("Дикій") дивізії. Останнє звання у російській армії — прапорщик.

Восени 1917 р. перевівся до 3-го драгунського Новоросійського полку, що
українізовувався. У грудні 1917 р. — січні 1918 р. — виборний командир цього
полку, перейменованого на 1-й Український драгунський Новоросійський. Згодом
виїхав до Києва, де брав участь у січневих (1918 р.) вуличних боях проти
більшовиків. 

Після відступу з Києва 09.02.1918 р. вступив до кулеметного відділу 3-го
Запорізького куреня ім. К. Гордієнка військ Центральної Ради (згодом — 1-й
Кінно-Гайдамацький полк ім. К.Гордієнка). 

У червні 1918 р. підбурював склад полку до виступу проти влади гетьмана
П.Скоропадського, через що полк було кадровано до сотні, а Козир-Зірка мусив
переховуватись на Катеринославщині. 

На початку листопада 1918 р. з'явився у розпорядження командувача Окремого
загону Січових стрільців Є. Коновальца та отримав від нього доручення на
формування 1-го кінно-партизанського полку Січових стрільців. Після
сформування полку очолив його на фронті під Києвом. Згодом полк розташувався
в Овручі як залога, де відзначився погромами

У січні 1919 р. полк Козира-Зірки залишив місто Овруч перед наступом червоних і
був переведений на Чернігівщину, однак тут відзначився таким мародерством, що
українське командування віддало наказ про його роззброєння. 

Отаман Козир-Зірка самовільно виїхав у район ст. Бирзула, де у лютому 1919 р.
разом з полком увійшов до складу Запорізької Січі отамана Божка (командував
кіннотою). У травні 1919 р. був заарештований за звинуваченням у Овруцькому
погромі, ув'язнений та засуджений до розстрілу. Під час розстрілу вдалося
втекти. 

За іншими даними – слідство проводилося у Кам'янці-Подільському, але не
завершилося. У жовтні 1919 р. місто було зайняте Збройними Силами Півдня
Росії і Козир-Зірку звільнили.

За деякими даними у 1924 р. жив в Катеринославі, де працював у губернському ЧК. Подальша доля невідома.

Середа М. Отаман Козир-Зірка//Літопис Червоної Калини. — Львів. — 1930. — Ч.
11. — С. 11–13; Петрів В. Спомини з часів української революції (1917–1921). —
Львів. — 1927. — Ч. 1. — С. 147; Криловецький І. Мої спогади з часів збройної
визвольної боротьби//За Державність. — Торонто. — 1964. — Ч. 10. — С. 220–230.

\zzrule

– Полковник Микола Чеботарів, перший начальник контррозвідки Армії УНР
("Петлюрівське ЧК"), начальник особистої охорони Симона Петлюри, організатор і
виконавець смертних присудів:

“Козір-Зірка, – повстанець проти гетьмана, але ще в 1919-му році перекинувся до
большевиків і як їхній агент працював на їхню користь. 

Симон Васильович Петлюра завжди до нього негативно ставився. 

І під час евакуації Камянця в 1919 році Козир-Зірка був заарештований і
засуджений до розстрілу. Виконання присуду відбувалося в ночі, завдяки чому
ранений в руку втік.

Большевики уміло використовували його українство для провокаційної роботи
середь українського населеня, зовнішній вигляд українця-запорожця найкраще
спріяли виконанню ганебних планів большевиків. До сього часу Козір-Зірка у
большевиків на урядовій посаді”.

 (ЦДАВО України, ф. 4453, оп. 1, спр. 19, арк. 68 – 69 зв.).

\zzrule

– Необхідно враховувати, що позасудовий розстріл Козир-Зірки був не за
"більшовизм" (про його ідейне українство пише і його противник М.Чеботарів
– голова контррозвідки Директорії) не залишив йому шансів продовжувати
боротьбу в складі Армії УНР. Лише після цього він перейшов на бік
більшовиків, які боролися проти денікінців і тих, хто прирік його на
смерть. 

P.S. «

– « ...И было другое — лютая ненависть. Было четыреста тысяч немцев, а вокруг них
четырежды сорок раз четыреста тысяч мужиков с сердцами, горящими неутоленной
злобой. 

О, много, много скопилось в этих сердцах. И удары лейтенантских стеков по
лицам, и шрапнельный беглый огонь по непокорным деревням, спины,
исполосованные шомполами гетманских сердюков, и расписки на клочках бумаги
почерком майоров и лейтенантов германской армии...

И реквизированные лошади, и отобранный хлеб, и помещики с толстыми лицами,
вернувшиеся в свои поместья при гетмане, — дрожь ненависти при слове
«офицерня».

Вот что было-с.

Да еще слухи о земельной реформе, которую намеревался произвести пан гетман.

Увы, увы! Только в ноябре восемнадцатого года, когда под Городом загудели
пушки, догадались умные люди, а в том числе и Василиса, что ненавидели мужики
этого самого пана гетмана, как бешеную собаку — и мужицкие мыслишки о том, что
никакой этой панской сволочной реформы не нужно, а нужна та вечная, чаемая
мужицкая реформа:

\obeycr
— Вся земля мужикам.
— Каждому по сто десятин.
— Чтобы никаких помещиков и духу не было.
— И чтобы на каждые эти сто десятин верная гербовая бумага с печатью — во владение вечное, наследственное, от деда к отцу, от отца к сыну, к внуку и так далее.
— Чтобы никакая шпана из Города не приезжала требовать хлеб. Хлеб мужицкий, никому его не дадим, что сами не съедим, закопаем в землю.
— Чтобы из Города привозили керосин.
— Ну-с, такой реформы обожаемый гетман произвести не мог.
Были тоскливые слухи, что справиться с гетманской и немецкой напастью могут только большевики, но у большевиков своя напасть:
— Жиды и комиссары.
— Вот головушка горькая у украинских мужиков!
\restorecr

Ниоткуда нет спасения!!!» (М.Булгаков.Белая гвардия).

\ii{07_11_2019.fb.sotnik_katerina.1.bulgakov_parad_kiev.cmt}
