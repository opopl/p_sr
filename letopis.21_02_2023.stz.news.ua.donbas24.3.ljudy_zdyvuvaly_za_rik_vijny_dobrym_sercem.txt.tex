% vim: keymap=russian-jcukenwin
%%beginhead 
 
%%file 21_02_2023.stz.news.ua.donbas24.3.ljudy_zdyvuvaly_za_rik_vijny_dobrym_sercem.txt
%%parent 21_02_2023.stz.news.ua.donbas24.3.ljudy_zdyvuvaly_za_rik_vijny_dobrym_sercem
 
%%url 
 
%%author_id 
%%date 
 
%%tags 
%%title 
 
%%endhead 

21_02_2023.jana_ivanova.donbas24.ljudy_zdyvuvaly_za_rik_vijny_dobrym_sercem
Яна Іванова (Маріуполь)
Україна,Війна,Волонтер,date.21_02_2023

Люди, які здивували за рік війни своїм добрим серцем (ФОТО)

Українці за рік війни та незламності показали щирі серця та рятували один
одного

Донбас24 пропонує дізнатися про надзвичайних людей, які з початку
повномасштабного вторгнення робили усе можливе, аби врятувати та допомогти
людям, які опинилися у біді через війну.

Російська агресія круто змінила життя та сенси українців. Ті, хто раніше
займався музикою, читав лекції або просто вів спортивні заняття, несподівано
брав зброю або починав волонтерити.

Читайте також: Міцніші за залізо: як ЗСУ героїчно боронять Донеччину в умовах
зими (ФОТО) 18-річний опікун В'ячеслав

В'ячеслава на Донеччині з 37-річною матір'ю ворожий обстріл застав по дорозі
додому. Надати медичну допомогу матері не вдалося вчасно — без неї залишилися
п'ятеро дітей. Коли родина 18-річного хлопця евакуювалася у Львівську область,
він взяв опіку над чотирма братами й сестрами, які вже ходять у школу.
Забезпечує їх сам В'ячеслав, який до того ж мріє стати лікарем.

Читайте також: Зеленський та українці на обкладинці Time: хто став символом
Духу України (ФОТО)

Денис Мінін — витягував людей з Маріуполя

Денис Мінін — відомий в Маріуполі та Україні телеведучий. Коли почалося
повномасштабне вторгнення, він прагнув врятувати з обложеного міста свою
родину. Він виїхав у Запоріжжя та почав шукати способи порятунку батьків.
Згодом його ініціатива виросла у великий волонтерський проєкт. Перша
експедиція, яку йому вдалося здійснити, сталася в середині березня 2022 року.
Відтоді волонтери скоординувалися та, ризикуючи життям, вивозили маріупольців.
Їм доводилося терпіти знущання окупантів, перебувати під обстрілами, дехто
потрапив у полон.

Вже в середині червня завдяки ініціативі Дениса Мініна було врятовано тисячі
маріупольців, зокрема й 200 дітей різного віку. Також у Запоріжжі він
організував гуманітарний штаб для допомоги переселенцям. У жовтні 2022 року у
це приміщення влучила російська ракета.

Читайте також: Нацгвардієць евакуював понад 500 воїнів з «гарячих» точок
Донбасу (ФОТО)

П'ятирічна волонтерка Марія

Переселенка 5 років Марія Макєєва з Кривого Рогу стала наймолодшою волонтеркою
України. Своїми співами разом з 9-річним братом вона назбирала 35 тисяч гривень
на тепловізор для ЗСУ. Марія у центрі Львова виконувала українські пісні. Їй
акомпанував брат — виступ тривав 9 годин 54 хвилини! Всі охочі могли підтримати
малечу та пожертвувати гроші на підтримку Збройних сил України.

Віктор Разживін — вивіз 25 тисяч українських книжок

Цей чоловік з Донеччини здивував усю Україну своїм вчинком. Коли у рідному для
нього Слов'янську почалися перші обстріли, він зрозумів, що варто рятувати не
тільки людей, але й українську мову. Книги для кандидата філологічних наук та
власника магазину Віктора Разживіна завжди були сенсом. Він мав книгарню у
Слов'янську і розумів, що в разі жахливого розвитку подій росіяни не будуть
берегти українські книжки.

«Уявімо, що це все залишилося у місті — для мене була б трагедія, якби всі
книжки згоріли. Рукописи не горять, а книжки горять і дуже гарно. Вони не
повинні горіти, вони повинні читатися. І мені взагалі не хотілося, щоб подібне
відбулося», — розповідав він свого часу сайту Донбас24.

Так, специфічний вантаж пакували три дні і за один раз доправили у Київ. Так 25
тисяч книжок з українськими творами були врятовані.

Читайте також: Слово 2022 року: який вираз став головним в рік війни

Нагадаємо, що «Укрпошта» випустила марку до річниці війни з посланням до
Путіна.

Ще більше новин та найактуальніша інформація про Донецьку та Луганську області
в нашому телеграм-каналі Донбас24.

Головне фото — pon.org.ua
