% vim: keymap=russian-jcukenwin
%%beginhead 
 
%%file 28_12_2021.yz.djadjka_chernomor.1.vojna_s_ukrainoj
%%parent 28_12_2021
 
%%url https://zen.yandex.ru/media/chernomorr/budet-li-voina-s-ukrainoi-somnitelno-no-vse-je-61c9d58f7cc2e0105ac0964a
 
%%author_id yz.djadjka_chernomor
%%date 
 
%%tags napadenie,rossia,ugroza,ukraina,vojna
%%title Будет ли война с Украиной? Сомнительно, но всё же...
 
%%endhead 
 
\subsection{Будет ли война с Украиной? Сомнительно, но всё же...}
\label{sec:28_12_2021.yz.djadjka_chernomor.1.vojna_s_ukrainoj}
 
\Purl{https://zen.yandex.ru/media/chernomorr/budet-li-voina-s-ukrainoi-somnitelno-no-vse-je-61c9d58f7cc2e0105ac0964a}
\ifcmt
 author_begin
   author_id yz.djadjka_chernomor
 author_end
\fi

Шутки-шутками, но ситуация с Украиной становится всё тревожнее. То, что, не так
уж давно, казалось совершенно невозможным, начинает приобретать грозные
очертания, хотя пока и неясные. Статьи о возможности возобновления военных
действий на Украине заполнили медиапространство, и в воздухе запахло порохом.

\ii{28_12_2021.yz.djadjka_chernomor.1.vojna_s_ukrainoj.pic.1}

Откуда появились эти тревожные слухи? В основном из украинских и западных СМИ
дружно поднявших шум о перемещении российских вооружённых сил вблизи границы с
Украиной. Они ссылаются на то, что их разведка из космоса зафиксировала
увеличение числа боевой техники и военнослужащих сразу в нескольких местах
вдоль украинской границы.

Ну и что? Само по себе размещение войск в непосредственной близости от
государства, проводящего отъявленную антироссийскую политику вполне оправдано.
Вряд ли есть смысл сосредотачивать мобильные войска где-нибудь в удалённом от
границ месте, например под Якутском или Сыктывкаром. От кого там защищаться?

Раздражает упорно навязываемая мировому сообществу мысль о том, что это злобная
Россия готовится напасть на миролюбивую Украину. Извините! Но разве вооруженные
силы используются только для нападения? Исторический опыт доказывает, что
российские войска в основном нацелены на защиту от внешнего вмешательства.

Возможна ли вообще война между Украиной и Россией? Какая необходимость России
воевать с этой недостраной. Была бы необходимость, мы бы легко, без
существенных потерь, ещё в 2014 году, вместе с Крымом освободили от украинских
националистов и Харьков и Одессу и еще с пяток украинских регионов.

Хотя теоретически война может случиться. Но только в том случае, если
украинские вооружённые силы захотят силой оружия вернуть под своё подчинение не
согласные со случившимся госпереворотом и восставшие против насильственной
украинизации Донецк и Луганск, в которых около трети населения уже имеют
российское гражданство. Россия наверняка этого не позволит, о чём она
неоднократно Украину предупреждала. О Крыме я даже не стану вовсе упоминать.

Из каких бы клоунов не происходило сегодняшнее политическое и военное
руководство Украины, они безусловно понимают разницу в их и наших силах, но
если они раньше надеялись на военную помощь с Запада, то сейчас получили вполне
определённый ответ о неготовности стран НАТО принимать участие в столь
серьёзном вооружённом конфликте.

Заявить о неучастии в военных действиях они-то заявили, но втихаря продолжают
подзуживать и поддерживать воинственное настроение Украины, как высказываниями
о своей безусловной поддержке в сфере международных отношений, так и поставкой
отдельных видов вооружения, наживаясь, при этом, на списанном военном
снаряжении. И, конечно, продолжая угрожать России санкциями.

Пригрозить России дополнительными санкциями, направить военных инструкторов и
поддержать Украину морально, а немного и материально – это пожалуйста. А вот
послать своих солдат воевать за надоедливую и непредсказуемую Украину – это
вряд ли.

Тут важно отметить, что все вышеперечисленные действия относятся к орудиям
гибридной войны. И она уже началась и вовсю ведётся. Хотя, по моему мнению,
перейти в стадию открытых боевых действий ей вряд ли суждено.

Впрочем, уроки истории говорят о том, что если условия для открытого
вооружённого конфликта созрели, то он может начаться внезапно, по воле
сложившихся обстоятельств, иногда случайных, а иногда специально
спровоцированных. Не зря же ходят слухи о появлении вблизи полосы разделения
натовского специального химического подразделения. Подобный сюжет мы уже не раз
наблюдали накануне обострения конфликтов в некоторых горячих точках планеты.

Так что, разжечь конфликт страны Североатлантического блока готовы, но
участвовать в нём не желают. В украинской же армии есть некоторая часть
подготовленных, агрессивно настроенных и бескомпромиссных националистических
батальонов, но значительная же часть призванных военнослужащих не готова
сложить голову, или идти на неоправданные жертвы за непонятные для них
ценности. Многие расценивают службу в армии, как просто способ выживания и
зарабатывания средств на существование, но не предполагают своего участия в
боевых действиях.

У нас тоже нет желания посылать под пули своих парней. Ради чего? Введение
российских войск на Украину только ещё больше настроит украинцев против России.
Они уже сейчас начинают готовить гражданское население к активному
сопротивлению. А украинцы имеют в этом вопросе огромный опыт. Поэтому, пока
есть возможность, Россия будет действовать бескровными методами. Сейчас пока их
вполне достаточно.

Но, даже в случае начала серьёзных военных действий, Россия применит
максимальные меры по сохранению гражданского населения Украины. У нас есть
большие возможности вести боевые действия «на удалёнке» точечно повредив пути
наступления (и отступления тоже) украинской армии, путей её снабжения, и
нарушения межвойсковых связей. Возможности нашей армии в этом плане многократно
превышают возможности Украины. В образовавшихся «котлах» моральных дух
украинской армии моментально погаснет, и они начнут сдаваться без боя вместе со
своим оружием. Тех же, кто пожелает стоять до «победного» конца, ожидает
незавидная участь.

Дальнейшее продолжение событий на Украине будет тесно связано с ответом Запада
на наши предложения по ограничению продвижения НАТО на восток и гарантиям нашей
безопасности.

В любом случае, лично мне противна сама мысль открытого вооружённого
столкновения между нашими странами, которые объединяет общая история и
многовековая дружба. Но и разместить вблизи наших границ стратегическое оружие
мы позволить тоже не можем.

ПОДПИСЫВАЙТЕСЬ - будем общаться в комментариях
