% vim: keymap=russian-jcukenwin
%%beginhead 
 
%%file 24_07_2021.fb.hvylja_oksana.1.uvolnenie_harkov_kiosk
%%parent 24_07_2021
 
%%url https://www.facebook.com/xunmeya/posts/3002741363335207
 
%%author Хвиля, Оксана
%%author_id hvylja_oksana
%%author_url 
 
%%tags harkov,mova,provokacia,ukrainizacia,uvolnenie
%%title АПД. Спілкувалася з власником. Він звільнив цю істоту в мене на очах
 
%%endhead 
 
\subsection{АПД. Спілкувалася з власником. Він звільнив цю істоту в мене на очах}
\label{sec:24_07_2021.fb.hvylja_oksana.1.uvolnenie_harkov_kiosk}
 
\Purl{https://www.facebook.com/xunmeya/posts/3002741363335207}
\ifcmt
 author_begin
   author_id hvylja_oksana
 author_end
\fi

АПД. Спілкувалася з власником. Він звільнив цю істоту в мене на очах. Справа в
тому, що вона хотіла мене обдурити, і дала мені телефон подружки, замість
телефона куратора. А обдурила власника, так що він теж жадає її крові. Все ОК,
він нормальний хлопець, зі Львова, залишився в Харкові після навчання. Коли
побачив відео, сам отетерів. Подробиці зараз розпишу. І привіт моїм хейтерам!

Ну, власне, я в історію вплуталася, тому вже негарно зіскочувати, і ми продовжимо.

Хотіла б написате "коротко в попередніх серіях...", але як раз там було
коротко, бо я була розлючена і принижена. Тому написала буквально пунктиром.
Давайте краще трохи розпишу, що саме відбувалося.

Я виступаю за розширення сфери української мови. Виступаю доволі дієво, але
практикую методи лагідної українізації. Наш ВуММ починався як практично
російськомовний науково-популярний проект, бо Харків же ж, але зараз ми робимо
контент виключно українською, і дякуємо нашим лекторам, які можуть і хочуть
поширювати науку українською.

В побуті теж не хочеться ходити за хлібом кожний раз, як у бій. Тому я вигадала
простий лайфгак, і користуся ним. Звертаюся українською,  замовляю товар,
розмовляю. А потім обов'язково доходить до фрази з боку продавчині "паіщітє
пять грівен пажалста!" На що я з незворушною посмішкою відповідаю: "Ні.
Дріб'язок шукаю лише для тих, хто обслуговує українською". Багато хто розуміє,
і "ой... пошукайте п'ять гривень! Дякую!" Але є такі, хто зтискає губи і
вишукує решту. В принципі, я не наїджаю, не лякаю, просто даю зрозуміти, що
повага з мого боку лише для тих, хто поважає мене. Для них я звісно напружуся,
і піду на зустрічні кроки.

Але те, що сталося в четвер у злощасному кіоску "Колосся. Долина смаку" мене
просто висадило. Продавчиня торохтіла російською "ой, у нас зрази
вкусниєєєєєє", але коли почула моє "ні, двадцять гривень лише для тих, хто
обслуговує українською", її порвало. Вона почала кричати, щоб я катилася до
Бандерштадту, що вона росіянка, приїхала з Росії, і що ніякої української тут
не буде. Вона відібрала назад пакунок з товаром і кинула мені сто гривень.
Сказати, що я прифігіла, це нічого не сказати.

Але, тому, хто живе у скляному будинку, не варто, як відомо, кидатися камінням.
Я повернулася сьогодні. З донькою, яка допомогла мені з відеофіксацією. І тепер
у мене на руках доказ її поведінки. Можете зацінити. І так, я готую офіційну
скаргу. І не тільки омбудсмену. Бо я хотіла залагодити конфлікт, але її
начальниця кинула слухавку зі словами "Разгаваріваєтє на украінскам? Да
свіданія!" Бо, власне, я стара лояльна покупчиня, яка скуповувалася тут багато
років, а це нова продавчиня, яка не встигла влаштуватися, як зчинила такий
скандал. А не мати касового апарату і бикувати, ну це реально тупо. Бо там не
тільки з мовою порушення.

Відео додаю. Власне, воно красномовне.

\ii{24_07_2021.fb.hvylja_oksana.1.uvolnenie_harkov_kiosk.cmt}
