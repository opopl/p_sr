% vim: keymap=russian-jcukenwin
%%beginhead 
 
%%file 25_12_2020.fb.bilchenko_evgenia.3.o_novoj_knige
%%parent 25_12_2020
 
%%url https://www.facebook.com/yevzhik/posts/3504152912953127
 
%%author Бильченко, Евгения
%%author_id bilchenko_evgenia
%%author_url 
 
%%tags bilchenko_evgenia,kniga
%%title БЖ. О новой книге
 
%%endhead 
 
\subsection{БЖ. О новой книге}
\label{sec:25_12_2020.fb.bilchenko_evgenia.3.o_novoj_knige}
\Purl{https://www.facebook.com/yevzhik/posts/3504152912953127}
\ifcmt
 author_begin
   author_id bilchenko_evgenia
 author_end
\fi

БЖ. О новой книге.

Кажется, она вышла в Таганроге на днях. Книга о Питере и его людях. Грандиозный
подарок моего друга Стефания Данилова. В основу названия положено любимое
стихотворение одного петербуржца, который мне безумно духовно близок и который
до сих пор привыкает к залихватской былинной ритмике моей тоники с трудом,
предпочитая четкий резец по мрамору классического метра.

\enquote{Беседка}. Книга вся написана классически: в нее не вошли  мятежные маршевые
стихи Питеру \enquote{до} и любовно-истерические стенания ему же \enquote{после}. Я могу
обозначить первый подростковый и третий женский периоды влюбленности в Город
конкретными лицами. Но в этой книге - только второй, четко выверенный строгим
метром и для меня переломный период. И за ним тоже стоят конкретные люди,
которых сквозь сумерки я высматривала по холодным набережным, дрожащими руками
набирая номер, и которые вернули мне Родину. Теперь это - так стойко, что
кажется: прошло сто лет. Я помню Питер все время: как в юности меня душил
камень и я упрекала Город, будто он - мой муж: так любящий упрекает предмет
мечты. Меня как скомороха непрерывно тянуло в разгильдяйское юродствование
златоглавых и кабацких Москвы и Киева. Так я отторгала Прагу. Сначала. Резалась
о готику. Но питерский ампир - это иное, оно - роднее было всегда, Моцарт с
Чайковским...

\ifcmt
  pic https://scontent-lga3-2.xx.fbcdn.net/v/t1.6435-9/132647703_3504152846286467_5036822931288535221_n.jpg?_nc_cat=104&ccb=1-3&_nc_sid=8bfeb9&_nc_ohc=m_dnXE2-wKkAX_26tby&_nc_ht=scontent-lga3-2.xx&oh=95283f53e6df1d91491acfa8c5e8228f&oe=60CA8BFB
	caption БЖ. О новой книге
\fi

А потом что-то бесповоротно и упоительно треснуло. Я даже помню где и при каких
обстоятельствах. Треснул мрамор, и я поняла, что камень - легок: настолько
легок, что из него со свистом просачивается воздух. А потом я сама стала
камнем. В этом - высшая суть беседки. Я говорю тебе вечное кровное
\enquote{спасибо} в Логосе, дорогая Стефания! И я говорю тебе
\enquote{спасибо}, брат и герой \enquote{Беседки}.  \enquote{Мы идем по Неве.
Мы почти подлатаны}. Я вас, правда, очень люблю. Я не смогу без вас.
