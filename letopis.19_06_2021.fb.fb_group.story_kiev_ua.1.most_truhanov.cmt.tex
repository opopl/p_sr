% vim: keymap=russian-jcukenwin
%%beginhead 
 
%%file 19_06_2021.fb.fb_group.story_kiev_ua.1.most_truhanov.cmt
%%parent 19_06_2021.fb.fb_group.story_kiev_ua.1.most_truhanov
 
%%url 
 
%%author_id 
%%date 
 
%%tags 
%%title 
 
%%endhead 
\subsubsection{Коментарі}

\begin{itemize} % {
\iusr{Андрей Надиевец}
Письма и рассказы написанные от сердца,не опаздывают никогда

\iusr{Ира Зеленина}
Петюня, спасибо за хорошие воспоминания! На Труханке прошла молодость! Встречи
и прогулки с друзьями и близкими! Как же было хорошо!

\begin{itemize} % {
\iusr{Петр Кузьменко}
Ира! Прямо таки, \enquote{молодость прошла}! Не прибедняйся! Ты, Ируся, сейчас
в самом соку. Как говорится, женщина на \enquote{Ять}!  @igg{fbicon.heart.sparkling} 

\iusr{Ира Зеленина}
\textbf{Петр Кузьменко} спасибо мой друг за комплеман !

\iusr{Ира Зеленина}
Когда будете в Киеве?
\end{itemize} % }

\iusr{Ира Зеленина}

\ifcmt
  ig https://i2.paste.pics/97c15226ff6d11e62c60c337de7d0b9c.png
  @width 0.2
\fi

\iusr{Gennady Henry Sergienko}
Очень тепло написано. Спасибо Вам.


\iusr{Tatiana Goncharova}

Спасибо за воспоминания! Как же, а про салют? На Пешеходном мосту мы всегда
смотрели салют и он был всегда самый красивый!

\begin{itemize} % {
\iusr{Петр Кузьменко}
\textbf{Tatiana Goncharova} несомненно!  @igg{fbicon.hands.shake} 

\iusr{Tatiana Goncharova}
\textbf{Петр Кузьменко} а вдоль набережной до моста самые красивые каштаны, помните? @igg{fbicon.heart.red}

\iusr{Петр Кузьменко}
\textbf{Tatiana Goncharova}, конечно! И про рыбалку там уже писал. Как всё было прекрасно в детстве и юности. Помним мы и наш Подол...

\iusr{Tatiana Goncharova}
\textbf{Петр Кузьменко} каштаны, 1977 год. Тогда наш Город был действительно зелёным:

\ifcmt
  ig https://scontent-frx5-1.xx.fbcdn.net/v/t39.30808-6/187835156_2714504198695767_4783256315477310135_n.jpg?_nc_cat=100&ccb=1-5&_nc_sid=dbeb18&_nc_ohc=inwibAiXxnkAX8_MQaT&_nc_ht=scontent-frx5-1.xx&oh=d43978e58068373ae826d9ef276fcc80&oe=61BD1766
  @width 0.4
\fi

\iusr{Раиса Карчевская}
Спасибо за прекрасные, добрые, теплые и трогательные воспоминания детства

\iusr{Ирина Архипович}
Просто прекрасно!! @igg{fbicon.heart.red}

\end{itemize} % }

\iusr{Світлана Проценко}

\ifcmt
  ig https://i2.paste.pics/799f011f678a46268b4fb024b87217dc.png
  @width 0.2
\fi

\iusr{Ирина Ещенко}
Петр, спасибо за приятные воспоминания детства !

\iusr{Олексій Вєровчук}
Душевно

\iusr{Елена Любкина}

\ifcmt
  ig https://i2.paste.pics/55b7cea60d03bc5e41a2a72d830f20d8.png
  @width 0.2
\fi

\iusr{Наталия Калатозишвили}
Душевные воспоминания спасибо

\iusr{Наталия Кебкал}
Спасибо за теплые воспоминания. Словно фильм посмотрела. Они Ваши личные, но близки многим...

\iusr{Konstantin Shamin}
Спасибо.

\iusr{Валентина Луконина}
Какие чудесные воспоминания! Этот мост, несомненно, в памяти любого киевлянина. Ну а для подолянина и подавно!
Как тепло и с какой любовью Вы, Пётр, вспоминаете своих родителей, обожаемую жену Нелли - и это не может не нравиться!
Спасибо Вам!

\iusr{Петр Кузьменко}
\textbf{Валентина Луконина} благодарю!  @igg{fbicon.hands.shake} 

\iusr{Оксана Писарчук}
Мы тоже сыном так ходили

\iusr{Ирина Иванченко}
Очень здОрово, респект!

\iusr{Валентина Андрусяк}
Благодарю за ваши чудесные воспоминания.

\iusr{Наталия Свирина}
І коли було багато людей міст трохи коливався...

\iusr{Людмила Аскерова}

С большим удовольствием прочитала ваши воспоминания и нахлынули воспоминания
детства с каким чувством шла я с мамой, а потом с подругами по мосту на пляж это
было настоящее путешествие сердце замирало от ощущения свободы.......

\iusr{Георгий Овсинский}

А у меня \enquote{тяжёлые} воспоминания... Пойдем с пацанами через мост, попросим
взрослых перевести, так как малых самих не пускали, возвращаемся домой и во
дворе всех встречают кого с ремнем, а кого и с резиновым шлангом. А меня не
встречали. Мама всегда на работе. Очень \enquote{завидовал}


\iusr{Natalia Volyanska}
Спасибо.

\iusr{Калерия Ходос}

Супер, спасибо людям которые делятся воспоминаниями о старом Подоле который был
прекрасен со своими маленькими домами и калорит неповторимого Подола. Мы жили
тоже на Подоле только в районе Житнего рынка на ул. Мирная это было просто
незабываемая жизнь.

\iusr{лилия шакалова}

А я ещё помню времена, когда не было этого моста. Мы жили на Подоле, и каждое
воскресенье точно также с молодой картошкой с маслом и укропом в банке и в
такой же банке - клубника в сахаре отправлялись по Спасской улице на Спасские
причал . Там мы покупали билеты и садились на маленький катер, который
переправлял нас на Подольский пляж. Все то же самое: потрясающе вкусная и
желанная еда, крики мамы, чтобы вылезла из воды, игры с песком... А вот вечером
нужно было отстоять длиннющую очередь на последний или предпоследний катер,
отправляющийся в Киев. Помню, на этом катере нужно было стоять в очень плотной
многолюдной толпе между взрослыми - ничего не было видно вокруг и помню мой
страх, что катер сейчас утонет. Вот так было ещё ДО появления Паркового моста


\iusr{Георгий Овсинский}
\textbf{лилия шакалова} Этот маленький катер имел гордое имя - ЛАПОТЬ!

\iusr{Алла Лукьянченко}
Пешеходный мост - это дорога в детство!

\iusr{Наталя Кудря}
\textbf{Alla Lukyanchenko} романтично і правдиво...

\iusr{Владислав Городецький}
\textbf{Петр Кузьменко}
Написано тепло, но вот только раки боком не ходят.

\begin{itemize} % {
\iusr{Петр Кузьменко}
\textbf{Владислав Городецький} мне маленькому казалось, что они передвигались именно так. @igg{fbicon.laugh.rolling.floor} 

\iusr{Валентина Горковенко-Спицына}
\textbf{Владислав Городецький} Почему не ходят? В случае опасности (и др.) раки передвигаются боком.
\end{itemize} % }


\end{itemize} % }
