% vim: keymap=russian-jcukenwin
%%beginhead 
 
%%file 18_01_2022.fb.ermolaev_andrej.1.noch_surka.cmt
%%parent 18_01_2022.fb.ermolaev_andrej.1.noch_surka
 
%%url 
 
%%author_id 
%%date 
 
%%tags 
%%title 
 
%%endhead 
\zzSecCmt

\begin{itemize} % {
\iusr{Viktor Shcherbina}

Думается, не менее важным является понимание того состояния, которое мы
определяем как бодрствование... Сон разума уже породил тех, кого мы наблюдаем.
Не спи - замерзнешь. Или, как учил Гераклит: «Не следует действовать и говорить
подобно спящим... Для бодр­ствующих существует один общий мир, а из спящих
каждый отворачивается в свой собственный... Не умеющие ни слушать, ни
говорить... Те, кто слышали, да не поняли, глухим подобны: \enquote{присутствуя,
отсутствуют}, — говорит о них пословица... Мудрым [Существом] можно считать
только одно: Ум, могущий править всей Вселенной». Мир, переживание которого
является общим для тех, кто бодрствует, — это мистическое единство,
единственность всех вещей, которые могут постигаться только разумом: \enquote{Должно
следовать общему... Здравый рассудок — у всех общий... Из всего — одно, из
одного — все...}

\iusr{Игорь Душин}
Бесконечно блящееся настоящее. Было такое название книжки  @igg{fbicon.smile} 

\begin{itemize} % {
\iusr{Андрій Єрмолаєв}
\textbf{Игорь Душин} в 1995-м вместо \enquote{б} было \enquote{д})

\iusr{Игорь Душин}
\textbf{Андрій Єрмолаєв} очепятка, но красиво, точно  @igg{fbicon.smile} 

\iusr{Игорь Душин}
\textbf{Андрій Єрмолаєв} Пойти, что ли, почитать её, наконец-то?

\iusr{Игорь Душин}
\textbf{Андрій Єрмолаєв} прошу простить меня , ворчливого. Всё будет хорошо, мы страна молодой демократии! Как заклинание повторяю сам себе  @igg{fbicon.smile} 

\iusr{Андрій Єрмолаєв}
\textbf{Игорь Душин} Да, Игорь, демократия у нас молодая, мы только слегка побелели, пока она алфавит учит)
\end{itemize} % }

\iusr{Виктор Мироненко}
\enquote{День Суркова}, Андрей. Так будет точнее. @igg{fbicon.face.tears.of.joy} 

\iusr{Геннадій Супіханов}
Хроники Нарнии... вечная зима...

\iusr{Dmitriy Raspopin}
Мы подобны сновидцу, который видит сон и сам живет во сне. Но кто этот сновидец?

\iusr{Олег Бондарь}

И ... доброе утро! Может нужно извлечь уроки - как в фильме от ужаса без конца?
Один, два, три... что мы делали не так. Не они - мы адекваты. Фактически нужна
глубокая рефлексия сродни психоанализу потому, что за формой мы потеряли -
суть. Вот и имеем результат - бесконечный сюр. Это тоже результат - но формы.

\iusr{Сергей Костенко}
Так может делать начнем???

\iusr{Александр Толстанов}
И о политике конечно же тоже... Как всегда спасибо!

\iusr{Снежана Егорова}

\textbf{Андрій Єрмолаєв}
Только это ночь сЮрка!!!
А он никогда не спит, поэтому и не посыпается....

\iusr{Igor Tenetko}

Один из моих любимых фильмов. Время от времени - к обязательному пересмотру.

Билл Мюррей бесподобен и говорит про Украину:

— А что бы ты делал, если бы застрял в одном месте, в одном дне и ничего бы
вокруг не менялось?

— А я примерно так и живу..


\iusr{Liliya Fomenko}

\enquote{День сурка} - комедия. А у нас сюрреалистическая трагедия. Или мы
все-таки выйдем из петли, или - нет, Скорее всего, нет.


\end{itemize} % }
