% vim: keymap=russian-jcukenwin
%%beginhead 
 
%%file 19_04_2023.stz.news.ua.donbas24.1.kyiv_vystavka_kult_spadchyna_mrpl
%%parent 19_04_2023
 
%%url https://donbas24.news/news/u-kijevi-vidkrilasya-vistavka-prisvyacena-kulturnii-spadshhini-mariupolya
 
%%author_id demidko_olga.mariupol,news.ua.donbas24
%%date 
 
%%tags 
%%title У Києві відкрилася виставка, присвячена культурній спадщині Маріуполя
 
%%endhead 
 
\subsection{У Києві відкрилася виставка, присвячена культурній спадщині Маріуполя}
\label{sec:19_04_2023.stz.news.ua.donbas24.1.kyiv_vystavka_kult_spadchyna_mrpl}
 
\Purl{https://donbas24.news/news/u-kijevi-vidkrilasya-vistavka-prisvyacena-kulturnii-spadshhini-mariupolya}
\ifcmt
 author_begin
   author_id demidko_olga.mariupol,news.ua.donbas24
 author_end
\fi

\ii{19_04_2023.stz.news.ua.donbas24.1.kyiv_vystavka_kult_spadchyna_mrpl.pic.front}
\begin{center}
  \em\color{blue}\bfseries\Large
У столиці відкрили виставку, присвячену пам'яткам архітектури та монументального мистецтва Маріуполя
\end{center}

\ii{19_04_2023.stz.news.ua.donbas24.1.kyiv_vystavka_kult_spadchyna_mrpl.pic.1}

18 квітня у \href{https://archive.org/details/18_04_2023.olga_demidko.donbas24.den_pamjatok_istorii_ta_kultury}{\emph{День пам'яток}}%
\footnote{День пам'яток історії та культури — традиції свята, Ольга Демідко, donbas24.news, 18.04.2023, \par%
\url{https://donbas24.news/news/den-pamyatok-istoriyi-i-kulturi-tradiciyi-svyata}, \par%
Internet Archive: \url{https://archive.org/details/18_04_2023.olga_demidko.donbas24.den_pamjatok_istorii_ta_kultury}%
} історії та культури в Києві відкрилася маріупольська
фотовиставка \enquote{Вписані в історію. Зруйновані війною}. Вона проходитиме у
Національному музеї історії України. Головна мета виставки — показати
зруйновану російськими окупантами культурну спадщину українського Маріуполя.

\textbf{Читайте також:} \href{https://donbas24.news/news/stalevar-veza-ta-more-u-lvovi-prezentuyut-pisanki-z-mariupolskimi-motivami-foto}{\emph{Сталевар, Вежа та море: у Львові презентують писанки з маріупольськими мотивами}}%
\footnote{Сталевар, Вежа та море: у Львові презентують писанки з маріупольськими мотивами, Наталія Сорокіна, donbas24.news, 17.04.2023, \par%
\url{https://donbas24.news/news/stalevar-veza-ta-more-u-lvovi-prezentuyut-pisanki-z-mariupolskimi-motivami-foto}%
}

Як відомо, до повномасштабного вторгнення рф саме в Маріуполі було зосереджено
одне з найбільших в Україні зібрань мозаїчної монументалістики. Станом на 24
лютого 2022 року в Маріуполі налічувалось: 206 історичних будівель, 6 з яких зі
статусом памятника архітектури; 106 памятників історії та монументального
мистецтва, 24 мозаїчниї полотна, 33 об'єкти археології. На фотовиставці були
представлені фото до повномасштабного вторгнення і після. Внаслідок авіаударів
і постійних обстрілів майже вщент знищені знакові будівлі центру міста, які
формували історичний ареал Маріуполя. Організаторами виставки виступили
Ма\hyp{}ріупольська міська рада, Національний музей історії України, проєкт \enquote{Я —
Маріуполь. Культура}, ККП mEHUB.

\begin{leftbar}
\emph{\enquote{Виставку почали готувати з 17 березня. Ідея належить саме Департаменту
культурно-громадського розвитку Маріупольської міської ради. Ми розуміли, що в
цей день — 18 квітня — ми маємо на національному рівні говорити про втрачені
історичні місця Маріуполя},} — зазначила Діана Трима.
\end{leftbar}

\ii{insert.read_also.elina_prokopchuk.donbas24.nava_povernulasja}
\ii{19_04_2023.stz.news.ua.donbas24.1.kyiv_vystavka_kult_spadchyna_mrpl.pic.2}
\ii{19_04_2023.stz.news.ua.donbas24.1.kyiv_vystavka_kult_spadchyna_mrpl.pic.3}

На виставці були присутні директор Національного музею історії України Федір
Андрощук та перший заступник міського голови Маріуполя Михайло Когут. На
відкриття завітав і Міністр культури та інформаційної політики України
Олександр Ткаченко, який уважно ознайомився з усіма світлинами. 

\begin{leftbar}
\emph{\enquote{Навіть зараз Маріуполь продовжує боротись за свою культуру, традиції та
історію. Місто, колись відоме своєю історичною мультикультурністю, зараз ми
згадуємо із великим болем у серці. Проте, такі фотовиставки несуть в собі
велику цінність, бо дають нам можливість запам'ятати місто таким, яким воно
було, аби в майбутньому його відродити}}, — наголосив Олександр Ткаченко. 
\end{leftbar}

\ii{19_04_2023.stz.news.ua.donbas24.1.kyiv_vystavka_kult_spadchyna_mrpl.pic.4}
\ii{insert.read_also.demidko.donbas24.mrpl_teatr_conception_vystava_ternopil}

Фото для виставки надали: муніципалітет Маріуполя, архіви Маріупольського
краєзнавчого музею, ККП mEHUB, маріупольські фотографи — Євген Сосновський та
Іван Станіславський. Фіксація руйнувань — з відкритих джерел.

\begin{leftbar}
\emph{\enquote{Світлини, представлені на виставці, зроблені протягом 2015−2021 років. Ці
світлини залишились, як важливі документи і наша пам'ять. Дуже боляче на все це
дивитись. Не хочеться порівнювати фото до і після}}, — підкреслив Євген
Сосновський.
\end{leftbar}

\ii{19_04_2023.stz.news.ua.donbas24.1.kyiv_vystavka_kult_spadchyna_mrpl.pic.5}

\ii{19_04_2023.stz.news.ua.donbas24.1.kyiv_vystavka_kult_spadchyna_mrpl.pic.6}
\ii{19_04_2023.stz.news.ua.donbas24.1.kyiv_vystavka_kult_spadchyna_mrpl.pic.7}
\ii{19_04_2023.stz.news.ua.donbas24.1.kyiv_vystavka_kult_spadchyna_mrpl.pic.8}

Виставка \enquote{Вписані в історію. Зруйновані війною} у Національному музеї історії
України триватиме до 1 травня 2023 року.

Раніше Донбас24 розповідав, що \href{https://donbas24.news/news/film-pro-mariupol-peremig-na-kinofestivali-u-ssa}{\emph{фільм про Маріуполь}}%
\footnote{Фільм про Маріуполь переміг на кінофестивалі у США, Еліна Прокопчук, donbas24.news, 07.04.2023, \par\url{https://donbas24.news/news/film-pro-mariupol-peremig-na-kinofestivali-u-ssa}} переміг на кінофестивалі у США.

Найсвіжіші новини та найактуальнішу інформацію про Донецьку й Луганську області
також читайте в нашому телеграм-каналі Донбас24.

ФОТО: з архіву Донбас24

\ii{insert.author.demidko_olga}
%\ii{19_04_2023.stz.news.ua.donbas24.1.kyiv_vystavka_kult_spadchyna_mrpl.txt}
