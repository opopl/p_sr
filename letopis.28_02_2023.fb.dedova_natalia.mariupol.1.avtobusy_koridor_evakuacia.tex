%%beginhead 
 
%%file 28_02_2023.fb.dedova_natalia.mariupol.1.avtobusy_koridor_evakuacia
%%parent 28_02_2023
 
%%url https://www.facebook.com/permalink.php?story_fbid=pfbid02Cgn86aCcNztXuqEbZZUs1ebN3Zo9fXRULqjqVXXvsHaZQUALXzCAwLo5NzyLFZqcl&id=100007662284921
 
%%author_id dedova_natalia.mariupol
%%date 28_02_2023
 
%%tags mariupol,mariupol.war,evakuacia,avtobus,05.03.2022,koridor.gumanitarnyj,katastrofa.gumanitarnaja
%%title Автобуси
 
%%endhead 

\subsection{Автобуси}
\label{sec:28_02_2023.fb.dedova_natalia.mariupol.1.avtobusy_koridor_evakuacia}

\Purl{https://www.facebook.com/permalink.php?story_fbid=pfbid02Cgn86aCcNztXuqEbZZUs1ebN3Zo9fXRULqjqVXXvsHaZQUALXzCAwLo5NzyLFZqcl&id=100007662284921}
\ifcmt
 author_begin
   author_id dedova_natalia.mariupol
 author_end
\fi

\#міймаріуполь 

\#автобуси

\#евакуація

\#склад

\#діти

Автобуси. 5-го березня, якби розпочалася евакуація, міський транспорт мав би
вивозити людей. Але автобуси - знищать. І вивозити не буде на чому. Про це
говорять Бойченко, Когут і Василий Клат 

Коридор. Його не буде. З 12:00 5 березня і до 13 - повна блокада і суцільне
вбивство мирного населення. Спроби виїхати - є. Проте на блокпостах нікого не
випускають. Адже на трасах - також вбивають. Маріуполь занурюється в пекло.
Маріупольці перетворюються на організми. Мета - вижити. 

Їжа. Запас продуктів влада зробить вже після 24 лютого.

1 березня. Обстріл складу в \enquote{Комунальнику}. Мінус тони їжі та води. Про це
говорять Бойченко, Когут і Ірина Коробка . 

А за кілька тижнів в місті розпочнеться гуманітарна катастрофа. 

Маріупольці їстимуть голубів, тухлу рибу, гнилі овочі, запліснявілий хліб.
Почнеться мародерство. 

Маріупольці питимуть воду з луж, батарей та бойлерів. Сотні загинуть на
криницях, від зневоднення та інфекцій. 

В березні загине водій \enquote{Міськводоканалу}. Він розвозитиме воду (прізвище?).
Тіло так і залишиться в машині. 

\enquote{ЯМаріуполь}. Обстріл складу, де знаходився теплий одяг для маріупольців до 14
років. Мінус куртки, мінус шапки. Влада пропонує компенсації грошима. 

За два дні до весни теплі дитячі набори та зимові з ковдрою та вітродуйчиком
отримають не всі маріупольці. Але пекельний 2022 та зиму 2023 - переживуть.
Звісно, хто як зможе. І, звісно, якщо пощастило залишитися в живих.

%\ii{28_02_2023.fb.dedova_natalia.mariupol.1.avtobusy_koridor_evakuacia.cmt}
