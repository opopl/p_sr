% vim: keymap=russian-jcukenwin
%%beginhead 
 
%%file 30_12_2021.fb.pekar_valerij.1.mova_jazyk_public.cmt
%%parent 30_12_2021.fb.pekar_valerij.1.mova_jazyk_public
 
%%url 
 
%%author_id 
%%date 
 
%%tags 
%%title 
 
%%endhead 
\zzSecCmt

\begin{itemize} % {
\iusr{Taras Shamayda}
Дякую!

\iusr{Anastasia Rozlucka}
Дякую!

\iusr{Максим Кобєлєв}
Підтримую!

\iusr{Іванна Кобєлєва}
Підтримую \#заТБукраїнською

\iusr{Наталка Федечко}
Підтримую!

\iusr{Сергій Оснач}
Підтримую!

\iusr{Юрій Зубко}
Повністю підтримую!

\iusr{Катруся Кравченко}

Повністю підтримуємо!!! За «Свати» я б посадила когось. І поки на ефірах будуть
різні падли на кшталт курви Бондаренко і решти.. ми самі цю війну програємо
дуже швидко.... а вони у 2022 буду реваншувати максимально активно!!! Нинішня
влада дала усі козирі в руки.. і мені вже здається- дає їх навмисно!! Аби
зімітувати державний переворот.

\iusr{Олег Слабоспицький}
Дякую!

\iusr{Тарас Марусик}
Дякую і підтримаю обома руками й обома ногами!

\iusr{Tetyana Rozumenko}
О, так, я дуже навіть згодна.

\iusr{Lyubov Vorobyova}
Підтримую!

\iusr{Олег Ровінець}
\textbf{Валерій Пекар}, 

дякуємо Вам за чітку проукраїнську громадянську позицію!

Команда ютуб каналу Хмаринка просить Вас про підтримку!

Ми створюємо освітні відео для молоді. Вся наша робота тримається на ентузіазмі
жмені відчайдухів, але ми робимо гарні відео ролики.

Ось приклад нашої роботи @igg{fbicon.right.arrow.curving.down}

\href{https://www.youtube.com/watch?v=HJOZ6u48Wh0}{%
Сонце - Зоря яка згасає, Хмаринка Science, youtube, 30.11.2021%
}

Сонце це серце нашої зоряної системи. Без нього життя на Землі було б не
можливим. Будова Сонця давно відома, як і його хімічний склад, та процеси, що
відбуваються всередині. У масштабах Всесвіту, Сонце це невелика зірка яка
тримає вісім планет на своїх орбітах, а також - це безперервне джерело світла
та тепла. І так триває вже мільярди років. Але вчені знають – так буде не
завжди, і вченим вже відомо коли Сонце згасне.

\ifcmt
  tab_begin cols=3,no_fig
     pic https://i2.paste.pics/154f4421338813aad0400b035447097c.png
		 pic https://i2.paste.pics/5d080d28080a86f1bbcf9c14a3fbd022.png
		 pic https://i2.paste.pics/6dc3eed0d04fbc071038c412ed7aa3a1.png
  tab_end
\fi

\ifcmt
  tab_begin cols=2,no_fig
	   pic https://i2.paste.pics/cd9f98ea0043367493d558ee432dbb6e.png
		 pic https://i2.paste.pics/e5a4bdf1a8c6e11932ee7c6062042ff3.png
  tab_end
\fi

\ifcmt
  tab_begin cols=3,no_fig
		 pic https://i2.paste.pics/0044589684b7d2d25d71d3f9035c752a.png
		 pic https://i2.paste.pics/f3b6e071589e62022e8c42bdf91ab026.png
		 pic https://i2.paste.pics/571235cfb10aa55652bc6f2b80dc56f9.png
  tab_end
\fi

Розкажіть про наш канал своїй аудиторії, від перегляду освітніх відео виграють всі!
Дякуємо!
Слава Україні!

\iusr{Ната Миха}
Так. Дякую. Вам

\iusr{Влад Тяжов}

\enquote{це плем'я, яке читає спільні книжки. Я би додав у наш час: ... і дивиться
спільні телепрограми}

Это дополнение уместно смотрелось бы в 80х. В наши дни \enquote{общие телепрограммы}
отмирают. Ютубом пользуется 20+ миллионов украинцев, включая уже и старшее
поколение (которое набивает миллионы просмотров видео вроде \enquote{ВАНГЕЛИЯ. ВСЕ
СЕРИИ. ЭТА ЖЕНЩИНА ЗНАЛА БУДУЩЕЕ. СЕРИАЛ ВЗОРВАЛ ИНТЕРНЕТ}).

В итоге получается шизофреничная картина. Телевидение со всё более строгими
ограничениями и цензурой - и свободный интернет, который в принципе невозможно
ограничить и который ориентируется на вкусы потребителей, а не законодателей
(аналогичная ситуация с радио и стримингами). И никого из фанатов квот такое в
принципе не смущает. А хотелось бы последовательности и цельной стратегии по
украинскому контенту, с опорой на реальность, а не на теоретические книги 80х.

\begin{itemize} % {
\iusr{Валерій Пекар}
Однак на виборах переміг герой телебачення, а не YouTube.

\iusr{Влад Тяжов}
\textbf{Валерій Пекар} 

А я и не про выборы, а про утопическую/устаревшую идею общих телепрограмм и
регулирования ТВ. В последние несколько лет и в ближайшие годы складывается
идеальный шторм, разрушающий традиционные каналы потребления контента:

1) все новые телевизоры - даже бюджетные - имеют предустановленный YouTube.

2) все новые автомобили оснащены Android Auto/Apple CarPlay.

3) скорость и проникновение интернета быстро растут.

Это сильно отличается от условного 2017, где интернет-видео смотрели в основном
с мониторов и ноутбуков (которые были далеко не у всех) и смартфонах (маленький
экран) либо через специальные приставки (неудобно, вариант не для всех), а ТВ
сохраняло монополию на телевизорах у большинства аудитории. Сейчас - включил
телевизор и смотри что хочешь. Реальная конкуренция только началась - и не
видно, за счёт чего ТВ/радио должно вывезти.

Отголоски этой конкуренции наблюдаю по своему окружению в последний год.
Крёстная, взахлёб рассказывающая как она купила новый телевизор и теперь
смотрит по ютубу сериалы и видео Жизнелюба Гарика Корогодского. Тесть,
смотрящий сериалы про военных, опять же, на ютубе с телевизора. Кум,
переключившийся с канала \enquote{Трофей} на ютуб-каналы аналогичной тематики.

Поэтому вопрос такой. Как вы предлагаете всё это регулировать?) Какую общую и
цельную стратегию, включающую в себя ТВ, ютуб, радио, стриминги вы видите?

P.S. Тем же квотам музыкальным 7 лет, но в стримингах и на улицах доминирует
российская (даже не русскоязычная, а именно российская) музыка, а новых
исполнителей, способных собрать Олимпийский, и близко не появилось.

\end{itemize} % }

\iusr{Владимир Мирошниченко}
Дякую, пане Валерію.

\iusr{Elena Grygoryeva}

рывок в будущее, это когда зритель сам выбирает на каком языке смотреть и
слушать. например Netflix - возможность выбрать язык и субтитры. Я за новости
на английском  @igg{fbicon.grin} . И выходить за паттерны одного языка.

\begin{itemize} % {
\iusr{Валерій Пекар}

Замечательная позиция, только не во время войны. Во время войны язык и культура
агрессора должны быть дискриминированы, если агрессор использует их как оружие.

\iusr{Elena Grygoryeva}
\textbf{Валерій} , при всём уважении к Вам.

Телевидение, новостные стримы только на украинском это путь в никуда. Украине
нужно выходить за рамки, один язык это всегда ограничения. И пока часть
человечества становиться более multicultural, Украина вводит ограничения в
коммуникации. Английский также язык бывшей империи, но сегодня это превосходный
инструмент коммуникации. На украинском Вы не поговорите с китайцами. Язык это
инструмент донесения смыслов и идентичности.

Но что сегодня важнее?

\iusr{Stas Klok}
\textbf{Elena Grygoryeva} 

російська мова - інструмент війни. Російськомовні українці - наслідок нашої
поразки на мовному фронті. Ваше не бажання це визнавати, лише підтверджує
важливість війни на мовному фронті. Якби мова у війні не була важлива, то цей
інструмент не використовував би ворог.

\iusr{Stas Klok}
\textbf{Elena Grygoryeva} 

я і з китайцями не спілкуватимусь російською. І з англійцями не спілкувався
англійською, якби була агресія з їх боку.

\end{itemize} % }

\iusr{Helevera Dmitry}
Не Ваше діло хто чим буде займатися в новорічну ніч і що дивитися.

\iusr{Ярослав Томин}
 @igg{fbicon.heart.blue}  @igg{fbicon.heart.yellow} 

\iusr{Oleg Bazylewicz}
Так наче ж переїхали держкіністи, на Кіото 27 вони тепер

\iusr{Виктор Кравцов}
 @igg{fbicon.100.percent}  @igg{fbicon.thumb.up.yellow}{repeat=3} 

\iusr{Евгений Борисенко}
І що конкретно пропонуєте робити у цій ситуації?

\iusr{Roman Romanow}
странно чото не звернув увагу раніше (( дуже слушно!!!

\end{itemize} % }
