% vim: keymap=russian-jcukenwin
%%beginhead 
 
%%file slova.nezavisimost
%%parent slova
 
%%url 
 
%%author 
%%author_id 
%%author_url 
 
%%tags 
%%title 
 
%%endhead 
\chapter{Независимость}
\label{sec:slova.nezavisimost}

Всі ці взаємні звинувачення, безперечно, є зручними пояснювальними схемами для
утримання електоральних баз лідерами відповідних політичних таборів, але вони
мають дуже мало спільного зі справжніми причинами чуттєвих та естетичних
провалів \emph{української незалежності}, з якими настав час розібратися саме
прихильникам українського проекту як головними відповідальними за все, що тут
відбувається,
\citTitle{Нелюбов та некраса української незалежності}, Анатолій Якименко, analytics.hvylya.net, 10.06.2021

%%%cit
%%%cit_head
%%%cit_pic
%%%cit_text
Очень знаковое решение накануне 30-летия \emph{независимости}.  Реальный позор на 30
летие \emph{независимости} когда судей, которые будут судить твоих граждан, отбирают
какие-то иностранцы. Какую ответственность они несут за результат своего
отбора? Такую же как понес отобранный иностранцами Коболев за 19 млрд убытков
Нефтегаза в 2020. То есть - никакую.  Ни он, ни эти эксперты здесь не будут
жить. Завтра Зеленский разрешит им отбирать полицию, потом министров и
президента. И эти люди поздравляли нас вчера с днем конституции и что-то мычали
про суверенитет, который сегодня дружно сдали(
%%%cit_comment
%%%cit_title
\citTitle{Отбор украинских судей иностранцами - это реальный позор! / Лента соцсетей / Страна}, 
Денис Иванеско, strana.ua, 29.06.2021
%%%endcit

%%%cit
%%%cit_head
%%%cit_pic
%%%cit_text
Отдельно порадовали попытки нардепов прикрыться рекомендациями Венецианской
комиссии. Когда речь шла, к примеру, о «языковом» законе, депутаты утверждали,
что рекомендации те обязательными не являются. И тут – на тебе! Куском
суверенитета готовы пожертвовать, чтоб тем «необязательным» угодить.  Чего ради
\emph{независимость} отдаём? На что меняем? На членство в ЕС? Так вход только по
билетам - посудомойками да сборщиками клубники. Мало нам ВАКС и НАБУ,
реализующих хотелки США и не прижавших ни одного серьёзного коррупционера.
Давайте им теперь судебную власть отдадим! Очнулись бы вы пока не поздно,
бабочки-однодневки!
%%%cit_comment
%%%cit_title
\citTitle{Чего ради независимость отдаем? На что меняем?  / Лента соцсетей / Страна}, 
Максим Могильницкий, strana.ua, 30.06.2021
%%%endcit

%%%cit
%%%cit_head
%%%cit_pic
%%%cit_text
Мінус 15 мільйонів населення за 30 років \emph{незалежності} – незлецький підсумок,
погодьтеся. Так, так, я знаю про всі застереження й причини, але часом добре
глянути на сухі цифри без пояснень, бо вони самі все пояснюють. Економічні
кризи, які в нас переходять одна в іншу, анексія території і частини населення,
війна, плачевний стан системи охорони здоров'я, низька тривалість життя і
низька народжуваність (частково через бідність, частково – як
загальноєвропейський тренд), виїзд на заробітки – та ви й самі про все це
знаєте.  Але не будемо тут про причини, я хотів би поговорити про наслідки, зокрема один
із них – візуальний. Мінус 15 мільйонів населення навіть на такій величезній
території не могли минути непоміченими. Вони залишили свій слід у кожному
куточку країни – поставили чорне клеймо запустіння
%%%cit_comment
%%%cit_title
\citTitle{Українці помирають, їдуть, зникають. Настане час, коли тут нікого не залишиться}, 
Андрій Любка, gazeta.ua, 31.07.2021
%%%endcit

%%%cit
%%%cit_head
%%%cit_pic
%%%cit_text
Я допускаю, що можуть бути серйозні акції проти України. Тим більше, що Росії
дуже важливо дискредитувати 30-ліття \emph{Незалежності} України. За логікою
російських ЗМІ, треба дискредитувати президента Зеленського, який не має
ніякого відношення до 30-ліття. Йому не так це буде боляче, це не його заслуги.
Це така російська логіка примітивна, цинічна.  Єдине, я не допускаю
широкомасштабного наступу по всьому кордону. Але пам'ятаємо слова Путіна
«організатори будь-яких провокацій, які загрожують корінним інтересам нашої
безпеки, пошкодують про скоєне так, як давно вже ні про що не шкодували». Це
неможливо перекласти на логіку нормальної людини зі здоровим глуздом»
%%%cit_comment
%%%cit_title
\citTitle{Про «диявола Росію», радикалізацію Лукашенка і чому Медведчук «не лох». Одне з останніх інтерв’ю Євгена Марчука}, 
Радіо Свобода, www.radiosvoboda.org, 05.08.2021
%%%endcit

%%%cit
%%%cit_head
%%%cit_pic
%%%cit_text
По сути говоря, у многих украинцев искусственно взращён психологический
комплекс \emph{фетишизма независимости}: никем и ничем не ограничиваемый суверенитет
(страны, президента, суда, личности и так далее) считается абсолютной и главной
ценностью общества. Дороже демократии, выше человечности и важнее здравого
смысла. Такое отношение, при помощи СМИ и других инструментов психологического
воздействия, эмоционально поддерживается в украинском обществе почти на
религиозном уровне. Любая попытка пересмотра или трансформации безудержно
продвигаемой парадигмы \emph{независимости} Украины характеризуется как предательство,
которое обязано наказываться жесточайшим образом.
Причём эта парадигма основывается на презумпции и сакрализации исторических
травм только некоторой части украинцев. Что породило новый господствующий тренд
современной национальной культуры – униженные и оскорблённые (особенно в
вопросе о языке). Теперь те, кто так жаждал оскорбиться, могут с уверенностью
считать себя оскорблённым независимо от того, был ли факт оскорбления. Хотя
социальная критика – это не дело вкуса или гражданская позиция, а моральная
обязанность интеллектуалов
%%%cit_comment
%%%cit_title
\citTitle{Замысел украинского государства: социальность, самодостаточность, независимость. Пятая часть}, 
Акулов-Муратов В. В., analytics.hvylya.net, 18.10.2021
%%%endcit
