% vim: keymap=russian-jcukenwin
%%beginhead 
 
%%file 09_01_2022.fb.fb_group.story_kiev_ua.1.pamjati_ok_antonova.cmt
%%parent 09_01_2022.fb.fb_group.story_kiev_ua.1.pamjati_ok_antonova
 
%%url 
 
%%author_id 
%%date 
 
%%tags 
%%title 
 
%%endhead 
\zzSecCmt

\begin{itemize} % {
\iusr{Сергій Криськов}

Одна из фотографий повторяется, а у меня не получается удалить дубль. Прошу
модераторов подредактировать.

\begin{itemize} % {
\iusr{Олег Коваль}
\textbf{Сергій}, модераторы не имеют такой возможности.

\iusr{Сергій Криськов}

Ладно, постараюсь справиться с этим самостоятельно.

\iusr{Люсянка Балашова}
\textbf{Сергій Криськов} а Вы не обращались к семье Ф. Ф... может они и отдали бы Вам плёнки

\iusr{Сергій Криськов}
\textbf{Люсянка Балашова} Но у меня нет их контактов
\end{itemize} % }

\iusr{Люсянка Балашова}
замечательная история... такая встреча навсегда запомнится !

\iusr{Люсянка Балашова}
В Киеве, в Соломенском р-не есть улица имени О. К. Антонова

\iusr{Надежда Владимир Федько}

+2

Чудова розповідь!

Мені теж довелося зустрітися з Олегом Антоновим в дещо несподіваних
обставинах...

Видавництво \enquote{Веселка} випускало книгу про О.К. Антонова... І в цій книзі
згадувалося про хобі конструктора - абстрактний живопис.

Мені доручили сфотографувати картини Антонова.

Завдання я виконав... про що напишу в КИ...

\iusr{Ковальская Татьяна}

А мені розповідав лікарь, професор Говалло, в кого лікувався Антонов, який він
був чудовою людиною!

\iusr{Юрий Гребенкин}

Более подробных фотографий летательного аппарата не сохранилось?

\begin{itemize} % {
\iusr{Сергій Криськов}
\textbf{Юрий Гребенкин} К сожалению. Много негативов осталось у кинооператора Забегайло Ф.Ф. Есть один снимок очень низкого качества и снятый с неудачного ракурса, поэтому малоинформативный. Нет смысла его выкладывать.
\end{itemize} % }

\iusr{Сергей Михайловский}
Он ещё книжки писал, одна так и называется \enquote{Десять раз сначала}.

\begin{itemize} % {
\iusr{Сергій Криськов}

Об этом и сказано в этой моей публикации, также в подписи к соответствующему
фотоснимку.
\end{itemize} % }

\iusr{Margarita Kaminsky}

В очень далёкой молодости около года работала в фотолаборатории и пела в
оркестре завода Антонова.

\end{itemize} % }
