% vim: keymap=russian-jcukenwin
%%beginhead 
 
%%file 11_01_2022.fb.fb_group.story_kiev_ua.3.zima_rybalka
%%parent 11_01_2022
 
%%url https://www.facebook.com/groups/story.kiev.ua/posts/1838241859705940
 
%%author_id fb_group.story_kiev_ua,dubinina_oksana
%%date 
 
%%tags dnepr,kiev,kievljane,rybalka,zima
%%title Зимняя рыбалка киевлян
 
%%endhead 
 
\subsection{Зимняя рыбалка киевлян}
\label{sec:11_01_2022.fb.fb_group.story_kiev_ua.3.zima_rybalka}
 
\Purl{https://www.facebook.com/groups/story.kiev.ua/posts/1838241859705940}
\ifcmt
 author_begin
   author_id fb_group.story_kiev_ua,dubinina_oksana
 author_end
\fi

Зимняя рыбалка киевлян.

Пришла настоящая зима со снегом и морозами и навеяла детские воспоминания о
папином увлечении зимней рыбалкой. Сейчас папе 80 лет, на зимнюю рыбалку не
ходит лет 10-12, но с удовольствием о ней рассказывает.

\ii{11_01_2022.fb.fb_group.story_kiev_ua.3.zima_rybalka.pic.1}

Хотя трескучие морозы прочно и глубоко сковывали водную гладь водоёмов,
настоящих рыбаков никогда это не останавливало. А даже наоборот, дождавшись
самых сильных морозов, дающих крепкий лёд, киевляне - настоящие мужчины
(потомки мужественных викингов, сарматов, скифов и казаков и т.д.) шли в пургу
и лютый холод на зимнюю рыбалку. 

Надо признаться, меня всегда удивляло, как можно провести на льду в -15°/-25°
мороза почти целый день, да еще и в тяжелой неудобной одёжке. Обледенелыми
руками нанизывать на манюсенькую мармышку-крючок мотыля - наживку для рыбы (это
такие красные маленькие червячки - личинки комаров), а потом часами сидеть у
лунки и выжидать клёва. 

А чего стоили зимние доспехи - металлический ящик, увесистый бур для льда,
пешня для разбивания лунки, ну и всякие специальные приспособления и снасти!

Металлический ящик сам по себе был тяжелым, но одновременно служил и
стульчиком, и чемоданчиком. Обычно верх рыбацкого ящика был с
сидушкой-подушкой, обтянутой кожей, кожзамом или тканью, а еще я видела
деревянные сиденья. Часто ящики были самодельные, обязательно с крепким ремнем
наискосок для ношения через плечо, иногда к днищу приделывались маленькие
саночки и тогда за ремень можно было тянуть ящик по льду. Особенно, когда ящик
был наполнен пойманной рыбой. 

\ii{11_01_2022.fb.fb_group.story_kiev_ua.3.zima_rybalka.pic.2_3}

Приготовления к походу основательные. Целый ритуал. Обмундирование тёплое,
инвентарь и всё нужное с собой. С друзьями обсуждение и споры куда ехать в этот
раз. Бутерброды, в основном чёрный хлеб с салом и чесноком, в термосе горячий
чай и, почти всегда, «чекушка» водочки – для «сугреву»)).  

Папа готовил для подкормки рыб пшеничную кашу и заправлял её растертыми в муку
жареными семечками. Аромат был очень приятный и моя младшая сестра часто
просила ей отсыпать немного «рыбной» каши))) 

Для подкормки рыбок использовался «самосвал» - такая сеточка со свинцовым
грузком, в которую насыпался прикорм, его опускали в лунку, почти на дно, и
кашу течением  вымывало, что привлекало рыб.

На рыбацком рынке Бухара у станции метро Днепр папа покупал мотыля, он должен
был быть живой и подвижный, только такой мог обеспечить хороший клёв. Хранился
он в специальной пенопластовой коробочке в чистой влажной тряпочке в укромном
месте холодильника. Если мотыль нравился, папа говорил, причмокивая языком: ах,
какой жирненький! Прям на хлеб намазывать! - мы с сестрой со смехом шарахались
от того мотыля!)) 

* Ещё была такая шутка о мотыле: чтобы проверить мальчик это или девочка, нужно
протянуть через передние зубы червячка. Если застрянет - значит мальчик, а
пролетит через щелочку - значит девочка))) 

Конечно же, бывали дни, когда рыбак, промерзнув весь день на льду, уходил с
рыбалки без улова, но бывал и такой удачного клёва, что  унести  всё было
невозможно и тогда уловом делились с другими, менее удачными рыбаками. 

Рыбы иногда действительно было много, подтверждаю. Частенько приходилось
чистить колючих окуней и ершей, разве только плотву легче было чистить. Но
самая вкусная уха была из окуней. Реже ловились судаки, чехонь, смотря какая
снасть была.

Из правдивых папиных рыбацких побасенок:

* начался клёв, подсекаю, вытягиваю из лунки почти килограммового окуня, хвост
которого проглотил окунь грамм 600, а в хвост второго вцепился зубками совсем
небольшой, грамм 200 окунёк. Вот хотите верьте, хотите нет! Сам не поверил
своим глазам!!

* клюнуло – тяну, а голова рыбы в лунке застряла! Рот окунь открыл, рванулся и
– леска оборвалась , а рыба исчезла в глубине! Вот такой рот! – и показывает,
соединяя большим и указательными пальцами. Чуть удочку не утянула!

* сидим, ловим рыбу. Вдруг, как подул ураганный ветер, несколько шапок по
ледяной поверхности так и покотились, как футбольные мячи!)

* пошёл я как-то на «броню». –Что за «броня»? – Это Брандвахта – пост, который
закрывал вход в шлюзовой канал возле ГЭС, там зимой очень даже много рыбы
водилось. Сижу, поесть захотелось, достал бутерброд с салом, откусил и тут клёв
начался, отложил еду на краю ящика. А тут ворона, так боком-боком, хвать
хлебушек с сальцом и отлетела на безопасное расстояние и быстренько ест. Те же
вороны выдергивали «хвосты» из улова, сложенного горкой на  льду.

* один раз на реке Сула на трещине во льду был такой клёв, что папе пришлось
снять верхнюю одежду, снять майку, снова быстро одеться, майку снизу завязать
узлом и использовать её, как сумку для рыбы. (Глядя в папины честные голубые
глаза, невозможно ж не поверить)). 

* папа зовёт нас с сестрой: «Я вам гостинець від зайчика приніс!» Мы маленькие
и всему ж верим. С довольным и хитрым лицом достаёт из рыбацкого ящика коробку
для бутербродов и аккуратненько по маленькому кусочку отрезает оставшийся
черный хлебушек с салом – с мороза  этот кусочек такой вкусный, ах, какой
хороший зайчик!)) Конечно же потом, повзрослев, мы смеялись над этим. 

* Как-то на зимней рыбалке рыбаки заметили, что мужчина подползает к берегу
по-пластунски и продолжает ползти по суше. Его окликнули, оказалось мужчина
провалился в трещину в холодную воду и еле выкарабкался на лёд. Долго полз,
никого рядом не было, вот и перенервничал. Ему помогли добраться до автобуса -
благо тот был рядом - и отогрели чаем и водкой.

* Анекдот: Рыбаки перед зимней рыбалкой совещаются: - Сколько будем брать
водки? В позапрошлом году мы брали по одной бутылке на брата - потеряли удочки,
в прошлом году брали по две - потеряли автобус. Что предлагаете? 

Выходит один рыболов: - Предлагаю в этом году взять по три, но удочки не брать
и из автобуса не выходить!

Увидеть стайки рыбаков  на льду мы можем и сейчас, проезжая через киевские
мосты, особенно в заливах Гидропарка. Еще из самых популярных мест рыбалки для
киевлян можно назвать Киевское море, реки Десну и Днепр, заливы Днепра,  в том
числе в Конча Заспе, река Припять, обводной канал возле ГЭС, Кременчуцкое  и
Каневское водохранилища, река Тетерев, Ирпень и Сула. А также различные
«рыбные» места у сел Пронозовка, Фрузиновка, Страхолесье, Осещина, Пуховка,
Леточки т.д., список можно продолжать). 

К некоторым местам лова можно было добраться, например, на метро, а к некоторым
возили специальные автобусы. На Лениградскую площадь самым ранним утром,
задолго до восхода солнца, сходились рыбаки-любители, чтобы несколько часов
ехать к какому-нибудь дальнему водоёму и всё это ради удачного лова и желания
побыть на природе и в мужском коллективе.

Современным рыбакам, охотникам и спортсменам на выручку пришло различное
термобельё и легусенькая одежда, которая не стесняет движения, а ведь еще 30-40
лет назад зимний рыбак должен был надеть столько тёплой одежды, что его походка
становилась почти как у пингвина, а еще добавьте неудобные валенки и частенько
калоши на них, а в довершение образа - шапка ушанка. 

Возвращавшиеся со льда рыбака, слегка разогретого  спиртным, можно было издали
заметить по распахнутому кожуху, у горла свитер ручной вязки с высокой
горловиной, а шапка ушанка чуть набекрень уже с развязанными ушками))

***Огромное спасибо Сергею Пятерикову за предоставленное им отличное фото из
собственного драгоценного архива киевских фотографий. 

Фото Сергея Пятерикова 6.02.2005г

\ii{11_01_2022.fb.fb_group.story_kiev_ua.3.zima_rybalka.cmt}
