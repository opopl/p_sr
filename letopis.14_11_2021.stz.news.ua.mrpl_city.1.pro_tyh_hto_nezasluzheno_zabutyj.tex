% vim: keymap=russian-jcukenwin
%%beginhead 
 
%%file 14_11_2021.stz.news.ua.mrpl_city.1.pro_tyh_hto_nezasluzheno_zabutyj
%%parent 14_11_2021
 
%%url https://mrpl.city/blogs/view/pro-tih-hto-nezasluzheno-zabutij
 
%%author_id demidko_olga.mariupol,news.ua.mrpl_city
%%date 
 
%%tags 
%%title Про тих, хто незаслужено забутий...
 
%%endhead 
 
\subsection{Про тих, хто незаслужено забутий...}
\label{sec:14_11_2021.stz.news.ua.mrpl_city.1.pro_tyh_hto_nezasluzheno_zabutyj}
 
\Purl{https://mrpl.city/blogs/view/pro-tih-hto-nezasluzheno-zabutij}
\ifcmt
 author_begin
   author_id demidko_olga.mariupol,news.ua.mrpl_city
 author_end
\fi

З цього матеріалу я хочу розпочати нову серію блогів, присвячених тим містянам,
чиї могили перебувають в занедбаному стані, проте за життя ці люди
користувалися неабиякою популярністю. З різних причин вони були поховані саме в
Маріуполі, на найстарішому цвинтарі міста, але, оскільки не було родичів,
залишилися одні на кладовищі, яке сьогодні поступово перетворюється на музей
просто неба. Сподіваюся, що розповідаючи про цих видатних діячів, хтось з
маріупольців вирішить відвідати своїх знаменитих земляків...

Почати хочу з театральних діячів міста, адже тема театру наразі залишається для
мене головною і в науковій, і в громадській діяльності.

\ii{14_11_2021.stz.news.ua.mrpl_city.1.pro_tyh_hto_nezasluzheno_zabutyj.pic.1}

Один з найстаріших пам'ятників, який зберігся на цвинтарі і перебуває в дуже
занедбаному стані, належить брату засновника першої професіональної театральної
трупи Маріуполя Василя Леонтійовича Шаповалова  \emph{\textbf{Іллі Васильовичу
Кечеджи-Шапова\hyp{}лову}}. Про нього збереглося набагато менше відомостей, ніж про
Василя Шаповалова. Немає і світлин. На жаль, і на пам'ятнику фото вилучено. Але
завдяки тодішній пресі вдалося зібрати деяку інформацію. Зокрема, рецензент
одного з провідних видань театральної періодики останньої третини ХІХ століття
Російської імперії \enquote{Артист} зазначав, що \enquote{Василь та Ілля Шаповалови – актори з
великим знанням сцени} і взагалі \enquote{трупа підібрана цілком добре, шкода тільки,
що внаслідок важкої економічної кризи театральні справи йдуть у Маріуполі
незадовільно}.

Також відомо, що брат засновника маріупольського театру – Ілля
Кечеджи-Шаповалов, акторське обдарування якого відзначала столична преса, –
організував у Маріуполі \emph{\enquote{Товариство драматичних артистів}} (1893), у складі
якого був і Василій Леонтійович. На сторінках журналу \enquote{Артист} знаходимо
наступну інформацію: \enquote{Спектаклі драматичного Товариства, що проходять у
Концертній залі В. Шаповалова, почалися 22 жовтня постановкою \enquote{Столичного
повітря}. У виставі бере участь у якості гастролерки пані Мазуровська}. Але це
була не перша спроба Івана Леонтійовича стати організатором театральної справи
в Маріуполі. За рік до цього викладач давніх мов Олександрійської чоловічої
гімназії \emph{О. Петрашевський} писав: \enquote{Остання антреприза Іллі Шаповалова при
непоганому складі трупи не мала успіху}. Наразі більше ніяких відомостей про
Іллю Леонтійовича не вдалося знайти. Але точно відомо, що він був талановитим
актором та вірним сподвижником театральної справи.

\ii{14_11_2021.stz.news.ua.mrpl_city.1.pro_tyh_hto_nezasluzheno_zabutyj.pic.2}

На старому цвинтарі можна побачити могили і інших театралів.  Зокрема, тут
поховане талановите театральне подружжя – режисер-постановник, заслужений
артист УРСР \emph{\textbf{Семен Андрійович Шейко}} та народна артистка УРСР \emph{\textbf{Роза Іванівна
Под'якова}}. Насправді приїзд до Маріуполя з Дніпра таких яскравих творчих
індивідуальностей з великим театральним досвідом дуже позитивно вплинув на
театральне життя Північного Приазов'я. На маріупольській сцені вони працювали
порівняно недовго: Семен Андрійович – чотири роки, Роза Іванівна – на рік
більше. Їхні могили розташовані поруч, але одразу ж помітно, що їх дуже давно
ніхто не відвідував. Цікава деталь на табличці бачимо: Семен Шейко, але за
життя режисера всі називали Михайлом. На жаль, ніхто з очевидців тих подій не
знає, чому режисер мав два імені. Народна артистка України \emph{Світлана Отченашенко}
припускає, що це було свідомим вибором режисера. Вона знала братів Семена
Андрійовича, які мешкали в Полтаві та Києві. Вони його називали і Михайлом, і
Семеном. Проте, коли почав працювати в маріупольському драматичному театрі
головним режисером, одразу представився як Михайло Андрійович Шейко, тому про
друге ім'я більшість акторів не знала. Як згадувала актриса \emph{Марія Алютова},
Михайло Шейко  дуже швидко порозумівся з трупою. Його першою роботою була
постановка \enquote{День народження Терези}. Зважаючи на те, що драматургічно п'єса
була слабенькою, новий режисер увів до вистави кубинські мелодії, кубинські
танці, і постановка вийшла яскравою та незабутньою, наповнилася кубинським
ментальним колоритом.

\ii{14_11_2021.stz.news.ua.mrpl_city.1.pro_tyh_hto_nezasluzheno_zabutyj.pic.3}

Багато позитивних відгуків про режисерську діяльність Семена Шейка я знайшла і
в пресі. До речі в усій маріупольській періодиці його називають саме Михайлом.
Зокрема, рецензент А. Козловський, характеризуючи виставу \enquote{Орфей спускається в
пекло}, зазначив, що весь виконавський склад вистави порадував глядача
злагодженою грою, переконливою життєвістю створених образів. Протягом усього
дійства відчувався індивідуальний творчий почерк режисера-постановника Михайла
Шейка.

\ii{14_11_2021.stz.news.ua.mrpl_city.1.pro_tyh_hto_nezasluzheno_zabutyj.pic.4}

Критики неодноразово відзначали талановиту гру і  народної артистки УРСР,
дружини Михайла Шейка Рози Под'якової. Так той же А. Козловський зазначає, що
вона 
\begin{quote}
\enquote{тонка і глибока майстриня, яка поєднує психологічну глибину з бездоганною
сценічною технікою. Маючи гарні зовнішні дані, актриса у виставі \enquote{Орфей
спускається в пекло} жестами, паузами, інтонаціями, рухами підкреслила
внутрішню красу Леді Торренс. Вдячні маріупольці захоплено сприйняли новий
творчий доробок, нагороджуючи акторів гучними оплесками і словами вдячності}.
\end{quote}

\ii{14_11_2021.stz.news.ua.mrpl_city.1.pro_tyh_hto_nezasluzheno_zabutyj.pic.5}

Також у пресі зазначалося, що \enquote{головний режисер маріупольського театру М. Шейко
знайшов свій оригінальний ключ для потрактування низки образів, багато сцен у
його виставах зіграні з вражаючою силою, схвильовано, емоційно...}.

\ii{14_11_2021.stz.news.ua.mrpl_city.1.pro_tyh_hto_nezasluzheno_zabutyj.pic.6}

Позитивно оцінили театрали і відому виставу \enquote{Собака на сіні}, поставлену
Михайлом Шейком. За словами критиків, головний режисер театру М. Шейко не
тільки передав зовнішній колорит епохи, правдиво зобразив героїв п'єси, але й
розкрив їхній внутрішній світ. Заслуга його в тому, що він, відтіняючи життєву
природність конфліктів, потрактовує їх у підкреслено комедійному, сатиричному,
а часом навіть у гротесковому ракурсі. На високому художньо-психологічному
рівні зіграла головну роль у цій виставі Роза Под'якова. Будучи обдарованою
акторкою досить широкого діапазону, для кожної ролі вона знаходила відповідну
тональність, точні жести, правдиві внутрішні і зовнішні інтонації.

\ii{14_11_2021.stz.news.ua.mrpl_city.1.pro_tyh_hto_nezasluzheno_zabutyj.pic.7}

Світлана Отченашенко зазначала, що 
\begin{quote}
\enquote{за порівняно короткий період роботи в
маріупольському театрі Михайло Шейко здійснив досить багато вистав (\enquote{Собака на
сіні}, \enquote{Гліб Космачов},\enquote{Шануй батька свого}, \enquote{Орфей спускається в пекло},
\enquote{Божественна комедія}, \enquote{Будинок божевільних}). Всі вони позначалися
талановитими постановками і новаторством. Але вершиною маріупольського періоду
творчості режисера була \enquote{Оптимістична трагедія} Всеволода Вишневського}.
\end{quote}

\ii{14_11_2021.stz.news.ua.mrpl_city.1.pro_tyh_hto_nezasluzheno_zabutyj.pic.8}

Розі Под'яковій завжди хотілося грати сучасниць, створювати яскраві образи,
сильні, вольові характери. Для неї кожна нова робота була проявом досвіду,
кристалізацією виграшних штрихів і інтонацій. \enquote{Це була дуже яскрава артистка,
неймовірно цікава. Напрочуд красива жінка, розумна, талановита}, – згадує С.
Отченашенко. Кореспондент \enquote{Приазовського робочого} С. Гольдберг підкреслив, що
особливо важливою для акторки стала роль Комісара у п'єсі В. Вишневського
\enquote{Оптимістична трагедія}: 

\begin{quote}
\enquote{Вона виходила на сцену зібрана, одухотворена; жила
цією роллю, цим образом, зливаючись з ним...}.
\end{quote}

Розу Под'якову ніяк не можна назвати актрисою однієї теми, одного амплуа.
Тільки на маріупольській сцені вона створила цілу галерею різноманітних
образів, часом зовсім несхожих один на одного. Але всіх її героїнь ріднить одна
спільна риса – визначеність характеру, пристрасність, незалежність, глибина
думки. І кожна зустріч з нею завжди викликала у маріупольців радість, роздуми,
переживання. У серпні 1968 року сталася дуже знакова для театральної культури
міста подія. За видатні заслуги в розвитку радянського театрального мистецтва
на той час заслуженій артистці УРСР Розі Іванівні Под'яковій Указом Президії
Верховної Ради Української РСР було присвоєне почесне звання народної артистки
республіки. Того ж року, на превеликий жаль, раптово помер Михайло Андрійович
Шейко. Йому було всього 55 років. А через рік у віці 44 років померла і Роза
Іванівна.

Це видатне подружжя поховане на найстарішому цвинтарі Ма\hyp{}ріуполя, яке сьогодні є
занедбаним і потребує упорядкування. І про Семена Шейка, і про Розу Под'якову
збереглося багато відомостей і спогадів, є фото. Вони були дійсно талановитими
і непересічними особистостями, які, на жаль, дуже рано пішли з життя.

Я особисто нещодавно відвідала могили Іллі Кечеджи-Шапова\hyp{}лова і талановитого
театрального подружжя. Сподіваюся, що попри те, що в місті не залишилося їхніх
родичів, обов'язково знайдуться ті маріупольці, які захочуть відвідати їхні
могили та вшанувати пам'ять видатних театральних діячів.
