% vim: keymap=russian-jcukenwin
%%beginhead 
 
%%file 29_01_2022.fb.fb_group.story_kiev_ua.3.shvejna_mashynka
%%parent 29_01_2022
 
%%url https://www.facebook.com/groups/story.kiev.ua/posts/1850277608502365
 
%%author_id fb_group.story_kiev_ua,grabar_sergij.kiev
%%date 
 
%%tags kiev
%%title Швейна машинка
 
%%endhead 
 
\subsection{Швейна машинка}
\label{sec:29_01_2022.fb.fb_group.story_kiev_ua.3.shvejna_mashynka}
 
\Purl{https://www.facebook.com/groups/story.kiev.ua/posts/1850277608502365}
\ifcmt
 author_begin
   author_id fb_group.story_kiev_ua,grabar_sergij.kiev
 author_end
\fi

Швейна машинка.

Коли мене привезли, після довгого шляху спочатку морем, а потім дорогами,
важкими та вибоїстими, і, знявши з вантажної хури, занесли, то відчула легке
розчарування. Я вважала, що заслуговую на краще, маючи непросте, звучне ім’я і
свій особистий номер. Але доля розпорядилася інакше і я потрапила до цієї
родини, десь на околиці губернського міста Києва.

\ii{29_01_2022.fb.fb_group.story_kiev_ua.3.shvejna_mashynka.pic.1}

Хоча околицею там, де я отримала проживання, назвати було важко. Місцевість ця
називалася Нова забудова і з’явилася порівняно недавно на місці селища Ямки.
Подейкували, що після розширення Печерської фортеці, частину людей з Печерська,
не питаючи їхньої згоди, наказово пересилили сюди, тим самим розширивши місто.

Будиночок, у якому я опинилася, був привітним, теплим, добре впорядкованим, а
та увага, що йшла від моїх господарів назавжди перекреслила перші розчарування.
Я відразу відчула себе членом родини, яка, судячи з їхніх розмов, декілька
років економила, збирала кошти, щоби  зрештою отримати подарунок у вигляді
чавунної станини, на якій містився швейний механізм з ножним приводом.

Вулиця, на якій стояв будиночок, називалася Ямською була веселою, з постійним
святом, що не припинялося ні вдень, ні вночі. Поруч із нами шелестіла невеличка
річка-струмочок з прізвиськом Ямка, яка метрів за тридцять зливалася з більшою
на ім’я Либідь.

Люди, які вперше заходили до господи, побачивши мене, завжди повторювали одну й
ту ж фразу: «О, у вас «Зінгер!» «Так, - відповідали з гордістю хазяї, і, трохи
помовчавши, додавали, - посаг донечки нашої». Донька була маленькою, веселою
дівчинкою, і постійно намагалася трохи покачатися на моєму металевому ножному
приводі. За що отримувала від батьків маленьких прочуханів.

Проходили роки, дівчинка підросла, вступила до Інституту шляхетних панянок,
вчилася добре і була гордістю батьків. А потім прийшла війна, а за нею
революція, чи як там у них воно все це називалося, і навколо стало моторошно і
небезпечно, бо кожної хвилини могло статися непоправне. Так вже бувало у
сусідніх будинках: стукіт у двері, на порозі озброєні люди і далі все, що
цінне, а я була великою цінністю, забиралося і зникало в нетрях великого,
ворожого міста. А люди? З ними бувало по-різному: когось полишали, когось
вбивали, когось забирали і вони вже ніколи не поверталися.

Так було й у нас, однієї зимової ночі грізно постукали, але той молодий
чоловік, що прийшов на чолі невеликого загону, якось дивно подивився на молоду
красиву дівчину, яка виросла з малої дівчинки. Він і сам був красенем:
стрункий, статний, з пишним волоссям і сміливим поглядом. Щось вмить промайнуло
між ними - двома молодими людьми. І молодик скомандував загону вийти з господи,
вибачився і пішов. Він прийшов наступного дня сам і ще наступного, і так до
того моменту, доки вони не одружилися, а я, як посаг, переїхала до нового
помешкання на Галицькому базарі.

Галицький базар в народі називався Єврейським базаром, а скорочено Євбаз. Місце
було гомінке, метушливе – тут можна було купити все що завгодно, і водночас
наразитися на будь-які неприємності. Ми оселилися на другому поверсі цегляного
будинку, що мав вихід безпосередньо до самого ринку, і все що відбувалося
ззовні, мало наслідки і всередині житла. Тож мало кому з мешканців будинку
вдалося не потрапити до сумнозвісного «Діда Лук’яна», тобто не відсидіти строк.

Наша родина, а я кажу «наша», бо стала частиною цієї сім’ї, якось зуміла
оминути всі ці перешкоди. Трійко діточок: два хлопчики і дівчинка, народилося у
цієї надзвичайно красивої пари. І всі вони, як колись це робила їхня мати,
намагалися покачатися на моєму ножному металевому приводі і, звичайно,
отримували прочуханів.

Пройшла ще одна війна, вбивча і кривава, але й вона закінчилася. Ринок закрили.
На його місці збудували «Цирк». Пішли до зносу навколишні будинки. Молодий
фронтовик мав все таке ж пишне, але цілковито сиве волосся. З фронту повернувся
з пораненням і трохи накульгував на праву ногу. Його кохана дружина роздобріла,
зберігаючи свою невимовну красу. Сини розлетілися по світу, а донечка
продовжувала жити з батьками, була красива тою вибуховою, чаруючою красою, що
зводить чоловіків з розуму.

Ми переїхали до нового масиву, що отримав назву Першотравневого (в народі
Чоколівки), отримавши досить пристойне житло з окремою ванною і туалетом. Мене
розмістили на кухні, біля вікна і я могла відчувати всі принади нового життя.
Площа, куди виходили вікна мала назву Космонавтів, бо на той час люди вже
полетіли до космосу. Але мій ножний привід так само залишався предметом
цікавості і нове покоління дітлахів намагалося качатися на ньому, інколи
отримуючи прочуханів.

Час йшов, колись молода, красива пара відійшла у вічність, і ми знову
переїхали, повернувшись до Євбазу, де ще залишалися пару-трійко помешкань
старої забудови.

Колишнє губернське місто швидко розросталося, і нові райони з’являлися з
неймовірною швидкістю. З кожним роком все щільніше забудовувався лівий берег
Дніпра – річки, що розділила Київ на дві частини. І мені, як корінній киянці (а
я себе вважаю саме такою), довелося пізнавати нові території. Так я опинилася
на Троєщині, на дев’ятому поверсі величезного будинку, і за вікном жовтіли
суцільні дюни: піщана рівнина з піщаними пагорбами.

Пройшло ще часу, і пустеля за вікном перетворилася на грандіозні кам’яні нетрі.
Покоління змінювалися. З’являлися нові бажаючі покачатися на ножному приводі і
мене ця постійність дітлахів покірно тішила. Прийшов час нового переїзду, що за
дивним збігом обставин привів мене майже до того самого місця, де я вперше
отримала київський притулок. Мене перевезли на вулиця Антоновича, в тій її
частині, де до Ямської пару хвилин ходу.

Що буде далі? Побачимо. Може я знову кудись переїду, може надовго залишуся тут.
Речі часто переживають своїх власників. Закінчуючи розповідь, я зрозуміла, що
нічого не сказала про своє життя швачки, але це зовсім інша історія зі своїми
цікавими пригодами, і про це наступним разом.

Мене звати «Зінгер», мій номер 5670379, призначення – швейна машинка,  я
приїхала здалеку, з-за океану, але вже більше ста років живу в цьому
прекрасному місті, з іменем Київ.

