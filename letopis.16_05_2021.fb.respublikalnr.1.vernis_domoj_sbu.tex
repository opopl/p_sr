% vim: keymap=russian-jcukenwin
%%beginhead 
 
%%file 16_05_2021.fb.respublikalnr.1.vernis_domoj_sbu
%%parent 16_05_2021
 
%%url 
 
%%author 
%%author_id 
%%author_url 
 
%%tags 
%%title 
 
%%endhead 
\subsection{Лживые обещания. В застенки СБУ, или Суть программы «Вернись домой»}

Тех, кто ведется на лживые обещания Украины в рамках программы «Вернись домой»,
отправляют прямо в лапы СБУшникам. «Возвращенцам» якобы гарантируется свобода в
обмен на отказ от своих убеждений. Но по факту – жесткое давление, вербовка и
работа на украинские спецслужбы. Поверить в то, что СБУ с распростертыми
объятиями принимает защитников Донбасса, может только очень наивный человек.
Хотя нужно отдать должное пиарщикам в погонах – играть на тонких струнах души
они умеют.


\ifcmt
  pic https://scontent-bos3-1.xx.fbcdn.net/v/t1.6435-9/186472684_133786408801467_2273136080252060398_n.jpg?_nc_cat=105&ccb=1-3&_nc_sid=825194&_nc_ohc=mBKQztzbnl4AX8CaRt7&_nc_ht=scontent-bos3-1.xx&oh=e1b2972676519452eea10ec1dfcf637b&oe=60C6560F
\fi


Если у намеченной жертвы программы имеются близкие родственники на территории
Украины – действуют через них. Другой способ – знакомства в социальных сетях
якобы с приятными парнями или девушками, которые на самом деле – обычные
фейковые персонажи спецслужбы. Третий – якобы «случайная» встреча в Сети с
давним знакомым, который уже вернулся домой и счастлив. В общем, для каждого
можно попытаться подобрать свой «ключик» – главное, выманить человека на
территорию Украины. А там – пыточная, где ломается воля человека, и согласие
работать на СБУ.

Кстати, тех, кто соблазнится «вернуться домой», ничего хорошего в жизни не
ожидает. Так что лучший совет для желающих попытаться вернуться домой через
программу «Вернись домой» – не искушать судьбу. СБУ мягко стелет, да вот только
«спать» потом придется много лет на жестких тюремных нарах.

Источник:  \url{https://t.me/respublika_LNR}
