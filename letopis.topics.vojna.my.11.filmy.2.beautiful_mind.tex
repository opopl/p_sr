% vim: keymap=russian-jcukenwin
%%beginhead 
 
%%file topics.vojna.my.11.filmy.2.beautiful_mind
%%parent topics.vojna.my.11.filmy
 
%%url 
 
%%author_id 
%%date 
 
%%tags 
%%title 
 
%%endhead 

\section{О Фильме Игры Разума (Beautiful Mind)}

Недавно я посмотрел и даже несколько раз пересмотрел прекрасный фильм Игры
Разума (Beautiful Mind - буквально, Прекрасный Разум - 2001 год - удостоенный
Оскара).  Это фильм про математика-ученого, Джона Неша, реально жившего,
который известен в науке, тем, что создал теорию игр (game theory),
математическую теорию, которая ныне активно используется во многих областях
человеческой жизни, за которую ему в 1994 присудили Нобелевскую Премию по
экономике.  Джон Неш также известен тем, что на какой то период своей жизни он
заболел шизофренией, но он смог побороть в итоге свой душевный недуг, в
конечном итоге вернувшись к активной жизни в науке. Вот про жизнь Неша в науке,
а также про то, как Неш заболел, боролся со своим недугом, и как он научился
справляться со своей болезнью, и был снят этот фильм. А касательно содержания
фильма, и впечатлений после просмотра, я перейду немного позже, сейчас хотелось
бы сделать небольшое философское отступление по поводу роли точных наук, и в
частности математики, в нашей современной жизни.

... Говорят, что математика - это царица наук, и это действительно так, хотя
тут есть парадокс, потому что математика - это в каком то смысле вообще не
наука, если принять точку зрения, что наука - это все, что проверяется и
подтверждается опытом, и что наука исследует явления природы. Ну ладно,
математика царица наук или нет, это не так важно, как важно то, что математика
лежит в основе нашего сегодняшнего понимания мира, потому что физические
законы, которые управляют всем, что происходит вокруг, выражены через
математические уравнения и формулы. И исторически развитие наших представлений
об окружающем мире, шло параллельно с развитием математики.  Более того,
технологии, которые мы пользуемся каждый день, даже например простой
выключатель света, покоятся на действии законов физики, записанных
математическим языком, как например уравнения электродинамики Максвелла.  В
общем, математика повсюду, она не видна совершенно, ее не замечают, но она
работает - и в том, как работает компьютер или смартфон, и в том, как и почему
движется вагон Киевского метро, и почему лифт в нашем доме едет вверх или вниз,
и почему, если бросить яблоко с балкона нашей квартиры на десятом этаже, оно
неизбежно упадет через несколько секунд на асфальт, громко шмякнувшись, и до
смерти испугав рядом проходящую старушку, и даже также в том, почему самолет
взлетает, когда мы летим в отпуск в Египет, хотя казалось бы, если он такой
тяжелый, так как он способен летать? Вот... математика и физика лежит в основе
нашего сегодняшнего технологичного и цифрового мира, и без Исаака Ньютона,
открывшего закон всемирного тяготения, Джорджа Кларка Максвелла, записавшего
уравнения электродинамики, или же Людвига Больцмана, записавшего законы
термодинамики, мы бы по прежнему бы бегали туда-сюда с факелами, свечками,
ездили бы на конях, и мерли бы немеряно от оспы, чумы, и прочих напастей, как
это и происходило например в Средневековье. Так что слава математике! Слава
физике! Слава точным наукам, и ученым, благодаря которым у нас есть столько
всяких технологий, облегчающих и делающих приятней нашу жизнь!

... А теперь, после такого философско-исторически-лирического отступления
перейду собственно к фильму.
