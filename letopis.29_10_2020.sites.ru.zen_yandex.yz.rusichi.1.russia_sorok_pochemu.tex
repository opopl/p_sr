% vim: keymap=russian-jcukenwin
%%beginhead 
 
%%file 29_10_2020.sites.ru.zen_yandex.yz.rusichi.1.russia_sorok_pochemu
%%parent 29_10_2020
 
%%url https://zen.yandex.ru/media/politinteres/pochemu-na-rusi-stali-govorit-sorok-vmesto-chetyredesiat-5f986d4dbaf78e79e7614f66
 
%%author Русичи (Яндекс Zen)
%%author_id yz.rusichi
%%author_url 
 
%%tags 
%%title Почему на Руси стали говорить сорок вместо четыредесят
 
%%endhead 
 
\subsection{Почему на Руси стали говорить сорок вместо четыредесят}
\label{sec:29_10_2020.sites.ru.zen_yandex.yz.rusichi.1.russia_sorok_pochemu}
\Purl{https://zen.yandex.ru/media/politinteres/pochemu-na-rusi-stali-govorit-sorok-vmesto-chetyredesiat-5f986d4dbaf78e79e7614f66}
\ifcmt
	author_begin
   author_id yz.rusichi
	author_end
\fi

{\bfseries 
Откуда взялся этот сбой?
}

\ifcmt
pic https://avatars.mds.yandex.net/get-zen_doc/3989805/pub_5f986d4dbaf78e79e7614f66_5f9a703090370858216cc970/scale_1200
\fi

Никогда не задумывались о том, что нумерация десятков в русском языке имеет
странный сбой на числе \textbf{сорок}? Вот смотрите: десять, двадцать, тридцать,
пятьдесят, шестьдесят, семьдесят, восемьдесят, девяносто... Везде в первой
части слова идет цифра, обозначающая номер десятка.

И только сорок никак не вписывается в этот ряд. По идее, на его месте должно
стоять слово четыредцать. Или четыредесят.

Самое интересное, что раньше такое слово действительно было. И до сих пор в
очень близком к нам болгарском языке есть слово \textbf{четыридесэт}. А в сербском -
\textbf{четрдесет}. То есть понятно, что по такому же принципу строился счет и в
древнерусском языке.

Но в какой-то момент старинное слово уступило место новому - сорок. Почему? И
откуда оно вообще взялось?

\ifcmt
pic https://avatars.mds.yandex.net/get-zen_doc/4004066/pub_5f986d4dbaf78e79e7614f66_5f9a70662603b20d514e1754/scale_1200
\fi

На этот вопрос отвечают лингвисты и историки. По их мнению, первопричиной стало
название некоего \textbf{мешка}, в который складывали шкурки с ценным мехом. Такие
шкурки на Руси служили не только предметом экспорта, но и аналогом денег.

Так вот, мешок этот назывался сорок. И в него помещалось обычно как раз где-то
четыре десятка шкурок. Поэтому очень быстро это превратилось в счетную единицу.
Так и писали потом в берестяных грамотах - передал кому-то два сорока кун или
три сорока белок. Или просто - \textbf{сорок кун}.

А потом слово сорок в значении четыре десятка стало настолько
общеупотребительным, что вытеснило из обихода свой аналог.

Кстати, слово \textbf{сорочка}, обозначающее нательную одежду, тоже происходит
именно от этого слова. К числам оно не имеет отношение, а вот к старинному
мешку - самое прямое.

\ifcmt
pic https://avatars.mds.yandex.net/get-zen_doc/1658056/pub_5f986d4dbaf78e79e7614f66_5f9a70cd2603b20d514ed08c/scale_1200
\fi

А еще можно вспомнить выражение \textbf{\enquote{родиться в сорочке}}. Оно вовсе не
подразумевало, что ребенок рождается уже одетым. Сорочкой назывался раньше
плодный мешок. Плацента, в общем.

Все это, как видите, восходит к одному и тому же понятию.
