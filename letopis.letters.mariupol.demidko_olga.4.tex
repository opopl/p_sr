% vim: keymap=russian-jcukenwin
%%beginhead 
 
%%file letters.mariupol.demidko_olga.4
%%parent letters.mariupol.demidko_olga
 
%%url 
 
%%author_id 
%%date 
 
%%tags 
%%title 
 
%%endhead 

Ви так чи інакше є ) і завтра і післязавтра. І я абсолютно впевнений, що Ваше
завтра буде пречудовим, де б Ви не знаходились! А якщо... раптом мій лист став
причиною... то... нічого вже тут я поробити не можу, вибачайте. Київ - то моє
місто і мій світ, мій Всесвіт, так! Я тут народився, виріс, і вже напевне
назавжди тут я буду! і я хожу куди хочу і де хочу ) Але повірте.. спеціально я
зустрічі не шукаю із Вами. Особливо якщо взяти до уваги, що я живу всього
навсього в 500 метрах від Вашого університету, кожного дня ходжу або їжджу повз
входу в університет на Преображенській, (ось... всього півгодини назад їхав із
поліграфії... роздрукував постер із Маріупольським муралом... ) і в принципі,
технічно немає ніякої проблеми якось підстерегти Вас (хм, може це не те
слово... )... Але я ніколи таке не планував, не планую, і не збираюсь робити,
бо то Ваше власне життя, Ваш життєвий простір, і я маю честь і гідність!  Але
чекаю, так, чекаю! Кожного дня чекаю, повірте щоби нарешті вибачитись
персонально  перед Вами, як це роблять зазвичай люди, як це прийнято, і як це
повинно бути! Без всяких телеграмів і фейсбуків, без всяких віртуальних
реальностей, по чесному, насправді! Хоча це й буде свого роду Страшний Суд для
мене! Весь час шукав можливості зустрічі із Вами, а не міг знайти ніякої
можливості... на Вашу виставку не пішов, хоча це 5 хвилин для мене обутись і
пройти... залишалось мучитись наодинці... бо було чорт забирай як соромно, і як
я згадую, скільки я Вам турботи завдав, так само тяжко на душі і так само
соромно стає, що аж лице пересмикує! Гарного дня!

Завтра буде цікаво на фесті ;) ...  оце от оскільки Вас завтра не буде на
фесті, трохи розскажу про те, що хотів при можливості показати Вам як
маріупольчанці і також як спеціалісту по культурі (див фото). (1) роздруковані
мурали (2) книжки - Маріупольські Листівки + Маріупольський Літопис Випуск 5 +
Збірник по Фестивалю Кролів та Писанок в Києві на Софійській Площі в 2018 (з
якого потім частина кроликів та яєць помандрувала в Маріуполь в 2019, ну а
потім частина з тих кроликів повернулась в Київ і осіла в Добропарку, і одного
з тих Маріупольських кроликів я персонально бачив в Музеї Міста Києва ще в
березні ).  Я все це вже досить ретельно дослідив, ну і власне вже все
роздрукував і зробив з того всього добра вже купу книжок, які я завтра повезу
завтра на фест, буду при можливості розсказувати. За всім цим стоїть
спеціалізована програма, яку я розробляв на протязі останніх трьох років, ну
тобто десь восени 2020 року я почав розробку цього літописного проєкту (перші строки програмного коду), ще
задовго до війни. Ну і звісно є вже 500 сторінок Сергія Давидовича Бурова у
вигляді двох товстих книжок, тобто фактично компіляція всіх збережених постів
(але вони на фото не влізли). Ідеологія за всім цим (1) знайти спосіб
розсказати людям, далеким від Маріуполя, а таких майже весь Київ, що таке є
Маріуполь, бо на жаль зараз така ситуація, що по перше Київ фактично відпав від
подій війни і таке відчуття, що місто живе саме по собі, як скажімо в році
якомусь 2010 (я як киянин кажу).  Потім (2) на жаль, в масовій свідомості людей
Маріуполь зараз асоціюється головним чином як (1) місце страшної трагедії (2)
також, місце героїчної оборони Азовсталі полком Азов.  І крім того, фактично, у
пересічних людей немає більше ніяких асоціацій про Маріуполь і це дуже шкода
бачити, оскільки в результаті немає усвідомлення тої ціни того, що було
втрачено, знищено, ну тобто...  як це пояснити... Всі от знають що таке Одеса -
море, Дерибасовская тощо. Всі знають що таке Львів - кава костьоли тощо всі
знають що таке Київ - Софія Лавра Столиця тощо. А от що таке Маріуполь? Яка
цінність цього міста для України? Чим там жили люди до війни, що це було за
місто? - а от про це фактично пересічні громадяни нічого не знають - окрім
того, що місто біля моря + Азовсталь ну і все...  і в результаті... ну... тут
багато наслідків... Але добре, вибачте вже напевне багато написав, це так,
трохи щоб пояснити що я про все це думаю... Бо з таксистами і зі своїми друзями
я вже це досить багато обговорював (до речі, я тоді ще в липні двох своїх
друзів відправив до Вас на виставку - такий собі чоловік років за 50 - його теж
звати Олександр - то мій друг + жінка вже похилого віку - він мені розсказував
потім про свої враження від Вашої виставки ) - так... з іншими я вже багато
чого вже обговорював про Маріуполь...  а з Вами всіми цими міркуванням якось не
було можливості поділитись останнім часом....

https://youtu.be/PDxxP9JmoVU


знаєте... тут справа не тільки в тому... що це місто як місто... багато чого...
але... вибачте, таке хочу сказати... Я знаю, напевне це неправильно я роблю, це
буде неприємно чути для Вас, особливо від мене, тому що я завдав Вам стільки
турботи і негативних переживань в минулому, і можливо Ви образитесь на мене
знову, але - в цьому випадку мені не соромно, за те що я зараз скажу Вам, тому
що - я як досвідчений автомобіліст (і я воджу також мотоцикл, у мене права з
2005 року, і багатолітній водійський стаж, також я деякий час працював в таксі
Уклон на своїй машині - заразі цікавості більше, грошей там багато на
заробиш... - десь близько 500 поїздок зробив, але потім облишив... - багато раз
возив і дітей в тому числі - і матерів із грудними дітьми також  - тому я знаю,
що я зараз скажу... ) я мушу таке сказати. Така їзда, як на відео вище (водій
із дитиною, причому водій дає дитині рулити) є грубим порушенням правил безпеки
перевозу дітей до 7 років. Вашу дитину можна перевозити по правилам лише в
спеціальному кріслі, причому на задньому сидінні лише (на переднє сидіння можна
садити лише після 12 років, див. витяг з правил нижче ). А садити її на переднє
сидіння і ще й давати рулити - це щось взагалі неприпустиме. Я би ніколи ні за
яких обставин не робив би таке... Подібний випадок легковажності... давно ще..
хоча це було в авіації - пілот літака дав своєму сину порулити, а в результаті
авіакатастрофа, загинуло 75 людей... Вибачте, що таке кажу... Тому що я ні в
якому разі не хочу кидати тінь на Вашого коханого, я дуже хочу, щоб Ви були
щасливі, щоб у Вашої родини, коханого, синочка, все було добре... і також, якщо
Ви знову образитесь на мене, мені буде дуже шкода, але... Мій досвід водіння
змушує мене все ж таки написати, бо правила поведінки для пишуться для всіх, і
тут немає виключень ні для кого... Ці правила, що і як треба робити правильно в
авто... знаєте... вони ж не з пустого місця впали... Я розумію, Ви дуже кохаєте
свого хлопця, і я тільки цьому радий, що Ви нарешті не сама, і що Вам набагато
легше піклуватись про своїх батьків та сина, коли у Вас є надійна опора, але у
нас в Києві дуже швидка їзда, тут їздять як попало, по тротуарах просто
гасають, підрізають, по встрєчке, короче, це Киев мать его - як співається в
одній пісні... і повірте мені, і надмірна легковажність в таких випадках,
навіть якщо Ваш коханий просто хотів таким чином продемонструвати Вам свою
любов і свою відданість до Вас і до Вашого сина, все ж таки, може призвести до
дуже аварійної ситуації... Я як автомобіліст, також, як таксіст, що багато раз
перевозив пасажирів із дітьми - а перевозити чужих дітей - то велика
відповідальність... - бо ж якщо щось станеться із дитиною... таксісту ж просто
голову відірвують.... от... як автомобіліст... та мотоцикліст із стажем я Вас
попередив, а Ви вже самі думайте... перевірте самі офіційні правила перевозу
дітей, якщо мені не довіряете в тому, що я зараз кажу... от, вибачте... (нижче
скріни із правил)

... короче, кажучи в двох словах, правила безпеки на пальцях і що може бути,
якщо їх порушувати... якщо би Ваш син раптом різко повернув руль, ви би виїхали
на встрєчку і загинули би всі разом за мить...

https://dmitri-obi.livejournal.com/11478986.html

дуже мало їздете... чорт забирай, через ідіотизм водія Ваша дитина могла загинути і Ви теж,
залишивши батьків Ваших напризволяще,
Ваш так званий коханий - хоча насправді він просто звичайний безмозглий дебіл і все (яких на жаль повним повно в Києві) - грубо порушив правила і закони України, а Ви сука дуже так мало їздете???
Вам що Сина свого не жалко??? Тупа безмозгла курка ти просто Оля!!!
Просто сором вселенський!!!
Прощавай і більше мене не турбуй!!! Ніколи!!!

дуже мало їздете... чорт забирай, через ідіотизм водія Ваша дитина могла загинути і Ви теж,
залишивши батьків Ваших напризволяще,
Ваш так званий "коханий" - хоча насправді він просто звичайний безмозглий дебіл і все (яких на жаль повним повно в Києві) -   грубо порушив правила і закони України, піддавши Вас і Вашу дитину смертельній небезпеці, - а Ви сука дуже так мало їздете???
Вам що Сина свого не жалко??? Тупа безмозгла курка ти просто Оля!!!
Просто сором вселенський!!!
Прощавай і більше мене не турбуй!!! Ніколи!!! А коханим твоїм дебілом, який дає чотирихрічним дітям просто так рулити на дорогах Києва, займуться відповідні органи, будь певна!!!


Оля, доброе утро. Вчера очень резко Вам написал, сейчас немного спокойнее. За
то, что сказал, мне абсолютно не стыдно, я написал что думаю, и мое мнение не
изменилось. Вы действительно полная дура без мозгов, что влюбились по уши в человека,
который не знает даже элементарных вещей перевозки малолетних детей в авто, который реально поставил под угрозу жизнь Вашу и Вашего ребенка. Вы влюбились в 
кретина какого то просто, вот и все.
И Вы что, действилельно думаете, что Вы готовы связать свою жизнь с человеком, 
- которому очевидно наплевать на безопасность и жизнь Вашего ребенка? ... Ну а почему я
так написал, Вы сами виноваты. Не скидывали бы видео - где в общем то чужому
человеку (вы только знакомы 2 месяца, как я понял) спокойно отдаете жизнь
своего ребенка в его руки - я бы не писал такое. Хотел бы добавить здесь
такое.  Я пересмотрел видео, сохранил его у себя как фактическое доказательство
умышленного создания возможной аварийной ситуации на дороге.  Как я сказал,
Вашим парнем вообще то должны заниматься правоохранительные органы, то есть
полиция, потому что он реально поставил под риск жизнь ребенка, Вашу,
естественно его самого, ну и также жизни людей, которые едут в других машинах.
Но к сожалению, полиция у нас довольно неповотливая в этих вопросах... В общем,
из-за таких вот безумных действий разных клинических идиотов, неспособных
прочитать и запомнить правила поведения в авто, и происходят аварии на дорогах,
уж поверьте мне. Далее, как я понял из видео, Вы отлично умеете говорить по
русски, поэтому пишу по русски. Вообще, хотя Вы превосходно говорите и пишете на украинском - Ваши публикации одно удовольствие читать, - не стесняйтесь также писать по русски, у нас в Киеве большинство на русском говорят, так уж сложилось исторически. Касательно вот этого человека.  Совершенно
ясно, что он не пара Вам, и все это скоро рассыпется как карточный домик, и Вы
останетесь снова сама. Это наверное очень печально для Вас в данный момент, но
это пройдет. И на самом деле, Вы очень талантливый и неординарный человек, Вы столько сделали для развития культуры, - и также, Вы очень красивая и обаятельная девушка, это правда, - и
Вы достойны быть счастливой и достойны найти действительно хорошего человека, с
которым Вы создадите новую семью, и с которым Вы вместе проживете долгие годы
счастливой жизни, я в этом совершенно уверен. Вы на это заслуживаете, и я
уверен, так оно и будет, обязательно будет. Но знаете что.  Для этого нужно
немного повзрослеть, найти время подумать, что то осознать, расставить
приоритеты, понимаете, да. Более вдумчиво подходить, что ли, не кидаться в
объятия первому попавшимуся человеку. И я знаете понимаю почему Вы так легковерно
отнеслись... я Вас очень хорошо понимаю, хоть и не вижу лично...  Потому что Вы
женщина, и Ваше Серце предназначено чтобы любить, так уж женщины устроены. И
учитывая, сколько Вам пришлось пережить, я думаю, да, Вы хотите снова любить,
Вы не хотите быть сама и это на самом деле очень хорошо... Потому что это знак
того, что у Вас есть и Сердце, и Душа - большая, щедрая, добрая Душа! Да! Так
что я Вас понимаю, поверьте.  Все у Вас будет хорошо, я верю в это.

... хоча я й знаю із практичного досвіду, що ймовірність поговорити із Вами у
Києві для мене рівна приблизно ймовірності польоту із Землі на Марс, (тобто
практично ніколи), все ж таки скажу, що у свій день народження (13/10), я буду
на Оболоні на прем'єрі Першого Плану Він. Вона. Божевільня, починається о
18-30. Я вже ходив на декілька вистав цього театру і мені дуже подобається гра
акторів!

%12:37:39 03-10-23
... і оце знаєте, щойно задумався про Ваші копирсайки. Дуже гарні відео!  А
щодо теми відео. Ви розсказуєте діткам там про те, що є слова з двома
наголосами. І дійсно, ці відео є унікальні і я їх собі зберіг локально... Але
знаєте, є також слова із одним наголосом, і одне із таких слів є слово поруч
(рядом російською). Ось про це я і хотів написати. Чудова штука оце слово -
поруч. А чому...  Ну як сказати... Тому що воно напевне найбільш точно
відображає моє відношення до Вас. Бо ж дійсно, ще з часу Дня Відкритих Дверей 1
квітня, коли я стояв поруч Вас - десь трохи позаду зліва (я добре пам'ятаю свої
враження від того дня), я був десь поруч Вас, недалеко. Ви тоді мене напевне
навіть не помітили, але я - Вас - помітив добре...  Вже тоді я якось собі
подумки сказав - о, оце та сама Ольга Олександрівна, яка так активно всім
займається...  Якщо в реальності, то і дні відкритих дверей, і на похоронах
Данила - Ви стояли справа від мене такі дуже заплакані - і мені було дуже
прикро бачити Вас в такому стані - і також - на концерті Сонце над Азовсталлю,
коли Ви сиділи справа від мене із своїм хлопцем, коли Ви навпаки прийшли прям
як казкова фея - от чесно - також - коли ми робили спільне фото і я навіть
ненароком поклав руку Вам на плече (ненавмисно, вибачте, так вже вийшло -
напевне воно Вам було дуже неприємно - хоча Ви й виду не подали) я теж був поруч від
Вас. Ну і звісно, кожного разу, коли Ви приїжджаєте в університет, я теж поруч
Вас, тому що корпуса університету знаходяться в 5 хвилинах пішки і поки Ви там
десь працюєте на кафедрі або сидите на якихось засіданнях або ведете заняття
або приймаєте екзамени у студентів, я сиджу у себе вдома і програмую собі в
своє задоволення, або ж десь ходжу по району, або десь чимось ще займаюсь у
своїй квартирі, хоча ми й знаходимось на відстані десь 500 метрів один від
одного.  Короче, ось таке от слово поруч. Чудове слово, чи не так? Але звісно,
якби я хотів лише сказати про ті декілька випадків, коли я мав щастя бачити Вас
вживу або ж про те, що Преображенська перетинається із Максима Кривоноса, то я
думаю, не варто було би про це писати.  А я також знаєте поруч із Вами - коли -
читаю Ваші публікації, особливо старі публікації із мирного довоєнного
Маріуполя - як Ви класно пишете, так! немовби п'єш із чистого джерела, так! або
ж Вашу монографію із тими самими неймовірними шрифтами...  Так! Тоді я теж
поруч із Вами - хоча я й не бачу Вас, але мене тішить те як Ви все це зробили -
як все красиво та талановито...  Одразу уявляєш собі - ось сидить Оля в тиші
кабінету і щось придумує, пише... Ось... я також поруч із Вами коли думаю чи
все у Вас добре і чи все добре у Вашого сина або Ваших мами або тата. Так! Я
теж тоді подумки із Вами...  Якось воно так склалось... якось так... кожного
дня подумки поруч із Вами, подумки спілкуюсь із Вами, хоча я й майже Вам нічого
не пишу, не дзвоню, не вислідковую де Ви там саме ходите і що робите...  нічого
не можу із цим подіяти... І схоже на то, що напевне вже нічого і не сможу
вдіяти, якби я навіть й дуже того хотів... а щодо того, що я був такий злий
на ситуацію із авто і назвав Вас поганими словами, насправді, я вважаю Вас дуже
розумною і я впевнений, Ви здатні розібратись в усьому і самі, якщо сядете і
все як слід розкладете по поличкам, повірте. Просто буває так... що втрачаєш
трохи розум від почуттів ось і все...  Але добре... А знаєте, яка у мене мрія
щодо Вас, якщо я колись таки зустріну Вас вживу?  А ось картинка знизу (фото з
парку в Харкові). А просто сидимо на лавочці і я Вам показую книжки, які я вже
показував багатьом людям ;) До речі, бачите, як скульптури понівечені
пострілами чи осколками?... Але живі, незважаючи ні на що! Гарного дня!

