% vim: keymap=russian-jcukenwin
%%beginhead 
 
%%file slova.polsha
%%parent slova
 
%%url 
 
%%author 
%%author_id 
%%author_url 
 
%%tags 
%%title 
 
%%endhead 
\chapter{Польша}

%%%cit
%%%cit_pic
%%%cit_text
Во второй половине XVIII века, когда Волынь перешла под власть Российской
империи, правительство получило в свое подданство совершенно новый народ,
имевший больше отношения к \emph{полякам}, а не к русским. Екатерина II, ознакомившись
с исследованиями этого народа, проведенного некоторыми российскими этнографами,
долго возмущалась тем, как сильно \emph{поляки} испортили эту часть русской нации.
Также она была недовольна тем, что по договору о разделе \emph{Польши} другая часть
«русской окраины» (нынешняя Галичина) перешла к австрийцам. В связи с этим и
были задуманы некоторые реформы по «обратной русификации» «испорченных», по
мнению российских чиновников, западных русских, которые с тем или иным успехом
(или неуспехом) применялись многими последующими властями России
%%%cit_title
\citTitle{Почему современную Украину назвали «окраиной», а не более престижно - Киевской Русью?}, 
Исторический Понедельник, zen.yandex.ru, 22.02.2021 
%%%endcit

