% vim: keymap=russian-jcukenwin
%%beginhead 
 
%%file 23_05_2021.fb.fb_group.story_kiev_ua.1.glotatel_knig
%%parent 23_05_2021
 
%%url https://www.facebook.com/groups/story.kiev.ua/permalink/1668773516652776/
 
%%author Киевские Истории
%%author_id fb_group.story_kiev_ua
%%author_url 
 
%%tags 
%%title О глотателе книг
 
%%endhead 
 
\subsection{О глотателе книг}
\label{sec:23_05_2021.fb.fb_group.story_kiev_ua.1.glotatel_knig}
\Purl{https://www.facebook.com/groups/story.kiev.ua/permalink/1668773516652776/}
\ifcmt
 author_begin
   author_id fb_group.story_kiev_ua
 author_end
\fi

\textbf{Gennady Henry Sergienko}
\url{https://www.facebook.com/groups/736908309839306/user/1454095861/}

Федя очень любил читать. По всей вероятности, это было его любимое занятие: он
читал и днём, и ночью, в любую свободную минуту. Он читал на ходу, в
транспорте, даже на переменках в школе, а во время приёма пищи, так тем более.
Он мог читать весь выходной напролёт, лишь изредка прибегая на кухню, чтобы
схватить яблоко или кусок хлеба с маслом.

Когда он ездил к бабушке и дедушке в Голосеево, он читал и в автобусе, и в
метро, и в троллейбусе. Изредка он ездил и к родственникам на Воскресенку, и
хотя она была близко от Лесного массива, где жил Федя, всё равно нужно было
добираться двумя видами транспорта: автобусом и трамваем. Конечно, при хорошей
погоде можно было махнуть и напрямик через лес, но в лесу на ходу не почитаешь.

В этот раз Федя читал \enquote{Мёртвые души}, хотя он учился в шестом классе, а их изучали в старших классах.

Доехав на трамвайчике до нужной улицы, Федя быстро зашагал вверх к дому своего
троюродного брата. Открыла ему бабушка брата, родная сестра Фединой бабушки, и
сразу же усадила за стол на кухне. Она очень вкусно готовила, и Федя не
возражал, тем более, что была возможность почитать за обедом. Они перекинулись
несколькими словами с братом, и брат пошёл смотреть футбол, а Федя остался на
кухне.

Через пару минут брат вбежал на кухню и радостно закричал:\enquote{Ты не поверишь - на воротах Рудаков!}

Дело в том, что известный вратарь киевского \enquote{Динамо} недавно объявил об уходе
из команды, и это было событием. Но Феде было неудобно встать со стола, не
закончив обед, и он пропустил этот важный момент. Оказалось, что Рудакова
поставили на ворота символически, а затем весь стадион с ним торжественно
попрощался.

© Copyright: Гена Сергиенко, 2019
