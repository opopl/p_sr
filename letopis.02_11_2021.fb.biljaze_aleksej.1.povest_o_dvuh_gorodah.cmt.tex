% vim: keymap=russian-jcukenwin
%%beginhead 
 
%%file 02_11_2021.fb.biljaze_aleksej.1.povest_o_dvuh_gorodah.cmt
%%parent 02_11_2021.fb.biljaze_aleksej.1.povest_o_dvuh_gorodah
 
%%url 
 
%%author_id 
%%date 
 
%%tags 
%%title 
 
%%endhead 
\subsubsection{Коментарі}

\begin{itemize} % {
\iusr{Konstantyn Shcherbakov}
Очень познавательно

\iusr{Anton Savidi}

Вместе добили Византию, помогая в том числе османам. Чтобы лучше контролировать
торговлю с Востоком. В конечном итоге османы им перекрыли кислород, и оба
великих города сдулись.

Поиск альтернативных торговых путей (экспедиции Колумба, Васко да Гама и так
далее) как раз пошли из-за того что османы всем в Европе перекрыли торговлю с
Востоком

\begin{itemize} % {
\iusr{Алексей Билязе}
\textbf{Anton Savidi} я сторонник гипотезы, что это был параллельный процесс. Даже если бы и не перекрыли, то все равно поиск имел бы место, т.к. были накоплены нужные ресурсы.

\iusr{Anton Savidi}
\textbf{Aleksey Bilyaze} ну в какой-то момент да, но в общем стимул был очень сильным когда османы решили все закрыть под себя

\iusr{Алексей Билязе}
\textbf{Anton Savidi} викингов в дальнюю дорогу явно не османы направили  @igg{fbicon.smile} 

% -------------------------------------
\ii{fbauth.solovjev_mykyta.harkov.ukraina.dem_sokyra}
% -------------------------------------

\textbf{Алексей Билязе} Вопрос все-таки не в ресурсах. Китай обладал кратно превышающим объемом ресурсов для морских путешествий значительно раньше. Но там отсутствовала институциональная мотивация.

\iusr{Алексей Билязе}
\textbf{Микита Соловйов} 

там еще и отсутствовала технология-как-ресурс. Тут очень глубокий вопрос:
чемпионами становятся те, кто больше работает, или больше работает тот, кому
это легче дается?


\iusr{Anton Savidi}
\textbf{Aleksey Bilyaze} 

китайцы имели серьезные морские экспедиции по Индийскому океану, но нигде не
закреплялись, как те же викинги в Северной Америке.

Я думаю, побеждают ещё и те, кому больше нужно. Европейцы знали, что хотят
попасть в Индию, потому что знали, что это \$\$\$ на торговле, потому что ещё со
времен римлян это был самый жир. И был стимул вкладывать в рисковые экспедиции.
И шла гонка кто доберется первым альтернативными путями, по дороге открыли
Америку и все остальное.

\end{itemize} % }

\iusr{Алексей Корнилов}

Соперничество друг с другом, что было неизбежно в средневековой конъюктуре,
привело к закату. Попустительство Венеции османам при падении Константинополя
привело к потере Генуей факторий и основного источника доходов из черноморских
факторий, но, неизбежно привёл к тому, что следующей оказалась сама Венеция со
своими средиземноморскими владениями. Ярчайший пример - победа в крупнейшей
морской битве Средневековья при Лепанто в 1571 не спасло от общего поражения в
войне с османами. После чего, фактически, начался этап экономической стагнации
и спада, консервации и формализации внутриполитических процессов, которые
превратились в декорацию власти олигархата на ближайшие 200 лет, до безвольного
падения перед Французской Республикой в 1797 году.

\iusr{Вячеслав Бутко}

Прогрессивная методика ценообразования от венецианских купцов (2018г,
магазинчик возле Basilica di Santa Maria della Salute)  @igg{fbicon.smile}.

\ifcmt
  ig https://scontent-frt3-2.xx.fbcdn.net/v/t39.30808-6/253128902_10227668935973997_6923233219749319769_n.jpg?_nc_cat=103&ccb=1-5&_nc_sid=dbeb18&_nc_ohc=4lLs0rRIm9MAX97fvYW&_nc_ht=scontent-frt3-2.xx&oh=889f59e7559ac3e854239ae472dad035&oe=619C36B5
  @width 0.4
\fi

\iusr{Павел Бойко}

\textbf{Олексій Білязе} Дуже крутий лонг рід. Із задоволенням тебе читаю!

\iusr{Michael Savchyn}
Добре написано

\iusr{Andy UnlinkabilityEnthusiast}
+1

\ifcmt
  ig https://scontent-frx5-1.xx.fbcdn.net/v/t39.30808-6/253359489_4746560952061251_3390618386814945584_n.jpg?_nc_cat=105&ccb=1-5&_nc_sid=dbeb18&_nc_ohc=qcy0xUkBavcAX81GAj9&_nc_ht=scontent-frx5-1.xx&oh=211ab0d5047d6bee0085699f14dba5e4&oe=619C7DE2
  @width 0.4
\fi

\iusr{Oleksandr Novokhatskyi}
Отличные системные выводы. Спасибо. С удовольствием Вас читаю.

\end{itemize} % }
