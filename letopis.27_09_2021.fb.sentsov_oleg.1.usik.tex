% vim: keymap=russian-jcukenwin
%%beginhead 
 
%%file 27_09_2021.fb.sentsov_oleg.1.usik
%%parent 27_09_2021
 
%%url https://www.facebook.com/oleg.sentsov/posts/4466714043412097
 
%%author_id sentsov_oleg
%%date 
 
%%tags __sep_2021.usik.pobeda.dzhoshua,nacionalizm,nenavist,obschestvo,rusmir,ukraina,usik_aleksandr,vospitanie
%%title Марно розраховувати на те, що "усики" самі усвідомлять всю безвихідь свого існування
 
%%endhead 
 
\subsection{Марно розраховувати на те, що \enquote{усики} самі усвідомлять всю безвихідь свого існування}
\label{sec:27_09_2021.fb.sentsov_oleg.1.usik}
 
\Purl{https://www.facebook.com/oleg.sentsov/posts/4466714043412097}
\ifcmt
 author_begin
   author_id sentsov_oleg
 author_end
\fi


\begin{center}
\begin{fminipage}{0.8\textwidth}
\Large\color{orange}\em\bfseries
...
Як ми будемо збільшувати відсоток свідомих патріотів? Зменшувати кількість
ватників фізично? Масовими розстрілами ще не було побудовано жодної успішної
держави. Чекати, коли вони самі поїдуть/помруть, і виросте нове покоління
справжніх українців?
...
\end{fminipage}
\end{center}

Недавня перемога Усика над Джошуа в боксерському матчі за титул чемпіона світу
у суперважкій вазі викликала багато суперечок в українському суспільстві.
Знову. Загальних точок зору дві. Одна - \enquote{це московський манкурт, після нього
треба прати прапор, хай котиться до свого Путіна}. Та друга: \enquote{байдуже на його
погляди, головне, що прославив Україну на міжнародному рівні}. Так, картинка
рукавичок з написами \enquote{Україна} та \enquote{Сімферополь}, які трощать щелепи
супротивника, обійшли світ, і це хороша реклама нашої держави. Але мені, як і
будь-якому патріоту України, неможливо погодитися з тією ахінеєю, яку регулярно
несе Усик, коли він відкриває рота. Його невизначена позиція по Криму та
путинські пропагандистські штампи, з яких складається свідомість цього боксера,
прямо протилежні моїм. 

Питання зараз в іншому, чи ми відштовхуємо зараз Усика запоребрік, чи
намагаємось працювати з його свідомістю та перетягувати на свій бік. Це питання
не окремої особистості, а взагалі стратегії, яку буде займати держава, а
головне - громадянське суспільство. Немає секрету в тому, скільки в Україні
людей з чіткою проукраїнською позицією (нажаль, зовсім не переважна більшість),
скільки прихильників відвертих чи прихованих проросійських сил, скільки вірян
ходять до храмів московського/фсбшного патріархату (окреме питання, чому ці
заклади взагалі досі існують на українській землі). Як ми будемо збільшувати
відсоток свідомих патріотів? Зменшувати кількість ватників фізично? Масовими
розстрілами ще не було побудовано жодної успішної держави. Чекати, коли вони
самі поїдуть/помруть, і виросте нове покоління справжніх українців? Розчарую,
що з цим теж не все гаразд - подивіться, яких ютуберів, переважно
російськомовних, дивляться діти та підлітки.

Марно розраховувати на те, що "усики" самі усвідомлять всю безвихідь свого
існування, перестануть дивитися російські канали й інших шаріїв, та почнуть
наступного року шанувати Степана Андрійовича. Що тоді робити? Точно не
відштовхувати, а роз’яснювати, особливо, коли людина вважає себе українцем, що
це насправді значить і чому не може співіснувати з \enquote{руським миром}. Щоб пройти
шлях від латентного ватника чи представника електорального болота, так званої
\enquote{какой разницы}, потрібні час, бажання та допомога тих, хто вже усвідомив свою
ідентичність. Пропоную працювати над цим, а не над знищенням можливих та
існуючих прихильників нашої державності. 

Слава Україні!
