%%beginhead 
 
%%file 06_01_2023.fb.rojz_svitlana.kyiv.1.r_zdvo
%%parent 06_01_2023
 
%%url https://www.facebook.com/svetlanaroyz/posts/pfbid0G5rPsq92jBLAbQk3YGQSQZewMDXRGepSat8xQk9jsLBTNfF1Qjxrdz7RfQTT1NsMl
 
%%author_id rojz_svitlana.kyiv
%%date 06_01_2023
 
%%tags rizdvo
%%title Різдво
 
%%endhead 

\subsection{Різдво}
\label{sec:06_01_2023.fb.rojz_svitlana.kyiv.1.r_zdvo}

\Purl{https://www.facebook.com/svetlanaroyz/posts/pfbid0G5rPsq92jBLAbQk3YGQSQZewMDXRGepSat8xQk9jsLBTNfF1Qjxrdz7RfQTT1NsMl}
\ifcmt
 author_begin
   author_id rojz_svitlana.kyiv
 author_end
\fi

Я не можу пояснити, чому я - єврейська дитина - в своєму зовсім асимільованому
та нерелігійному дитинстві так любила Різдво. Я доньці розповідала нещодавно
про цей стан - передчуття. Не подарунків, не зовнішнього свята - ми не
святкували Різдво. А того, що в тобі самій наче пробуджується і сходить - "із
внутрішньої печери" смутку, безнадійності, холоду ця "внутрішня Зірка". І
кожного року (а цього року і поготів) - це майже фізичне відчуття - наче ти зі
свого стиснутого, мов кулак, серця починаєш народжувати знов це "святе
немовля".  І вдих, і видих  стає, як при переймах - все потужніше і глибше.
Наче виштовхуєш із себе зайве, що мало б давно вийти із сльозами. Щоб
прочистити очі та серце.

А всередині у мене звучать слова моєї ж пісні: "із самої любові Дитина Свята
рождається. Нам Життя через очі її усміхається. Придивись - і в кожного з нас в
очах - коли є любов, відступає страх..."

...У нас зараз немає світла. При ліхтарях ми, наче, в печері. Я дивлюсь на нашу
Різдвяну Шопку і прошу всі світлі сили, в які я вірю - допомогти не втратити
контакт із цією Зіркою що береже контакт із Любов'ю. 

А потім - кожного року - раптом, наче все завмирає... І "починається"... Ти
переходиш за новий внутрішній поріг, виходиш з "печери" і кажеш собі - о, не
так вже і скурвилася, і трохи пришелепкувато, відкрито  і ніжно дивишся
навколо, як ми окситоциново-сп'яніло дивимося на тільки народжених малюків
людей та тварин.

Я так люблю це передчуття - надії на поєднання з цією осяяною, незаплямованою
ненавистю та болем частиною. Передчуття любові і "початку".

Я колись в Віфлеємі дивилась на ікону - єдину, де Діва Марія посміхається. (Не
знаю, чи було їй відоме майбутнє Сина). Мені здається, в її посмішці -
благословіння ніжністю. Як це - жити серед болю і зберігати в собі ніжність?

... У мене зберігається рослина - Ієрихонська троянда. За легендою її
благословила Діва Марія на безсмертя. Квітка може зберігатись сухою, стиснутою
в "кулак", а якщо її покласти в воду - швидко починає розкриватись і зеленіти.
Без коріння. Без ґрунту. А потім, якщо її витягти з води, знов стане сухою
грудочкою. Для мене ця рослина саме зараз стала символом переселенців, символом
відновлення, життєвості та життєстійкості. Навіть, якщо сьогодні я суха,
знесилена, без ґрунту - не означає, що я не зможу зазеленіти, знов відкритись
світлу. Цю рослину "оживлюють" напередодні Різдва та Великодня.

Я шукала таку в подарунок, узнала, що у нас її продають, як Selaginella
lepidophylla. "Вічна рослина" (на фото - як вона оживає)

З Різдвом, Родино ❤️ Обіймаю, наче ми вже Перемогли.

А на останніх словах, які я писала - світло включили
