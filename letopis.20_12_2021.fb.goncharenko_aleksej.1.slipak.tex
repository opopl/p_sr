% vim: keymap=russian-jcukenwin
%%beginhead 
 
%%file 20_12_2021.fb.goncharenko_aleksej.1.slipak
%%parent 20_12_2021
 
%%url https://www.facebook.com/alexeygoncharenko/posts/4788976087808257
 
%%author_id goncharenko_aleksej
%%date 
 
%%tags ato,donbass,slipak_vasyl.pevec.voin,ukraina,vojna
%%title «Міф», ми пам'ятаємо і не підведемо
 
%%endhead 
 
\subsection{«Міф», ми пам'ятаємо і не підведемо}
\label{sec:20_12_2021.fb.goncharenko_aleksej.1.slipak}
 
\Purl{https://www.facebook.com/alexeygoncharenko/posts/4788976087808257}
\ifcmt
 author_begin
   author_id goncharenko_aleksej
 author_end
\fi

«Міф», ми пам'ятаємо і не підведемо.

Сьогодні свій 47 день народження мав святкувати український оперний співак
Василь Сліпак – зірка світового масштабу та щирий патріот України. Мав
святкувати, але 2016 року загинув від російської кулі на Донбасі, де проливав
кров за рідну країну.

\ii{20_12_2021.fb.goncharenko_aleksej.1.slipak.pic.1}

Відомий співак та велика людина. Американські тури, концерти в Карнеґі-Холі,
французький етап кар'єри та всесвітнє визнання – все не цікавило Василя, коли
ворог напав на Україну. І справжній козак одразу відклав ноти.

Ім'я він мав відоме на весь світ, а «Міфом» став уже в бронежилеті та зі зброєю
в руках. Воював в Пісках та коло Авдіївки, не зупинився навіть після першого
поранення. Це та людина, яку вважаєш взірцем – йому точно не була «какая
разніца».

На жаль, життя Василя обірвав снайпер. Мені байдуже, чи знав той, кого вбиває:
російським загарбникам все одно, по кому стріляти. А нам має бути байдуже, які
балачки вони точать, ховаючи за спину закривавлені руки.

Спи спокійно, «Міф». Ми за тебе помстимось.

\ii{20_12_2021.fb.goncharenko_aleksej.1.slipak.cmt}
