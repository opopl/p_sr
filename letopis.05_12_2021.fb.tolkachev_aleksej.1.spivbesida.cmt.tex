% vim: keymap=russian-jcukenwin
%%beginhead 
 
%%file 05_12_2021.fb.tolkachev_aleksej.1.spivbesida.cmt
%%parent 05_12_2021.fb.tolkachev_aleksej.1.spivbesida
 
%%url 
 
%%author_id 
%%date 
 
%%tags 
%%title 
 
%%endhead 
\zzSecCmt

\begin{itemize} % {
\iusr{Алена Яремчук}
Классс @igg{fbicon.flame}{repeat=6} 

\iusr{Vita Golova}
Генії не прохідні,...

\iusr{Дмитрий Мурин}
Круто) дякс за текст)

\iusr{Ольга Паржицкая}
 @igg{fbicon.flame} 

\iusr{Оксана Байдуж}
Так, Григорій Сковорода завжди сучасний. Геній....

\iusr{Олексій Толкачов}
\textbf{Оксана Байдуж} Він дуже крутий. Але, на жаль, ще не пізнаний на гідному рівні.

\iusr{Lesja Bas}
Занадто розумний і вільнодумнмй

\iusr{Олексій Толкачов}
\textbf{Lesja Bas} Таким зараз теж немає місця  @igg{fbicon.wink} 

\iusr{Lesja Bas}
\textbf{Oleksiy Tolkachov} 

місце є - лише на вільних хлібах. Бо як для найманого працівника, надто
розумний... планктон виводить з зони комфорта, сіє невдоволення в колективі
самим лише філософським, а не споживацьким поглядом на життя...

\iusr{Василь Шиянюк}
Неперевершений українець!

\iusr{Ольга Малиновская}
Геніально!

\iusr{Ольга Соколовская}
Мне почему то грустно стало.......

\iusr{Олексій Толкачов}
\textbf{Ольга Соколовская} Таке життя. Можливо, тоді його світ не впіймав, тому що Сковорода не дуже вписувався у тогочасний світ. Тепер було б так само

\iusr{Alla Vyshynska}
\textbf{Oleksiy Tolkachov} Все же нужно помнить, что его отец был священником и думаю, что это имело свое влияние и по маме тоже были священники в семье.  @igg{fbicon.smile} 

\iusr{Ольга Соколовская}
Я сравнила с современностью......

\iusr{Надежда Волошина}
Еще бы не грустно - самая что ни на есть реальная реальность!

\iusr{Halyna Makukh}
Дуже влучно) люблю Сковороду. Рекомендую \enquote{Благодарний Еродій} ( постановка театру Л. Курбаса)

\iusr{Ирина Ярменюк}
От тільки кінцівка все ж сумна... вільнодумство і зараз не в пошані...

\iusr{Олексій Толкачов}
\textbf{Ирина Ярменюк} таке життя  @igg{fbicon.frown} 

\iusr{Тетяна Новак}
Геніїв оцінюють посередності, які себе вважають геніями.

\end{itemize} % }
