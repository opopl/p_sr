% vim: keymap=russian-jcukenwin
%%beginhead 
 
%%file 14_02_2022.fb.svetlichnyj_alipij.kiev.1.prazdnik_sretenia_gospodnja
%%parent 14_02_2022
 
%%url https://www.facebook.com/igalipiy/posts/3104224703152505
 
%%author_id svetlichnyj_alipij.kiev
%%date 
 
%%tags cerkov,pravoslavie,prazdnik
%%title Праздник Сретения Господня
 
%%endhead 
 
\subsection{Праздник Сретения Господня}
\label{sec:14_02_2022.fb.svetlichnyj_alipij.kiev.1.prazdnik_sretenia_gospodnja}
 
\Purl{https://www.facebook.com/igalipiy/posts/3104224703152505}
\ifcmt
 author_begin
   author_id svetlichnyj_alipij.kiev
 author_end
\fi

Праздник Сретения Господня следует рассматривать, как праздник- попразднство
Рождественскому циклу. Им завершается торжество ради  Воплощения в нашем земном
мире Бога Слова от Пречистой Девы Марии. Но не прекращается переживание
благодатного действа любви Божьей на земле.

\ii{14_02_2022.fb.svetlichnyj_alipij.kiev.1.prazdnik_sretenia_gospodnja.pic.1}

Богомладенец приносится в Храм ради ритуального жертвоприношения и тут же, в
обители Бога, окружается мистическими действиями людей, пророчествующих о Нём.
И Симеон, и Анна были долголетны, а потому стары. Об Анне (Хана Бат- Пнуэль из
колена Ашера) сказано, что она заматорела в годах своих не покидая пределы
Храма, пребывая в посте и молитве. Но именно сейчас она, подобно волхвам,
распознавшим звезду, возвещает  всем тем, кто ожидал освобождение Иерусалима,
об этом Младенце. Ведь заматорелость тела не коснулось трепещущего в руке
Божьей сердца!

Иногда служение Господу с многими ограничениями себя, ради Бога, через много
лет может выразится именно такого характера вспышкой, пронзающей своим светом
восторга всю последующую историю! 

У Бога нет «вчера» или «завтра», у Него всегда «сейчас». Люди, живущие на
земле, не всегда знают когда они войдут в область Божьего «сейчас». Но
искренние служители Всевышнего  готовят себя к этому всю жизнь. А когда это
случается, то понимают, что их сердца коснулся Господь и ради этой встречи
стоило жить!

Пусть такие Встречи станут для каждого из нас естественным итогом нашему
предстоянию Богу! Пусть наше сердце достигнет этого зенита благодати и
возгорится благородным пламенем Божественной всепобеждающей любви!

Нет нужды чего либо боятся в мирке замерзших душ, имея сердце пламенеющее
ожиданием встречи со своим Спасителем! Это Сретение состоится- в нём смысл
нашей жизни!

С Праздником, дорогие!
