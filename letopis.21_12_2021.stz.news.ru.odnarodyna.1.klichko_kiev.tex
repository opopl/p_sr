% vim: keymap=russian-jcukenwin
%%beginhead 
 
%%file 21_12_2021.stz.news.ru.odnarodyna.1.klichko_kiev
%%parent 21_12_2021
 
%%url https://www.odnarodyna.org/article/mer-v-stupore-ili-batko-klichko-sedlaet-konya
 
%%author_id mirnyj_dmitrij
%%date 
 
%%tags 
%%title Мэр в ступоре, или Батько Кличко седлает коня
 
%%endhead 
\subsection{Мэр в ступоре, или Батько Кличко седлает коня}
\label{sec:21_12_2021.stz.news.ru.odnarodyna.1.klichko_kiev}

\Purl{https://www.odnarodyna.org/article/mer-v-stupore-ili-batko-klichko-sedlaet-konya}
\ifcmt
 author_begin
   author_id mirnyj_dmitrij
 author_end
\fi

\ifcmt
  ig https://www.odnarodyna.org/sites/default/files/styles/adaptive/public/article/2021/dm21122101.jpg?itok=hjYQTPpg
  @width 0.4
  %@wrap \parpic[r]
  @wrap \InsertBoxR{0}
\fi

Градоначальник с президентскими амбициями, мэр Киева Виталий Кличко дал
развёрнутое интервью для немецкого таблоида Bild, где рассказал о подготовке
столицы к возможному российскому вторжению. По признанию мэра, уже
активизирована работа по набору резервистов и идёт подготовка к возможному
вводу чрезвычайного положения. Многие киевляне иронично ответили в социальных
сетях на воинственные заявления бывшего боксёра. Некоторые спрашивают у Виталия
Владимировича: а приедут ли ваши дети, живущие заграницей, чтобы принять
участие в отражении атаки агрессора?

Из ленты новостей

Общая мозговая и физическая недостаточность, раздвоение личности, постоянный
бред, эротомания. Но при этом бургомистр вольного города, избираюсь 17-й раз
пожизненно.

Из к/ф «Убить дракона»

\begin{multicols}{3} % {
\setlength{\parindent}{0pt}
\obeycr
Типичный немец, местный герр
Беседовал легко.
Отмечен строгостью манер,
Угрюм был мэр Кличко.
\smallskip
Он ждал суровых перемен
И от Москвы угроз.
Расслаблен телом был спортсмен,
Но напряжён был мозг.
\smallskip
– Так, значит, Путин у ворот,
Ведёт сто тысяч он?
– О, найн! Когда он нападёт,
С ним будет миллион!
\smallskip
Один удар и нам капут – 
Так шепчет мне мой ум.
Ведь президент у нас не крут,
И в боксе ни бум-бум.
\smallskip
Другого нужно дать стране.
– Гаранта? – Йя! Йя! Йя!
И фрау Меркель как-то мне
Сказала – буду я.
\smallskip
– Но Меркель нет уже. – Не знал.
Что ж, пухом ей земля.
– А есть в столице арсенал?
Защита от Кремля?
\smallskip
– О, киевлянам не впервой!
Известно с давних пор:
Когда у них есть мэр-герой,
Они дают отпор.
\smallskip
Возьмут бюджетники ломы
И будут рыть окоп,
Враги, которых тьмы и тьмы,
Не просочились чтоб.
\smallskip
А на Печерске будет схрон
(Бандеры в нём портрет)
Для тех, кто пишет нам закон – 
Мне их милее нет.
\smallskip
– Дас ист фантастиш! Алес гут!
Як кажуть, зашибись!
А киндер ваши тоже тут
Сражаться собрались? 
\smallskip
И здесь Виталя наш чуток
В беседе затупил.
В мозгу как-будто был щелчок, –
И ступор наступил.
\smallskip
И бедный мэр на этот раз
Про то сказать не смог,
Как от прорыва теплотрасс
Польётся кипяток.
\smallskip
И русским станет горячо,
Вскричат до хрипоты.
Потом (на это есть расчёт)
На них падут мосты.
\smallskip
Те, что «устали» среди дел
Не тронул их ремонт.
Но мэр молчал и так глядел,
Как проглотивший зонд.
\smallskip
Вопрос последний был: – Герр мэр,
Что там ни говори,
А может Путин, например,
Прибыть часа за три?
\smallskip
И что тогда? Каков ваш план
При скорости такой?
В глазах Кличко стоял туман,
Но в сердце был покой.
\smallskip
– Возьму хлеб-соль и во весь дух
Помчусь я на вокзал,
Встречать врага. Однако вслух
Мэр это не сказал.
\restorecr
\end{multicols} % }

Карикатура, художник Илья Гельд
САТИРИКОН
