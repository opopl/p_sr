% vim: keymap=russian-jcukenwin
%%beginhead 
 
%%file 25_10_2021.fb.karevin_aleksandr.1.plastilin_ukraincy.cmt
%%parent 25_10_2021.fb.karevin_aleksandr.1.plastilin_ukraincy
 
%%url 
 
%%author_id 
%%date 
 
%%tags 
%%title 
 
%%endhead 
\subsubsection{Коментарі}
\label{sec:25_10_2021.fb.karevin_aleksandr.1.plastilin_ukraincy.cmt}

\begin{itemize} % {
\iusr{Platon Sidorov}
Волга всегда был таким... котом Леопольдом

\iusr{Александр Пашин}

Угу. Все это замечательно. Только непонятно, отчего Украина в нынешних границах
так резко разделена- электорально, ментально, ценностно. Видимо, в Центре и на
Северо-Западе все сплошь пластилиновые, а на Ю-В все- из кремня и стали. Те
прислоняются, а эти отчего- то манали это дело. И телевизор вроде один и тот
же, а поди ж ты... Может, есть объяснение проще- у тех просто другой стержень??

\begin{itemize} % {
\iusr{Александр Каревин}
\textbf{Александр Пашин} Я утверждал, что всё замечательно? А пластилиновые везде есть. А где больше, где меньше - ещё и от срока обработки зависит.
\end{itemize} % }

\iusr{Михаил Погребинский}

Спасибо, Саша! Очень толковый и правильный пост.

\begin{itemize} % {
\iusr{Александр Тищенко}

Текст и толковый и правильный, только на перемены в векторах силы, пошлют тех
самых, якобы "пластилиновых", проливать вполне реальную, не пластелиновую
кровь.

В этом и заключается "растерянность Путина", или во всяком случае её публичная
демонстрация.

Растерян ли он в самом деле, это бабушка на-двое сказала ...

\iusr{Анатолий Попов}
\textbf{Михаил Погребинский} именно по этому, со стержнем (каким?) голосовали за Зе.  @igg{fbicon.smile} 

\iusr{Александр Каревин}
\textbf{Анатолий Попов} Надо было за По?

\iusr{Александр Тищенко}
За Пу!!!

\iusr{Анатолий Попов}
\textbf{Александр Каревин} 

надо бы не признавать захватчиков и не легитимизировать "дем.процедурами" их
существование. Мы, не голосовавшие, со стержнем, которая совестью называется. У
нас она чиста

\iusr{Александр Каревин}
\textbf{Александр Тищенко} Пу в бюллетене не было. А По говорил, что Зе - это и есть ПУ.

\iusr{Александр Каревин}
\textbf{Анатолий Попов} В тех условиях любое неголосование оборачивалось по факту голосованием за Порошенко. Так что насчёт чистой совести, Вы, уж извините, несколько преувеличиваете, на мой взгляд.

\iusr{Анатолий Попов}
\textbf{Александр Каревин} Если вы считаете, что имеют значения фамилии президентов, то без Пу будете выбирать всегда в позе между по и зе

\iusr{Александр Каревин}
\textbf{Анатолий Попов} 

Я так не считаю. Но и пророком себя не считаю. Не могу точно знать, что будет в
будущем. Не мог знать и того, окажется ли Зе вторым По или будет кем-то другим.
Маленькая надежда, что он будет другим, у меня была. Она не оправдалась.Но,
повторю, я не пророк и могу только предполагать, а не знать будущее наверняка.

\end{itemize} % }

\iusr{Alexander Violin}
 @igg{fbicon.thumb.up.yellow} 

\iusr{Ирина Александровна Чернобай}
Спасибо , Александр, замечательная статья!)

\iusr{Gavrilova Irina}

Ну спорно... Скорее с Волгой соглашусь. Вот эти все " у нас пока не получается
стать сильными.. их много, а нас мало, ... когда мир прогнется плд нас" я слышу
с 2004 года. Но реальной силой антимайдановцы не стали и не становятся, нет
лидера. Один сбежал, другие перешли на сторону силы, третьи маскируются под
наших. А жизнь-то идет. Одна, наша единственная жизнь. Без внешней помощи не
обойтись. Потому что мы- давно в гетто, построенном для нас пластилиновыми
человечками. Могут сбежать пару человек, да. Но разрушить концлагерь
заключенные не могут - давайте уж правду. И тот же Путин стоит перед выбором-
освобождать концлагерь, или договориться с вертухаем, чтоб выпускали
помаленьку, закрывая глаза на нарушения дисцплины

\begin{itemize} % {
\iusr{Александр Каревин}
\textbf{Gavrilova Irina} Да, лидера пока нет. Да, жизнь идёт. Да, заключенные не могут (во всяком случае, пока) концлагерь разрушить. Как это посту противоречит?
\end{itemize} % }

\iusr{Зоя Горенко}
Спасибо! Становитесь поскорее сильными.

\iusr{Igor Krichtafovitch}
Вот как можно из пластилина соорудить флюгер?

\begin{itemize} % {
\iusr{Александр Каревин}
\textbf{Igor Krichtafovitch} Как сказал когда-то чудик: "Посмотрите на нас! Возможно всё!"

\iusr{Igor Krichtafovitch}
\textbf{Александр Каревин} и нет предела.
\end{itemize} % }

\iusr{Пётр Гринёв}

Ну, про "Украина переродилась" мы слышим еще с 2014-го. Это из серии" не так
восстали - не так поднялись - сами виноваты - Украина отрезанный ломоть - там все
криптобандеровцы". Про растерянного Путина тоже из этой же серии. Ну а
пассионариев в любом обществе мало, поэтому о "пластилиновых" можно говорить в
каждом государстве, в том числе и в РФ, где это ярко проявилось в начале 90-х(да
и сейчас тоже).

\iusr{Вавилова Елена}

Саша, а Вам не кажется, что тех, кого Вы обозначили местоимением "мы"
становится все меньше? Одних затягивает пластилиновое болото, другие - массово
покидают эту территорию. А иных уж нет. Ещё немного и некому будет лепить
государство заново, как Вы говорите. Поэтому, как говорится, свежо предание, но
верится с трудом. А Вам, конечно, спасибо за мужество жить в этой стране,
работать по свести, и находить в себе силы верить.

\begin{itemize} % {
\iusr{Александр Каревин}
\textbf{Вавилова Елена} Это не "кажется". Это так и есть. Но это не повод опускать руки.

\iusr{Ира Гаврилова}
вот-вот
\end{itemize} % }

\iusr{Эдуард Ципищук}

Согласен с этой точкой зрения. У меня немного другая терминология. Есть 5\%
гениев, 5\% идиотов. 5\% умных, 5\% дураков. И между ними все остальные градации.
Эту шкалу можно применить к чему угодно. Так вот большинство - в центре, и это
большинство при любом раскладе. Будут другие передачи по тв, будет другое
мнение у большинства.

И да, от страны не зависит, повсюду так.

% -------------------------------------
\ii{fbauth.tominec_mihail.kiev.ukraina.mezhgorje.hirurg}
% -------------------------------------

Аналогичным образом пластилиновые подстраиваются под ковид-террор и глобалисты
лепят новый мировой порядок в виде цифрового концлагеря.

\iusr{Виктор Парандюк}

В точку!

\iusr{Всеволод Шимов}

Конформизм - естественная эволюционная стратегия. Будь как все, подчиняйся сильному.

% -------------------------------------
\ii{fbauth.pashin_aleksandr.poltava.ukraina.inzhener}
% -------------------------------------

Хотите экскурс в эмпатию? Я прекрасно понимаю, где истоки подобного
необоснованного оптимизма у киевских, черкасских, полтавских русских, помнящих
"бывшее время", когда милые пейзане не были гегемоном, а знали свое место- на
поле и у базарного прилавка. Мол, дело за малым, маленьким, малюсеньким- стоит
только явиться некоему "сильному" мессии, цыкнуть на зарвавшихся аборигенов, и
все вернется на круги своя, и Киев снова станет городом их юности, и злобная
бандерня растворится и выпадет в осадок. Понимаю, что вот это психологическое
плацебо комфортнее и удобнее, чем признаться самому себе, что родной город
перестал быть родным, что он чужой, и таким и останется, ибо времена
изменились, и тех обстоятельств, которые мотивировали украинцев стремиться
изжить в себе архаику, интегрироваться в русскую технологическую цивилизацию,
стать её частью, а не отторгать- больше нет и, вероятнее всего, в обозримом
будущем не возникнет. Надежда сохранить идентичность еще какое-то время
осталась только у русских, живущих в восьми областях Юго-Востока, где они пока
еще в большинстве. У русских Центра и Запада этой надежды увы, нет. Фарш
невозможно провернуть назад. Впрочем, право на веру и надежду имеет каждый.

\begin{itemize} % {
\iusr{Александр Каревин}
\textbf{Александр Пашин} А для меня мой родной город остался моим родным городом. Просто он оккупирован чужими. Потому и болеет, и выглядит неважнецки. Но всё равно это мой родной город.

\iusr{Александр Пашин}
\textbf{Александр Каревин} Вы ведь не желаете их поубивать, правда? Оттого и убеждаете себя и других, что их легко и элементарно изменить и переделать. Но этого не получится, увы.

\iusr{Ира Гаврилова}
\textbf{Александр Пашин} солидарна. потому что это не пластилин, а дерьмо

\iusr{Александр Пашин}
\textbf{Александр Каревин} Пост, аналогичный Вашему, мог бы изваять любой из политических украинцев в отношении, скажем, Донбасса или Крыма- с теми же эпитетами про "пластилиновые мозги" и "стержни"- только с противоположным знаком. Мол, стоит, убрать "оккупантов" и включить другой телевизор- и местные проникнутся украинской идеей и солидарно закричат СУГС. Догматизм, Александр Семенович.

\iusr{Александр Каревин}
\textbf{Александр Пашин} Ну да. Один утверждает, что дважды два четыре, а другой - что десять. Аналогичные же утверждения, правда?

\iusr{Фируза Утяшева}
\textbf{Александр Пашин} 

В отношении Крыма с теми же эпитетами про "пластилиновые мозги" НЕПРАВДА! Там
зелёных человеков встречали со слезами на глазах! Насильственная украинизация
Крыма еще в советские годы, после того как полуостров был подарен Украине
волюнтаристским решением Хрущева, не сделал их пластилиновыми. А Донбасс
дубасят 8-ой год и что то мозги у них тоже не стали пластилиновыми!

\iusr{Александр Пашин}

В математике, ув. Александр Семенович, существуют доказательства, напр.,
обратным действием. Доказательство теорем основано на формальной логике. Чем Вы
доказываете, что утверждаемое Вами- дважды два-четыре? Или Ваши утверждения
аксиомичны?

\iusr{Александр Каревин}
\textbf{Александр Пашин} 

Вы не затрудняете себя тем, чтобы подкрепить свои утверждения доказательствами,
а мне претензии предъявляете.. Но раз уж заговорили о логике: вполне логично
будет допустить, что внушённое некоему субъекту некими внешними факторами
мнение потеряет силу и убедительность, как только прекратится действие тех
самых факторов на того самого субъекта.

\iusr{Александр Пашин}

Секундочку. Это Вы прибегли к математическим метафорам, утверждая, что Ваше
мнение так же верно, как дважды два- четыре. И разве итог прекращения действия
внешних факторов не универсален? Или Ваша логика избирательна? Это одно- а ЭТО
ДРУГОЕ))?

\iusr{Александр Каревин}
\textbf{Александр Пашин} 

Да хоть две секундочки. Я сказал немножко не так. Это Вы поставили знак
равенства между пластилиновыми жителями Украины и совсем уж не пластилиновыми
жителями Донбасса. Я всего лишь попробовал показать с помощью метафоры,
насколько такое сравнение абсурдно.

\iusr{Александр Пашин}
\textbf{Александр Каревин} 

В таком случае - без метафор. Факты и доказательства даже не нужно искать- они у
Вас под ногами, просто не нужно переступать через них, как не укладывающиеся в
Вашу теорию. 

Что может быть нагляднее в вопросе о "толщине стержня" @igg{fbicon.wink} , чем
добровольческое движение? Не будем трогать регулярные войска - ВСУ или корпуса
НМ ЛДНР - там призыв, подневольщина и прочая субъективщина. Но записаться
добровольцем - это ли не эксцесс пассионария в чистом виде? 

В составе противостоящих в нынешней гражданской войне вооруженных сил имеются
структурные элементы, комплектовавшиеся добровольцами- это т.наз. "батальоны
территориальной обороны", прекрасно Вам известные "добро"- или тербаты. В ДНР с
населением 2 245 000 их сумели сформировать шесть-суммарной численностью около
3000 л/с. Это 0,13\% населения. 

Вна Украине численность 46-ти тербатов ( не считая ДУК ПС, УДА, батальона
"ОУН", "батальона имени Джохара Дудаева" и пр.- порядка 35 000. Это 0.1\%
населения Украины на весну 2014 ( без Крыма и ЛДНР, но с Харьковом, Одессой и с
Вашим родным Киевом, кстати). Так у кого стержень толще и длиннее? При том, что
как ВСУ, так и корпуса НМ ЛДНР имеют абсолютно аналогичные проблемы с
комплектованием по призыву. Ссылочки давать или сами нагуглите? Например, на
известное интервью Ходаковского "Московскому комсомольцу"? Какие будут Ваши
доказателства их пластилиновости и нашей упругости стержня? Только без
исторических экскурсов - во-первых, потому, что я, в отличие от экзальтированных
дам-с, лайкающих Ваши посты, с отличием закончил исторический факультет и
ориентируюсь в предмете не хуже Вашего, а во-вторых, если Вы все-таки
попробуете экстраполировать в современность дела дней давно минувших, я могу
подумать, что Вы сторонник современных ремейков пацификации, действий ЧОН в
1919-1921гг., или войсковых операций НКГБ-МГБ СССР в 1944-1956 годах. А это
примерно то же самое, что желать закрытия назад открытой Колумбом Америки.

\iusr{Александр Каревин}
\textbf{Александр Пашин} 

"Дипломом с отличием" козырять не надо. Во всяком случае, передо мной. У меня
такой же и я знаю, что это не показатель реального уровня знаний. Попытка
сводить здесь счёты с "экзальтироваными дамами" тоже неуместна. А по сути - Вас
просто понесло "не в ту степь".

\iusr{Александр Пашин}
\textbf{Александр Каревин} Это все Ваши "аргументы" и "доказательства"))?

\iusr{Александр Каревин}
\textbf{Александр Пашин} 

Вам бесполезно приводить доказательства. Вы трактуете их так, как Вам хочется,
видите в них то, что Вам хочется, а не то, что есть на самом деле. Предъявляете
самые разнообразные и, как правило, необоснованные претензии. Перескакиваете с
одного тезиса на другой. Мне такая "дискуссия" не интересна.


\iusr{Александр Пашин}
\textbf{Александр Каревин} 

Самдурак - любимый и единственный аргумент догматиков. Амплуа непререкаемого
оракула для упомянутых дам-с комфортнее, я понимаю)). За сим и завершим.


\iusr{Фируза Утяшева}
\textbf{Александр Пашин} 

Дорогой Александр, вы стёб Каревина не уловили, он просто хотел объяснить, что
украинские украинцы сами с усами?? Нет необходимости помогать им
самоуничтожаться - они для этого вытащили недобитых Сталиным, освободивших
потом их из тюрем Хрущёвым бандеровцев, свастику, карательных отрядов,
откровенный фашизм – все это вехи германской истории. Мы тут бессильны... Не
будем же вешать Знамя Победы как в Рейхстаге в Киеве... вот если нападут на
нас.... а так.... они сами... сами. Вас просто понесло "не в ту степь".

\end{itemize} % }

\iusr{Кира Берестенко}

Согласна и с Вами, и с Волгой - Путин не будет больше вкладывать в Украину
деньги, а без них никакой трансформации не произойдёт. Поэтому для таких как мы
все достаточно безнадёжно.

\begin{itemize} % {
\iusr{Александр Каревин}
\textbf{Кира Берестенко} А я думаю, что не безнадёжно.

\iusr{Ира Гаврилова}
\textbf{Александр Каревин} а чем докажете? Тем, что мы живы и пишем в фейсбучек?

\iusr{Арсений Родына}

Это всего-навсего демократия. Она везде "пластилиновая". В России её
значительно меньше, поэтому там больше настоящей свободы. Вы, кстати, тоже
(извините за напоминание) вели себя как "пластилиновый", когда голосовали за
Зеленского. "Непластилиновые" - это те, кто не верят ни в какие идеологии.
Они-то и есть украинский (малороссийский) народ. С ними, я думаю, всё в
порядке. Их мало, а в фейсбуке так и вовсе они незаметны, но они-то по крайней
мере хоть что-то значат. Благодаря им Украина останется русской.

\iusr{Александр Каревин}
\textbf{Ира Гаврилова} То, что мы живы и пишем - это уже не мало. А время покажет, кто ошибся - я или Вы?

\iusr{Александр Каревин}
\textbf{Арсений Родына} Извиняться за напоминание не надо. Я помню и не отрицаю того, что делал. Однако лучше жалеть о том, что сделал, чем о том, чего не сделал.

\iusr{Кира Берестенко}
\textbf{Александр Каревин} 

то, что мы живы и пишем в ФБ - вопрос времени и нашей сравнительно небольшой
популярности. Зацепим какого-то служку - он это быстро исправит.

Вот уже 21 год я каждый день вспоминаю фразу одной мудрой женщины, которая
просила меня не лезть на рожон, а я не понимала, чем это мне может грозить (я,
мол, что, Гонгадзе?).

- в отличие от Гонгадзе, тебя искать никто не будет.

\iusr{Арсений Родына}
\textbf{Александр Каревин} 

Вам не кажется, что вы оказываете слишком большую честь таким, как Зеленский,
Порошенко и иже с ними? Они из кожи лезут, чтобы, угождая заокеанским хозяевам,
как можно сильнее раздразнить русских. Вы, обращая внимание на каждый их чих,
невольно помогаете им утверждаться. И вот уже сами в азарт вошли, захотели
"непластилинового" электората. Уважаю вас как честного и мужественного историка
и потому задаюсь вопросом: зачем вы себя на них распыляете? Украина отнюдь не
пустыня, там всё еще много достойных русских людей, они нуждаются в мудрой и
неазартной русской интеллектуальной элите. Не обидеть хотел вас, а подзадорить.

\iusr{Александр Каревин}
\textbf{Арсений} Насчёт чести - нет, не кажется, что оказываю. Просто из РФ не всё видится так, как из Украины.
\end{itemize} % }

% -------------------------------------
\ii{fbauth.butorina_irina.st_peterburg.rossia.prepodavatel}
% -------------------------------------

Респект автору за пластелиновых. Украинцы, как зависимая нация, всегда тянется
к тем, кто сильнее. К счастью, сейчас есть все основания ждать ослабления
Европы. Этого хочет США, Китай, да и мы не против, а европейцы со своей
толерантностью и зеленой доктриной заводят себя в тупик. Главное, чтобы не
решили с боями из него выбираться, тогда Украина традиционно станет ареной
военных действий.

\begin{itemize} % {
\iusr{Нина Евсеева}
\textbf{Ирина Буторина} Независимую нацию не подскажете? Нацию.

\iusr{Александр Каревин}
\textbf{Ирина Буторина} Я не считаю украинцев нацией. На мой взгляд, это часть русской нации, сильно подпорченная долгим пребыванием под иноземным игом. Так бывает.

\iusr{Ирина Буторина}
\textbf{Александр Каревин} это ваше право считать или нет, но факт остается фактом: отличаются внешне, языком, ментальностью. Я долго жила на Украине, знаю тему.

\iusr{Ирина Буторина}
\textbf{Нина Евсеева} независимой можно считать ту нацию, которая имела свою государственность, даже если какое-то время была под другим государством.

\iusr{Александр Каревин}
\textbf{Ирина Буторина} 

Ну, я тоже немножко знаю тему. А насчёт отличий - факт, что 100 назад немцы
Нижней Германии отличались от немцев Верхней Германии гораздо сильнее, чем
малорусы от великорусов. Но ведь никому в голову не приходит считать верхних и
нижних немцев двумя разными народами.

\iusr{Нина Евсеева}
\textbf{Ирина Буторина} 

Государственность с нацией путаете. Бла, бла, бла. Примерчик?

\iusr{Ирина Буторина}
\textbf{Александр Каревин} 

на Украине я не встретила ни одного украинца, который бы считал себя русским.
Да и в России таких немало. А вот русских, которые считают себя украинцами по
одному из родителей тоже встречала. Сейчас вообще всем лучше так считать, чтобы
не было проблем.

\iusr{Александр Каревин}
\textbf{Ирина Буторина} 

А тут нет ничего удивительного. Все годы советской власти нам внушали, что
украинцы - не русские. Это внушалось на всех уровнях и всеми способами: на
школьных уроках, на лекциях в вузах, через СМИ, через научную и художественную
литературу... И на протяжении нескольких поколений. Головы задурили почти всем.
А вновь обретать истину каждый должен был самостоятельно. Так что не надо
удивляться, что Вы не встретили "украинцев" считающих себя русскими. Таких
немного. И не могло быть много. Но постепенно таких становится больше.

\iusr{Ирина Буторина}
\textbf{Александр Каревин} 

вы, часом, не советник нашего президента, он тоже все говорит: мы единая нация,
безумно раздражая этим украинцев. Корень у нас один, но мы уже давно разные в
силу разного географического положения, разной истории на протяжении столетий.

\end{itemize} % }

\iusr{Alex Dudchak}
Практически каждое современное общество - это на 70\% пластилиновые.

\iusr{Владимир Бурдукин}

Это явление называется конформизм. И говорит о качестве человеческого
материала. А пластилин вообще-то напоминает по виду и консистенции одну всем
доступную субстанцию....

\begin{itemize} % {
\iusr{Александр Каревин}
\textbf{Владимир Бурдукин} Напоминает. Но из той субстанции вообще что-то слепить трудно.

\iusr{Владимир Бурдукин}
\textbf{Александр Каревин} , вот и получается не пойми что.
\end{itemize} % }

\iusr{Виталия Козина}

Вы правы, у нас есть шанс, но прежде всего, мы должны поставить цели и
элементарно написать план действий. Только план не встроиться тушкой или
чучелком в чужую повестку. Нужно осознать свои приоритеты и интересы и
использовать свой потенциал для блага страны, а не кучки олигархов! Вся
патетика вокруг пластилиновости, просто, ни о чем! Вопрос, как всегда, ключевой
у кого собственность и кому принадлежит страна? Пока, это будет территорией
которую грабит капитал, причём не принципиально, свой или чужой, так и будем
искать сильное плечо, куда облокотиться.

\begin{itemize} % {
\iusr{Татьяна Давыдова}
\textbf{Виталия Козина} ну да, времени для осознания мало было, всё исчо впереди.

\iusr{Виталия Козина}
\textbf{Татьяна Давыдова} судя по действиям, ни народ ни власть этого не понимают и этим противоречием замечательно пользуются внешние игроки.
\end{itemize} % }

\iusr{Пётр Гринёв}
\textbf{Ирина Буторина} 

"...украинцы, как зависимая нация, всегда тянется к тем, кто
сильнее...". Малорусы(украинцы) такие же русские, как великорусы и белорусы. И не
вина малорусов(как и белорусов), что из них после 1917 года начали
лепить"отдельную самостийную нацию", создавая на территории Юго-Западной
России"самостийную Украину". А что касается"пластилиновых",то таковым можно
назвать любой народ. Возьмите РФ-РСФСР начала 90-х, промыли мозги людям и пошли
покорно за ЕБНом. Или Германия 1933-1945 гг., то же самое. А взять мелких удельных
князей, переходивших на службу к Великому Князю Московскому, тогда ведь еще и
России как таковой не было, а была разделённая Русь: часть под Польшей и
Литвой, часть под Москвой + "самостийные" удельные княжества. На мой взгляд, вы
просто не поняли посыл Александр Каревин .Не в украинцах дело, а в том, что нам
внушают мысль об"отрезанном ломте",что"Украина давно переродилась". Такое себе
пораженчество + работа на врага.

\begin{itemize} % {
\iusr{Ирина Буторина}
\textbf{Пётр Гринёв} 

я не против братского Союза, но считаю называть нас одной нацией - это обижать
и украинцев и белоруссов, которые претендуют на самостийность.


\iusr{Пётр Гринёв}
\textbf{Ирина Буторина} 

интересная позиция. А вы не думаете, что обидите тех, кто не принимает украинства
и считает себя русским? "Брасткий союз", как и "три братских народа", это вредная и
бесполезная затея, которая уже показала свою нежизнеспособность. Отстаивать нужно
русскость и бороться за общерусское единство, а "три братских народа" это путь в
никуда.

\iusr{Ирина Буторина}
\textbf{Пётр Гринёв} вы не мне русской это втирайте, а скажите это в Киеве, Минске. Интересно, успеете ли до границы добежать?

\iusr{Пётр Гринёв}
\textbf{Ирина Буторина} 

вы бы как-то повежливее были, без всех этих "втирайте "и" успеете добежать до
границы". Естественно, что сейчас в Киеве (как и в Минске) власть отнюдь не
русская, поэтому доказывать и дискутировать в этой ситуации
проблематично. Однако, как верно заметил Александр Чаленко, если в Киеве будет
русская власть, 95\% киевлян снова станут русскими. Украинство с его
идеологией, "мовой", которая является чужой как минимум для половины
страны, держится только на терроре и принуждении. Уйдёт всё это и Малороссия
опять будет русской.

\iusr{Ирина Буторина}
\textbf{Пётр Гринёв}, пробачте друже, но е така не дуже приемна украиньска байка: дурень думкой богатие.

\iusr{Пётр Гринёв}
\textbf{Ирина Буторина} вот это как раз о вас поговорка.
\end{itemize} % }

\iusr{Сергей Белашко}
Они подстраиваться не под власть или силу, а под тренд, моду. А пока мы, увы, не в моде.

\begin{itemize} % {
\iusr{Александр Каревин}
\textbf{Сергей Белашко} Мода - это нечто легковесное. А тут что-то потяжелее.

\iusr{Сергей Белашко}
\textbf{Александр Каревин} Вес моды определяется значимостью сферы жизнедеятельности. Сейчас закачиваются тренды, сформировавшиеся ещё в 1960-х.

\iusr{Злата Власова}
\textbf{Сергей Белашко} Власть у них всегда в моде. Через три дня после ее смены они все дружно снова будут кричать " славакпсс" как до этого кричали " понадусе"
\end{itemize} % }

\iusr{Зоя Михайлова}

80\% людей-конформисты. Ленин сказал:"Когда мы будем сильными,. все будут с
нами" Так и получилось. Что касается окраины-как основной элемент и часть
древней Руси, она будет зависеть от остальных элементов. Как самостоятельная
,она будет поглощена поляками и Литвой, как это было в 14 веке. Тем более
Люблинскую унию Зе подписал, ждем Берестейскую


\iusr{Wladimir Kruschkow}
\textbf{Александр Каревин}, 

возможно, Украине в перспективе пригодится опыт Австрии. Там, как известно,
очень многие были активистами нацистского движения, включая Гитлера. Однако, по
мере продвижения Красной армии к границам этой страны, процесс денацификации
этой страны ускорялся. В ходе оккупации Австрии войсками союзников
денацификация вышла на уровень национального законодательства. От массовой
поддержки нацизма страна проделала путь к массовому его неприятию. Подробности
здесь:

\url{https://interaffairs.ru/jauthor/material/2198}

\iusr{Natalia Gavrylenko}

Как у Стугацких: когда вышки переключат на другой посыл, они стройными рядами
изменятся. Другое дело у кого власть над вышками.

\iusr{Сергей Миронов}

За время, когда руководство Союза сплошь состояло из украинских
выходцев, вырастили поколение иждивенцев. Теперь это устойчивое мировоззрение
небратьев-весь мир им что-то должен, все им должны что-то дать, причем на
халяву! Не приведи Господь, эти миллионы иждивенцев окажутся в России! Рисунок
автора в тему.

\ifcmt
  ig https://scontent-lhr8-2.xx.fbcdn.net/v/t1.6435-9/248681488_3091485667731809_148162180340584153_n.jpg?_nc_cat=105&ccb=1-5&_nc_sid=dbeb18&_nc_ohc=aDPvk9yDcOAAX9t7ZmL&_nc_ht=scontent-lhr8-2.xx&oh=4bd7bdc86870a30f79ab3222e3a70815&oe=619D9084
  @width 0.4
\fi

\iusr{Артём Алексеев}
вот и я думаю, что всё дело в Путине))

\iusr{Мирослава Александровна Бердник}

Согласна. Правда, пластилиновость, которую часто называют конформизм, присуща ок. 80\% любого общества любой страны.

\iusr{Любовь Чуб}

\obeycr
Я не знаю почему, но этот ваш друг уже на протяжении лет 3-х настойчиво утверждает мысль, что Миргород не просто не той, а уже чисто Львовский)).
Пыталась спорить, бесполезно).
Вы написали очень хорошо о пластилиновых).
Все так и есть, все так и будет.
Добавьте ещё тех, кто со стальным характером, поэтому переобувается на лету), тех, кто боится признаться в подлинных чувствах... - и картина маслом для оптимизма.
Ну, и лично меня потешила "растерянность Путина")
\restorecr

\iusr{Пётр Гринёв}

Ну вот, выше уже нарисовался "тоже-россиянин" (не путать с русскими), для которого
Россия это РФ, а малороссы вовсе не русские, а "иждивенцы". Вот эта слякоть, особо
не отличающаяся от своих таких же собратьев-кастрюлек из Украины, чётко работает
на раскол России и общерусского единства.

\iusr{Валерий Васильевич Кошевой}

Растерянность - для лохов. Кремль знает что делает. Даже время играет на него.

\iusr{Ни Соо}

\obeycr
грядут большие перемены
сказал седой хамелеон
и вдруг окрасился цветами
соседней дружеской страны
\restorecr

© Arhistratig

\iusr{Ни Соо}

\obeycr
ученый сверстник галилея
был галилея не глупее
он знал что вертится земля
но у него была семья
\restorecr

\end{itemize} % }
