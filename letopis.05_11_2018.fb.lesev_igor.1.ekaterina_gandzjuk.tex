% vim: keymap=russian-jcukenwin
%%beginhead 
 
%%file 05_11_2018.fb.lesev_igor.1.ekaterina_gandzjuk
%%parent 05_11_2018
 
%%url https://www.facebook.com/permalink.php?story_fbid=2159804670717291&id=100000633379839
 
%%author_id lesev_igor
%%date 
 
%%tags gandzjuk_ekaterina.ukr,ukraina
%%title По Екатерине Гандзюк
 
%%endhead 
 
\subsection{По Екатерине Гандзюк}
\label{sec:05_11_2018.fb.lesev_igor.1.ekaterina_gandzjuk}
 
\Purl{https://www.facebook.com/permalink.php?story_fbid=2159804670717291&id=100000633379839}
\ifcmt
 author_begin
   author_id lesev_igor
 author_end
\fi

По Екатерине Гандзюк.

События вокруг нее настолько быстро раскручиваются, что вызывают ряд вопросов.

Первый – это покушение на убийство. Даже не само покушение, а демонстративная
степень жестокости. Мы живем в такой чудесной стране, где за 300 долларов можно
найти киллера. За 100 можно заказать избиение. Но если хотят кого-то убить или
«наказать», кислотой обычно не пользуются.

\ifcmt
  ig https://scontent-frt3-1.xx.fbcdn.net/v/t1.6435-9/45520441_2159804514050640_6121955527980220416_n.jpg?_nc_cat=104&ccb=1-5&_nc_sid=730e14&_nc_ohc=hF71gG4a9QMAX8AzaD5&_nc_ht=scontent-frt3-1.xx&oh=f4646ce2acf0c7e145c45b5da8bc8b12&oe=61BB8F7E
  @width 0.4
  %@wrap \parpic[r]
  @wrap \InsertBoxR{0}
\fi

А пользуются кислотой, когда хотят вызвать эффект. И его вызвали лишь частично,
если говорить об июле, когда и было совершено покушение. Гандзюк, прямо скажем,
не была широко известной медиа-персоной. В своей тусовке, да. Но кто сейчас с
кондачка назовет активиста-евромайдановца из Николаева или Житомира?

Поэтому, когда на Гандзюк шли с кислотой, задача-максимум была – вызвать
эффект. Тогда он был лишь частичным.

Но произошло второе – Гандзюк умерла. Официальная версия – тромб. Но
теоретически могли ведь и помочь. У нас врачи плохо лечат. Но убить человека,
находящегося и без того в критическом состоянии, для специалиста-врачевателя не
самая большая задача. Это, конечно же, не обвинение. Просто констатация того,
что теоретически это могло случиться.

А теория эта косвенно подтверждается третьим пунктом – вокруг имени Гандзюк
начинается информационная феерия. В воскресенье свои первые спонтанные акции
провели активисты. Возможно, они самостоятельно собрались. Активисты – они же
активные. Но люди, которые работают в журналистике – сфере, которая по идее
должна держать руку на информационном пульте – знают, что в воскресенье хрен
когда соберешь оперативно редакцию. До половины вообще не дозвонишься. Другие
пьяные или «уехали». А тут – встали и пошли/пришли.

В воскресенье же появилась оперативная реакция из посольства Штатов. Посольство
США в Украине – это чуть больше чем посольство. Это как «поэт в России больше
чем поэт» у Евтушенко. И вот в Штатах требуют наказать заказчиков.

А вот это самое интересное. Потому что исполнители давно уже задержаны. С ними
тоже вышел конфуз – среди них нет симпатиков Путина или хотя бы Януковича.
Впрочем, мы уже приучены, что агенты и симпатики России убивают в Украине таким
образом, что исполнителей найти нет возможности. Во всех остальных случаях в
преступлениях начинают фигурировать ветераны АТО и добробатов. С Гандзюк
ситуация такая же.

Но Штаты требуют заказчика. А еще чуть раньше появляется пост любителя кофе
Найема, который служит не то, чтобы не украинскому народу, но американскому
правительству – это бесспорно. И вот Найем в своей пространной и эмоциональной
речи пока что никого не обвиняет. Но он акцентирует направление конфликта – это
было убийство и оно управляется кем-то коварным и злым.

Правда, у Найема есть еще рассуждение о том, что ментовское руководство, если
не справляется со всей этой чехардой, могло бы и свалить в отставку. И
заметьте, что и активисты в воскресенье по городам Украины собирались под
областными управлениями МВД. Случайность? Ну, конечно же.

Сама власть по итогу воскресенья сильно бздит. Порохоботы гнут линию, что
Гандзюк поддерживала мэра Херсона и вообще, была преданной сторонницей Петра
Алексеевича. Уже успел дать заявление Луценко, что в ближайшие дни будут
названы/найдены заказчики убийства Гандзюк. Ну и, естественно, скорбящее письмо
по горячим следам с обещаниями быстро все порешать, опубликовал Порошенко.

И у власти ситуация сейчас предельно сложная. Формально преступление давно уже
раскрыто. Заказчик, если отойти от конспирологической версии, вполне может быть
мелким местным фунтом. Гандзюк ведь не розы на рынке продавала. Она была
девушкой жесткой и ее точно далеко не все любили в Херсоне и окрестностях. Но
теперь назвать просто какого-то держателя трех ларьков и обладателя свободных
300 долларов заказчиком преступления – вот такое не пропетляет.

Не пропетляет, потому что настоящие мотивы преступления, обычно самые скучные,
банальные и очевидные. А общество сейчас целенаправленно накачивают. Ассоциации
с Врадиевкой уже гуляют. Технологии те же. И самое паршивое для Петра
Алексеевича, что к процессу подключается американское посольство. 

И центральный вопрос теперь не в том, кто настоящий заказчик Гандзюк, а в том,
на кого укажут. А вот тут могут быть интересные и очень неожиданные варианты.

\ii{05_11_2018.fb.lesev_igor.1.ekaterina_gandzjuk.cmt}
