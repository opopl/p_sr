% vim: keymap=russian-jcukenwin
%%beginhead 
 
%%file 06_11_2021.fb.chugajenko_tatjana.kiev.1.den_osvobozhdenia_kiev
%%parent 06_11_2021
 
%%url https://www.facebook.com/tanja.lilo/posts/4450491708353090
 
%%author_id chugajenko_tatjana.kiev
%%date 
 
%%tags 1943,1943.kiev.osvobozhdenie,6_nov,dedushka,istoria,kiev,mirvojna2,nacizm,rkka,semja,sssr,ukraina,vov
%%title Сегодня День освобождения Киева от немецко-фашистских захватчиков
 
%%endhead 
 
\subsection{Сегодня День освобождения Киева от немецко-фашистских захватчиков}
\label{sec:06_11_2021.fb.chugajenko_tatjana.kiev.1.den_osvobozhdenia_kiev}
 
\Purl{https://www.facebook.com/tanja.lilo/posts/4450491708353090}
\ifcmt
 author_begin
   author_id chugajenko_tatjana.kiev
 author_end
\fi

Сегодня День освобождения Киева от немецко-фашистских захватчиков. 

Событие знаковое и очень важное. Многие мои друзья об этом уже написали. Я
только немного добавлю. 

Одними из первых ранним утром 6 ноября 1943 года в Киев ворвались советские
танки. На одном из них был мой дедушка, Евгений Афанасьевич. Потом в память об
освобождении столицы Украины в городе  появилась улица Танковая. Так назвали
одну из улиц, по которой наши танки тогда  двигались на Крещатик. А 10 лет
назад улицы Танковой в Киеве не стало. Нет, она никуда не делась. Просто ей
дали другое название, поскольку на ней вот-вот должно было открыться посольство
США, и поэтому срочно искали что-нибудь "более благозвучное". Мой дед об этом
не знает, поскольку к тому времени он уже умер. 

\ifcmt
  ig https://scontent-frx5-1.xx.fbcdn.net/v/t39.30808-6/252362967_4450491628353098_8422708203332058588_n.jpg?_nc_cat=110&ccb=1-5&_nc_sid=8bfeb9&_nc_ohc=z5MhmhFCoxEAX__nW32&_nc_oc=AQl_0bOkE6m18fvZZ0_ao2IKvI-kN17q-0qAD_nl2qyLXRfglv6PdhmN6vrR7VHqPEY&_nc_ht=scontent-frx5-1.xx&oh=9c1224e3c279be0fe7ca83b671dba0cc&oe=61A3E7C4
  @width 0.5
  @wrap \InsertBoxR{0}
\fi

А я знаю. И от таких знаний мне очень грустно. И стыдно. 

С годовщиной освобождения, киевляне! Обращаюсь к тем, кто ещё "при памяти" и в
состоянии оценить значимость этой даты.

P.S. Поскольку в прошлом году кое-кто здесь у меня пытался поязвить на тему,
что эту улицу, мол, переименовывали «регионалы» и лично Александр Попов, я
повторю. Так сказать, во избежание.   @igg{fbicon.face.grinning.squinting}  

Граждане, вынуждена вас разочаровать. И в виде политпросвещения сообщаю, что
решения о переименовании улиц принимает городской совет. Мэром Киева в то время
(то бишь главой горсовета) был не Александр Попов, а Леонид Черновецкий. А по
фракциям картина тогда была такой: 

Большинство (30,45\%) у Блока Леонида Черновецкого. На II месте Блок Юлии
Тимошенко, у них 22,79\%. Третье место у Блока Виталия Кличко- 10,61\%. Далее
идут Блок Владимира Литвина - 8,71\% и «Гражданский актив Киева»- 5,95\%. А у
Партии регионов тогда было лишь VI место и 3,95\%. Так что выпендриваться по
этому поводу даже не начинайте. Не получится.

\ii{06_11_2021.fb.chugajenko_tatjana.kiev.1.den_osvobozhdenia_kiev.cmt}
