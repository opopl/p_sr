% vim: keymap=russian-jcukenwin
%%beginhead 
 
%%file slova.surzhik
%%parent slova
 
%%url 
 
%%author 
%%author_id 
%%author_url 
 
%%tags 
%%title 
 
%%endhead 
\chapter{Суржик}
\label{sec:slova.surzhik}

%%%cit
%%%cit_head
%%%cit_pic
%%%cit_text
Решение языковой проблемы в Украине Юлия Мендель видит с помощью избавления
России от монополии на русский язык. Нужно внедрить \enquote{украинский русский
язык}.  С подобным заявлением Мендель уже выступала, когда занимала пост
пресс-секретаря президента. Но в своей книге она объяснила, каким видит
украинский русский – \enquote{\emph{сочным суржиком}}: "Перші роки свого життя
я провела в херсонському селі. А там, в українських селах, мова взагалі
неймовірна. Це \emph{соковитий суржик}, насичений енергією свободи і
всерозуміння, практичності й кмітливості. Українська фонетика з украпленнями
російських слів, з українськими вставками, прислів’ями й органічними
неологізмами. Українська? Російська? Ні тобі, ні мені".  При этом, в ее
понимании, русскоязычный гражданин – это человек, который верит в российскую
пропаганду, в \enquote{фашизм} на Майдане, в ужасы националистов с факелами и
так далее
%%%cit_comment
%%%cit_title
\citTitle{\enquote{Я їб...в Гітлера в зад}. Что пишет Юлия Мендель в своей книге о двух годах у Зеленского}, 
Екатерина Терехова, strana.ua, 07.07.2021
%%%endcit
