%%beginhead 
 
%%file 02_12_2022.fb.kovaleva_anna.odessa.klinicheskii_psiholog.2.nastaet_den__kogda_t
%%parent 02_12_2022
 
%%url https://www.facebook.com/permalink.php?story_fbid=pfbid02MTKq8NyiSRBKpvK3d7ZW281bRfJPj1vpyF3Z8T5hzvTkhJFAjhztPtxAEwLiWBNal&id=100007996034347
 
%%author_id kovaleva_anna.odessa.klinicheskii_psiholog
%%date 02_12_2022
 
%%tags 
%%title Настает день, когда ты начинаешь видеть вещи, которые раньше не замечала
 
%%endhead 

\subsection{Настает день, когда ты начинаешь видеть вещи, которые раньше не замечала}
\label{sec:02_12_2022.fb.kovaleva_anna.odessa.klinicheskii_psiholog.2.nastaet_den__kogda_t}

\Purl{https://www.facebook.com/permalink.php?story_fbid=pfbid02MTKq8NyiSRBKpvK3d7ZW281bRfJPj1vpyF3Z8T5hzvTkhJFAjhztPtxAEwLiWBNal&id=100007996034347}
\ifcmt
 author_begin
   author_id kovaleva_anna.odessa.klinicheskii_psiholog
 author_end
\fi

Настает день, когда ты начинаешь видеть вещи, которые раньше не замечала.

Обвисшие обои, облупленную тумбочку, желтый потолок, свою жизнь, в некоторых ее
неподходящих аспектах.

И что самое интересное, начинаешь все это видеть, когда нет света, когда темно,
потому что есть время и место посмотреть (зачеркнуто) почувствовать где и что
идет не так.

Война не время для ремонта?

Херня)

Решение своих вопросов подождёт до победы?

Херня)

Именно здесь и именно сейчас.

Пока нет света и какие то привычные действия и беготня отложились, самое время
решать эти вопросы, чтобы они после победы уже не беспокоили.

Покрасить - пожалуйста.

Побелить - не вопрос.

Помыть заскорузлый жир - милости просим, налобный фонарик в помощь.

Обнять, поговорить, принять решение, обсудить, «посмотреть и увидеть», не
откладывать, жить сейчас, без света, тепла и привычного комфорта - это можно
делать уже сегодня.

Помните, раньше, при ковиде, при изоляции - если ты думал, что раньше тебе не
хватало времени, то сейчас ты точно знаешь, что ты просто ленивая жопа.

Наш квест усилился в миллионы раз, давайте попробуем его пройти по уровню, по
чуть чуть, не спеша.

Короче, я начала делать ремонт, сама, с фонариком, на кухне и в коридоре.

Не знаю почему, но меня это очень вдохновило)

Наверное по сравнению с тем, когда я задумываюсь - а что я буду делать этими
длинными вечерами?

Сейчас приятная усталость и дурные мысли ушли вон.

Может быть это и психологический обход, но пока я с этим разберусь, в квартире
будет пахнуть новым ремонтом и реставрированными своими руками тумбочками.

Вот так, включается утром свет, а живешь уже в новом пространстве и новой
реальности.

По утрам видится как чудо, которое я творю своими руками.

Не раскисаем.

Делаем ремонт своими силами, моем, скребем, перебираем, очищаем пространство,
лучше работает когда поешь вслух песни.

Реально все выглядит по другому, а вечером волшебно.

Прохожу по дому с фонариком, любуюсь и обязательно хвалю себя за проделанную
работу, порой такую откладываемую.

Когда думаешь - может ну его нахер?

Нет, давай, тихонечко, кофеек, сигаретка, молитва о благословении на дела
лучшие.

Молитва о воинах, детях, близких, Украине, обо всех наших людях, о Победе..

Вооот, и силы пошли, и желание,  Он всегда слышит, всегда с нами. 

В самой глухой ночи, в самой непроглядной тьме, которая кажется никогда не
закончится.

Всегда с нами, всегда рядом, мы не всегда с Ним, поэтому возвращаемся,
попросили сил и пошли дальше заниматься своими делами.

Пусть Он рулит, Он направит самым наилучшим образом 🙏🏻🇺🇦💛💙🇺🇦

Кстати, именно Он научил меня клеить, резать, мерить 3d панели. Покажу
результат.

До этого я ведь не занималась этим, а пошло как по маслу, просто попросила и
поверила 🙏🏻

А еще я заметила, что свет включается именно тогда, когда я поставила намерение
поработать. И выключается тогда, когда видимо пора отдохнуть)
