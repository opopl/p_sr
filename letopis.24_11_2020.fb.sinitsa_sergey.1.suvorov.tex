% vim: keymap=russian-jcukenwin
%%beginhead 
 
%%file 24_11_2020.fb.sinitsa_sergey.1.suvorov
%%parent 24_11_2020
 
%%url https://www.facebook.com/sergiysinitsa/posts/2734082303514374
 
%%author 
%%author_id sinitsa_sergey
%%author_url 
 
%%tags 
%%title День рождения Суворова
 
%%endhead 
 
\subsection{День рождения Суворова}
\label{sec:24_11_2020.fb.sinitsa_sergey.1.suvorov}
\Purl{https://www.facebook.com/sergiysinitsa/posts/2734082303514374}
\Pauthor{Синица, Сергей}

\index[names.rus]{Суворов!Александр Васильевич(1730-1800), полководец}

\ifcmt
pic https://scontent-waw1-1.xx.fbcdn.net/v/t1.0-9/127442581_2734082266847711_2108633342159065248_n.jpg?_nc_cat=104&ccb=2&_nc_sid=8bfeb9&_nc_ohc=-TX5goUZanYAX8ut7vl&_nc_ht=scontent-waw1-1.xx&oh=541cfd96da5246526e0c8b6f580030fc&oe=5FE2C2FA
caption Александр Васильевич Суворов
\fi

Великий военачальник времен Российской Империи \textbf{Александр Васильевич Суворов}
(1730 - 1800) является символом целой эпохи побед русских войск. Его
боготворили солдаты, он пользовался уважением у знати, о нем ходили легенды в
народе.

Год рождения полководца достоверно неизвестен. В официальных документах указывается 24.11.1730 года.

Его отец генерал тайной канцелярии происходил из древнего шведского рода. В своей служебной деятельности отличался жестким нравом.

Мать Александра Васильевича Евдокия (Авдотья) была потомком одного из московских дворян, служивших при дворе.

Свое имя Суворов получил в честь великого Александра Невского. Молодой
Александр был слаб здоровьем и часто болел, поэтому все его детство прошло в
деревенской усадьбе отца. С юных лет Суворов тяготел к военной науке. По книгам
отца изучал основы военной подготовки, тактику, стратегию и все что связано со
службой. Также он начал приводить свое здоровье в надлежащее состояние, усилено
занимался физической подготовкой и закаливанием. Родители же готовили Суворову
жизнь гражданского служащего, но один из друзей семьи, разглядев в Александре
задатки полководца, сумел их переубедить. Это был знаменитый генерал А.
Ганнибал.

\subsubsection{Военная карьера Суворова}

1742 год стал для Александра переломным. Он поступает на службу в Семеновский
полк, мушкетером. В этом военном соединении он служит шесть лет. Все свое время
он посвящает изучению военной науки, ее тонкостей и хитростей.

В 1754 году Суворов получает офицерский чин поручика и переводится служить в
пехотный Ингерманландский полк.

Свое первое боевое крещение Александр Васильевич получил в разгар Семилетней
войны, а именно - летом 1759 года. Молодой офицер в составе драгунского
эскадрона вступили в схватку с противником и одержали сокрушительную победу над
прусскими войсками. После этого Суворова повышают в звании.

Первая воинская слава же пришла к молодому офицеру при осаде Кольберга. Город,
осаждаемый русскими войсками, ждал подкрепления. Командование российской армии
выдвинуло наперехват прусским войскам Донской полк, где находился Александр. В
состоявшемся сражении он проявил себя как бесстрашный и смелый офицер,
способный принимать волевые решения в пылу сражения и вести солдат за собой.
Также Суворов принимает участие в сражениях за Берлин и Кунерсдорф. К концу
этой войны он получает должность начальника штаба при кавалерийском корпусе.

В начале осени 1762 года Александр Васильевич получает звание полковника и
принимает под командование Пехотный Астраханский полк. Необходимо отметить, что
в этот чин Суворова производит сама Екатерина вторая.

Следующим шагом в его военной карьере стало командование Суздальским полком.
Здесь он издает «Полковое учреждение», где описываются основные правила
поведения солдат и основы их боевой подготовки.

В сентябре 1768 года его производят в чин бригадира.

Во время боевых действий на территории Польши Суворов проявляет великие
тактические способности. Он побеждает практически в любом сражении.

Зимой 1770 года его производят в звание генерал- майора. В том же году он получает свой первый орден.Суворов Александр Васильевич
Летом и весной 1773 года Суворов со своим отрядом дважды переходят Дунай и
одерживают победу над турецкими войсками под Туртукаем.

Во времена восстания Емельяна Пугачева, Александр Васильевич получает приказ
подавить бунт от императрицы. После разгрома войск Пугачева Суворов лично
конвоирует главаря бунтарей в Симбирск.

С 1774 по 1786 он руководил различными военными соединениями практически по всей территории Российской империи.

В 1778 году Суворов сумел разбить турецкий десант при попытке того высадится в
Ахтирской бухте, чем предотвратил начало новой войны с Турцией, которая в то
время была крайне невыгодна России.

В 1786 году он получает звание генерала. При осаде Очакова Суворов получает ранение.

Считавшаяся до этого неприступной, крепость Измаил была взята Суворовым в 1790
году. В 1791-94 годы Александр Васильевич занимается укреплением оборонных
сооружений на границах своей родины.

Летом 1794 года Суворов назначают на командование войсками российской империи и
отправляют подавить восстание в Польше. После чего получает звание
фельдмаршала.

В 1796 году он закончил написание знаменитого своего произведения «Наука побеждать».

Суворов Александр ВасильевичС приходом к власти императора Павла первого
Суворов попадает в его немилость. Он отказывается исполнять приказы императора
по новому обустройству армии, тем самым вызывая на себя гнев правителя. Его
разжалуют из армии и отсылают в собственное имение. Вскоре новопреставленный
император вновь возвращает его на службу.

Суворов получает назначение главнокомандующего союзными войсками на территории
Италии, которая к тому времени была занята французами. В 1799 году Александр
Васильевич благодаря своим умелым действиям и природному таланту сумел
освободить Италию от захватчиков.

В том же году состоялся его последний боевой поход, Швейцарский. Суворов и в
нем сумел отличиться, совершив, казалось, немыслимое. Он перебросил свои войска
через высокогорные Альпийские хребты Росштокс и Паникс. Это предприятие
принесло полководцу не былую славу и вписало его подвиг в историю.

По исходу Швейцарского похода Павел Первый произвел Александра Васильевича в
генералиссимусы. Самый высший ранг военного руководства.

\subsubsection{Личная жизнь Александра Суворова}

В отличие от военной карьеры, события на личном фронте великого полководца оставляли желать лучшего.

В 1774 году Суворов венчается с княжной Прозоровской В. И.. Их семейная жизнь
не сложилась изначально. Вдобавок ко всему Александр Васильевич узнает о
неверности его жены. Он принимает решение развестись и даже начинает
бракоразводный процесс. Через 10 лет после свадьбы процесс был закончен. От
этого брака у Суворова оставались сын Аркадий 1784 г. р. и дочь Наталья 1775 г.
р. Сын полководца утонул, не дожив до 30 лет в звании генерала, оставив после
себя четырех детей. Дочь же родила Суворову шестерых внуков.

\subsubsection{Смерть Великого Полководца}

Великий полководец скончался весной 1800 года в Петрограде. Гроб с его телом
несли ветераны, служившие под его началом. Процессия состоялась в присутствии
огромного количества народа. Погребение состоялось в Александро-Невской Лавре.
