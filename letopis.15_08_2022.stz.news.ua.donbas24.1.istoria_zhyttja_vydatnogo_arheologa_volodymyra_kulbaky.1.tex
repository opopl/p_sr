% vim: keymap=russian-jcukenwin
%%beginhead 
 
%%file 15_08_2022.stz.news.ua.donbas24.1.istoria_zhyttja_vydatnogo_arheologa_volodymyra_kulbaky.1
%%parent 15_08_2022.stz.news.ua.donbas24.1.istoria_zhyttja_vydatnogo_arheologa_volodymyra_kulbaky
 
%%url 
 
%%author_id 
%%date 
 
%%tags 
%%title 
 
%%endhead 

\subsubsection{Життєвий шлях}

\ii{15_08_2022.stz.news.ua.donbas24.1.istoria_zhyttja_vydatnogo_arheologa_volodymyra_kulbaky.pic.1}

Біографічних відомостей про Володимира Костянтиновича залишилося небагато.

Народився В. К. Кульбака в м. Маріуполь (тоді — Жданов) 22 березня 1954 року.

У 1971 році закінчив тут же курс середнього навчання в школі № 51.

У 1972 — 1974 роках служив у лавах Радянської армії.

Сьогодні важко визначити, коли саме у молодого Володимира з'явився інтерес до
археології. Однак відомо, що освоювати ази польових досліджень, розширювати
свій кругозір, захоплюватися \emph{\textbf{дослідженням пам'яток катакомбної культури епохи
бронзи}} він почав з 1976 року, коли працював лаборантом Сіверсько-Донецької
експедиції Інституту археології АН УРСР.

У 1980 р. В. К. Кульбака почав навчання в Донецькому державному університеті.

Перші самостійні розкопки він провів у 1984 році як керівник створеного ним
археологічного гуртка при тоді ще Орджонікідзевському районному Будинку
піонерів м. Жданова. Гурток переріс у Жданівську археологічну експедицію.
Рятувальні розкопки частково зруйнованого кургану «Виноградники» на західній
околиці м. Жданова фінансувалися Донецьким відділенням Товариства охорони
пам'яток.

Після завершення навчання в 1986 році Володимир Кульбака працював у складі
науково-дослідного сектора кафедри археології, історії стародавнього світу та
середньовіччя ДонНУ (1986−1987), де він виконував обов'язки інженера.

Важливим етапом становлення для молодого спеціаліста як організатора
археологічних робіт було створення власної експедиції. У 1989−1991 роках В. К.
Кульбака очолив відділ археології Донбасу в Республіканській лабораторії
охоронних досліджень пам'яток археології та культури Українського Фонду
культури (м. Київ). А в 1991 — 1993 рр. очолив відділ археології Півдня
України.

Протягом 1989 — 1993 років Маріупольська археологічна експедиція під його
керівництвом провела охоронні розкопки низки курганів Приазовського краю.

У 1993 — 1995 рр. В. К. Кульбака завідував відділом археології Півдня України
кооперативу \enquote{Археолог}, що працював під егідою Інституту археології НАН
України.

З 1997 р. В. К. Кульбака починає працювати старшим викладачем кафедри
історичних дисциплін історичного факультету Маріупольського державного
гуманітарного коледжу, а пізніше — інституту й університету. Експедиція
Маріупольського гуманітарного інституту під його керівництвом завершила в 1998
році розкопки ґрунтового золотоординського могильника, розпочаті в 1987-му.
Тоді ж у мікрорайоні \enquote{Східний} були розкопані два кургани зрубного часу (період
бронзи). А в 1999 році він продовжив розкопки кургану \enquote{Дід}. Тут дослідив
всього три поховання, одне з них — з моделюванням особи, а також парне
поховання епохи бронзи. Археолог очікував сенсацій від цієї потужної
стародавньої архітектурної споруди, розраховував знайти й скіфське золото...

До 2009 року, коли він передчасно пішов з життя Володимир Костянтинович
продовжував працювати в МДУ та активно займатися польовими та лабораторними
дослідженнями археологічної спадщини України.

