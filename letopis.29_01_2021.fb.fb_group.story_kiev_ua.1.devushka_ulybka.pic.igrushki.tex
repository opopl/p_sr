% vim: keymap=russian-jcukenwin
%%beginhead 
 
%%file 29_01_2021.fb.fb_group.story_kiev_ua.1.devushka_ulybka.pic.igrushki
%%parent 29_01_2021.fb.fb_group.story_kiev_ua.1.devushka_ulybka
 
%%url 
 
%%author_id 
%%date 
 
%%tags 
%%title 
 
%%endhead 

\begin{center}
	\begin{fminipage}{0.9\textwidth}
\ifcmt
	tex \begin{center}
  ig https://scontent-frx5-1.xx.fbcdn.net/v/t1.6435-9/143543105_3997312883635622_7546973299362185007_n.jpg?_nc_cat=105&ccb=1-5&_nc_sid=b9115d&_nc_ohc=ifEbYlJuehAAX9rnmjp&_nc_ht=scontent-frx5-1.xx&oh=524966c45050259d21744ea34f5ae8a1&oe=61B3447E
	@width 1.0
  %@wrap \parpic[r]
  %@wrap \InsertBoxR{0}
	tex \end{center}
\fi

\iusr{Ольга Писанко}
И у нас такие сосульки были!!!!

\iusr{Елена Чайко}
И сосульки, и часы живы у нас по сей день

\iusr{Galina Krynitskaya}
И я училась в КИСИ в 1974-1979г, и игрушки у. меня такие есть. Спасибо за воспоминании о молодости, студенчесских годах

\iusr{Ирина Петрова}
\textbf{Galina Krynitskaya} а какой факультет?

\iusr{Galina Krynitskaya}
\textbf{Ирина Петрова} технологичесский ПСИиК

\iusr{Анна Шустерман}
Такие раритеты сохранились у меня по сей день.)

\iusr{Aleks Faershtein}
\textbf{Анна Шустерман} 

У моей сестры в Киеве есть точно такие же игрушки и плюс мордочка клоуна...
Клоун был ёлочной игрушкой моей мамы где-то с 30-х годов прошлого века!!! Самое
интересное в этой игрушке то, что как его не повесь - он всегда отворачивается
затылком....
		
	\end{fminipage}
\end{center}
