% vim: keymap=russian-jcukenwin
%%beginhead 
 
%%file 31_12_2021.stz.news.ua.tyzhden.1.cennosti
%%parent 31_12_2021
 
%%url https://tyzhden.ua/Columns/50/254008
 
%%author_id marinovich_miroslav
%%date 
 
%%tags cennosti,ukraina,obschestvo,spasenie
%%title Нас врятують цінності
 
%%endhead 
\subsection{Нас врятують цінності}
\label{sec:31_12_2021.stz.news.ua.tyzhden.1.cennosti}

\Purl{https://tyzhden.ua/Columns/50/254008}

\ifcmt
 author_begin
   author_id marinovich_miroslav
 author_end
\fi

Матеріал друкованого видання № 53 (736) від 17 грудня 2021 р.

\begin{zznagolos}
Сьогодні Україна опинилася в доволі неоднозначній ситуації гострої
геополітичної конкуренції з Росією
\end{zznagolos}

Часом складається враження, що ми — дві сполучені посудини. Піднесення Москви
означає занепад української держави, а злет Києва прирікає на крах Росію. Наш
північний сусід нині робить усе можливе й неможливе, щоб не дати чаші вкотре
перехилитися на український бік, як це було за часів Київської Русі. Для цього
Росія не гребує методами, які я можу назвати відверто диявольськими. Це брехня,
ненависть і насильство — готовність негайно та без вагань вдатися до агресії за
будь-яких обставин. Ці елементи — три «іпостасі» чистого зла, але саме на них
Російська Федерація зробила ставку. Саме в них тамтешній уряд сліпо бачить усю
державну «велич і могутність». І саме в них рано чи пізно Росія віднайде свій
кінець. Натомість для України нагода перемогти з’явиться тоді, коли ми
максимально віддалимося від понять, за якими живе сусідня країна-агресор. Поки
«вєлікій русскій народ» падатиме в прірву духовної ницості, Україна знайде
порятунок у цінностях. Нині в нашій державі укладений корупційний суспільний
договір. За ним український народ дотримується первинних інстинктів виживання:
корупції, брехні, підлабузництва, кумівства — усього того, що характерне для
найгірших періодів нашої історії. Натомість для країни вже настав час ретельно
придивитися до тих осередків, які визначили цінності розвитку як основу своєї
життєдіяльності. І одним із таких місць є Український католицький університет.
Ми розуміємо: знання, що не спираються на цінності, можуть бути навіть
небезпечними. Погодьтеся, серед членів уряду Януковича було чимало людей із
доброю освітою. Щоправда, їхні знання та вміння лягли на хибну етичну основу,
тож спричинили зло країні. Головне гасло УКУ — «Віра та розум». Розум як
професійні навички та віра як етична основа всього. Коли ці два складники
злагоджено працюють в одній інституції (зокрема в УКУ), тоді суспільство
гармонійно розвивається.

\begin{zznagolos}
Наше суспільство хоче жити в майбутньому, а не женеться за сумнівним і
примарним минулим. Україна прагне миру, свободи й порозуміння, і вже в цьому її
перемога	
\end{zznagolos}

Українська нація звикла жити за патерналістськими інстинктами. Ми без вагань
делегуємо абикому відповідальність навіть за своє життя. А згодом доходимо
висновку: причина всіх наших бід — ті, хто нами керує. Варто пам’ятати: саме ми
відповідальні за наслідки наших рішень. У критичних ситуаціях шукати когось,
щоб зробити його винним у всіх своїх бідах, — це стратегія переможених. Через
це українську молоду націю я можу порівняти з дитиною. Коли малеча тільки
вчиться ходити, ми бачимо, як вона падає, піднімається й знову падає. Хіба
після другого чи третього падіння дитині скажуть: «Зупинися. Далі й не пробуй,
адже все одно не вийде. Скільки ж можна падати!»? Звісно, ми такого не
говоримо, бо розуміємо, що це етап, який має пройти (чи спочатку проповзти)
кожен. Так само й наше суспільство. Дуже довгий час українці були недержавним
народом. Нині ми не маємо характерних для такої форми рефлексів, тож мусимо їх
набути й не в комфортних штучних умовах, а в процесі реальних зіткнень та
реальних випробувань. Звісно, ми могли б бути цілком спокійними за Україну,
якби жили у своєрідному вакуумі та якби тільки наші внутрішні процеси визначали
нинішнє й майбутнє. Але це не так. Тож для перемоги українській нації варто
навчитися не тільки ходити, а й твердо триматися на ногах. Сьогодні Україна не
має такої розкоші, щоб можна було нескінченно довго визрівати — це треба
зробити якомога швидше. Звідси знову й знову постає питання цінностей та
етичної відповідальності за свої дії. Наш університет прагне сформувати
студентів так, щоб вони розуміли причинно-наслідковий зв’язок між персональними
діями та суспільними наслідками, адже саме таких людей потребує держава. Я
люблю повторювати фразу блаженнішого Любомира Гузара, що вразила мене багато
років тому: «Дорогі мої, дайте людям вирости у свободі». І цим реченням сказано
все, через що проходить Україна. Треба зрозуміти: зі стану раба одразу стати
вільною та відповідальною людиною неможливо, адже перший рух раба — свавілля: я
вільний, тож тепер можу робити все, що заманеться. Друга реакція — одразу
когось покарати, вчинити якесь насильство, знайти винного. Натомість у свободі
треба зростати, вчитися бути відповідальним, бо свобода без відповідальності
стає сваволею. Чому ще я не вірю в путінську Росію? Вона зробила ставку на
минуле й хоче повернути форму імперії, яка вже суперечить нинішнім законам
цивілізації. Для утвердження цієї мети Російська Федерація вдається до жахливих
дій. Натомість Україна цього і не хоче, і, на щастя, не робить. Наше
суспільство хоче жити в майбутньому, а не женеться за сумнівним і примарним
минулим. Україна прагне миру, свободи й порозуміння, і вже в цьому її перемога. 
