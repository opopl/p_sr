%%beginhead 
 
%%file 01_03_2023.fb.krutenko_maryna.mariupol.1.shestoi_den_voini_01
%%parent 01_03_2023
 
%%url https://www.facebook.com/marinakrytenko/posts/pfbid0Sxu746uqVrmqFAKPHFmd8kshoSwieFyEzEkeoKrErUQ6Sq6nUCkpNxKdL6zUHaXDl
 
%%author_id krutenko_maryna.mariupol
%%date 01_03_2023
 
%%tags mariupol,mariupol.war,dnevnik
%%title ШЕСТОЙ ДЕНЬ ВОЙНЫ 01.03.22
 
%%endhead 

\subsection{ШЕСТОЙ ДЕНЬ ВОЙНЫ 01.03.22}
\label{sec:01_03_2023.fb.krutenko_maryna.mariupol.1.shestoi_den_voini_01}

\Purl{https://www.facebook.com/marinakrytenko/posts/pfbid0Sxu746uqVrmqFAKPHFmd8kshoSwieFyEzEkeoKrErUQ6Sq6nUCkpNxKdL6zUHaXDl}
\ifcmt
 author_begin
   author_id krutenko_maryna.mariupol
 author_end
\fi

ШЕСТОЙ ДЕНЬ ВОЙНЫ 01.03.22

Утром нас всех попросили уйти из бомбоубежища. Это здание оказалось
стратегически важным.  Периодически отключали свет, но потом опять включали. 

Женя узнал, что наши друзья собираются присоединится к колоне автомобилей,
греческой диаспоры. Греческое посольство, договорились с Россией, о вывозе всех
желающих греков из Сартаны и города. Женя спросил у меня, не хочу ли я уехать.
Сам же муж и сын никуда не собирались уезжать, они были уверены, что они нужны
городу.... Я сказала, что спасаться без них не буду, смысл мне жить, если они
погибнут в Мариуполе. Будем умирать вместе....

Утром «Градами» обстреляли район Бахчиванжи  и Морского училища. 

В обед к нам пришли друзья, мы предложили им переехать к нам. Вместе легче
справляться с новыми проставленными задачами. Они жили на шестом или седьмом
этаже девятиэтажного дома, на центральном проспекте через который заходили
русские войска. Их дом был в зоне риска. 

Нас стало шестеро.  Наша семья: я, Женя и Никита и семья Шишигиных: Сергей,
Ольга и Алина. 

Они побежали домой за вещами, водой и продуктами и все перенесли к нам. 

После обеда, мы с Женей решили поехать на источник, набрать питьевой воды. Мы
поехали не по улице Нахимова, а по другой, параллельной ей. Мы заметили, что
предприятие в котором фильтровалась вода и воду развозили по магазинам,
обстреляли и люди набирали воду. Мы остановились, заняли очередь и два или три
часа простояли. Народ зверел и наглел. Женя взял на себя ответственность за
толпу, построил всех и наладил разлив воды. Позже мы узнали, что источник на
который мы ехали был обстрелян и там лежало куча трупов.

С приходом друзей, нам стало немного веселее. Вечерами мы играли в игры,
шутили, рассказывали какие-то истории и молились. Молитва стала обязательной.
Город оказался в глубоком окружении, с одной стороны Азовское море, с другой
ДНР, а там где раньше была свободная Украина, 150 км оккупации. Город был в
блокаде!!!

Ночевали мы уже дома. Мне кажется эту ночь в кладовке спали только девочки, не
считая Ридика 🐕. 

Продолжение следует.....

%\ii{01_03_2023.fb.krutenko_maryna.mariupol.1.shestoi_den_voini_01.cmt}
