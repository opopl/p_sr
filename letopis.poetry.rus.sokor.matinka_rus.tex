% vim: keymap=russian-jcukenwin
%%beginhead 
 
%%file poetry.rus.sokor.matinka_rus
%%parent poetry.rus.sokor
 
%%url http://maysterni.com/publication.php?id=146529
%%author 
%%tags 
%%title 
 
%%endhead 

\subsubsection{Я, Русь-Україна}
\label{sec:poetry.rus.sokor.matinka_rus}

\Purl{http://maysterni.com/publication.php?id=146529}

Я, Русь-Україна.
Любіть мене, що звана Русь,
Для вас я є одна єдина.
Для вас я Матінка-Русь,
Слов'янська ви моя родина.

Я вам дала природній світ.
Ви вільні, ладили у мирі
Не знали рабський гніт,
Перед богами були щирі.

З богами ви були в путі,
В труді, на полі битви.
А зброя у житті,
Плуг, меч, молитва.

Любіть мене, що звана Русь,
Сад вишневий, квіт калини,
Дорідним ланом я стелюсь
Щоб колосились ниви.

Чому ж на зламі тисяч літ,
Моє ім'я забули.
Забули свій духовний світ,
Мене назвали Україна.

Ви ж обирали вожаків
На вічі всього роду.
Тепер обрали чужаків
Тай втратили свободу.

В чужих руках віки томлюсь,
Покрита вашими ж тілами.
Казалось вільна становлюся,
Та керована не вами.

Святе у вашому житті
Земля, молитва і родина.
Долі наші не легкі,
Коли править чужа людина.

Ваш бувши славний древній рід,
Свята криниця,
Хай в серцях незгасне світ,
З ковтком свяченої водиці.

Я ваша Ненька, Матір-Русь,
Чекаю батька, доньку і сина.
Розквітну я і оновлюсь
В новітній, вільній Русь-Україні.
Червень 2020р.
