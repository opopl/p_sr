% vim: keymap=russian-jcukenwin
%%beginhead 
 
%%file 12_09_2018.stz.news.ua.mrpl_city.1.240_rokiv_legendy_mrpl_1
%%parent 12_09_2018
 
%%url https://mrpl.city/blogs/view/do-240-richchya-mariupolya-legendi-mista-chastina-i
 
%%author_id demidko_olga.mariupol,news.ua.mrpl_city
%%date 
 
%%tags 
%%title До 240-річчя Маріуполя: Легенди міста. Частина I.
 
%%endhead 
 
\subsection{До 240-річчя Маріуполя: Легенди міста. Частина I.}
\label{sec:12_09_2018.stz.news.ua.mrpl_city.1.240_rokiv_legendy_mrpl_1}
 
\Purl{https://mrpl.city/blogs/view/do-240-richchya-mariupolya-legendi-mista-chastina-i}
\ifcmt
 author_begin
   author_id demidko_olga.mariupol,news.ua.mrpl_city
 author_end
\fi

\vspace{0.5cm}
\ii{12_09_2018.stz.news.ua.mrpl_city.1.240_rokiv_legendy_mrpl_1.pic.1}

У кожного міста є власні легенди, які створюють містяни. Вони переходять з
покоління в покоління, зберігаючи власну самобутність, загадковість і
унікальність. Виявилося, що Маріуполь має не тільки багату історію, але й
велику кількість цікавих легенд, які мені вдалося зібрати завдяки статтям
краєзнавців, журналістів та розповідям маріупольців. Пропоную почати з ними
знайомитися. Проте деякі з легенд можуть бути справжнісінькою правдою, а не
вигадкою... Сьогодні ми можемо лише здогадуватися, що ж відбувалося насправді і
як виникла та чи інша легенда про наше місто.

\begin{center}
\textbf{\enquote{Домаха}}
\end{center}

Поки історики сперечаються про те, як виникло і хто побудував поселення в гирлі
Кальміуса Домаху (Адомаху, Адомахію), жителі сучасного Маріуполя, на території
якого і було це поселення, розповідають легенду про кохання лицаря-козака і
красуні Домахи. Запорожець кілька років ніс варту біля сторожового поста на
березі Азовського моря та одного разу, поспішаючи з дорученням до товаришів на
Січ, зустрів дівчину, красиву, як квітуча калина.

З першого погляду закохався юнак в Домаху, і вона відповіла йому взаємністю.
Свою любов запорожець і Домаха пронесли через все життя, тому, коли на місці
сторожового козацького поста запорожці звели фортецю, козак, який на той час
вже став отаманом, назвав фортецю ім'ям своєї коханої.

\textbf{Читайте також:} \emph{Історія і легенди Азовського моря}\footnote{%
Історія і легенди Азовського моря, Ольга Демідко, mrpl.city, 23.06.2018, %
\url{https://mrpl.city/blogs/view/istoriya-i-legendi-azovskogo-morya}%
}\footnote{Internet Archive: \url{https://archive.org/details/23_06_2018.olga_demidko.istoria_i_legendy_azovskogo}}

За іншою легендою, назву козацької фортеці Домаха подарувала шабля-Домаха з
дамаської сталі, з якої козак-запорожець ніколи не розлучався в походах і яка
не один раз врятувала йому життя.

Згідно з ще одним переказом, в цих місцях жила самотня жінка. Будинок її
відокремлювався від Кальміусу широким і глибоким яром. Коли жінці набридло
ходити по настільки незручному шляху до річки, вона вирішила направити воду
Кальміусу в яр. З цією метою вирішила вона перекопати перемичку, що відділяла
річку від яру. Їй це вдалося завдяки наполегливій і довгій праці.

Таким чином у Кальміуса з'явилося два русла при впадінні в Азовське море. Жінку
звали Домаха, таку ж назву отримала й утворена нею протока, а також козацька
паланка.

Ця легенди описані в статті історика Івана Джухи.

\begin{center}
\bfseries\enquote{Маріупольські катакомби}	
\end{center}

\ii{12_09_2018.stz.news.ua.mrpl_city.1.240_rokiv_legendy_mrpl_1.pic.2}

За розповідями старожилів, від центральних вулиць Маріуполя під Міським садом
проходять тунелі, що виходять на Слобідку. За одними чутками, вони існували ще
з козацьких часів, за іншими - вириті перед Кримською війною 1853–1856 років.
Під час Кримської війни й обстрілу міста маріупольці ховалися в цих тунелях.
Але де входи та виходи цих тунелів? У статті журналістки Тетяни Фаустової
міститься інтерв'ю видатного краєзнавця Аркадія Проценка, який поділився своїми
думками стосовно загадкових катакомб. За його словами, тунелів було два. І в
післявоєнні роки він сам особисто, бувши підлітком, проходив по ним. Обидва
мали виходи на Слобідку, а входи – в нинішньому Старому місті. Один з тунелів
дійсно проходив під Міським садом. На жаль, і входи, і виходи в підземні тунелі
були засипані будівельниками та ремонтниками. Довше протримався другий тунель,
вхід в який знаходився в одному з дворів по вулиці Італійській, але і цей вхід
зараз засипаний. Тобто тунелі й сьогодні існують, але входів і виходів в них
немає.

\begin{center}
\bfseries\enquote{Дід-Веселун}
\end{center}

Маріупольці похилого віку розповіли мені дуже добру легенду про діда-Веселуна,
який завдяки яскравому образу запам'ятався багатьом. З'являтися він почав у
1950-ті роки дуже несподівано, завжди з олівцями в руках і чистими аркушами. На
обличчі посмішка, але справжня, жива, щира. Років йому можна було дати від 65
до 70. Добрий дідусь з білою бородою, в красивій жилетці та яскраво-коричневим
метеликом на ній. Під жилеткою виднілася біла сорочка. Іноді у дідуся можна
було побачити капелюшок, правда, головний убір у нього завжди відрізнявся і за
кольором, і за формою. Дідусь серед тих, хто його бачив, отримав прізвисько
\enquote{Веселун} не тільки через свою веселу вдачу, а й ще тому, що він часто гуляв по
лівобережному маріупольському парку \enquote{Веселка}. Він з'являвся серед самотніх
чоловіків чи жінок, бабусь або дідусів, які сумно сиділи та чогось очікували,
малював їхні портрети та писав, що все налагодиться, після чого вручав свій
презент, залишаючи автограф на пам'ять. За словами маріупольчанки Лариси
Петрівни Силової, особисто у неї, і правда, все налагодилося. З'являвся дідусь
не тільки в парку, а й в міському транспорті, на зупинках... Здавалося, він був
скрізь, де на нього чекали, і там, де він був по-справжньому потрібен.

\textbf{Читайте також:} \emph{Мариупольский \enquote{Санта} сменил экскаватор на хулахуп, придумал 10 ярких шляп и устроил шоу}%
\footnote{Мариупольский \enquote{Санта} сменил экскаватор на хулахуп, придумал 10 ярких шляп и устроил шоу, mrpl.city, 11.07.2018, \url{https://mrpl.city/news/view/mariupolskij-santa-smenil-e-kskavator-na-hulahup-pridumal-10-yarkih-shlyap-i-ustroil-shou-foto-plusvideo}}

\begin{center}
\bfseries\enquote{Арка бажання}
\end{center}

Навпроти сучасного ПК \enquote{Молодіжний} знаходиться могутня маріупольська арка
бажання. Прикрашає її маскарон (маска тварини, чи людини, яка має сакральну
функцію) лева. Серед маріупольців було прийнято на цьому місці загадувати
найрізноманітніші бажання і вірити в їх виконання. За легендою завдяки леву всі
бажання завжди здійснювалися, тому це місце є одним з найбільш сакральних,
загадкових і чарівних.

\begin{center}
\textbf{Маріупольський скарб Нестора Махна}
\end{center}

Існує легенда, що батько Махно, після того, як взяв Маріуполь, нібито закопав
тут два (з трьох) ящика білогвардійського золота.

Чи існують ті два ящики, чи ні і де вони закопані, – про це історики та
краєзнавці сперечаються досі. Як, втім, і про те, на якій будівлі в місті
знаходиться балкон, з якого Нестор Махно звертався до жителів Маріуполя в 1919
році. Є версія, що цей балкон знаходиться на будівлі, в якому нині
розташовується редакція газети \enquote{Приазовский рабочий}. Маріупольці досі
шукають загублений скарб, який так і не був ніким знайдений.

Про інші легенди, які зберігає Маріуполь, ви дізнаєтеся згодом. Важливо, що
кожна з них є яскравим і потрібним доповненням до історії нашого міста.

\textbf{Читайте також:} \emph{Мистический Мариуполь: жуткие легенды и загадочные места города}%
\footnote{Мистический Мариуполь: жуткие легенды и загадочные места города, Анастасія Селітріннікова, mrpl.city, 13.04.2018, \url{https://mrpl.city/news/view/misticheskij-mariupol-zhutkie-legendy-i-zagadochnye-mesta-goroda}}
