% vim: keymap=russian-jcukenwin
%%beginhead 
 
%%file 15_02_2023.fb.fb_group.mariupol.pre_war.3.dom_gampera.cmt
%%parent 15_02_2023.fb.fb_group.mariupol.pre_war.3.dom_gampera
 
%%url 
 
%%author_id 
%%date 
 
%%tags 
%%title 
 
%%endhead 

\qqSecCmt

\iusr{Ирина Долгополая-Красовицкая}

\ifcmt
  igc https://i.paste.pics/a40719e4f72b68a3718acf1babafc75d.png
	@width 0.1
\fi

\iusr{Yuliya Varshavskaya}

В этом доме есть что-то волшебное, когда в детстве мимо проходила аж дух
захватывало. Очень надеюсь что он не разрушен.

\begin{itemize} % {
\iusr{Роман Внуков}
\textbf{Yuliya Varshavskaya} он разрушен

\iusr{Natalia Tkachenko}
\textbf{Yuliya Varshavskaya} он разрушен, но не под снос...

\iusr{Yuliya Varshavskaya}
\textbf{Natalia Tkachenko} Спасибо 🙏
\end{itemize} % }

\iusr{Leonid Ehdelshteyn}

Дом Гампера на Гамперском спуске. Все уже знают, что на Гамперском. А раньше,
через одного: Гамбургский спуск, Гамбургский спуск, так ухо, по аналогии с г.
Гамбург, воспринимало. Я жил на 3-й Слободке, поблизости этого дома. А в самом
доме, жил мой одноклассник. Наша школа № 54, в которой мы вместе учились и
которой сильно прилетело во время рашистского нашествия в прошлом году,
находилась на Кронштадтской. Но так получилось, что мы стали соседями на
Слободке, познакомившись и женившись на самых красивых девчонках - выпускницах,
из 37 слободской школы, соответственно 1975 и 1976 года. Он жил у бабушки и
дедушки своей избранницы в Гамперовском доме, а я пройдя дорогами двух лет
шабашек, купил полдома на Донецкой. Мы частенько, встречались с ним, и с
удовольствием попивали, сидя у него в квартире Гамперовского дома, 90 -
градусную эссенцию,используемую для приготовления ситра, родом из ситроцеха,
расположенного напротив дома Гампера и которую легко можно было купить на
проходной.

Ещё, какие воспоминания связывают меня с домом Гампера и с Гамперским спуском?
Однажды, зимой, в 1979 году, возвращаясь с вечерних занятий в автошколе
ДОСААФа, в лёгком подпитии, я подскользнулся на камнях Гамперовского спуска и
проехал стоя на колене какое-то, расстояние, чем стёр всю голубую краску с
колена на новёхоньких фирмовых джинсах, за неделю до того, купленных мною за
200 рублей.

И прийдя домой, гневный и хмельной, разразился разящей наотмашь поэмой, о
затруднительных перспективах спуска на Слободку, зимой, по обледенелым камням
Гамперовского спуска, в то время как раньше, за год до этих щемящих душу
событий, можно было объехать этот спуск на трамвае №4, и который за полгода до
этого сняли, а надо бы вернуть. И на следующий день, сдал всё написанное, в
общественную приёмную газеты \enquote{Приазовский рабочий,} настоятельно посоветовав
её напечатать, а то я ещё и не такое напишу. За давностию лет, в памяти
застряли и сохранились, только эти, наполненные болью от ободранного колена,
строки:

\obeycr
Давным-давно, во время оно
Маршрут четвёртый не мешал,
Никто Слободку, как района,
Права на транспорт не лишал.
Тогда, с работы, на работу,
К вокзалу, летом и зимой,
Скитаться не было заботы,
И транспорт был удобный, свой.
\restorecr

.... дальше было

\iusr{Leonid Ehdelshteyn}

\obeycr
Когда-то здесь лежали рельсы,
Когда-то здесь бежал трамвай!
Теперь отбегался, всё чётко,
Продумано и решено,
Богом забытую Слободку,
Горисполком забыл давно!
\smallskip
.....и ещё
\smallskip
Опошлили всё, разве можно,
Уже который день подряд,
Ухабы Гамбургского спуска,
Зимой и летом штурмовать,
Где даже танк перевернётся,
(Как люди ходят здесь пешком!)
Ведь кроме Гамбургского спуска,
Есть ещё Гамбургский подъём!
\smallskip
... и т. д! ...и в том же духе!
\restorecr

Общественная приёмная и горисполком отреагировали. Письмом. Правда не сразу. Я
предполагаю, что поэма, какое-то время, так месяца четыре, ходила по кабинетам
\enquote{Приазовки}, заглядывала в горисполком, вызывая заслуженное восхищение слогом и
смыслом. И вот прихожу, уже летом, а упал на спуске зимой, с работы, а тут
жена: Тебе письмо! Из горисполкома!

Распечатываю и читаю.

В общем, наобещали, много чего, сделали ещё меньше. Во-первых, оговорились
сразу: трамвай не вернём, спуск асфальтировать и делать вдоль спуска перила -
не будем. Но зато: заасфальтируем ул. им. Розы Люксембург, начиная от
исторических камней Гамбургского спуска до ж/д переезда. И 3-Слободку, она же
улица Донецкая, тоже заасфальтируем, раз вы по ней ходите.

Первое сделали, второе - нет. Зато по бокам нового асфальта по улице Розы
Люксембург, от исторических камней до переезда, прорыли и забетонировали
канавы, по которым знатно подтапливало третью и четвёртую Слободку, и дома и
дворы, во время морских наводнений.

\iusr{Олексій Гаруда}

Він вистояв? Є світлини який він зараз?
