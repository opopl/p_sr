% vim: keymap=russian-jcukenwin
%%beginhead 
 
%%file slova.bolnica
%%parent slova
 
%%url 
 
%%author 
%%author_id 
%%author_url 
 
%%tags 
%%title 
 
%%endhead 
\chapter{Больница}
\label{sec:slova.bolnica}

%%%cit
%%%cit_head
%%%cit_pic
\ifcmt
  pic https://img.strana.ua/img/article/3400/studentka-marharita-iz-10_main.jpeg
\fi
%%%cit_text
\enquote{Я перенервничала в пятницу и в субботу попала в \emph{больницу}. Сейчас на
больничном, нахожусь в стационаре. Сбой анализов на фоне стресса, капали разные
растворы. Сейчас уже полегче, сегодня без капельниц}, - сообщила она.  Девушка
также поделилась новостями о ситуации с увольнением. Как ранее писала \enquote{Страна},
медработнику обещали проблемы в случае отказа от добровольного увольнения
%%%cit_comment
%%%cit_title
\citTitle{Студентка Маргарита из Харькова попала в больницу после нападок из-за языка}, Юлия Корзун, strana.ua, 23.06.2021
%%%endcit

%%%cit
%%%cit_head
%%%cit_pic
%%%cit_text
Статья 49 Конституции гарантирует, что сеть государственных и коммунальных
\emph{лечебных учреждений} не может быть сокращена. И помощь там должна оказываться
бесплатно.  На деле же реформа экс-министра Ульяны Супрун, которую продолжают и
сейчас, предусматривает фактическое закрытие ряда \emph{больниц}, поскольку их
финансирование урезают. В 2020 году начался второй этап этой реформы, когда
начали закрывать \emph{психлечебницы} и \emph{туберкулезные диспансеры}.  В целом, по данным
Госстата, за пять лет количество \emph{больниц} сократилось на треть.  Также в 2017
году Супрун предложила на третьем этапе реформы ввести плату за ряд
государственных услуг в \emph{больницах} - в частности, операций. Цены были заложены
космические - по сто и больше тысяч за операцию. После скандала, прайс из
закона о медреформе вычеркнули
%%%cit_comment
%%%cit_title
\citTitle{День Конституции Украины 28 июня - какие статьи нарушаются сильнее всего}, 
Оксана Малахова; Максим Минин, strana.ua, 28.06.2021
%%%endcit

