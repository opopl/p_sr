% vim: keymap=russian-jcukenwin
%%beginhead 
 
%%file moje.kremlevskie_narrativy.jazyk.skovoroda
%%parent moje.kremlevskie_narrativy.jazyk
 
%%url 
 
%%author_id 
%%date 
 
%%tags 
%%title 
 
%%endhead 

\paragraph{Русский язык - язык Григория Сковороды}
\label{sec:moje.kremlevskie_narrativy.jazyk.skovoroda}

Так вот, знаете, сейчас как раз уже 300 лет нашему
великому философу, и это поразительно, просто поразительно, что язык, на
котором он писал о самых сокровенных вещах - Человеке, Боге, Душе -
систематически оплевывается и унижается. Некоторые недалекие люди говорят, а
иные даже исступлено кричат, испуская потоки желчи по бескрайнему интернету,
что это язык московских оккупантов, говорят, что это московский (москвинский,
мокшанский) язык. Чушь собачья! Русский язык, раз уж вспомнили о Сковороде, это
- прежде всего язык Сковороды, поскольку на 90 процентов или более язык его
произведений совпадает со современным русским языком. Не верите? Ну так
почитайте сами, откройте в оригинале Сад Божественных Песен, глаза ж вам даны
не только для того, чтобы смотреть передачи ТСН, не так ли? Конечно, в его
произведениях есть примеси украинских (малороссийских) слов, и грамматические
конструкции немного старомодны, все-таки писал он более двух столетий назад; но
по факту, язык его произведений - с нашей (Киевской) точки зрения, это самый настоящий русский язык, это уж
будьте уверены. И значит, если унижают русский язык, то по факту унижают язык
Григория Саввича, а значит, это плевок в самого Сковороду.  Задумайтесь об
этом, господа патриоты. Вот, например, песнь 18-ая, красиво, правда?

\raggedcolumns
\begin{multicols}{2} % {
\setlength{\parindent}{0pt}
\obeycr
Ой ты, птичко жолтобоко,
Не клади гнезда высоко!
Клади на зеленой травке,
На молоденькой муравке.
\smallskip
От ястреб над головою
Висит, хочет ухватить,
Вашею живет он кровью,
От, от! кохти он острит!
\smallskip
Стоит явор над горою,
Все кивает головою.
Буйны ветры повевают,
Руки явору ломают.
\smallskip
А вербочки шумят низко,
Волокут мене до сна.
Тут течет поточок близко;
Видно воду аж до дна.
\smallskip
На что ж мне замышляти,
Что в селе родила мати?
Нехай у тех мозок рвется,
Кто высоко в гору дмется,
\smallskip
А я буду себе тихо
Коротати милый век.
\smallskip
Так минет мене все лихо,
Щастлив буду человек.
\restorecr
\end{multicols} % }

