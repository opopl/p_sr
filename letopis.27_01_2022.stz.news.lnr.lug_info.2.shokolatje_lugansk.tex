% vim: keymap=russian-jcukenwin
%%beginhead 
 
%%file 27_01_2022.stz.news.lnr.lug_info.2.shokolatje_lugansk
%%parent 27_01_2022
 
%%url https://lug-info.com/news/luganskij-shokolat-e-otkryla-shkolu-konditerskogo-iskusstva-i-priglashaet-na-obuchenie
 
%%author_id news.lnr.lug_info
%%date 
 
%%tags konditer,lnr,lugansk,obuchenie,shkola,shokolad,donbass
%%title Луганский шоколатье открыла школу кондитерского искусства и приглашает на обучение
 
%%endhead 
 
\subsection{Луганский шоколатье открыла школу кондитерского искусства и приглашает на обучение}
\label{sec:27_01_2022.stz.news.lnr.lug_info.2.shokolatje_lugansk}
 
\Purl{https://lug-info.com/news/luganskij-shokolat-e-otkryla-shkolu-konditerskogo-iskusstva-i-priglashaet-na-obuchenie}
\ifcmt
 author_begin
   author_id news.lnr.lug_info
 author_end
\fi

Луганский кондитер-шоколатье Кристина Самонина открыла в столице Республики
школу кондитерского искусства и приглашает всех желающих обучиться изготовлению
шоколадных изделий. Об этом она рассказала корреспонденту ЛИЦ.

\enquote{У нас нет таких школ, я сама с этим столкнулась, что негде учиться – это
первый момент. Второй момент – это то, что мне не хотелось быть домашним
кондитером, я хотела себе свой угол, чтобы у меня все было красиво. Когда я
решила открыть школу, то сперва начала искать помещение, зарегистрировалась в
налоговой, чтобы вся документация была в норме, завезла оборудование и
необходимые ингредиенты}, - сказала она.

Первый мастер-класс в школе прошел по теме изготовления корпусных конфет. Он
длился два дня с 09:00 до 18:00, его посетили три ученика.

\enquote{Это и теория, и практика. Работы было много, мы учились темперировать шоколад
разными способами, разобрали необходимые инструменты для работы с шоколадом. В
первый день мы подготавливали формы для конфет, занимались окрашиванием,
начинкой – все готовилось с нуля и без каких-либо готовых пюре и так далее}, -
рассказала Самонина.

По окончании первого дня конфеты отправились в холодильник для загустения и
стабилизации, этот процесс занимает минимум 6-8 часов.

\enquote{Во второй день мы закрывали донышки конфет, готовили трюфель, делали
презентацию изделий. Процесс приготовления конфет занимает два дня, сократить
до одного или же нескольких часов его нельзя}, - отметила кондитер.

Она рассказала, что в ходе мастер-класса ученики изготовили 10 видов корпусных
конфет и 2 вида трюфелей с разными начинками.

\enquote{По итогам мастер-класса мы приготовили три килограмма готовых изделий, каждый
из учеников забрал по килограмму конфет}, - сказала Самонина.

Шоколатье подчеркнула, что мастер-классы рассчитаны как на новичков, так и на
профессионалов, а главным условием специалист назвала желание \enquote{научиться делать
шоколад}.

\enquote{Частота проведения мастер-классов будет зависеть от их актуальности, как будут
набираться группы. К 14 февраля будет мастер-класс по изготовлению сладостей ко
Дню всех влюбленных, будет курс к 8 Марта}, - анонсировала она.

Самонина уточнила, что курсы рассчитаны на людей старше 16 лет, но в скором
времени планируется проведение мастер-классов и для детей.

\enquote{Это будут упрощенные курсы, на два-три часа, но обязательным их условием будет
присутствие и участие в них родителей детей}, - добавила кондитер-шоколатье.

Школа кондитерского искусства находится по адресу: Луганск, улица Демехина,
27б. Более подробную информацию о курсах можно получить в группе
кондитера-шоколатье в социальной сети \enquote{ВКонтакте}, в Instagram или по телефону:
(072) 179 79 79. 
