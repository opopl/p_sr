% vim: keymap=russian-jcukenwin
%%beginhead 
 
%%file slova.planeta
%%parent slova
 
%%url 
 
%%author_id 
%%date 
 
%%tags 
%%title 
 
%%endhead 
\chapter{Планета}

%%%cit
%%%cit_head
%%%cit_pic
%%%cit_text
Урбанизация этой \emph{планеты} достигла своего максимума. Вся поверхность Трантора
протяженностью 75 ООО ООО кв. миль представляла собой один громадный город.
Население превышало 40 миллиардов человек. Это огромное количество людей
занималось исключительно административными нуждами планеты и Империи. Тем не
менее их было недостаточно для решения многих насущных проблем. Каждый день
целые флотилии, состоящие из десятков тысяч кораблей, привозили продукты с
сельскохозяйственных \emph{планет} на обеденные столы Трантора...
Зависимость \emph{планеты} от внешних миров не только в сельскохозяйственной,
но и в других областях, связанных с жизненной необходимостью, сделало Трантор
исключительно уязвимым в случае нападения и долгой осады. За последнее
десятилетия императоры хорошо поняли это, подавляя восстание за восстанием, и
вся политика императорского двора свелась к тому, чтобы как-то защитить слабое
место Империи... Галактическая Энциклопедия
%%%cit_comment
%%%cit_title
\citTitle{Основание}, Айзек Азимов
%%%endcit

%%%cit
%%%cit_head
%%%cit_pic
\ifcmt
  tab_begin cols=4
		 pic https://avatars.mds.yandex.net/i?id=dbbd2a10586874e9f0e80170042ffb4e-4432889-images-thumbs&n=13

     pic https://img.strana.news/img/article/3618/v-zhashkove-v-72_main.jpeg

     pic https://upload.wikimedia.org/wikipedia/commons/8/81/%D0%90%D0%BA%D1%82_%D0%B2%D1%96%D0%B4%D0%BD%D0%BE%D0%B2%D0%BB%D0%B5%D0%BD%D0%BD%D1%8F_%D0%A3%D0%BA%D1%80%D0%B0%D1%97%D0%BD%D1%81%D1%8C%D0%BA%D0%BE%D1%97_%D0%94%D0%B5%D1%80%D0%B6%D0%B0%D0%B2%D0%B8_%28%D0%B7_%D0%B0%D0%B2%D1%82%D0%BE%D0%B3%D1%80%D0%B0%D1%84%D0%BE%D0%BC_%D0%AF%D1%80%D0%BE%D1%81%D0%BB%D0%B0%D0%B2%D0%B0_%D0%A1%D1%82%D0%B5%D1%86%D1%8C%D0%BA%D0%B0%29.jpg

		 pic https://avatars.mds.yandex.net/i?id=be2b55fdd5aad0357ab0e5d8580441f9-3537197-images-thumbs&n=13

  tab_end
\fi
%%%cit_text
Просто две новости одного дня.  В Жашкове, Черкасская область, в школьном
классе повесили бумажную свастику - после чего гордо рассказали об этом на
сайте школы.  А в Киеве, возле здания парламента, установили стенд, где
воспроизводится клятва верности нацистам, которую давали в сорок первом вожди
ОУН: "Відновлена Українська Держава буде тісно співдіяти з
Націонал-Соціялістичною Велико-Німеччиною, що під проводом Адольфа Гітлера
творить новий лад в Європі". Увы, это даже не сенсация. Ультраправые символы и
идеологемы превратились для Украины в новую нормальность - на фоне тотального
запрета всего левого. И сейчас страна представляет собой самое большое
коричневое пятно на карте \emph{планеты}
%%%cit_comment
%%%cit_title
\citTitle{Нацистские символы и идеологемы превратились для Украины в новую нормальность / Лента соцсетей / Страна}, 
Андрей Манчук, strana.news, 13.11.2021
%%%endcit
