% vim: keymap=russian-jcukenwin
%%beginhead 
 
%%file 07_09_2021.fb.moskovskije_zapiski.1.vladimir_golicyn_moskva
%%parent 07_09_2021
 
%%url https://www.facebook.com/mskstory/posts/1960668637431214
 
%%author_id moskovskije_zapiski
%%date 
 
%%tags golicyn_vladimir.1847_1932,istoria,moskva,rossia,sssr
%%title Потомок древнего княжеского рода, московский голова Владимир Михайлович Голицын (1847-1932)
 
%%endhead 
 
\subsection{Потомок древнего княжеского рода, московский голова Владимир Михайлович Голицын (1847-1932)}
\label{sec:07_09_2021.fb.moskovskije_zapiski.1.vladimir_golicyn_moskva}
 
\Purl{https://www.facebook.com/mskstory/posts/1960668637431214}
\ifcmt
 author_begin
   author_id moskovskije_zapiski
 author_end
\fi

В это трудно поверить, но холеный усач на портрете кисти Валентина Серова и
худой старик на снимках 1930 годов - один и тот же человек. Потомок древнего
княжеского рода, московский голова Владимир Михайлович Голицын (1847-1932). Его
заслуги перед Москвой можно выразить фразой, сказанной благодарным
современником: «Он принял город с керосиновыми фонарями и водоразборными
фонтанами, а оставил с электрическим освещением, водопроводом и телефоном».

\ii{07_09_2021.fb.moskovskije_zapiski.1.vladimir_golicyn_moskva.pic}

В бытность его руководства в Москве появилась первые очистные сооружения, 50
артезианских скважин с питьевой водой, телефонная станция. При нем московские
власти выкупили у частников права на эксплуатацию конки и заменили ее
электрическим трамваем. Две первые московские электростанции (на Раушской
набережной и Георгиевская), четыре новых ж/д вокзала (Курский, Павелецкий,
Рижский, Савеловский), первый в России хоспис – все это тоже стараниями
Владимира Михайловича. Так что даже скупая на награды Первопрестольная за все
эти заслуги удостоила князя Голицына звания почетный гражданин Москвы. Таких до
революции было лишь 12 человек.

Князь очень гордился тем, что сдержал слово, данное Павлу Третьякову – тот
завещал городу картинную галерею при условии, что вход в нее будет свободным.
Голицын принял галерею на баланс Москвы, из казны выделял немалые средства на
ее содержание и не взял денег ни с одного посетителя. И как же сокрушался
князь, когда после революции вход в национализированную галерею стал платным.

После 1917 года князь не покинул Россию, постоянно переезжал с семьей, часто
арестовывался. Однажды его вызвал московский глава Лев Каменев, чтобы
поговорить как со знатоком городского хозяйства. Каменев проникся симпатией к
старику и даже выдал «охранную грамоту». Правда, когда в опалу попал уже сам
Каменев, Голицыны сожгут этот опасный документ.

В 1929 году Голицыных лишили продовольственных карточек и выслали из Москвы:
сначала на станцию Хлебниково, затем в Загорск, потом в Дмитров, где и сделаны
последние снимки. Бывший князь подрабатывал переводчиком с французского. Умер
Владимир Голицын 29 февраля 1932 года. Ему было 84 года. Похоронили его на
кладбище близ Казанской церкви в селе Подлипичье. Место погоста давно застроено
домами.

\begin{verbatim}
	#МосковскиеЗаписки
\end{verbatim}
