% vim: keymap=russian-jcukenwin
%%beginhead 
 
%%file 04_11_2021.fb.fb_group.story_kiev_ua.3.zhiznetorzhskaja_ploschad_podol
%%parent 04_11_2021
 
%%url https://www.facebook.com/groups/story.kiev.ua/posts/1790476361149157/
 
%%author_id fb_group.story_kiev_ua,opolskaja_nina
%%date 
 
%%tags gorod,istoria,kiev,podol
%%title Житнеторжская площадь на Подоле
 
%%endhead 
 
\subsection{Житнеторжская площадь на Подоле}
\label{sec:04_11_2021.fb.fb_group.story_kiev_ua.3.zhiznetorzhskaja_ploschad_podol}
 
\Purl{https://www.facebook.com/groups/story.kiev.ua/posts/1790476361149157/}
\ifcmt
 author_begin
   author_id fb_group.story_kiev_ua,opolskaja_nina
 author_end
\fi

Житнеторжская площадь на Подоле существует много веков. В летописях она
упоминается как "Торжище" - здесь велась масштабная торговля рожью (житом). Этот
товар был самым популярным. Благодаря близости к Днепру рожь в Киев доставляли
на баржах и продавали на базаре. Торговая площадь занимала целый квартал, а ее
эпицентром служило место, где сейчас расположена автостанция "Подол". Есть
сведения, что в старину сюда наведывались Николай Гоголь, Тарас Шевченко. В
советские времена торговцы перешли под крышу нового крытого рынка, возведённого
в 1980 году. Рядом с житним рынком на ул. Верхний вал, 16 находится небольшое
задание персиково-бордового цвета. 

\ifcmt
  tab_begin cols=3

     pic https://scontent-frt3-1.xx.fbcdn.net/v/t39.30808-6/252082619_1806469609560652_3404724439683395804_n.jpg?_nc_cat=104&ccb=1-5&_nc_sid=b9115d&_nc_ohc=CS5qy1eA_HkAX_E85PV&_nc_ht=scontent-frt3-1.xx&oh=c65fb6487f4778d7c955b292539055a9&oe=618AF3C8

     pic https://scontent-frx5-1.xx.fbcdn.net/v/t39.30808-6/252514538_1806469639560649_1537971963833675509_n.jpg?_nc_cat=111&ccb=1-5&_nc_sid=b9115d&_nc_ohc=8lYU3TGE6vMAX89ipIG&tn=lCYVFeHcTIAFcAzi&_nc_ht=scontent-frx5-1.xx&oh=1e686010f89187562bd2c8bc61865856&oe=618B383A

		 pic https://scontent-frt3-1.xx.fbcdn.net/v/t39.30808-6/253172590_1806469686227311_1111238485996632809_n.jpg?_nc_cat=102&ccb=1-5&_nc_sid=b9115d&_nc_ohc=KD6inwqv7oQAX-dCpA4&tn=lCYVFeHcTIAFcAzi&_nc_ht=scontent-frt3-1.xx&oh=8ca07ab5f3e3928371fb381c0bfd1656&oe=6189651A

  tab_end
\fi

История этого дома связана с купеческим
родом. Известно, что с 1831 года особняк принадлежал купцу Домарацкому, который
спустя некоторое время передал его своим внукам, а они к усадьбе решили сделать
несколько пристроек - так появился двухэтажный особняк по адресу Верхний
вал, 16/4. На первом этаже действовали магазины ,а второй этаж предназначался для
жилья. После реставрации особняка достроили третий этаж, что кардинально изменило
внешний облик дома. Нижний и Верхний вал разделяет бульвар, в старину здесь
протекала речка Глыбочица, которая со временем превратилась в сточную канаву, а
затем была закована а бетон. Рядом появилась одна из автостанций Города
-,,Подол",в одноэтажном розовом домике. Люди выстраивались в длинные
очереди, чтобы купить билет. Сейчас отсюда ходят автобусы, но автостанция потеряла
популярность. Скорее всего, она будет закрыта. 

\ifcmt
  tab_begin cols=4

     pic https://scontent-frt3-2.xx.fbcdn.net/v/t39.30808-6/253107029_1806469749560638_2006514598289222377_n.jpg?_nc_cat=103&ccb=1-5&_nc_sid=b9115d&_nc_ohc=Jl-G0AM_bY0AX-lOenQ&_nc_ht=scontent-frt3-2.xx&oh=8382c84da778247b9a680288ea45bf10&oe=618A52DE

     pic https://scontent-frx5-1.xx.fbcdn.net/v/t39.30808-6/251999745_1806469792893967_2303734731317296538_n.jpg?_nc_cat=100&ccb=1-5&_nc_sid=b9115d&_nc_ohc=8Ca6X9X6qW4AX87zDyI&_nc_ht=scontent-frx5-1.xx&oh=d98c9a6f24bcc8ec7d1fd1e4a2f56d5e&oe=618A8F5C

		 pic https://scontent-frt3-2.xx.fbcdn.net/v/t39.30808-6/252769597_1806469836227296_1170384533887534381_n.jpg?_nc_cat=103&ccb=1-5&_nc_sid=b9115d&_nc_ohc=xsgY8wrDrHsAX-Ky6mv&_nc_ht=scontent-frt3-2.xx&oh=09f886a4548cf9a2364ddfb594a31402&oe=618A7968

		 pic https://scontent-frx5-2.xx.fbcdn.net/v/t39.30808-6/250994290_1806469876227292_3911756521687077925_n.jpg?_nc_cat=109&ccb=1-5&_nc_sid=b9115d&_nc_ohc=z5Hofc-KWMoAX9p7VIo&_nc_ht=scontent-frx5-2.xx&oh=77e1766d9720cf8204b86d0bda8a378c&oe=618A6CEA

  tab_end
\fi

На углу Житнеторжской и Нижнего вала, 17, находится новое пятиэтажное здание с
башенкой под старину, где находятся магазины и офисы. А ранее на этом месте (до
2011 года) находился комплекс красивейших малоэтажных зданий под номерами 17 и
17/6. Они были построены в конце 19 века и принадлежали киевскому купцу Носову.
Известно, что верхняя часть дома была жилой, а нижней располагались магазины и
подвалы ...Жаль, снесли...  Трехэтажный дом на улице Житнеторжской, 8 построен
в 1917 году. Столетний особняк украшает красивейший зубчатый карниз. Сейчас
здание нуждается в капремонте. Как достучаться до власти, чтобы сохранили
историческую ценность... 

Киев начинается с Подола.... Уже в середине 13 века Подол становится главной
частью и культурным центром Киева.... Бродить его улочками... Это большая честь
для нас, живущих в самом красивом Городе на земле! И так хочется сберечь эти
исторические места для будущих поколений!

\ii{04_11_2021.fb.fb_group.story_kiev_ua.3.zhiznetorzhskaja_ploschad_podol.cmt}
