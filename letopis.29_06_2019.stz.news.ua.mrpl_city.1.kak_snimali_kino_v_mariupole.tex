% vim: keymap=russian-jcukenwin
%%beginhead 
 
%%file 29_06_2019.stz.news.ua.mrpl_city.1.kak_snimali_kino_v_mariupole
%%parent 29_06_2019
 
%%url https://mrpl.city/blogs/view/kak-snimali-kino-v-mariupole
 
%%author_id burov_sergij.mariupol,news.ua.mrpl_city
%%date 
 
%%tags 
%%title Как снимали кино в Мариуполе
 
%%endhead 
 
\subsection{Как снимали кино в Мариуполе}
\label{sec:29_06_2019.stz.news.ua.mrpl_city.1.kak_snimali_kino_v_mariupole}
 
\Purl{https://mrpl.city/blogs/view/kak-snimali-kino-v-mariupole}
\ifcmt
 author_begin
   author_id burov_sergij.mariupol,news.ua.mrpl_city
 author_end
\fi

% 1 - Боря Александров в роли Артемки
\ii{29_06_2019.stz.news.ua.mrpl_city.1.kak_snimali_kino_v_mariupole.pic.1}

В 1956 году на экраны страны вышла кинокартина студии \enquote{Ленфильм} \textbf{\enquote{Приключения
Артемки}}. В основу ее сценария, написанного Леонидом Соловьевым, была положена
повесть писателя Ивана Василенко \enquote{Артемка в цирке}. В фильме, как и в повести,
рассказывается о событиях нескольких дней из жизни Артемки,
мальчишки-подмастерья сапожника. Его роль исполнил ленинградский школьник Боря
Александров.

По сюжету  Артемка помогает революционерам распространять листовки, знакомится
и дружит с заезжим борцом Чемберсом Пепсом – в этой роли зрители увидели
политэмигранта из Бразилии Тито Ромалио. Друзьями Артемки становятся клоун
передвижного цирка и его дочка, юная цирковая наездница Леся. Исполнительницей
этой роли была школьница Тамара Ларкина. Режиссер-постановщик фильма — Андрей
Апсолон. Кроме \enquote{Приключений Артемки}, он поставил еще несколько фильмов, но в
историю отечественного кино вошел, прежде всего, как талантливый актер. Одна из
лучших его работ в этом качестве – роль радиста Оси в фильме Сергея Герасимова
\enquote{Семеро смелых}. А еще Андрей Николаевич написал несколько сценариев и тексты
песен для  кинофильмов.

\ii{29_06_2019.stz.news.ua.mrpl_city.1.kak_snimali_kino_v_mariupole.pic.2}

Прямо скажем, фильм \enquote{Приключения Артемки} не стал событием в кинематографе того
времени, а тем более – классикой. Так, рядовая картина. Но для нас,
мариупольцев, фильм интересен тем, что его натурные съемки происходили в нашем
городе. И было это в июне 1955 года. Для съемок ряда сцен авторы фильма
облюбовали небольшой скверик в начале Торговой улицы близ здания, где в то
время находился военкомат. Здесь были построены декорации кафе, в котором
главный герой фильма Артемка по сюжету встречался с борцом Пепсом. Несколько
эпизодов было снято на Подгорной улице. Оператор фильма Дмитрий Месхиев
запечатлел и стадо гусей, и небольшие вросшие в землю хатенки, украшением
которых были снежной белизны стены. Нужно заметить, что их не нужно было
специально белить для съемок. В ту пору стены там всегда были такими.

\ii{29_06_2019.stz.news.ua.mrpl_city.1.kak_snimali_kino_v_mariupole.pic.3}

Съемки происходили и на бывшей базарной площади, занятой позже зданием ДОСААФ,
апелляционным судом и другими строениями. А тогда, в 1955 году, она была пуста,
только где-то в глубине площади сиротливо виднелось здание дореволюционной
гостиницы, где сейчас располагается офис DTЕK. Вот эту пустоту кинематографисты
заполнили огромной массовкой и ларьками, сколоченными на скорую руку, которые
изображали дореволюционный базар. Там было все: и груды арбузов, и корзины с
помидорами, и разносчик лубочных книжек, и дамы под кружевными зонтиками, и
торговка рыбой. А также кухарки, посланные состоятельными горожанами за снедью,
и кобзарь с кобзой, и революционеры, разбрасывающие листовки, и, конечно же,
городовые, гоняющиеся за \enquote{нарушителями порядка}.

\ii{29_06_2019.stz.news.ua.mrpl_city.1.kak_snimali_kino_v_mariupole.pic.4}

В фильме многое мариупольское узнаваемо: здание нашего старого железнодорожного
вокзала, берег моря где-то у поселка Песчаного. Участок Торговой улицы, выше
пересечения ее с улицей Кафайской, художник фильма Семен Малкин \enquote{перенес} из
середины ХХ века в его начало за счет афишных тумб, вывесок с надписями с
соблюдением дореволюционной орфографии.

Рабочие моменты съемок зафиксировал своим фотоаппаратом \enquote{Смена} наш земляк
Георгий Сергеевич Котельников. Он с детства был влюблен в кинематограф, много
читал об этом феномене человеческой культуры. Мало того, будучи студентом
Ждановского металлургического института, своими руками соорудил кинокамеру. И
затем довольно успешно снимал ею любительские фильмы. В свое время были
записаны воспоминания Георгия Сергеевича:

\begin{quote}
\em
\enquote{Когда в наш город приехала съемочная группа \enquote{Ленфильма}, у нас были каникулы.
Я только что  окончил четвертый курс, был совершенно свободен. Узнав, что у нас
в городе  начались съемки, я, прихватив фотоаппарат, отправился на ее поиски.
Нашел. С этого момента я буквально бегал за съемочной группой. И, конечно,
щелкал затвором своего фотоаппарата. Так я увидел, как записывают людей в
массовку. Это происходило во дворе средней школы № 2, тогда она находилась  на
улице Харлампиевской. Там стоял стол, за которым сидело два человека, как я
понял позже, помощники режиссера. И  они записывали всех желающих сниматься в
массовках. Народ толпился возле них.

Был еще такой момент интересный, когда актеры после съемок, которые проходили
возле старого военкомата, уселись на скамейку в скверике, а фотограф киногруппы
должен был их сфотографировать. Один из актеров был чернокожий, который,
очевидно, не хотел, чтобы его снимали посторонние. А я в это время подлез
буквально под руку фотографа киногруппы, нажал на спуск камеры. Но чернокожий
актер, увидев мой фотоаппарат, начал протестовать. Так этот кадр и получился.
Но самое интересное, конечно, на самих съемках было, когда снимали на улице
Подгорной. Там были проложены рельсы, по которым каталась тележка, на тележке
ездил оператор со своей кинокамерой. Рядом стояло несколько огромных
осветительных приборов, хотя был солнечный летний день. Вокруг суетились люди.
Мне удалось снять все это}.
\end{quote}

В фильме снимались, кроме названных Бори Александрова, Тамары Ларкиной и Тито
Ромалио, актеры Сергей Плотников, Пётр Савин, Леонид Галлис, Виталий
Полицеймако, Владимир Муковозов, Олег Жаков, Борис Дмоховский, Тамара Алёшина,
Анатолий Федосеев и другие. Музыку к фильму написал композитор Венедикт Пушков.

\textbf{Читайте также:}

\href{https://mrpl.city/news/view/v-mariupole-v-sapogah-i-myachah-vyrosli-tsvety-fotofakt}{%
В Мариуполе в сапогах и мячах выросли цветы, Яна Іванова, mrpl.city, 26.06.2019}
