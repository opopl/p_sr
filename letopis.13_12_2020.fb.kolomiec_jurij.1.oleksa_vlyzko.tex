% vim: keymap=russian-jcukenwin
%%beginhead 
 
%%file 13_12_2020.fb.kolomiec_jurij.1.oleksa_vlyzko
%%parent 13_12_2020
 
%%url https://www.facebook.com/permalink.php?story_fbid=859632018162442&id=100023469540550
 
%%author Коломієць, Юрій
%%author_id kolomiec_jurij
%%author_url 
 
%%tags 
%%title 14 грудня минає 86 років з дня загибелі (1934) в радянських катівнях українського поета, прозаїка, футуриста Олекси Федоровича Влизька
 
%%endhead 
 
\subsection{14 грудня минає 86 років з дня загибелі (1934) в радянських катівнях українського поета, прозаїка, футуриста Олекси Федоровича Влизька}
\label{sec:13_12_2020.fb.kolomiec_jurij.1.oleksa_vlyzko}
\Purl{https://www.facebook.com/permalink.php?story_fbid=859632018162442&id=100023469540550}
\ifcmt
	author_begin
   author_id kolomiec_jurij
	author_end
\fi

\ifcmt
pic https://scontent-frt3-1.xx.fbcdn.net/v/t1.0-9/130715085_859631788162465_269085949357614372_n.jpg?_nc_cat=109&ccb=2&_nc_sid=730e14&_nc_ohc=an-TqAAYeMAAX_hnkBY&_nc_ht=scontent-frt3-1.xx&oh=bbd3d91f545539dec6934fd4cc22f0e9&oe=600582F7
\fi

14 грудня минає 86 років з дня загибелі (1934) в радянських катівнях
українського поета, прозаїка, футуриста Олекси Федоровича Влизька (17 лютого
1908 р., станція Боровьонка Крестецького повіту Новгородської губернії – Київ).

Життєпис. Народився на станції Боровьонка Крестецького повіту Новгородської
губернії, де його батько служив дяком, псаломщиком. 1917 р. родина переїхала на
батьківщину діда – в село Сигнаївка Звенигородського повіту Київської губернії
(нині – Шполянський район Черкаської області). У 1919 р. – в с. Станіславчик,
далі в с. Лебедин, де у 13 років важко перехворів на скарлатину і втратив слух,
потім, з 1921 р. – знов у Сигнаївці. У 1923 р. разом зі старшим братом,
Олександром, та матір'ю тікає від батька, який почав зловживати спиртним, до
Києва.

Втрата слуху майбутнім письменником компенсувалася вольовим розвитком пам'яті,
начитаністю: книжка стала для хлопця одним з основних джерел формування
художньої й соціальної свідомості. Закінчив мовно-літературний факультет
Київського інституту народної освіти. Одружився з правнучатою небогою
Т.Г.Шевченка Фотіною (Тіною) Красицькою.

1927 р. в пресі з'явилося повідомлення про загибель О.Влизька в Дніпрі. Вістка
справила приголомшливе враження. Чільний тогочасний критик В.Коряк у
центральній харківській газеті «Комуніст» умістив некролог, у якому назвав
юного поета «українським Пушкіним». Однак інформація виявилася звичайнісінькою
містифікацією, що їх так полюбляли футуристи, О.Влизько згодом спростував її
весело-іронічною заявою в пресі та новими книжками.

1928 р. подорожував по Німеччині, згодом – по нагір'ях Паміру.

Арештований 1934 р. Після вбивства С.Кірова, у грудні 1934 р. страчений нарівні
з багатьма діячами української культури, серед яких Г.Косинка, Д.Фальківський,
К.Буревій, Т. та І.Крушельницькі, М.Лебединець, Р.Шевченко, М.Оксамит та ін.
Виїзна сесія Військової колегії Верховного Суду СРСР на закритому засіданні, що
розпочалося 14 грудня 1934 р. в Києві, офіційно ухвалила О.Влизьку смертний
вирок за приналежність до міфічної контрреволюційної терористичної організації.
Реабілітований посмертно 1958 р.

Ю.Лавріненко в антології «Розстріляне відродження» писав: «Влизько-поет – це
тільки обірваний початок. Або, як він сам про себе писав у передмові до «Живу,
працюю!», - «тільки етап і шукання нових форм... До синтетичної рівноваги ще
далеко...» З його жадібністю, темпераментом, естетичним поліморфізмом нелегко
дійти «синтетичної рівноваги» за кілька даних йому долею літ. А все ж елементи
власної синтези в ньому вже починали проявлятися. Я.Савченко писав з приводу
О.Влизькової «Дев'ятої симфонії»: «Я не знаю нічого кращого в українській
поезії останнього десятиліття щодо такої шляхетності думок, такого міцного й
суцільно-пафосного піднесення і, нарешті, такої широти й людяності мислення. Це
тим паче вражає, що Влизькові всього 19 років». Коротенький восьмирічний
літературний шлях О.Влизька позначений динамічними шуканнями, різнорідністю
форм, жанрів, тем. Класицизм, футуризм, «виробнича поезія» й агітка, а над усім
і передусім активний вітаїстичний романтизм.

Творчість. Рано почав писати вірші: спершу російською мовою, поступово зі
словником удосконалював свою українську. У літературу О.Влизька ввів
Б.Антоненко-Давидович, надрукувавши в київському журналі «Глобус» (1925. - №
22) його вірш «Серце на норд». Відтоді Влизько регулярно друкується в газетах
«Комсомолець України», «Культура і побут», «Літературна газета», журналах
«Життя й революція», «Молодняк», «Нова громада», «Червоний шлях». Зблизився з
Михайлем Семенком і українськими футуристами, активно співпрацював у їхньому
органі журналі «Нова генерація». Належав до літературних організацій «Молодняк»
і ВУСПП. У першому номері органу ВУСППу - «Літературній газеті» - було
надруковано напутні рядки з вірша Олекси Влизька «Поетові»: «Не лицемір, поете,
серцем, і не роби із нього шарж, - замало глянути крізь скельця, на бунт,
виспівуючи марш» (1927).

1927 р. вийшла перша збірка поезій О.Влизька «За всіх скажу». Тогочасна критика
напрочуд високо оцінила поетичний дебют молодого автора. Збірку відзначено
премією Наркомосу УСРР на конкурсі з нагоди 10-річчя Жовтня. Попри тяжку
недугу, вірші О.Влизька сповнені бадьорими інтонаціями, поет оспівував екзотику
морської стихії, поривання в небуденність. Був невтомним шукачем образів і
ритміки, суголосних добі національного відродження та революційної романтики.
Автор збірок віршів «За всіх скажу» (1927), «Поезії» (1927), «Hoch,
Deutschland!» (1930), «Живу, працюю!» (1930), «Книга балад» (1930), «Рейс»
(1930), «Моє ударне» (1931), «П'яний корабель» (1933), «Мій друг Дон-Жуан»
(1934), книжки нарисів «Поїзди ідуть на Берлін» (1931).
