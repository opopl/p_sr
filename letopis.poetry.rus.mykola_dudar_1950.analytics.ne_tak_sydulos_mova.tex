% vim: keymap=russian-jcukenwin
%%beginhead 
 
%%file poetry.rus.mykola_dudar_1950.analytics.ne_tak_sydulos_mova
%%parent poetry.rus.mykola_dudar_1950
 
%%url http://maysterni.com/publication.php?id=122326
%%author 
%%tags 
%%title 
 
%%endhead 
\subsubsection{Не так судилось, як хотілось...}
\label{sec:poetry.rus.mykola_dudar_1950.analytics.ne_tak_sydulos_mova}

\Purl{http://maysterni.com/publication.php?id=122326}

Ось тільки не вчіть мене якою мовою мені писати, думати чи спілкуватись… ваша
«мова» ніякого відношення до моєї мови не має… ви спілкуєтесь виключно мовою
окупанта: брехні, жадібності, безкарності, вседозволеності, блюзнірства і
ненажерства. А ще приплюсуйте сюди звірине своє відношення до природи Землі і
Неба, і помножте все на, вами, покалічені долі… Не вчіть яблуню родити яблука.
Бо тільки вчора ви проклинали носіїв німецької, а сьогодні наче жебраки
просите, благаєте милостині у них. Так буде і з російською. Мені набагато
ближче і рідніше любий з вас, котрий своїми вчинками вказав, допоміг, спас
нужденного, чим рідний молодший брат, якого віддірвали, ледь віддірвали від
стакана… І не кажіть, що це хвороба. У світі, у цьому грішному світі, а він
був, є, і буде після любого з нас - кожен вибирає свій тернистий шлях. Мова тут
ні - до - чо - го… Ціль і мета, кожної з мов, має значення: що саме вона несе,
коли оживає в устах того чи іншого горла. Завчасно, щоб не втрапити в пастку,
робіть висновки на короткі і довші перспективи себе і свого оточення.
Обростайте невидимим саркофагом своєї Душі. Вона цього потребує. Вона тут не
надовго. За ваші вчинки, вашого мирського гріховного відповідатиме Вона. Чому
про Неї всі знають, але ніхто і ніколи досі ЇЇ не тримав, і як Вона виглядає -
темінь. Бо якби Творець бажав би - давно відкрив ЇЇ лице. Але знаючи людську
сутність, навіки скрив від людського ока. Чому у любові, на будь якій мові,
один колір - світлий? А всі тирани усіх часів і народів мають одне походження?
Чому? І при чому тут мова? Агов, друзі??? Я простий землянин. Мені б шмат
землі, жінку і для початку батьківське знаряддя, досвід. По необхідності освіта
і подібне. Все інше разом з дружиною ми напрацюємо. Від нас - людям, від них -
нам. І все. Не чіпайте і не заважайте. Але не так судилось, як хотілось. І не
тільки нам і вам. 
