% vim: keymap=russian-jcukenwin
%%beginhead 
 
%%file 08_12_2021.stz.edu.lnr.lgaki.1.vystavka_den_hudozhnika
%%parent 08_12_2021
 
%%url https://lgaki.info/novosti/v-mezhdunarodnyj-den-hudozhnika-v-akademii-matusovskogo-otkrylas-vystavka
 
%%author_id lgaki
%%date 
 
%%tags 
%%title В Международный день художника в Академии Матусовского открылась выставка
 
%%endhead 
\subsection{В Международный день художника в Академии Матусовского открылась выставка}
\label{sec:08_12_2021.stz.edu.lnr.lgaki.1.vystavka_den_hudozhnika}

\Purl{https://lgaki.info/novosti/v-mezhdunarodnyj-den-hudozhnika-v-akademii-matusovskogo-otkrylas-vystavka}
\ifcmt
 author_begin
   author_id lgaki
 author_end
\fi

8 декабря, в Международный день художника в холле Академии Матусовского
открылась выставка работ призеров и победителей Республиканского
фестиваля-конкурса художественного творчества «Осенний вернисаж». Конкурс уже
много лет проводит Академия при поддержке Министерства культуры, спорта и
молодежи ЛНР.

\ii{08_12_2021.stz.edu.lnr.lgaki.1.vystavka_den_hudozhnika.pic.1}

В этом году организаторы приняли 222 заявки от юных живописцев, графиков и
скульпторов со всей Республики.

Открывая вернисаж, заведующий кафедрой станковой живописи, художник Олег
Безуглый пожелал участникам конкурса: «Дерзать! Творить и, конечно, больше
работать!».

\ii{08_12_2021.stz.edu.lnr.lgaki.1.vystavka_den_hudozhnika.pic.2}

Руководитель учебно-методического центра Мария Чепрасова подчеркнула, что
вернисаж – это территория творчества, и поэтому работы юных художников подарят
посетителям выставки заряд праздничного настроения и желание творить!

\ii{08_12_2021.stz.edu.lnr.lgaki.1.vystavka_den_hudozhnika.pic.3}

«Девять детей из нашей школы приняли участие в конкурсе. И у нас 4 призовых
мест: Богдан Дарчук, Илья Нерушенко, Виктория Белова, София Бережняк. Их работы
можно увидеть на выставке. Для ребят участие в конкурсе – это стимул. Подобные
мероприятия развивают воображение и воспитывают волю к победе», – поделилась
впечатлениями преподаватель детской школы искусств города Первомайска
Владислава Боргулёва.

\ii{08_12_2021.stz.edu.lnr.lgaki.1.vystavka_den_hudozhnika.pic.4}

Фото – Марина Машевски.

\ii{08_12_2021.stz.edu.lnr.lgaki.1.vystavka_den_hudozhnika.pic.5}
