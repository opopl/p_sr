% vim: keymap=russian-jcukenwin
%%beginhead 
 
%%file 21_01_2022.stz.news.ua.fromua.1.lnr_dnr_marionetki.1.navischo
%%parent 21_01_2022.stz.news.ua.fromua.1.lnr_dnr_marionetki
 
%%url 
 
%%author_id 
%%date 
 
%%tags 
%%title 
 
%%endhead 

\subsubsection{Навіщо Росії визнавати ДНР і ЛНР}

19 січня депутати російської Думи опублікували проєкт звернення до президента
Росії Володимира Путіна про визнання так званих ДНР і ЛНР. На фоні
геополітичного протистояння з Заходом та концентрації армій РФ поблизу України
така ініціатива – ще один крок гібридної стратегії Кремля. Після провальних
переговорів з США та НАТО, Москва піднімає ставки у грі. Для неї Донбас і його
жителі – звичайний розхідний матеріал для реалізації своїх глобальних цілей.
Заручники та інструмент гібридної війни, яку веде Путін проти України.

У документі йде мова про нібито сформовані «демократичні та легітимні органи
влади» на території ДНР і ЛНР. Хто оцінював рівень демократії та легітимності
на окупованих територіях, автори звернення не вказують. Бо демократії у ДНР і
ЛНР не більше, ніж у Північній Кореї.

\ii{21_01_2022.stz.news.ua.fromua.1.lnr_dnr_marionetki.pic.2}

За задумом ініціаторів, визнання донбаських квазіреспублік \enquote{створить підстави
для забезпечення гарантій безпеки та захисту цих народів від зовнішніх загроз
та реалізації політики геноциду щодо жителів республік}. Про який геноцид йде
мова і хто його організовує, депутати Думи теж не уточнюють. Для них не існує
сотень тисяч проукраїнськи налаштованих жителів Донбасу, які були змушені
покинути свої домівки через страх переслідувань з боку російських бойовиків і
їх пособників.

Після визнання ДНР і ДНР Росія має встановити з ними міждержавні стосунки. Далі
у фантазіях російських законодавців йде мова про мир та безпеку, які нібито
запанують на Донбасі. Звісно, цей «мир та безпека» будуть підтримуватися
завдяки російській армії, яка зможе вже цілком легітимно увійти на територію
«республік». У Москві вступ російських армій назвуть «захистом народу Донбасу
від української загрози, від нацистів і фашистів, від агресивного блоку НАТО».
А далі можливі різні сценарії – від остаточної анексії цих територій до спроб
організувати нову агресію вглиб України. В такому разі ДНР і ЛНР стануть
плацдармом для атаки і постачальником дешевого гарматного м’яса, яке Росія з
задоволенням кине проти України.

