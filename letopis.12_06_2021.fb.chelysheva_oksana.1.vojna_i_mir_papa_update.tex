% vim: keymap=russian-jcukenwin
%%beginhead 
 
%%file 12_06_2021.fb.chelysheva_oksana.1.vojna_i_mir_papa_update
%%parent 12_06_2021
 
%%url https://www.facebook.com/permalink.php?story_fbid=2757638011184009&id=100008135108849
 
%%author 
%%author_id chelysheva_oksana
%%author_url 
 
%%tags chelovek,chtenie,kniga,kniga.vojna_i_mir.tolstoj,kultura,literatura,otec,rusmir
%%title Апдейт: папа перечитывает "Войну и мир"
 
%%endhead 
 
\subsection{Апдейт: папа перечитывает "Войну и мир"}
\label{sec:12_06_2021.fb.chelysheva_oksana.1.vojna_i_mir_papa_update}
\Purl{https://www.facebook.com/permalink.php?story_fbid=2757638011184009&id=100008135108849}
\ifcmt
 author_begin
   author_id chelysheva_oksana
 author_end
\fi

Апдейт: папа перечитывает \enquote{Войну и мир}. 

Сегодня - один из самых важных для меня дней. День рождения папы. Он сегодня
попробует услышать звонок по телефону. Он уже давно ходит с палочкой. Он
продолжает читать. В прошлый раз маму спросила, \enquote{А папа что делает?}
\enquote{Читает \enquote{Анну Каренину}}, - ответила мама.

Папа, Анатолий Пантелеевич Колошманов, жизнью не балован. Война, лагерь,
послевоенный голод, школа киномехаников - да, папа показывал кино! А потом -
тяжёлая работа на ВАЗе. Но папа очень гордился своей рабочей профессией -
шлифовщик с личным клеймом качества. Знаете, у папы уже много лет назад
появилась проблема: он не ощущает веса чашки, потому что привык к тяжёлым
деталям. И глуховат из-за шума на заводе.

Папа и мама сделали всё, чтобы у их дочерей было образование. Мы любим читать,
потому что папа любит читать. Потому что они с мамой делали все, чтобы в доме
были книги, а мы с сестрой не вылазили из библиотек. Отпуск родителей: это
всегда долгая поездка к одной из бабушек. Порой - к обеим: сначала в Суземку, а
потом в Запорожье. Дорога лежала через Москву. Поверьте: никогда не было
такого, чтобы мы бежали с вокзала на вокзал или просто стояли в очереди в
какой-то магазин. Родители всегда вели нас в музей, в какой-то красивый парк.
Если в Тольятти приезжал на гастроли театр, у нас точно были билеты. Так я
увидела спектакли Малого театра и театра Моссовета, танцы Махмуда Эсамбаева и
услышала, как прекрасна скрипка.

С днем рождения, папа. Я очень скоро тебе позвоню.
