% vim: keymap=russian-jcukenwin
%%beginhead 
 
%%file 03_12_2020.news.ua.strana.1.obval_krivoj_rog_shahta
%%parent 03_12_2020
 
%%url https://strana.ua/news/304607-v-krivom-rohe-obval-porody-sluchilsja-na-shakhte-kotoraja-suditsja-s-hornjakami-iz-za-zabastovki.html
 
%%author 
%%author_id 
%%author_url 
 
%%tags 
%%title Обвал породы травмировал рабочего на криворожской шахте, которая судится с горняками из-за забастовки
 
%%endhead 
 
\subsection{Обвал породы травмировал рабочего на криворожской шахте, которая судится с горняками из-за забастовки}
\label{sec:03_12_2020.news.ua.strana.1.obval_krivoj_rog_shahta}
\Purl{https://strana.ua/news/304607-v-krivom-rohe-obval-porody-sluchilsja-na-shakhte-kotoraja-suditsja-s-hornjakami-iz-za-zabastovki.html}

\ifcmt
pic https://strana.ua/img/article/3046/7_main-v1607020229.jpeg
caption ЧП произошло 3 декабря на шахте "Терновская". Фото: Первый городской
\fi

Сегодняшний обвал пород на шахте \enquote{Терновская} в Кривом Роге, при котором тяжело
травмировался горняк, обрастает новыми подробностями.

Как выяснилось, ЧП произошло на шахте, работники которой недавно устроили акцию
протеста и требовали улучшения условий работы. А после забастовки администрация
предприятия подала на горняков в суд. 

Это уже третий несчастный случай на шахте с момента начала суда над шахтерами,
сообщила \enquote{Стране} машинистка подъема на шахте \enquote{Октябрьская} Майя Чабан.

\enquote{Сколько еще несчастий должно произойти, чтобы 
\enquote{большое начальство}
отреагировали на жуткие условия труда. Или деньги превыше совести? Именно об
этом мы и говорили, когда стояли возле комбината, а ребята сидели под землей.
Жизнь человека ничего не стоит?} - возмущается сотрудница шахты. 

По ее словам, на 15 декабря запланировано очередное заседание суда.

\enquote{А теперь мы - шахтеры ждем объяснений адвоката от КЖРК про то, как наши ребята
создавали аварийную ситуацию. Да, и эксперта своего пусть не забудут
пригласить, того, который знает: когда аварийная ситуация в шахте бывает}, -
говорит Майя Чабан. 

Напомним, сегодня на криворожской шахте в результате обвала породы был
травмирован горняк, которого увезли в больницу в тяжелом состоянии.

Также \enquote{Страна} писала о том, что шахтеры в Кривом Роге начали протестовать еще
3 сентября. Они потребовали у руководства повышения зарплат. Уже 8 сентября
около 400 горняков шахт \enquote{Батькивщина}, \enquote{Гвардейская}, \enquote{Терновская} и
\enquote{Октябрьская} объявили о забастовке. Они остались под землей с требованиями
увеличить зарплаты и улучшить условия труда. Забастовку прекратили 16 октября.
