% vim: keymap=russian-jcukenwin
%%beginhead 
 
%%file 18_07_2020.fb.lnr.13
%%parent 18_07_2020
 
%%endhead 

\subsection{Украина продолжает оправдывать звание страны-террориста: срочное заявление от МГБ ДНР}
\url{https://www.facebook.com/groups/LNRGUMO/permalink/2855590391219183/}
  
\vspace{0.5cm}
{\small\LaTeX~section: \verb|18_07_2020.fb.lnr.13| project: \verb|letopis| rootid: \verb|p_saintrussia|}
\vspace{0.5cm}

Сегодня в Министерстве информации ДНР прошла пресс-конференция представителя
Министерства государственной безопасности ДНР Михаила Попова, который раскрыл
ряд подробностей диверсионных и террористических актов, организованных Главным
управлением разведки Министерства обороны Украины на территории Республики Крым
и ДНР.

«9 июля 2020 года в украинских СМИ были приданы огласке аудиозаписи телефонных
разговоров экс-президента Украины Порошенко с бывшим вице-президентом США
Байденом, где американская сторона фактически подвергла обструкции попытки
Киева развязать диверсионную войну в российском Крыму в августе 2016 года.

16 июля 2020 года на ряде интернет-ресурсов были размещены совершенно секретные
документы Главного управления разведки Министерства обороны Украины, которые
раскрывают все обстоятельства неудачной диверсионно-террористической акции в
Крыму 6-8 августа 2016 года.  В связи с этим МГБ ДНР считает важным сделать
официальное заявление.  Материалы ГУР МО Украины, опубликованные в интернете,
являются абсолютно достоверными. МГБ ДНР располагает копиями всех указанных
документов, подлинность которых не вызывает сомнений.

Из этих документов прямо следует, что в начале 2016 года начальник ГУР
Кондратюк дал указание фактически совершить террористический акт на территории
Российской Федерации. Этот теракт был сорван спецслужбами ценой жизни
сотрудника ФСБ.

Документально подтверждено, что этот теракт готовило и пыталось осуществить
подразделение украинской разведки --- Специальный резерв ГУР МОУ, база которого
расположена в городе Лютеж Киевской области.  Кроме того, документально
установлены имена всех причастных к этому преступлению должностных лиц военной
разведки Украины. Более того, эти сотрудники Специального резерва хорошо
известны МГБ ДНР по преступлениям, совершенным в Донбассе», — сказал Михаил
Попов.  Далее пресс-офицер МГБ ДНР подробно рассказал о задачах, стоявших перед
украинскими разведчиками на территории ДНР в 2016-2019 годах, а также о том,
как эта диверсионно-разведывательная деятельность была пресечена благодаря
профессиональной работе МГБ и МВД ДНР.

«Полагаем, что обнародованные документы, разоблачающие преступную сущность
специальных служб Украины, послужат дополнительным доказательством по уголовным
делам, ведущимся в России», — подытожил Михаил Попов.

Авторская съемка военкора команды News Front Катерины Катиной. 
