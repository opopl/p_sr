% vim: keymap=russian-jcukenwin
%%beginhead 
 
%%file 03_11_2018.stz.news.ua.mrpl_city.1.dom_u_dorogi_o_mariupol_dvorce_morjakov
%%parent 03_11_2018
 
%%url https://mrpl.city/blogs/view/dom-u-dorogio-mariupolskom-dvortse-moryakov
 
%%author_id burov_sergij.mariupol,news.ua.mrpl_city
%%date 
 
%%tags 
%%title Дом у дороги - о мариупольском Дворце моряков
 
%%endhead 
 
\subsection{Дом у дороги - о мариупольском Дворце моряков}
\label{sec:03_11_2018.stz.news.ua.mrpl_city.1.dom_u_dorogi_o_mariupol_dvorce_morjakov}
 
\Purl{https://mrpl.city/blogs/view/dom-u-dorogio-mariupolskom-dvortse-moryakov}
\ifcmt
 author_begin
   author_id burov_sergij.mariupol,news.ua.mrpl_city
 author_end
\fi

Это двухэтажное здание стоит на небольшом пригорке у развилки дорог, одна из
которых ведет на поселок Песчаный, а другая сворачивает на улицу Азовскую. С
дороги его не разглядеть: от взоров пассажиров легковых автомобилей и
маршрутных такси заслоняют разросшиеся деревья.

\ii{03_11_2018.stz.news.ua.mrpl_city.1.dom_u_dorogi_o_mariupol_dvorce_morjakov.pic.1}

Если не полениться и подойти поближе, можно рассмотреть здание подробнее. К
нему ведет лестница, которая оканчивается балюстрадой перед фасадом.
Центральная часть фасада украшена колоннами и аттиком с полуциркульными
проемами. Слева и справа от центральной части находятся довольно обширные
балконы. Всемогущественное время и бурная эпоха перемен в нашей стране сильно
потрепали произведение довоенных зодчих. Нынешние владельцы здания, а
принадлежит оно будто бы теперь ООО \enquote{Торгмортранс АМП}, пытались вернуть зданию
первоначальный вид, делали ремонт и окраску фасадов, но соседство с угольными
причалами порта все усилия свели на нет. Видимо, архитекторы, выбирая место для
его постройки, не учли последствия от близкого соседства с портом. И все же дом
у дороги и сейчас по-своему красив.

\textbf{Читайте также:} 

\href{https://mrpl.city/news/view/odna-iz-mariupolskih-ulits-budet-nosit-imya-olimpijskogo-chempiona}{Одна из мариупольских улиц будет носить имя олимпийского чемпиона, Анастасія Папуш, mrpl.city, 02.11.2018}

От старожилов Приморского района удалось узнать, что он был построен до войны
для Дворца моряков, который был открыт в 1940 году. Сохранился фотоснимок тех
лет, который сделал фотолюбитель, впоследствии известный скульптор Иван
Савельевич Баранников. Помогли найти это редчайшее изображение Дворца
педагоги-пенсионеры Нелли Петровна и Николай Федорович Савранские. А Нелли
Петровна еще и поведала, что было внутри дома с колоннами. Вот ее рассказ:
\emph{\enquote{Довоенное время, начиная с тридцать седьмого года, я помню хорошо, тогда я
пошла в первый класс 32-й школы. Помню и Дворец моряков. Это был шикарный
дворец. Плоская крыша, на ней была устроена танцплощадка. Каждый вечер,
особенно в выходные и праздничные дни, там было полным-полно молодежи. Играл
духовой оркестр, молодежь танцевала. Внутри был довольно большой наклонный
зрительный зал со сценой. Во Дворце были всевозможные кружки для молодежи и
учащейся детворы: спортивный, художественно-литературный, струнный оркестр,
хор. Я посещала художественно-литературный кружок. Мы ставили спектакли,
литературные монтажи к праздникам. Что поразило нас, детей, когда мы впервые
попали в зал, где проводил спевку детский хор? Открыли перед нами дверь, и мы в
изумлении остановились: посреди зала стоял огромный белый лакированный рояль.
Говорили, что его купили в Америке и привезли морем в Мариуполь. Было несколько
комнат отдыха, в них около диванов лежали шкуры белых медведей. Тогда
директором Дворца был Донской. Фамилию прекрасно помню, а вот имя и отчество, к
сожалению, забылись}}.

Довелось услышать и совершенно невероятную историю, связанную с бывшим Дворцом
моряков. Будто он был построен по личному распоряжению И.В. Сталина, мало того,
вроде бы вождь выступал с балкона Дворца при его открытии. Конечно, это
чистейшей воды вымысел. Но вот то, что высокие гости из Москвы были здесь –
факт, подтвержденный несколькими свидетелями этого события. Это были Герои
Советского Союза океанограф и гидробиолог Петр Ширшов и радист Эрнст Кренкель -
двое из четверки советских полярников, которые с 21 мая 1937 года по 19 февраля
1938 года дрейфовали на льдине с научно-исследовательской станцией \enquote{Северный
полюс} по Северному Ледовитому океану. В предвоенные годы отважные
исследователи Арктики Иван Папанин, Евгений Федоров и названные выше Ширшов и
Кренкель были поистине народными героями. Не случайно рабочие порта, жители
припортовых поселков с воодушевлением встретили Петра Петровича и Эрнста
Теодоровича, а затем, затаив дыхание, слушали их выступления.

\textbf{Читайте также:} 

\href{https://mrpl.city/news/view/fotografii-knigi-tetrapody-iz-mariupolya-stali-chastyu-blagotvoritelnoj-aktsii-v-kieve-foto}{%
Фотографии, книги, тетраподы из Мариуполя стали частью благотворительной акции в Киеве, Ігор Романов, mrpl.city, 02.11.2018}

С началом Великой Отечественной войны деятельность Дворца моряков
прекратилась, а в его помещениях разместился флагманский пункт (штаб) Азовской
военной флотилии, которая была сформирована 20 июля 1941 года в составе
Черноморского флота. Командующим флотилией был назначен капитан I ранга А. П.
Александров, военкомом – бригадный комиссар А. Д. Рощин.

В первых числах октября 1941 года здание опустело. Последние части Красной
армии покинули Мариуполь. Передовые подразделения немецких войск уже захватили
Ильичевский район. Их разведчики-мотоциклисты достигли центра города, но до
поселка Песчаный не дошли. Воспользовавшись моментом, местные жители, кто
пошустрее, ринулись тащить из Дворца все, что попадало под руки. Но вскоре
оккупанты пришли и в эти места. Их представители под угрозой сурового наказания
потребовали немедленно сдать награбленное в комендатуру. О том, как был
исполнен приказ, ничего неизвестно. В книге В. М. Зиновьевой \enquote{Чтобы жизнь
продолжалась. Приазовье в период оккупации. 1941 – 1943} сообщается, что
гитлеровцы устроили в очаге культуры моряков публичный дом. В сентябре 1943
года оккупанты, покидая город под ударами соединений и частей Красной армии и
морских десантников, сожгли более половины городских строений, в том числе и
Дворец моряков. Погорелка долго стояла одиноко в окружении одноэтажных домиков.

Лишь в самом начале шестидесятых годов ХХ века здание было восстановлено и
одновременно перепланировано внутри. В его комнаты, пахнущие краской, вселилась
контора \enquote{Торгмортранса} - торговой организации, одной из лучших в нашем городе.
В период повального дефицита в его магазинах и на базах можно было даже
человеку, далекому от торгового флота, \enquote{достать} те товары, которые никогда не
бывали в обычных торговых точках. Но не для этого, конечно, была предназначена
эта организация. Ее главная задача состояла в том, чтобы обеспечить
продовольственными и промышленными товарами первой необходимости экипажи судов
Азовского морского пароходства, а также работников судоремонтных заводов и
портов Мариуполя, Таганрога, Бердянска и Керчи. Главная контора \enquote{Торгмортранса}
находилась в том здании, о котором здесь идет разговор. Наряду с ней в нем
размещались в разное время плавлавка, магазин, столовая.

\textbf{Читайте также:}

\href{https://mrpl.city/news/view/310-let-so-dnya-baturinskoj-tragedii-mariupol-chtit-pamyat-pogibshih}{%
310 лет со дня Батуринской трагедии. Мариуполь чтит память погибших, Анастасія Папуш, mrpl.city, 02.11.2018}

Пока процветало Азовское морское пароходство, и торговая организация со звучным
названием \enquote{Торгмортранс} крепко стояла на ногах. Но потом пароходство
обанкротилось, и торговая организация потеряла своего главного клиента. Часть
помещений пришлось сдать в аренду различным фирмам. Они, судя по всему, не
очень-то пекутся о внешнем виде своего местопребывания. Все же хочется верить:
рано или поздно историческое здание обретет первоначальный вид.
