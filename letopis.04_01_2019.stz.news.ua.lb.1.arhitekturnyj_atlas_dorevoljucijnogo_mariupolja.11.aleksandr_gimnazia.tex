% vim: keymap=russian-jcukenwin
%%beginhead 
 
%%file 04_01_2019.stz.news.ua.lb.1.arhitekturnyj_atlas_dorevoljucijnogo_mariupolja.11.aleksandr_gimnazia
%%parent 04_01_2019.stz.news.ua.lb.1.arhitekturnyj_atlas_dorevoljucijnogo_mariupolja
 
%%url 
 
%%author_id 
%%date 
 
%%tags 
%%title 
 
%%endhead 

\subsubsection{Александровская гимназия}

Здание Александровской гимназии для Мариуполя является одним из наиболее
уникальных и значительных в архитектурном отношении. Его главная особенность
заключается в наличии арочного входного портала, аналогов которому в Мариуполе
больше нет. Смета на строительство нового корпуса Александровской гимназии была
утверждена в 1895 году, а в октябре 1899 года состоялось его торжественное
освящение. Автором проекта стал академик Н. К. Толвинский, правда, несколько
поправок внес городской архитектор Бер. Которого, к слову, позже сняли с
должности, так как, по определению современников, постройки его были ужасны. В
1919 году гимназию закрыли. В послереволюционное время в ее здании
располагались рабфак и совпартшкола. Во время оккупации корпус был сильно
разрушен, а в 50-х годах восстановлен и занят Индустриальным техникумом.
Сегодня в здании проводит занятия колледж Технического Университета. Так что в
определенном смысле работа гимназии продолжается. Внешний фасад благодаря
отсутствию отделки кажется неплохо сохранившимся, но при ближайшем рассмотрении
заметны повсеместные разрушения элементов кирпичного декора. Состояние этой
архитектурной памятки, не плачевное, но и далеко не выставочное.

\ii{04_01_2019.stz.news.ua.lb.1.arhitekturnyj_atlas_dorevoljucijnogo_mariupolja.11.aleksandr_gimnazia.pic.1}
\ii{04_01_2019.stz.news.ua.lb.1.arhitekturnyj_atlas_dorevoljucijnogo_mariupolja.11.aleksandr_gimnazia.pic.2}
\ii{04_01_2019.stz.news.ua.lb.1.arhitekturnyj_atlas_dorevoljucijnogo_mariupolja.11.aleksandr_gimnazia.pic.3}
