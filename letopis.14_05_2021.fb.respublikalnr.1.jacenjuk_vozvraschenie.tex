% vim: keymap=russian-jcukenwin
%%beginhead 
 
%%file 14_05_2021.fb.respublikalnr.1.jacenjuk_vozvraschenie
%%parent 14_05_2021
 
%%url https://www.facebook.com/groups/respublikalnr/permalink/812261332742993/
 
%%author 
%%author_id 
%%author_url 
 
%%tags 
%%title 
 
%%endhead 
\subsection{ГАЗЕТА \enquote{РЕСПУБЛИКА} (№19, 2021г).  ВОТ ТЕ РАЗ. «КУЛЯВЛОБ» ГОТОВ ВЕРНУТЬСЯ?}
\label{sec:14_05_2021.fb.respublikalnr.1.jacenjuk_vozvraschenie}
\Purl{https://www.facebook.com/groups/respublikalnr/permalink/812261332742993/}

Надо сказать, что череда майских праздников в «незалежной» прошла не скучно. А
началось все с 1 000 долларов за рэпера Басту. Украинские радикалы ловили
Василия Вакуленко по всему Киеву. История банальная: одесский «правосек»
Стерненко должен по замыслу президента Зеленского стать самым главным
украинским националистом, то есть – подмять под себя всё украинское радикальное
движение. А для такой масштабной задачи нужен громкий пиар. Использовать «в
тёмную» имя российского исполнителя – не такая уж и глупая затея. Медийный
интерес, да и «хайп» в социальных сетях обеспечены.


\ifcmt
  pic https://scontent-bos3-1.xx.fbcdn.net/v/t1.6435-9/186324567_133257908854317_8847975362751872454_n.jpg?_nc_cat=110&ccb=1-3&_nc_sid=825194&_nc_ohc=YZGfMAR9pZMAX8Wg8R1&_nc_ht=scontent-bos3-1.xx&oh=f1430e35eade23bb3d518b8367e0b1d7&oe=60C4F61A
\fi


Разогнанная Стерненко информационная волна – это спланированная операция.
Выглядела она так: у киевского ресторана, который должен был посетить Баста,
находился в «засаде» провокатор и те, кто должен был запечатлеть провокацию на
видео. Некий «активист» стал задавать вопросы Василию Вакуленко о Крыме и
обвинять в том, что рэпер посещает якобы «оккупированный» Россией полуостров.
Разумеется, охрана исполнителя вытолкала агрессивного провокатора из ресторана.
Но главная задача была выполнена: эмоциональное видео оказалось в руках
Стерненко, который немедленно объявил «охоту на Басту».

По социальным сетям распространили обещание выплатить от 200 до 1 000 долларов
за любую информацию о местонахождении музыканта. Активист Синицын рассказал,
что 3 мая Вакуленко чуть не поймали на Крещатике – типичный приём для
разжигания интереса околофутбольной тусовки к дармовым деньгам, которые могут
оказаться в кармане ультрас. Что и говорить – карманные радикалы Зеленского
смогли раскачать молодняк на «движ» по поиску Басты. Не нужно забывать, что в
команде Зели работают профессиональные пиарщики и сценаристы. Их фантазии
хватит на многое, лишь бы только удержать своего патрона у власти. Но как
говорится в поговорке – телята с дубом не бодаются. Шансов на единоличное
управление своим шутом у украинских вкладчиков в проект нет.

Инсайдеры с Банковой говорят о плане по досрочным выборам президента, на
которых собирается победить Зеленский. Как это будет реализовано технически –
ясности нет. Ярлык на гетьманство от Вашингтонского обкома он не получил, да и
с Лондонским генеральным комитетом у Зеленского в последнее время не ладится. А
тут ещё Виктория Нуланд в Киев пожаловала: заместитель госсекретаря
сопровождала в поездке на Украину Энтони Блинкена. Печенька и Блинчик прибыли
расставлять нужные акценты и нужных Вашингтону людей. Пока Блинкен проводил
официальные встречи, Нуланд занималась закрытыми переговорами. 

Главная смотрящая за Украиной от глобалистов намерена хорошенько перетрясти
кабмин. Её главная цель – поставить на должность премьер-министра одну из своих
кандидатур. Дип стэйт устроят и Яценюк, и Яресько. Если удастся протолкнуть
родственника Нуланд, Наталье Яресько достанется должность первого вице-премьера
и министра финансов. А если даже сверхусилий не хватит на продвижение Яценюка,
премьером станет Яресько. 

Олигарх Ахметов уже потирает руки в предвкушении очень больших барышей.
Предусмотрительный Ринат Леонидович сильно «подружился» с Яценюком, понимая его
перспективность в украинских политических проектах США. Раздавательница
американских «печенек» родом с Буковины, как и Терезия Яценюк. Обе дамы
являются представительницами богатых еврейских кланов – Гур и Нудельман,
состоящих в дальнем родстве. Ахметов давно приятельствует с Викторией Нуланд и
тепло относится к семейству Яценюков. В общем – очередной джекпот у олигарха! 

Порошенко отлично отработал на карман Ахметова, теперь пришла очередь и
Зеленского. Тем более что «кнопка включения – выключения» игрушечного клоуна
теперь полностью в его руках. Сейчас Ахметов очень умело убирает из окружения
Зеленского всех тех, кто имеет на президента влияние.

Мальчика будут «танцевать» только те, кто когда-то заплатил за майдан в Киеве.
И кстати – непотопляемый Арсен снова с золотой акцией. Ему нужно «помочь»
своим американским друзьям немножечко «прогнуть» через колено верховную раду.
Яресько устраивает американцев, но всё-таки в приличных еврейских семьях
принято рассаживать на денежные посты родню.

\subsubsection{НАПОМНИМ}

В 2014 году премьер Арсений Яценюк заявил о строительстве «Стены», которая
должна защитить Украину от возможного вторжения России. Она должна была
обойтись бюджету в 4 млрд грн.

«Стену» планировали завершить в 2018 году, но из-за недостаточного
финансирования перенесли сроки на 2025 год. Пока работы выполнены как максимум
на 30\%. На строительство потратили уже более 2 млрд грн, ежегодно выделяя по
400–500 миллионов. Но уже в 2016 году стало известно, что объемы работ очень
низкие. Было сделано только 12\% от запланированного. И только в одной области
– Харьковской. А когда показали саму «Стену» – это стало поводом для
многочисленных шуток.

Вместо «Европейского вала» страна увидела небольшой и тонкий железный забор с
колючей проволокой наверху. Ограждение было похоже на забор дачного участка и
явно не могло служить препятствием для российских танков. Пиарился на этом
сооружении также однопартиец Яценюка – Арсен Аваков. Который, к слову, и
отвечает за весь проект от его начала и до сих пор – поскольку Госпогранслужба,
обустраивающая «Европейский вал», входит в структуру МВД.

Игорь МИРТОВСКИЙ

ГАЗЕТА \enquote{РЕСПУБЛИКА} (№19, 2021г).

\verb|#газета #республика #политика_Украины #возвращение_Яценюка|
