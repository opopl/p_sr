% vim: keymap=russian-jcukenwin
%%beginhead 
 
%%file 13_02_2022.fb.druzenko_gennadiy.1.chy_pochnetsja_velyka_vijna
%%parent 13_02_2022
 
%%url https://www.facebook.com/gennadiy.druzenko/posts/10158760816343412
 
%%author_id druzenko_gennadiy
%%date 
 
%%tags napadenie,rossia,ugroza,ukraina
%%title ЧИ ПОЧНЕТЬСЯ ВЕЛИКА ВІЙНА?
 
%%endhead 
 
\subsection{ЧИ ПОЧНЕТЬСЯ ВЕЛИКА ВІЙНА?}
\label{sec:13_02_2022.fb.druzenko_gennadiy.1.chy_pochnetsja_velyka_vijna}
 
\Purl{https://www.facebook.com/gennadiy.druzenko/posts/10158760816343412}
\ifcmt
 author_begin
   author_id druzenko_gennadiy
 author_end
\fi

ЧИ ПОЧНЕТЬСЯ ВЕЛИКА ВІЙНА?

Суперечки навколо цього питання втратили будь-який сенс. Як і на виборах,
позицію абсолютної більшості моїх співгромадян формують не раціональні
аргументи, тверезі оцінки чи логічні висновки. Її формує підсвідоме бажання
вірити чи не вірити у велику війну. А віра, як відомо, майже не піддається
раціональному доведенню, спростуванню чи корегуванню.

\ii{13_02_2022.fb.druzenko_gennadiy.1.chy_pochnetsja_velyka_vijna.pic.1}

Врешті-решт, ніхто у світі наразі не знає, чи віддасть Путін наказ наступати чи
ні. Навіть, якщо сили та засоби, накопичені біля українських кордонів,
здаватимуться йому достатніми для бліцкригу.

Своє суто субʼєктивне відчуття моменту я описав вчора. А сьогодні знайшов
просто геніальну фразу, яка максимально влучно формулює те, що я намагаюсь
донести до своєї авдиторії:

\enquote{Якщо війна чомусь і вчить, то лише тому, що завжди слід бути готовим
зустрітись з неочікуваним та зіткнутися з немислимим} (© Christopher Fowler,
сучасний британський письменник).

Мене в нинішній ситуації стратегічної невизначеності тішить інша річ. Якщо
початок війни залежить виключно від Путіна, то її результат від нього майже не
залежить. Припускаю, якщо наважиться, він може узяти Київ і навіть посадити на
Печерських пагорбах маріонетковий уряд. Припускаю, що цей уряд може бути навіть
формально легітимним. 

Але це нічого не змінить, бо наше головне прокляття – взаємне відчуження влади
та посполитих – перетвориться на нашу головну перевагу. Українці НЕ визнають
будь-який колабораціоністський уряд за підставово легітимний. І не стануть йому
коритися. 

Будь-який умовний Мінськ-3 – це розірвання суспільного договору, згідно з яким
українці терплять корумповану, непрофесійну, крадійкувату та хамовити
центральну владу як доконечне зло. Бо краще державність з такою владою, аніж
руїна. Втім щойно українська влада укладе своє Комп'єнське перемир'я (1940),
вона вибʼє стілець легітимності з-під власних ніг і стане частиною окупаційної
влади.

Повторюсь, я не пророк і не ясновидець, і тому не знаю напевно, чи трапиться
велика війна між Україною та Росією. Але якщо трапиться, вона має всі шанси
стати початком кінця ресентиметної недоімперії з центром у Москві, але також –
початком великого відродження нашої землі. Це будуть перейми нової України, які
завжди, перефразовуючи Черчилля, \enquote{піт, сльози, кров та гарування}.  

Тож, готуємось \enquote{зустрітись з неочікуваним та зіткнутися з немислимим}. Бути
готовим завжди краще.
