% vim: keymap=russian-jcukenwin
%%beginhead 
 
%%file 16_10_2020.fb.yura_zaika.1.ukraine_rerih_parhomivka
%%parent 16_10_2020
%%author zaika iurii
%%url https://www.facebook.com/zaika.iurii/posts/4070540452962459
%%tags rerih,cerkva,parhomivka,ukraine,pokrova

%%endhead 

\subsection{Пархомівка - Церква - Реріх - автор Юра Заїка}

\url{https://www.facebook.com/zaika.iurii/posts/4070540452962459}

На жаль, для багатьох співвітчизників Україна-terra incognita.  Така собі
замурзана Попелюшка, що всім "должна"...

Неподалік райцентру Володарка, за 120 кілометрів від  Києва, між річечкою
Тарган, і якимось безіменним струмком, розкинулося звичайне українське село
Пархомівка. 

Звичайне, глухе, хоч і старе(16 ст.) українське село.

Коли їхати з Києва, на в'їзді у Володарку, буде круговий рух-їдьте по колу
наліво, між дво-, і триповерховим будинками, вулицею далі, попадете в село.

Наприкінці 19 століття, село дісталося інженеру-промисловцю Віктору Голубєву,
який побудував тут школу, лікарню, бібліотеку, училище ремісниче...і не встиг
побудувати цегельний завод...бо помер у 1903-му.

Там його і поховали, біля церкви, як він і заповідав.

І було у нього два сини, Віктор-молодший, і Лев. В молодшого Віктора були
приятелі-Володимир Покровський, архітектор, і дивний мандрівний філософ,
художник Микола, який захоплювався якимись далекими незрозумілими горами,
міфами, якоюсь сакральщиною. 

От ця трійця друзів задумала побудувати храм, як заповідав Віктор-старший. Храм
мав бути символом релігійного поєднання людства.

Покровський побудував Храм з капличкою, огорожу, хату священника, в'їзні
ворота...церкву довго не хотіли освячувати, бо та була не така(не так!-давня
наша звичка), як канони пишуть церковні. В церкві для зручності навіть були
встановлені лави, що було нечуваним порушенням. Зрештою, незвичайний храм був
урочисто освячений в 1907 році митрополитом Київським і Галицьким Флавіаном.

Поєднання готики, і східних культур, паркетна підлога, змії, пшениця, та
свастика на стінах, мозаїка...  Храм і досі унікальний! Дзвіниця-50 метрів
заввишки! 

Оздобленням храму займався мандрівний філософ Микола, який намалював аж 12
ескізів, з яких було втілено в життя всього два-мозаїки “Покров Богоматері”, і
“Спас Нерукотворний”, смальту для яких виготовив один з найкращих майстрів того
часу - В.А.Фролов.  

А потім сюди прилізли "пєвуни"-швондєри, і шарикови...і в 1929 храм закрили.

У Храмі був і склад міндобрив, і вівчарня, і клуб...огорожу розібрали, щось
покрали, малі серуни кидали камінням по мозаїках, змагаючись, хто більше виб'є
кольорових камінчиків. Якась п'янь стріляла по мозаїках...пішли тріщини...

Храм із 1979 року-пам'ятник національного значення, з 1982 почалися
реставраційні роботи...відновили куполи, дах...одну мозаїку...

Храм, за рівнем акустики-один із найкращих в Європі. У стіни вмуровані
керамічні посудини-голосники, такі технології відомі в давньоруській
архітектурі ще з Х століття, і прийшли в Київську Русь, найімовірніше, з
Візантії.

На щастя, у 1987 році, в Москві, проходив фестиваль Індії, на якому був
присутній Святослав, син мандрівного філософа Миколи, того, що малював ескізи.
Віктор Геннадійович Киркевич – український журналіст і краєзнавець, був на
зустрічі, з Святославом, і розказував, що той, взнавши про долю храму,
зателефонував Михайлу Міченому, на прізвисько Горбачов...

І здійснявся ґвалт! Найкращі фахівці, реставратори, купа виділених
грошей-почалася реставрація Храму. Всі чекали на високих гостей.

А вони не приїхали.

То була єдина реставрація... яку так і не закінчили. Встигли тільки поновити
Покров Пресвятої Богородиці. Спас так і залишився стріляним, і неосвоєним...

Мозаїки мандрівного філософа Миколи, збереглися лише в двох(4-в колишньому
Союзі) місцях України - тут, у селі Пархомівка, і в Троїцькому соборі Почаївської
Лаври (Тернопільська область) - мозаїка "Спас і князі Святі".

Над головним входом Храму Покрови, в Пархомівці, унікальна мозаїка "Покров
Пресвятої Богородиці". Над ґанком притвору Святого Віктора - панно "Спас
Нерукотворний з вибраними святими": перед образом Спаса стоять на колінах
чотири фігури - Богородиця, і апостол Іоанн Богослов справа, сотник Лонгин
каппадокійський, і Марія Магдалина – зліва. 

В правому куті мозаїки видно підпис "Нік. Реріх".  Прізвище мандрівного
філософа Миколи, і його сина Святослава - Реріх.  Ось така, малята, х... такий от
Колізей, з Бранденбурзькими воротами, в Попелюшки, під боком.

Е, якось воно та й буде! - писав Остап Вишня.

\ifcmt
pic https://scontent.fiev21-1.fna.fbcdn.net/v/t1.0-9/121803112_4070517722964732_7025207717740055152_n.jpg?_nc_cat=108&_nc_sid=730e14&_nc_ohc=G8r8xv77BtoAX-vGArf&_nc_ht=scontent.fiev21-1.fna&oh=f6b405581b1f87b5bac8980482131626&oe=5FB1A831

pic https://scontent.fiev21-1.fna.fbcdn.net/v/t1.0-9/121683367_4070517509631420_3635640716978942113_n.jpg?_nc_cat=104&_nc_sid=730e14&_nc_ohc=dCW3zPLECfQAX_Bz_qo&_nc_ht=scontent.fiev21-1.fna&oh=d4bf340dc3e9e9066876d6b5b27a1a0d&oe=5FB18AD2

pic https://scontent.fiev21-1.fna.fbcdn.net/v/t1.0-9/121740112_4070517832964721_3931135567963664668_n.jpg?_nc_cat=109&_nc_sid=730e14&_nc_ohc=YSo0-pLSUHwAX9MvPd4&_nc_ht=scontent.fiev21-1.fna&oh=e449f3fbba8c8dc4e1bd82d584b69c41&oe=5FB05458

\fi
