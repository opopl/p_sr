% vim: keymap=russian-jcukenwin
%%beginhead 
 
%%file 07_05_2021.fb.buzhanskij_maksim.1.vov_pepel_mirvojna
%%parent 07_05_2021
 
%%url https://www.facebook.com/permalink.php?story_fbid=1939079002923236&id=100004634650264
 
%%author 
%%author_id 
%%author_url 
 
%%tags 
%%title 
 
%%endhead 
\subsection{Груда пепла}
\Purl{https://www.facebook.com/permalink.php?story_fbid=1939079002923236&id=100004634650264}

Когда журналисты спрашивают, как украинцы относятся к Второй Мировой Войне,
когда вас спрашивают, всегда представляйте себе груду пепла.

А потом посмотрите вокруг, на пробегающих мимо детей, на своих близких, собаку
свою потрепайте за холку, есть же у вас собака?

Её бы не было.

Не было бы детей, жен, мужей, родителей, ничего бы не было.

Груда пепла.

Представили?

Ну всё, теперь можно смело отвечать на вопрос о своём отношении к Второй
Мировой Войне и о том, что для вас День Победы.

Это ведь не абстрактный какой то пепел где то там.

Это вы, ваш пепел, все самое дорогое вам - пепел.

И ничего больше.

Или вот это вот, всё вокруг.

Так важно, или не важно?
