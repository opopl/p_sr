% vim: keymap=russian-jcukenwin
%%beginhead 
 
%%file 21_07_2020.fb.lnr.8
%%parent 21_07_2020
 
%%endhead 
\subsection{Крым Украине не нужен. О чем свидетельствует позиция Зеленского по пресной воде для полуострова}
\url{https://www.facebook.com/groups/LNRGUMO/permalink/2865530483558507/}
  
\vspace{0.5cm}
{\small\LaTeX section: \verb|21_07_2020.fb.lnr.8| project: \verb|letopis| rootid: \verb|p_saintrussia|}
\vspace{0.5cm}

\index{Авторы!Егор Леев}

Крым Украине не нужен. О чем свидетельствует позиция Зеленского по пресной воде для полуострова

Президент Украины Владимир Зеленский выступает против подачи днепровской воды в Крым. Таким образом, он признает, что завладеть полуостровом Украине не удастся, да и сами украинские власти этого не хотят

Президент Украины Владимир Зеленский выступает против подачи воды в Крым, сообщил народный депутат от «Слуги народа» Богдан Яременко в Facebook.

«Я против», — так лаконично президент Зеленский только что ответил на мой вопрос о его отношении к восстановлению водоснабжения оккупированного Крыма», — написал Яременко поздно вечером 20 июля Facebook.

Таким образом, надеждам на нормализацию украино-российских отношений при президентстве Зеленского был нанесен очередной удар.

Однако в этих словах Зеленского можно найти подтверждение того, что на данном этапе в Киеве признают свою неспособность взять Крым под свой контроль.

Повышая ставки

В последнее время американские СМИ, специализирующиеся на Крыме (например, подразделение «Радио Свобода» — «Крым Реалии»), задают тон в повестке дня, касающейся Крыма. Издание Украина.ру уже описывало, как «Крым Реалии» выпустило ряд материалов о якобы реальной военной угрозе Украине со стороны России из-за днепровской воды.

Так, в мае это СМИ опубликовало видео «В Крыму готовят канал для подачи воды с материка?», в июне — «Сможет ли Украина отразить атаку России?» и «Водная «катастрофа» — повод для войны?». До этого — в апреле — это СМИ выпустило сюжет «Засуха ударит по Крыму сильнее, чем коронавирус».

Но на всём этом решили не останавливаться. И тему воды в Крыму украинские сотрудники американского СМИ продолжили и в июле. Однако сместили акценты. Теперь говорят не только о военной угрозе, но и о том, что «Россия выкачивает всю воду из Крыма». Сюжет с таким заголовком появился 18 июля.

В нем среди прочего говорилось о том, что бурение скважин и использование полученной оттуда воды может привести к засолению крымских почв, а также к исчерпанию запасов подземных пресных вод на полуострове.

Посыл сюжета прост: крымская власть своими решениями наносит вред экологии Крыма. Расплачиваться за это будут следующие поколения крымчан.

«За все годы после аннексии чиновники и ученые озвучили много идей, как можно обеспечить Крым водой: это и перебрасывание в Крым воды из Днепра, Дона — по дну моря, водопровод с Кубани. Еще вариант — опреснение морской воды и применение самолета, который бы вызывал искусственные дожди. Но ни один из них этих планов не реализовали», — говорилось в сюжете.

При этом там же подвергали жесткой критике существующие и довольно успешные проекты по обеспечению Крыма водой.

Казалось бы, Украина, которая считает полуостров своим, должна была бы озаботиться поставкой туда пресной воды из Днепра и, таким образом, не только «противодействовать засолению крымских почв», но и получить еще один источник денег, что немаловажно для страны, пребывающей в перманентном экономическом кризисе.

Однако, судя по заявлению Зеленского, подавать воду в Крым Украина не намерена. Отсюда вывод: либо вся информация о засолении крымских почв и вреде экологии при добыче воды из-под земли в Крыму преувеличена, а ее распространение служит обоснованием панических слухов о якобы готовящемся нападении России на Херсонскую область; либо Киев не спешит помогать экологии Крыма, так как уже смирился с тем, что полуостров — отрезанный ломоть, он никогда не будет украинским. А согласно одной из народных украинских поговорок, «У самих хата горит, зато приятно, что у соседа корова сдохла».

Речь также может идти не только о признании невозможности вернуть Крым, но и нежелании этого делать вообще. И тому есть причины.

Проблемный красный регион

По состоянию на 2014 год население Крыма равнялось почти 2 млн человек. Это одна двадцатая часть нынешнего населения Украины. И эти два миллиона крымчан в подавляющем большинстве своем не принимали украинский национализм и голосовали за коммунистов или Партию регионов.

В понедельник, 20 июля, бывший глава украинского Генштаба Владимир Замана, который возглавлял его как раз в 2014 году, рассказал, почему Украине не удалось удержать Крым. Замана рассказал, что Генштаб ВСУ тогда подготовил план «антитеррористической операции» в Крыму и доложил об этом Александру Турчинову, который в феврале после госпереворота стал и.о. президента Украины. Однако Турчинов так и не отдал соответствующий приказ.

«Воинские части были укомплектованы контрактниками. Определенное количество частей и подразделений были укомплектованы контрактниками и могли применяться не через неделю или две, а буквально в тот же день, когда получен приказ. Операция была подготовлена, я об этом докладывал товарищу Турчинову и руководству государства 28 февраля. Материалы того заседания СНБО до сегодняшнего дня полностью не раскрыты. То, что им надо, они раскрыли, а что невыгодно, они засекретили», — рассказал Замана.

По его словам, для располагавшейся в Крыму группировки украинских войск были поставлены дополнительные запасы топлива и продовольствия

«Если бы была политическая воля, не было бы такой постыдной сдачи своей территории. Мне сложно рассуждать, как бы развивалась эта операция, но результаты были бы однозначно <…> Основной причиной сдачи Крыма я считаю отсутствие политических решений по объявлению военного положения в Крыму и применения вооруженных сил. Командир не может сам применять свои подразделения и боевые уставы. Для этого нужно соответствующие политические решения, которых не было», — подчеркнул Замана.

Одной из причин, по которым таких решений не было, могла быть обеспокоенность политиков не только в сопротивлении майданной власти на местах, но и в реванше «регионалов». Перед Турчиновым и компанией маячил опыт 2006 года, когда выборы в Верховную Раду Украины запомнились ошеломительной для многих победой Партии регионов — политсилы, активистов и лидеров которой травили все годы с победы «Оранжевой революции». Юго-восток, который еще в 2004 году попробовал заявить об автономии, на время притих, но на выборах сказал свое слово. И повторил его уже на президентских выборах 2010 года.

С учетом же того, что Крым в большинстве своем был «красным» регионом (электоральной вотчиной коммунистов), несложно предусмотреть, какие последствия имел бы анонсированный в 2014 году тогдашним лидером запрещенного в РФ «Меджлиса» Мустафой Джемилевым «ленинопад» — снос памятников Владимиру Ленину.

Спустя шесть лет после воссоединения Крыма с Россией население полуострова увеличилось. По данным Росстата на 1 января 2020 года, оно составляло 2,3 млн человек. И все эти люди весьма негативно настроены к украинским властям. Это нелояльные киевским властям избиратели, что прекрасно понимает украинское руководство. Поэтому все разговоры о возвращении Крыма — это политическое пустословие. Украина не только не может, но и, самое главное, не хочет возвращения Крыма. И ведет себя как неродная мать из притчи о Соломоне, двух женщинах и одном ребенке. А это еще раз показывает, что в марте 2014 года крымчане не ошиблись с выбором.

Егор Леев
  
