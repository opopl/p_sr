% vim: keymap=russian-jcukenwin
%%beginhead 
 
%%file 24_11_2020.fb.maslov_evgenij.1.nikolaj_nosov
%%parent 24_11_2020
 
%%url https://www.facebook.com/maslovevgeniy14/posts/1356403004701404
 
%%author Маслов, Евгений
%%author_id maslov_evgenij
%%author_url 
 
%%tags 
%%title 23 ноября исполнилось 112 лет со дня рождения детского писателя Николая Николаевича Носова
 
%%endhead 
 
\subsection{23 ноября исполнилось 112 лет со дня рождения детского писателя Николая Николаевича Носова}
\label{sec:24_11_2020.fb.maslov_evgenij.1.nikolaj_nosov}
\Purl{https://www.facebook.com/maslovevgeniy14/posts/1356403004701404}
\Pauthor{Маслов, Евгений}

\index[names.rus]{Носов!Николай, Николаевич!Писатель}

\ifcmt
pic https://scontent-waw1-1.xx.fbcdn.net/v/t1.0-9/127444145_1356407078034330_9044768741745954661_n.jpg?_nc_cat=107&ccb=2&_nc_sid=8bfeb9&_nc_ohc=Hb2ofQFshyQAX9F3Jzp&_nc_ht=scontent-waw1-1.xx&oh=b497015256c3eadaedf75a21efa651fc&oe=5FE0D909
caption Незнайка
\fi

23 ноября исполнилось 112 лет со дня рождения детского писателя Николая Николаевича Носова.
Несколько пророческих цитат из известной книги писателя:

* * *
" --- К чему же богачам столько денег? --- удивился Незнайка. --- Разве богач
может несколько миллионов проесть? --- «Проесть»! --- фыркнул Козлик. --- Если бы
они только ели! Богач ведь насытит брюхо, а потом начинает насыщать своё
тщеславие. --- Это какое тщеславие? --- не понял Незнайка. --- Ну это когда
хочется другим пыль в нос пустить. Например, один богач построит себе
большой дом, а другой посмотрит и говорит: «Ах, ты такой дом построил, а я
отгрохаю вдвое больше!». Один заведёт себе повара да лакея, а другой
говорит: «Ну так я себе заведу не только повара и лакея, а ещё и швейцара».
Один наймёт целый десяток слуг, а другой говорит: «Ну так я найму два
десятка, да ещё сверх того пожарника в каске у себя во дворе под навесом
поставлю». Один заведёт три автомобиля, другой тут же заведёт пять. Да ещё и
хвастает: «Я, говорит, лучше его. У него только три автомобиля, а у меня
целых пять». Каждому, понимаешь, хочется показать, будто он лучше других, а
так как ум, доброта, честность у нас ни во что не ценятся, то хвалятся друг
перед другом одним лишь богатством. И тут уж никакого предела нет. Тщеславие
такая вещь: его ничем не насытишь."
* 
Пончик до такой степени уставал на работе, что, придя домой, растягивался на койке и вставал только для того, чтобы чего-нибудь пожевать. Даже еда не доставляла ему прежнего удовольствия. Теперь единственным для него наслаждением было отправиться в выходной день на берег и самому повертеться на каком-нибудь чёртовом колесе, параболоиде или хотя бы на водяной колбасе. --- Вот и чудесно! --- со злорадной усмешкой бормотал он. --- Целую неделю я вертел разных бездельников, а теперь пусть другие бездельники повертят меня! 
*
Мы не хотим также сказать, что, приобретая акции, коротышки ничего не приобретают, так как, покупая акции, они получают надежду на улучшение своего благосостояния. А надежда, как известно, тоже чего-нибудь да стоит. Даром, как говорится, и болячка не сядет. За все надо платить денежки, а, заплатив, можно и помечтать. 
*
Каждому, понимаешь, хочется показать, будто он лучше других, а так как ум, доброта, честность у нас ни во что не ценятся, то хвалятся друг перед другом одним лишь богатством. 
*
В те времена как для господина Жулио, так и для господина Спрутса самым большим удовольствием было усесться вечерком, после дневных забот, у телевизора и начать проклинать новые порядки. 
*
- С полицией, братец, лучше не связываться. От полиции никакой выгоды --- ни мне, ни тебе, леший её дери! 
*
«По настоящему, обязанность полицейских --- защищать население от грабителей, в действительности же они защищают лишь богачей. А богачи то и есть самые настоящие грабители. Только грабят они нас, прикрываясь законами, которые сами придумывают. А какая, скажите, разница, по закону меня ограбят или не по закону?» 
*
– За-за-за что я арестован? --- спросил, заикаясь от страха, Фикс. --- За то, что задаёте идиотские вопросы, --- объяснил Мига. 
*
Полицейский комиссар сказал, что все это увёртки, так как отличить полицейского от бандита не так уж трудно. В ответ на это стрелявший из пистолета сказал, что теперешнего полицейского не отличишь от бандита, так как полицейские часто действуют заодно с бандитами, бандиты же переодеваются в полицейскую форму, чтоб удобнее было грабить. В результате честному коротышке уже совершенно безразлично, кто перед ним: бандит или полицейский. 
*
Что выгодно для бедняков, то невыгодно для богачей. 
Н.Н.Носов
"Незнайка на Луне"
1965 г.
