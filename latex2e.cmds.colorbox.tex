% vim: keymap=russian-jcukenwin
%%beginhead 
 
%%file cmds.colorbox
%%parent body
 
%%endhead 

\section{\textbackslash colorbox}
\url{joshua.smcvt.edu/latex2e/Colored-boxes.html}

Synopses:

\begin{verbatim}
	\colorbox{name}{...}
	\colorbox[model name]{box background color}{...}
\end{verbatim}

or

\begin{verbatim}
	\fcolorbox{frame color}{box background color}{...}
	\fcolorbox[model name]{frame color}{box background color}{...}
\end{verbatim}

Make a box with the stated background color. The \verb|\fcolorbox| command puts a frame around the box. 
For instance this

\begin{verbatim}
	Name:~\colorbox{cyan}{\makebox[5cm][l]{\strut}}
\end{verbatim}

makes a cyan-colored box that is five centimeters long and gets its depth and
height from the \verb|\strut| (so the depth is \verb|-.3\baselineskip| and the height is
\verb|\baselineskip|). This puts white text on a blue background.

\begin{verbatim}
	\colorbox{blue}{\textcolor{white}{Welcome to the machine.}}
\end{verbatim}

The \verb|\fcolorbox| commands use the same parameters as \verb|\fbox| (see
\verb|\fbox| and \verb|\framebox|), \verb|\fboxrule| and \verb|\fboxsep|, to
set the thickness of the rule and the boundary between the box interior and the
surrounding rule. LaTeX’s defaults are \verb|0.4pt| and \verb|3pt|,
respectively.

This example changes the thickness of the border to 0.8 points. Note that it is
surrounded by curly braces so that the change ends at the end of the second
line.

\begin{verbatim}
	{\setlength{\fboxrule}{0.8pt}
	\fcolorbox{black}{red}{Under no circumstances turn this knob.}}
\end{verbatim}

