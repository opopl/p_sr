% vim: keymap=russian-jcukenwin
%%beginhead 
 
%%file 09_12_2018.fb.fb_group.story_kiev_ua.1.petr_nesterov.cmt
%%parent 09_12_2018.fb.fb_group.story_kiev_ua.1.petr_nesterov
 
%%url 
 
%%author_id 
%%date 
 
%%tags 
%%title 
 
%%endhead 
\zzSecCmt

\begin{itemize} % {
\iusr{Юрий Симоненко}
Был как- то у его могилы...

\ifcmt
  ig https://scontent-lga3-2.xx.fbcdn.net/v/t1.6435-9/47579768_353754585179431_2442432087442063360_n.jpg?_nc_cat=109&ccb=1-5&_nc_sid=dbeb18&_nc_ohc=8Ysbr2K3vNwAX-N7OE7&_nc_ht=scontent-lga3-2.xx&oh=00_AT-Uc0oGtSjM7h4FYIKFFwvK-6Zxrcolde3ZIAwdXuXQ6Q&oe=61E9068E
  @width 0.4
\fi

\begin{itemize} % {
\iusr{Татьяна Петрюк}
\textbf{Юрий Симоненко}, да на Лукьяновском кладбище ..

\iusr{Александра Пихур}
\textbf{Юрий Симоненко} Захоронен на Лукьяновском кладбище недалеко от моих близких
\end{itemize} % }

\iusr{Марина Лабунец}

Еще был живописец Михаил Васильевич Нестеров, который росписывал Владимирский
собор. Собор был освящен в 1896 году. После этого живописец покинул город. В
мае 1913 года поручик Петр Нестеров был назначен в авиационный отряд,
формировавшийся в Киеве. Так, два разных человека с одинаковой фамилией
примерно в одно и тоже время (с разницей в одно десятилетие) прославили наш
город.


\iusr{Елена Сидоренко}
Всего 27 лет прожил! Герой, красавец, настоящий мужчина!!!

\begin{itemize} % {
\iusr{Сергей Пятериков}

Пережив би першу світову - ймовірно, його б, як і десятки тисяч таких як він
офіценів під час громадянької розстріляли б більшовики. А як не у 18-20р.р., то
у 37-39, бо надто вже був показний красень...


\iusr{Елена Сидоренко}
\textbf{Сергей Пятериков} Шкода таких хлопців, таких хоробрих красенів бракує і зараз!
\end{itemize} % }

\iusr{Ольга Гураль}

У Києві, впродовж 1913 - 1914 рр., П. Нестеров мешкав у прибутковому будинку
по вул. Московська, 5, у квартирі № 23. А аеродром, де він вперше здійснив
\enquote{мертву петлю} також називався і Святошинським, розташовувався на Шулявці, за
нинішньою станцією метро.

\begin{itemize} % {
\iusr{Оксана Дітковська}
Здається Сирецький аеродром

\iusr{Оксана Дітковська}
\url{https://uk.wikipedia.org/wiki/Мертва_петля}

\iusr{Оксана Дітковська}
\href{https://dimlzapovidnuk.files.wordpress.com/2019/04/d09ad180d0b0d194d0b7d0bdd0b0d0b2d181d182d0b2d0be_4-2018.pdf}{%
До 100-річчя завершення Першої світової війни, КРАЄЗНАВСТВО, 2018}

\iusr{Оксана Дітковська}
Стаття до 100 річчя завершення 1 Св. війни

\end{itemize} % }

\iusr{Людмила Филоненко}

Моя бабушка жила (с довоенных лет и до 1978г.) в доме на Московской, 5, кв.5.
Честно, не представляю, где могла быть кв.23??? В 1978 этот дом \enquote{перешёл} в
ведение завода \enquote{Арсенал}, а потом ещё несколько раз менялись \enquote{хозяева} этого
дома... Радует то, что дом внешне ухожен, правда, внутрь попасть не
получилось.....


\iusr{Сергей Пятериков}
А цікаво було б дізнатися, що це за портрет, його походження...

\begin{itemize} % {
\iusr{Феликс Лидерман}
\textbf{Сергей Пятериков} 

Есть такой уникальный человек Ольга Ширнина, которая восстанавливает и
колоризирует исторические фото и делает их живыми.

\href{https://rosphoto.com/history/klimbim-5276}{%
ОЛЬГА ШИРНИНА: «МНЕ ИНТЕРЕСНА ИСТОРИЯ РОССИИ», rosphoto.com, 26.09.2016%
}

\iusr{Сергей Пятериков}
\textbf{Felix Liderman} 

- щиро дякую, колись вже бачив цю техніку розфарбовування ч/б фото - вона мені
здалася занадто кропіткою, але результат потрясаючий! \enquote{Игра стоит свеч!}

\end{itemize} % }

\iusr{Анатолий Пецух}

В Нижнем Новгороде в местном краеведческом музее хранится кортик Петра
Нестерова!

\begin{itemize} % {
\iusr{Юрий Симоненко}
\textbf{Анатолий Пецух} Интересно было бы узнать, как он туда попал...

\iusr{Анатолий Пецух}

Трудно сказать. В Горьком (ныне Нижний Новгород) я был в командировке. Это было в
70х годах 20 столетия. Может быть недалеко от места падения самолёта или просто
как в память о родине авиатора. Интересно, можно ли послать запрос в местный
краеведческий музей об этом кортике?

\iusr{Анатолий Пецух}
Там даже был снимок места падения и искарёженный кортик.
\end{itemize} % }

\end{itemize} % }
