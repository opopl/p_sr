% vim: keymap=russian-jcukenwin
%%beginhead 
 
%%file 18_07_2020.lj.chervonec_001.rabstvo_usa
%%parent 18_07_2020
 
%%endhead 
\subsection{В США рабство окончательно было отменено лишь в 7 лет назад}
\label{sec:18_07_2020.lj.chervonec_001.rabstvo_usa}
\url{https://chervonec-001.livejournal.com/3347591.html?fbclid=IwAR1ciwkL7WOeB04t82HeF_8xQw9I1lPdiHGmC3-0DS01NWSVvfsod1XwDSs}
  
\vspace{0.5cm}
{\small\LaTeX~section: \verb|18_07_2020.lj.chervonec_001.rabstvo_usa| project: \verb|letopis| rootid: \verb|p_saintrussia|}
\vspace{0.5cm}

Американцы так ратуют за свободу и демократию, но вот рабство у себя в стране
отменили только шесть лет назад!

Право слово, даже как-то неудобно выходит. Соединённые Штаты все из себя якобы
светоч демократии, «град сияющий на холме», а начнёшь в их законодательстве
копаться, так такие несуразицы находятся, что хоть стой, хоть падай. Но о
законодательстве в другой раз. А пока просто посмотрим, что же в этой стране с
таким простым по сути вопросом как рабство.

Тринадцатая поправка к Конституции США, отменяющая рабство, была принята
Конгрессом во время Гражданской войны 31 января 1865 года и вступила в силу 6
декабря 1865 года после того, как её ратифицировали необходимые три четверти
штатов --- 27 из существовавших тогда 36.

Т.е. этот даже не закон, а поправка к Конституции, которая должна была быть
ратифицирована прежде чем вступить в силу. Однако некоторые штаты неторопливо
присоединялись к этой инициативе на протяжении более чем ста лет. Более того,
беглых рабов должны были задерживать на территории тех штатов, где рабство было
отменено и возвращать хозяевам в те штаты, где отмена рабства не была
раьтфицирована.

И тут самый интересный факт - штат Кентукки ратифицировал 13-ю поправку только
в 1976 г, а в Миссисипи ратифицировать поправку и вовсе забыли, аж до 30 января
2013 года! Лишь 7 февраля 2013 года директор Федерального реестра Чарльз Барт
объявил о том, что поправка официально зарегистрирована, т.е. рабство
окончательно отменено на всей территории США.

Кто-то скажет - "ну подумаешь, забыли закнодательно оформить!". Однако упустят
из внимания тот факт, что в 70-е годы без всякой забывчивости в отдельных
штатах рабство оставалось на законодательном уровне.


Но даже формальная отмена рабства ещё сотню лет не решала проблем с правами
чернокожего населения. Большинство политических деятелей Америки того времени,
включая борцов за отмену рабства, исходили из постулата о превосходстве белой
расы над чернокожими. Поэтому личная свобода для рабов не означала обретения
ими гражданских прав. Сразу после принятия Тринадцатой поправки в южных штатах
были приняты так называемые «Черные кодексы», определявшие порядок жизни
чернокожего населения.

Например, в штате Миссисипи неграм под страхом пожизненного тюремного
заключения отказывалось в праве вступать в брак с белыми, запрещалось носить
оружие, ограничивалось их право владеть землей. «Закон о подмастерьях» гласил,
что все негры --- подростки до 18 лет, не имеющие родителей, или дети бедных
родителей, отдавались в услужение белым, которые могли их насильно удерживать в
услужении, возвращать в случае побега по суду и подвергать телесным наказаниям.
Отдельно стоит сказать о входивших в «Черные кодексы» «Законах о
бродяжничестве».

Поскольку освобождение бывших рабов происходило без наделения землей, вчерашние
хозяева выбрасывали свободных людей на улицу, оставляя их без куска хлеба и
крыши над головой. Тут они попадали под действие «Закона о бродяжничестве».
Согласно ему, чернокожие, не имевшие постоянной работы, объявлялись бродягами,
заключались в тюрьму и отправлялись в каторжные бригады, либо они оказывались
на плантациях у прежних хозяев. Альтернативой была уплата штрафа за
бродяжничество, но денег у несчастных просто не было. При этом эксплуатация
«бродяг» была порой еще более жестокой, нежели до отмены рабства.

Чернокожие оставались де факто поражены в правах. Даже после формального
разрешения им голосовать. Формально Пятнадцатая поправка наделяла чернокожее
население южных штатов избирательными правами, однако местное законодательство
было выстроено таким образом, что подавляющее большинство афроамериканцев
оставались бесправными. Например, на выборах 1900 года в Алабаме из 181500
человек чёрного населения допустили к голосованию лишь 3000.

Особую популярность приобрели так называемые «суды Линча» --- убийства человека,
подозреваемого в преступлении или нарушении общественных обычаев, без суда и
следствия. После Гражданской войны в США главными жертвами «судов Линча»
становились афроамериканцы. Излюбленным методом убийц на таких судах было
повешение несчастных или даже их сожжение.

Точной статистики по «судам Линча» нет. Специалисты Университета Миссури,
изучая этот вопрос, пришли к выводу, что в период между 1882 и 1920 годами
линчеванию подверглись около 3500 афроамериканцев. Критики полагают, что в
данном случае речь идет лишь о самых громких публичных случаях, а общее
количество чернокожих, убитых расистами, измеряется десятками тысяч человек.

Принять законы против линчевания пытались сразу несколько американских
президентов, включая Франклина Рузвельта, однако их попытки потерпели неудачу.
Лишь в 1960-х годах линчевание стало рассматриваться как убийство с отягчающими
обстоятельствами. Лишь в июне 2005 года Сенат США принял резолюцию, в которой
официально принес извинения за бездействие в отношении линчевания нескольких
тысяч человек, в основном чернокожих.

Так что ущемление прав чернокожих продолжались уже в современной истории. К
примеру. 1 декабря 1955 г. Роза Паркс, 42-летняя чернокожая швея одного из
универмагов Монтгомери, столицы штата Алабама, была задержана и затем
оштрафована за отказ уступить место в автобусе белому пассажиру, как от неё
требовалось по местному закону.

Для того, чтобы в 1957 году в городе Литтл-Рок штата Арканзас чернокожие
ученики смогли пойти в одну школу с белыми, в населенный пункт была введена
101-я воздушно-десантная дивизия. Десантникам было приказано выполнить решение
суда, несмотря на сопротивление местных властей.

СССР уже в космос активно летал, но только в 1962 году и то, после решения
Верховного суда США, был принят первый чернокожий в американский университет
Миссисипи. Чернокожий студент Джеймс Мередит появился в университете в
сопровождении федеральных маршалов, однако для полного урегулирования ситуации
в Миссисипи были направлены несколько тысяч солдат федеральной армии. Во время
беспорядков погибли два человека, 375 человек были ранены и около 200
арестованы. Некоторое количество военнослужащих оставались для охраны Мередита
до его выпуска.

Так называемый "автобус ненависти" с членами американской нацистской партии,
которые пытаются всячески запугивать непокорных черных (1961 год)

Расисты оказывали ожесточенное сопротивление. В 1963 году губернатор Алабамы
Джордж Уоллес заявил: «Сегрегация сегодня, сегрегация завтра, сегрегация
навсегда». 11 июня 1963 года произошёл инцидент на входе в университет, когда
Уоллес закрыл собой дорогу для первых двух чернокожих студентов Университета
Алабамы --- Вивиан Мелоун и Джеймса Худа. 12 июня 1963 чернокожий активист Медгар
Эверс погиб на пороге своего дома в городе Джексоне (штат Миссисипи) от пули
белого расиста Байрона де ла Беквита.

Только в 60-е годы 20-го века реально началась борьба за права чернокожих.
Законодательно она закончилась только к 1968 году, уже после гибели борца за их
свободу Мартина Лютера Кинга. Но фактические экономическое неравенство черных,
разумеется, нельзя было устранить только принятием тех или иных законов. А вот
ратификация отмены самого рабства затянулась аж до 21-го века, как видим.

П.С.

В первоначальном виде демократия гарантировала не менее пяти рабов.

Местные фермеры сдавали своих рабов в аренду государству по 55 долларов в год.
С их помощью были возведены Капитолий, Белый дом, здание министерства финансов.

Рабы валили деревья на Капитолийском холме, выкорчевывали пни на месте будущих
улиц, работали в каменоломнях, где добывали песчаник и помогали каменотесам
трудиться над строительным материалом для новых резиденций Конгресса и
президента. Рабов также использовали при расширении Капитолия в конце 1850-х
годов.

На этом выросла западная "демократия". Должны платить и каяться.

И при этом американцы очень любят снимать кино и делать игры о том как, скажем,
в первую мировую негр смело сражался против немцев. Или во вторую мировую
генерал-негр из Америки собственноручно уничтожал сотни нацистов. Или как негр
стал Стрелком в Тёмной Башне.

То есть вместо признания собственной истории и проблем они предпочитают
создавать альтернативную историю, и многие даже готовы наказывать, если эту
альтернативную историю отказываются создавать или признавать истинной.

Зато вот залезть в историю СССР и диктовать, как надо ещё преподносить, они
тоже очень любят.
  
