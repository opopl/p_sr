% vim: keymap=russian-jcukenwin
%%beginhead 
 
%%file slova.prezident
%%parent slova
 
%%url 
 
%%author 
%%author_id 
%%author_url 
 
%%tags 
%%title 
 
%%endhead 
\chapter{Президент}
\label{sec:slova.prezident}

Я думала о том, нужно ли мне идти на эту пресс-конференцию и что спросить у
\emph{президента}. Но вопросов к нему у меня нет. За два года мы успели убедиться в
том, что наш гарант не умеет отвечать на вопросы. Не только потому что
косноязычный. Он не умеет мыслить, думать. У него отсутствует мировоззрение.
Нет плана. Нет стратегии развития страны. Так что и вопросов к нему нет. Кроме,
разве что, риторического \enquote{доколе?}.  А главный вопрос для нашей
редакции – это вопрос о незаконном закрытии трех телеканалов (ZIK, NewsOne,
112). И абсолютного беспредела в отношении блокировки \enquote{Першого
незалежного} по звонку СБУ. Но надежд на то, что \emph{гарант} даст ответ на этот
вопрос, у меня нет. Потому что это абсолютно политическое решение и расправа
над независимыми СМИ и над оппозицией. Эти вопросы наша редакция планирует
решать в судебном порядке,
\textbf{\enquote{Если вас взяли в заложники - подайте знак}. Что украинские журналисты спросили бы у Зеленского},
strana.ua, 18.05.2021

Идея \emph{Президентского} университета, где будут готовить людей будущего,
выглядит странной, если сопоставить ее с настоящим. Например, по заявленным
специальностям (нанотехнологии, искусственный интеллект, кибербезопасность,
космос) уже сейчас готовит специалистов КПИ. Думаю, что готовит хорошо. Но об
уровне образования в действующих вузах можно узнать уже сегодня, тогда как
\emph{Президентский} университет остается загадкой.  На него планируют
потратить 7,2 миллиарда гривен за три года. В то же время годовой бюджет
Национальной академии наук – 6,2 миллиарда, из которых лишь полмиллиарда идут
на приоритетные направления научных исследований,
\citTitle{На президентский университет собираются потратить 7.2 миллиарда гривен}, Алексей Гавриленко, strana.ua, 08.06.2021

