% vim: keymap=russian-jcukenwin
%%beginhead 
 
%%file 15_06_2020.fb.fedina_sofia.1.mova_protest
%%parent 15_06_2020
 
%%url https://www.facebook.com/sofiya.fedyna/posts/10158214538815390
 
%%author 
%%author_id 
%%author_url 
 
%%tags 
%%title 
 
%%endhead 

\subsection{ЗУПИНИТИ НАСТУП НА УКРАЇНСЬКУ МОВУ ТА ІНФОРМАЦІЙНИЙ ПРОСТІР. Відкрите звернення.}
\label{sec:15_06_2020.fb.fedina_sofia.1.mova_protest}
\url{https://www.facebook.com/sofiya.fedyna/posts/10158214538815390}

Відкрите звернення небайдужих людей, яке я всеціло підтримую!


\ifcmt
  pic https://scontent-amt2-1.xx.fbcdn.net/v/t1.6435-9/103411461_10158214541310390_6864012414064041615_n.jpg?_nc_cat=109&ccb=1-3&_nc_sid=730e14&_nc_ohc=2VLw1ROMNkQAX8M2LY0&_nc_ht=scontent-amt2-1.xx&oh=67e86c5a2b88d48a39ae307c6795b639&oe=608EF49D
  width 0.4
\fi

Новітня історія України засвідчує: якщо влада пропонує «винести за дужки» питання української ідентичності, а лейтмотивом мовної політики стає «яка різниця?», невдовзі цій владі українська мова і все, що робить країну Україною, чомусь стає поперек горла.
Так було в часи Януковича-Табачника і Ківалова-Колесніченка, так відбувається і тепер.
Упродовж першого року президентства Зеленського жевріла надія, що владі вистачить відповідальності зосередитися на вирішенні гострих питань безпеки та економіки і не чіпати закон про мову та норми інших законів, що захищають українську освіту, культуру, медіапростір.
Ці закони містять важливі механізми єднання нації, захисту державної незалежності й національної безпеки України. Доки українське військо боронить нашу незалежність на фронті, ці закони зміцнюють її всередині країни. Вони працюють і підтримуються більшістю людей.
Єдине, що потрібно від влади – неухильно їх виконувати.
Проте останнім часом влада почала озвучувати тези, досі характерні лише для пропаганди держави-агресора і пов’язаних з нею олігархічних кланів. Президент України, Голова Верховної Ради, новий міністр культури й інформаційної політики та інші посадовці раптом почали розповідати про потребу переглянути закон про мову, про якусь «іншу сторону» (?!) яку треба «почути», про таємничих «вчених», руками яких влада хоче розворушити вже вирішене на законодавчому рівні питання державної мови і ощасливити українців «правильними» мовними ініціативами.
«Слуги народу» не обмежилися словами і переходять до практичних руйнівних дій, які потребують належної реакції з боку суспільства.
По-перше, влада з маніакальною впертістю намагається протягнути законопроєкт №2362 одіозного депутата Максима Бужанського, який уже втретє (!) виносять на засідання комітету з гуманітарної та інформаційної політики. Цей проєкт передбачає зміни в закони про мову і про освіту, щоб скасувати перехід з 1 вересня 2020 року на навчання українською мовою тих учнів 5-11 класів, які досі навчалися російською.
Проти цього антиукраїнського проєкту виступають не лише громадські й експертні середовища, але й Міністерство освіти й науки. Адже за три роки надруковані підручники та посібники, проведені курси для вчителів і ніщо не заважає переходу шкіл на українську мову. Очевидно, що форсований розгляд проєкту 2362 на комітеті потрібен для того, щоб проштовхнути його на пленарному засіданні Ради до канікул.
Причому в останній «доопрацьованій» редакції законопроєкт став ще небезпечнішим. Він до всього іншого руйнує мовні норми ухваленого вже нинішнім складом парламенту і підписаного Зеленським Закону «Про повну загальну середню освіту», відкриваючи шлюзи для русифікації освітнього процесу.
Не менше обурення викликає і підготовлена до реєстрації і розгляду на комітеті нова версія законопроєкту №2693 «Про медіа». У тексті доопрацьованого законопроєкту не лише значно погіршені порівняно з попередньою версією і з чинним законодавством механізми захисту української мови, але й практично знищується такий важливий інструмент протидії російській інформаційній агресії як Перелік осіб, що створюють загрозу національній безпеці.
Законопроєкт пропонує скасувати чинний перелік 150 осіб, фільми, програми, концерти за участю яких не мають права демонструватися в Україні. Натомість сформувати за дуже вузькими критеріями куций Перелік осіб, що створюють загрозу медіа-простору, потрапити в який можна буде лише через голосування Національної Ради з питань телебачення й радіомовлення, більшість якої становлять вихідці з «Кварталу 95» і студії «1+1». Це означає масове повернення на українські екрани й у концертні зали «артистів-рашистів» з усіма наслідками для інформаційної безпеки держави.
Це – червоні лінії, які влада не має права перетинати ні заради комерційних інтересів олігархічних медіа, ні, тим більше, на догоду Москві і її п’ятій колоні.
Ми закликаємо керівництво держави і парламенту, народних депутатів України:
1. Зняти з розгляду провокаційний законопроєкт Бужанського №2362 і припинити спроби відновлення русифікації освіти.
2. Під час реєстрації та розгляду в першому читанні нового законопроєкту «Про медіа» не допустити послаблення жодної чинної законодавчої норми, що захищає українську мову та національний інформаційний простір, зберегти чинний Перелік осіб, що загрожують національній безпеці та механізми його наповнення і застосування.
3. Відмовитися від спроб ревізії Закону «Про забезпечення функціонування української мови як державної» і зосередитися на його неухильному виконанні. А в цілому замість збурення і розколу суспільства на мовному ґрунті – сконцентруватися на розв’язанні соціальних та економічних проблем українських громадян.
Захист нашої мови і культурно-інформаційного простору – це речі, які українці боронитимуть за будь-яку ціну всіма передбаченими законом засобами. Ми закликаємо владу дослухатися до наших справедливих вимог і не провокувати громадянське протистояння.
Володимир Василенко, доктор юридичних наук, професор, надзвичайний і повноважний посол України
Ігор Козловський, релігієзнавець, громадський діяч, учасник Ініціативної групи "Першого грудня", член PEN Україна
Вахтанг Кіпіані, журналіст, головний редактор проєкту “Історична правда”
Віталій Портников, журналіст, публіцист
Йосип Зісельс, колишній політв'язень, учасник ІГ "Першого грудня
Ірма Вітовська, акторка театру та кіно, заслужена артистка України
Тарас Компаніченко, народний артист України, лідер гурту “Хорея козацька”
д-р Уляна Супрун, голова ГО "Арк.ЮЕЙ", в.о. міністра охорони здоров'я України (2016-2019)
Володимир Балух, громадський активіст, політичний в'язень путінського режиму
Ахтем Сеітаблаєв, кінорежисер, заслужений артист АР Крим
Володимир Тихий, кінорежисер, лавреат Національної премії ім. Т.Г. Шевченка, засновник об‘єднання кінодокументалістів «Вавилон 13»
Юлія Кириченко, Центр політико-правових реформ, співголова Ради Реанімаційного пакету реформ
Ірен Роздобудько, письменниця
Римма Зюбіна, актриса, телеведуча, громадська діячка
Юрій Журавель, лідер гурту OT VINTA, автор спільноти ЗНАЙ НАШИХ
Юрій Винничук, письменник
Ганна Гопко, мережа захисту національних інтересів АНТС, голова Zero Corruption Conference
Андрій Кокотюха, письменник
Тарас Антипович, письменник, сценарист
Анна Заклецька, українська співачка, вокалістка гурту "Vroda", телеведуча, модель
Лариса Масенко, професор НУ "Києво-Могилянська академія", доктор філологічних наук
Тарас Шамайда, рух “Простір свободи”
Аліна Михайлова, БФ Армія SOS, волонтер
Ярина Чорногуз, ініціаторка руху протесту «Весна_на_граніті», бойовий медик
Максим Кобєлєв, ініціатива “Дріжджі”
Євген Дикий, науковець, ветеран українсько-російської війни
Катерина Чепура, театральна режисерка, громадянський рух “Відсіч”
Іванна Кобєлєва, Портал мовної політики
Тетяна Строй, Донецький прес-клуб
Іван Патриляк, історик, доктор історичних наук, професор
Світлана Поваляєва, письменниця
Олег Слабоспицький, Громадський сектор Євромайдану
Олександр Іванов, ініціатива “Переходь на українську”
Сергій Оснач, член Експертної групи з питань мовної політики при Кабінеті міністрів України
Сергій Стуканов, журналіст
Павло Вольвач, письменник
Сергій Пантюк, письменник
Ольга Купріян, письменниця
Ярина Скуратівська, журналістка
Богдан Логвиненко, ГО Українер
Павло Подобєд, керівник благодійного фонду "Героїка"
Нестро Воля, суспільно-політичний блогер
Світлана Єременко, Інститут демократії імені Пилипа Орлика
Інна Юр'єва, газета "Громада Схід"
Тетяна Швидченко, ГО Експертний корпус
Марія Давиденко, ІА Вчасно
Анастасія Мазниченко, редакторка
Альона Заіка, журналістка
Ольга Кирилова, журналістка
Юлія Костюченко, журналістка
Влад Солдатенко, журналіст
Олександр Цахнів, журналіст
Денис Блощинський, Фундація соціальних інновацій “З країни в Україну”
Сергій Костинський, член Національної ради з питань телебачення і радіомовлення (2015-20)
Світлана Кашенець, ГО “ЗміниЄ”
Оксана Левкова, Всеукраїнська громадська організація "Не будь байдужим!"
Юрій Гончаренко, Фонд підтримки демократичних ініціатив.
Святослав Липовецький, публіцист
Костянтин Рєуцький, благодійний фонд Восток-SOS
Микола Марченко , експерт Асоціації українських правників з конституційного права
Андрій Левус, Рух опору капітуляції
Магновський Ігор, доктор юридичних наук, експерт з публічного права
Погребиський Олександр, ветеран війни проти Росії
Сергій Пархоменко, Центр зовнішньополітичних досліджень ОПАД ім. О.Никанорова
Марія Шейко, ЗМГО "Півострів змін"
Сергій Підмогильний, мережа Greenways Ukraine
Анастасія Розлуцька, ГО "Український світ"
Ольга Андрусенко, ініціатива “Є-мова”
Надія Зубатюк, ГО "Асоціація вчителів української мови та літератури Донеччини"
Надія Басенко, Українська спеціалізована загальноосвітня школа з поглибленим вивченням окремих предметів
Наталія Хоменко, фольклорист, кандидат філологічних наук
Андрій Юсов, Школа відповідальної політики
Тарас Марусик, експерт з питань мовної політики, ТО перекладачів Національної спілки письменників України, журналіст
Христина Сударенко, волонтерський рух "Безкоштовні курси української мови"
Муравльова Оксана, ГО "Український розмовний клуб "Файно", ВГО ДВ "Союз українок"
Олександр Мартиненко, боєць батальйону спецпризначення "Донбас", ініціатива "Ветерани плюс"
Ярослава Братусь, ГО "Центр соціального розвитку " Ініціатива"
Павло Вишебаба, музикант One Planet Orchestra, голова ГО Єдина Планета
Євген Карась, рух «Суспільство майбутнього»
Олександр Войтко, Спілка ветеранів війни з Росією
Ігор Луценко, народний депутат України 8 скликання
Денис Поліщук, керівник моніторингової групи з дотримання прав ув‘язнених Міністерства у справах ветеранів
Євген Бондаренко, ГО Школа Медіапатріотів
Мирон Спольський, підприємець
Микола Панченко, ГО "Освітня асамблея"
Борис Бабін, доктор юридичних наук, професор
Тарас Стецьків, народний депутат 5-ти скликань, голова Зарваницької громадської ініціативи
Юрій Шевчук, мовознавець, викладач Колумбійського та Єльського університетів, засновник і директор Українського кіноклубу при Колумбійському університеті
Сергій Висоцький, народний депутат України 8 скликання
Михайло Гончар, Центр глобалістики “Стратегія ХХІ”, головний редактор журналу “Чорноморська безпека”
Ліліана Вежбовська, мистецтвознавиця
Андрій Щекун, Кримський центр “Український дім”, головний редактор газети “Кримська світлиця”
Марина Мірзаєва, історикиня, голова ГО "Валькірія"
Андрій Фендик, голова Крайової Управи Спілки Української Молоді в Україні
Олександр Северин, кандидат юридичних наук, публіцист
Марія Давиденко, ГО "Медіа-Погляд"
Євген Межевікін, підполковник Збройних сил України, Герой України
Наталя Васько, акторка театру та кіно
Михайло Басараб, політолог
Леонтій Шипілов, член ЦВК (2018-2019)
Дмитро Лиховій, журналіст, головний редактор інтернет-видання "Новинарня"
Анастасія Мельниченко, ГО “Студена”
Олександр Семирга, кандидат технічних наук, старший науковий співробітник
Юрій Міндюк, громадський діяч. співзасновник Національно-консервативного руху
Ігор Артюшенко, народний депутат 8-го скликання, Центр національного відродження
Тарас Гребеняк, ГО “Українська справа”
Лада Каневська, медіаторка, фасилітаторка, співавторка курсів про діалог і управління конфліктами
