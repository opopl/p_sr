% vim: keymap=russian-jcukenwin
%%beginhead 
 
%%file 08_08_2021.fb.miroslava_iltjo.1.chelovek_poezia
%%parent 08_08_2021
 
%%url https://www.facebook.com/muroslavamix/posts/1787068084814141
 
%%author Ильтьо, Мирослава
%%author_id miroslava_iltjo
%%author_url 
 
%%tags chelovek,mova,poezia,ukraina
%%title Жив собі відлюдькуватий чоловік
 
%%endhead 
 
\subsection{Жив собі відлюдькуватий чоловік}
\label{sec:08_08_2021.fb.miroslava_iltjo.1.chelovek_poezia}
 
\Purl{https://www.facebook.com/muroslavamix/posts/1787068084814141}
\ifcmt
 author_begin
   author_id miroslava_iltjo
 author_end
\fi

\begin{multicols}{2}
\obeycr
Жив собі відлюдькуватий чоловік. 
Ну як жив, він і досі собі живе.
У нього звичайне ім'я, середній вік, 
У будні працює, на вихідних п'є.
\smallskip
Не дебоширить, говорить скупо.
Висить на турніку після роботи,
Сортує сміття на мільйон купок.
Раз на рік ходить в церкву навпроти.
\smallskip
Пересічний чоловік без харизми.
Нікому до нього ніякого діла,
З дня у день він розгадує механізми,
Як позбутися свого тіла.
\smallskip
Ну скільки можна, – каже він, –
Вже набридло носити одне і те ж.
Я готовий до змін, чуєш, змін!
Я готовий віддати його, візьмеш?
\smallskip
Заплющує очі. Стоїть. Чекає,
Заручившись мовчазною згодою стін.
І нічого не відбувається, змін немає.
Ось він. Дзеркало. За свідка тінь.
\smallskip
На роботу, з роботи, у магазин,
Купує пиво, чіпси, вмикає тупе кіно.
Чорт! Якщо я улюблений твій син,
То чому я ношу на собі це лайно?
\smallskip
Сідає в тролейбус, їде до кінцевої,
Кришить печиво, кормить голубів.
Бачите, яке він дав лице мені?
Такі лиця хіба в дурнуватих жлобів.
\smallskip
Голуби їдять, воркочуть до чоловіка,
Який бурчить і годує їх безліч років.
Він не потвора, не чмир, не каліка.
Має гарні очі, глибокі-глибокі.
\smallskip
Глибокі такі, як старий океан,
Добрі такі, мов в бездомного пса.
Поглядом може сцілювати від ран,
Поглядом гойдати на небеса,
\smallskip
Поглядом купати в медовому молоці,
Поглядом огортати теплом Сахари,
Має погляд божественний на лиці,
А ходить з дня у день, мов примара.
\smallskip
Бо, бачте, тіло сковує його міць!
Бо, бачте, тіло носити йому остогидло.
У тіло запхали його силоміць,
За тілом, бачте, душі не видно.
\smallskip
Відрощує бороду, збриває, носить щетину.
Часом запускає себе, нажирає живіт.
Тіло буває ніжне, як у дитини,
Буває обвисле, як перестиглий плід.
\smallskip
На нього кажуть старий холостяк,
Живе сам один, нікому не підійшов.
Умілий механік, хоч трохи й пияк. 
Син Бога, який увесь світ обійшов –
І прийшов жити в під'їзді навпроти.
\smallskip
Здавати макулатуру, кришки, склотару, 
Дратувати на тротуарі водіїв чорноротих,
Купувати за грубі гроші сигари. 
\smallskip
Натирати вузькими туфлями мозолі,
Заглядатися на красивих жінок,
Він аж так давно живе на землі,
Що написав аквареллю геть всіх зірок!
\smallskip
Він дуже втомився за довгий вік,
Йому мало хто дивиться в очі бездонні,
Бо ж на вигляд простий собі чоловік –
Часом звичайний бездомний.
\smallskip
Він і є дім. З руками, ногами, головою.
Ходить тими ж вулицями, що й ти,
Щоб стрітися поглядом якось з тобою –
І далі собі піти.
\restorecr
\end{multicols}

\verb|#і_м_і|

\ii{08_08_2021.fb.miroslava_iltjo.1.chelovek_poezia.cmt}
