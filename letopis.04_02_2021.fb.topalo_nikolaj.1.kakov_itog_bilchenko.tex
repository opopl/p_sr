% vim: keymap=russian-jcukenwin
%%beginhead 
 
%%file 04_02_2021.fb.topalo_nikolaj.1.kakov_itog_bilchenko
%%parent 04_02_2021
 
%%url https://www.facebook.com/permalink.php?story_fbid=782829155640551&id=100017404206730
 
%%author 
%%author_id topalo_nikolaj
%%author_url 
 
%%tags bilchenko_evgenia,obschestvo,skandal,ukraina
%%title "Ну, браток, каков итог?..."
 
%%endhead 
 
\subsection{\enquote{Ну, браток, каков итог?...}}
\label{sec:04_02_2021.fb.topalo_nikolaj.1.kakov_itog_bilchenko}
\Purl{https://www.facebook.com/permalink.php?story_fbid=782829155640551&id=100017404206730}
\ifcmt
 author_begin
   author_id topalo_nikolaj
 author_end
\fi

\enquote{Ну, браток, каков итог?...}

А итог ожидаемо никакой. Дело Евгении Витальевны Бильченко буксует и утопает в бюрократической трясине. 

Учёный Совет решил... что решать не вправе. Потому, временно отстранил БЖ от
преподавания. Дело передадут \enquote{дальше}, то бишь, \enquote{экспертам} (точно не в сфере
культурологии).

Само заседание Совета прошло просто и чинно, по схеме пятидесятилетней
давности, в лучших традициях былой эпохи. Вспомнили все недостатки Евгении
Витальевны. Коих вспомнили так много, что даже стало интересно - как же её
взяли на работу и как же она столько лет успешно работала на факультете? И что
интересно - все замечания были: а) выражены в самых беглых, расплывчатых и
абстрактных терминах; б) не имели за собой никаких оснований.

Если только у нас не принято обвинению в \enquote{зраді} верить на слово. Или верить на
слово письменным жалобам, не соответствующим действительности. 

Вообще, Совет прошел достаточно интересно. Там были и поэтические сочинения на
тему \enquote{Бильченко - предатель родины}, и попытки политизации чисто научного
дискурса, и обвинение в нарушении чисто человеческой морали (апеллируя к памяти
погибших на Майдане и в АТО), и введение ситуации в политические бинарные
оппозиции аля \enquote{чей Крым} что, естественно, к критической теории никакого
отношения не имело. А столько обвинений в \enquote{аморальном поведении} и \enquote{нарушении
этики} наверное не получают даже преступники, чьё дело дают огласке в СМИ. 

Студентов, что решили защищать профессора, постоянно обвиняли в демагогии,
манипуляции, запугивании (кого? где? когда?) и в результате призвали \enquote{не
создавать конфликт} - так теперь называют отстаивание собственного мнения.
Изюминкой стало что нас назвали \enquote{левыми радикалами} (прости Господи), что
является интересным названием для людей отстаивающих науку своей же страны. 

Примечательно что пожелания смерти, многочисленные интервью в СМИ и письменный
жалобы - это всё мнения \enquote{обеспокоенных граждан} что имеют место. А любое мнение
в поддержку - это необоснованный акт агрессии, усугубляющий и без того
непростую ситуацию. Интересно, не правда ли? 

В общем и целом - весело и потешно... могло бы быть, если бы это была
постановка-карикатура на тоталитарный режим. А это наша действительность за
окном. Так что печально и обидно.

\ifcmt
  pic https://scontent-frx5-2.xx.fbcdn.net/v/t1.6435-0/p180x540/145956353_782827212307412_6013248601177462460_n.jpg?_nc_cat=109&ccb=1-3&_nc_sid=730e14&_nc_ohc=82MQOekSQuIAX8-UQN6&_nc_oc=AQniBtRvufmTeR6VbGxllzE-8umRGPnsmaFdrlxv9Auy4ghB_GDBZdVWYtkKfgs4i8k&_nc_ht=scontent-frx5-2.xx&tp=6&oh=33570f80ee6aa2057d0b29fc664bd323&oe=60D41F1E
\fi
