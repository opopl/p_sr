% vim: keymap=russian-jcukenwin
%%beginhead 
 
%%file 25_11_2021.fb.kyshtymov_aleksandr.1.chub_spasenie_iz_ada
%%parent 25_11_2021
 
%%url https://www.facebook.com/permalink.php?story_fbid=270930078323017&id=100062180802989
 
%%author_id kyshtymov_aleksandr
%%date 
 
%%tags kazaki,ukraincy
%%title Чуб-спасение из Ада
 
%%endhead 
 
\subsection{Чуб-спасение из Ада}
\label{sec:25_11_2021.fb.kyshtymov_aleksandr.1.chub_spasenie_iz_ada}
 
\Purl{https://www.facebook.com/permalink.php?story_fbid=270930078323017&id=100062180802989}
\ifcmt
 author_begin
   author_id kyshtymov_aleksandr
 author_end
\fi

Чуб-спасение из Ада. 

\ifcmt
  ig https://scontent-frx5-2.xx.fbcdn.net/v/t39.30808-6/259515089_270930041656354_5371783771425445391_n.jpg?_nc_cat=109&ccb=1-5&_nc_sid=730e14&_nc_ohc=nuzJgMSIo8cAX9DUKPo&_nc_ht=scontent-frx5-2.xx&oh=00c49c6d19a886339572a38d624f183d&oe=61A78274
  @width 0.4
  %@wrap \parpic[r]
  @wrap \InsertBoxR{0}
\fi

Знаменитый длинный чуб оселедец — это особо важный атрибут внешности
запорожского казака. Отращивали его лишь опытные и закаленные в боях запорожцы.
Смысл в него закладывали сакральный: матерый казак в результате своих
похождениях был слишком грешен, и ада ему было не избежать; лишь за длинный
чуб, по легенде, милостивый Бог мог вытащить его из адских котлов.

\begin{itemize} % {

% -------------------------------------
\ii{fbauth.ovchinnikova_valentina.sevastopol.rossia}
% -------------------------------------

\ifcmt
  ig https://scontent-frx5-1.xx.fbcdn.net/v/t39.30808-6/261193665_1978638312317445_3524938630818558614_n.jpg?_nc_cat=105&ccb=1-5&_nc_sid=dbeb18&_nc_ohc=sPTNUHt0b_sAX_hGSjs&_nc_ht=scontent-frx5-1.xx&oh=2d2a795587999b4964f64d32abb21452&oe=61A6AC73
  @width 0.4
\fi

% -------------------------------------
\ii{fbauth.legkov_igor.simferopol.rossia}
% -------------------------------------

Потом Бог оторвал оселедец и бросил на майдан.

\iusr{Вячеслав Липатенков}

Чуб -то же хохол

\end{itemize} % }
