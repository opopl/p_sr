% vim: keymap=russian-jcukenwin
%%beginhead 
 
%%file 30_06_2021.fb.sajchuk_andrej.1.shevchenko_mova_sssr_futbol
%%parent 30_06_2021
 
%%url https://www.facebook.com/andriy.saychuk/posts/4588023977909610
 
%%author Сайчук, Андрей
%%author_id sajchuk_andrej
%%author_url 
 
%%tags evro2020.futbol.chempionat,futbol,jazyk,mova,shevchenko_andrej.ukr.futbolist,ukraina,ukrainizacia
%%title Для адептів "какая разніца, головне - результат": так давайте відродимо збірну СССР. А чо?
 
%%endhead 
 
\subsection{Для адептів \enquote{какая разніца, головне - результат}: так давайте відродимо збірну СССР. А чо?}
\label{sec:30_06_2021.fb.sajchuk_andrej.1.shevchenko_mova_sssr_futbol}
\Purl{https://www.facebook.com/andriy.saychuk/posts/4588023977909610}
\ifcmt
 author_begin
   author_id sajchuk_andrej
 author_end
\fi

Андрій Шевченко як тренер національної збірної зобов'язаний законом знати
українську мову і говорити українською на публічних заходах, де він бере участь
як посадова особа, чия діяльність оплачується з державного бюджету. Якшо в
хлопця з Яготина вже така тяжка біда з українською - він міг виступати на
брифінгу англійською чи італійською.

P. S. Для адептів \enquote{какая разніца, головне - результат}: так давайте відродимо збірну СССР. А чо?

\ifcmt
  pic https://scontent-mia3-2.xx.fbcdn.net/v/t1.6435-9/210458146_4588023921242949_5195860298387150011_n.jpg?_nc_cat=107&ccb=1-3&_nc_sid=8bfeb9&_nc_ohc=z1J07H6AI4cAX8zAMAs&_nc_oc=AQlO_Cr0baxtLMatI4BVR9QdUUz25Q0D_p-zeZgJs4MhC2cZRTmlgbkdCeC--Cb-aak&tn=lowUrFCbCbt-jOWu&_nc_ht=scontent-mia3-2.xx&oh=6bb900d5894dccd923d9213688f013d2&oe=60E60CC2
  width 0.6
\fi

\begin{itemize}
\item Upd. Як мені пояснили, УАФ, яка є роботодавцем Шевченка, не отримує
фінансування з держбюджету. Я не знав про цей нюанс - не надто цікавлюся
футболом, радше як культурним явищем. Але в такому випадку все є навіть
цікавішим. Виникає багато запитань. До прикладу: а хто фінансує УАФ? Хто
платить Шевченку (за даними ЗМІ 60 тис євро в місяць) зарплату? Чи означає
відповідь \enquote{футбольні клуби} те саме що й \enquote{олігархи}? І наскільки може
називатися \enquote{національною} організація, якій нецікава ні національна мова, ні
мовне законодавство?

\item Upd.2 Є потреба невеличкого дисклеймеру: шановні вболівальники, дозвольте
пояснити. Я  в жодному разі не хотів зачепити ваші почуття і підважити гордість
за перемогу збірної. Просто для мене все це - про ідентичність. Гейзинга писав,
що в сучасному світі футбол є безкровним замінником національних воєн менш
цивілізованих часів. А кожний українській війні - збройовій чи культурній -  я
так в першу чергу бачу змаг за ідентичність. Саме цим мені цікавий футбол. Тому
я так болісно реагую на російськомовність Шевченка: мені вона наче якщо не
перекреслює, то знецінює вчорашню блискучу перемогу України.

\end{itemize}

\ii{30_06_2021.fb.sajchuk_andrej.1.shevchenko_mova_sssr_futbol.cmt}
