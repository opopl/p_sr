% vim: keymap=russian-jcukenwin
%%beginhead 
 
%%file 04_12_2020.news.ua.obozrevatel.bondarenko_karina.1.kovid_statistics
%%parent 04_12_2020
 
%%url https://www.obozrevatel.com/society/koronavirusom-v-ukraine-zarazilis-bolee-15-tyisyach-za-sutki-statistika-minzdrava-na-4-dekabrya.htm
 
%%author Бондаренко, Карина
%%author_id bondarenko_karina
%%author_url 
 
%%tags 
%%title Коронавирусом в Украине заразились более 15 тысяч за сутки: статистика Минздрава на 4 декабря
 
%%endhead 
 
\subsection{Коронавирусом в Украине заразились более 15 тысяч за сутки: статистика Минздрава на 4 декабря}
\label{sec:04_12_2020.news.ua.obozrevatel.bondarenko_karina.1.kovid_statistics}
\Purl{https://www.obozrevatel.com/society/koronavirusom-v-ukraine-zarazilis-bolee-15-tyisyach-za-sutki-statistika-minzdrava-na-4-dekabrya.htm}
\ifcmt
	author_begin
   author_id bondarenko_karina
	author_end
\fi

\ifcmt
pic https://i.obozrevatel.com/news/2020/12/4/rt.jpg?size=972x462
\fi

По состоянию на утро пятницы, 4 декабря, в Украине официально \textbf{подтвержден 787
891} случай коронавирусной инфекции. Из них 15 131 --- обнаружены за прошедшие
сутки.

От COVID-19 в стране умерли уже 13 195человек (+235 за сутки), а 397 809
пациентов (+13 383 за сутки) полностью побороли болезнь. Об этом сообщил
министр здравоохранения Максим Степанов на своей странице в Facebook

За последние сутки наибольшее количество подтвержденных случаев
зарегистрировано в городе Киев (1374), Днепропетровской (1258), Одесской
(1211), Запорожской (1106) и Киевской (988) областях.

Проведено тестирований за сутки --- 74880 (в частности методом ПЦР --- 48 028,
методом ИФА --- 26 852).

\textbf{Количество заболеваний в регионах:}

\begin{itemize}
  \item г. Киев --- 76750 случаев;
  \item Винницкая область --- 18271 случай;
  \item Волынская область --- 25465 случаев;
  \item Днепропетровская область --- 42224 случаи;
  \item Донецкая область --- 24595 случаев;
  \item Житомирская область --- 31183 случая;
  \item Закарпатская область --- 24067 случаев;
  \item Запорожская область --- 34721 случай;
  \item Ивано-Франковская область --- 38588 случаев;
  \item Киевская область --- 41415 случаев;
  \item Кировоградская область --- 5888 случаев;
  \item Луганская область --- 6726 случаев;
  \item Львовская область --- 48819 случаев;
  \item Николаевская область --- 20358 случаев;
  \item Одесская область --- 49131 случай;
  \item Полтавская область --- 24532 случая;
  \item Ривненская область --- 32702 случая;
  \item Сумская область --- 32080 случаев;
  \item Тернопольская область --- 29007 случаев;
  \item Харьковская область --- 58030 случаев;
  \item Херсонская область --- 11398 случаев;
  \item Хмельницкая область --- 31412 случаев;
  \item Черкасская область --- 25492 случая;
  \item Черновицкая область --- 34710 случаев;
  \item Черниговская область --- 20327 случаев.
\end{itemize}

Данные с временно оккупированных территорий АР Крым, Донецкой, Луганской
областей и города Севастополя отсутствуют.
