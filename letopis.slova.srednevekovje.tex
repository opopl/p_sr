% vim: keymap=russian-jcukenwin
%%beginhead 
 
%%file slova.srednevekovje
%%parent slova
 
%%url 
 
%%author 
%%author_id 
%%author_url 
 
%%tags 
%%title 
 
%%endhead 
\chapter{Средневековье}
\label{sec:slova.srednevekovje}

%%%cit
%%%cit_head
%%%cit_pic
%%%cit_text
Теперь понимаете, почему на картинах всевозможных \enquote{рубенсов} так много
пышных, ни разу не худеньких женщин?  А поскольку у нас на дворе \emph{Новое
Средневековье}, то, возможно, диетологам скоро придется сильно
подкорректировать свои советы. А всем нам - подкорректировать мышление в
сторону более \enquote{научного}, но исключительно некритического.  Что ж это
за \emph{Новое Средневековье} без идеологических магов, патриотических
охотников на ведьм, инновационных метл, цифровой инквизиции и высокотехнолочных
аутодафе?  Хорошо еще, что украинский мир, населенный панночками и салом на
ночь, к такому развитию событий традиционно готов
%%%cit_comment
%%%cit_title
\citTitle{Украинский мир с его салом на ночь готов к новому Средневековью!}, 
Владислав Михеев, strana.ua, 23.06.2021
%%%endcit

