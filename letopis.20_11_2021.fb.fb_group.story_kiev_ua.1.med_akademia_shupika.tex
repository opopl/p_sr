% vim: keymap=russian-jcukenwin
%%beginhead 
 
%%file 20_11_2021.fb.fb_group.story_kiev_ua.1.med_akademia_shupika
%%parent 20_11_2021
 
%%url https://www.facebook.com/groups/story.kiev.ua/posts/1802173899979403/
 
%%author_id fb_group.story_kiev_ua,opolskaja_nina
%%date 
 
%%tags gorod,kiev,shupik_platon,ukraina
%%title Национальная медицинская академия имени Платона Лукича Шупика
 
%%endhead 
 
\subsection{Национальная медицинская академия имени Платона Лукича Шупика}
\label{sec:20_11_2021.fb.fb_group.story_kiev_ua.1.med_akademia_shupika}
 
\Purl{https://www.facebook.com/groups/story.kiev.ua/posts/1802173899979403/}
\ifcmt
 author_begin
   author_id fb_group.story_kiev_ua,opolskaja_nina
 author_end
\fi

ФБ напомнил и появилось желание написать об этом. Национальная медицинская
академия последипломного образования имени Платона Лукича Шупика. В ее стенах
каждые пять лет фармацевты и врачи повышают свою квалификацию. Имея опыт
работы, медики получают новые знания, ибо  в нашем мире все стремительно
развивается и оновляется. Занятия продолжаются около месяца и, получив заветный
сертификат, имеешь право на дальнейшую профессиональную деятельность. А ещё там
происходят и приятные встречи с однокурсниками, воспоминания о студенческих
годах. 

\ifcmt
  tab_begin cols=3

     pic https://scontent-frt3-1.xx.fbcdn.net/v/t39.30808-6/259357369_1819058181635128_2774250062394662476_n.jpg?_nc_cat=104&ccb=1-5&_nc_sid=b9115d&_nc_ohc=dZUy8FrEr0YAX9S2cka&_nc_ht=scontent-frt3-1.xx&oh=5c1349a1c62628a4f266f1670abed77a&oe=61A2DD75

     pic https://scontent-frx5-1.xx.fbcdn.net/v/t39.30808-6/259108259_1819058214968458_8424695524576645131_n.jpg?_nc_cat=111&ccb=1-5&_nc_sid=b9115d&_nc_ohc=DiRk38Ot8D4AX-PRw75&tn=lCYVFeHcTIAFcAzi&_nc_ht=scontent-frx5-1.xx&oh=1b094a1f5373c075339c8fe8432da517&oe=61A4110B

		 pic https://scontent-frt3-2.xx.fbcdn.net/v/t39.30808-6/258789668_1819058258301787_8008690300268850792_n.jpg?_nc_cat=103&ccb=1-5&_nc_sid=b9115d&_nc_ohc=PDeLliO61LEAX_ccVCT&_nc_ht=scontent-frt3-2.xx&oh=12f2a2e55dabd56a021d654149523f8b&oe=61A2F640

  tab_end
\fi

Академия- это одно из первых научных и  учебных заведений Украины, основанных в
период украинской государственности (1917-1919гг). Она основана на базе
Клинического института, который был создан в 1918 году, по инициативе
медицинской общественности Киева - профессионального союза киевских врачей.

Зарубежными партнёрами академии являются Колорадский университет
(США), Ягелонский университет (Польша), Шотландский королевский колледж врачей
общей практики (Великобритания) и многие другие. 

\ifcmt
  tab_begin cols=3

     pic https://scontent-frt3-1.xx.fbcdn.net/v/t39.30808-6/258835844_1819058301635116_8025770095628052115_n.jpg?_nc_cat=102&ccb=1-5&_nc_sid=b9115d&_nc_ohc=BcY6-EEsg40AX9jflvo&_nc_ht=scontent-frt3-1.xx&oh=d6add0ff59ed3e6f4663180083504ba2&oe=61A33051

     pic https://scontent-frx5-1.xx.fbcdn.net/v/t39.30808-6/259441454_1819058341635112_1955936119680080945_n.jpg?_nc_cat=110&ccb=1-5&_nc_sid=b9115d&_nc_ohc=53Nveayds60AX9FfK9s&_nc_ht=scontent-frx5-1.xx&oh=b93fa3f4d41f9abc15be6c29b84fa20b&oe=61A2D645

		 pic https://scontent-frx5-1.xx.fbcdn.net/v/t39.30808-6/258414109_1819058621635084_3267716569424759343_n.jpg?_nc_cat=110&ccb=1-5&_nc_sid=b9115d&_nc_ohc=4DEzxga70z4AX8WGLoo&_nc_ht=scontent-frx5-1.xx&oh=6df0413ea30d646d291d97ca5c639d8f&oe=61A40A87

  tab_end
\fi

В стенах академии с 1966 по 1980 гг работал Платон  Лукич Шупик, советский
врач-хирург, министр здравоохранения УССР, кандидат медицинских наук,
профессор, депутат Верховного Совета 3-го, 5 и 7 созывов. В 1989 году академия
названа его именем. Платон Лукич Шупик внёс великий вклад в развитие
отечественной медицины, покоится в Киеве. 

В академии помнят и чтят Платона  Лукича, стенды напоминают о его работе здесь.
С большим удовольствием обучаюсь здесь, когда приходит время... Как- будто
возвращаюсь туда, где молодость и беззаботные студенческие годы ...И горят
глаза, познавая что-то новое ...И делимся своими навыками по работе... И смех
лунает под сводами этого почтенного заведения, радость встреч и воспоминаний...
Особенно, с однокурсниками, которых не видел давно, но узнаешь с первого
мгновения. .Ах, какие были времена-говорим мы... Ой, какие сейчас прекрасные
мгновения общения... 

Занятия заканчиваются и мы произносим - До новых встреч, коллеги! Низкий поклон
преподам академии за данные знания и до новых встреч!

\ii{20_11_2021.fb.fb_group.story_kiev_ua.1.med_akademia_shupika.cmt}
