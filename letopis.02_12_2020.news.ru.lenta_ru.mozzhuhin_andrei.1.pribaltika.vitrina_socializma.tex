% vim: keymap=russian-jcukenwin
%%beginhead 
 
%%file 02_12_2020.news.ru.lenta_ru.mozzhuhin_andrei.1.pribaltika.vitrina_socializma
%%parent 02_12_2020.news.ru.lenta_ru.mozzhuhin_andrei.1.pribaltika
 
%%url 
 
%%author 
%%author_id 
%%author_url 
 
%%tags 
%%title 
 
%%endhead 
\subsubsection{Витрина социализма}
\label{sec:02_12_2020.news.ru.lenta_ru.mozzhuhin_andrei.1.pribaltika.vitrina_socializma}

\lenta{Как Москва проводила повторную советизацию Прибалтики после освобождения от
немцев и в послевоенные годы? Она учла ошибки 1940-1941-го?}

Конечно. Ресоветизация там проходила осторожно и продуманно, с учетом местной
специфики. Например, коллективизация проводилась в очень мягких формах.

\ifcmt
tab_begin cols=3
	caption Лесные братья

	pic https://icdn.lenta.ru/images/2020/10/20/16/20201020162245239/pic_04e95a2792dfda67790a07e587d97dfa.jpg
	caption Руководители штаба отряда «лесных братьев» в Литве. 5 апреля 1949 г.  Фото: genocid.lt

	pic https://icdn.lenta.ru/images/2020/10/20/16/20201020162248332/pic_7727c580ed29a40a1a098d1204267675.jpg
	caption Один из лидеров литовских «лесных братьев» Антанас Случка-Шарунас со своими девушками-связными. Не позднее 1949 г. Фото: genocid.lt

	pic https://icdn.lenta.ru/images/2020/10/20/16/20201020162251731/pic_cb1f382a3c4579d5f13ea188b2329cea.jpg
	caption «Лесные братья» из Дайнавы (Дзукии) — юго-восточной части Литвы. Апрель 1948 г.  Фото: genocid.lt

tab_end
\fi

Особый акцент советская власть сделала на возрождении национальных культур
прибалтийских народов. В частности, университет в Вильнюсе вновь открыли сразу
же после освобождения Литвы от немецкой оккупации, не дожидаясь окончания
войны.

На восстановление лежащих в руинах Литвы, Латвии и Эстонии Москва бросила
колоссальные ресурсы, нередко в ущерб другим республикам. И такая политика
продолжалась все годы советской власти. В развитие местной промышленности,
повышение жизненного уровня и благоустройство союзные власти постоянно
вкладывали огромные средства.

\begin{leftbar}
				\large
В результате Прибалтика превратилась в зажиточную советскую
«витрину социализма» и стала одним из немногих привилегированных
регионов, регулярно получавших помощь из центра. В СССР она считалась
«нашим Западом» — вроде и своей территорией, но все-таки немного чужой
\end{leftbar}

Это чувствовалось и в быту — там даже к дачникам из России относились с
прохладцей. Не зря и многие фильмы про заграничную жизнь снимались в Риге,
Таллине и Вильнюсе.

\lenta{Современные литовцы, латыши и эстонцы наверняка в ответ вам возразят, что все
это достигалось за счет принудительной русификации и поощрения миграции из
России, Украины и Белоруссии.}

Современные литовцы, латыши и эстонцы — очень разные. Не надо всех причесывать
под одну гребенку. Вопрос, какова официальная историческая политика. Насчет
притока приезжих — не будем забывать, что именно стараниями этих людей удалось
не только очень быстро поднять Прибалтику из руин, но и превратить ее из
преимущественно аграрного края в самый экономически развитый промышленный
регион Советского Союза.

Поощрение Москвой массовой русскоязычной миграции способствовало встраиванию
Прибалтики в советскую систему. Языковая проблема — больной вопрос,
обоюдоострый. Скажем прямо — люди, приезжавшие в Прибалтику из других регионов
СССР, не считали нужным учить национальные языки, хотя оставались там на всю
жизнь или как минимум жили там десятилетиями. Это, конечно, не могло не
раздражать местное население.

Заметим: сами коренные жители Литвы, Латвии и Эстонии тоже весьма охотно
приспосабливались к советской власти. Многие представители не только
политических, но и культурных элит этих республик сознательно уезжали учиться в
Москву, Ленинград или, допустим, в Казань. Немало артистов из Прибалтики
сделали успешную карьеру в советском кино, театре или на эстраде. Получить
всесоюзную известность для них считалось очень престижным. Прибалтику удалось
надолго замирить, но, как показали события конца 1980-х годов, отнюдь не
усмирить.

Теперь мы видим, что во всех странах Прибалтики, несмотря на все их различия,
память о Второй мировой войне стала одной из самых болезненных точек массового
сознания. Заметьте, именно война, а не события после революции 1917 года, не
краткий период независимости, не ее восстановление после перестройки.
