% vim: keymap=russian-jcukenwin
%%beginhead 
 
%%file 13_02_2022.stz.news.ua.strana.3.tenj_kabula
%%parent 13_02_2022
 
%%url https://strana.news/news/376444-chto-zapadnye-smi-pishut-ob-anonsirovannoj-ssha-vojne-rossii-s-ukrainoj-.html
 
%%author_id harchenko_aleksandra
%%date 
 
%%tags rossia,ugroza,ukraina,vtorzhenie
%%title "Тень Кабула". Что пишут о "вторжении" России в Украину западные СМИ
 
%%endhead 
 
\subsection{\enquote{Тень Кабула}. Что пишут о \enquote{вторжении} России в Украину западные СМИ}
\label{sec:13_02_2022.stz.news.ua.strana.3.tenj_kabula}
 
\Purl{https://strana.news/news/376444-chto-zapadnye-smi-pishut-ob-anonsirovannoj-ssha-vojne-rossii-s-ukrainoj-.html}
\ifcmt
 author_begin
   author_id harchenko_aleksandra
 author_end
\fi

\raggedcolumns
\begin{multicols}{3} % {
\setlength{\parindent}{0pt}

Информационная лавина на тему якобы близящегося \enquote{вторжения Путина} заполонила
передовицы западных СМИ.

Британские, американские, немецкие, французские и издания других стран мир с
обложек транслируют озвученные Вашингтоном призывы покинуть Украину.  

При этом некоторые прочат повторения афганской истории с позорной сдачей Кабула
\enquote{Талибану} и провальной эвакуацией, когда люди цеплялись за самолеты.

\ii{13_02_2022.stz.news.ua.strana.3.tenj_kabula.pic.1}

\enquote{Помочь американцам в случае бегства}

\enquote{Украинский кризис: тень Кабула нависла над планами эвакуации США}, - с таким заголовком вышла британская Times. 

Издание пишет, что президент Байден одобрил планы размещения американских войск
в Польше, чтобы \enquote{помочь эвакуировать тысячи американцев из Украины в
случае вторжения России, поскольку Белый дом стремится избежать повторения
хаотического вывода войск из Афганистана}. 

Польские и американские войска начнут строить палаточные лагеря и
контрольно-пропускные пункты, \enquote{чтобы помочь американцам в случае
бегства}.

Впрочем, в Украине пока нет паники, чем искренне удивляются западные СМИ.

\enquote{В условиях, когда иностранные посольства отзывают сотрудников, а ряд
стран теперь призывают своих граждан покинуть Украину, Киев все еще не
чувствует себя городом, находящимся в кризисе}, - отмечает дипломатический
корреспондент BBC Паули Адамс.

\ii{13_02_2022.stz.news.ua.strana.3.tenj_kabula.pic.2}

\enquote{Правительство здесь говорит людям сохранять спокойствие и единство и, как
говорится в сегодняшнем заявлении, воздерживаться от действий, которые
подрывают стабильность и сеют панику. Президент Зеленский заявил, что страна
должна быть готова к любым неожиданностям. По всей Украине иностранные граждане
сейчас строят поспешные планы. Стюарт Маккензи, который живет в Киеве уже 28
лет и ведет успешный бизнес, надеется отправить жену и двух сыновей на самолет.
Но он готов, если надо, запихнуть семью в машину и проехать 300 верст до
Польши. Он любит Украину и не может поверить, что до этого дошло}, - продолжает
корреспондент BBC.

В британском посольстве молчаливые сотрудники загружают сумки в машину и
уезжают. Казалось, никто не хочет говорить.

\enquote{Недалеко к северу, за границей в Белоруссии, сейчас полным ходом идут военные
учения России. На фотографиях Министерства обороны России, опубликованных
сегодня утром, видны выстрелы из нескольких реактивных установок. Москва
по-прежнему заявляет, что не планирует вторжения. Но русские могут многое
сделать, даже не ступая на территорию Украины}, - считает автор BBC.

\ii{13_02_2022.stz.news.ua.strana.3.tenj_kabula.pic.3}

\end{multicols} % }
