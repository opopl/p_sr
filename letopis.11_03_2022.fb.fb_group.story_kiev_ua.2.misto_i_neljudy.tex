% vim: keymap=russian-jcukenwin
%%beginhead 
 
%%file 11_03_2022.fb.fb_group.story_kiev_ua.2.misto_i_neljudy
%%parent 11_03_2022
 
%%url https://www.facebook.com/groups/story.kiev.ua/posts/1879171105613015
 
%%author_id fb_group.story_kiev_ua,grabar_sergij.kiev
%%date 
 
%%tags 
%%title Місто і нелюди
 
%%endhead 
 
\subsection{Місто і нелюди}
\label{sec:11_03_2022.fb.fb_group.story_kiev_ua.2.misto_i_neljudy}
 
\Purl{https://www.facebook.com/groups/story.kiev.ua/posts/1879171105613015}
\ifcmt
 author_begin
   author_id fb_group.story_kiev_ua,grabar_sergij.kiev
 author_end
\fi

Місто і нелюди

Цей день таки прийшов. Ми його боялися, ми не вірили, що це може статися за
нашого життя. Ми знаходили безліч причин, щоби не чекати його. Говорили про
час, у якому не може бути подібного страхіття, твердили про людей розумних і
розвиток цивілізацій. Зрештою, закликали до здорового глузду. І не вірили самі
собі, коли десь усередині, у закутках нашої свідомості цівотіла непрохана,
захекана думка: «А що, як раптом?..»

Але цей день прийшов. І ім’я йому – Війна...

\ifcmt
  ig https://scontent-mxp1-1.xx.fbcdn.net/v/t39.30808-6/275384445_4905487086236607_8531460986764841922_n.jpg?_nc_cat=105&ccb=1-5&_nc_sid=5cd70e&_nc_ohc=G57fozgWJEkAX81rd0V&_nc_ht=scontent-mxp1-1.xx&oh=00_AT_ZDVIFPLQnbuKv_rlUtDfb2D0HB3m-HB1OQKohTAzVig&oe=62644689
	@caption Картина народного художника України Володимира Слєпченка
  @wrap center
  @width 0.8
\fi

У цей день застиглого лихоліття, я ще більше відчув подих свого міста,
затиснутого з усіх боків здичавілою ордою, яка так і залишилася на тому
первинному рівні життя, яке ми давно пройшли, проминули. Настільки давно, що
навіть і не пам’ятаємо, чи колись теж були такими. Ні, мабуть, такими, як вони,
ми ніколи не були, бо нелюдами народжуються, від дитинства напиваючись
нелюдським отруйним молоком, дихаючи нелюдським повітрям, слухаючи нелюдських
пророків. А як може бути інакше, коли і батько, і мати, і діди, і прадіди – всі
нелюди.

Київ у житті своєму не раз бачив нелюдів. Така вже доля великого Вічного міста:
страждати, горіти, гинути, щоби з часом повставати з попелу і  відроджуватися.

«Року 1096 від Різдва Христового, травня місяця, двадцять п’ятого дня, у неділю
звечора прийшов хан Боняк із половцями до Києва. І спустошили вони окіл Києва,
і спалив він на селі Берестовім княжий двір».

Невеличке літописне сповіщення, за яким  -- смерть, згорьовані обличчя, спалене
житло і героїчний захист. Київ тоді встояв, зберігся, зумів дати відсіч
нелюдам. Ми нічого не знаємо про захисників міста, але знаємо, що за кілька
днів відступили каральні загони кіпчаків. Так, було зруйновано Берестове, але
величний Печерський монастир залишився незламним, новозбудований храм Успіння
Пресвятої Богородиці у ньому – неушкодженим, Богородиця Оранта у Святій Софії
стояла непорушною стіною. І осяював місто його небесний покровитель –
Архистратиг Михаїл...

«Року 1169 від Різдва Христового. Узятий же був Київ місяця березня у
дванадцятий день, у середу другої неділі посту. І грабували вони два дні увесь
город – і Поділ, і Гору, і монастирі, і Софію, і Десятинну Богородицю. І не
було помилування анікому і нізвідки? Церкви горіли, християн убивали, а других
в’язали, жінок вели в полон, силоміць розлучали із мужами їхніми, діти ридали,
дивлячись на матерів своїх. І взяли вони майна безліч, і церкви оголили од
ікон, і книг, і риз, і дзвони познімали всі – і всі святині було забрано».

Те, що діялося тоді у Києві об’єднаними військами кількох князівств під орудою
суздальського князя Андрія Боголюбського -- сина Юрія Довгорукого, онука
Володимира Мономаха, сьогодні назвали би геноцидом. До Києва прийшли нелюди, і
не якісь зайди з далеких земель, а такі самі слов’яни-християни. Прийшли нищити
і грабувати, ґвалтувати жінок  і плюндрувати святині, сиротити дітей, лишаючи
їх опіки та відганяючи у рабство. Тоді ніхто не думав, що ці ніби свої –
смольняни, суздальці, чернігівці, виявляться жорстокішими за будь-яких
найманців і принесуть Києву та його мешканцям «стогін, тугу і скорботу невтішні
та сльози безперестанні».

Одну із найдавніших святинь землі Руської – ікону Вишгородської Божої Матері
було викрадено і перевезено спочатку до Суздаля, а згодом до Володимира. Вона і
до сьогодні там, під ім’ям Володимирської Божої Матері, у нелюдів-загарбників.

Пройшов час і місто відновилося, бо завжди тут жили сміливі, працьовиті та
доброчесні люди. Народжувалися діти, до земель наших з усього світу потягнулися
торгові каравани. І все такою же Непорушною стіною височіла Богородиця Оранта у
Святій Софії, і все так же боронив місто очільник воїнства Небесного –
Архистратиг Михаїл…

«Року 1240 від Різдва Христового прийшов Батий до Києва з великою силою і
окружив город. І обступила Київ сила татарська, і був город в облозі великій. І
не було чути нічого од звуків скрипіння теліг його, ревіння безлічі верблюдів
його, і од звуків іржання стад коней його, і сповнена була земля Руська
ворогами».

У Києві стояла пізня осінь. Ручаї та маленькі річки вже почали промерзати.
Військо, що стояло під Києвом, було велелюдним. Огром людей, коней, верблюдів
зосередився біля найменш укріплених воріт міста -- Лядських. Усі, задіяні під
час воєнної кампанії монголів, 32 метальні машини – пороки стояли напоготові.
Саме вони й розпочали вдень і вночі методичне бомбардування міста. Київ,
відкинувши пропозиції про капітуляцію, тримався довго. Один за одним спадали
оборонні заборола. Останньою твердинею захисників міста стала кам’яна Десятинна
церква, яка від бомбардування пороками поглинула киян.

Море нелюду розпливлося по Києву, знищуючи все живе на своєму шляху, грабуючи і
ґвалтуючи. З сорока тисяч населення міста живими залишилося трохи більше двох
тисяч.

Пройшло більше 80-и років, коли після битви на річці Ірпені з новою силою
засяяв золотом у Святій Софії величний лик Оранти-Заступниці та розкрив свої
крила на небі Архистратиг Михаїл…

«Року 1482 від Різдва Христового, прийшов Менґлі-Ґірей под град на день Семена
Лєтопроводца, в першу годину дня, і відправив полки і підступив к граду. І град
спалив, і загинули люди і казна».

Цього разу десяти тисячне військо татарського хана Менґлі-Ґірея у спільноті із
загонами московського царя Івана ІІІ оточили Київ, в якому за розрахунками
перебувало тоді трохи більше п’яти тисяч мешканців. Місто, не зважаючи на
відчайдушний опір, було взято і пограбовано. Нелюди отримали наказ не залишати
нічого живого.

Коли з усього Великого князівства Литовського на допомогу Києву прибуло сорока
тисячне військо, їх вийшли зустрічати всього декілька сотень киян.

Московський цар Іван ІІІ отримав від свого татарського союзника на знак дружби
золоті потир та дискос із сплюндрованої Софії Київської.

До відновлення давнього міста було залучено близько двадцяти тисяч сокир. Київ
швидко піднявся з руїн. І знову матір Оранта простягнула до Неба свої руки, і
небесний Архистратиг  Михаїл підніс свого меча…

«Року 1918 від Різдва Христового місяця січня, у двадцять третій день під
Києвом зосередилися каральні війська російської армії під керівництвом Михайла
Муравйова. Артилерія почала обстрілювати місто».

І знову загарбники прийшли до Києва. Те, що це були саме нелюди розповідає
запис, що полишив головнокомандувач спеціальної операції Михайло Муравйов:

«Ми йдемо вогнем і мечем встановлювати радянську владу. Я зайняв місто, бив по
палацах і церквах... бив, нікому не даючи пощади! 28 січня Дума Києва просила
перемир’я. У відповідь я наказав душити їх газами. Сотні генералів, а може й
тисячі, було безжалісно вбиті. Так, ми мстилися. Ми могли зупинити гнів помсти,
проте ми не зробили цього, тому що наше гасло – бути нещадними!»

За даними джерел тими днями у Києві було страчено близько тисячі офіцерів і
генералів. Жертви серед цивільного населення сягали до п’яти тисяч осіб.

Після встановлення радянської влади Оранта-Заступниця на багато десятиліть
лишалася зачиненою за монастирськими мурами, а Архистратигу Михаїлу намагалися
знайти альтернативу у вигляді квіточки каштана...

«Року 1941 від Різдва Христового, місяця вересня, у дев’ятнадцятий день – після
довгих і напружених боїв радянські війська залишили місто Київ».

Київ разом з більшістю мешканців опинився сам на сам перед німецькими
загарбниками -- нелюдами із Західної Європи. Битва за Київ була частиною
величезного наступального плану «Барбаросса» і тривала 72 дні. Після приходу
окупантів, почалися репресії: арешти, розстріли, табори, Бабин Яр, відправлення
остарбайтерів. Так тривало всі 778 днів і ночей фашистського терору. Але
мешканці міста – кияни пройшли крізь пекло і вижили.

І знову засяяла усіма проявами сонця Матір Богородиця у Святій Софії, і десь
поміж хмарами почало проблискувати окрилля покровителя Києва – Архистратига
Михаїла...

«Року 2022 від Різдва Христового, місяця лютого, у двадцять четвертий день, о
п’ятій годині ранку четверга почалося вторгнення до України нелюдів із
сусідньої з нами держави, що зветься Росія».

Я пишу ці рядки, а за вікном чутно канонаду – йдуть бої у передмісті Києва.
Тривають вже не перший день, навіть, вже не перший тиждень. Нелюди не можуть
взяти наше місто – і це знак того, що ми непохитні, непереможні.

День війни прийшов до нас зненацька, але потрібно було зовсім трохи часу, щоби
відчути себе пліч-о-пліч з такими же, як ти  -- громадянами вільної країни,
громадянами вільного міста.

Нелюди завжди були ласими до наших чеснот, не розуміючи їх, до нашої краси, не
відчуваючи її. Нелюди приходили звідусіль: зі сходу і заходу, півдня і півночі.
Їм так хотілося Києва, його величі. Інколи вони захоплювали місто, грабували,
нищили, але ніколи не відчували його дихання, не могли осягнути Вічності його.
І йшли, поволі помираючи у тяжких муках, не розуміючи, чому до них так швидко
прийшла кара Небесна.

Вони знову прийшли сюди, до Києва, до своєї погибелі... сьогодні, може завтра,
може за три дні, але вже скоро...

Місто наше Вічне -- Києве! Благослови на подвиг, дітей твоїх, Матір Богородиця
-- Оранта-Заступнице! Захисти крилами своїми -- покровителю Києва Архистратиг
Михаїл!

08-11 березня 2022 року

Картина народного художника України Володимира Слєпченка

\ii{11_03_2022.fb.fb_group.story_kiev_ua.2.misto_i_neljudy.cmt}
