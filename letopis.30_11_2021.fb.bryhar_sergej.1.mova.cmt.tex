% vim: keymap=russian-jcukenwin
%%beginhead 
 
%%file 30_11_2021.fb.bryhar_sergej.1.mova.cmt
%%parent 30_11_2021.fb.bryhar_sergej.1.mova
 
%%url 
 
%%author_id 
%%date 
 
%%tags 
%%title 
 
%%endhead 
\subsubsection{Коментарі}

\begin{itemize} % {
\iusr{Оксана Ковалишин}
Співчуваю. \enquote{Мужичьонка} той нагадує мені опариша.

\iusr{Olga Tumenko}

Вони всі ті так не люблять ні нашу мову, ні Україну, але їх дрином з України не
вигониш. Тому, що в раші вони будуть ніким і ніхто, а тут можуть качати права. І
ніякий закон для них не закон, тому, що не має ніякого за цим державного
контролю.


\iusr{Yaroslav Datsko}

\enquote{Думаю, якби обставини склалися відповідним чином, ті, що там, дійсно могли б і
повторити жахіття 1932-1933 років чи новий 1937-й} 
\enquote{С удовольствием}

\iusr{Semeniuk Oleksandr}
Креол й***ний, так я їх називаю.

\iusr{Євгенія Чуприна}
Це все публіка, на яку треба лиш трохи натиснути, і вона тебе полюбить назавжди

\begin{itemize} % {
\iusr{Людмила Демянчук}
\textbf{Євгенія Чуприна}
Та ніколи не полюбить !
Затаїться до слушного часу, і піде асвабадітєлєй з квітами зустрічати ...

\iusr{Olga Tumenko}
\textbf{Людмила Демянчук} Це точно,як в Криму.

\iusr{Volodymyr Tarnavskyi}
Не полюбить)

\iusr{Євгенія Чуприна}
\textbf{Людмила Демянчук} полюбить, якщо просто показати

\iusr{Людмила Демянчук}
\textbf{Євгенія Чуприна}
Показати що ?
\end{itemize} % }

\iusr{Тетяна Лук'янова}

А у самого його батько з якої дерьовні? Забула вже прізвище того покидька, який
в Одесу з якоїсь ass-трахані у 2011 чи 12-му приперся, то теж тюльку травив про
село і який він корінний одесит. А як війна почалася, то чкурнув, аж закурилося
за ним.


\iusr{Людмила Демянчук}
Зі мною працювала корінна росіянка фельдшером. Не говорила українською ні слова.
Вже була старша, воювала підчас ВВв, і дуже плутала в документації, і коли я їй в надцятий раз звернула увагу , вона мені каже :
- Людміла Андрєєвна !
Как ви мнє надоєлі, єслі би била война, я би взяла калаш і стрєляла би крєст накрєст !
А була ця фельдшерка в розстрільній трійці.
І нічого її життя не навчило ...

\begin{itemize} % {
\iusr{Анна Винокурова}
Не плутаймо - ДСВ а не ВВВ
А взагалі так, це страшні психічно хворі люди

\iusr{Людмила Демянчук}
\textbf{Анна Винокурова}
Не плутаєм, я назвала так, як війну називала моя фельдшерка.
\end{itemize} % }

\iusr{Tanya Volska}

Я б впровадила Тест на знання державної мови для отримання права голосу. Хочеш
голосувати - пройди тестування. Хочеш балотуватися - пройди тестування. Ще й
тест на знання загальної історії України а також історії регіону, де живеш. І
ніяких питань не буде.

\iusr{Вадим Руденко}

Держава - не щось окреме, далеке, абстрактне. Державу складають ті, хто мають
її громадянство. Тобто - усі ми.

У державі вирішує більшість. За демократичною процедурою.

З \enquote{нацменами} - картина вкрай сумна. Але ж вирішальне слово не за \enquote{нацменами}.
За останнім переписом тих, хто вважає себе українцем, у країні - 78\%. От де
собака заритий!..

У країнах Прибалтики серед титульних націй людей, що йдуть проти своєї мови, -
майже нуль. Не теє в нас... Про Білорусь вже й не кажу!..

\begin{itemize} % {
\iusr{Анна Винокурова}
\textbf{Вадим Руденко} 78\% вважають українську рідною. А українцями себе вважають майже 90\%

\iusr{Вадим Руденко}
\textbf{Анна Винокурова} і - що ж так слабо видно-чутно?..
За кого голос віддають?..
Чи ж зеля - в чомусь українець?..

\iusr{Volodymyr Tarnavskyi}
Ну так поняття українець в кожного різне... В багатьох з оцих 78 чи 90 воно наближається до росіян...

\iusr{Вадим Руденко}
\textbf{Volodymyr Tarnavskyi} саме так. Тому і мали вибори-2019. І ще, на жаль, буватиме...

\iusr{Volodymyr Tarnavskyi}

В нас українців є така дуже хрінова риса характеру як марнославство. От
назвались ватники українцями, а ми сидимо і пишаємось, що нас 78, 90, 150\%.
Хоча це найсправжнісінька московська маніпуляція, яка тільки збиває з
пантелику, і викликає парадокс - якщо вони теж українці, то теж мають право на
свою візію України. Тоді розмивається саме поняття українець і Україна. Користі
ніякої, а шкоди ой як багато.


\iusr{Вадим Руденко}
\textbf{Volodymyr Tarnavskyi} а хто - пишається-то? Лиш - статистика.

\iusr{Volodymyr Tarnavskyi}
\textbf{Вадим Руденко} ну погодьтеся, нас всіх гріють ці фіктивні цифри?)))

\iusr{Вадим Руденко}
\textbf{Volodymyr Tarnavskyi} кого - вас?
\end{itemize} % }

\iusr{Ronald Niberg}

Тут ще один цікавий аспект. Хоча в селі точно що нічого поганого немає; вихідці
із міських люмпенів, лакеїв та прислуги точно що значно поступаються розвитком
селянам. Але на даний момент біля 70\% мешканців Московії живе у селах або т.з.
поселеннях міського типу. Тоді як в Україні близько 50\% живуть у містах або
спальних приміських супутниках. Тому якщо вже говорити про мову дєрьовні, це
беззаперечно московська.

\begin{itemize} % {
\iusr{Svitlana Chub-Krywuzka}
А якщо говорити про колхозний язик, то це й подавно московська. Колхози це їхнє.

\iusr{Volodymyr Tarnavskyi}
Звідки такі дані?  @igg{fbicon.face.tears.of.joy}  3/4 кацапів живуть в містах, в нас десь 2/3
\end{itemize} % }

\iusr{Валерій Корнєєв}
Чекають і ждуть повернення імперійки...

\begin{itemize} % {
\iusr{Анатолий Чабан}
\textbf{Валерій Корнєєв} Свого часу чекали комунізму. Навіть приблизно час озвучили.

\iusr{Валерій Корнєєв}
\textbf{Анатолий Чабан} та тут хіба мідни таз накриє, а років через 800 цивілізація як у «Дюні» відродиться.

\iusr{Анатолий Чабан}
\textbf{Валерій Корнєєв} На теренах Раші?)))

\iusr{Валерій Корнєєв}
\textbf{Анатолий Чабан} там буде Великий Валдайсько-Тюменський океан
\end{itemize} % }

\iusr{Павло Шубарт}

Московська мова - це говірка аборигенів підмосковних дєрєвєнь. До Пушкіна уся
московська аристократія послуговувалась французькою та німецькою, вважаючи
т.зв.руССкій ізик мовою кріпаків.

\begin{itemize} % {
\iusr{Наталья Федоркова}
\textbf{Павло Шубарт} руццкій язик вигадали в часи Каткрини ||, щоб змусити різноплеменне фіно-угро-тюркське населення розмовляти однією мовою, щоб легше ними було управляти. Так і з'явився штучний, примитивний руцццкій язик.
\end{itemize} % }

\iusr{Олена Єрьоменко}
Підтримую

\iusr{Сергій Гуцалюк}

Вони не візьмуть до рук зброю проти України, і це головне бо не має в них духу
та сміливості, все вже відбулось у 2014. А патякати все одно що на паркані
написати х.. й, а за ним дрова. Думаю не потрібно на них реагувати, лише до
того моменту як що вони не почнуть закликати окупанта. Тоді різка відсіч.

\begin{itemize} % {
\iusr{Solomia Shtephan}
\textbf{Сергій Гуцалюк} візьму ь, ще й як. А ще є атршники, які проти української мови

\iusr{Анна Шептинська}
\textbf{Сергій Гуцалюк} А згадайте недалекий час: Маріуполь і їх барикади, крізь які проривалися наші танки, а реакція їх на полонених в Донецьку, а полонений льотчик.... Аж не хочеться згадувати все решта.. На мою думку це як троянський кінь в середині держави.

\iusr{Світлана Перч}
\textbf{Сергій Гуцалюк} 

зброю у руки не візьмуть. Візьмуть ікони і ляжуть під танки. Під українські
танки, не пускаючи українців, захищати Україну. Вже так було. Ворог, який у
тебе стріляє це набагато кращий ворог. Бо у нього, можна стріляти у відповідь.
А що робити з \enquote{божими одуванчиками}, у гіпюрових хусточках, які з іконами і
молитвами лягають під танк. Бо вони - не допустять \enquote{братоубійство}. І що з ними
робити ? Танками переїхати, на радість Скабєєвій ? Чи довго і нудно умовляти ?
Доки тим бабцям поясниш, то к@ц@п до Києва дійде.

\iusr{Лариса Артем'єва}
\textbf{Сергій Гуцалюк} Справа не в тому що візьмуть зброю. а в тому, що мають поважати мову, історію, культуру держави в якій живуть. чи вони тут живуть як паразити.?
\end{itemize} % }

\iusr{Алла Шубрікова}

Мене теж бісять люди, які вже років зо 50 бовтаються у цій теплій ванні неньки
України, але тупо та вперто ні слова українською, а дивляться виключно руські
канали до цих пір!!! При тому що і вони, і їх діти дуже-дуже добре
живуть-поживають не на мокшандії, а тут, "гдє тєпло і много фруктов" і звідки
їх ( як дуже влучно тут хтось написав їх дрином не вигониш. Вони тут чекають
расєю. І так, ще й зверхньо дивляться на мєстних аборигенів, які не розмовляють
чомусь! Руським язиком... бо всі повинні розмовляти ним, а не мовою села.
Просто тьфу та ще раз тьфу

\begin{itemize} % {
\iusr{Тетяна Пашко}
\textbf{Алла Шубрікова} з цією нацією щось не так . Недавно їхала в турпоїздку. Чоловік і жінка, які мають троє дітей, вихвалялися як їх діти навчилися російської "дякуючи" телефону, російським сайтам. І як діти поправлять батьків, бо "нада так гаваріть" . А кажу , а що далі буде? Та ми їх потім перевчимо (???!!!). Шизофренія! Горе! І знаєте, руки опускаються, коли дивишся і чуєш як мислить молода людина. Абсолютно нема в них бажання берегти українське. Воно саме таке, сидить на російських сайтах, слухає оцю дурну попсу, за хвилину ти їх ні в чому не переконаєш. Воно нічого не читає, от на таких і розрахований "руськіймір" - легка здобич.
\end{itemize} % }

\iusr{Ntina Ntoubrova}
Ну так і є. З 1848го року світ живе в національних державах. Або спрямований на це. Росіяни, що є вихідцями з РФ якщо вони самі на цьому наголошують постійно - їм має нагадуватися, що вони ІМІГРАНТИ. Такі самі, як араби - у Франції.

\iusr{Вікторія Ільченко}
Згодна

\iusr{Jurko Zełenyj}

\obeycr
qassaby 14 stolitia i XXI - odne j te same!
I ne zminiujut sì vony ne tiko w sij \#бувшій\_УССР, ale j po cilim śviti!!
Szo w Warszawi, szo w Berlini, szo Londoni, szo w Novomu Yorku - vże povirte!
Same tomu jihnia SPRAWŻNIA nazva maje buty jek «biośmitė Lüdstva»!
A vidomo, że śmitė tra zbyraty organizovano j utylizuvaty go!
Vid sego Śvit ta Lüdstvo stalo by tiko szystiszym, myrniszym, vyvażeniszym, kulturniszym, etc, etc, etc.
ALE...
Nażal, śvit sze do sego nijak ne goden dozrity - osoblyvo po II Śvitovij,
koly posadyw qassabiw za stił suddiw w Nürnberġu,
a na na lavu pidsudnyh....  @igg{fbicon.frown} 
\restorecr

\iusr{Анна Винокурова}
\#рашизм \#мова\_має\_значення

\iusr{Yuriy Yastreb}

Він не просто ворог, він - нацист, і подібних покидьків слід регулярно так
називати. Це дещо змінить ставлення до них в супільстві - не на користь їм.


\iusr{Галина Шередько}

\ifcmt
  ig https://scontent-lhr8-1.xx.fbcdn.net/v/t39.30808-6/263071345_1067394110759004_8922036364670119160_n.jpg?_nc_cat=110&ccb=1-5&_nc_sid=dbeb18&_nc_ohc=H4EdFu1rmuoAX83y9EB&_nc_ht=scontent-lhr8-1.xx&oh=dfc7f6d3ecf1cc81092a770615bea14e&oe=61B4D1BE
  @width 0.4
\fi

\iusr{Надія Зволінська}

Підтримую. В питанні мови толерантність ні до чого, тільки жорсткі закони без
виключень. На всіх рівнях державна мова одна - українська і крапка.

\iusr{Віктор Іліка}

З тими хто з ,, середньої полоси Росії,, все ясно. До них питань нема. Для них
Росія понад усе, хоча вони і не кричать про це на кожному кроці.

А як бути з нашими,доморощеними манкуртами, які аж кінчають від всього
російського, таке воно їм миле. Адже не ті що ,,понахалі,, роблять касу
проросійським блогерам, артистам, попам, журналістам, співакам, політикам, а
наші рідні, від плуга і серпа,так сказати, земляки.
З документами у них все нормально.

Як з ними бути.

\iusr{Volodymyr Tarnavskyi}

Сергію, ну не буде тесту на знання мови для отримання громадянства, не буде
статусу негромадян... Це все утопія і фантастика...

\begin{itemize} % {
\iusr{Петро Степанець}
\textbf{Volodymyr Tarnavskyi} 

яка утопія, проукраїнська влада і як миленькі заговорять. І не треба примушувати
розмовляти ,треба просто примусити виконувати закон. За кордоном ніхто не стоїть
з нагайкою, тебе просто без відповідного рівня знання мови не візьмуть на
відповідну роботу. Тобі просто без відповідного рівня знання не дадуть дозвіл на
проживання ,тобі без відповідного рівня знання не дадуть громадянство. Яка
утопія?????? То ввас в голові, утопія. Всі хто повиїзжав з Донбасу за кордон, всі
знають мову тієї держави в якій вирішили приютитись.

Яка утопія!!!!!Лише треба проукраїнську владу.А ось тут уже питання до народу......

\iusr{Volodymyr Tarnavskyi}
\textbf{Петро Степанець} 

от у вашому коментарі прекрасно все  @igg{fbicon.face.tears.of.joy}  Найперше, місце проживання - далека і
сонячна Португалія. Думаю, сидячи на березі атлантичного океану і попиваючи
португальський портвейн, можна не тільки паспортизацію провести десь в далекій
Україні, а й замахнутися на більше - експансію у світі, можливо навіть на інших
планетах. Наступне. Ваша фантазія як модель нежиттєздатна - вона не враховує
багато змінних внутрішнього і зовнішнього середовищ. Ваша модель вимагає багато
ресурсів - фінансових і людських. Ваша модель нічого не говорить навіть про
найпростіше - як отримати владу у цій країні. Ваша модель не враховує бажання і
настрої людей, в т. ч. жителів південного сходу. Тощо тощо тощо. Я вам більше
скажу, така модель, про яку ви мрієте вже існує, але в цій моделі домінуючу
роль відіграє російська. Я розумію, з далекої Португалії не все чітко можна
розібрати, але повірте, ніхто з нагайкою не стоїть і не б'є українців, але їм
все одно треба знати російську, а в багатьох регіонах тільки її, бо на роботу
не візьмуть. І законів не має, і примусу теж. Такий консенсус в Україні, така
рівновага. А таким як ви дозволяють пофантазувати. На здоров'я.


\iusr{Mashinskaya Anna}
\textbf{Volodymyr Tarnavskyi} Ее, панечку, ви якийсь дивний. Я от ні разу не в сонячній португалії, але з коментом пана Степанця абсолютно згодна, все так і є. І перші кроки зроблені, і треба йти далі. І будемо йти, аби не коровяче бажання деяких знову під батіг.

\iusr{Volodymyr Tarnavskyi}
\textbf{Mashinskaya Anna}  @igg{fbicon.face.tears.of.joy}  що тут дивного? Прості практичні питання. Натомість я завжди чую якісь дивні безсенсові речі про те що треба кудись іти, процес пішов тощо...
\end{itemize} % }

\iusr{Vitalii Havrylyshyn}

Згідний на 100\%! + "ускаґаварящіє" ще й "ускамислящіє": їм важко второпати, що
їх, їхня московія використала як інструмент для приєднання нових територій. І з
їхніми дітьми теж проблема, бо, як на мене, існує генетична пам'ять і вони
завжди будуть почувати себе тут чужими... Принаймні до якогось там коліна.

\begin{itemize} % {
\iusr{Volodymyr Tarnavskyi}
\textbf{Vitalii Havrylyshyn} діти призвичаюються... В нормальних країнах типу Литви чи Естонії. Чому цього не відбувається в нас інше питання...

\iusr{Helena Dubravina}
\textbf{Vitalii Havrylyshyn} О ні. Вони не по, уваються тут чужимивоеи почуваються господарями, а українці їм це дозволяють.
\end{itemize} % }

\iusr{Влада Порхомчук}
Підтримую!!!!

\iusr{Валентина Знахарева}

Який зразу паспорт не громадянина ? Спочатку короткотермінові ТИМЧАСОВІ дозволи
на проживання, а для їх видачи перевірка до сьомого коліна, і така видача
робиться 2-3 рази ! Потім довгостроковий ДОЗВІЛ на проживання з перевірками ! А
для громадянства скільки треба, щоб подати документи....... Потім це питання
потребує перевірок і не факт, що отримаєш його.... Отак захищає закон свою
державу ! А у нас можна на базарі купити громадянство.....

\begin{itemize} % {
\iusr{Volodymyr Tarnavskyi}
\textbf{Valentina Znahareva}  @igg{fbicon.face.tears.of.joy}{repeat=3} 

\iusr{Валентина Знахарева}
\textbf{Volodymyr Tarnavskyi}, смієшся чому.? Я живу в Італії, тому знаю про що пишу !!!

\iusr{Олена Потайчук}
\textbf{Valentina Znahareva} якби можна було, я би сорок вподобайте вам поставила. Завжди захоплювалась позицією Прибалтики. Наші ці "правителі" ніколи на таке не підуть - раптом московія обидиться? А карлик захищати їх піде.

\iusr{Валентина Знахарева}
\textbf{Олена Потайчук},

\ifcmt
  ig https://scontent-lhr8-1.xx.fbcdn.net/v/t39.30808-6/263156651_2710205962459629_2510849606291012756_n.jpg?_nc_cat=103&ccb=1-5&_nc_sid=dbeb18&_nc_ohc=6ixrdm3u8AIAX9szCtM&_nc_ht=scontent-lhr8-1.xx&oh=c2b2d08627b03634e3f25b30b1987a29&oe=61B3DDF2
  @width 0.4
\fi

\end{itemize} % }

\iusr{Лідія Берник}

Просто потрібно прийняти закон, як Прибалтика, в паспорті москаль, паспорт
москальський- заборона на все і всюди: керівні посади, викладання, без права
голосування і т.д. Лиш можна підмітати українські вулиці і не більше. Мова.
Мова це окремо. Чуєш чужинську мову відразу в пику. Повірте через місяць буде
результат позитивний. І головне не боятися, що скаже світ. Вони тихенько
поаплодують Україні.

\begin{itemize} % {
\iusr{Volodymyr Tarnavskyi}
Чого в пику?))) стріляти відразу)))

\iusr{Лідія Берник}
\textbf{Volodymyr Tarnavskyi} Погоджуюся і підтримую.

\iusr{Лідія Берник}
\textbf{Volodymyr Tarnavskyi}

\iusr{Alex Goosak}
\textbf{Лідія Берник} потрібно припинити вживати топонім "прібалтіка", створенний окупантами щоб знищити у пвдсвідомісті сприйняття суверенитету цих держав.

\iusr{Лідія Берник}
\textbf{Alex Goosak}

\ifcmt
  ig https://i2.paste.pics/a665c80e75a41be71a72996f5d39230b.png
  @width 0.4
\fi

\end{itemize} % }

\iusr{Світлана Гончарова}

Питання надання паспорта не громадянина проситься давно до вирішення. Якось це
треба просувати до ВР, бо або ми їх мовою та іншими обмеженнями, або вони нас
московією додавлять.

\iusr{Ольга Дідик}
\textbf{Світлана Гончарова} 

Воно то так! Але де взяти ту ВР, яка проголосує за такий закон? Там закон про
мову, аж занадто лояльний до московитів, насилу за 28 років проголосували, а
про такий залишається тільки мріяти.

\iusr{Світлана Гончарова}
\textbf{Ольга Дідик} Погоджуюся. Але ж треба вимагати.

\iusr{Олена Слюсаренко}
Тільки так!!!!

\iusr{Владислав Ткаченко}
Тільки так - тільки по українськи!  @igg{fbicon.smile} 

\iusr{Олександр Прадеус}

Потрібно просто трішки видозмінити процедуру отримання паспорта громадянина
України: для неповнолітніх - у 14 - тимчасова \enquote{посвідка для неповнолітніх} (яка
обов'язково втрачає силу, скажімо у 18,5 чи 19 років). Для отримання паспорта -
потрібно до посвідки додати ЗНО з української мови, історії України та права.
Для дорослих - видається посвідка на 5 років з тією ж метою. У випадку, якщо
термін дії посвідки сплив - без варіантів надається паспорт негромадянина. І
тоді той, хто зможе отримати паспорт громадянина України - буде це цінувати.

\begin{itemize} % {
\iusr{Валентина Бугаєнко}
\textbf{Олександр Прадеус} Уточнення не в 18,5 чи 19, а в 21.

\iusr{Олександр Прадеус}
\textbf{Валентина Бугаєнко} взагалі то у нас повноліття настає у 18.

\iusr{Валентина Бугаєнко}
\textbf{Олександр Прадеус} Повноліття так , а спиртне продають з 21. Просто в 21 вже трохи свідоміші.
\end{itemize} % }

\iusr{Валентин Марущак}

Почекай, але тут ви всі робите типову ідеалістичну українську наївняцьку
помилку.

Як це \enquote{держава нічого не робила в цій сфері}?? Та згадай хоча би декомунізацію!
Та ще в 2013р ми могли лише мріяти про це, тоді комуняки були реальною владою в
Україні і керували ВР! За декомунізацією логічно йшла дерусифікація, і вона
ЙШЛА! Вони нас 300 років морили, я не кажу, що тепер треба такі ж 300 років на
нашу користь - але за короткий період 2014-2018 вони ВЖЕ почувалися некомфортно
і ВЖЕ були змушені так чи інакше реукраїнізовуватись, \enquote{обставлені} зі всіх
сторін централізованою неухильною державною політикою повернення до
українського-україномовного.

Просто треба було все це продовжувати, блін..

\iusr{Nelly Osadchuk}
+++++++++++
Українців більше чим отих:)))

\iusr{Зеня Кіт}
Свята правда. Вартує зробити їм лиш \enquote{відповідні умови,} то відразу заговорять.




\end{itemize} % }
