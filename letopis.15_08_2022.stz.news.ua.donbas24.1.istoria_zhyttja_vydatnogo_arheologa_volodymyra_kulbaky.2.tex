% vim: keymap=russian-jcukenwin
%%beginhead 
 
%%file 15_08_2022.stz.news.ua.donbas24.1.istoria_zhyttja_vydatnogo_arheologa_volodymyra_kulbaky.2
%%parent 15_08_2022.stz.news.ua.donbas24.1.istoria_zhyttja_vydatnogo_arheologa_volodymyra_kulbaky
 
%%url 
 
%%author_id 
%%date 
 
%%tags 
%%title 
 
%%endhead 

\subsubsection{Наукові здобутки вченого-археолога}

\ii{15_08_2022.stz.news.ua.donbas24.1.istoria_zhyttja_vydatnogo_arheologa_volodymyra_kulbaky.pic.2}

Завдяки науковій діяльності Володимира Кульбаки в Маріупольському краєзнавчому
музеї з'явився сектор археології, на базі університету почала працювати
археологічна лабораторія, спеціальна аудиторія для проведення лекцій з
археологічної тематики. Йому вдалося підготувати низку навчальних посібників,
створити демонстраційну наочність, зручну для студентів-істориків, які під час
літньої практики робили зі своїм викладачем унікальні відкриття. У 1993 р. він
почав досліджувати порушений будівельними роботами курган \enquote{Дід} на західній
околиці Маріуполя. 90-ті роки стали кризовими для розвитку археології, далеко
не всім у ті часи вдавалося без жодної підтримки та фінансування залишатися
відданими своїй справі. Але Володимир Костянтинович зміг впоратися з цим
випробуванням.

Впродовж 1984−2000 років Володимир Кульбака досліджував кургани з похованнями
енеоліту, епохи бронзи, раннього залізного віку, кочівників і ґрунтовий
могильник періоду Золотої Орди. А здобуті яскраві матеріали лягли в основу його
розробок з індоєвропейської тематики. Вчений неодноразово робив спроби
організувати регулярні розкопки в Приазов'ї й водночас — повноцінну практику
для майбутніх істориків-студентів МДУ. Так у 2000 році вдалося дослідити другу
ділянку золотоординського могильника на східній околиці Маріуполя. Але
відсутність фінансування зривала ці плани.

Варто відзначити, що Володимир Костянтинович мав \emph{\textbf{альтернативний погляд на
історію рідного краю}}, що було вкрай небажаним в СРСР. Працюючи в експедиціях з
1976 року і дослідивши 67 курганів, він отримав можливість видати свої
монографії тільки на початку 2000-х років. Всього було видано чотири книги, в
яких зібрані результати роботи над 274 похованнями з епохи енеоліту до пізнього
середньовіччя та одним могильником золотоординського часу.

Однією з унікальних знахідок Володимира Костянтиновича\par\noindent Кульбаки є дерев'яні
чотириколісні вози з суцільними дерев'яними дисковими колесами. Вони були
виявлені в кургані бронзової доби на околиці нашого міста — біля дороги, що
веде в селище Володарське. \emph{Ці знахідки є не тільки одним з найдавніших видів
колісного транспорту Приазов'я, але й одними з найдавніших возів у світі.}

Володимир Костянтинович довів, що Приазов'я, і зокрема Маріуполь, має багату та
унікальну давню історію, а не виник кілька століть тому. Тільки людина з
великим серцем, безмежно віддана обраній справі буде завжди в доброму гуморі,
незважаючи на всі життєві труднощі. Адже Володимир Костянтинович пов'язав своє
життя з наукою, що не мала фінансової підтримки з боку держави. Розраховувати
залишалося на власні сили та невичерпний ентузіазм...
