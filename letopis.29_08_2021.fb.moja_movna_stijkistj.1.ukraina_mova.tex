% vim: keymap=russian-jcukenwin
%%beginhead 
 
%%file 29_08_2021.fb.moja_movna_stijkistj.1.ukraina_mova
%%parent 29_08_2021
 
%%url https://www.facebook.com/mmstiykist/posts/368533024854848
 
%%author_id moja_movna_stijkistj
%%date 
 
%%tags mova,ukraina,ukrainizacia
%%title «Не було б української мови, то не було б України»
 
%%endhead 
 
\subsection{«Не було б української мови, то не було б України»}
\label{sec:29_08_2021.fb.moja_movna_stijkistj.1.ukraina_mova}
 
\Purl{https://www.facebook.com/mmstiykist/posts/368533024854848}
\ifcmt
 author_begin
   author_id moja_movna_stijkistj
 author_end
\fi

«Не було б української мови, то не було б України». 

Співкоординатор Руху добровольців Простір свободи та громадський активіст Тарас
Шамайда розповідає про мову та державу.

За його словами, через українську мову поширюються зовсім інші сенси та
наративи, ніж російською.

Шамайда впевнений, якщо Україна буде об’єднана мовою, то у Кремля не буде
жодних підстав розповідати про «один народ».

\ifcmt
  ig https://scontent-frx5-1.xx.fbcdn.net/v/t39.30808-6/240588464_368532948188189_1779502618641793792_n.jpg?_nc_cat=100&ccb=1-5&_nc_sid=8bfeb9&_nc_ohc=T2smrWJrNSIAX_9pxzl&_nc_ht=scontent-frx5-1.xx&oh=cacbf4c68e7a490e5f1fc3877484756f&oe=61515072
  @width 0.4
  %@wrap \parpic[r]
  @wrap \InsertBoxR{0}
\fi

«Міжнародне дослідження, яке було опубліковане в 15 країнах світу, показало, що
у всіх цих країнах люди визнали мову основним та ключовим елементом своєї
ідентичності. Для Східної Європи таке формулювання є ще більш важливішим.
Оскільки не було б української мови, то не було України як такої. Саме тому
зрозуміло, що Російська імперія, СРСР та Російська Федерація завжди намагалися
спочатку русифікувати населення України.» — зазначив Шамайда.

«Під впливом Москви Україна не лише не спромоглася ухвалити сучасний закон про
мову, а зробила це лише в 2019 році, так ще в часи Януковича був ухвалений
відверто русифікаторський закон Ківалова-Колесніченка, людей, які відверто
працювали на Кремль. Очевидно, що така русифікаторська політика є елементом
гібридної війни. Саме тому, коли зараз діє закон про мову, важливо, щоб
телебачення виконувало цей закон.» — додав Шамайда.

Джерело: espreso.tv

\#мовніновини
