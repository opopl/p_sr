% vim: keymap=russian-jcukenwin
%%beginhead 
 
%%file 30_01_2022.stz.ludi.uljanov_anatolij.1.uslovia_mira
%%parent 30_01_2022
 
%%url https://dadakinder.com/txt/2022/1/conditions-of-peace
 
%%author_id uljanov_anatolij
%%date 
 
%%tags donbass,minsk_dogovor,mir,obschestvo,ukraina
%%title Условия мира
 
%%endhead 
 
\subsection{Условия мира}
\label{sec:30_01_2022.stz.ludi.uljanov_anatolij.1.uslovia_mira}
 
\Purl{https://dadakinder.com/txt/2022/1/conditions-of-peace}
\ifcmt
 author_begin
   author_id uljanov_anatolij
 author_end
\fi

Смешно и примечательно, что те же люди, которые выступают против «совка», одним
из инструментов которого был централизованный контроль, выступают и против
Минских соглашений. Ведь это как раз тот случай, когда нужно либо снять
крестик, и перестать изображать из себя демократов, либо надеть трусы, и
следовать заявляемым демократическим принципам.

Наши патриоты боятся, что «особый статус» Донбасса открывает возможность для
прихода к власти в этом регионе «пророссийских сил», а, в перспективе, его
выхода из состава Украины, или даже присоединения к России.

В нижеследующем тексте я хочу поговорить о такой перспективе; и том, почему
децентрализация является последним шансом для Украины реализовать свой
демократический потенциал в нынешних географических координатах; сделать
украинское общество привлекательным для всех его граждан.

\subsubsection{1}

В Украине живут очень разные люди. В том числе те, кто не поддерживает дискурс
Майдана, движение Украины в НАТО, уничтожение памятников и переименование улиц,
идею «титульной нации» и единый для всех язык. 

Да, эти политические вопросы вторичны по отношению к экономическим, от которых
зависит жизнь тела, но это не значит, что их можно игнорировать, ведь они
касаются человеческого мироощущения, идентичности – бытия собой.

Что из этого следует? Что разные граждане по-разному смотрят на жизнь, и с
разнообразием этих взглядов необходимо считаться. 

Да, в украинской палитре есть люди с «пророссийскими взглядами», – как
приписываемыми, так и вполне реальными; те, кого у нас пренебрежительно
называют ватой, колорадами, сепарами, совкодрочерами и т.д.

Разделять взгляды этих людей не обязательно. Но их права должны быть защищены,
а голос представлен – не только на уровне взглядов, за которые у нас
наказывают, и не только в медиа, которые у нас закрывают, но и в самой системе
управления страной. Если, конечно, мы – демократическое общество.

\subsubsection{2}

Одним из инструментов демократизации является децентрализация. Её смысл в том,
чтобы не центр диктован регионам порядок их жизни, а чтобы регионы сами решали,
ставить им у себя памятник Бандере или Ленину, говорить на украинском, русском
или ещё каком-то языке, и т.д.

Политика, которая исходит из понимания особенностей и интересов сообществ,
–жителей конкретного контекста, – позволяет выстраивать демократию снизу –
создавать систему из нюансированных региональных блоков, а не спускать на людей
некий единый для всех режим и порядок.

Децентрализация динамизирует систему, позволяя экспериментировать с теми или
иными законами и механизмами на региональном уровне, прежде чем
имплементировать наиболее успешные из них в национальном масштабе.

\subsubsection{3}

Тем, кто боится, что «пророссийские» регионы сбегут от Украины, необходимо
задуматься о том, почему у целого ряда украинских граждан может возникать
желание «сбежать» из страны, которая не может предложить им ничего, кроме
нищеты и насилия в статусе граждан второго сорта.

Конечно, можно риторически лишить «пророссийских» граждан Украины agency, и
сказать, что всё, что они думают и говорят – это кремлёвская пропаганда, что
они «зомбированное ящиком быдло», но такое высокомерие не сделает украинское
общество удобным и приятным местом для жизни разных людей.

73\% голосов, отданных Зеленскому в знак протеста против националистического
милитаризма, выраженного в лозунге «Армiя, Мова, Вiра», сообщают нам то, что
большинство граждан Украины – мирные люди, а не фанатики нейшнбилдинга любой
ценой. Их голос был использован, но по сей день не реализован. За них говорят
«лидеры мнений» и «гражданское общество». Их интересы не  учитываются, и не
становятся частью системного узора.  

Положение, при котором часть граждан подвергается попыткам насильственной
ассимиляции; инакованию, исключению и маргинализации за отказ жить по правилам,
на которые она не влияет, нельзя назвать демократическим.

\subsubsection{4}

Когда мы слышим, что приход к власти «пророссийского» кандидата, которым в
Украине может быть объявлен любой, кто не соответствует известному набору
стандартов, будет расцениваться как российское вторжение, у людей моего
этнического и идеологического состава возникает вопрос: каковы вообще
перспективы политической субъектности «неправильных» украинских граждан в
«правильной» Украине, и почему такие украинцы должны ратовать за ту Украину,
где им, их памяти, культуре, языку и взглядам нет места? Что Украина может им
предложить? Почему они должны уступать ей свою жизнь, или гибнуть за её тесный
проект, а не жить по-своему, ради себя и своих детей?

Всё это – открытые вопросы. Можно сказать, что их нет. Не слышать их. Но они от
этого не исчезнут. Главный из них таков: хотим ли мы вообще жить вместе?

Если да, тогда у этого «да» должно быть практическое выражение. В частности, в
виде уважения прав, голосов и желаний людей, которые иначе смотрят на мир и
страну. Одной из форм такого выражения является децентрализация.

Если нет – что ж, тогда нужно расходиться. Желательно, полюбовно. Или хотя бы
без углубления мокрухи. Жители Донбасса не должны уезжать в Россию, чтобы
освободить его для «титульной нации». Это их земля.

Вопросы будут обостряться до тех пор, пока не получат ответа. Уверен, что
сценарий отказа от своего этнокультурного фундамента неприемлем ни для одной из
социальных групп: ни для тех, кому Бандера «батько», ни для тех, кому сейчас
выписывают штраф за символы, под знамёнами которых «деды воевали». 

Над этим можно посмеяться. Но долгим такой смех не будет, и закончится тем, что
«ватная» часть задумается о необходимости самоопределения за пределами Украины.
Потому что для любого «ватника» житель Ростова понятнее жителя Львова, и это не
что-то, над чем стоит смеяться. Это что-то, что нужно понять, принять, и с чем
нужно работать, чтобы остановить распад страны.

\subsubsection{5}

Децентрализация не решает всех проблем. Но создаёт условия для их решения на
уровне регионов, а, в перспективе, и на уровне страны. И является испытанием на
предмет нашей способности быть разными, и жить под одним зонтом. Не на словах.
Не в лозунгах. В повседневной практике мирного неба.

Есть ли в этом сожительстве смысл? Каким клеем нас клеить, если на месте
советской идентичности за 30 лет украинской независимости так ничего и не
возникло, кроме коробочки с украинством, в которое многие из нас просто не
помещаются? Не все мы думаем и чувствуем в единстве вышиванки. Это не плохо, и
не хорошо. Это – факт, который необходимо учитывать.

«Но что если Донбасс таки решит уйти из Украины?!» 

Что ж, если таковой окажется воля жителей Донбасса, у Украины нет иного способа
доказать свою верность демократическим принципам, кроме как уважить эту волю,
обеспечив её реализацию. 

Скажу более – выход того или иного региона из состава Украины в процессе
демократического волеизъявления кажется мне куда более предпочтительным
сценарием расставания, чем аннексия или война.

Хочу подчеркнуть, что я по-прежнему надеюсь на возможность единой страны. И
именно поэтому выступаю за децентрализацию не только для Донбасса, но за
большую автономию каждого региона Украины. 

Пусть тернопольская область заклеит Бандерой хоть каждый дом. У себя. В
тернопольской области. Но не нужно навязывать своих героев всем. У разных
украинцев – разные герои, разная историческая память, разные взгляды, и это –
нормально. Потому что современное общество – сложное и цветное. 

Понимание этого, и реализация этого понимания на пратике являются условиями
мира в Украине. Без равных прав и равного представительства, смысла клеиться
друг к другу нет. И тогда нужно расходиться. 

Мы родились под одним флагом. Росли под другим. И вполне можем умереть под
третьим, пятым, двенадцатым. Потому что, в конечном итоге, значение имеют не
флаги, а люди. Их счастье, здоровье, мечты и возможности.

От того, способна ли Украина удовлетворить эти фундаментальные чаяния своих
разнообразных граждан зависит не только их, но и её будущее. 
