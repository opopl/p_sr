% vim: keymap=russian-jcukenwin
%%beginhead 
 
%%file 27_05_2021.fb.master_i_margarita.1.voland_film_basilashvili
%%parent 27_05_2021
 
%%url https://www.facebook.com/groups/masterimargaritabulgakov/permalink/3799264500200154/
 
%%author 
%%author_id master_i_margarita
%%author_url 
 
%%tags 
%%title Олег Басилашвили о роли Воланда в фильме  "Мастер и Маргарита"
 
%%endhead 
 
\subsection{Олег Басилашвили о роли Воланда в фильме  "Мастер и Маргарита"}
\label{sec:27_05_2021.fb.master_i_margarita.1.voland_film_basilashvili}
\Purl{https://www.facebook.com/groups/masterimargaritabulgakov/permalink/3799264500200154/}
\ifcmt
 author_begin
   author_id master_i_margarita
 author_end
\fi

\textbf{Миром Владеющая}
\url{https://www.facebook.com/groups/3504809972978943/user/100043885923593}

Олег Басилашвили о роли Воланда в фильме  "Мастер и Маргарита."

-  Вы не боялись, соглашаясь на роль Воланда?

- Конечно, хочется сказать: все это ерунда! Но, думаю, с романом действительно
все обстоит не так просто. Честно говоря, мне тоже было немного не по себе,
когда получил предложение сыграть Воланда. Все вокруг говорили: «Мистика! Бог
вас покарает!» Тогда я как раз был в Киеве на гастролях, вот и решил пойти в
только что восстановленный храм в Киево-Печерской лавре попросить у Бога
разрешения сниматься в «Мастере и Маргарите». Прихожу, а храм закрыт. Делать
нечего, зашел в находящийся рядом. Стою. Вдруг подходит ко мне женщина. «А вы
не хотите посмотреть восстановленный храм?» — спрашивает. «Мечтал об этом,
только там закрыто», — говорю. А она мне: «Пойдемте, вдруг пустят». И точно,
только мы подошли, какой-то служка двери открывает: «Проходите». Я оказался в
храме один и воспринял это как знак: мне разрешают.

- На ваш взгляд, кто такой Воланд? Дьявол?

- Ни в коем случае! Дьявол — искуситель, он даже Христа искушал в пустыне. А
Воланд никого не совращает, он просто наказывает порок. Убивает стукача барона
Майгеля, казнит атеиста Берлиоза, а из талантливого, но малограмотного поэта
Бездомного делает философа. Ему много тысяч лет отроду, он прожил гигантскую
жизнь, говорил с самим Богом. И вот спустя сотни тысяч лет является на Землю,
чтобы посмотреть на людей. В социалистической Москве, говорят, живут свободные
люди, отрицающие Бога. Интересно! И что же он видит? Люди такие же, какими они
были тысячи лет назад. Одеваются иначе, на автомобилях ездят, но в остальном
все, как прежде. «Я прав», — думает он. Весело ему или грустно? Грустно, потому
что человеческую природу, в которой заложено много подлого, не изменить. И
вдруг он встречает двоих, которые не похожи на остальных людей. Талантливого и
честного писателя (не от мира сего) и женщину, которая кожу с себя готова
содрать, только бы ему было хорошо. Такой любви Воланд тоже не встречал. И он
понимает, что этим двоим не место на земле, их просто затопчут в этом
«социалистическом раю». Жаль. Значит, надо им помочь. И он помогает. Разве это
плохой поступок?..

(Инет)

\ifcmt
  pic https://scontent-bos3-1.xx.fbcdn.net/v/t1.6435-9/193108874_335273107945551_9038777012324173344_n.jpg?_nc_cat=103&ccb=1-3&_nc_sid=825194&_nc_ohc=Jkmshfdhua8AX-zmp9d&_nc_ht=scontent-bos3-1.xx&oh=83cbd6e1c01e81729d546c31a80f9a63&oe=60D47799
\fi
