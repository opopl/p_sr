% vim: keymap=russian-jcukenwin
%%beginhead 
 
%%file 16_10_2020.news.ua.bbc.1_izrael_covid_marriage
%%parent 16_10_2020
%%url https://www.bbc.com/ukrainian/news-54568404
%%tags bbc,marriage,israel,covid
 
%%endhead 

\subsection{Кров і сльози. В Ізраїлі поліція "штурмувала" весілля - порушували карантин}

\url{https://www.bbc.com/ukrainian/news-54568404}

В Ізраїлі правоохоронцям довелося жорстко зупиняти весільне святкування,
організатори та гості якого порушували правила карантину, запровадженого в
країні для боротьби з Covid-19.

Як повідомляє місцева газета Haaretz, інцидент стався у селищі неподалік
Єрусалима - поліцію на місце викликали сусіди. Вони повідомили, що на весіллі
поряд із ними перебуває понад 20 людей.

Карантинні правила в Ізраїлі передбачають, що збиратися дозволено максимум 10
людям у приміщенні та не більш як 20 людям на вулиці.

Поліцейським, які прибули на виклик, довелося застосувати силу.

На відеозаписах, які поширюють у соцмережах, видно, що принаймні в одного з
гостей весілля закривавлене обличчя.

Правоохоронці кажуть, що одна з жінок, яка була невдоволена їхнім приїздом,
накинулася на них.

Після цього інші гості почали жбурляти в поліцейських пляшки, внаслідок чого,
один із гостей послизнувся і поранився розбитим склом. Схожих ушкоджень зазнав
і один із правоохоронців.

Представник поліції Мікі Розенфельд розповів виданню The Guardian, що у пляшках
була оливкова олія.

Ситуацію активно обговорюють ізраїльські користувачі соцмереж, на адресу
правоохоронців лунає багато критики.

Засудив дії поліцейських навіть міністр внутрішніх справ Ар'є Дері: "Немає
жодних причин для того, щоб поліція вдиралася на весілля із гвинтівками в руках
і завдавала шкоди людям", - заявив він.

Міністр громадської безпеки Амір Огана пообіцяв провести розслідування
інциденту. 

\url{https://twitter.com/i/status/1316422902812733442}
