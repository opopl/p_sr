% vim: keymap=russian-jcukenwin
%%beginhead 
 
%%file 27_01_2022.fb.lukshic_jurij.1.tragedia_dnepr
%%parent 27_01_2022
 
%%url https://www.facebook.com/permalink.php?story_fbid=1172836736582286&id=100015679115504
 
%%author_id lukshic_jurij
%%date 
 
%%tags dnepropetrovsk,murder,nacgvardia,obschestvo,strana,tragedia,ukraina
%%title 27 января ранним утром в Днепре произошла трагедия
 
%%endhead 
 
\subsection{27 января ранним утром в Днепре произошла трагедия}
\label{sec:27_01_2022.fb.lukshic_jurij.1.tragedia_dnepr}
 
\Purl{https://www.facebook.com/permalink.php?story_fbid=1172836736582286&id=100015679115504}
\ifcmt
 author_begin
   author_id lukshic_jurij
 author_end
\fi

27 января ранним утром в Днепре произошла трагедия. Военнослужащий Национальной
гвардии Украины, солдат срочной службы 2001 г. р., при выдаче оружия по
неустановленным причинам выстрелами из автомата Калашникова расстрелял караул
военнослужащих НГУ, после чего с оружием скрылся. В результате этого пять
человек погибли, ещё пятеро получили ранения.

События произошли в казарме Нацгвардии. Солдат-срочник, который вырос без отца,
взял оружие и начал расстреливать людей, убил бабушку-вахтёра, которая была на
выходе из казармы. Известно, что он родом из Одесской области.

Какие выводы можно сделать из этой информации? Первое. Солдат 2001 года
рождения. Второе. Солдат-срочник. Третье. Вырос без отца. Четвёртое. Пятеро
погибших, вероятно, сослуживцы. Несложная задачка. Впрочем, подождём
результатов официального расследования. Уголовное дело уже заведено.

\ii{27_01_2022.fb.lukshic_jurij.1.tragedia_dnepr.cmt}
