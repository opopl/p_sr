% vim: keymap=russian-jcukenwin
%%beginhead 
 
%%file slova.krasota
%%parent slova
 
%%url 
 
%%author 
%%author_id 
%%author_url 
 
%%tags 
%%title 
 
%%endhead 
\chapter{Красота}
\label{sec:slova.krasota}

%%%cit
%%%cit_pic
%%%cit_text
Четвертое место - песня \enquote{Несе Галя воду}. В этой песне сошлось все и
сразу: певучесть украинского языка, невероятная мелодичность, доходчивость,
нежность и проницательность. А как она звучит на несколько голосов, невероятно
красиво. Впервые на телевидении эту песню исполнило трио Мариничей. Она сразу
полюбилась. Удивительно, что как только начинаешь ее петь, то сразу возникает
образ села, невероятная природа и ситуация, где девушка несет воду, а парень
пытается комплиментами привлечь к себе внимание. Вроде все просто, но как
\emph{красиво} и выразительно. Вспомним эту песню вместе в исполнении Людмилы
Сенчиной,
%%%cit_title
\citTitle{5 задушевных украинских песен, которые пели наши родители, а теперь поем мы}, 
Кино Вояж И Не Только, zen.yandex.ru, 07.06.2021
%%%endcit

%%%cit
%%%cit_head
%%%cit_pic
\ifcmt
  pic https://lgaki.info/wp-content/uploads/2021/10/Lyubota-Polina-690x1024.jpg
  @width 0.4
\fi
%%%cit_text
Сколько бы поэты и художники ни воспевали \emph{красоту} осени, все равно будет
мало!  Внесем свою лепту и мы. Жюри Республиканского фестиваля-конкурса
художественного творчества «Осенний вернисаж» во главе с заведующим кафедрой
станковой живописи нашей Академии, прекрасным художником Олегом Безуглым,
отсмотрело присланные на конкурс работы. С удовольствием! Хоть и труд это был
немалый – организаторы приняли 222 заявки от юных живописцев, графиков и
скульпторов со всей ЛНР. И готово назвать победителей. И список их будет
длинным! Что, согласитесь, чудесно!  Конкурс, напомним, уже много лет проводит
наша Академия при поддержке Министерства культуры, спорта и молодежи ЛНР
%%%cit_comment
%%%cit_title
\citTitle{Открываем наш «Осенний вернисаж»}, , lgaki.info, 16.10.2021
%%%endcit

%%%cit
%%%cit_head
%%%cit_pic
%%%cit_text
\emph{Краса} Карпат допомогла мені. Довгі дні роздумів, зосередження. Все
довкола підказувало: не бійся, відкинь страх! І грандіозні хвилі гір, і
смерекова варта, і мудра тиша, що пливла зоряними ночами над Чорногорою, — все
веліло: починай!  І я сказав сам собі: а чому б і не почати? Чому б не
спробувати? Теоретизувати може всякий, а полетить в небо лише птах! Якщо в
моєму єстві присутня здатність нового вияву— успіх буде, прийде! Якщо така
здатність відсутня — вона повинна зачатися, запліднитися від мого палкого
бажання, воління. Інакше воля нічого не варта! Вона — фікція. І тоді людина —
лише раб природи, її жалюгідна маріонетка, прив’язана До пуповини кривавого
насичення. Безкомпромісно — чи не так? Але правда була для мене понад усе. І
ще… я відчував у собі, помимо власної волі, клекотіння тисяч інших воль, інших
життів. Вони вимагали, штовхала, сподівалися...  Я повторив свій експеримент з
голодом. Шістдесят днів утримання від їжі, чиста вода, повітря, обмивання в
потоці прохолодного водоспаду. Раніше я мав мету — Лікування, повернення до
руху, до нормального людського життя. Тепер воля кликала мене далі, за межі
відомого...
%%%cit_comment
%%%cit_title
\citTitle{Вогнесміх}, Олесь Бердник
%%%endcit
