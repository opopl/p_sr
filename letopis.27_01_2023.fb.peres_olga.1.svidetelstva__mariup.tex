%%beginhead 
 
%%file 27_01_2023.fb.peres_olga.1.svidetelstva__mariup
%%parent 27_01_2023
 
%%url https://www.facebook.com/olga.peres.395/posts/pfbid028XAvX8hKTUMHZYUpm2GiUxRaxWrsrWVME6Ln9Z8qzXuHzaoujoYT3VMn85RtHdYGl
 
%%author_id peres_olga
%%date 27_01_2023
 
%%tags mariupol,mariupol.war,svidchennja
%%title Свидетельства  мариупольцев
 
%%endhead 

\subsection{Свидетельства  мариупольцев}
\label{sec:27_01_2023.fb.peres_olga.1.svidetelstva__mariup}

\Purl{https://www.facebook.com/olga.peres.395/posts/pfbid028XAvX8hKTUMHZYUpm2GiUxRaxWrsrWVME6Ln9Z8qzXuHzaoujoYT3VMn85RtHdYGl}
\ifcmt
 author_begin
   author_id peres_olga
 author_end
\fi

Свидетельства мариупольцев.
***

Одна мысль не даёт мне покоя, в Хиросиме после ядерного взрыва погибло 200 тыс
человек. 

Погибли они менее, чем за минуту. 

В Мариуполе в течение нескольких месяцев гибли люди, которых уничтожали изо
дня в день, из часа в час, из минуты в минуту. Это было самое ужасное,
примерно, как у смертников на электрическом стуле. 

Ты никогда не был уверен проснешься ли сегодня утром, доживешь ли до этого
самого утра. Информации ноль, элементарных условий ноль, еды и воды ноль.
Представьте себе состояние родителей, которым нечем накормить своих детей в 21
веке!!! Люди не мылись месяцами. Никто из нас не понимал, когда все это
закончится и будет ли конец!!! Полное отсутствие информации, ты не знаешь живы
ли твои родные, будут ли защищать город, будет вода или еда!!! Ты не знаешь
ничего!!! Ты видишь горящие дома вокруг, ты слышишь крики людей, которые горят
, кто-то успел выскочить, и ты понимаешь, что это вопрос времени, когда
прилетит в твой дом!!! Пожалуй самое ужасное во всем этом, что наши дети тоже
были в этом аду!!! И, да, страшно не было, когда ты находишься в эпицентре
событий, был страх за детей, не за себя. Гораздо страшнее жить сейчас, когда
адреналин больше не зашкаливает, и наступает осознание, как мы там
выжили?!!!! Каким чудом?!!! Благодаря чему?!!! Только сейчас появилось
осознание всего ужаса происходящего и понимание того, что погибло только 200
тыс, а, ведь, могли погибнуть там все мы и наши дети!!! Нас всех постепенно,
планомерно уничтожали рашисты день за днём, час за часом. Трупы на улицах,
могилки во дворах домов больше не ужасали, мы понимали, что бояться надо живых
, особенно тех, что в самолётах сбрасывают мины на твой город!!! Можно
бесконечно об этом писать. Потому, что эта война навсегда останется с нами, в
наших сердцах и душах. Потому, что это и есть мы. И все мы немного умерли там,
в Мариуполе((

*****

Коли я побачила перший труп на вулиці, у центрі Маріуполя, я була в шоці...
чому на вулиці?  і хтось прикрив білим простирадлом.  

А потім до мене дійшло, центр міста, цвинтар далеко, лопати немає, а може й
нікому копати й вибухи... з усіх боків чути.  Це був тільки початок і людина, я
думаю, померла своєю смертю.  А потім, коли я побачила величезний, чорний
9-поверховий будинок після пожежи, ось тут у мене звучало одне питання, де всі
ці люди, які там жили, де старі і хворі, де діти?  Скільки у цьому будинку
залишилося людей у своїх квартирах? Вони і досі там... Ніколи не пробачу.
Ніколи не забуду!  І онукам нагадуватиму, щоб не забули, вони наше місто теж
дуже любили...

***

Люди в подвалах, без еды, без воды, без связи, без новостей, в неизвестности. 

А потом каждое утро в четыре утра самолёты. 

Страшно, не знаешь, где сегодня скинут авиабомбы....с нами в подвале была
женщина с двумя маленькими детьми. Муж её полицейский, привозил нам питьевую
воду. 

Как то приехал и сказал, что эти самолёты разбомбили центр города. Дома
рушились, как карточные домики..... страшно. 

Нас в подвале 25 человек. 

Со мной муж, сын 18 лет, дочь приехала из Киева в гости на несколько дней и
тоже попала с нами, внук 8 месяцев. Хорошо, что дочь кормит грудью. На первом
этаже в квартире прилёт. Все убежали тушить пожар. Воды нет. Сливали из системы
отопления. Мы в подвале втроём, я, дочь и внук. На улице мороз -10 С. Сижу и
думаю, вдруг пожар не потушат и нам придется бежать из нашего этого
укрытия..... мысли.... надо одеть ребенка... обувь должна быть удобной, чтобы
не упасть, если бежать. 

А в углу комнаты сумка с документами на квартиру и обручальные кольца. Это всё,
что пока смогли взять дома. 

Я так и не забрала наши свидетельства о рождении, свидетельство о браке, мой
диплом о высшем образовании. Всё сгорело дома. 

Нам чудесным образом повезло остаться в живых...но как правильно сказано, мы
все немного там умерли. Был зелёный коридор. И мы смогли покинуть город. Целые,
невредимые. Но очень испуганные, измученные морально. Душа болит. Сердце рвется
на части. Мы лишились всего. Дом, друзья, любимая работа, семейные фотографии,
семейные архивные видео, какие то мелочи, которые напоминали нам о взрослении
наших детей. Всё забрали.

А теперь я часто слышу, это не самое главное в жизни.

Но у других это не самое главное не забрали.....

Мариуполь. хочу домой ...

(С)
