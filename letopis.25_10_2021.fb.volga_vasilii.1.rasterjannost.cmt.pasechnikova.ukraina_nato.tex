% vim: keymap=russian-jcukenwin
%%beginhead 
 
%%file 25_10_2021.fb.volga_vasilii.1.rasterjannost.cmt.pasechnikova.ukraina_nato
%%parent 25_10_2021.fb.volga_vasilii.1.rasterjannost.cmt
 
%%url 
 
%%author_id 
%%date 
 
%%tags 
%%title 
 
%%endhead 
\paragraph{Инна Пасечникова - я могу понять для чего Украине в НАТО ...}

\begin{itemize} % {
\iusr{Инна Пасечникова}

я могу понять для чего Украине в НАТО, но вот для чего Украина для НАТО понять не могу.

\begin{itemize} % {
\iusr{Василий Александрович Волга}
да просто все. Россия без Украины не Империя. И чем короче подлетное время, тем противник сильнее.

\iusr{Инна Пасечникова}
\textbf{Василий Александрович Волга} про время я в курсе. Зачем принимать в альянс, если Украина и без его на коротком поводке и в строгом ошейнике

\iusr{Павло Котвіцький}

Василий, дело не в подлетном времени, НТП идёт вперёд и подлетное время будет неумолимо сокращаться.
Вопрос в человеческом ресурсе без которого Россия не империя,
Но нынешнее состояние украинского человеческого ресурса такое, что при его интеграции разорвет Россию, и полетят ее клочки по закоулочкам, как по мне чем подальше от Украины, тем целее будет)

\iusr{Василий Александрович Волга}
\textbf{Инна Пасечникова} не знаю. это Путин чего-то переживает.

\iusr{Галина Скрипник}
\textbf{Василий Александрович Волга} Это Вы Пилсудского цитируете? Один сказанул - другие повторяют бездумно. Россия и без Украины Империя. И без Украины прекрасно развивается. А вот Украина без России попробовала жить в 1918 году пару лет - не получилось. Сейчас пробует - опять фигня какая-то. Так что формулу эту наоборот читать надо.

\iusr{Василий Александрович Волга}
\textbf{Галина Скрипник} НУ, дай Бог.
\end{itemize} % }

\iusr{Sergey Sergeyev}

Сомневаюсь, что Путин принимает какие-либо важные решения, не то что на
международном уровне, а у себя в стране - на местном уровне!

\begin{itemize} % {
\iusr{Василий Александрович Волга}
думаю, что Вы, Сергей, транслируете традиционное заблуждение.

\iusr{Sergey Sergeyev}
\textbf{Василий Александрович Волга} Не поверите - ну, очень хочу верить в то, что это моё заблуждение. К сожалению, сложившаяся практика наталкивает именно на такие суждения.

\iusr{Жужа Мартиросова}
\textbf{Sergey Sergeyev} рептилоиды?

\iusr{Sergey Sergeyev}
\textbf{Жужа Мартиросова} Это кто или что?

\iusr{Жужа Мартиросова}
\textbf{Sergey Sergeyev} сама не знаю, но они все решают

\iusr{Sergey Sergeyev}
\textbf{Жужа Мартиросова} "Шедевральный" ответ  @igg{fbicon.face.grinning.squinting} 

\iusr{Жужа Мартиросова}
\textbf{Sergey Sergeyev} что за обидные кавычки!
\end{itemize} % }

\iusr{Ирина Ромоданова}
Я не увидела никакой растерянности. И он знает,что делать.

\begin{itemize} % {
\iusr{Василий Александрович Волга}
а говорил он все свои слова просто так? или обманывал?

\iusr{Ирина Ромоданова}
\textbf{Василий Александрович Волга} 

Василий Александрович, возможно, что ещё не время для каких- либо громких
заявлений, хотя они не за горами. Но то, что он знает, что делать- больше, чем
уверена, на то он- ВВП. Это ,лишь -мое мнение.

\end{itemize} % }

\iusr{Валентина Титова}

Думаю, что Вы, Василий, правы. Поздно пить боржоми... @igg{fbicon.cry}  России надо было
начинать думать об Украине 20 лет назад и предпринимать какие-то шаги. Даже в
14-ом ещё были какие-то шансы на другой курс Украины. Россия ( Путин) тоже его
про@рала. Увы((

\begin{itemize} % {
\iusr{Василий Александрович Волга}
для меня это особая тема, начиная с моего выступления в Госдуме в 2001 году, когда я настаивал поддержать идею Карты Русского. Отказали.

\iusr{Марина Хлебникова}
\textbf{Валентина Титова} надо было 20 лет назад России об Украине думать?.... Ааа. Вон оно что, Петрович.... А не напомните ли нам тут, что с Россией было 20 лет назад?

\iusr{Валентина Титова}
\textbf{Марина Хлебникова} Ну, если России всё равно,что будет с Украиной, тогда не о чем и говорить. Пусть Украина и в ЕС вступает и в НАТО .... Ну тогда России и о Донбассе лучше забыть.

\iusr{Марина Хлебникова}
\textbf{Валентина Титова} это что такое? Предлагаете России напасть на Украину?

\iusr{Валентина Титова}
\textbf{Марина Хлебникова} Это вы предлагаете? Лично у меня такого и в мыслях не было.

\iusr{Марина Хлебникова}
А что тогда имеете ввиду?

\iusr{Валентина Титова}
Так прочитайте мой первый комментарий. Просто выразила своё мнение, совпадающее с мнением Василия. А что вам надо от меня- не понимаю.

\iusr{Марина Хлебникова}
\textbf{Валентина Титова} кстати, думаете, что Россия не терпела какие то лишения и понижение уровня жизни после украинского Майдана. Поддерживая республики, затянув поясок? Обустраивая Крым..? Но, мы все понимаем, что это нужно, в массе своей и переживаем, так как свои.

\iusr{Марина Хлебникова}
\textbf{Валентина Титова} поделитесь мыслями, как надо сделать?

\iusr{Валентина Титова}
\textbf{Марина Хлебникова} ПОЗДНО!!! ЧТО НЕПОНЯТНО???

\iusr{Марина Хлебникова}
\textbf{Валентина Титова} ну, раз пишите о том, что Россия должна сделать, напишите, как. И что поздно? Россия сама поднималась, Никого не обвиняя. Всего хватало. Чего вам поздно то?

\iusr{Марина Хлебникова}
\textbf{Валентина Титова} мне понятно, что Вы пишите ерунду.
\end{itemize} % }

\iusr{Людмила Петровская}
Да уж... трусливый Путин. Не грозит, как Крамп-Карренбауэр ядерной дубинкой.

\begin{itemize} % {
\iusr{Василий Александрович Волга}
причем тут это?
\end{itemize} % }

\iusr{Жужа Мартиросова}
Да, именно так

\iusr{Alexander Balabas}

Вот как раз об этом мы с вами уже несколько лет Василий Александрович общаемся.

Я вам о том, что у России нет ясного представления и понимания, что делать по
отношению к Беларуси и Украине. А вы про националистов и про не понятно чего
там еще...

Россия без Украины и Беларуси - это Московия. В своем геополитическом значении
это Московия. И "пещерные" методы цензуры истории 19 века в 21 в век интернета
и просвещения, не работают. Да и суть то не в том, что кто-то за Лукашенко или
против. Они в обществе белорусском были и до августа и сидели в соцсетях
ругались - жили одним словом. Что произошло в августе? Пришло просто время
"собирать камни", так всегда и везде бывает с библейских времен. Как мы с вами
учились, просто пришло время экзамена, все построились строем и пошли сдавать.
И сегодня тема Беларусь и это касается не только стран и их руководства - это
касается каждого из нас. Каждый из нас должен выбрать. Вот почему у кого-то
растерянность, а кто-то сегодня как "флюгер" обозначает буйную деятельность.
Как торговали с Лукашенко так и торгуют, а кому они все показывают это
возмущение? Беларускому народу показывают, чтобы потом сказать и рассказать как
они ему "помогали". И с этой точки зрения будущего мироустройства Европы, как
раз Украина молодцы, не отстает от Польшы, Литува молодцы ... да... пусть
"нехрена" не делают, но хоть ветер гоняют - вид показывают. Вот так Василий она
политика в долгую и делается. Если бы в августе, кто-то на Востоке сказал ...
"Досыць!", сказали бы "Григорьевич ... ну надо хоть как -то вид демократии
устраивать", Григорьевич перестань избивать народ, это не по христиански ...

Тогда бы Василий и дорожные карты не нужны были бы ... было бы у нас Союзное
государство Беларусь - Россия и без Лукашенко. И Украина бы затылок чесала. А
так имеем то что имеем, имеем то к чему пришли сами без "западных кукловодов"

Беларусь і Украіна - разам і назаўсёды. Білорусь і Україна - разом і назавжди.

Мы еще с вами доживем, как в одном новом европейском Союзе (Унии) жить будем.
Все идет к этому.

\begin{itemize} % {
\iusr{Василий Александрович Волга}
возможно

\iusr{Василий Сокол}
\textbf{Alexander Balabas} 

...Да распадётся Евросоюз задолго до того, как наступит очередь вступления в ЕС
Украины. Выйдут из ЕС Великобритания, Франция и ещё некоторые государства-
крупные финансовые покровители ЕС и наступит крах этого союза.

\iusr{Alexander Balabas}
\textbf{Василий Александрович Волга} 

Ну... это дело такое. Лет пять назад - это был только один из возможных
вариантов. И как-то подробно писал об этом, компилировал всех мудрецов
геополитики. Василий разговор не о том нравится или не нравится. Мы с вами
зрители в театре. Возможна ли геополитическая сборка Россия, Украина, Беларусь?
Да конечно - это было уже. Возможна ли сборка Беларусь, Украина и Литува?
Конечно - это было уже в истории. Вопрос, что "великие мудрецы" за 30 лет
сделали для первого или второго? 30 лет в политике - это вообще "миг". За 30
лет - русский с украинцем стали врагами. Еще пять лет тому назад Лукашенко
стоял нейтрально, пытался примирить, пытался исправить ситуацию между Россией и
Украиной. Прошел август 2020 ... какое отношение беларусов к России? Кто
поддержал избиение, кто кредиты дает? На чьи деньги бьют беларуса? Да можно
меня лично хоть в "змагары" или в "сьвядомые". Масштаб в Беларуси достиг таких
размеров, что это коснулось не кучки маргиналов с бчб флагами на Октябрьской
площади, они там с 1994 года стояли, это коснулось всех, каждого беларусы. И
нет у нас с вами инструмента, нет той ложки, которой можно взять это из головы
"вычерпать". Это вам как пропагандист замполит пишу. Есть критическая точка,
есть эта красная линия, когда хоть кол на голове чеши, хоть закрой всей сайты и
газеты, языки отрежь чтобы люди не говорили, зашей рот ... это психология, нас
такими Бог создал. Если в 1995 году 80\% беларусов были за объединение и именно
они своим решением делили украинское общество, именно они беларусы давали
сомнение украинцу, что может и мы что-то не так делаем.... То сегодня все...
тормозов нет. Россия для украинцев оккупант, а те кто не верил или сомневался,
вот тебе пример с Беларусью ... как Россия поступила ...

Но это выбор и результат деятельности Кремля и их "говорящих голов" на российском ТВ.

Понимаете Василий, тут нет сакрального. Пойдем с вами на подводную лодку, у вас
спрошу: Василий а что будет если я вот "этот" кран зарою..." Вы мне ... ну
реактор перегреется и потом рванет ... А я вам: Не верю! А давайте попробуем
... Ну вот так и в политике, и в пропаганде, которая является проводником
политики власти в обществе. Просто она есть или нет эта политика, просто есть
цели и задачи или их нет. И просто есть пропагандисты, которые строят систему
пропаганды в рамках поставленной цели и задачи: сформировать, достичь ...
информировать население, есть "демагоги хайпанутые". Точно так же как и у вас
есть инженеры профессионалы, а есть тем кому и гаечный ключ в руки давать
нельзя.

\iusr{Alexander Balabas}
\textbf{Василий Сокол} 

Не будет Украина в ЕС и НАТО.  @igg{fbicon.smile}  Это заблуждение, да есть
определенная часть общества в Украине, которая еще пока в это верит как в
мечту. Но приходит понимание, потихоньку. Это социология, это психология, тут
надо время, просто так устроенно общество. Кто-то из несознательных политиков
использует этот "тезис". Не хочу много писать, тема вообще про Россию и про
Путина - про его результат работы. Но если что могу парочку железнобетоных
аргумента привести, почему ЕС не нужна Украина, более того она нужна слабая.
Ослабить Украину - это геополитическая задача Европы как Германии так и
Франции. А если есть общие задачи, то не распадутся они.

\iusr{Александр Добровольский}

бедная Московия, что она без балабасов делать будет  @igg{fbicon.frown} 

\iusr{Alexander Balabas}
\textbf{Александр Добровольский} 

Да не чего...  @igg{fbicon.smile}  В 21 веке Кровавый потом не устроишь, войну не начнешь. Вот и
будет дальше бесится и сдавать Кавказ, Волгу - тюркскому миру.

\end{itemize} % }

\iusr{Наталия Карпенко}

Путин так хотел Крым что признал власть майданную и Порошенко это была первая и
единственная ошибка но роковая. А дальше поехало... торговля, газ споры и не
споры. Но признали. Отмотать нельзя. По этому же поводу случилась и сдача
Новороссии, изначально начали но потом притормозили. А ведь на момент 2014 года
реально половина была за Россию точно. Ну а когда порох всех кровью повязал то
ясно что сейчас это нереально на 50 лет вперед. И вот результат никто не знает
что делать с ненькой в России. Воевать ясно не хотят да и страшно думать о
таком но ситуация именно такая. Сша подталкивает неньку к мщению и вечной войне
и оружие дадут и все дадут только воюйте. Боюсь что однажды Путин не выдержит и
начнёт. И всем мало не покажется

\begin{itemize} % {
\iusr{Василий Александрович Волга}
и это может быть
\end{itemize} % }

\iusr{Alexander Shvyrkov}

Плана не было никогда. Плана не было и нет даже для России. Потому что у мелких
жуликов никогда не бывает плана. Они просто хватают то, что плохо лежит.

\begin{itemize} % {
\iusr{Наталия Карпенко}
\textbf{Alexander Shvyrkov} это примитивный ответ в духе кастрюль

\iusr{Наташа Маркова}
\textbf{Alexander Shvyrkov} это вы, конечно же, о нашем окраинном соседе @igg{fbicon.laugh.rolling.floor} !!! Это точно!
\end{itemize} % }

\iusr{Галина Соловьёва}

Я, как будто украинское ТВ посмотрела, прочитав Ваш пост. Одну и ту же речь
можно по разному преподнести. Я услышала совсем другую речь и совсем другой
посыл и намёки. Вы преподносите это, как те 60\% о которых Погребинский по моему
не говорил.

\begin{itemize} % {
\iusr{Василий Александрович Волга}
я просто цитирую. если я неверно привожу цитаты, то укажите мне на это. спасибо.

\iusr{Галина Соловьёва}
\textbf{Василий Александрович Волга} о какой растерянности Путина Вы говорите?

\iusr{Василий Александрович Волга}
\textbf{Галина Соловьёва} "Я Вас разочарую. Я не знаю ответа на этот вопрос"(с)

\iusr{Галина Соловьёва}
\textbf{Василий Александрович Волга} 

вы пишите: "Жаль, но Путин вновь повторяет избитую и ложную мантру о том, что
националистическое меньшинство на Украине заказывает музыку, а большинство
боится даже честно отвечать на вопросы социологов. " - Тысячу раз согласна с
Путиным в этом. Я читаю комментарии в одной одесской группе. Адекватные люди
боятся говорить, что они думают о ситуации на Украине. Даже если целый стадион
будет за дружбу с Россией, а у двух арбитров бандеровцев по автомату, то весь
стадион будет молчать.

\iusr{German Gorozhanski}
\textbf{Василий Александрович Волга} 

По вашему, Путин должен вывалить на паблик все планы, чтоб кто-то где-то не
подумал, что он в полной растерянности?)

\iusr{Василий Александрович Волга}
\textbf{Галина Соловьёва} 

Да читать Вы можете какую угодно группу - одесскую, львовскую, житомирскую.
Факты вещь упрямая - почти 80\% украинцев негативно или крайне негативно
относятся к России и русским. Топонимика изменена в честь титульных нацистов и
русофобов и население Украины это принимает с радостью. Протестующих единицы,
как бы громко им не давали кричать с российских СМИ. Русского языка не осталось
в системе образования и украинцы с пониманием к этому отнеслись, поскольку
русский язык - это язык агрессора. Даже те родители, которые испытывают
сложности с образованием своих детей на украинском языке, говорят, что это
правильное решение. На выборах все больше и больше побеждают партии и политики
откровенно русофобского толка. В Киеве за нацистов и русофобов проголосовали на
прошлых выборах больше 70\% киевлян.

\end{itemize} % }

\iusr{Руслан Шнуренко}
\textbf{Василий Александрович Волга} , а как Вы считаете, сколько Путин у руля* пробудет еще?Может быть с его уходом и появятся те, кто знают что делать...

\begin{itemize} % {
\iusr{Василий Александрович Волга}

не думаю. уход Путина грозит катастрофой планетарного масштаба. Чечня,
например, держится только авторитетом Путина. Кому еще Кадыров готов будет
поклониться?

\end{itemize} % }

\iusr{Лариса Власова}

Погребинский всегда осторожен. Ему на Украине жить. А у Путина за спиной
страна, о которой он в первую очередь должен думать.

Но считаю Донбасс надо защитить, там российские граждане проживают. И уже пора
плюнуть на то что подумают заокеанские "друзья". Но надо запустить Северный
поток. Надеюсь будет принято правильное решение и оружие США не будет
направлено на Донбасс.

\begin{itemize} % {
\iusr{Василий Александрович Волга}
так запуск СП2 зависит от Запада, а не от России.

\iusr{Лариса Власова}
Возможно поэтому Путин и "растерян"...пока не запустили.

\iusr{Александр Добровольский}
\textbf{Лариса Власова} при чем тут запуск СП-2? Он уже окупился, даже не будучи запущен, благодаря высокоумной политике ЕС.

\iusr{Лариса Власова}
\textbf{Александр Добровольский} , может и не при чём. Я же не Путин и даже близко не стратег, но одно понимаю, запустить СП выгоднее, чем не запустить.)))

\iusr{Александр Добровольский}
\textbf{Лариса Власова} это да. Я к тому, что его запуск имеет критическое значение для Европы, а не для России. И Европа его запустит, если будет холодно.

\iusr{Лариса Власова}
\textbf{Александр Добровольский} Время- деньги!

\iusr{Антонина Бажукова}
\textbf{Лариса Власова} Так вы уже с 2014-го Донбасс "защищаете". Результат налицо...
\end{itemize} % }

\iusr{Галина Евгеньевна Запорожцева}
а если знает, то должен по телевизору объявить?)))

\begin{itemize} % {
\iusr{Василий Александрович Волга}

я не знаю. я просто повторяю его слова о том, что он не знает. мне кажется, что
когда человек не хочет говорить что-то, он может промолчать. не так?

\iusr{Галина Евгеньевна Запорожцева}
\textbf{Василий Александрович Волга} 

не обязательно))) мы когда встретимся (давно мечтаю) и за чашечкой кофе я
выскажу свою точку зрения о бихевиоризме и политическом айкидо Путина)


\iusr{Евгений Гарец}
\textbf{Галина Евгеньевна Запорожцева} 

всё верно, "...использовать "силу" врага против него, не затрачивая своих
сил... "если по айкидо. Всё Он прекрасно знает, но дело в том, что на сегодня
Россией втрачено преимущество в том смысле, что ещё три года назад вопрос
Украины был бы решен безболезненно, а сейчас увы. И Он понимает, в отличии от
других 'президентов',последствия.

\iusr{Галина Евгеньевна Запорожцева}
\textbf{Евгений Гарец} Станиславский говорил если взяли паузу то держите ее сколько можете)))

\iusr{Paul Isaev}
\textbf{Василий Александрович Волга} Обычный человек, да, может и промолчать. Но не глава ядерной сверхдержавы.
Да и зачем говорить об этом, Путин просто сделает то, что считает правильным и необходимым.

\iusr{Paul Isaev}
\textbf{Евгений Гарец} Слишком плоско смотрите. Путин играет в 3-х мерные политические шахматы и никогда не решает только одну проблему или задачу. Незеркально, многоходово, растянуто во времени и ассимтрично. Высший пилотаж.

\iusr{Евгений Гарец}
\textbf{Paul Isaev} извини, брат, не по адресу. Поддерживаю твои слова, но не смотря на трехмерное мышление..... оперативное преимущество в вопросе Донбасса утерено и это факт.

\iusr{Paul Isaev}
\textbf{Евгений Гарец} 

утеряно в чем? Создано минимум 3 корпуса, несколько танковых батальонов,
разведка, РЭБы, артиллерия, саперные части, современное вооружение, форма,
грамотные инструктора, прекрасные фортификационные сооружения, не уступающие
сирийским, афганским, отлично зарекомендовавшим себя в реальных боях!?

Какое преимущество утеряно?) Наоборот, нарощено. Просто ждём 2024. Закончится
транзит через территорию страны 404, и вопрос Донбасса будет решен.

\iusr{Евгений Гарец}

Всё выше перечисленное радует, но можно было избежать человеческих смертей,
которые продолжаются, увеличивая количество гибнувших. А ожидание до 2024 года
это даже не знаю как назвать...

По поводу преимущества - это люди, не забывайте референдум и его результаты
были не только в Крыму. Если бы вы видели массовость голосования и результаты,
а чем закончилось? Активистов референдума уничтожили на глазах их детей, прямо
в огородах их домостроений и тогда Донбасс был в своих границах. Да и ещё много
чего... я единственно боюсь, что бы Донбасс не ожидала судьба "Коловрата "

\iusr{Галина Шулаева}
\textbf{Василий Александрович Волга} да не растерян ВВП, а опечален, огорчен, обескуражен... что угодно, только не растерян.

\iusr{Андрей Иванов}
\textbf{Галина Евгеньевна Запорожцева} ага!)Поедьте в Одессу расскажите что нужно подождать А смысл?

\iusr{Василий Александрович Волга}
\textbf{Галина Шулаева} пусть так

\iusr{Leniev Madalid}
\textbf{Василий Александрович Волга} или соврать

\iusr{Giorgio Alessandro Torlonia}
\textbf{Василий Александрович Волга} Путин играет тонко. Ждать недолго - узнаете.

\iusr{Галина Евгеньевна Запорожцева}
\textbf{Андрей Иванов} сударь, разве я вас держу за ноги? выйдете на Куликовое поле и расскажите там свою позицию, что Вас сдерживает? диван?

\iusr{Василий Александрович Волга}
\textbf{Giorgio Alessandro Torlonia} да я не против. И уже 8 лет жду. Могу ещё 8.

\iusr{Евгений Гарец}
\textbf{Василий Александрович Волга} ничего личного, но благодаря таким суждениям мы в полной ж.... Не могу поверить, бывший военный, флотский, и перешел в режим ожидания, да, не хорошо. Беда, что вы не одиноки. Ждите...

\iusr{Василий Александрович Волга}
\textbf{Евгений Гарец} а у Вас есть предложения?

\iusr{Евгений Гарец}
\textbf{Василий Александрович Волга} открыто компрометирующий вопрос, но отвечу - предложений много, средств нету  @igg{fbicon.smile} 

\iusr{Галина Евгеньевна Запорожцева}
\textbf{Евгений Гарец} отож))) собрались воевать, а патроны не получили))))

\iusr{Василий Александрович Волга}
\textbf{Евгений Гарец} так хоть одно для примера

\iusr{Андрей Иванов}
\textbf{Галина Евгеньевна Запорожцева}
Да простите.
Погорячился.
Видел вас сегодня в телевизоре!
Прекрасная женщина!

\end{itemize} % }

\iusr{Лариса Минькова}

Смешно, где Вы увидели растерянность? Путин - профессиональный игрок по
политической сцене. Кроме того, сторонник восточной философии, которая Вам все
объяснит. И он в первую очередь думает о россиянах, а уж потом о бедных
приблудных родственниках.

\begin{itemize} % {
\iusr{Василий Александрович Волга}
да куда мне... я ведь по глупости своей все больше привожу цитаты.

\iusr{Лариса Минькова}
\textbf{Василий Александрович Волга} Вы вкладываете в них свой смысл.
\end{itemize} % }

\iusr{Валерий Волощук}
Дядька с 1947 года, если не ошибаюсь....
Чего ему бояться?
 @igg{fbicon.shrug} 

\begin{itemize} % {
\iusr{Василий Александрович Волга}
это Вы о чем, Валерий?

\iusr{Валерий Волощук}
\textbf{Василий Александрович Волга} ну Вы же Василий пишите, что мол, говорит боязно... боится, что скажет слишком много правды об Украине...
Я о Погребинском
\end{itemize} % }

\iusr{Антон Деч}

Проблема не в том, что Украина переродилась, а что украинское общество и
экономика сгнили на корню.

Люди не знают, как жить без тотального хитрованства и нaeбaтельcтвa всех всеми,
экономика изношена до предела. Люди охотно воспринимают любые манипуляции
политиков. Дезориентированными людьми вообще легко манипулировать.

Нет страны, которая может вытянуть Украину с тем количеством населения, которое
там есть, из этой ямы.

Запад, играя с Украиной, тоже ничего хорошего не имеет.

Неподготовленным европейцам Украина сносит крышу, от неограниченной коррупции и
отсутствия сдерживающих факторов. Вон у Байдена Украина практически разрушила
личность сына. Не думаю, что Байден от этого кайфует.

Украина - это источник социальных болезней и бездонная экономическая черная
дыра.

Путин прав. Ничего делать не надо. Надо ждать, пока это перегниет само. Тем кто
бежит, давать гражданство. Для них вон решили целый город строить под
Владивостоком. Кстати советую, широта чуть южнее Сочи.

А остальные пусть живут, как им нравится, поют гимн, красят заборы и выбирают
очередного спасителя из клоунов и популистов, пока не надоест. Сколько на это
уйдет время, х/з. Может 10 лет, может 50.

\iusr{Владимир Приказчиков}
Ещё не вечер, время покажет, так вам Путин всё и сказал...

\iusr{Крень Виктор}

Как Вы могли повестись на слова ( я не знаю) и ответы у него есть как решил вопросы в
САР Афганистана если коротко всё будет после 2024 года

\iusr{Сергей Концеропятов}
Я воспринял Его ответ иначе. Он знает, что делать, но не считает нужным это озвучивать СЕЙЧАС...

\begin{itemize} % {
\iusr{Василий Александрович Волга}
Ага. Спасибо, что разъяснили. теперь когда Путин будет говорить "Я не знаю", понимать это мы будем наоборот.

\iusr{Крень Виктор}
\textbf{Сергей Концеропятов} он всегда знает настоящий разведчик

\iusr{Крень Виктор}
\textbf{Василий Александрович Волга} Василий не обижайтесь но это факт
Просто хочу одну вещь открыть медведь единственное животное который перед нападением не выражает эмоции...

\iusr{Светлана Матвеева}
\textbf{Сергей Концеропятов}, я тоже так думаю

\iusr{Сергей Концеропятов}
\textbf{Василий Александрович Волга} я и не думал вас поучать, либо УМНИЧАТЬ. Англоговорящие говорят IMHO - "in my humble opinion" - по моему скромному мнению то есть. Это вы - ЛОМ. Я - обычный смертный, имею право на ошибку, нет?)

\iusr{Василий Александрович Волга}
\textbf{Сергей Концеропятов} все мы грешны, и уж тем более я. Но никогда я не был человеком, который в угоду толпе будет говорить то, что хочет слышать толпа, а не то, что есть на самом деле. С уважением.
\end{itemize} % }

\iusr{Gennady Anatolievich}

зомби пропаганда плюс тупизм населения на выходе ненависть к РФ да и вообще
всему, что соответствует здравому смыслу

\iusr{Максим Граник}

Если драка неизбежна - надо бить первым. (В.В.Путин) Если враг не сдается - его
уничтожают. (А.М.Горький) Но это все в теории красиво звучит. На практике это
горы трупов и моря крови. Которых никто не хочет. Но решение принимать все
равно придется.

\iusr{Елена Сенникова}

ММне показалось, что этот ответ-только для публики. Для широкой
аудитории... Где-то ТАМ в глубинах его проектов ,наверняка есть
план... Просто, пока ,не пришло время ни нам его узнать, ни его
реализовать. думаю, что и с его стороны есть слабая, но, НАДЕЖДА на то, что
укровласть как-то, придёт в себя и очнётся от морока, в который впала.

\iusr{Yury De Sign}

Я не знаю, доживу ли я до того момента, когда украинская элита поймёт: только
она одна ( в этом и есть миссия национальной элиты) определит, что делать со
своей страной, не оглядываясь ни на кого. Пошли они все на х\#й - Эвропа, страна
п\#здно-волосатого флага и Расея. Да, но для этого надо уничтожить одним махом
большую часть нонешней элитки - в том числе и Вас, милостивый государь,
придётся зарыть в одной яме с Сергеем Лещенко, Мучтафой Найемом и Давидом
Арахамией. Нет у нас сорока лет и пустыни тоже, впрочем. А на вашего Путина
ладошки потеют и сердцебиение учащается только у украинских путиноидов - в
России он уже обрыдл большинству обнищавшего народа. Если б Кремль не мухлевал,
то был сейчас президентом Грудинин, а в Думе большинство занимало КПРФ.

\begin{itemize} % {
\iusr{Любовь Чуб}
\textbf{Yury De Sign} у нас есть элита?

\iusr{Olga Vlas}
\textbf{Yury De Sign} про Грудинина - смешно

\iusr{Марина Вадимовна Медведева}
\textbf{Yury De Sign} , начни с себя...)))

\iusr{Yury De Sign}
\textbf{Марина Вадимовна Медведева} , я давно заметил, что Волге аплодирует по большей части быдло типа тебя

\iusr{Paul Isaev}
\textbf{Yury De Sign} Ворюга Грудинин? @igg{fbicon.man.facepalming}  @igg{fbicon.shrug}  @igg{fbicon.face.tears.of.joy}{repeat=4} 

\iusr{Yury De Sign}
\textbf{Paul Isaev} меня более смешит оксюморон честный Путин.

\iusr{Yury De Sign}
\textbf{Любовь Чуб} , пока очень мало - до незаметности. А вообще речь я вёл об "элитке".

\iusr{Yury De Sign}
\textbf{Анатолий Жуков} , сам-то понял, чё написал?

\iusr{Марина Вадимовна Медведева}
\textbf{Yury De Sign} , у вас что - то со зрением? ))

\iusr{Марина Вадимовна Медведева}
\textbf{Yury De Sign} , смешной здесь ты.))

\iusr{Paul Isaev}
\textbf{Yury De Sign} кого бы ещё интересовало твоё эксперДное мнение @igg{fbicon.shrug}  @igg{fbicon.man.facepalming}  @igg{fbicon.face.tears.of.joy} 

\iusr{Yury De Sign}
\textbf{Марина Вадимовна Медведева} -тупой волгинской бот
\end{itemize} % }

\end{itemize} % }
