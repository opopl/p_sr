% vim: keymap=russian-jcukenwin
%%beginhead 
 
%%file 16_03_2019.stz.news.ua.mrpl_city.1.park_gurova_razmyshlenia_dosuzhego_passazhira_tramvaja
%%parent 16_03_2019
 
%%url https://mrpl.city/blogs/view/mariupolskij-park-im-na-gurova-razmyshleniya-dosuzhego-passazhira-tramvaya
 
%%author_id burov_sergij.mariupol,news.ua.mrpl_city
%%date 
 
%%tags 
%%title Мариупольский парк им. Н. А. Гурова: размышления досужего пассажира трамвая
 
%%endhead 
 
\subsection{Мариупольский парк им. Н. А. Гурова: размышления досужего пассажира трамвая}
\label{sec:16_03_2019.stz.news.ua.mrpl_city.1.park_gurova_razmyshlenia_dosuzhego_passazhira_tramvaja}
 
\Purl{https://mrpl.city/blogs/view/mariupolskij-park-im-na-gurova-razmyshleniya-dosuzhego-passazhira-tramvaya}
\ifcmt
 author_begin
   author_id burov_sergij.mariupol,news.ua.mrpl_city
 author_end
\fi

Если приходилось ехать на городском транспорте вдоль парка имени Н.А. Гурова,
вспоминались те времена, когда старая часть Мариуполя была отделена от завода
Ильича, - именно так называли коренные мариупольцы теперешний Кальмиусский
район с его заводами и жилыми кварталами. Границей районов была речка Кальчик.
За ней находилась череда пустырей и огромный фруктовый сад. Сейчас при поездке
в трамвае вдруг возникает картина разъезда трамваев перед мостом через Кальчик
с двухколейного пути на одноколейный. Всплывают      Т-образные опоры,
поддерживающие контактный провод, трамваи с открытыми порой даже в движении
дверями, с сиденьями из лакированных брусков, с кондукторами, узнаваемыми по
стандартными сумкам на груди. Но самое сильное впечатление из прошлого – море
цветущих деревьев весной. Старики называли пространство с фруктовыми деревьями
садом Ликаки. Горожане помоложе – совхозным садом. Какого совхоза? Да как-то не
было желания узнавать. И так ли это важно? Другое дело - фамилия Ликаки.
Порывшись в краеведческих книгах, можно узнать, что \textbf{Иван Демьянович Ликаки} был
гласным Мариупольской думы, по нашим понятиям, депутатом городского совета. У
Ивана Демьяновича был подворье на Торговой улице. А еще он владел кирпичным
заводом.

\vspace{0.5cm}
\begin{minipage}{0.9\textwidth}
	
\textbf{Читайте также:}

\href{https://mrpl.city/blogs/view/mariupolski-parki-stanut-the-best}{Маріупольські парки стануть the best, Олександр Куць, mrpl.city, 12.10.2018}
\end{minipage}
\vspace{0.5cm}

Естественно, в наши дни вряд ли можно встретить людей, помнивших сады Ликаки, а
фруктовые насаждения какого-то совхоза на памяти еще довольно многих
мариупольцев. Но случилось так, что фруктовые деревья постепенно стали
исчезать. И образовалось пустое пространство.  В 1982—1990 годах директором
комбината имени Ильича был \textbf{Николай Алексеевич Гуров}. Вот при нем и,
естественно, при его участии начались работы по преобразованию территории, где
когда-то были сады. Так появился новый парк в Мариуполе...

Однажды при встрече с добрым давним знакомым \textbf{Николаем Леонтьевичем Гахом}
довелось услышать следующее: 

\begin{quote}
\em\enquote{А ты знаешь, как в народе до войны называли
остановку трамвая, известную сейчас как \enquote{5-й микрорайон}? А называлась она
\enquote{Чанки}. Почему? Оказывается, названа она была по фамилии хозяина угодий,
которые были вблизи этой остановки: на пригорке располагались его виноградники.
Хозяин – Чанков, болгарин по происхождению. Между прочим, он доводится прадедом
моей жене Елене Ивановне. Когда поселился в нашем городе \textbf{Чанков Коню
Михайлович}, переименованный по-нашему в \textbf{Константина Михайловича}, - сведений
нет. Известно только, что ему были выделены в то время не очень удобные, по
мнению местных жителей, угодья в пойме реки Кальчик, где сейчас парк имени
Гурова. Будучи грамотным ирригатором, он быстро обустроил все дела
ирригационные и начал заниматься сельским хозяйством. Завез сюда из Болгарии
перец, баклажаны, отменные сорта капусты, огурцов, помидоров и других овощей. И
буквально через какой-то промежуток времени он стал очень богатым и влиятельным
человеком. Начал приобретать другие участки земли, а также жилые дома с
подворьями. Короче говоря, повел очень бурную такую предпринимательскую жизнь.
К нему приехала семья - жена и четверо детей. В том числе и любимая дочь Коню -
Стояна, Стояна Константиновна.}

\end{quote}

\textbf{Читайте также:} 

\href{https://mrpl.city/news/view/park-im-gurova-v-mariupole-obnovyat-fontanami-besedkami-magnoliyami-i-barbarisom-foto}{Парк им. Гурова в Мариуполе обновят фонтанами, беседками, магнолиями и барбарисом, Яна Іванова, mrpl.city, 05.03.2018}

\begin{quote}
\em\enquote{У потомков Коню сохранилось кое-что из документов, которые приоткрывают
некоторые страницы мариупольского периода жизни Чанкова. В удостоверении,
выданном 13 ноября 1909 года помощником командира стражи, ротмистром (подпись
которого разобрать теперь невозможно), записано: \enquote{Дано сие
болгарско-подданому (Коню) Константину Михайловичу Чанкову 42 лет,
православного вероисповедания для проживания впредь до исходатайствания
национального паспорта от надлежащих властей}. Есть еще один документ: о
приобретении Чанковым у Евфимии Ивановны Коссе фруктового сада. Эта бумага
датирована 20 сентября 1905 года. Из этого можно сделать вывод, что Чанков
прибыл в наши края не позднее 1905 года. Фруктовый сад, купленный у Евфимии
Коссе, и дом на Бахмутской – подарок дочери - были не единственными объектами
недвижимости Коню Михайловича в Мариуполе. Чанков стремился, и это ему
удавалось, расширить, говоря современным языком, свою производственную и
материальную базы. В 1922 году он взял в аренду у Мариупольского земельного
отдела шесть десятин земли, прилегающих к его усадьбе. В 1927 году Константин
Михайлович купил у Марии Федоровны Поповой домовладение на улице Советской,
рядом с бывшей гостиницей \enquote{Континенталь}. 

Бурная деятельность Коню Чанкова в Мариуполе, успешно начатая в дореволюционные
годы, продолжалась и в период Новой экономической политики (НЭП),
провозглашенной X съездом РКП(б) весной 1921 года. Но к началу 30-х годов НЭП
был фактически свернут. Он почувствовал неладное, по доброй воле или по
принуждению сдал государству все свои огороды, сады и виноградники, за бесценок
продал домовладения, а подворье на улице Бахмутской переписал на Стояну, сам же
уехал в Болгарию. Перебравшись на родину, Чанков, сам того не зная, избежал
участи тех мариупольских болгар, которые в 1937 году были арестованы, а потом
расстреляны по ложному обвинению в контрреволюции, шпионаже в пользу
монархо-фашистской Болгарии и в других надуманных обвинениях. Дальнейшая его
судьба неизвестна}...

\end{quote}

\textbf{Читайте также:} 

\href{https://mrpl.city/news/view/park-gurova-v-mariupole-prodolzhit-191-ga-hvojnogo-lesa}{Парк Гурова в Мариуполе продолжит 191 га хвойного леса, Роман Катріч, mrpl.city, 19.10.2018}

Такие воспоминания иногда роились в голове досужего пассажира в трамвае марке
№1 при поездке из города на завод Ильича или обратно. И вот явилась благая
весть. Парк будут реконструировать. Хочется верить, что в обновленном парке
будут не только красивые по новой моде дорожки, но и много зелени, такой
желанной в жаркую пору.
