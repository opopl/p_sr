% vim: keymap=russian-jcukenwin
%%beginhead 
 
%%file 27_04_2021.fb.fb_group.story_kiev_ua.1.podvig_chaes
%%parent 27_04_2021
 
%%url https://www.facebook.com/groups/story.kiev.ua/posts/1649191888610939
 
%%author_id fb_group.story_kiev_ua,jabloneva_tonja.kiev
%%date 
 
%%tags avaria_na_chaes,chaes,chernobyl,geroizm
%%title Історія подвигу трьох працівників ЧАЕС
 
%%endhead 
 
\subsection{Історія подвигу трьох працівників ЧАЕС}
\label{sec:27_04_2021.fb.fb_group.story_kiev_ua.1.podvig_chaes}
 
\Purl{https://www.facebook.com/groups/story.kiev.ua/posts/1649191888610939}
\ifcmt
 author_begin
   author_id fb_group.story_kiev_ua,jabloneva_tonja.kiev
 author_end
\fi

26 квітня 00 год. 23 хв., 35 років назад, сталася найбільша катастрофа в історії
людства. Не буду повторювати свої спогади про чорнобильський Київ – моторошно
згадувати, та й вже написано чимало. Але оскільки ця трагедія  буде обпікати
свідомість ще багато років, бо і досі її наслідки толком не виявлені, - хочу
поділитися інформацією про маловідомих героїв, про яких дізналася з  газети
«Новини ЧАЕС» трирічної давнини. (2018). (Можливо, ви про них знаєте?).

\ii{27_04_2021.fb.fb_group.story_kiev_ua.1.podvig_chaes.pic.1}

...Ми й до сьогодні чуємо про забуті подвиги і  розвінчуємо міфи Протягом
багатьох років історія подвигу трьох працівників ЧАЕС, яка висвітлювалася в
різних ЗМІ була спотворена... 

Трохи передісторії. Після катастрофи виникла ще одна загроза, ще один вибух,
після якого навряд чи був шанс на порятунок у мільйонів людей...

Близько 185 тонн розплавленого ядерного матеріалу зі зруйнованого реактора
пропалювали бетонну плиту під реактором. Під ним в басейні-барботері
накопичувалася вода, що використовувалась під час гасіння пожежі в перші дні
після аварії. Її об'єм оцінювався  в 19 тис. тонн. У випадку, якщо б
розплавлена активна зона дісталася води, була висока ймовірність повторного,
парового вибуху.

Урядова комісія після консультацій з академіками Є. П. Веліховим і В. А.
Легасовим визнала загрозу такого перебігу подій. 

На десятий день після аварії, 6 травня 1986 року, силами персоналу зміни ЧАЕС,
військових і пожежників була здійснена операція зі звільнення приміщень від
води. Після монтажу гнучкої траси велику частину рідини за короткий термін було
відведено з басейну в спеціальне безпечне місце, але залишилася чимала
кількість води, яку можна було спустити лише одним способом—відкривши зливні
вентилі басейнів Четвертого енергоблоку, що знаходяться в неосвітлених
приміщеннях, з високим рівнем радіації і під півметровим шаром радіоактивної
води з концентрацією до 10 кюрі на літр. На всю роботу відводилося 15 хв.,
виконати її визвалися добровольці: інженери ЧАЕС  Олексій Ананенко і Валерій
Беспалов Іначальник зміни станції Борис Баранов. Одягнені в гідрокостюми,
«озброєні» ліхтарями і необхідним інструментом, вони мали відправитись через
затоплені камери четвертого реактора, знайти два запірних клапани й відкрити
їх, щоб спустити воду. Вони спустилися в басейн-барботер під зруйнованим
реактором. Один із них тримав підводну лампу,  двоє інших відкривали клапани.
Не без зусиль їм вдалося спустошити смертоносний бассейн, отримавши за цей
короткий час близько 10 річних доз радіації (!) ... Шансів вижити не було ні у
кого, і вони про це знали... Мабуть,  саме тому в подальших публікаціях ( і у
фільмі «Чорнобиль») відзначалося, що після операції по спорожненню басейна,
старший інженер-механік реакторного цеху № 2 Олексій Михайлович Ананенко і
старший інженер управління блоком № 3 Валерій Олексійович Беспалов померли
через десять днів в одній з московських лікарень, а начальник зміни станції
Борис Олександрович Баранов прожив трохи довше. І що нібито усіх трьох ховали в
щільно запаяних цинкових трунах... 

Але насправді ця історія мала  зовсім інший фінал. Після успішно проведеної
операції, герої не померли!  (І тут питання: або вчені помилилися і вони не
отримали таких величезних доз радіації, або були ефективними гідрокостюми, або
зіграв його величність випадок. Історія про це і досі мовчить). Тим не менше,
заступник Голови Ради міністрів колишнього СРСР, голова Урядової комісії з
ліквідації наслідків аварії на ЧАЕС Б.Щербина особисто подякував героям,
пообіцявши високі нагороди. Однак, їх подвиг так і не був тоді гідно
відзначений... 


Ці троє героїв-добровольців  запобігли ще одному  катастрофічному вибуху...

АНАНЕНКО ОЛЕКСІЙ МИХАЙЛОВИЧ– зараз йому 62 р., український енергетик, старший
інженер-механік   на Чорнобильській АЕС. Після аварії продовжував працювати  за
спеціальністю – в Держатомнагляді, різних установах. У 2017 його збив
автомобіль, і він кілька місяців знаходився у комі... У 2018 вийшов на пенсію.

БЕСПАЛОВ ВАЛЕРІЙ ОЛЕКСІЙОВИЧ – зараз йому 64 р.. інженер-теплоенергетик,
старший інженер-механік турбінного цеху під час аварії.  З 2012 р. працював
диспетчером на Енергоатомі.

БАРАНОВ  БОРИС ОЛЕКСАНДРОВИЧ -1940-2005. Інженер-теплоенергетик. На час аварії
– начальник зміни станції. Після ліквідації наслідків аварії, продовжував
працювати на станції до своєї смерті. Помер від серцевого нападу у 2005 р.

У 2019 всі троє були удостоєні, нарешті , звання «Герой України», Б.О.Баранов –
посмертно...

... Які ще таємниці і міфи досі зберігає Чорнобиль? 

На фото: Ананенко О.М.

Беспалов В.О. 

Баранов Б.О.

ДЯКУЄМО ВАМ ЗА ЖИТТЯ!!!

\ii{27_04_2021.fb.fb_group.story_kiev_ua.1.podvig_chaes.cmt}
