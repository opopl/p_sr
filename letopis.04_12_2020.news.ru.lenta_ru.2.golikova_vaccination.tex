% vim: keymap=russian-jcukenwin
%%beginhead 
 
%%file 04_12_2020.news.ru.lenta_ru.2.golikova_vaccination
%%parent 04_12_2020
 
%%url https://lenta.ru/news/2020/12/04/vaccregss/
 
%%author 
%%author_id 
%%author_url 
 
%%tags 
%%title Голикова объявила сроки начала вакцинации во всех регионах России
 
%%endhead 
 
\subsection{Голикова объявила сроки начала вакцинации во всех регионах России}
\label{sec:04_12_2020.news.ru.lenta_ru.2.golikova_vaccination}
\Purl{https://lenta.ru/news/2020/12/04/vaccregss/}

\index[names.rus]{Голикова, Татьяна!Вице-премьер России!04.12.2020, вакцинация}
\index[rus]{Коронавирус!Россия!Вакцинация}

\ifcmt
pic https://icdn.lenta.ru/images/2020/12/04/18/20201204182327067/pic_4673fd803454f9b2c83581b51cdea993.jpg
caption Татьяна Голикова. Фото: Александр Миридонов / «Коммерсантъ»
\fi

Все регионы России подключатся к вакцинации в конце следующей недели. Такие
сроки объявила вице-премьер Татьяна Голикова во время правительственного
брифинга, передает ТАСС.

Голикова добавила, что вакцинация уже началась в армии и Москве.

2 декабря президент России Владимир Путин поручил начать вакцинацию от
коронавируса. Он заявил, что это возможно благодаря объемам произведенной в
стране вакцины Спутник V. Российский лидер добавил, что в ближайшее время
количество выпущенных доз достигнет двух миллионов.

4 декабря в Москве открылась запись на вакцинацию от коронавируса на портале
mos.ru. По словам мэра столицы Сергея Собянина, за первые пять часов запись
совершили пять тысяч человек. Прививочные пункты заработают в Москве 5 декабря,
всего в городе их 70.

Для получения прививки надо быть прикрепленным к московской поликлинике.

Первыми препарат получат основные группы риска — врачи, учителя, сотрудники
городских служб, социальные работники. Категории расширят после поступления
новых партий вакцин.
