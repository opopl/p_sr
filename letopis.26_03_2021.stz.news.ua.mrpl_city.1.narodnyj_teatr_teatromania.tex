% vim: keymap=russian-jcukenwin
%%beginhead 
 
%%file 26_03_2021.stz.news.ua.mrpl_city.1.narodnyj_teatr_teatromania
%%parent 26_03_2021
 
%%url https://mrpl.city/blogs/view/narodnij-teatr-teatromaniya
 
%%author_id demidko_olga.mariupol,news.ua.mrpl_city
%%date 
 
%%tags 
%%title Народний театр "Театроманія"
 
%%endhead 
 
\subsection{Народний театр \enquote{Театроманія}}
\label{sec:26_03_2021.stz.news.ua.mrpl_city.1.narodnyj_teatr_teatromania}
 
\Purl{https://mrpl.city/blogs/view/narodnij-teatr-teatromaniya}
\ifcmt
 author_begin
   author_id demidko_olga.mariupol,news.ua.mrpl_city
 author_end
\fi

Бажання знайти оригінальні засоби сучасного мистецького втілення відчуваються в
роботі Народного театру \enquote{Театроманія}, засновником та керівником якого є
\emph{\textbf{Тельбізов Антон Миколайович}}.

Народний театр \enquote{Театроманія} був створений у січні 2011 року і цьогоріч
відзначив \emph{\textbf{10-річний ювілей}}. У 2016 році колектив отримав звання \emph{\enquote{Народний}}. За
9 років праці колективу набули навичок акторської майстерності більше 200
учасників. Проте основний склад трупи залишається незмінним та нараховує 15
осіб основного складу та одну підготовчу групу у кількості 10 учасників.

\ii{26_03_2021.stz.news.ua.mrpl_city.1.narodnyj_teatr_teatromania.pic.1}

Колектив відіграє важливу роль у художньо-творчому та духовно-естетичному
розвитку молоді міста, постійно бере участь у різноманітних фестивалях,
конкурсах, міських заходах. Проте основний напрям роботи – пошук нових форм в
сценічному мистецтві, гастролі та розробка соціальних проєктів.

Впродовж 10 років НТ \enquote{Театроманія} створив наступні вистави:  \enquote{13 хвилин тиші},
\enquote{Омлет з гірчицею}, \enquote{Дорогою до щастя}, \enquote{Королівські ігри}, \enquote{Готель двох
світів}, \enquote{Як це робиться в Одесі}, \enquote{Гупьошка}, \enquote{Викрадення Дженні}, \enquote{Тіні
забутих предків}.

\ii{26_03_2021.stz.news.ua.mrpl_city.1.narodnyj_teatr_teatromania.pic.2}

Також колектив завжди бере участь у поданні грантів та їх реалізації. За
останній рік колектив спільно з Організацією Об'єдна\hyp{}них Націй провів гастролі у
Донецькій, Луганській та Запорізькій  областях, де підняв дуже важливу
соціальну тему домашнього насильства. У 2016 та 2019 роках театр отримав
міні-гранти від Маріупольської міської ради  на створення арт-виставки \enquote{Вибір
за тобою}, та діджиталізованої вистави \enquote{Тіні забутих предків}.

У репертуарі театру є авторські постанови: \enquote{Сни о}, \enquote{\#соТворение},  \enquote{Прощавай
зброя}, Арт-виставка \enquote{Вибір за тобою}.

Для молодшої аудиторії розроблені казки.

У репертуарі театрального колективу також є естрадно-театраль\hyp{}ні номери:
\enquote{Вагонетка}, \enquote{Втеча}, \enquote{Зірка}, \enquote{Розлука}, \enquote{Універ}, ритмічний етюд – \enquote{Годинник
стукає} та естрадно-пародійний номер \enquote{Інтерни}.

Цікаво, що актори театру брали участь у зйомках кліпу \enquote{Дим багаття}, \enquote{Не буває
Рута червоною}, \enquote{Хочу жити без війни}.

Водночас колектив є активним учасником створення міждисциплінарних
міжнародних проєктів за участі іноземних митців та театрів. Так у 2019 році
створено диджиталізовану виставу\par\noindent\enquote{Lorenc.io} між Маріуполем та Дортмундом.

\enquote{Театроманія} впродовж свого існування отримала безліч нагород, подяк і
сертифікатів. Міська влада неодноразово відзначала великий внесок народного
театру у культурний розвиток міста.  Зокрема, у серпні 2012 році колектив
отримав диплом переможця міського конкурсу \enquote{Майбутнє Маріуполя 2012}, в
номінації \enquote{Найкращий творчий колектив}. У лютому 2018 році Народний театр взяв
участь у Міжнародному фестивалі народної творчості та отримав диплом першого
ступеня.

2019 рік приніс також приніс \enquote{Театроманії} багато нагород, серед яких: дипломи
за найкращу режисерську роботу та акторський ансамбль на фестивалі аматорського
театру \enquote{День театру} та диплом першого рівня на фестивалі \enquote{Лавина талантів}.

\ii{26_03_2021.stz.news.ua.mrpl_city.1.narodnyj_teatr_teatromania.pic.3}

У 2020 році театр виграв міні-грант від Маріупольської міської ради на
створення диджиталізованої вистави \enquote{Тіні забутих предків}. Спектакль з
використанням сучасних 3D-технологій, безумовно, запам'ятався глядачам. Однак
режисер Антон Тельбізов  наголосив, що всі ефекти потрібні саме для цілісності
картинки і в цій виставі є органічними. Зокрема, для спектаклю придбали
голографічний 3D-вентилятор, сніг-машину, неонові лампи і музичні інструменти.
З цією унікальною та актуальною виставою, присвяченої стосункам людини і
природи, колектив планує гастролювати містами України. Також до цієї вистави
колектив створив фільм, тривалістю 10 хвилин, присвячений процесу створення
вистави, завдяки якому вдалося показати наскільки створення однієї вистави
копітка робота. Адже глядач в залі бачить вже результат і не замислюється
скільки роботи проведено до прем'єри. Водночас у фільмі представлені роздуми на
теми життя, смерті, взаємин і покликання актора і режисера.

\ii{insert.read_also.demidko.teatralna_artil_dramkom}

Наразі \enquote{Театроманія} – це сучасний молодий театр, що завжди шукає нові форми й
засоби для діалогу з глядачами. Основний напрям роботи – висвітлення проблем
сучасності та розробка соціальних проєктів. Колектив активно гастролює, охоче
освоює не\hyp{}стандартні майданчики. В репертуарі є масштабні й камерні вистави, а
також провокаційні перформанси. \enquote{Театроманія} завжди прислуховується до серця
та тонко відчуває кожного й кожну. До Міжнародного дня театру, 27 березня,
\enquote{Театроманія} проведе аудіо зустріч зі своїм глядачем.

\ii{26_03_2021.stz.news.ua.mrpl_city.1.narodnyj_teatr_teatromania.pic.4}
