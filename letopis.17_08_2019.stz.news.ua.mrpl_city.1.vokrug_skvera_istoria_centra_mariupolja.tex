% vim: keymap=russian-jcukenwin
%%beginhead 
 
%%file 17_08_2019.stz.news.ua.mrpl_city.1.vokrug_skvera_istoria_centra_mariupolja
%%parent 17_08_2019
 
%%url https://mrpl.city/blogs/view/vokrug-skvera-istoriya-tsentra-mariupolya
 
%%author_id burov_sergij.mariupol,news.ua.mrpl_city
%%date 
 
%%tags 
%%title Вокруг сквера: история центра Мариуполя
 
%%endhead 
 
\subsection{Вокруг сквера: история центра Мариуполя}
\label{sec:17_08_2019.stz.news.ua.mrpl_city.1.vokrug_skvera_istoria_centra_mariupolja}
 
\Purl{https://mrpl.city/blogs/view/vokrug-skvera-istoriya-tsentra-mariupolya}
\ifcmt
 author_begin
   author_id burov_sergij.mariupol,news.ua.mrpl_city
 author_end
\fi

\ii{17_08_2019.stz.news.ua.mrpl_city.1.vokrug_skvera_istoria_centra_mariupolja.pic.1}

О мариупольском сквере широкой публике известно много. О нем писали. Не
пересчитать сколько раз его снимали фотоаппаратами, кинокамерами любительскими,
профессиональными и камерами телевизионными. И как-то в тени оказалось то, что
окружает сквер. Чтобы хотя бы частично пополнить наши знания об исторической
части нашего города, давайте совершим воображаемую прогулку вокруг сквера.

Начнем с подворья № 56 по проспекту Мира. До Октябрьской революции оно
принадлежало купцу Трегубову. В его состав входили двухэтажный особняк,
выходивший на улицу, и ряд строений в глубине двора. Среди них выделялся дом с
островерхой башней. Этот дом сравнительно недавно отреставрирован и сейчас
радует глаз прохожих. Помнится, что в послевоенное время там была детская
поликлиника. В этот же промежуток истории Мариуполя в двухэтажном особняке
также была поликлиника, но для взрослых. А вот в послереволюционные годы там
располагалось местное отделение ЧК, от упоминания о котором у людей холодела в
жилах кровь. После того, как в 60-х годах прошлого века для поликлиники было
построено новое здание рядом с больницей № 3, в опустевшие помещения вселилась
\enquote{Ремобувь} и другие организации.

Несколько слов о владельце столь заметного подворья. Абрам Михайлович Трегубов
был собственником паровой мельницы на Покровской площади, она была где-то рядом
с современным проспектом Металлургов. А также владел по одному домостроению  на
Георгиевской, Митрополитской и Евпаторийской улицах. Трегубов был человеком
прогрессивным – оплачивал три телефона, которые находились в его квартире, на
его мельнице, в его мучных амбарах. Заметим, стоило это удовольствие немалые
деньги. А. М. Трегубов был заметным общественным деятелем в городе. Являлся
гласным (депутатом) городской думы, одним из семи директоров уездного отделения
губернского тюремного комитета, секретарем Общества пособия нуждающимся
воспитанникам Мариупольской Александровской мужской гимназии. Следует заметить,
что в руководство подобного рода обществ выбирали особо щедрых благотворителей.

\textbf{Читайте также:} 

\href{https://mrpl.city/news/view/30-mln-grn-na-dve-zhemchuzhiny-kak-idet-rekonstruktsiya-zdanij-v-tsentre-mariupolya-foto}{%
30 млн грн на две \enquote{жемчужины}: как идет реконструкция зданий в центре Мариуполя, Анастасія Селітріннікова, mrpl.city, 12.08.2019}

Дом № 60, сложенный из красного кирпича – жилое здание, построенное, скорее
всего, в советское время. Впрочем, может быть и в предреволюционные годы. Перед
этим домом - памятный знак Жертвам голодомора и политических репрессий. Здесь
собираются члены общества \enquote{Мемориал} и другие горожане в траурные дни. Дом №
62/3 знаменит был тем, что на его первом этаже располагался магазин \enquote{Ткани}.
Его еще называли \enquote{киевским}. Теперь в нем находится то ли ресторан, то ли кафе
под названием \enquote{Нильсен}. Только непонятно – причем здесь уважаемый городской
архитектор?

В этом же доме жил профессор, доктор технических наук Владимир Федорович
Зубарев. Генеральский сын, прапорщик, удостоенный Георгиевского креста за
подвиг в Первую мировую войну, красный командир в войну Гражданскую. Затем
ленинградский Технологический институт, диплом инженера, высылка в Саратов на
комбайновый завод. В 1938 году – Новокузнецк, Сибирский металлургический
институт. Работая там, Владимир Федорович защитил кандидатскую, а через время и
докторскую диссертацию. Почему-то запомнилась фраза Владимира Федоровича:

\begin{quote}
\emph{\enquote{Каждый инженер должен знать, по меньшей мере, три иностранных языка:
английский, немецкий и французский, а инженер-металлург — еще и шведский. Я, к
сожалению, шведский не знаю}}.
\end{quote}

Где-то по соседству с профессором Зубаревым была квартира Капустиных. Отец
семейства – Евгений Александрович Капустин, вошедший в историю нашего города,
как известный отечественный ученый-металлург и теплофизик, доктор технических
наук, профессор местного металлургического института, а с 1969 по 1981 год и
его ректор. Его жена – Маргарита Ивановна, доцент, кандидат технических наук.
Несколько поколений ее питомцев - прокатчиков, ученых и производственников - с
теплотой и искренней благодарностью вспоминают великолепные педагогические
способности, доброту, отзывчивость Маргариты Ивановны, ее готовность помочь
студентам в разрешении проблем в учебе или быту. Дети Маргариты Ивановны и
Евгения Александровича также посвятили себя науке. Галина Евгеньевна стала
доктором физико-математических наук, Алексей Евгеньевич профессором, доктором
химических наук.

На верхотуре этого дома под башенкой находилась мастерская, в которой работали
молодые художники Олег Ковалев и Евгений Скорлупин, недавние выпускники
Ростовского художественного училища. Порой там собирались друзья, чтобы
поговорить об искусстве, обменяться мнением о прочитанных книгах. Это были
живописец Виктор Кофанов, получивший образование в Саратовском художественном
училище, Эдуард Сахаров - инженер и талантливый самодеятельный художник, Вартан
Вартанян, геолог с высшим образованием, который учился и окончил в Ленинграде
искусствоведческое отделение Академии художеств, сотрудник краеведческого музея
историк Юрий Тельбизов и другие. Через годы Виктору Кофанову, Олегу Ковалеву и
Евгению Скорлупину было присвоено звание Заслуженных художников Украины.

\vspace{0.5cm}
\begin{minipage}{0.9\textwidth}
\textbf{Читайте также:} 

\href{https://mrpl.city/news/view/dvor-za-million-poyavitsya-v-tsentre-mariupolya-foto}{%
Двор за миллион появится в центре Мариуполя, Кіра Булгакова, mrpl.city, 16.08.2019}
\end{minipage}
\vspace{0.5cm}

Дом № 63/1. В нем находился хлебный магазин \enquote{Наталка}, в котором всегда был
свежий хлеб, изумительного вкуса сдобные булочки и слойки. Все эти произведения
местного хлебозавода источали аппетитный аромат. Иногда истинные ценители
вкусного хлеба приходили в \enquote{Наталку} издалека. Этот дом примечателен еще и тем,
что в нем находилась квартира Ивана Георгиевича Казанцева - ученого-металлурга,
заслуженного деятеля науки и техники УССР, доктора технических наук, профессора
Ждановского металлургического института, автора более 80 научных работ, многие
из которых актуальны и сейчас. О его научном авторитете свидетельствует такой
факт. В 1957 году Иван Георгиевич в составе очень немногочисленной делегации
авторитетных ученых представлял Украину, как члена ООН, на втором Международном
конгрессе металлургов в Чикаго (США).

Покровский храм, строящийся по инициативе Александра Владимировича Савчука,
бывшего Генерального директора ОАО \enquote{Азовмаш}. Рядом с ним стоит деревянная
церквушка. А сразу за церквушкой строящееся здание, о котором кто-то сказал,
что это будет ресторан. Именно на этом месте стоял дом Ивана Алексеевича
Попова, городского головы Мариуполя с 1896 по сентябрь 1917 года. При его
власти в Мариуполе появились электрические и телефонные сети, водопровод, были
сооружены городская больница, здания Мужской гимназии, Епархиального училища и
многое, многое другое. В послевоенное время в доме Попова находился детский сад
\enquote{Голубок}. В конце 80-х он выглядел довольно крепким. Это видно на фотографии
1989 года, приложенной к документу под названием \enquote{Предложения по реконструкции
и реставрации исторической застройки города Мариуполя}. Похоже, что строение
переходило из рук в руки, пока не разрушилось.

Дом со шпилем по адресу - ул. Куинджи, 48. Он был построен на месте сожженного
гитлеровцами особняка, принадлежавшего Давиду Александровичу Хараджаеву. Самому
богатому горожанину Мариуполя, вместе с тем и самому большому благотворителю.
На этом мы и завершим прогулку вокруг сквера.

\textbf{Читайте также:} 

\href{https://mrpl.city/news/view/mariupolskij-hram-pokrovasamyj-vysokij-sobor-v-ukraine-planirovalos-stroit-eshhjo-vyshe-foto}{%
Мариупольский храм Покрова - самый высокий собор в Украине, планировалось строить ещё выше, mrpl.city, 10.07.2019}
