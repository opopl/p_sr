% vim: keymap=russian-jcukenwin
%%beginhead 
 
%%file 12_02_2019.stz.news.ua.mrpl_city.1.oleksandr_chernov_hranytel_hud_cinnostej_mrpl
%%parent 12_02_2019
 
%%url https://mrpl.city/blogs/view/oleksandr-chernov-hranitel-hudozhnih-tsinnostej-mariupolya
 
%%author_id demidko_olga.mariupol,news.ua.mrpl_city
%%date 
 
%%tags 
%%title Олександр Чернов: хранитель художніх цінностей Маріуполя
 
%%endhead 
 
\subsection{Олександр Чернов: хранитель художніх цінностей Маріуполя}
\label{sec:12_02_2019.stz.news.ua.mrpl_city.1.oleksandr_chernov_hranytel_hud_cinnostej_mrpl}
 
\Purl{https://mrpl.city/blogs/view/oleksandr-chernov-hranitel-hudozhnih-tsinnostej-mariupolya}
\ifcmt
 author_begin
   author_id demidko_olga.mariupol,news.ua.mrpl_city
 author_end
\fi

\ii{12_02_2019.stz.news.ua.mrpl_city.1.oleksandr_chernov_hranytel_hud_cinnostej_mrpl.pic.1}

Серію нарисів, присвячених унікальним сучасникам Маріуполя, пропоную продовжити
розповіддю про відомого маріупольського колекціонера-дослідника,
мистецтвознавця, краєзнавця, члена Національної спілки краєзнавців \textbf{\emph{Олександра
Михайловича Чернова}}.

Наш герой народився у Мангуші, закінчив Приазовський державний технічний
університет, працював у науково-дослідних і проектних інститутах Одеси та
Маріуполя. Незважаючи на те, що за освітою Олександр Михайлович – інженер, його
призначення завжди було більш творчим та художнім. Він пройшов усі стадії
колекціонування: починав з марок, етикеток, монет, значків, потім захопився
іконами. Вірусом збирання його заразили батьки. Коли він був у другому класі,
вони привезли йому з Донецька марки, після чого маленький Олександр став
філателістом. Пізніше він захопився книгами – у студентські роки почав збирати
бібліотеку з мистецтва.

\textbf{Читайте також:} \emph{Маріуполець Євген Сосновський: таке красиве життя}%
\footnote{Маріуполець Євген Сосновський: таке красиве життя, Ольга Демідко, mrpl.city, 19.01.2019, \url{https://mrpl.city/blogs/view/mariupolets-evgen-sosnovskij-take-krasive-zhittya}} %
\footnote{Internet Archive: \url{https://archive.org/details/19_01_2019.olga_demidko.mrpl_city.sosnovsky_take_krasyve_zhyttja}}

\ii{12_02_2019.stz.news.ua.mrpl_city.1.oleksandr_chernov_hranytel_hud_cinnostej_mrpl.pic.2}

З кінця 1970-х років Олександр починає збирати фарфор, живопис і графіку
радянського та пострадянського часу. Колекціонер згадує, що все починалося з
Одеси, де він на барахолці у 1975 р. придбав мало не Айвазовського. Там навіть
рама характерна для робіт цього визначного художника. Цікаво, що у Маріуполі
(та й не тільки) Олександр Михайлович почав колекціонувати фарфор одним з
перших, навіть організував невелику ретроспективну виставку – російський і
радянський фарфор (починаючи з середини XIX століття). До колекціонування у
нього підхід ґрунтовний. І живопис, і графіка, і фарфор – все систематизовано
та каталогізовано. 

\ii{12_02_2019.stz.news.ua.mrpl_city.1.oleksandr_chernov_hranytel_hud_cinnostej_mrpl.pic.3}

Колекціонер наголошує, що це дуже копітка робота. Вважає, що
є \emph{\enquote{люди, які народжуються з геном колекціонера}}. Сьогодні колекція Олександра
Михайловича нараховує більше 100 художників (при цьому в одного автора може
бути як декілька робіт, так і більше десяти). Кожна робота для колекціонера
дуже цінна і має свою історію, адже часто Олександр знаходив роботу невідомого
автора, але завдяки наполегливим пошукам зміг відкрити світу мистецтва таких
художників, як А. В. Стирн, С. А. Татарников, І. Д. Бертяков, І. Я. Бондарєв, І. Є.
Власенко та ін. Головними труднощами для себе колекціонер вважає обмежені
можливості експонувати та презентувати всі роботи.

Більше ніж десять років Олександр Чернов зі своїми однодумцями виношував мрію
про створення приватної галереї. У 2011 році, нарешті, цю мрію вдалося втілити
в життя. Спочатку, коли тільки народилася ідея про її створення, Олександр
знайшов меценатів, заразив їх цією ідеєю, і вони вклали гроші в реставрацію
приміщення. Вони викупили приміщення в старому будинку, побудованому в
сталінські часи. Колись в цьому будинку був головпоштамт Маріуполя. Картинна
галерея стала справжньою прикрасою міста. Олександр Михайлович наповнив галерею
своїми колекціями та колекціями меценатів. Це була важлива і значуща подія для
маріупольців. Двісті квадратних метрів, два поверхи, три виставкові зали.
Однак, на жаль, свою роль зіграв людський фактор. Незважаючи на високий рівень
галереї, засновники вирішили проект згорнути. Хоча маріупольці й гості міста
досі телефонують Олександру Михайловичу і питають, коли можна прийти до Центру
мистецтв \enquote{АртЛюкс} (ARTLUX), адже сайт галереї дотепер працює...

\textbf{Читайте також:} \emph{В Мариуполе работает уникальный музей о меткомбинате (ЭКСКУРСИЯ 360˚)}%
\footnote{В Мариуполе работает уникальный музей о меткомбинате (ЭКСКУРСИЯ 360˚), Роман Катріч, mrpl.city, 11.02.2019, \par%
\url{https://mrpl.city/news/view/v-mariupole-rabotaet-unikalnyj-muzej-o-metkombinate-e-kskursiya-360}%
}

Творчий багаж нашого земляка досить багатий, він автор численних публікацій,
монографій та альбомів. Олександр Чернов взяв участь у створенні 19
телевізійних фільмів про митців на маріупольському телеканалі \enquote{Сигма} в рубриці
\enquote{Маріуполь. Минуле}. Водночас галеристу вдалося заповнити \enquote{білі плями} в
біографії всесвітньо відомого скульптора епохи арт-деко Жоржа Лаврова,
висвітливши його період життя і творчості в Маріуполі. Надихають Олександра
Михайловича зустрічі з цікавими та яскравими особистостями, адже після таких
зустрічей завжди з'являються нові ідеї. Улюбленим місцем у Маріуполі завжди був
Міський сад.

\ii{12_02_2019.stz.news.ua.mrpl_city.1.oleksandr_chernov_hranytel_hud_cinnostej_mrpl.pic.4}

Якщо ви хочете ближче познайомитися з Олександром Михайловичем, то це з
легкістю можна зробити завдяки його книзі \enquote{Роман з антикваріатом. Записки
збирача}, адже в ній автор ділиться своїми знахідками й відкриттями в галузі
колекціонування.

\ii{12_02_2019.stz.news.ua.mrpl_city.1.oleksandr_chernov_hranytel_hud_cinnostej_mrpl.pic.5}

Сьогодні Олександр Михайлович продовжує збирати та зберігати художні цінності
Маріуполя, підтримувати маріупольських митців та не перестає думати про
відкриття Музею приватних колекцій, який у півмільйонному місті є справжньою
необхідністю. Сподівається на підтримку й активну участь всіх маріупольців,
небайдужих до художнього мистецтва. Після спілкування з Олександром
Михайловичем особисто мені стало спокійніше за культурну спадщину рідного
міста, адже завдяки таким ентузіастам, відданим своїй справі, невідомим авторам
повертається імена та визнання, а втрачені пам'ятки культури отримують друге
життя...

\textbf{Читайте також:} \emph{Парки Мариуполя украсят скульптуры, придуманные детьми}%
\footnote{Парки Мариуполя украсят скульптуры, придуманные детьми, Яна Іванова, mrpl.city, 12.02.2019, \par%
\url{https://mrpl.city/news/view/parki-mariupolya-ukrasyat-skulptury-pridumannye-detmi-foto}
}

\underline{\textbf{Опубліковані праці:}} монографія \enquote{Леонід Гади. Життя і творчість},

альбоми \enquote{Валентин Константинов}, \enquote{Віктор Кофанов}, \enquote{Нюанси Валентина Малецького},
публікації в краєзнавчих збірниках, журнали \enquote{Серед колекціонерів}
(Москва), \enquote{Парус} (Маріуполь), різна періодика.

\ii{12_02_2019.stz.news.ua.mrpl_city.1.oleksandr_chernov_hranytel_hud_cinnostej_mrpl.pic.6}

Книга \enquote{Роман з антикваріатом. Записки збирача} (2016 р.).

\underline{\textbf{Улюблені книги Олександра Чернова:}} 

\enquote{Чорні дошки} та \enquote{Листи з російського музею} В. Солоухіна.

\underline{\textbf{Улюблені фільми:}} радянські кінокомедії Л. Гайдая та Е. Рязанова.

\underline{\textbf{Курйозний випадок з життя:}} Не всі знають, але довгоочікуване відкриття
художнього музею ім. А. І. Куїнджі пов'язано саме з рішучістю Олександра
Михайловича. Наш герой втомився від бездіяльності тих, від кого це залежало, що
написав статтю \enquote{Картинна галерея як дзеркало маріупольської влади}.
Звернувся в місцеву газету \enquote{Вечірній Маріуполь} з проханням
опублікувати. Редактор прочитала, сказала, що готова надрукувати, але порадила
зробити цю статтю колективною. Олександр Михайлович підключив знакових людей –
художників, архітекторів, почесних громадян Маріуполя, - і вони із задоволенням
підписалися. Тогочасного президента В. А. Ющенка Олександр Чернов знав як
постійного відвідувача зборищ колекціонерів в Києві. Коли вийшла газета з
колективним листом, він особисто відвіз її до Києва і вручив примірник
президентові, який пообіцяв ознайомитися. Коли Олександр Михайлович приїхав
додому, протягом тижня була тиша, проте незабаром почався пошук автора статті.
Автора так і не знайшли, адже прізвище Олександра Михайловича було після всіх
десяти інших прізвищ небайдужих маріупольців. У результаті через кілька місяців
в місті відкрився художній музей імені А. І. Куїнджі.

\ii{12_02_2019.stz.news.ua.mrpl_city.1.oleksandr_chernov_hranytel_hud_cinnostej_mrpl.pic.7}

\underline{\textbf{Порада колекціонерам-початківцям:}} Перш ніж почати колекціонувати,
необхідно володіти потрібними знаннями в цій галузі, слід придбати книги,
каталоги, виховувати й тренувати смак. Постійно відкривати щось нове. Не
піддаватися данині моди.

\textbf{Читайте також:} \emph{Фотомистецтво Маріуполя: екскурс у минуле}%
\footnote{Фотомистецтво Маріуполя: екскурс у минуле, Ольга Демідко, mrpl.city, 05.02.2019, \url{https://mrpl.city/blogs/view/fotomistetstvo-mariupolya-ekskurs-u-minule} } %
\footnote{Internet Archive: \url{https://archive.org/details/05_02_2019.olga_demidko.mrpl_city.fotomystectvo_mrpl_ekskurs_v_mynule}}
