% vim: keymap=russian-jcukenwin
%%beginhead 
 
%%file slova.karta
%%parent slova
 
%%url 
 
%%author 
%%author_id 
%%author_url 
 
%%tags 
%%title 
 
%%endhead 
\chapter{Карта}
\label{sec:slova.karta}

%%%cit
%%%cit_head
%%%cit_pic
\ifcmt
  pic https://avatars.mds.yandex.net/get-zen_doc/4888095/pub_60bdf74e7b0ba72b6fd0d2cb_60be012bbcbf42494e3b1c20/scale_1200
\fi
%%%cit_text
Хочу добавить к перечисленному \emph{карту} Африки. Хорошо очерченное
побережье, с указанием городов.  В Сахаре достаточно много небольших озёр и
рек. Обозначен пролив Гибралтар.  То есть \emph{карта} написана не из фантазий
автора, а по, достаточно точным, измерениям.  В Африке в это время идёт борьба
с работорговлей против европейских колонизаторов. Как и во всём мире, в Африке
16 века сформировано несколько государств. Развита металлургия, животноводство
и земледелие.  Но с приходом европейских колонизаторов эти государства пришли в
упадок
%%%cit_comment
%%%cit_title
\citTitle{Карты 1570 года. Тартария, Америка и материк на Северном Полюсе}, 
Нешкольная История, zen.yandex.ru, 08.06.2021
%%%endcit

%%%cit
%%%cit_head
%%%cit_pic
\ifcmt
  tab_begin cols=3
     pic https://avatars.mds.yandex.net/get-zen_doc/1880126/pub_60f29964cf9db26dda7a8641_60f2a1f54b5d8b6f29283db2/scale_1200
     pic https://avatars.mds.yandex.net/get-zen_doc/3414453/pub_60f29964cf9db26dda7a8641_60f2a2856de09944845431f5/scale_1200
		 pic https://avatars.mds.yandex.net/get-zen_doc/916951/pub_60f29964cf9db26dda7a8641_60f2a42b4b5d8b6f292d6bc5/scale_1200
  tab_end
\fi
%%%cit_text
И это, заметьте, \emph{карта} 1375 года!  А вот \emph{карта} Николая Кузанского
1491 года, которую можно найти в Национальной библиотеке Франции. И тут тоже
Россия. Причем к России тут относят и милый сердцу каждого уважающего себя
украинского националиста Львов. А Киев обозначен как Chiovea - Киовия,
соответственно, жить там должны "киовиты".  А вот на Херефордской \emph{карте}
(изготовлена до 1300 года) из Херефордского собора, Британия можно обнаружить
одновременно и Россию и славян и медведя (ну само собой, медведь же наше все).
Ориентация \emph{карты} - восток сверху, в центре Иерусалим. Это самая большая
существующая средневековая \emph{карта}, она нарисована на пергаменте из
цельной телячей кожи 158 на 133 см, её автор - Ричард из Хардингема (есть
подпись в левом нижнем углу).  И таких \emph{карт} можно еще много найти. То
есть "московия" появляется на \emph{картах} именно во время и благодаря
антироссийской литовско-польской пропаганде, что в принципе довольно наглядно
видно по тому, когда вообще впервые появились \emph{карты} с термином
"московия". И любой, кто прежде всего человек увлекающийся историей, а не
нацист, это прекрасно знает
%%%cit_comment
%%%cit_title
\citTitle{300-летию \enquote{московии} посвящается}, Илья Duke, zen.yandex.ru, 19.07.2021
%%%endcit

%%%cit
%%%cit_head
%%%cit_pic
%%%cit_text
А вот в Киевраде, в отличие от Верховной, есть не только
"национально-сознательное" большинство, но и, похоже, пронацистское. Только
этим можно объяснить тот факт, что с \emph{карты} столицы одну за другой
убирают улицы, которые не подпадают ни под закон о декоммунизации, ни даже под
какие-либо антироссийские тенденции.  К примеру, со вчерашнего дня больше не
существует улицы Народного ополчения. В чем провинились ополченцы, защищавшие
Киев от немецких захватчиков, понять сложно.  То же касается улицы
Молодогвардейцев – переименовать ее могли только те, кто считает деятельность
юных подпольщиков Краснодона диверсией против вермахта.  Ну, а улицу
легендарного советского разведчика Николая Кузнецова и вовсе переименовали в
честь писателя-коллаборациониста Олеся Бабия, сбежавшего с нацистами в 1944-м.
И это еще не все. Уже известно, что на очереди у Киеврады находятся улицы
Героев Сталинграда и Маршала Тимошенко, начались разговоры и в отношении улицы
Героев Днепра и площади Героев Севастополя. Судя по тенденциям, доберутся и до
них. Вопрос только в том, какую из этих улиц назовут именем гауляйтера Украины
Эриха Коха
%%%cit_comment
%%%cit_title
\citTitle{СБУ на службе у "активиста", Рада "разгромила" ЗСТ с ЕС, Киеврада уничтожила Молодогвардейцев. Итоги "Страны"}, 
, strana.news, 05.11.2021
%%%endcit
