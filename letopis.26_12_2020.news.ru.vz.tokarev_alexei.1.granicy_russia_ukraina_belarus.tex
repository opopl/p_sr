% vim: keymap=russian-jcukenwin
%%beginhead 
 
%%file 26_12_2020.news.ru.vz.tokarev_alexei.1.granicy_russia_ukraina_belarus
%%parent 26_12_2020
 
%%url https://vz.ru/opinions/2020/12/26/1077561.html
 
%%author 
%%author_id tokarev_alexei
%%author_url 
 
%%tags russia,ukraina,belarus
%%title Между Россией, Белоруссией и Украиной нет границ
 
%%endhead 
 
\subsection{Между Россией, Белоруссией и Украиной нет границ}
\label{sec:26_12_2020.news.ru.vz.tokarev_alexei.1.granicy_russia_ukraina_belarus}
\Purl{https://vz.ru/opinions/2020/12/26/1077561.html}
\ifcmt
  author_begin
   author_id tokarev_alexei
  author_end
\fi

Алексей Токарев, к.п.н., старший научный сотрудник Института международных исследований МГИМО МИД России

Каждый год в декабре крупнейшие поисковые системы подводят итоги, сравнивая
количество запросов по конкретным событиям, явлениям и словам, которые
поступили из разных стран. Я думаю, это отличный инструмент для поисков степени
родства «братьев-славян». Что ищут жители России, Украины и Белоруссии?

Интернет знает про нас намного больше, чем друзья, психологи и священник на
исповеди. Пожалуй, поисковая строка – единственное пространство в мире, где мы
не врем сами себе, спрашивая о том, что нас интересует действительно. Бывший
дата-аналитик Google Сет Стивен-Давидовиц вообще предлагает верифицировать
посредством анализа больших данных отдельные элементы фрейдизма, ведь иллюзия
анонимности и отсутствия общественного взгляда позволяет нам быть откровенными
с поисковой системой.

Частота запросов, которые мы ежедневно отправляем куда-то в глубину «паутины»,
помогает судить о тех границах, которых нет на картах. «Украина – не Россия»,
«Белорусский федеральный округ РФ», «братство восточных славян», «комплексы
младших братьев», «недоимперия» и прочие политические, исторические и
культурные стереотипы, прочно вошедшие в дискурсы всех социальных групп, от
националистов до глобалистов, отлично измеряются при помощи анализа данных. 

Долгие годы жители России, Украины и Белоруссии ищут одинаковые события,
явления и персоны. В 2017 году украинцы интересовались Дианой Шурыгиной,
Михаилом Задорновым и Дмитрием Марьяновым. Первая по количеству запросов
украинка Ирина Бережная была лишь на четвертом месте. «Почему… Малахов ушел с
Первого канала?» – третий по популярности запрос-2017 в стране, декларирующей
полный разрыв с Россией. За рамками интереса к личностям запросом № 1 на
Украине был «сериал Физрук 4 сезон», на втором месте – украинское шоу
«Холостяк». В 2019 году все три страны интересовались Юлией Началовой, Децлом,
Аллой Вербер – виртуальная жизнь похожа на реальную, в которой смерть
объединяет давно живущих порознь родственников за общим столом.

Главное явление в мире 2020 года было одинаковым для всех. Интернет восточных
славян не стал исключением: «коронавирус» победил остальные запросы в России,
на Украине и в Белоруссии. Вице-чемпион этой гонки за внимание пользователей
общий у мира, России и Белоруссии, а украинцы «выборы в США» гуглили меньше,
чем «коронавирус рекомендации», которые получили второе место в национальной
доменной зоне. «Симптомами коронавируса» пандемия отметилась на третьей строчке
в Белоруссии и пятой на Украине. В 2019 году украинские выборы «объединяли»
наши интернет-пространства хуже американских – соревнование Зеленского с
Порошенко в России было на первом месте, на Украине – на пятом, а в Белоруссии
вообще не вошло в топ-10. 

\ifcmt
pic https://img.vz.ru/upimg/m10/m1077561.jpg
\fi

Сама Белоруссия в 2020-м, конечно, привлекла к себе внимание
«восточнославянского» Интернета. В российском сегменте Google «события в
Белоруссии» на пятом месте, «новости Минск» в самой стране – на четвертом, на
Украине – за пределами десятки. Чаще всего запрос «Belarus» вбивали в
Прибалтике и Восточной Африке: на пике протестов в августе Литва, Уганда,
Замбия, Кения, Нигерия, Эстония и Гана интересовались темой в англоязычном
сегменте намного сильнее, чем Армения и Россия.

«Борис Клюев» и «Коби Браянт» – самые частые запросы рейтинга утрат в России и
на Украине (умерший американский баскетболист – на третьем месте в мире по
количеству запросов). Россияне и украинцы одинаково искали информацию о смерти
Ирины Антоновой, Михаила Жванецкого, Армена Джигарханяна, Валентины
Легкоступовой, Чедвика Боузмана, Диего Марадоны, Джорджа Флойда, Шона Коннери –
девять из десяти позиций в поиске общие. 

«Фильм года» в России и Белоруссии – «Холоп». У Белоруссии и Украины общие
поиски покупок. «Антисептик», «маска медицинская», «респиратор»,
«пульсоксиметр» в обеих странах входят в топ-5 запросов, который отличается
одной строкой: белорусы искали «парацетамол», украинцы «IPhone 12». Футболист
Артем Дзюба вошел в топ-10 в обеих странах, оттеснив в Белоруссии оппозиционера
Светлану Тихоновскую, а на Украине – умершего мэра Харькова Геннадия Кернеса. В
России нападающий в десятку не попал, а белорусский президент и американский
вице-президент обогнали российского премьера (Александр Лукашенко – 7-е место,
Джо Байден – 9-е, Михаил Мишустин – 10-е). В 2019 году главным политиком,
которого россияне искали в Сети, был Владимир Зеленский (Google отдавал ему
четвертую позицию в рейтинге персон, «Яндекс» – первую).

Происходит ли онлайн-суверенизация России, Украины и Белоруссии в интернете с
течением времени? Для точного ответа нужен доступ к большим данным. Но если
судить по той аналитике, которая доступна публично, интернет продолжает стирать
границы. Сеть можно уподобить вечному двигателю: с одной стороны, она является
порождением когда-то общего культурного пространства империи. С другой – она же
продолжает «склеивать» массовые сознания разных стран, отменяя национальные
суверенитеты в онлайне. Неважно, какой у пользователя паспорт или место
жительства. Неважно даже, как пользователь относится к своему и соседним
правительствам. Россияне, украинцы и белорусы одинаково интересуются смертями
советских актеров, способами сделать в домашних условиях марлевые маски,
роликами спортивных звезд и выборами в США. 

Конечно, в деталях мы ищем разное. Россия в этом смысле была сосредоточена в
2020 году в большей степени на собственной повестке (в топ-10 событий вошли бои
Хабиба Нурмагомедова, обращения к народу Владимира Путина, игры футбольной
сборной, поправки в Конституцию). Но в отношении больших трендов мы похожи с
украинцами и белорусами.

Мы сообща голосуем за выбор сериалов и кино, наконец, будто сговорившись, мы
отдаем одинаковые строчки рейтинга одним и тем же героям общественного
внимания. И пока в телевизионных шоу побеждают границы и классическое
разделение политики на своих и чужих, интернет доказывает: общее информационное
пространство россиян, украинцев и белорусов сохраняется. 
