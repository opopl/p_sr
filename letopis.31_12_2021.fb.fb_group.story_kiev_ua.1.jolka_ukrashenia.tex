% vim: keymap=russian-jcukenwin
%%beginhead 
 
%%file 31_12_2021.fb.fb_group.story_kiev_ua.1.jolka_ukrashenia
%%parent 31_12_2021
 
%%url https://www.facebook.com/groups/story.kiev.ua/posts/1830331283830331
 
%%author_id fb_group.story_kiev_ua,chekalkina_viktoria.kiev
%%date 
 
%%tags igrushka.jolka,jolka,kiev,novyj_god
%%title Эти чудесные фигуры из моей коллекции - елочные украшения
 
%%endhead 
 
\subsection{Эти чудесные фигуры из моей коллекции - елочные украшения}
\label{sec:31_12_2021.fb.fb_group.story_kiev_ua.1.jolka_ukrashenia}
 
\Purl{https://www.facebook.com/groups/story.kiev.ua/posts/1830331283830331}
\ifcmt
 author_begin
   author_id fb_group.story_kiev_ua,chekalkina_viktoria.kiev
 author_end
\fi

С 1936 года, после того как елку вернули как новогоднюю, а не рождественскую,
местную промышленность развернули на производство елочных игрушек. Вспомнили
старые технологии и материалы для производства елочных украшений. 

\raggedcolumns
\begin{multicols}{3} % {
\setlength{\parindent}{0pt}

\ii{31_12_2021.fb.fb_group.story_kiev_ua.1.jolka_ukrashenia.pic.1}
\ii{31_12_2021.fb.fb_group.story_kiev_ua.1.jolka_ukrashenia.pic.1.cmt}

\ii{31_12_2021.fb.fb_group.story_kiev_ua.1.jolka_ukrashenia.pic.2}
\ii{31_12_2021.fb.fb_group.story_kiev_ua.1.jolka_ukrashenia.pic.2.cmt}

\end{multicols} % }

\ii{31_12_2021.fb.fb_group.story_kiev_ua.1.jolka_ukrashenia.pic.2.cmt.goncharova}

\raggedcolumns
\begin{multicols}{3} % {
\setlength{\parindent}{0pt}

\ii{31_12_2021.fb.fb_group.story_kiev_ua.1.jolka_ukrashenia.pic.2.cmt.continue}

\end{multicols} % }

Киевская артель «Игрушка» стала производить маленькие ёлочки по старой
дореволюционной технологии из гусиных перьев. Киевские артели «Радянский
Китай», «Щастливе дитинство», «Дитячий світ» с 1936 года и артели «9 травня»,
«Побєда» с 1945 года производят чудесные игрушки и подставочные фигуры под елку
из Ваты и Папье Маше.  

% figs 2,3,4
\ii{31_12_2021.fb.fb_group.story_kiev_ua.1.jolka_ukrashenia.pic.3}

Прошу всех желающих делиться фотографиями своих семейных ватных игрушек,
дедов морозов, снегурочек, изготовленных до 1965 года. Так как после 1966
артельный ручной труд был заменён на штампованный пластик. Очарование, тепло
человеческих рук художников было утрачено.  Эти чудесные фигуры из моей
коллекции. Всех с наступающим Новым годом! Крепкого здоровья и удачи!

\ii{31_12_2021.fb.fb_group.story_kiev_ua.1.jolka_ukrashenia.cmt}
