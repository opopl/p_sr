% vim: keymap=russian-jcukenwin
%%beginhead 
 
%%file 22_10_2022.stz.news.ua.donbas24.1.okupanty_zbyrajutsja_znyschyty_mural_milana
%%parent 22_10_2022
 
%%url https://donbas24.news/news/okupanti-zbirayutsya-znishhiti-mariupolskii-simvol-nadiyi-na-mir-mural-milana
 
%%author_id demidko_olga.mariupol,news.ua.donbas24
%%date 
 
%%tags 
%%title Окупанти збираються знищити маріупольський символ надії на мир — мурал "Мілана"
 
%%endhead 
 
\subsection{Окупанти збираються знищити маріупольський символ надії на мир — мурал \enquote{Мілана}}
\label{sec:22_10_2022.stz.news.ua.donbas24.1.okupanty_zbyrajutsja_znyschyty_mural_milana}
 
\Purl{https://donbas24.news/news/okupanti-zbirayutsya-znishhiti-mariupolskii-simvol-nadiyi-na-mir-mural-milana}
\ifcmt
 author_begin
   author_id demidko_olga.mariupol,news.ua.donbas24
 author_end
\fi

\ii{22_10_2022.stz.news.ua.donbas24.1.okupanty_zbyrajutsja_znyschyty_mural_milana.pic.front}
\begin{center}
  \em\bfseries\Large
Окупанти продовжують знищувати пам'ять про власні злочини
\end{center}

В окупованому Маріуполі \href{https://donbas24.news/news/okupanti-planuyut-vivoziti-muzeyi-z-krimu-a-zsu-znishhili-rosiisku-texniku-v-antraciti-genstab}{\emph{росіяни планують}}%
\footnote{Окупанти планують вивозити музеї з Криму, а ЗСУ знищили російську техніку в Антрациті - Генштаб, Еліна Прокопчук, donbas24.news, 14.10.2022, \par\url{https://donbas24.news/news/okupanti-planuyut-vivoziti-muzeyi-z-krimu-a-zsu-znishhili-rosiisku-texniku-v-antraciti-genstab}} демонтувати мурал \enquote{Мілана}, який став
символом трагедії війни та надії на мир і відновлення. Цей стінопис присвячений
маріупольській дівчинці, яка постраждала від обстрілів проросійських бойовиків
мікрорайону \enquote{Східний} у 2015 році. Офіційним приводом окупанти назвали
\enquote{відновлення теплового контуру}. Про це \href{https://t.me/mariupolrada/11434}{повідомила}%
\footnote{\url{https://t.me/mariupolrada/11434}} Маріупольська рада.

\begin{leftbar}
\emph{\enquote{Насправді — намагання знищити пам'ять про власні злочини та обстріл нашого
міста ще у 2015 році. Адже мурал став символом трагедії війни та надії на мир й
відновлення}}, — зауважили у міськраді.
\end{leftbar}

\ii{insert.read_also.demidko.donbas24.spilka_arheologiv_ukr_zasudyla_pretenzii_rf}

\ifcmt
  ig https://i2.paste.pics/PSR6G.png?trs=1142e84a8812893e619f828af22a1d084584f26ffb97dd2bb11c85495ee994c5
  @wrap center
  @width 0.7
\fi

\ii{22_10_2022.stz.news.ua.donbas24.1.okupanty_zbyrajutsja_znyschyty_mural_milana.pic.1}

\subsubsection{Історія створення муралу}

7 років тому, 24 січня 2015 року проросійські бойовики обстріляли \enquote{Градами} та
\enquote{Ураганами} мікрорайон \enquote{Східний} у Маріуполі. Того трагічного дня 31 людина
загинула, понад сотня отримали поранення. Прокуратура класифікувала цей обстріл
як теракт, тодішній президент України Петро Порошенко оголосив 25 січня Днем
жалоби.

Серед постраждалих у Маріуполі були діти, зокрема Мілана Абдурашитова. Для
дівчинки це був звичайний день. Вона зранку йшла з мамою на роботу, оскільки
залишити її не було з ким. Тоді, близько 9:30, по житловому мікрорайону
\enquote{Східний} відкрили вогонь.

Поранена осколками снарядів, Мілана залишається живою завдяки героїзму матері:
жінка в останній момент кинулася до дочки і захистила своїм тілом дитину. Ольга
Абдурашитов загинула, а Мілана отримала тяжкі поранення. Дитині довелося
ампутувати частину лівої ноги. Вона перенесла кілька операцій, отримала протез
та психологічну допомогу. Всі витрати на лікування Мілани взяв на себе Фонд
Ріната Ахметова.

\textbf{Читайте також:} \href{https://donbas24.news/news/pro-vtraceni-mozayiki-mariupolya-znyali-film-zevrinnya-derevo-zittya-unikalni-kadrivideo}{\emph{Про втрачені мозаїки Маріуполя зняли фільм \enquote{Жевріння. Дерево життя}: унікальні кадри}}%
\footnote{Про втрачені мозаїки Маріуполя зняли фільм \enquote{Жевріння. Дерево життя}: унікальні кадри, Яна Іванова, donbas24.news, 29.09.2022, \par%
\url{https://donbas24.news/news/pro-vtraceni-mozayiki-mariupolya-znyali-film-zevrinnya-derevo-zittya-unikalni-kadrivideo}%
}

\ii{22_10_2022.stz.news.ua.donbas24.1.okupanty_zbyrajutsja_znyschyty_mural_milana.pic.2}

Мурал був зображений у 2018 році на пятнадцятиповерхівці, яка розташована на
проспекті Миру, 101. Автором став київський стріт-арт художник \href{https://www.facebook.com/Sasha.Korban2}{Олександр
Корбан}.%
\footnote{\url{https://www.facebook.com/Sasha.Korban2}}

\begin{quote}
\emph{\enquote{Мілана — сама чарівність. Я перейнявся усією душею до дитини. А коли
дізнався про її історію, спробував ці емоції вкласти в малюнок. Щоб глядач міг
відчути тепло дитини та всіх постраждалих дітей. Спочатку деякі маріупольці
скептично поставилися до муралу, але потім зізналися: дивлячись на нього їм
хочеться жити}}, — \href{https://mrpl.city/news/view/devochka-s-mariupolskogo-murala-novaya-zhizn-posle-tragedii-na-vostochnom-foto-plusvideo}{розповів}%
\footnote{Девочка с мариупольского мурала: новая жизнь после трагедии на \enquote{Восточном}, Кіра Булгакова, mrpl.city, 23.01.2019, \par\url{https://mrpl.city/news/view/devochka-s-mariupolskogo-murala-novaya-zhizn-posle-tragedii-na-vostochnom-foto-plusvideo}}
автор муралу Олександр Корбан.
\end{quote}

\ii{22_10_2022.stz.news.ua.donbas24.1.okupanty_zbyrajutsja_znyschyty_mural_milana.pic.3}

\textbf{Читайте також:} \href{https://donbas24.news/news/rozberi-rusnyu-na-suveniri-pereselenec-z-mariupolya-vigotovlyaje-unikalni-breloki}{\emph{\enquote{Розбери русню на сувеніри}: переселенець з Маріуполя виготовляє унікальні брелоки}}%
\footnote{\enquote{Розбери русню на сувеніри}: переселенець з Маріуполя виготовляє унікальні брелоки, Алевтина Швецова, donbas24.news, 29.09.2022, \par%
\url{https://donbas24.news/news/rozberi-rusnyu-na-suveniri-pereselenec-z-mariupolya-vigotovlyaje-unikalni-breloki}%
}

\subsubsection{Реакція маріупольців}

Рішення окупантів знищити мурал обурило маріупольців, які втомилися
спостерігати за знищенням власного міста. Адже через численні зруйнування в
Маріуполі цінних архітектурних та культурних пам'яток, місто безповоротно
втрачає своє історичне обличчя.

\begin{leftbar}
\emph{\enquote{Знищено обличчя міста, його шарм, його самобутність. Тепер все змусять
новими коробками. І буде просто безлике місто, яких на карті багато. Місто
померло, вмирає море. Навряд чи сюди хтось захоче приїхати. Ну хіба що ми, чиє
коріння тут так і залишилося. Дочекатися б перемоги...}}, — наголосила
маріупольчанка Вікторія Аверченко. 
\end{leftbar}

Український громадський діяч, журналіст, голова громадської організації
\emph{\enquote{Архі-Місто}} Андрій Марусов підкреслив, що у Маріуполі повним ходом йде
\enquote{зачистка історії} — від давньої до сучасної. У місті вже відкрилося відділення
Російського історичного товариства та зносяться пам'ятники...

\begin{leftbar}
\emph{\enquote{Але мені все одно незрозуміло — ну бідолашна Мілана чим вам не догодила?!?
дайте їй спокій... але — ні, її теж \enquote{зачищають...}...}}, — зазначив Андрій Марусов.  
\end{leftbar}

\ii{insert.read_also.demidko.donbas24.kamjani_polovecki_baby_oberigajut_vid_bombarduvanj}
\ii{22_10_2022.stz.news.ua.donbas24.1.okupanty_zbyrajutsja_znyschyty_mural_milana.pic.4}

Нагадаємо, що нещодавно окупанти знищили пам'ятник Жертвам голодомору і
політичних репресій радянського тоталітарного режиму в тимчасово окупованому
Маріуполі, як такий, що не є релевантним. Окупанти заявили, що граніт
перероблять на будматеріали. Це свідчить лише про правонаступництво сучасного
російського режиму попереднім в його злочинах проти людяності та українського
народу. Міністерство культури та інформаційної політики України засудило
руйнування пам'ятника. Загалом знищення і вивіз цінних пам'яток культури та
історії є злочином, скоєним російськими військовими, які спрямовані проти
культурної спадщини Маріуполя та загалом України.

Раніше Донбас24 розповідав про злочини росіян \href{https://archive.org/details/11_08_2022.olga_demidko.donbas24.zlochyny_rosian_proty_kulturn_spadchyny_donbasu}{\emph{проти культурної спадщини Донбасу}}.%
\footnote{Злочини росіян проти культурної спадщини Донбасу, Ольга Демідко, donbas24.news, 11.08.2022, \par%
\url{https://donbas24.news/news/zlocini-rosiyan-proti-kulturnoyi-spadshhini-donbasu}, \par%
Internet Archive: \url{https://archive.org/details/11_08_2022.olga_demidko.donbas24.zlochyny_rosian_proty_kulturn_spadchyny_donbasu}%
}

Ще більше новин та найактуальніша інформація про Донецьку та Луганську області
в нашому телеграм-каналі Донбас24.

ФОТО: з відкритих джерел.

\ii{insert.author.demidko_olga}
%\ii{22_10_2022.stz.news.ua.donbas24.1.okupanty_zbyrajutsja_znyschyty_mural_milana.txt}
