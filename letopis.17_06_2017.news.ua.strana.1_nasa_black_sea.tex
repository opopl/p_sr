% vim: keymap=russian-jcukenwin
%%beginhead 
 
%%file 17_06_2017.news.ua.strana.1_nasa_black_sea
%%parent 17_06_2017
%%url https://strana.ua/news/76729-v-nasa-pokazali-sputnikovye-foto-so-strannym-cvetom-chernogo-morya.html
%%tags nasa,black sea,strana.ua
 
%%endhead 

\subsection{В НАСА показали спутниковые фото со странным цветом Черного моря}

\url{https://strana.ua/news/76729-v-nasa-pokazali-sputnikovye-foto-so-strannym-cvetom-chernogo-morya.html}

16:08, 17 июня 2017

Ученые NASA обнародовали космические снимки, сделанные спутником Aqua Satellite
с помощью спектрографа Moderate Resolution Imaging Spectroradiometer. Об этом
сообщает Daily Mail.

"Aqua Satellite NASA получил потрясающий образ Черного моря из космоса, на
котором видно как бирюзовые разводы доминируют, среди, как правило, темно-синей
воды", - говорится в сообщении.

\ifcmt
img_begin 
	tags nasa,black sea,strana.ua
	url https://strana.ua/pub/a/2/b/2bb81b241e0c42ebd9915931ee025348.jpg
	width 0.5
	caption Снимок Черного моря со спутника Фото: NASA
img_end
\fi

На снимках видно, как самое темное в мире море изменило свой цвет. В некоторых
местах вместо темно-синего отчетливо прослеживается бирюзовый - гуще всего
такие светлые пятна у берегов Румынии, Болгарии и Украины.

По мнению ученых, это связано с мощным морским течением, которое принесло в
Черное море огромное количество фитопланктона. Эти микроорганизмы и стали
причиной изменения цвета воды.

"Пролив Босфор в Стамбуле становится бирюзовым в летнее время. NASA сообщает,
что уникальный цвет стал преобладать из-за фитопланктона в Черном море", -
написано в сообщении TrtWorld в Twitter.

Как сообщала "Страна", китайцы собрались вырастить картошку на обратной стороне
Луны уже в следующем году.

Напомним, NASA из 18000 человек отобрало 12 будущих колонизаторов Луны и Марса.

Ранее мы писали, что в NASA раскрыли детали космического полета к Солнцу в 2018
году.
