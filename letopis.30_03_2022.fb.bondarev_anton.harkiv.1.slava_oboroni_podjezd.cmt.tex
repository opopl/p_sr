% vim: keymap=russian-jcukenwin
%%beginhead 
 
%%file 30_03_2022.fb.bondarev_anton.harkiv.1.slava_oboroni_podjezd.cmt
%%parent 30_03_2022.fb.bondarev_anton.harkiv.1.slava_oboroni_podjezd
 
%%url 
 
%%author_id 
%%date 
 
%%tags 
%%title 
 
%%endhead 
\zzSecCmt

\begin{itemize} % {
\iusr{Anna Sivokon-Zelenova}

Я теж бейсбольну биту тримаю біля дверей. А ще каску залізну воєнну, образца
1953 року. Уявила себе з битою та в касці))

\iusr{Жигунова Оксана}
\textbf{Anna Sivokon-Zelenova} самый модный тренд!

\iusr{Олександр Костьо}
Тримайтесь, це головне

\iusr{Тарас Гавришко}
Було-б смішно, якби не було так сумно... Залишається тільки триматись!!!

\iusr{Svetlana Klimova}
@igg{fbicon.heart.red}

\begin{itemize} % {
\iusr{Svetlana Klimova}

Люблю ваші тексти. Взяла на свою сторінку. Сорі, без дозволу. Обіймаємо. Сумую
за нашим містом.

\iusr{Anton Bondarev}
\textbf{Svetlana Klimova} ну я просто не міг такі історії не зафіксувати)))

\iusr{Svetlana Klimova}
\textbf{Anton Bondarev} чекаємо на інщі хоч і сумні

\iusr{Anton Bondarev}
\textbf{Svetlana Klimova} дуже хочется щоб приводу писати сумні тексти не було
\end{itemize} % }

\iusr{Оксана Рудковська}

Життя пише нову історію про нас самих)

\iusr{Inna Romenska}
Це мені нагадує ті статті, що ти присилав для редактури про Першу Світову))

\iusr{Тетяна Шпеник}

Геніально! Розсмішили! Я живу на 3 поверсі, 5 поверхівки. За 18 років на 5-му
поверсі була 2 рази. Роки три тому, щось сталося з параболою на даху, так мого
чоловіка, який живе в будинку більше 40 років, ледве пустили на дах бдітельні
пінзіяшки! Мусили зняти ту параболу і пришпондьорити (трохи закарпатської) на
балкон.

\iusr{Vadim Komarov}
Волонтерки- кохання @igg{fbicon.beaming.face.smiling.eyes}  @igg{fbicon.biceps.flexed}  @igg{fbicon.thumb.up.yellow}{repeat=3} 

\iusr{Вася Кор}
\textbf{Vadim Komarov} Мабуть безплатно @igg{fbicon.beaming.face.smiling.eyes} 

\iusr{Yana Shershun}
Ключове питання: які все ж таки квіти вирішили садити на клюмбі?  @igg{fbicon.face.grinning.smiling.eyes} 

\begin{itemize} % {
\iusr{Anton Bondarev}
\textbf{Yana Shershun} кажись піони, або тюльпани @igg{fbicon.beaming.face.smiling.eyes}{repeat=3} 

\iusr{Yana Shershun}
\textbf{Anton Bondarev} ну вибір і справді нелегкий  @igg{fbicon.face.grinning.squinting} 
\end{itemize} % }

\iusr{Мирослав Меденцій}

Антон, радію Вашому гумору, значить справи йдуть на краще!

Як на мене, на клумбу краще саджати \enquote{Гвоздику} (2С1), бо у
\enquote{Піона} (2С7) калібр більший, при стрільбі шибки у будинку точно
повилітають, а у \enquote{Тюльпана} (2С4) - і поготів  @igg{fbicon.face.smiling.eyes.smiling} .

\iusr{Anton Bondarev}
Пост відреагував та відкрив доречі @igg{fbicon.beaming.face.smiling.eyes} 

\iusr{Arina Voina}
Все норм ))) всі бдять!

\iusr{Вікторія Ільченко}

В нас в будинку в Києві майже ніхто ні з ким не спілкувався до війни, а тепер
створили чат будинку і ми ще додатково чат парадного створили. В кожному
парадному ходять двічі на день оглядають, чи немає мвток від мародерів, якщо є,
то прибирають. І вже домовилися, що після перемоги гарно всі разом
відсвяткуємо.

А підвали теж облаштували і навіть інтернет туди на провайдер провів, за що
йому щіра подяка. @igg{fbicon.flag.ukraina}

\iusr{Вікторія Ільченко}

А чат села під Кривим рогом, доволі ватного села до війни, не зговорюючись
перейшов на українську, коли орки стали наближатися до кордонів області. А коли
в село стали ЗСУ, щоб зупинити наступ орків, то всі разом носили ЗСУ найкращу
їжу, зокрема купили й забили свиню, щоб годувати захисників свіжим і смачним.

@igg{fbicon.flag.ukraina}

\iusr{Елена Даньшина}
Ну все... Починаю стежити за вашими постами))) Доконали)))

\end{itemize} % }
