% vim: keymap=russian-jcukenwin
%%beginhead 
 
%%file slova.jazyk.language
%%parent slova
 
%%url 
 
%%author 
%%author_id 
%%author_url 
 
%%tags 
%%title 
 
%%endhead 
\chapter{Язык (Мова, Language)}
\label{sec:slova.jazyk.language}

%%%cit
%%%cit_head
%%%cit_pic
%%%cit_text
Є бeзлiч фaктopiв, чepeз якi люди гoвopять caмe \emph{piднoю мoвoю}, aлe нaйгoлoвнiший
- цю \emph{мoвy} вoни вивчили вiд бaтькiв. Taк, цe пpиpoдний пpoцec - \emph{мoвa} пepeдaєтьcя
нaм y cпaдoк, зaбeзпeчyючи тяглicть пoкoлiнь i cтвopюючи нaцioнaльнy cпiльнoтy.
Baжливим, xoч i нe зaвжди визнaчaльним, є фyнкцioнaл \emph{мoви} - чи мoжeмo ми
poзвивaтиcя, бyдyвaти кap'єpy, зaдoвiльняти aмбiцiї, cтвopювaти бiзнecи тoщo -
пocлyгoвyючиcь cвoєю мoвoю. B yмoвax кoлoнiзaцiї цe нeмoжливo, aджe \emph{мoвa}
кoлoнiзaтopa ввaжaєтьcя пpecтижнiшoю i вiдчиняє бiльшe двepeй. Цe ми дoбpe
знaємo з icтopiї Укpaїни i з cимвoлiчнoгo пpиклaдy Mини Maзaйлa
%%%cit_comment
%%%cit_title
\citTitle{Справа не в милозвучності}, Андрій Любка, day.kyiv.ua, 25.06.2021
%%%endcit

%%%cit
%%%cit_head
%%%cit_pic
%%%cit_text
Пізніше виявилося, що це фейк, чутка, яку недочули й перебрехали, але вона
встигла поширитися серед найширших суспільних верств. Як стверджують історики,
почалася вона з публікацій в українській газеті «Свобода», яка видавалася в
Америці. В одній із статей була така фраза: «Нашу \emph{руска мова} якъ каже одинъ
ученый прфесоръ Бантишъ-Камҍнскiй есть второю по италіяньской що до звучности,
що до солодкости, що до придатности подъ безсмертни ноти всесвҍтных музикальних
композиторовъ»
%%%cit_comment
%%%cit_title
\citTitle{Справа не в милозвучності}, Андрій Любка, day.kyiv.ua, 25.06.2021
%%%endcit

%%%cit
%%%cit_head
%%%cit_pic
\ifcmt
  pic https://gdb.rferl.org/BE77E58B-9F52-49C8-BD7F-589F144B44E6_w1597_n_r1_st.jpg
	caption Під час однієї з акцій проти проєкту так званого «мовного закону Ківалова-Колесніченка», які передували «Мовному майдану». Київ, 24 травня 2012 року 
	width 0.4
\fi
%%%cit_text
З ухваленням Основного закону закінчився один з етапів на довгому шляху
унормування \emph{мовної} ситуації в Україні, що триває досі. Очевидно, що
Конституція 1996 року стала одним із ключових \emph{мовних актів}, які змінили
і змінюють нашу державу.  Те, що переслідувана Москвою \emph{українська мова} є
надзвичайно важливою для суспільства, засвідчили перші такі акти, проголосовані
ще за існування маріонеткової держави УРСР. Під тиском різних факторів, які
збіглися в часі – вулиці, перебудовчих процесів у Москві, справжніх змагальних
виборів до останнього совєтського парламенту і першої демократичної Верховної
Ради України – офіційний маріонетковий Київ ухвалив зміни до «Конституції
Української РСР»
%%%cit_comment
%%%cit_title
\citTitle{Бастіон державності і мови. Конституція України витримує наскоки руйнівників}, 
Тарас Марусик, www.radiosvoboda.org, 02.07.2021
%%%endcit


%%%cit
%%%cit_head
%%%cit_pic
%%%cit_text
На самом деле, Романчук зафиксировал в своей работе объективный процесс,
который называется \emph{языковой конвергенцией} – увеличением числа общих черт в
языках или в диалектах. В случае с \emph{украинским языком} этот процесс проявляется в
том, что он крайне близок \emph{русскому языку}. Романчук яро критикует молодёжь,
которая, по его мнению, больше других групп населения употребляет заимствования
из русского; также доцент-филолог ругает новые технологии, прежде всего
интернет: «Молодёжь уже не представляет себя без импортируемого из-за
«поребрика» маргинального вставного слова «типа» ... Через всемирную
электронную паутину – интернет – также осуществляется электронная русификация»
%%%cit_comment
%%%cit_title
\citTitle{Украинским националистам не нравится близость украинского языка к русскому}, 
Сергей Бондаренко, odnarodyna.org, 30.06.2021
%%%endcit

%%%cit
%%%cit_head
%%%cit_pic
\ifcmt
  pic https://strana.ua/img/forall/u/0/36/2021-07-03_15h50_47.png
	width 0.4
\fi
%%%cit_text
Еще одна женщина при ответе на наш вопрос свободно переходила на несколько
\emph{языков} и считает, что любовь к Украине не зависит от \emph{языка} общения:
\enquote{Я сама говорю на русском \emph{языке}. Очень люблю украинский язык. Очень люблю
русский \emph{язык}. Очень люблю английский \emph{язык}. Этот \emph{язык} для меня прекрасен.
Поэтому у меня совсем другое отношение к \emph{языку}. А любовь к Украине не означает,
что обязательно нужно общаться только на украинском \emph{языке}}
%%%cit_comment
%%%cit_title
\citTitle{Что говорят украинцы о пресс-конференциях футболистов на русском языке. Опрос Страны}, 
Антонина Белоглазова, strana.ua, 03.07.2021
%%%endcit



%%%cit
%%%cit_head
%%%cit_pic
%%%cit_text
Русский \emph{язык} — \emph{язык} граждан Украины всех национальностей — не
может быть \emph{языком} лишь одной национальной меньшины в Украине. Более 30\%
граждан Украины, для которых \emph{русский язык} — родной, и более 70\% граждан
Украины, для которых \emph{русский язык} — \emph{язык общения}, не могут быть
национальной меньшиной в своей стране.
\emph{Русский язык} в Украине — это \emph{язык межнационального общения}. Это
факт, даже если его не признавать. Поэтому статья 10 Конституции Украины должна
быть изменена: \emph{государственный язык} Украины — украинский, \emph{язык
межнационального общения} в Украине — русский. Пора прекращать дискриминацию
миллионов украинцев и \emph{русского языка} в Украине
%%%cit_comment
%%%cit_title
\citTitle{Русский язык в Украине — это язык межнационального общения / Лента соцсетей / Страна}, 
Александр Скубченко, strana.ua, 03.07.2021
%%%endcit

%%%cit
%%%cit_head
%%%cit_pic
%%%cit_text
Йшов двадцять перший рік двадцять першого століття. А ми досі з’ясовували, яка
\emph{мова} в нашій багатомовній країні була \enquote{правильною}. Всі знали і
ніхто не заперечував щодо державності однієї \emph{мови} – української.
Всі знали, якими \emph{мовами} спілкуються в тому чи іншому регіоні. Всі знали,
що є села, де люди говорять тільки угорською чи румунською. Але при цьому
вперто й далі вели боротьбу
%%%cit_comment
%%%cit_title
\citTitle{Українці не розуміють одне одного не через мову, а через небажання слухати, чути і сприймати}, 
Юлія Мендель, www.pravda.com.ua, 07.07.2021
%%%endcit

%%%cit
%%%cit_head
%%%cit_pic
%%%cit_text
За спостереженнями соціолінгвістів, \emph{мова}, яка перебуває у наступі на чужій
території, впроваджує там не високі культурні зразки, а низькі,
звульгаризовані. \emph{Чужа мова}, вторгаючись на колонізовані землі, розкладає
концептуальну сферу мови підкореного народу, вихолощує й спотворює закладені в
ній смисли, денаціоналізуючи й деморалізуючи завойовану спільноту.
Такі згубні наслідки культурного завоювання України Росією художньо відтворила
Леся Українка в драмі «Оргія» на моделі стосунків Давньої Греції й Риму
%%%cit_comment
%%%cit_title
\citTitle{Українська мова в культурному просторі держави. Протистояння триває}, 
Лариса Масенко, www.radiosvoboda.org, 11.07.2021
%%%endcit
