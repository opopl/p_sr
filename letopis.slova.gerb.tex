% vim: keymap=russian-jcukenwin
%%beginhead 
 
%%file slova.gerb
%%parent slova
 
%%url 
 
%%author 
%%author_id 
%%author_url 
 
%%tags 
%%title 
 
%%endhead 
\chapter{Герб}
\label{sec:slova.gerb}

%%%cit
%%%cit_head
%%%cit_pic
\ifcmt
tab_begin cols=2
  width 0.4

  pic https://avatars.mds.yandex.net/get-zen_doc/4452915/pub_60ca5bab8f45bb66915f6fe7_60ca8cc7b4a95c467df45217/scale_1200
  caption Орнитологи уверяют, что это именно он - орёл могильник. Впрочем, всё не так и плохо. Почему - расскажем дальше

  pic https://avatars.mds.yandex.net/get-zen_doc/3047751/pub_60ca5bab8f45bb66915f6fe7_60ca65abf7ddc664b15e2918/scale_1200
  caption «Царь Иоанн III разрывает ханскую грамоту». А.Кившенко. 1879 год. У Ивана III орёл был без регалий, но в XIX веке художник об этом ещё не знал
  width 0.32

tab_end
\fi
%%%cit_text
Иванова грамота. Известный правовед конца XIX века В. А. Градовский писал:
«\emph{Герб} по государственному праву есть внешний символ, видимый отличительный знак
известного государства, эмблематично изображённый на государственной печати,
монете, знамени и т.п. В качестве такого символа \emph{государственный герб} выражает
отличительную идею и основы, осуществлять которые государство считает себя
призванным».  Первым официально сохранившимся документом, где изображён
двуглавый орёл в качестве российского символа, является жалованная грамота
Ивана III, датированная 1497 годом. Эта дата считается официальным «рождением»
\emph{гербового} двуглавого орла на Руси. По поводу его пришествия на Русь существует
много версий. Самой устойчивой является легенда, согласно которой русские
государи – единственные наследники византийских кесарей-императоров
%%%cit_comment
%%%cit_title
\citTitle{Мало кто знает, какой именно орёл, изображён на гербе России}, 
Белорус И Я, zen.yandex.ru, 17.06.2021
%%%endcit
