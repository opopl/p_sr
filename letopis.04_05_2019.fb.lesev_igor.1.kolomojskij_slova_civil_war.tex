% vim: keymap=russian-jcukenwin
%%beginhead 
 
%%file 04_05_2019.fb.lesev_igor.1.kolomojskij_slova_civil_war
%%parent 04_05_2019
 
%%url https://www.facebook.com/permalink.php?story_fbid=2429537183744037&id=100000633379839
 
%%author_id lesev_igor
%%date 
 
%%tags civil_war,donbass,kolomojskii_igor,ukraina,vojna
%%title У нас тут оскорбились словами Коломойского о гражданской войне
 
%%endhead 
 
\subsection{У нас тут оскорбились словами Коломойского о гражданской войне}
\label{sec:04_05_2019.fb.lesev_igor.1.kolomojskij_slova_civil_war}
 
\Purl{https://www.facebook.com/permalink.php?story_fbid=2429537183744037&id=100000633379839}
\ifcmt
 author_begin
   author_id lesev_igor
 author_end
\fi

У нас тут оскорбились словами Коломойского о гражданской войне. Оскорбились не
все, понятное дело, а те, кто у нас 5 лет «в домике».

Это хорошо было видно по реакции Бигуса. Ладно бы, о гражданском конфликте
говорили какие-то сепары и прочая мразь второго сорта. А так об этом говорит
кошерный человек, обещавший когда-то 10 штук денег за сепара. И тут, сука,
«гражданская война».

\ifcmt
  ig https://scontent-frx5-1.xx.fbcdn.net/v/t1.6435-9/57444650_2429537057077383_347284634578452480_n.jpg?_nc_cat=111&ccb=1-5&_nc_sid=730e14&_nc_ohc=CbgZ9T11ofUAX89zRQx&_nc_ht=scontent-frx5-1.xx&oh=38112a0a742d52d06ff0a77048ce1967&oe=61B99E75
  @width 0.4
  %@wrap \parpic[r]
  @wrap \InsertBoxR{0}
\fi

Но смотрите. Для того, чтобы нам хоть куда-то продвинуться в этой беде, вот
хоть на полшишечки, нам все равно придется об этом говорить. При этом не на
кухне и не с теми группами слушателей, где свои и где комфортно.

Говорить надо именно с оппонентами. И если не кидаются в ответ – это уже не
плохо.

Мы или хотя бы пытаемся поставить себе правильный диагноз. Хотя бы. Или
возвращается обратно к шаману и бубнам.

Итак, что такое вообще гражданский конфликт? Вот обобщенно. Наверное, это когда
обладатели паспорта одной страны херячат друг друга. Примеров... да вся история
человечества состоит из таких вот примеров. Даже лень что-то брать за основу.

Зато мне не лень было посмотреть биографии самых известных сепаратистов ЛДНР.
Вот где они родились и проживали до весны 2014 года.

Беднов (Бэтмен) – Луганск. 

Ищенко (Малыш) – это «народный мэр» Первомайска. Родился там же, в Первомайске.
Мозговой – Луганская область. 

Дремов – Луганская область. 

Жилин – Харьков. 

Болотов – родился в Таганроге (это Ростовская область), но был гражданином
Украины и проживал до войны в Луганской области. 

Гиви – Донецкая область (кстати, Иловайск).

Захарченко – Донецк.

Плотницкий – вообще в Черновицкой области, но проживал до войны на Донбассе.

Пушилин – Макеевка.

Пасечник – Луганская область.

Повторюсь, взял только тех, кто на слуху. Из известных инородцев только двое –
Моторола и Стрелков-Гиркин. У всех остальных был паспорт гражданина Украины.

Можно зайти на низовой уровень и посмотреть обменные списки Медведчука. Там
тоже очень некрасивая картинка получается. Потому что происходит обмен граждан
Украины на... граждан Украины. Процентов так на 95.

Есть ли граждане России на Донбассе? Есть. Много? Наверное, много.
«Отпускники», добровольцы, просто идейные. Но почему-то, как только мы говорим
о статистических данных, которые ЛЕГКО можно проверить – публичные лица
сепаратистов и списки пленных – граждан соседнего государства выявляется очень
небольшой процент.

Смотрите. Участие России в вооруженном конфликте на Донбассе переоценить
практически невозможно. Оно ключевое. Без всесторонней поддержки сепаратистские
образования были бы сметены. Это настолько очевидно, что не нужно даже как-то
обосновывать.

Но между Поддержкой и межгосударственной Войной разница все-таки космическая.
Мы говорим о почве. ЛДНР появилась на благодатной почве. Там отличный
сепаратистский чернозем. Если его удобрять вооружением, кураторами и
гуманитарными конвоями, то появится забористый урожай АнтиУкраины. Он там и
появился.

Вот возьмем для контрпримера Сумскую область. Протяженность общей границы с
Россией – огромна. Но есть там хотя бы теоретическая возможность появления
Сумской Народной Республики? Ну разве что очень теоретически. России там нужно
было бы делать Южный Вьетнам с официальным вводом регулярных войск. Потому что
базовая идентификация в Сумской области – украинская.

Да, мантра «мы воюем с Россией» звучит красиво. Это ведь облагораживает. Но
концепция все равно остается рыхлой.

Вот тут у нас Донбасс и мы воюем с российскими оккупантами. А тут чуть севернее
не воюем, потому что у нас уже идет газопровод «Дружба». И мы боремся на
международной арене, чтобы коварный оккупант продолжал его эксплуатировать. Нам
ведь нужны деньги за транзит газа, чтобы мы могли воевать с оккупантом.

\ii{04_05_2019.fb.lesev_igor.1.kolomojskij_slova_civil_war.cmtfront.1.ni_soo}

Да, а южнее у нас Крым. Он не оккупирован, а аннексирован. Никто не знает в чем
разница, но раз так – мы здесь не воюем, и наши граждане могут мотаться туда
отдыхать.

Еще у нас ездят поезда Львов-Москва. Они всегда переполнены. А еще Россия
по-прежнему наш главный экономический партнер. Да, и главный инвестор в нашу
экономику. И 2/3 банковской системы тоже контролирует РФ.

И дипломатические отношения мы тоже не разрываем. А так да, у нас идет война с
Россией.

Однако у нас (и не только у нас) стесняются говорить о другом. На Донбассе идет
война. Только не межгосударственная российско-украинская. А этническая.
Русско-украинская.

Это самый банальный этнический конфликт. Русские убивают украинцев. Украинцы
убивают русских. Противостояние идет именно по линии самоидентификации.

Вот только война идет на территории одного государства. Одного его региона. И
абсолютное большинство участников этой кровавой бани – обладатели паспорта
одной и той же страны.

\ii{04_05_2019.fb.lesev_igor.1.kolomojskij_slova_civil_war.cmtfront.2.elena_fedorova}

В Киеве, Москве, Брюсселе и Вашингтоне об этом не хотят говорить. Потому что
для всех это рушит удобные конструкции. Минск-2. Астана – 56. Сан-Франциско –
до востребования.

Но если мы сами – граждане одной страны – не научимся в табурете видеть
табурет, а не что-то иное, мы никуда не сдвинемся. Вот вообще.

Хотите видеть войну с Россией? Ну, вперед. Удобно воевать за «русский мир» и
Новороссию? Давай. Благородный пафос ведь так греет слух.

\ii{04_05_2019.fb.lesev_igor.1.kolomojskij_slova_civil_war.cmtfront.3.elena_fedorova}

В мире полно пещерных стран. Сомали, Йемен, Афганистан, Судан (оба), Мозамбик,
Нигерия... Там живут чудесные люди. Они стоят у истоков цивилизации. Ходят в
самые правильные церкви. Говорят на самом лучшем языке. Но им всегда кто-то
мешает нормально жить. Всегда есть враги. И их всегда не понимают.

И если у нас стесняются табурет называть табуретом, а этно-гражданский конфликт
этно-гражданским конфликтом, добро пожаловать в чудесный мир пещерных стран.

\ii{04_05_2019.fb.lesev_igor.1.kolomojskij_slova_civil_war.cmtfront.4.rustam_isakov}

\ii{04_05_2019.fb.lesev_igor.1.kolomojskij_slova_civil_war.cmt}
