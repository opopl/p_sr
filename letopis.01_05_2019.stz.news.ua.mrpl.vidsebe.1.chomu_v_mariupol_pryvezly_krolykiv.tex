% vim: keymap=russian-jcukenwin
%%beginhead 
 
%%file 01_05_2019.stz.news.ua.mrpl.vidsebe.1.chomu_v_mariupol_pryvezly_krolykiv
%%parent 01_05_2019
 
%%url https://vidsebe.media/blogs/30969/chomu-v-mariupol-privezli-velikodnih-krolikiv-i-scho-mi-pro-nih-znaemo
 
%%author_id news.ua.mrpl.vidsebe,petrushkina_tetjana.mariupol
%%date 
 
%%tags mariupol,mariupol.event.2019.pysanky_velykden,mariupol.pre_war,pysanka,velykden
%%title Чому в Маріуполь привезли великодніх кроликів. І що ми про них знаємо
 
%%endhead 
 
\subsection{Чому в Маріуполь привезли великодніх кроликів. І що ми про них знаємо}
\label{sec:01_05_2019.stz.news.ua.mrpl.vidsebe.1.chomu_v_mariupol_pryvezly_krolykiv}
 
\Purl{https://vidsebe.media/blogs/30969/chomu-v-mariupol-privezli-velikodnih-krolikiv-i-scho-mi-pro-nih-znaemo}
\ifcmt
 author_begin
   author_id news.ua.mrpl.vidsebe,petrushkina_tetjana.mariupol
 author_end
\fi

\ifcmt
  ig https://i2.paste.pics/debcc8bd4283d039defa53392a9ea386.png
  @wrap center
  @width 0.8
\fi

\textbf{На Театральній площі триває великодня виставка: 30 півметрових писанок і 30
розмальованих кроликів приїхали до Маріуполя з Києва.}

Це частина проекту \enquote{Писанка Арт-Фест} – щорічного фестивалю, під час якого
художники та аматори розмальовують великодні арт-об'єкти.

З писанками усе зрозуміло. Щороку до свята українські господині роблять писанки
або їх полегшений варіант – крашанки. Це досить давня традиція: у 2013 році
\href{https://www.radiosvoboda.org/a/25102278.html}{археологи знайшли на Рівненщині писанку}, виготовлену 950 років тому.\footnote{%
\href{https://www.radiosvoboda.org/a/25102278.html}{На Рівненщині знайшли писанку, якій 950 років, Валентина Одарченко, radiosvoboda.org, 11.09.2013}\par %
Рівне – Глиняну писанку, виготовлену 950 років тому, вдалося знайти археологам
на березі водосховища у Демидівському районі Рівненщині. Експедицію здійснив
Інститут археології Національної академії наук. Керівник експедиції, заступник
директора Інституту археології НАН України Денис Козак зазначив в інтерв'ю
Радіо Свобода, що унікальна знахідка збереглася таким чином, що \enquote{ніби вчора
зробили}.
}

\ii{01_05_2019.stz.news.ua.mrpl.vidsebe.1.chomu_v_mariupol_pryvezly_krolykiv.pic.1}

Писанка – це традиційний символ Великодня для українців. А ось кролик – це
персонаж західної культури, який поволі стає впізнаваним у всьому світі. Він
нині такий самий символ Великодня, як Санта Клаус – Різдва.

Шукати крашанки в саду – популярна великодня розвага в Америці та європейських
країнах. Щоправда, зараз це здебільшого шоколадні яйця або спеціальні тістечка,
а не розфарбовані курячі яйця.

\ii{01_05_2019.stz.news.ua.mrpl.vidsebe.1.chomu_v_mariupol_pryvezly_krolykiv.pic.2}

За однією з легенд, зайці ховають у саду красиві яйця, щоб вибачитися перед
людьми за те, що їдять городину. Інша легенда розповідає про бідну жінку, яка
не мала коштів, щоб купити своїм дітям красиві подарунки до свята. Тож вона
пофарбувала яйця і заховала в саду, а коли діти їх шукали, то поряд пробігав
заєць і вони вирішили, що це він їх приніс.

\ii{01_05_2019.stz.news.ua.mrpl.vidsebe.1.chomu_v_mariupol_pryvezly_krolykiv.pic.3}
%\ii{01_05_2019.stz.news.ua.mrpl.vidsebe.1.chomu_v_mariupol_pryvezly_krolykiv.3.font_17pt}

А взагалі, великодній кролик чи заєць – символ, витоки якого прослідковують аж
до давньогерманських язичницьких вірувань. Заєць був супутником Остари, богині
весни і родючості. Її вшановують на весняне рівнодення, коли оживає природа.
Від імені Остари (в давньому варіанті написання Eostre або Eastre) походить
назва Easter, тобто Великдень англійською. Заєць тоді символізував плодючість,
а яйце – саме життя. З приходом християнства традиції святкування дивним чином
переплелися – відроджуватися стала не лише природа, а й Син Божий. Ну і зайчики
залишилися, хоч і перетворилися на домашніх кроликів.

Фото: листівки – thegraphicsfairy.com, \enquote{Маріуполь – туристичне місто}

%\ii{01_05_2019.stz.news.ua.mrpl.vidsebe.1.chomu_v_mariupol_pryvezly_krolykiv.pic.4.font_17pt}
