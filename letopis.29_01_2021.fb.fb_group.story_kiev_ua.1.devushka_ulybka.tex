% vim: keymap=russian-jcukenwin
%%beginhead 
 
%%file 29_01_2021.fb.fb_group.story_kiev_ua.1.devushka_ulybka
%%parent 29_01_2021
 
%%url https://www.facebook.com/groups/story.kiev.ua/posts/1586768274853301/
 
%%author_id fb_group.story_kiev_ua,petrova_irina.kiev
%%date 
 
%%tags kiev,studenty,universitet.kiev.kisi.knuba,zhizn
%%title Девушка! Девушка! У вас невероятная улыбка!
 
%%endhead 
 
\subsection{Девушка! Девушка! У вас невероятная улыбка!}
\label{sec:29_01_2021.fb.fb_group.story_kiev_ua.1.devushka_ulybka}
 
\Purl{https://www.facebook.com/groups/story.kiev.ua/posts/1586768274853301/}
\ifcmt
 author_begin
   author_id fb_group.story_kiev_ua,petrova_irina.kiev
 author_end
\fi

Как причудливо порой...                   

"Совпадение — это падение сов." (Л.Д.Ландау)
"Совпадение? Не думаю!" (17.05.2015)

Мир соткан из совпадений и случайностей...

Первый курс, не слишком вожделенного, но, отвоеванного в борьбе с препятствиями
, института. Инженерно- строительный !!! Факультет - санитарно-технический!!! А
специальность-то какая - водоснабжение и водоотведение!!  Вот оно - счастье...

Первые дни послешкольной, тихой, знакомой жизни в огромном, чужом, равнодушном
пространстве, блуждания по запутанному лабиринту этажей, аудиторий...  Эта, что
ли?

\ifcmt
  tab_begin cols=2

     pic https://scontent-frt3-1.xx.fbcdn.net/v/t1.6435-9/143506607_3997309036969340_4185187417268989296_n.jpg?_nc_cat=102&ccb=1-5&_nc_sid=b9115d&_nc_ohc=a22mpiDVGBIAX_iBDpA&_nc_ht=scontent-frt3-1.xx&oh=bbdcd862a681151330ab33ed28f0390a&oe=61B3DA09

     pic https://scontent-frt3-1.xx.fbcdn.net/v/t1.6435-9/144371268_3997309480302629_4255707641464699080_n.jpg?_nc_cat=102&ccb=1-5&_nc_sid=b9115d&_nc_ohc=UylXDZQCZIEAX-khNr8&tn=lCYVFeHcTIAFcAzi&_nc_ht=scontent-frt3-1.xx&oh=af1fafabcf5514dd5f0e437177f93041&oe=61B55423

  tab_end
\fi

Ариша переступает порог, неееет, лица  совсем незнакомые, ретируется, скорее к
лифту. Окликают.

- Девушка! Девушка! У вас невероятная улыбка!

(Да ладно, дешевенький подкат, но, симпатичный, высокий, ладный, в очках -
интеллигент, хи-хи).

- Вы почему сбежали?

- Перепутала аудиторию, мне 607-я нужна!

- Это же 7-й, тут 706-я...

- Уже поняла.

- Телефончик оставьте, пожалуйста! Я бы вас проводил, но, уже звонок... А я -
староста, мне опаздывать нельзя.

- А у меня телефона нет! (не врёт)

- Ой, жаль, запишите мой - 74-80...

Она быстро царапает на обрывке обертки рулончика проекта, и бегом.


И, конечно же, этот обрывок канул в Лету... Долго сожалеть не пришлось,
студенческие дни к этому не располагали.

УРА!! Первый курс окончен!!! Сессия! Где приличные киевские студенты к сессиям
готовятся? Праааавильно! В Гидропарке!

"Обережно, двері зачиняються. Наступна станція - Дніпро!"

С подружками - ха-ха-ха, хи-хи-хи!

- Девушка! А я так и не узнал, как вас зовут!

- А должны были узнать? Где?

- Нууу, вот, конечно, девичья память! А в декабре 607 и 706 аудитории помните?

- А! Ой, простите, я тогда потеряла бумажку, а вашу группу не запомнила!

- Тогда сразу - Юрий, группа СМО, уже 31, и теперь пишите в записную книжку!

- Ой,  книжки  нет с собой! Запишу на конспекте по химии, идет?

- Отлично, итак - 74-80... Я очень жду звонка! У вас еще и волосы роскошные!

\ii{29_01_2021.fb.fb_group.story_kiev_ua.1.devushka_ulybka.pic.lug_kozinka}

Верьте-не верьте, на следующий день сдали химию, конспект попросил парнишка из
соседней группы, унес в общагу и...канул в Лету  конспектик...

Лето, солнце, Ялта, потом студенческий лагерь в Пятихатках, практика по
геодезии - да ладно, и без того телефончика есть, чем заняться.

Вечер, костер, всё, что положено студентам в лагере... Смех, напитки,
взгляды...

- Девушка!!! Может, всё-таки скажете . как вас зовут???

Ржание однокурсников : "Её Петя зовут!!!"

- ???

- Ой, привет! Я тогда потеряла конспект! А зовут меня Ариша.

- Значит так, теперь ты уже не потеряешься! Тут территория охраняемая!

Команда велосипедистов была в лагере на сборах, у сантехников - месяц
геодезической практики...Завертелось...

После практики, как было модно в те годы, в подшефный колхоз на месячишко,
первые морозцы, капустка звенящая на рядах, уходящих за горизонт. Жизнь в
сельской хате, два шага от печки - иней по утрам на одеялах!

Но, весело!!!

- Петя!!! К тебе Киса приехал!

(клички, или, как сказали бы сейчас, никнеймы, заменяли имена.)

Киса, он же Юрий, он же обладатель смешной фамилии Кисевич, приехал на
велосипеде в колхоз в Березань. Киевляне оценят расстояние!

Это уже любовь?

Девчонки засматриваются, рост, стать, юмор, мордаха симптаичная, весёлый - ну,
набор неплохой. Та пусть будет, не лишний.

Весной - ужос-ужос-ужос!!! Первая любовь, с самого первого класса любоффь!
Причина порчи пуховых подушек солеными литрами слёз, автор первого поцелуя
и...и даже на свадьбу не позвал... вот это горе! Вот это - всё пропало!

И, как-то в шутку , от Юрки : " Слушай, Петюня, а давай поженимся? А? Квартирку
снимут родители! А?"

- А, давай!

- Когда?

- Пошли в "шоколадку", узнаем что и когда?

3 сентября - ок? Ок!

Родителям, как обухом!

"КАК??? ЗАЧЕМ??? ПОЧЕМУ??? ЗА КОГО??? Что за фамилия - Кисевич???"

Юрий умудрился расположить еще больше потенциальную тёщу на ужине сватовства,
когда довольно отрыгнув после холодцов-буженин-наполенов, изрёк : "Всё было
вкусно, особенно - сметана!"

Всё! БОльшего он сделать не мог, тёща была уже в надлежащем состоянии
"доброжелательства"!

Свадьба в "Метро", сто с лишним гостей - и одноклассники (кроме ЭТОГО, ууу...),
и однокурсники, и родственники, и знакомые...так было принято.

\ifcmt
  tab_begin cols=2

     pic https://scontent-frt3-1.xx.fbcdn.net/v/t1.6435-9/143522222_3997310000302577_2140617694223094851_n.jpg?_nc_cat=102&ccb=1-5&_nc_sid=b9115d&_nc_ohc=lNyrRQNw6zIAX844kIN&_nc_ht=scontent-frt3-1.xx&oh=d83728f347f561e105970f5d60637464&oe=61B65020

     pic https://scontent-frt3-2.xx.fbcdn.net/v/t1.6435-9/144306496_3997310260302551_3086791745816301277_n.jpg?_nc_cat=101&ccb=1-5&_nc_sid=b9115d&_nc_ohc=qqyq3wTP7qEAX9E9uYn&_nc_ht=scontent-frt3-2.xx&oh=aad40f16c166039969001272e53f0006&oe=61B4BDB2

  tab_end
\fi

Квартирку мама Ариши сняла моментально, чтобы пореже видеть зятька - один из
первых домов Оболони - край света, пески, ветер, один автобусик раз в час -
райский уголок.

Звоночек был прямо на свадьбе - страстный танец с подружкой детства Юрия не
остался незамеченным мамой Ариши, что повлекло много бесед, мягко говоря...
дальше - больше...

Есть в народе выражение "ходок", а есть научное разъяснение - "мужчины -
полигамны".

Ариша выучилась отвечать адекватными поступками...да...вот такая семейка
получалась.

Институт окончен, дипломы, и молодого специалиста призывают на год в армию,
офицером.

С превеликим удовольствием вернулась Ариша в родной дом, к маме, к папе, работа
отличная, молодежный коллектив, грустить некогда, вечера, КВНы, фестивали,
народный театр, поездки!

Ой, а почему-то и письма стали реже приходить, не заметила... ой, а чего совсем
не пишет?

Ужасная поездка в глубь черниговских лесов, военная часть на краю цивилизации,
участливая комендантша офицерской общаги : "Деточка, а Кисевич сейчас на
полигоне, будет к вечеру! А ты, девонька, поговори с ним, поговори! А то тут
такааая медсестра у нас ушлая, такая..."

Вот и поговорили...

Медсестра стала женой, Ариша - вольной птицей, Но, они остались отличными
друзьями. Развод и несколько годовщин свадьбы 3 сентября - аккуратно отмечали в
"Метро"! Даже сыновья родились в один и тот же год, и имена дали сыночкам
одинаковые.  "Совпадение? Не думаю!"

\ii{29_01_2021.fb.fb_group.story_kiev_ua.1.devushka_ulybka.pic.igrushki}

Как писал старик Дюма  - "Двадцать лет спустя", да нет, не двадцать, а все
тридцать...

2013-й Ариша с подружкой  встречали  в Москве, на обратном пути, в насквозь
проледенелом поезде РЖД , сходив через два вагона за чайком, Ариша выливает на
руку две чашки кипятка...Натурализма не надо, вид руки - ммм...специфический !

Много советов, самые участливые - от проводницы, в обязанность которой и была
доставка кипятка пассажирам, но, что уж тут.

Ариша не принимает доморощенные советы, набирает Кисевича, помня, что его
супруга (увы, к тому времени, покойная) работала в отделении комбустилогоии.

- Юр, что делать?

- Ничего, через два часа Киев, я приеду к поезду, поедем в ожоговый.

Всё обошлось, спасибо рукам и знанию врачей. И Юрке за помощь.

Семь лет еще пролетело...

Опять Новый Год, Ариша украшет ёлочку, в группе фейсбучной "Киевские истории"
выкладывает фотографии ретроигрушек.

Пишет участница группы : "Вы не можете продать ненужные старые игрушки? Те,
которыми уже не пользуетесь? Хочу украсить ёлку на работе в винтажном стиле."

"Продать? Да я так отдам!" Несколько игрушек, остались от ёлочек в той, далекой
, семейной с Кисевичем жизни,  это он приносил из дома. Это не Аришина история,
лежат всегда в коробке.

Договаривается с Еленой, 30-го декабря встреча не получается, 31-го тем более.

- Леночка, а вы сейчас хотите елочку нарядить?

- Нет, я на следующий год придумала, а сейчас прошу, потому что люди достают
как раз коробки с игрушками.

- А, тогда может подождать до после праздников?

- Да, конечно.

- Я могу даже подвезти Вам, я человек вольный, Вы где работаете?

- На левом берегу, в ожоговом центре !

!!!!????

- Ой, я семь лет назад как раз 2 января у вас там была, меня Юрий Кисевич
привез, муж Милы, покойной... не помните такую?

- Конечно, помню, надо же. А вы, Ариша, у кого тогда лечились, не у меня,
случайно?

- А как Ваша фамилия?

Лена называет красивую, редкую фамилию, и Ариша понимает, что именно это и есть
та самая добрая волшебница, спасшая запястье!

"Совпадение? Не думаю!"

И только уже закончив разговор, Ариша понимает, что Леночке отдаст игрушки
Кисевича..

\ii{29_01_2021.fb.fb_group.story_kiev_ua.1.devushka_ulybka.cmt}
