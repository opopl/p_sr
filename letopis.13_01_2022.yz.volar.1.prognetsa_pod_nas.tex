% vim: keymap=russian-jcukenwin
%%beginhead 
 
%%file 13_01_2022.yz.volar.1.prognetsa_pod_nas
%%parent 13_01_2022
 
%%url https://zen.yandex.ru/media/id/5fce8cdc60e9760ab703b1f4/pust-luchshe-on-prognetsia-pod-nas-61dfc2bcb5b8731ca6e53a81
 
%%author_id yz.volar
%%date 
 
%%tags rossia,zapad
%%title Пусть лучше он прогнётся под нас
 
%%endhead 
 
\subsection{Пусть лучше он прогнётся под нас}
\label{sec:13_01_2022.yz.volar.1.prognetsa_pod_nas}
 
\Purl{https://zen.yandex.ru/media/id/5fce8cdc60e9760ab703b1f4/pust-luchshe-on-prognetsia-pod-nas-61dfc2bcb5b8731ca6e53a81}
\ifcmt
 author_begin
   author_id yz.volar
 author_end
\fi

То ли у нас новости составляют иностранные агенты, специально подбирая такие,
чтоб аж до печёнок пробрало, то ли запад обнаглел по самое болото.

\begin{itemize}
  \item Литва заподозрила, что Россия хочет восстановить СССР
  \item Америка требует прекратить военные учения на территории России вблизи (400 км) от украинских границ и вернуть солдат в казармы
  \item Южная Корея хочет взять у России в аренду Дальний Восток
  \item Конгресс обсуждает санкции в отношении Президента России
\end{itemize}

Вашингтон не позволит.... Евросоюз требует.... Польша настоятельно
рекомендует.... Латвия категорически не согласна.... Франция возмущена....

\ii{13_01_2022.yz.volar.1.prognetsa_pod_nas.pic.1}

Вам какое дело хочется спросить? Что мы делаем на своей территории вас вообще
не должно волновать. Хотим мы восстановить Союз или не хотим.... Где наша армия
бродит по просторам своей Родины... Что мы изобрели или построили....Вас это
как касается?!

Когда все такие бо́рзые-то стали, я не пойму?!  @igg{fbicon.face.flushed}  Своими делами не пробовали
заняться?

Нет, правда! Запад до такой степени уверен, что он может, а самое главное,
имеет право так нагло разговаривать с Россией, что просто диву даёшься!
Бессмертные чтоль?!

Хоть новости не читай и не слушай. Ловишь себя на том, что при просмотре орешь
на телевизор. @igg{fbicon.beaming.face.smiling.eyes} 

Вот сейчас, да? Переговоры прошли. Два дня что-то там перетирали. Результат,
судя по заявлениям, нулевой. И опять-таки... мало-мало-мало огня...Опять наши
дипломаты слишком дипломатичные. Выражения подбирают. Конструктивно
поговорили... Жёстче надо, товарищи. Жёстче. Риторика должна быть такая, чтоб
прочувствовали и перекрестились (если ещё помнят как это делается), а не
"калечащими" санкциями угрожали.

Конечно, я понимаю, что мы сами виноваты. Допартнёрствовались. Наша мягкость и
неконфликтность стала восприниматься западом, как слабость и бесхребетность. В
этом жестоком, жестоком мире мы нарушили главное правило- выживает сильнейший.

Долгое время Россия уступала, молчала, терпела, пыталась договариваться и не
грызла глотки за свои интересы. Сейчас же вдруг показала клыки. А мир не верит.
Нет-с, не верит. Войну начнёте? Ой, не верим. А что ещё у вас есть против нас?
А у нас есть...И продолжает хамить.

Бесит.

Перефразируя популярный фильм, надеюсь и уповаю на нашего Президента, что он
приготовил для них очень хороший военно-технической ответ. Никто не знает что
он задумал, но я уже хочу обрадоваться.
