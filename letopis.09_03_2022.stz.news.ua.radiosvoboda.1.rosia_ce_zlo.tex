% vim: keymap=russian-jcukenwin
%%beginhead 
 
%%file 09_03_2022.stz.news.ua.radiosvoboda.1.rosia_ce_zlo
%%parent 09_03_2022
 
%%url https://www.radiosvoboda.org/a/viyna-bizhentsi-volonter-ahresor-henotsyd/31744296.html
 
%%author_id tereschuk_galina
%%date 
 
%%tags 
%%title Те, що робить Путін, Росія ‒ це зло ‒ доброволець зі США
 
%%endhead 
 
\subsection{Те, що робить Путін, Росія ‒ це зло ‒ доброволець зі США}
\label{sec:09_03_2022.stz.news.ua.radiosvoboda.1.rosia_ce_zlo}
 
\Purl{https://www.radiosvoboda.org/a/viyna-bizhentsi-volonter-ahresor-henotsyd/31744296.html}
\ifcmt
 author_begin
   author_id tereschuk_galina
 author_end
\fi

«Не втікай. Захищай!» – написано на великому рекламному щиті перед воротами у
пункті пропуску «Шегині» (українсько-польський кордон).

\ii{09_03_2022.stz.news.ua.radiosvoboda.1.rosia_ce_zlo.pic.1}

Два тижні тут не припиняється автомобільний рух і нескінченний потік людей на
пішому переході. Переважно у черзі стоять жінки з дітьми. Дітей дуже багато.
Малеча ‒ загорнута на руках у матерів, старші дітлахи пересувають валізи чи
торбини. Чимало людей взяли своїх собак.

Люди приїжджають до пункту пропуску, що на Львівщині, з різних регіонів
України. До пішого переходу, а відкриті вони по всьому українсько-польському
кордоні, добираються самостійно або ж організованим транспортом зі Львова. Щоби
дійти до паспортного контролю, розповідають, простоюють у черзі від 5 до 7
годин.

\ii{09_03_2022.stz.news.ua.radiosvoboda.1.rosia_ce_zlo.pic.2}

Оксана тримає трирічного сина на руках, а її старший син «тримає» чергу, поки
вона з сестрою Оленою та однорічним племінником гріються у наметі. Сестри з
дітьми приїхали у пункт пропуску зранку і у черзі стоять 4 години. Ще година –
і вони перетнуть кордон. Оксана розмовляла телефоном із мамою, не припиняючи
плакати і дякувати. Бо мама з татом допомагають у рідному місті українським
військовослужбовцям, роздаючи гарячі чай і каву. Обороняти країну залишився і
її чоловік.

Жінки розповідають Радіо Свобода свою історію, постійно витираючи сльози. Вони
покинули своє рідно місто Южноукраїнськ Миколаївської області.

«Ми живемо біля Вознесенська. Сказали знайомі, що на вулицях лежать мертві,
страшно ходити. А з Южноукраїнська – нам туди три години. Тому ми злякалися. Ми
їдемо з Миколаївської області. Живемо, де атомна станція. Там напружена
ситуація. Вирішили 6 березня виїжджати. Сестра приїхала і сказала, що я маю 10
хвилин, мовляв, збирайся. Собі майже нічого не взяла з речей, лише дітям, бо
місця у машині не так багато. Чоловік сестри привіз нас сюди, а сам
повертається додому захищати нашу землю. Допоміг нам у черзі. Ніколи не думала,
що сидітиму ось так, як біженка. Ми любимо Україну. Чоловіки наші там захищають
країну, вони записались у тероборону. Я ‒ підприємець. Кафе – у нас. Мама
залишилась і з татом роздають безкоштовно чай і каву. Щойно телефонувала і
розповіла, що українські військові гроші їй хочуть давати. Вона відмовляється.
І вони їй квіти принесли», ‒ говорить Оксана.

«Моя товаришка працює у Польщі і нас чекає у Вроцлаві. Недавно туди поїхала.
Коли стане мирно, то ми одразу повернемось додому. Їдемо, бо страшно за дітей.
Лише життя налагодилося ‒ купили квартиру, народився син...», ‒ каже Олена.

\ii{09_03_2022.stz.news.ua.radiosvoboda.1.rosia_ce_zlo.pic.3}

До пункту пропуску «Шегині» постійно підходять люди, під’їжджають автобуси з
втікачами від війни. Багатьох перевозять через кордон організовано зі Львова.

«Я з 7-ї ранку ‒ на кордоні. А зараз – 12-та. Їдемо у Краків до друзів. Ще
трішки. Загалом із Харкова вже їдемо три дні. Дуже довго і важко, навіть не
хочу розповідати. Частково родина тут залишилась, частково ‒ у Харкові.
Сподіваюсь квартира залишиться. Хочу повертатись назад, але якщо Харкова не
буде, то не повернусь. Ніколи не думала, що стану біженкою», ‒ розповіла одна з
пасажирок автобуса.

\subsubsection{Польські волонтери створили Центр допомоги}

Сьогодні ситуацію у пункті пропуску «Шегині» не порівняти з тією, яка була у
перші дні масштабної війни, коли на пішому переході люди стояли добу і більше,
коли іноземці проривали кордон, створювали тисняву, автомобільна черга сягала
50 км, не було туалетів, пунктів обігріву. Українська сторона зорганізувала
безперешкодний доїзд до кордону.

\ii{09_03_2022.stz.news.ua.radiosvoboda.1.rosia_ce_zlo.pic.4}

А 5 березня з українського боку у пункті пропуску почав діяти центр
гуманітарної допомоги в Шегині, який організували польські волонтери. Вони
встановили мобільні намети, де біженці можуть отримати медичну допомогу,
встановили великий намет для обігріву, де можна випити чаю чи кави, полежати,
посидіти, зарядити телефон.

За п’ять днів волонтери рознесли людям кілька десятків тисяч склянок гарячого
чаю. Порахувати кількість випитого чаю ‒ просто нереально, зауважує координатор
Центру гуманітарної допомоги в «Шегинях», польський волонтер \textbf{Войцех Гурецький}.

«Найбільший рух у нашому гуманітарному центрі, коли на вулиці робиться темно,
холодно. У пункті можна зігрітись, діти отримують їжу, можна випити чай. Маємо
амбулаторію, можемо надати медичну допомогу, обробити рани. До нас їдуть лікарі
з Польщі. Робимо все, щоб якомога більше допомогти», ‒ каже Радіо Свобода
Войцех Гурецький.

У Центрі гуманітарної допомоги в «Шегинях» працює від 20 до 25 волонтерів із
Польщі. Вони змінюють один одного. Але такої кількості людей не вистачає, щоб
допомогти біженцям і підтримувати діяльність пункту обігріву. Тому поляки
просять долучатись до них українців, які допоможуть ще й краще порозумітись з
україномовними та російськомовними втікачами від війни.

\ifcmt
  ig https://gdb.rferl.org/02220000-0aff-0242-d06c-08da01a85fb8_w650_r0_s.jpg
  @caption Люди очікують перетину кордону
  @wrap center
  @width 0.8
\fi

\subsubsection{Українські чоловіки повертаються з Європи}

Від початку вторгнення Росії в Україну, за даними Державної прикордонної
служби, на батьківщину повернулись 77 тисяч чоловіків з Європи. Більшість
працювала за кордоном, але, почувши про війну, не знаходили собі місця.
Протягом години, коли Радіо Свобода перебувало перед пішим переходом, ним
перейшли в Україну більш як 10 чоловіків, які покинули спокійне життя у Європі
і будуть захищати батьківщину. Найбільше чоловіків повертались у перші дні
війни, повідомила працівниця прикордонної служби у «Шегинях» Ольга Павлюк.

Чоловіки розповіли Радіо Свобода, чому вони зробили цей крок.

«Їду у Черкаси. У мене троє дітей ‒ двом по 5 років, а меншому півтора. Хотів
забрати всіх до Польщі. Але мені сказали, що зі своєї землі нікуди не підуть.
Доведеться мені йти і боронити їх. Перемога лише за Україною», ‒ каже
Олександр.

«Працював 5 років у Польщі. Я ‒ зі Стрия. Повертаюсь захищати свою неньку
Україну, до своєї родини, до всіх. Ми переможемо. Повернувся, бо настав
критичний момент. Хлопців вистачає, хоч їх теж вбивають, я зрозумів, що треба
допомогти, підтримати. Маю військовий досвід, служив у прикордонних військах.
Не знаходив собі місця у Польщі, коли почалась війна. У мене внукові буде 20
березня рік. Їду його захищати, щоб жив у незалежній Україні. Завтра піду у
військкомат і подам заявку, що хочу стати на захист України», ‒ говорить
Андрій.

\ii{09_03_2022.stz.news.ua.radiosvoboda.1.rosia_ce_zlo.pic.5}
