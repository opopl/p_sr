% vim: keymap=russian-jcukenwin
%%beginhead 
 
%%file 18_02_2022.stz.news.ua.pravda.1.lavrov_pacan
%%parent 18_02_2022
 
%%url https://www.pravda.com.ua/news/2022/02/18/7324481
 
%%author_id kizilov_evgen
%%date 
 
%%tags lavrov_sergei,rossia,zapad
%%title Лавров почав говорити з Заходом на фені: "Пацан сказав - пацан зробив"
 
%%endhead 
 
\subsection{Лавров почав говорити з Заходом на фені: \enquote{Пацан сказав - пацан зробив}}
\label{sec:18_02_2022.stz.news.ua.pravda.1.lavrov_pacan}
 
\Purl{https://www.pravda.com.ua/news/2022/02/18/7324481}
\ifcmt
 author_begin
   author_id kizilov_evgen
 author_end
\fi

Міністр закордонних справ Росії Сергій Лавров закликав Захід до діалогу про
безпеку \enquote{по поняттям}, щоб було так - \enquote{пацан сказав - пацан зробив}.

Джерело: інтерв'ю Лаврова пропагандистському телеканалу RT, яке цитує ТАСС

\ii{18_02_2022.stz.news.ua.pravda.1.lavrov_pacan.pic.1}

Пряма мова: \enquote{Будемо добиватися, щоб усе було чесно. Не хочу звертатися
до жаргону, але у нас є поняття - \enquote{пацан сказав - пацан зробив}.
Принаймні, \enquote{поняття} повинні дотримуватися і на міжнародному рівні }.

Деталі: Лавров заявив, що Москва домагатиметься, аби США та НАТО виконували
\enquote{прийняті на себе зобов'язання} з безпеки та \enquote{надані Росії
обіцянки} у зв'язку з цим.

Пряма мова: \enquote{Повторю, ми зацікавлені в тому, щоб докладніше пояснити
американським колегам і всім їхнім союзникам по НАТО, що не можемо
задовольнитись обіцянками. Тим більше, що письмові зобов'язання глав держав та
урядів, які вимагають від натовців враховувати наші інтереси повною мірою, не
кажучи про усні гарантії, про які президент Росії Володимир Путін неодноразово
згадував, виявляється, нічого не варті. Так справа не піде}.
