% vim: keymap=russian-jcukenwin
%%beginhead 
 
%%file 06_06_2021.fb.maslov_evgenij.1.shkola_pedagogika_kultura_pomeranc
%%parent 06_06_2021
 
%%url https://www.facebook.com/maslovevgeniy14/posts/1495594637448906
 
%%author Маслов, Евгений
%%author_id maslov_evgenij
%%author_url 
 
%%tags 
%%title Слово "кала" на санскрите - омоним: и время, и смерть. Культура, не нашедшая опоры в вечном, падет под напором перемен
 
%%endhead 
 
\subsection{Слово \enquote{кала} на санскрите - омоним: и время, и смерть. Культура, не нашедшая опоры в вечном, падет под напором перемен}
\label{sec:06_06_2021.fb.maslov_evgenij.1.shkola_pedagogika_kultura_pomeranc}
\Purl{https://www.facebook.com/maslovevgeniy14/posts/1495594637448906}
\ifcmt
 author_begin
   author_id maslov_evgenij
 author_end
\fi

"...Одна из проблем, которую нельзя решить высокоточными ракетами, - миллиарды
недорослей, недоучек, недоразвитков. Примитивные народы умели воспитывать своих
мальчиков и девочек. Простая культура целиком влезала в одну голову, и в каждой
голове были необходимые элементы этики и религии, а не только техническая
информация. Культура была духовным и нравственным целым. Естественным примером
этой цельности оставались отец с матерью. Сейчас они банкроты. Тинейджер,
овладевший компьютером, считает себя намного умнее деда, пишущего авторучкой.
Мир изменился, каждые пять лет он другой, и все старое сбрасывается с корабля
современности. Растут миллиарды людей, для которых святыни, открывшиеся
малограмотным пастухам, не стоят ломаного гроша. Полчища Смердяковых, грядущие
гунны, тучей скопились над миром. И они в любой день готовы пойти за Бен
Ладеном или Баркашовым. Записку Иконникова гунны не прочли (а если б и прочли -
что им Иконников? Что им князь Мышкин?). Судьбу Другого они на себя не
возьмут...

\ifcmt
  pic https://scontent-cdg2-1.xx.fbcdn.net/v/t1.6435-9/196803638_1495594607448909_9143839619333698734_n.jpg?_nc_cat=107&ccb=1-3&_nc_sid=730e14&_nc_ohc=jg_9nIHAdeAAX_QvRUP&_nc_ht=scontent-cdg2-1.xx&oh=d15c66a0f91de63ff9a4fb2ccd9f53d6&oe=60E378EF
\fi

Одно из бедствий современности - глобальная пошлость, извергаемая в эфир.
Возникает иллюзия, что глобализм и пошлость - синонимы. И глобализм уже поэтому
вызывает яростное сопротивление. Не только этническое. Не только
конфессиональное... 

Одна из особенностей великих культурных миров - способность к историческому
повороту, переходу от расширения вовне к внутреннему росту, от захваченности
центробежными процессами к созерцанию духовного центра и покаянию за отрыв от
него... Лидерами станут страны, которые лучше других сумеют создать новый стиль
жизни, включить паузу созерцания в череду дел, избавиться от лихорадки
деятельности, от \enquote{блуда труда, который у нас в крови} (Мандельштам). Пионерами
могут быть и большие и маленькие страны, сильные и слабые. Мы не знаем, кто
вырвется вперед. Но начинать должны все. 

Решающей становится не экономика, а педагогика, начиная с детского сада. Дети
схватывают начатки нового быстрее взрослых. Я вспоминаю слова девочки четырех
лет: \enquote{Мама, не говори громко, от этого засыхают деревья}... С самого
раннего детства можно воспитывать понимание радости, которую дает созерцание. И
это подготовит людей к переоценке ценностей, к переходу от инерции
неограниченного расширения техногенного мира к цивилизации созерцания,
духовного роста и равновесия с природой. 

Если мы будем просто звать людей ограничить свои потребности, ничего не выйдет,
кроме раздора. Петр кивнет на Ивана, Европа на Америку, Азия на Европу. Поворот
может дать только открытие ценности созерцания, паузы созерцания в делах, в
диалогах и дискуссиях, в развитии мысли….  Школа не может отвлечься от
сегодняшнего дня, не может не готовить программистов, юристов, менеджеров. Но
сегодняшний день скоротечен, и течение несет его к смерти. Слово \enquote{кала}
на санскрите - омоним: и время, и смерть. Культура, не нашедшая опоры в вечном,
падет под напором перемен.  Школы могут и должны учить паузе созерцания: через
искусство, через литературу. Со временем - используя телевидение, если оно
повернется к величайшей проблеме века..."

Григорий Соломонович Померанц (13 марта 1918, Вильно, Литва — 16 февраля 2013, Москва, Россия)

\emph{Lioudmila Setteducato}

УМНО! ДА УСЛЫШАТ И ПРИМУТ!!!

\emph{Marina Novgorodska}

Прекрасный философ! Только кто сейчас понимает слово ,, созерцание,,? А ведь
оно ,ключевое в мысли Григория Померанца и ключ к пониманию сказанного. Можно
порекомендовать прочесть его книгу ,, Записки гадкого утёнка,, для начала .
