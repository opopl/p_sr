% vim: keymap=russian-jcukenwin
%%beginhead 
 
%%file poplavskii_dmitrii_markovich.bronepoezd_liter_B.wikipedia_ru_05_08_2020
%%parent poplavskii_dmitrii_markovich
 
%%endhead 
\subsection{Бронепоезд Литер Б}
\url{https://ru.wikipedia.org/wiki/%D0%91%D1%80%D0%BE%D0%BD%D0%B5%D0%BF%D0%BE%D0%B5%D0%B7%D0%B4_%C2%AB%D0%9B%D0%B8%D1%82%D0%B5%D1%80_%D0%91%C2%BB}
  
Бронепоезд «Литер Б» — второй бронепоезд народного ополчения, построенный в
Киеве и принимавший участие в обороне города в ходе Великой Отечественной
войны.  

% align: l,c,r 
\begin{longtable}{ll}
\multicolumn{2}{c}{Бронепоезд «Литер Б»} \\
Принадлежность      & СССР                                    \\
Подчинение          & соединениям Юго-Западного фронта        \\
Эксплуатация        & 20 июля — 22 сентября 1941 года         \\
Изготовитель        & Киевский паровозо-вагоноремонтный завод \\
Участие в           & Киевская оборонительная операция        \\
Известные командиры & Василевский, Леонид Владиславович       \\
\multicolumn{2}{c}{Технические данные}                  \\
Силовая установка       & Бронепаровоз типа Ов          \\
Мощность                & 600 лошадинных сил            \\
Бронирование            & незакаленая сталь             \\
Количество броневагонов & 2 артиллерийско-пулемётных    \\
Экипаж                  & 77 человек\cite{Kajnaran2012} \\
\multicolumn{2}{c}{Вооружение} \\
Лёгкое вооружение         & 28 пулемётов \cite{Kolomiec2010}\\
Артиллерийское вооружение & 
2 76-мм пушки образца 1914/15 годов (Лендера) \cite{Kajnaran2012,Kolomiec2010}\\
\end{longtable}
