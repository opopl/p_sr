% vim: keymap=russian-jcukenwin
%%beginhead 
 
%%file 29_08_2023.stz.news.ua.mrpl_city.1.vikno_u_majbutnje_mural_kyiv_mariupol
%%parent 29_08_2023
 
%%url https://mrpl.city/news/view/vikno-u-majbutne-u-kievi-stvorili-mural-prisvyachenij-ukrainskomu-mariupolyu
 
%%author_id news.ua.mrpl_city,prokopchuk_elina.mariupol
%%date 
 
%%tags 
%%title "Вікно у майбутнє" - у Києві створили мурал, присвячений українському Маріуполю
 
%%endhead 
 
\subsection{\enquote{Вікно у майбутнє} - у Києві створили мурал, присвячений українському Маріуполю}
\label{sec:29_08_2023.stz.news.ua.mrpl_city.1.vikno_u_majbutnje_mural_kyiv_mariupol}
 
\Purl{https://mrpl.city/news/view/vikno-u-majbutne-u-kievi-stvorili-mural-prisvyachenij-ukrainskomu-mariupolyu}
\ifcmt
 author_begin
   author_id news.ua.mrpl_city,prokopchuk_elina.mariupol
 author_end
\fi

\begin{quote}
\em\color{blue}
\textbf{До Дня Незалежності в лівобережному центрі \enquote{ЯМаріуполь} в Києві створили мурал
про майбутнє міста. Малюнки для проєкту підготували мешканці - вони зображали
Маріуполь таким, яким мріють його побачити після деокупації та відновлення.}
\end{quote}

\ii{29_08_2023.stz.news.ua.mrpl_city.1.vikno_u_majbutnje_mural_kyiv_mariupol.pic.1}

Як \href{https://t.me/yamariupol_kyiv/8546}{розповідають у центрі}, за малюнками маріупольська художниця Марія Дьяконова
створила візуалізацію задумів містян, а художниця з Бердянська Марія Удінова
допомогла перенести зображення на стіну.

\ii{29_08_2023.stz.news.ua.mrpl_city.1.vikno_u_majbutnje_mural_kyiv_mariupol.pic.tg}

\vspace{0.5cm}
\begin{minipage}{0.9\textwidth}
	
\textbf{Читайте також:} 

\href{https://mrpl.city/news/view/u-mariupoli-okupanti-znishhili-shhe-odin-mural}{%
У Маріуполі окупанти знищили ще один мурал, Еліна Прокопчук, mrpl.city, 11.08.2023}
\end{minipage}
\vspace{0.5cm}

Майбутнє маріупольці бачать з відновленою історичною архітектурою, жовто-синіми
прапорами та морем, за яким так скучили усі мешканці.

\ii{29_08_2023.stz.news.ua.mrpl_city.1.vikno_u_majbutnje_mural_kyiv_mariupol.pic.2}

\emph{\enquote{Чотири райони міста видніються у вікнах, які намальовані за ескізами
маріупольців. Тут - море та веселка в Приморському та Лівобережному, квіти та
яскраве солнце у Кальміуському, історична архітектура та улюблені місця
маріупольців у Центральному районі. А по центру - двері в український, сучасний
та комфортний Маріуполь. Таким його бачать мешканці}}, - розповіли у центрі
\enquote{ЯМаріуполь. Київ}.

Робота над муралом тривала тиждень - до оздоблення стіни долучилися
маріупольці.

Нагадаємо,
\href{https://mrpl.city/news/view/u-poltavi-stvoryuyut-mural-na-chest-zahisnika-mariupolya-morpiha-volini}{раніше
у Полтаві створили мурал на честь захисника Маріуполя морпіха
\enquote{Волини}}.

Фото: \enquote{ЯМаріуполь. Київ}

\ii{29_08_2023.stz.news.ua.mrpl_city.1.vikno_u_majbutnje_mural_kyiv_mariupol.pic.3}
