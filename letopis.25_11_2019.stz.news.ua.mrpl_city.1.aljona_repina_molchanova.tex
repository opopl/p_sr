% vim: keymap=russian-jcukenwin
%%beginhead 
 
%%file 25_11_2019.stz.news.ua.mrpl_city.1.aljona_repina_molchanova
%%parent 25_11_2019
 
%%url https://mrpl.city/blogs/view/alona-repina-molchanova-nemae-nichogo-nemozhlivogo
 
%%author_id demidko_olga.mariupol,news.ua.mrpl_city
%%date 
 
%%tags 
%%title Альона Репіна-Молчанова: "Немає нічого неможливого"
 
%%endhead 
 
\subsection{Альона Репіна-Молчанова: \enquote{Немає нічого неможливого}}
\label{sec:25_11_2019.stz.news.ua.mrpl_city.1.aljona_repina_molchanova}
 
\Purl{https://mrpl.city/blogs/view/alona-repina-molchanova-nemae-nichogo-nemozhlivogo}
\ifcmt
 author_begin
   author_id demidko_olga.mariupol,news.ua.mrpl_city
 author_end
\fi

\ii{25_11_2019.stz.news.ua.mrpl_city.1.aljona_repina_molchanova.pic.1}

Продовжуючи нариси про яскравих маріупольців, хочу познайомити вас з дуже
неординарною і унікальною особистістю. У неї багато прізвиськ і одне велике
серце. Друзі і знайомі називають її \emph{\enquote{жіночка-катастрофа}, \enquote{танк},
\enquote{жінка-термінатор}}, але для більшості вона \emph{\enquote{Людина, що не може відмовити у
допомозі}}. З'яв\hyp{}ляється там, де потрібна, намагається завжди бути корисною. Ну
що ж, дорогі читачі, пропоную вам відкрити для себе маріупольську диво-жінку –
\textbf{Альону Репіну-Молчанову}.

Народилася Альона в Лисичанську на Луганщині. З 2 місяців до 4 років жила на
Кольському півострові. Батько – військовий, тому часто переїжджали. Мама за
освітою - вчитель російської мови і літератури, але в Абхазії і Маріуполі
працювала у військкоматі. Сестра, молодша на 5 років, народилася в
Азербайджані. Пізніше переїхали в Абхазію, де Альоні подобалося найбільше. У
військовому містечку було все дуже спокійно, дівчинка полюбила людей, з якими
її сім'я жила поруч, адже всі вони були добрими і дружніми. Вона з дитинства
відрізнялася особливою чуйністю, звикла допомагати іншим і боротися за
справедливість. Зокрема, в Абхазії збирала речі і віддавала малозабезпеченим
людям.

\textbf{Читайте також:} \emph{Ольга Белкова: \enquote{Быть толерантным – это не значит покорно терпеть}}%
\footnote{Ольга Белкова: \enquote{Быть толерантным – это не значит покорно терпеть}, Анна Хижникова, mrpl.city, 16.11.2019, \par%
\url{https://mrpl.city/blogs/view/olga-belkova-byt-tolerantnyme-to-ne-znachit-pokorno-terpet}%
}

До Маріуполя родина переїхала 24 серпня 1987 року. З 8-го по 10-й клас
навчалася в 63 школі. Альону дратувало, що однокласників розрізняли за
соціальною і національною ознаками. Вона намагалася довести, що це нісенітниця
і, якщо хтось з біднішої сім'ї, ображати його не можна. Такої несправедливості
вона не бачила в Абхазії, куди намагалася після переїзду декілька раз втекти.
Але згодом наша героїня все ж таки змогла полюбити Маріуполь і зрозуміла, що
люди в місті проживають різні, серед яких багато і щирих, добрих та відкритих.
У дитинстві дівчина хотіла стати істориком. В юності – військовою. Але батьки
відмовили. Поступила в торговий технікум, після якого закінчила Донецький
національний університет економіки і торгівлі імені Михайла Туган-Барановського
(спеціальність – економіст-маркетолог). 

\ii{25_11_2019.stz.news.ua.mrpl_city.1.aljona_repina_molchanova.pic.2}

У 21 рік Альона вийшла заміж. З чоловіком познайомилася в лікарні, коли
провідувала спільну з ним знайому. Завдяки своїм знанням і навичкам в 24 роки
стала завідувачкою магазину \enquote{Каскад}. Там працювала до 2001 року (до 28 років),
поки не пішла в декрет. Після декрету продовжила працювати на посаді
завідувачки магазину. З 2005 року починає працювати в житлово-комунальному
підприємстві бухгалтером квартального відділу (з 2005 до 2016 року).

\ii{25_11_2019.stz.news.ua.mrpl_city.1.aljona_repina_molchanova.pic.3}

За своєю природою наша героїня - справжній емпат. Але один випадок в її житті
довів, що вона все правильно робить і буде продовжувати допомагати іншим. Так,
коли в 2010 році її оперували, дуже багато людей, навіть зовсім незнайомих,
кинулися їй допомагати. Це не могло не вразити жінку. Вона й досі щиро вдячна
всім, хто не відвернувся від неї і допоміг у скрутних життєвих обставинах.
Відтоді Альона намагається ще частіше надавати допомогу всім, хто її потребує.
Каже, що коли відчуває, що дійсно допомогла, радіє неймовірно.

\textbf{Читайте також:} \emph{Как сделать социальные проекты успешными и не сгореть в процессе их реализации?}%
\footnote{Как сделать социальные проекты успешными и не сгореть в процессе их реализации?, Дарья Касьянова, mrpl.city, 08.01.2019, \par%
\url{https://mrpl.city/blogs/view/kak-sdelat-sotsialnye-proekty-uspeshnymi-i-ne-sgoret-v-protsesse-ih-realizatsii}
}
\ii{25_11_2019.stz.news.ua.mrpl_city.1.aljona_repina_molchanova.pic.4}

Під час подій майдану та війни Альона не залишалась осторонь. Вона активно
брала участь в допомозі військовим та переселенцям. У 2016 році потрапила в
Благодійний Фонд \enquote{Карітас Маріуполь}. Про її широку волонтерську діяльність
почув і отець Ростислав Спринюк, який вирішив, що такі небайдужі люди повинні
працювати з ним. Жінка займається збором речей, допомагає переселенцям і
малозабезпеченим маріупольцям, які цього потребують. Також вона неодноразово
надавала допомогу військовим. Їздила в госпіталі, збирала кошти на нагороди.
Навіть ремонтувала БТР і самотужки збирала машини. Талантів у волонтерки
вистачає. Також здає кров як донор. У майбутньому планує реалізувати власний
соціальний проєкт. Альона мало спить (4-5 годин), дуже рано встає, тому
встигає зробити безліч справ.

\ii{25_11_2019.stz.news.ua.mrpl_city.1.aljona_repina_molchanova.pic.5}

Її син \textbf{Ілля} вчиться на другому курсі на
спеціальності \enquote{Політологія} в Харківському національному університеті імені В.
Н. Каразіна. Син маму в усьому підтримує. Альона дуже любить своїх котів: Льоню
(11 років) і Адольфа (1,5 роки). Найближчим часом хоче вивчити іспанську та
італійську. Мріє відвідати Марракеш. У Маріуполі любить парки, море та мечеть.

\ii{25_11_2019.stz.news.ua.mrpl_city.1.aljona_repina_molchanova.pic.6}

\textbf{Плани на майбутнє:} відкрити свою громадську організацію, адже серед знайомих
маріупольчанки дуже багато волонтерів, дру\hyp{}зів, які допомагають і підтримують.

\textbf{Улюблені фільми:} \enquote{Собаче серце} (1988 рік), \enquote{Майстер і Маргарита} (2005 рік).

\textbf{Улюблена книга:} \enquote{Людина в пошуках справжнього сенсу} австрійського психіатра
Віктора Франкла.

\textbf{Цікаві факти:} найближча подруга називає Альону \enquote{жіночка-катастрофа}, для
знайомих вона \enquote{танк}, адже часто бачить мету і при цьому не помічає перешкод. А
відомий маріупольський фотограф Дмитро Чичера називає Альону
\enquote{жінка-термінатор}.

\ii{25_11_2019.stz.news.ua.mrpl_city.1.aljona_repina_molchanova.pic.7}

\textbf{Курйозні випадки:} Пам'ятаєте фільм з П'єром Рішаром і Жераром Депардьє
\enquote{Невдачливі}, героїня якого володіла властивістю притягувати до себе всі біди і
нещастя? Так от наша Альона теж має таку властивість. Але про нещасливі випадки
в своєму житті вона розповідає з гумором. Зокрема, на Кольському півострові
(3–4 роки) пірнула з пірсу за рибкою. На щастя, тато встиг зловити за шарф.
Засунула в ніс мозаїку, після чого везли катером на велику землю. Азербайджан
(6–7 років) – мало не провалилася в підземелля. Абхазія (13 років) – випадково
зіштовхнули з пірсу, ледве врятували. Маріуполь (16 років) – лікувала горло (є
такий спосіб роздавлений часник в бинті покласти у вуха на 20 хв) і поклала
шматочок у вухо. Горло вилікувалося, а от з вуха часник діставав вже лікар. 17
років – вмивалася, шматок скла застряг в оці, діставали під наркозом. Під час
декрету їла рибу, кістка застрягла, рятували в лікарні. Під час пересадки
кактусу загнала колючку, що призвело до операції і місяця лікарняного. Ось таке
\enquote{веселе} життя у нашої героїні.

\textbf{Порада маріупольцям:} 

\begin{quote}
\em\enquote{Пам'ятайте, що нічого неможливого немає! Бажаю бути мудрішими, перемоги над
всіма негараздами}.
\end{quote}

\textbf{Читайте також:} \emph{НеМовчи: в Мариуполе презентуют провокационный социальный проект}%
\footnote{НеМовчи: в Мариуполе презентуют провокационный социальный проект, Олена Онєгіна, mrpl.city, 22.11.2019, \par%
\url{https://mrpl.city/news/view/nemovchi-v-mariupole-prezentuyut-provokatsionnyj-sotsialnyj-proekt-foto}
}

\emph{Автор останнього фото - Євген Сосновський.}
