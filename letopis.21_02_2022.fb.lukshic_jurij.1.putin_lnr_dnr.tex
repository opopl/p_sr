% vim: keymap=russian-jcukenwin
%%beginhead 
 
%%file 21_02_2022.fb.lukshic_jurij.1.putin_lnr_dnr
%%parent 21_02_2022
 
%%url https://www.facebook.com/permalink.php?story_fbid=1188458255020134&id=100015679115504
 
%%author_id lukshic_jurij
%%date 
 
%%tags __feb_2022.putin.priznanie,dnr,donbass,lnr
%%title Путин подписал указы о признании независимости "ЛНР" и "ДНР"
 
%%endhead 
 
\subsection{Путин подписал указы о признании независимости \enquote{ЛНР} и \enquote{ДНР}}
\label{sec:21_02_2022.fb.lukshic_jurij.1.putin_lnr_dnr}
 
\Purl{https://www.facebook.com/permalink.php?story_fbid=1188458255020134&id=100015679115504}
\ifcmt
 author_begin
   author_id lukshic_jurij
 author_end
\fi

Что ж, теперь можно комментировать по факту. Путин подписал указы о признании
независимости \enquote{ЛНР} и \enquote{ДНР}.

1. Украина провалила переговорный процесс о восстановлении территориальной
целостности за последние 8 лет. Порошенко и Зеленский и их команды, я бы
сказал, кто вы, но такое на \enquote{Фейсбуке} не публикуется.

2. Жители \enquote{ЛДНР} и ОРДЛО, ваша жизнь станет немного легче. Для них открываются
социальные, политические и экономические возможности за счёт поддержки со
стороны России. Так что признание для них определённо плюс.

3. У России успех, как ни странно, половинчатый. А всё потому, что разрушение
Минского формата приведёт к ужесточению антироссийских санкций. 

4. Создание частично признанных квазиреспублик на Донбассе может запустить
дестабилизацию всего региона. Во-первых, \enquote{ЛДНР} могут заявить права на весь
Донбасс, а то и юго-восток. Если начнётся полномасштабная война, расклад, увы,
не в пользу Украины. Во-вторых, вероятен открытый ввод войск России (и
Беларуси; менее вероятен - ОДКБ) на территорию ОРДЛО.
