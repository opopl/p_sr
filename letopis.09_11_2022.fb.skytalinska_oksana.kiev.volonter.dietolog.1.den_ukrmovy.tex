% vim: keymap=russian-jcukenwin
%%beginhead 
 
%%file 09_11_2022.fb.skytalinska_oksana.kiev.volonter.dietolog.1.den_ukrmovy
%%parent 09_11_2022
 
%%url https://www.facebook.com/O.Skytalinska/posts/pfbid02HNJfdn2iiKFpBoACaX38BWVnPbd2Aa9ERRjQX7Sa4s7E1aLA9CJuuczGBLF6jCNpl
 
%%author_id skytalinska_oksana.kiev.volonter.dietolog
%%date 
 
%%tags mova
%%title Сьогодні -- День української мови та писемності
 
%%endhead 
 
\subsection{Сьогодні -- День української мови та писемності}
\label{sec:09_11_2022.fb.skytalinska_oksana.kiev.volonter.dietolog.1.den_ukrmovy}
 
\Purl{https://www.facebook.com/O.Skytalinska/posts/pfbid02HNJfdn2iiKFpBoACaX38BWVnPbd2Aa9ERRjQX7Sa4s7E1aLA9CJuuczGBLF6jCNpl}
\ifcmt
 author_begin
   author_id skytalinska_oksana.kiev.volonter.dietolog
 author_end
\fi

Сьогодні -- День української мови та писемності.

Українська -- моя мова до глибини кісток, я нею думаю, сміюсь і плачу.

Наприклад, вчора моя любов до України і українськок мови проявилась у відправці
4 монокулярів, тепловізора, 3 шоломів, 8 потужних павербанків, ящика сухої
ковбаси та різних смачних соусів на Бахмут, і, окремо , в якості "швидкої
допомоги" -- відправці шолома, навушників, рюкзака, одягу, кави на Миколаїв. 

Мовою цифр це 270 000 грн, які Ви надсилали.  Це все, на що вистачило коштів.

В дитинстві зі мною всі розмовляли українською. А мій тато -- ще й англійською.

Мене виховали на українських казках, піснях, козацьких думах, співах капели
бандуристів.

Я завжди розмовляла українською, навіть тоді, коли вона була не модна, не
вигідна, шкодила кар'єрі,  вважалась меншовартісною. У 90-х я брала участь у
конкурсі краси, конкурс був російською, всі учасниці розмовляли російською,
багато хто -- із сильним акцентом:) Коли дійшла моя черга дати мені слово,
ведучий поблажливо попередив "а Аксана будєт на укрАінском" @igg{fbicon.smile} 

Міс краси я не стала, але стала "міс глядацьких симпатій" @igg{fbicon.smile} 

Тому я щиро радію, коли багато людей переходить на українську, коли чую уривки
розмов між батьками та дітьми українською.

В дитинстві мій дитячий патріотизм був таким сильним, що я сказала якось "тату,
я готова померти за Україну", на що тато відповів: "за Україну треба жити,
Оксаночко".

Пишіть диктант, читайте, розмовляйте українською. Вона прекрасна, милозвучна,
барвиста, багата. Як наша земля, на яку століттями зазіхають, але спіймають
облизня @igg{fbicon.heart.red}@igg{fbicon.flag.ukraina}

@igg{fbicon.exclamation.mark} ️Допомагайте фронту!
@igg{fbicon.exclamation.mark} ️Ваші гривні перетворюються на "очі" і "вуха" наших захисників, знайте, що це і частинка кожного з Вас захищає Україну.
@igg{fbicon.exclamation.mark} Найбільш потрібні речі -- монокуляри, павербанки, тепловізори -- допомагайте їх купувати. Або купуйте їх і передавайте на передову, контакти всі дам.

-------------------------

@igg{fbicon.flag.ukraina}

\obeycr
PAYPAL: o.skytalinska@gmail.com
4149 5001 4943 7181 Райфайзен
5375 4114 1959 3632 Монобанк
\restorecr

-------------------------

"Слава Україні!" та "Героям Слава!" -- це не просто гарні слова. Це сила і мрії
багатьох поколінь, які передали нам цю естафету -- вберегти Україну
@igg{fbicon.heart.red} @igg{fbicon.flag.ukraina}

🥦
Вже наступний пост буде продовження теми Українська Дієта" -- про наукові
обґрунтування вживання українських традиційних продуктів, про помилкові
"тренди" харчування, нав'язані ринком та маркетингом і про те, як це шкодить
нашому здоров'ю. Початок тут:
\url{https://m.facebook.com/story.php?story_fbid=5911440295542749&id=100000305102907}

\#україна \#українапонадусе \#українськамова \#деньписемності  \#довголіття \#здоровехарчування \#їжа \#дієтолог \#скиталінська
