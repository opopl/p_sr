% vim: keymap=russian-jcukenwin
%%beginhead 
 
%%file 12_09_2022.stz.news.ua.donbas24.1.festival_teatralna_brama_detali
%%parent 12_09_2022
 
%%url https://donbas24.news/news/yak-v-mariupoli-proxodiv-festival-teatralna-brama-detali
 
%%author_id demidko_olga.mariupol,news.ua.donbas24
%%date 
 
%%tags 
%%title Як в Маріуполі проходив фестиваль "Театральна брама" - деталі
 
%%endhead 
 
\subsection{Як в Маріуполі проходив фестиваль \enquote{Театральна брама} - деталі}
\label{sec:12_09_2022.stz.news.ua.donbas24.1.festival_teatralna_brama_detali}
 
\Purl{https://donbas24.news/news/yak-v-mariupoli-proxodiv-festival-teatralna-brama-detali}
\ifcmt
 author_begin
   author_id demidko_olga.mariupol,news.ua.donbas24
 author_end
\fi

\ifcmt
  ig https://i2.paste.pics/e7c951762b720d6ec33058af3a1d9cb3.png
  @wrap center
  @width 0.9
\fi

\begin{center}
  \em\color{blue}\bfseries\Large
11−19 вересня 2021 року на сцені драмтеатру відбувся обласний відкритий
фестиваль театрального мистецтва \enquote{Театральна брама}
\end{center}

Фестиваль \href{https://www.facebook.com/TeatralnaBrama}{\enquote{Театральна брама-2021}} мав на меті показати глядачу унікальний
всесвіт з традиційними цінностями та новими підходами, інноваціями, мистецьким
проривом, а також налагодити співпрацю між театральними колективами для
подолання стереотипів та єднання між регіонами.

\subsubsection{Скільки колективів об'єднав фестиваль?}

\ii{12_09_2022.stz.news.ua.donbas24.1.festival_teatralna_brama_detali.pic.1}

10 професійних театральних колективів з різних регіонів України представили
маріупольцям та гостям міста свої вистави. Серед учасників конкурсної програми
— театри з Києва, Миколаєва, Дніпра, Сум, Львова, Кропивницького, Харкова,
Херсона. Відкривав фестиваль маріупольський драматичний театр своєю яскравою
виставою \enquote{Біла ворона}, яка як і інші вистави театру в Маріуполі, була
представлена українською мовою. Якщо на першому фестивалі \enquote{Театральна брама},
що проходив теж у Маріуполі, було представлено один національний театр, то 2021
року їх було три. Це Київський національний академічний театр російської драми
імені Лесі Українки, Миколаївський національний академічний український театр
драми та музичної комедії і Сумський національний академічний театр драми та
музичної комедії імені М. Щепкіна. Актори та режисери цих театрів були приємно
вражені театральною культурою міста та загальним розвитком Маріуполя.

\begin{leftbar}
	\begingroup
		\bfseries
\qbem{Я давно не бачив таких вдячних глядачів. У Маріуполі вперше, але рівень
культури, театральна сцена, краса міста загалом дуже здивували.
Очікував побачити промислове містечко, а відкрив для себе розвинуте
українське туристичне місто}, — поділився враженнями актор з
Миколаївського національного академічного українського театру драми.
	\endgroup
\end{leftbar}

\subsubsection{Хто оцінював вистави?}

\ii{12_09_2022.stz.news.ua.donbas24.1.festival_teatralna_brama_detali.pic.2}

Оцінювало всі представлені на фестивалі вистави високоповажне журі. До складу
журі увійшли \href{https://www.facebook.com/profile.php?id=100000612002046}{Дмитро Дощинський} — відомий театральний критик, театрознавець,
який оцінює вистави й XXIII міжнародного фестивалю \enquote{Мельпомена Таврії}. Також
серед журі була і \href{https://www.facebook.com/helena.libo}{Олена Либо} — театрознавиця, кураторка Міжнародних і
Всеукраїнських театрально-мистецьких проєктів, кураторка театральних
фестивалів, старша викладачка кафедри театрознавства Харківського національного
університету мистецтв ім. І.П.Котляревського, членкиня Міжнародної Спілки
діячів театру ляльок \enquote{UNIMA-Україна}, членкиня національної спілки театральних
діячів України (м.Харків). Представляли журі і педагог, художній керівник
акторсько-режисерського курсу \enquote{Майстерня 55} Харківського національного
університету мистецтв імені І. П. Котляревського Валерій Гувін, заступниця
начальника обласного управління культури та туризму Світлана Шмельова та відома
маріупольчанка, народна артистка України, лауреатка премії ім. М.
Заньковецької, театрознавиця Світлана Отченашенко, яка загинула 7 квітня 2022
року в Маріуполі. Експертне журі одноголосно вирішило, що фестиваль 2021 року
переріс статус обласного і наблизився до всеукраїнського. Водночас вони
наголосили на актуальності всіх запропонованих вистав та висловили сподівання,
що фестиваль стане щорічним. 

\subsubsection{Підсумки фестивалю}

У фестивалі \enquote{Театральна брама-2021} Донецький академічний обласний
драматичний театр (м. Маріуполь) за свою виставу \enquote{Маруся} за Л.
Костенко, режисерки-постановниці, заслуженої артистки України \textbf{Людмили Колосович}
отримав найвищу нагороду — \emph{гран-прі.}
\href{https://www.facebook.com/profile.php?id=100023187227038}{Віра Шевцова} за
роль Марусі Чурай була відзначена в номінації \enquote{За високохудожнє
створення образу}.

\ifcmt
  ig https://i2.paste.pics/2ae92b7dfde25dd0175537f03b61950c.png
  @wrap center
  @width 0.8
\fi

\ii{12_09_2022.stz.news.ua.donbas24.1.festival_teatralna_brama_detali.pic.3}

\begin{leftbar}
	\begingroup
		\bfseries
\qbem{Приємно вразила його і гра Віри Шевцової та всієї театральної трупи, яка
помітно виросла. Вважаю цей спектакль найпотужнішим з усього
фестивального репертуару}, — наголосив Валерій Гудвін.
	\endgroup
\end{leftbar}

Також журі відзначили й іншу роботу Донецького академічного обласного
драматичного театру (м. Маріуполь) — рок-оперу \enquote{Біла Ворона}
(режисерка-постановниця \textbf{Анжеліка Добрунова}), якою фестиваль було відкрито.
Насправді багато маріупольців вболівали саме за цей спектакль, адже він давно
для багатьох містян став найулюбленішим. Втім і цю виставу не залишили без
номінацій. Так, актриса маріупольського театру \href{https://www.facebook.com/profile.php?id=100028732189242}{Дар'я Недавня} за талановите
виконання ролі Жанни д'Арк отримала перемогу в номінації \emph{\enquote{За високохудожнє
виконання ролі}}. Народна артистка України, театрознавиця та членкиня журі
\href{\urlSvitlanaIvanivnaOtchenashenkoIA}{Світлана Отченашенко} не могла оцінювати спектаклі свого театру, але від вистави
\enquote{Біла ворона} була у повному захваті. 

\ifcmt
  ig https://i2.paste.pics/9c953028692c37d6c0c14239454a0705.png
  @wrap center
  @width 0.9
\fi

\begin{leftbar}
	\begingroup
		\bfseries
\qbem{Під час перегляду я ловила себе на думці, що щаслива бути частиною
такого талановитого і потужного театру. Вся трупа працювала на
максимальному видиху}, — підкреслила рік тому \href{\urlSvitlanaIvanivnaOtchenashenkoIA}{Світлана Іванівна}.
	\endgroup
\end{leftbar}

\ii{12_09_2022.stz.news.ua.donbas24.1.festival_teatralna_brama_detali.pic.4}

Виконавці головних ролей у виставах, які сподобалися маріупольським глядачам
найбільше, серед них і \enquote{Маргарита й Абульфаз} за С. Алексієвич (Львівський
академічний обласний музично-драматичний театру імені Юрія Дрогобича), \enquote{Баба
Пріся або на початку і наприкінці часів} П. Ар'є (Херсонський обласний
академічний музично-драматичний театр імені Миколи Куліша), \enquote{Хлібне перемир'я}
С. Жадана (Харківський державний академічний український драматичний театр ім.
Т. Г. Шевченка) та інші були відзначені журі в номінаціях \enquote{За високу
майстерність, високопрофесійний акторський ансамбль, творчий потенціал та
високохудожнє створення образу}. Після завершення фестивалю керівництво
маріупольського драматичного театру висловило сподівання, що цей фестиваль
протягом наступних років стане ще більш масштабним...

\ii{12_09_2022.stz.news.ua.donbas24.1.festival_teatralna_brama_detali.pic.5}

\subsubsection{Які були плани і що змінилося?}

Насправді \enquote{Театральна брама — 2021} дав можливість маріупольському драматичному
театру успішно заявити про себе. За прогнозами театрознавців, після фестивалю
маріупольці могли б швидше долучитися до всеукраїнських та міжнародних
фестивалів, адже такий потужний колектив хотіли б бачити в своїх стінах й інші
театри. Членкиня журі та театрознавиця \textbf{Олена Либо} підкреслила, що наступний
фестиваль повинен стати більш розгалуженим і включати в себе соціальні,
культурно-мистецькі, освітні та інклюзивні проєкти. На жаль, провести фестиваль
\enquote{Театральна брама} у 2022 році не вдалося, завадило повномасштабне вторгнення
росії в Україну. Наразі Маріуполь окупований, а деякі театральні діячі міста
добровільно обрали для себе роль колаборантів.

Рівно рік тому тодішній генеральний директор театру Володимир Кожевніков
пишався своїм колективом та наголовшував, що цей фестиваль дуже важливий для
театральної справи Маріуполя і загалом для колективу драмтеатру, про який
повинні знати в Україні. 

\begin{leftbar}
	\begingroup
		\bfseries
\qbem{Ця подія для нас дуже важлива, як і важлива оцінка провідних
театрознавців країни, адже необхідно розуміти, чи в правильному напрямі
рухається наш театр}, — підкреслив Кожевніков рік назад. 
	\endgroup
\end{leftbar}

Тобто на той час керівництво Донецького академічного обласного драматичного
театру (м. Маріуполь) наголошувало на необхідності співпраці колективу з іншими
театрами України. А вже сьогодні і директор, і заступник директора, і головна
бухгалтерка театру вирішили обрати іншу співпрацю — з окупантами, що ще рік
тому для них було непримустимим. Своїм вибором вони посприяли роз'єднанню
всього колективу на декілька угруповань (до речі, одне таке угруповання
залишилося в Маріуполі) і втратили здобуті можливості для розвитку. Лишається
сподіватися, що після деокупації маріупольський драмтеатр зможе долучитися до
нових фестивалів та надолужити втрачене на чолі зі справжніми керівниками,
відданими своїй країні та власній справі. 

\ii{12_09_2022.stz.news.ua.donbas24.1.festival_teatralna_brama_detali.pic.6}


Нагадаємо, раніше Донбас24 розповідав, \href{https://donbas24.news/news/potyag-do-peremogi-yak-istoriyi-regioniv-vtililisya-v-malyunkax-ukrayinskix-mitciv-foto}{як історії регіонів втілилися в малюнках
українських митців}.

Ще більше новин та найактуальніша інформація про Донецьку та Луганську області
в нашому \href{https://t.me/donbas24/}{телеграм-каналі Донбас24}.

ФОТО: з відкритих джерел.

\ii{insert.author.demidko_olga}
