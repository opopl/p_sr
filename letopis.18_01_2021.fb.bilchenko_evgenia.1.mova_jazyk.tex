% vim: keymap=russian-jcukenwin
%%beginhead 
 
%%file 18_01_2021.fb.bilchenko_evgenia.1.mova_jazyk
%%parent 18_01_2021
 
%%url https://www.facebook.com/yevzhik/posts/3566865556681862
 
%%author 
%%author_id 
%%author_url 
 
%%tags 
%%title 
 
%%endhead 

\subsection{БЖ. О русском языке без банальности}
\label{sec:18_01_2021.fb.bilchenko_evgenia.1.mova_jazyk}
\Purl{https://www.facebook.com/yevzhik/posts/3566865556681862}

\ifcmt
  pic https://scontent-bos3-1.xx.fbcdn.net/v/t1.6435-9/140313177_3566865490015202_1803062037982538616_n.jpg?_nc_cat=104&ccb=1-3&_nc_sid=8bfeb9&_nc_ohc=Bb9PvSNH994AX-PU1Dy&_nc_ht=scontent-bos3-1.xx&oh=2734364df86f7ce267c990ca12e3ac2e&oe=60B15B7F
	width 0.2
\fi

Пора и мне высказаться на наболевшую тему закона о языке от 16.01. Я опасаюсь,
что мое мнение покажется слишком категоричным, но... Родные мои, а при чем тут
конкретно этот непристойный закон? Гниют экономика, региональная и внешняя
политика, культура и образование, медицина... В квартирах - холодно, в карманах
- пусто, тарифы - выше крыше, учащиеся путают Милан с Нью-Йорком (причем,
Нью-Йорк - это страна!), уличные этнорадикалы руководят правительством, внешние
глобалистические силы руководят уличными радикалами.  

Процветает человеконенавистническая идеология. Ксенофобия по отношению к
собственному населению достигла военного и гражданского апогея.
Националистическая цензура овладела не только медиа, но и наукой.
Цивилизационная память уничтожается на всех уровнях...

Вот о чем надо говорить. А этот закон - лишь звено в цепи, очередной опухший
лимфоузел в теле онкобольной американской колонии. И стыдливые полумеры в
оценках ситуации посредством попытки отстоять... хотя бы свою систему знаков -
недостаточны: так мы сами себя обрекаем на лузерство, на выпрашивание того, что
и так нам принадлежит по праву.

Особенно поражает меня как представителя русской цивилизации, который владеет
украинским как вторым родным и говорит на нем часами каждый день, реакция тех
русскоязычных \enquote{гибридных} писателей, которые, едва научившись \enquote{промовляти на
кшталт},  продравшись из унизительного (не природного, а именно унизительного)
суржика, теперь выдавливают из себя нечто вроде: \enquote{Ничего страшного, можно и
потерпеть}. 

Обращусь к этим писателям. Вас раком ставят, господа литераторы, не меня, -
мне-то что? Именно вас. Я, наоборот, сейчас сознательно перехожу в сфере
обслуживания на русский: я годами на автомате говорю на мове, я не задумываюсь,
на чем я думаю именно сейчас, но тут больно смотреть, как испуганно
перестраивается под новый маразм мой бедный непобедимый народ. А вы тем
временем бормочите что-то самоуспокоительное про \enquote{ничего страшного}. 

Потому что именно вам - страшно. Только знаете что? Наци, они ведь вам этого не
зачтут, если возьмут реванш, если их не удалят с шахматной доски местечковой
истории носители цветного тренда. Но, ребята, soros-people вы тоже не нужны:
ученикам Фукуямы надо что-то покруче, чем \enquote{лилии-линии}. Так, стоит ли, вот за
это вот всё, душу нечистому продавать, а?

\ii{18_01_2021.fb.bilchenko_evgenia.1.mova_jazyk.cmt}
