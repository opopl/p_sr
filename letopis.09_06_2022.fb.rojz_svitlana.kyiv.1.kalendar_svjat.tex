% vim: keymap=russian-jcukenwin
%%beginhead 
 
%%file 09_06_2022.fb.rojz_svitlana.kyiv.1.kalendar_svjat
%%parent 09_06_2022
 
%%url https://www.facebook.com/svetlanaroyz/posts/pfbid0yBnadaRF5o3rCFXcRrjipTqa8NxwDcRLq4keV3Qsu2uKsah5CknWwB34pJi36bZCl
 
%%author_id rojz_svitlana.kyiv
%%date 
 
%%tags 
%%title Календар свят на кожен день. Щоб додати радості та легкості.
 
%%endhead 
 
\subsection{Календар свят на кожен день. Щоб додати радості та легкості}
\label{sec:09_06_2022.fb.rojz_svitlana.kyiv.1.kalendar_svjat}
 
\Purl{https://www.facebook.com/svetlanaroyz/posts/pfbid0yBnadaRF5o3rCFXcRrjipTqa8NxwDcRLq4keV3Qsu2uKsah5CknWwB34pJi36bZCl}
\ifcmt
 author_begin
   author_id rojz_svitlana.kyiv
 author_end
\fi

Календар свят на кожен день. Щоб додати радості та легкості. Допустити її
можливість. це для нас так важливо зараз! Червень. У мене сьогодні була складна
розмова з родиною, яка важко проходить етап адаптації до нової країни...Я думаю
про досвід дитини та батьків, з якими  спілкувалася – і зловила себе на тому,
як часто я повторюю вголос і подумки – ми зараз не просто адаптуємося до нових
умов, ми знов і знов адаптуємося до Життя. 

Це – сторінка нашого з художницею Diana Dubossarska Ресурсного календаря на
червень  – який я ще обов'язково надрукую.  Все, що тут є – можна
використовувати - думати, грати, втілювати, присвячувати день,читати разом з
дітьми і вигадувати - а як би це відсвяткувати  – де б ми не знаходились. 

\ii{09_06_2022.fb.rojz_svitlana.kyiv.1.kalendar_svjat.pic.1}

Червень

\begin{itemize}
  \item 1. День літніх думок
  \item 2. День ліхтарного стовпа
  \item 3. День рахування сходинок
  \item 4. День слухання міста
  \item 5. День аромату літа
  \item 6. День синього кольору
  \item 7. День ходіння пішки
  \item 8. День кульбабкового дихання
  \item 9. День діставання сніговика з холодильника (чи відкривання холодильника, щоб охолодитись, чи створення крижаних кубиків із квітами та ягодами)
  \item 10. День танцю сонячних зайчиків
  \item 11. День музики з нічого
  \item 12. День сміху над дурним жартом
  \item 13. День улюблених туфель (чи день ходіння по траві босоніж)
  \item 14. День книжкових хробаків
  \item 15. День \enquote{чому б і ні}
  \item 16. День вигулювання сукні (одягу)
  \item 17. День морозива
  \item 18. День поєднання непоєднуваного
  \item 19. День ходіння по брівці
  \item 20. День просто так
  \item 21. День усміхнених птахів
  \item 22. День легкості
  \item 23. День дотику до кактусу
  \item 24. День роблення секретиків
  \item 25. День здорових снів
  \item 26. День топтання як слон
  \item 27. День лоскотання пером
  \item 28. День відпускання думок – на вітер
  \item 29. День перетворення на метелика
  \item 30. День незвичайних соків – фрешів
\end{itemize}

Наша Перемога – не тільки військова. Ще одна важлива перемога кожного дня – не
втратити надії і можливості радіти. 

Обіймаю, Родино, як хочу Перемоги! @igg{fbicon.heart.red}

%\ii{09_06_2022.fb.rojz_svitlana.kyiv.1.kalendar_svjat.orig}
%\ii{09_06_2022.fb.rojz_svitlana.kyiv.1.kalendar_svjat.cmtx}
