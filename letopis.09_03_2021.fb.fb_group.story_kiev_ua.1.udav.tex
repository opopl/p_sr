% vim: keymap=russian-jcukenwin
%%beginhead 
 
%%file 09_03_2021.fb.fb_group.story_kiev_ua.1.udav
%%parent 09_03_2021
 
%%url https://www.facebook.com/groups/story.kiev.ua/posts/1614114395452022
 
%%author_id fb_group.story_kiev_ua,namestnik_myhailo
%%date 
 
%%tags kiev
%%title "В мене є думка і я її думаю" - як казав удав з відомого мультфільму
 
%%endhead 
 
\subsection{\enquote{В мене є думка і я її думаю} - як казав удав з відомого мультфільму}
\label{sec:09_03_2021.fb.fb_group.story_kiev_ua.1.udav}
 
\Purl{https://www.facebook.com/groups/story.kiev.ua/posts/1614114395452022}
\ifcmt
 author_begin
   author_id fb_group.story_kiev_ua,namestnik_myhailo
 author_end
\fi

\enquote{В мене є думка і я її думаю} - як казав удав з відомого мультфільму.

З цікавістю гортаю сторінки нашої в цілому толерантної групи, яка об'єднує
людей різного віку і освіти, віри, національності і мови - людей, які вважають
себе киянами, хоча багато хто став ними нещодавно, а інші і зовсім живуть
далеко від улюбленого міста - міста, яке лежить не тільки на семи пагорбах, але
й на перехресті семи вітрів.

І за пару тисячоліть існування через нього хвилями перекотилися гуни і готи,
печеніги і половці, варяги і греки, монголи і росіяни, німці, червоні, білі,
зелені та сіро-буро-малинові.  За ці часи сотні разів населення Києва корінним
чином мінялося, як мінялися  і складові частини його крові. А як казав
Коров'єв: \enquote{Вопросы крови - самые сложные вопросы в мире}. 

Може саме тому ми такі різні та й  історії у нас різні. Але у всіх, принаймні в
нашій різночинній групі, є один спільний знаменник - сам Київ. 

До речі, торік дочка на день народження подарувала мені можливість зробити
генетичний тест.

І виявилося, що я - етнічний українець з діда-прадіда - належу аж до чотирьох
генетичних груп та шести етносів : на 62,1\% до східноєвропейського етносу
(українці, білоруси, поляки),  на 12,1\%  до балтійського (литовці, латвійці,
естонці),  на 11\% до британського (шотландці, валлійці, ірландці), на 9,2\% до
балканського (хорвати, серби, македонці), на 4,7\% до центральноазійського (гуни
і монголи) та на 0,9\% до амазонського (племена Південної Америки!!!).

І мене, як амазонського дикуна пожежна машина,  постійно дивує зарозуміла пиха
деяких так званих корінних киян, які хизуються  тим, що їхні прадіди на якусь
сотню років раніше за інших \enquote{понаїхали} до нашого міста. Хоча це так смішно і
безглуздо, як пишатися довжиною власного кишечника.

Тож закінчу думку  приказкою моєї тітки Явдохи, яка казала: \enquote{Не хвалися, а Богу
молися}.

P.S. Припускаю, що можна гордитися досягненнями своїх предків і успіхами
нащадків. З великою натяжкою,  але можна, мабуть, пишатися  власними заслугами
- усілякими дипломами та брязкальцями типу ромбиків, кружечків або квадратиків.
Але в цілому, відносно почуття гордості,  закрадається підозра, що це просто
внутрішня або, якщо хочете, лицьова сторона пересічної неповноцінності. Втім,
фрейдисти можуть мене поправити.

P.P.S. До відома тих закоренілих киян, чия персональна гордість була уражена
моїми словами: колись я примудрився працювати на посаді, яка відповідно до
\enquote{Табели о рангах},  дорівнювала  чину надвірного радника. Тож при зверненні до
мене прошу без усяких церемоній використовувати вираз \enquote{Ваше високоблагородіє}.

\ii{09_03_2021.fb.fb_group.story_kiev_ua.1.udav.cmt}
