% vim: keymap=russian-jcukenwin
%%beginhead 
 
%%file 17_06_2021.fb.bilchenko_evgenia.4.post_panoptikum
%%parent 17_06_2021
 
%%url https://www.facebook.com/yevzhik/posts/3987509137950833
 
%%author Бильченко, Евгения
%%author_id bilchenko_evgenia
%%author_url 
 
%%tags bilchenko_evgenia,internet,konferencia,nauka,obschestvo
%%title БЖ. Пост-Паноптикум, сетевая цензура, либеральная слежка
 
%%endhead 
 
\subsection{БЖ. Пост-Паноптикум, сетевая цензура, либеральная слежка}
\label{sec:17_06_2021.fb.bilchenko_evgenia.4.post_panoptikum}
\Purl{https://www.facebook.com/yevzhik/posts/3987509137950833}
\ifcmt
 author_begin
   author_id bilchenko_evgenia
 author_end
\fi

БЖ. Пост-Паноптикум, сетевая цензура, либеральная слежка. 

\ifcmt
  pic https://scontent-lga3-2.xx.fbcdn.net/v/t1.6435-9/202051234_3987506897951057_2173300810918671507_n.jpg?_nc_cat=109&ccb=1-3&_nc_sid=730e14&_nc_ohc=lcM-NZ2C38wAX_jnF_L&_nc_ht=scontent-lga3-2.xx&oh=39d05837370a343ea266bdc609ff163b&oe=60D1214B
  caption Атомарность - Качество Человека
\fi

А пока не начался мой доклад на международной конференции по дигитальным
манипуляциям в Югре с 60 странами, чтобы немного успокоиться вместо выкуривания
второй пачки сигарет, размещу здесь свою презентацию. 


\ifcmt
  tab_begin cols=2
     width 0.4

     pic https://scontent-lga3-2.xx.fbcdn.net/v/t1.6435-9/201494434_3987506857951061_5380513300647750376_n.jpg?_nc_cat=105&ccb=1-3&_nc_sid=730e14&_nc_ohc=2nmIX9VgMrgAX8kTYfI&_nc_ht=scontent-lga3-2.xx&oh=8a15d2435316d22c0113235a6fbdfb27&oe=60CF450E

     pic https://scontent-lga3-2.xx.fbcdn.net/v/t1.6435-9/201370283_3987506851284395_1279852520808653780_n.jpg?_nc_cat=104&ccb=1-3&_nc_sid=730e14&_nc_ohc=G-oH1NbH-5MAX-xI78E&_nc_ht=scontent-lga3-2.xx&oh=e271d2f6b1c4ec21311f4c36fb8b4063&oe=60D0F7F3

  tab_end
\fi

Чуть позже размещу и сам доклад: здесь - порядок такой, что выступаешь вживую и
очень четко, кратко, импровизационно для десятка профессиональных синхронных
переводчиков, но все материалы с терминологией готовишь заранее - они идут в
письменную версию. Присутствуют все отрасли знаний: от физики до психоанализа.

\ifcmt
  tab_begin cols=2

     pic https://scontent-lga3-2.xx.fbcdn.net/v/t1.6435-9/202323332_3987506877951059_5220689419573571178_n.jpg?_nc_cat=103&ccb=1-3&_nc_sid=730e14&_nc_ohc=kiOXOXAqptkAX9AOkQK&_nc_ht=scontent-lga3-2.xx&oh=a99f4979cb1623fb96b5b10878a52e8a&oe=60CF3D14

     pic https://scontent-lga3-2.xx.fbcdn.net/v/t1.6435-9/201855360_3987506871284393_7854897242973868917_n.jpg?_nc_cat=106&ccb=1-3&_nc_sid=730e14&_nc_ohc=Q6z148vismkAX-DG3qn&_nc_ht=scontent-lga3-2.xx&oh=4f1d532e62021a21b6d96828a85d200c&oe=60CFB3F8

  tab_end
\fi

\ifcmt
  pic https://scontent-lga3-2.xx.fbcdn.net/v/t1.6435-9/202349584_3987506887951058_5083092786530401090_n.jpg?_nc_cat=107&ccb=1-3&_nc_sid=730e14&_nc_ohc=GT_CteNNDUgAX8KCxyt&_nc_ht=scontent-lga3-2.xx&oh=a3f606cccd2760a30aa74c74cfe0eca8&oe=60CFE35B

  pic https://scontent-lga3-2.xx.fbcdn.net/v/t1.6435-9/202068480_3987506954617718_2356618113198172776_n.jpg?_nc_cat=103&ccb=1-3&_nc_sid=730e14&_nc_ohc=IElU6XBX1scAX-NXrf0&_nc_ht=scontent-lga3-2.xx&oh=de9fc5d4a0437d7f71e6e36acce585b5&oe=60CFDB3E

  pic https://scontent-lga3-2.xx.fbcdn.net/v/t1.6435-9/201582604_3987507024617711_1217467091595747729_n.jpg?_nc_cat=102&ccb=1-3&_nc_sid=730e14&_nc_ohc=wJHeJL6Jsz0AX9R41iN&_nc_ht=scontent-lga3-2.xx&oh=b59dc8ac83902e807bb93909681ebaa6&oe=60CFE10D

  pic https://scontent-lga3-2.xx.fbcdn.net/v/t1.6435-9/201617435_3987506987951048_547011911837485136_n.jpg?_nc_cat=102&ccb=1-3&_nc_sid=730e14&_nc_ohc=kTcgLnpLSMMAX-ECm7L&_nc_ht=scontent-lga3-2.xx&oh=20fcbe1309616efb6f1d6eab2cc90b6f&oe=60D010A9

  pic https://scontent-lga3-2.xx.fbcdn.net/v/t1.6435-9/202316675_3987507001284380_1172316997065301010_n.jpg?_nc_cat=102&ccb=1-3&_nc_sid=730e14&_nc_ohc=BrPUJaD3MFEAX-zTwaL&_nc_ht=scontent-lga3-2.xx&oh=16b2d49134c3c74eb400fcff9a650dcf&oe=60CF71B2
\fi

\emph{Алексей Бажан}

Почему же, Вирильо очень актуальный автор, из Классика: "Совершенно очевидно,
что в условии сжатия времени до бесконечно малой величины, рендеринг всегда
проиграет сканеру, тому самому, что Вирильо назвал Машиной-Зрения. Вот почему
метод познания мира посредством мифа безнадёжно устарел. Он так и остался
маркером модерна, где форма любого «анализа» – по сути – тот же миф, живущий во
временном континууме, который, никак не поспевает за постоянно меняющимся
реальным временем медийно-сетевой эпохи, оперирующей исключительно пакетами
ссылок, они же координаты клиповых «нарезок», скреплённых временным конструктом
момента. И пока вы не научились конструировать время и не нашли способа
описания постоянно меняющейся со-бытийной энтропии, все ваши выкладки уже не
стоят ровным счётом ничего и представляют собой обыкновенный конспирологический
спам, как у тех же панариных, фурсовых, дугиных и прочих кургинянов несть им
числа.  Что же делать? Как понять какие-то системные вещи, находясь внутри
системы? Гуманитарный космос трёхмерен, и мы никак не можем выйти в четвёртое
измерение, выучив даже все книги Успенского, – что нам остаётся? Правильно, мы
можем попытаться временно разомкнуть систему введением неких дополнительных
элементов, недискурсивных и «нетелесных» по типу математического «мнимого
числа» – со-бытие, обладающее своим внутренним шифром, которое аннигилируется
сразу же по использовании собственного пакета смысловых вариаций."

\emph{Евгения Бильченко}
\textbf{Алексей Бажан} Отчасти - да. Но. Это откровенная апология
темпоральности капитала, который, несмотря на видимую динамику второй
космической скорости, - предельно квазимифологичен и крипторелигиозен. И о
компрессии и событии (нулевой фонеме) еще раннее писали Якобсон, Миллер, Бадью
и Бауман - мысль не нова. Актуальность не всегда значит истинность. Я не
отрицаю Вирильо, но я не хочу абсолютизировать его культ аннигиляции,
пренебрежение к пространству и превращать его в волюнтариста глобализации.
Именно это делают его поклонники - больше, чем сам Вирильо.

\emph{Алексей Бажан}
\textbf{Евгения Бильченко} В контексте - речь о стратегии Партизана.

\emph{Настя Бузиашвили}

Спасибо!! Наглядно и очень интересно

