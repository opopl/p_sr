% vim: keymap=russian-jcukenwin
%%beginhead 
 
%%file 19_01_2022.stz.news.lnr.lug_info.1.spasenie_pozhar
%%parent 19_01_2022
 
%%url https://lug-info.com/news/luganskie-pozharnye-spasli-iz-goryashego-doma-zhenshinu-i-ee-vos-miletnyuyu-doch
 
%%author_id news.lnr.lug_info
%%date 
 
%%tags donbass,lnr,mns_lnr,pozhar,spasenie
%%title Луганские пожарные спасли из горящего дома женщину и ее восьмилетнюю дочь
 
%%endhead 
\subsection{Луганские пожарные спасли из горящего дома женщину и ее восьмилетнюю дочь}
\label{sec:19_01_2022.stz.news.lnr.lug_info.1.spasenie_pozhar}

\Purl{https://lug-info.com/news/luganskie-pozharnye-spasli-iz-goryashego-doma-zhenshinu-i-ee-vos-miletnyuyu-doch}
\ifcmt
 author_begin
   author_id news.lnr.lug_info
 author_end
\fi

Луганские спасатели спасли молодую женщину и ее восьмилетнюю дочь из горящего
дома в поселке Косиора. Об этом сообщила пресс-служба МЧС ЛНР.

Сообщение о загорании пристройки на улице Очаковская поступило на линию 101
Луганска рано утром 19 января. К месту происшествия был направлен дежурный
караул государственной пожарно-спасательной части (ГПСЧ) № 3. Из-за угрозы
распространения огня на жилой дом в помощь были запрошены силы и средства ГПСЧ
№ 1.

\enquote{Хозяева указали, что в доме находятся 2 человека: их 28-летняя дочь и 8-летняя
внучка. Они не смогли покинуть жилище по причине задымленного выхода. Начальник
караула ГПСЧ № 3 Олег Медьевский и командир отделения Александр Пиньков через
оконный проем эвакуировали девушку и ребенка на свежий воздух. Помощь медиков
не понадобилась. Одновременно со спасением производились тушение пожара и поиск
очага возгорания}, - рассказали в МЧС.

В ведомстве добавили, что огнем уничтожены домашние вещи на площади 10 кв. м в
пристройке, причина пожара устанавливается.

Ранее в МЧС ЛНР отмечали, что наиболее частыми причинами пожаров являются
неосторожность при курении, в том числе в состоянии алкогольного опьянения,
неосторожное обращение с огнем и нарушение правил пожарной безопасности при
эксплуатации электроприборов.

