%%beginhead 
 
%%file 15_03_2023.fb.vitko_olena.mariupol.1.r_k_tomu__vs__spogad
%%parent 15_03_2023
 
%%url https://www.facebook.com/olena.vitko.3/posts/pfbid02cXhmwYa7HcZE41RQCeFKTcXkS4LXrkdTrbvQymcSZ9csnWkA3RfTLQevSjhcxL4cl
 
%%author_id vitko_olena.mariupol,kovalchuk_oksana.mariupol
%%date 15_03_2023
 
%%tags mariupol,mariupol.war,text.forward,text.story
%%title Рік тому. Всі спогади, як один страшний сон (Оксана Ковальчук)
 
%%endhead 

\subsection{Рік тому. Всі спогади, як один страшний сон (Оксана Ковальчук)}
\label{sec:15_03_2023.fb.vitko_olena.mariupol.1.r_k_tomu__vs__spogad}
 
\Purl{https://www.facebook.com/olena.vitko.3/posts/pfbid02cXhmwYa7HcZE41RQCeFKTcXkS4LXrkdTrbvQymcSZ9csnWkA3RfTLQevSjhcxL4cl}
\ifcmt
 author_begin
   author_id vitko_olena.mariupol,kovalchuk_oksana.mariupol
 author_end
\fi

Рік тому.

Всі спогади, як один страшний сон.

Вилітають скло та шибки вікон. Все навкруги палає. Наш будинок тепер перший від
дороги, був п'ятим. Поранена такса.

Був приліт осколків в мамину кімнату. Матуся вся в склі, я намагаюся все це
прибрати, другий приліт, вся кімната в осколках, годинник на стіні - вщент, посуд
на столі перед ліжком матусі - вщент, і диво - одна дирочка на боці піалки, вона
ціла, а дирочка збоку... Дивлячись на цю піалку, згадую ватних експердів - \enquote{як це
можливо - стіна зруйнована, а вікна цілісінькі!}, ага!))))

Диво не в піалці, диво, що я з мамою взагалі залишились живими, бо я стояла
навпроти вікна під час прильоту, але про це я не думаю чомусь....

Літаки літають у режимі нон-стоп, бомби кидають близько, будинок здригається.

Я не знаю, що відбувається в інших районах міста, тому здається, що обстрілюють
тільки наш район.

Лєпьошки смажимо під обстрілами - налила тісто, вискочила на двір, поставила на
багаття і назад. Часто двері зачинялися від вибухової хвилі. З якого переляку
ми вирішили, що дерев'яні двері нас врятують, хз...

Паралельної вулиці немає зовсім. 

В сторону Новоазовська ідуть люди. З дітками, стариками, з пустими очами. Це
ті, які втратили все. Чутки про загиблих сприймаються спокійно, без емоцій.

Найстрашніша думка - А хто перший буде у нас?

Молю бога, щоб це була я, і це не героїзм, це трусість, бо я дуже боюся втратити
їх, тому і згодна померти сама.

Без емоцій радимося з чоловіком, де поховаємо бабусю, якщо що.... Мені дуже
тяжко, холодно, страшно. Часто кричу на маму, і це вже не виправити і не
відмолити....

Молодший син, сидячи в погребі сказав - А ведь я ничего не успел! Я так много
хотел, и ничего не успел!.... Чути від нього таку дорослу промову було важко...

Я заздрила подрузі, яка живе на Правому біля річки. Тому що річка - це Вода, і
вона там не закінчується. Я не знаю, що з подругою, але заздрю...

Бажання одне - Тиша! Я дуже хочу ТИШІ!........

\href{https://www.facebook.com/profile.php?id=100070393907205}{Oxana Kovalchuk}

%\ii{15_03_2023.fb.vitko_olena.mariupol.1.r_k_tomu__vs__spogad.cmt}
