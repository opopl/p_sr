% vim: keymap=russian-jcukenwin
%%beginhead 
 
%%file 06_12_2020.news.ua.mykyivregion.1.bojarka_junyy_futbolist
%%parent 06_12_2020
 
%%url https://mykyivregion.com.ua/news/pisov-z-zittya-yunii-futbolist-z-boyarki-foto?fbclid=IwAR0j0rswdJs4yiu1_WoglWOg5YJY4TKRN4EFPwSTroWK5h3Y_9NpDIqQN7I
 
%%author 
%%author_id 
%%author_url 
 
%%tags 
%%title Пішов з життя юний футболіст із Боярки (ФОТО)
 
%%endhead 
 
\subsection{Пішов з життя юний футболіст із Боярки (ФОТО)}
\label{sec:06_12_2020.news.ua.mykyivregion.1.bojarka_junyy_futbolist}
\Purl{https://mykyivregion.com.ua/news/pisov-z-zittya-yunii-futbolist-z-boyarki-foto?fbclid=IwAR0j0rswdJs4yiu1_WoglWOg5YJY4TKRN4EFPwSTroWK5h3Y_9NpDIqQN7I}

2 грудня обірвалося життя молодого боярчанина та перспективного футболіста Артема Лєзніка

Про це повідомляє інформаційний портал \enquote{Моя Київщина} з покликанням на
спільноту \enquote{Києво-Святошинська районна федерація футболу}.\Furl{http://ksrff.com.ua/2017-04-12-12-11-49/2017-04-12-12-12-42/2361-dusha-jogo-na-mit-odnu-rozkvitla?fbclid=IwAR3xSufSvTWtPjBVsTOfAH6YMeq4HBbJFd3El-bGrNYXcP7R01Cn5OVu064}

\ifcmt
tab_begin cols=2
pic https://mykyivregion.com.ua/imagecache/large/img/e855c97d18b6090b635c5f28dc3d601e1607239995.jpg
pic https://mykyivregion.com.ua/uploads/img/59e4e4b1b4435703ec49c29c43a5cdbf1607239991.png
caption Артем Лєзнік (08.07.2005-02.12.2020)
tab_end
\fi

Артем Лєзнік (08.07.2005-02.12.2020) – спортсмен, футболіст-юніор, за ігровим
амплуа – правий захисник, потім – центральний півзахисник, універсал, учасник
дитячо-юнацьких  змагань з футболу та футзалу в чемпіонаті України, Київської
області, Києво-Святошинського району, першості міста Києва. Срібний призер
чемпіонату Київської області, бронзовий призер чемпіонату міста Києва, чемпіон
Києво-Святошинського району в різних вікових категоріях. Визнавався кращим
гравцем дитячо-юнацьких районних турнірів. Вихованець ДЮСШ \enquote{Боярка}. Виступав
за клуби \enquote{ДЮСШ} (Боярка) – 2015-2019, \enquote{Любомир} U-15 (Ставище) – 2019-2020.

Артем народився в місті Боярка Києво-Святошинського району, з дитинства  марив
футболом. У Артема була спортивна родина, батько свого часу серйозно займався
боротьбою і рано прищепив йому любов до спорту, тож з самого дитинства хлопчик
ріс фізично міцним та витривалим. У віці 8-ми років почав займатися дзюдо,
згодом – ходити на футбольну секцію. Пізніше любов до футболу переважила.
Перший тренер – Олександр Дмитрович Зубавленко. Досвідчений спеціаліст
розпізнав у Артема хороші задатки і передав хлопчика в Боярську ДЮСШ, де з
Артемом займалися відомі районні тренери Сергій Бабурін та Ігор Рибіннік.

Артем з малих років проявляв себе в команді справжнім лідером. Він був
фізично міцним хлопчиком, дуже обдарованим від природи, з чудовою
координацією та бійцівським характером. Прийшов до нас, як дзюдоїст, досить
\enquote{сирим} у футбольному розумінні, але за рахунок важкої праці на тренуваннях
швидко наздогнав своїх ровесників. Починав правим захисником, потім
перейшов у середню лінію, а з часом спробував стихію атаки, відчув потяг до
забивання голів. А взагалі був таким собі універсалом, який однаково добре
міг закрити на полі будь-яку позицію. І цим дуже часто виручав нас,
тренерів, коли з різних причин виникали проблеми зі складом. Такого бійця,
який ніколи не пасував перед труднощами, перед сильними суперниками, мати у
складі для будь-якого тренера, особливо у дитячій команді – це велика
удача. Ми завжди ставили Артема в приклад його партнерам по команді, як
зразок \enquote{лицаря без страху і докору}, яким, наприклад, свого часу був Олег
Лужний. До речі, Артем з дитинства вболівав за київське \enquote{Динамо} і був
відданим фанатом цього клубу. Часто ходив на стадіон, щоб підтримати свою
улюблену команду, – згадує тренер Боярської ДЮСШ Ігор Рибіннік.

Нагадаємо, що у Боярці працівники Києво-Святошинського відділу поліції викрили
та задокументували злочинну діяльність
дилера.\Furl{https://mykyivregion.com.ua/news/diler-z-boyarki-prodavav-metadon-foto}
