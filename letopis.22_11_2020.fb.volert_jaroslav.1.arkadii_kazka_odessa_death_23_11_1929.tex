% vim: keymap=russian-jcukenwin
%%beginhead 
 
%%file 22_11_2020.fb.volert_jaroslav.1.arkadii_kazka_odessa_death_23_11_1929
%%parent 22_11_2020
 
%%url https://www.facebook.com/permalink.php?story_fbid=749419768986077&id=100017541812795
 
%%author Волерт, Ярослав
%%author_id volert_jaroslav
%%author_url https://www.facebook.com/profile.php?id=100017541812795
 
%%tags 
%%title 23 листопада 1929 року в Одесі загинув український поет, перекладач, педагог Аркадій Васильович Казка

%%id writers.kazka_arkadii_vasylovych
 
%%endhead 

\subsection{23 листопада 1929 року в Одесі загинув український поет, перекладач, педагог Аркадій Васильович Казка}
\label{sec:22_11_2020.fb.volert_jaroslav.1.arkadii_kazka_odessa_death_23_11_1929}
\Purl{https://www.facebook.com/permalink.php?story_fbid=749419768986077&id=100017541812795}

\index[writers.rus]{Казка, Аркадій Васильович!Український поет, перекладач, педагог (23.09.1890-23.11.1929)}

23 листопада 1929 року в Одесі загинув український поет, перекладач, педагог
Аркадій Васильович Казка.  Доведений слідчим НКВС Григоренком до самогубства
(повісився 23 листопада, за іншими даними --- убитий в камері слідчого відділу
НКВД СРСР).  Аркадій Васильович народився  11 (23, за Григоріанським стилем)
вересня 1890, в м. Седнів, Чернігівської губернії у багатодітній родині

\ifcmt
  author_begin
   author_id volert_jaroslav
  author_end
\fi

сільського шевця, козацького походження. Закінчив Чернігівське реальне училище,
де познайомився з Павлом Тичиною, разом співав у церковному хорі. Відвідував
«суботи» Михайла Коцюбинського. Вступив до таємної молодіжної організації
«Братство самостійників». 

В серпні 1914 р. у Спаському соборі Чернігова, за дорученням Василя
Елланського, прийняв присягу на вірність «Братству»  в юнака-гімназиста Романа
Бжеського. Після закінчення училища Аркадій вступив до Київського комерційного
інституту, але не закінчив навчання через нестатки.  Повернувся до Чернігова,
працював креслярем у земській управі. Брав активну участь у українських
революційних і культурно-просвітницьких організаціях, виступав у концертах і
виставах, організованих «Просвітою». 

За дорученням Чернігівського Комітету охорони памяток старовини малював
історично-мистецькі пам’ятники Чернігівщини. Був близький до кола історика,
генеаолог й мистецтвознавця В.Л. Модзалевського. Згодом учителював у селах
Київщини, Дніпропетровщини, в Одесі (викладав співи та українську мову). Вночі
10 вересня 1929 р. Казку заарештували у сфабрикованій справі Спілки визволення
України, відправили до Одеської в’язниці. 

Аркадій гідно поводився на допитах, нікого не обмовив, спілкувався українською.
Матеріали слідства в архівах були знищені.

Поет був реабілітований 27 листопада 1997 р., коли Чернігівська обласна
прокуратура скасувала постанову одеських чекістів. Ще до 1917 р. перекладав із
західноєвропейських мов, російської літератури. В УСРР власні вірші і поеми
друкував у журналах «Літературно-науковий вісник», «Нова громада», «Плуг». На
початку 1920-х рр. підготував рукописну збірку поезій «Сумливе», яку подарував
Григорію Верьовці. Готував до видання збірки «Розірване намисто» й «Васильки».
Архів Казки, який зберігала його удова, загинув під час Другої світової війни.

\ifcmt
tab_begin cols=2
		caption Аркадій Казка, Український поет, перекладач, педагог (23.09.1890-23.11.1929)
		pic https://scontent-waw1-1.xx.fbcdn.net/v/t1.0-9/127035655_749418912319496_6664662140961379543_n.jpg?_nc_cat=101&ccb=2&_nc_sid=730e14&_nc_ohc=f8-8QqAc_KAAX_6QDLw&_nc_ht=scontent-waw1-1.xx&oh=c406a934129a7aa30a011b5029721e1c&oe=5FEB6F53

		pic https://scontent-waw1-1.xx.fbcdn.net/v/t1.0-9/127048041_749418985652822_1417210299745125187_n.jpg?_nc_cat=109&ccb=2&_nc_sid=730e14&_nc_ohc=o6rg_WJaAqYAX8ohKE8&_nc_ht=scontent-waw1-1.xx&oh=87f0a58b27ee551be506cdcb3d6b2d43&oe=5FEBCD01

		pic https://scontent-waw1-1.xx.fbcdn.net/v/t1.0-9/126939683_749419055652815_6584192226331376423_n.jpg?_nc_cat=101&ccb=2&_nc_sid=730e14&_nc_ohc=12TOzp2sOcAAX9Rlres&_nc_ht=scontent-waw1-1.xx&oh=a829f6e991715b27bae63b75cb92a5de&oe=5FEA4D27

		pic https://scontent-waw1-1.xx.fbcdn.net/v/t1.0-9/127060636_749419125652808_7374133622507627920_n.jpg?_nc_cat=106&ccb=2&_nc_sid=730e14&_nc_ohc=PCKdUi-MIKEAX9Sp7mX&_nc_ht=scontent-waw1-1.xx&oh=1e8c3f20df343d46ff56da9a23c869a9&oe=5FE9C0EB

		pic https://scontent-waw1-1.xx.fbcdn.net/v/t1.0-9/127051318_749419238986130_4510346491682810399_n.jpg?_nc_cat=102&ccb=2&_nc_sid=730e14&_nc_ohc=8zUIjXvG1C8AX9seBdZ&_nc_ht=scontent-waw1-1.xx&oh=6eceb7227b41fa144743153f5af13560&oe=5FE9DD15

		pic https://scontent-waw1-1.xx.fbcdn.net/v/t1.0-9/126940172_749419252319462_4265099129874284086_o.jpg?_nc_cat=106&ccb=2&_nc_sid=730e14&_nc_ohc=huwTeBjnfXAAX9CTtVF&_nc_ht=scontent-waw1-1.xx&oh=7b72c4d198b180353d5787c0f7da03f3&oe=5FEBB22A

		pic https://scontent-waw1-1.xx.fbcdn.net/v/t1.0-9/126954903_749419382319449_5997007091388768444_n.jpg?_nc_cat=100&ccb=2&_nc_sid=730e14&_nc_ohc=gkfMpK5U1fIAX-hswOx&_nc_ht=scontent-waw1-1.xx&oh=b98f50f9e10375aac09c5fa307e2e9f8&oe=5FE8F7E1

		pic https://scontent-waw1-1.xx.fbcdn.net/v/t1.0-9/126939118_749419622319425_2364292791748515240_n.jpg?_nc_cat=107&ccb=2&_nc_sid=730e14&_nc_ohc=TPwWyzTefBIAX9jjZd9&_nc_ht=scontent-waw1-1.xx&oh=af86e546e8d1118bdd3b5e06994dd9a4&oe=5FEAC824

		pic https://scontent-waw1-1.xx.fbcdn.net/v/t1.0-9/127120919_749419618986092_5304189938394362045_n.jpg?_nc_cat=100&ccb=2&_nc_sid=730e14&_nc_ohc=RyY42ioNdkMAX-Ozq6s&_nc_ht=scontent-waw1-1.xx&oh=6402187b2b431a770eee94205546656e&oe=5FE9D451

		pic https://scontent-waw1-1.xx.fbcdn.net/v/t1.0-9/126997696_749419718986082_1580534331501485978_n.jpg?_nc_cat=106&ccb=2&_nc_sid=730e14&_nc_ohc=UWwmithZ7goAX8-TlH9&_nc_ht=scontent-waw1-1.xx&oh=af3d30a4b20e80b038ae8aa6d36f3253&oe=5FEA31A9

		pic https://scontent-waw1-1.xx.fbcdn.net/v/t1.0-9/127035655_749423168985737_6322185379667712833_n.jpg?_nc_cat=106&ccb=2&_nc_sid=730e14&_nc_ohc=-nNhjTgWEx8AX-Nvtcj&_nc_ht=scontent-waw1-1.xx&oh=68193b6cc3f5ea8aab9d78283ddef3d0&oe=5FEB19A0
		caption Пам'ятник Аркадію Казці у Седневі.

tab_end
\fi

\subsubsection{Коментарі}

\paragraph{Лариса Панченко}

\ifcmt
pic https://scontent-waw1-1.xx.fbcdn.net/v/t1.0-9/126899199_397561754990182_690361089327810488_o.jpg?_nc_cat=100&ccb=2&_nc_sid=dbeb18&_nc_ohc=6xaS1Rz_q68AX987m9U&_nc_ht=scontent-waw1-1.xx&oh=2e425deaf3b3df31a5d78e25d3ec7b6d&oe=5FEAF755
\fi

