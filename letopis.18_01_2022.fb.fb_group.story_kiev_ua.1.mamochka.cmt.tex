% vim: keymap=russian-jcukenwin
%%beginhead 
 
%%file 18_01_2022.fb.fb_group.story_kiev_ua.1.mamochka.cmt
%%parent 18_01_2022.fb.fb_group.story_kiev_ua.1.mamochka
 
%%url 
 
%%author_id 
%%date 
 
%%tags 
%%title 
 
%%endhead 
\zzSecCmt

\begin{itemize} % {
\iusr{Лина Агаркова}
Светлая память

\iusr{Владимир Новицкий}
\textbf{Лина Агаркова} Спасибо Лина

\iusr{Виктория Русакова}
Какие лица замечательные!

\begin{itemize} % {
\iusr{Владимир Новицкий}
\textbf{Виктория Русакова} Спасибо Виктория

\iusr{Виктория Русакова}
\textbf{Владимир Новицкий} 

И Вам, Владимир, спасибо огромное за замечательный воспоминания. Моя мама
пережила оккупацию в Киеве, и очень многие ее воспоминания перекликаются с
Вашими. И да, я Ваша землячка по Русановке. Мы переехали на остров в 1965 году.
Моя очень добрая подружка, Ирочка Парасий-Вергуненко Ваша добрая соседка.


\iusr{Владимир Новицкий}
\textbf{Виктория Русакова} Ирочку мы считаем родственницей!!!

\iusr{Ирина Парасій-Вергуненко}
\textbf{Владимир Новицкий} как мир тесен!
\end{itemize} % }

\iusr{Natalya Tarasenko}
Такие воспоминания греют душу. Спасибо Вам...@igg{fbicon.exclamation.mark}

\iusr{Oksana Goranina}
Спасибо Вам огромное!

\iusr{Ассоль Грей}
Господи! Какой трогательный рассказ, наполненный любовью, уважением и благодарностью! Спасибо Вам)

\iusr{Александра Немченко}
Светлая память.

\iusr{Марина Резник}
Интересно, приятно, красиво и очень близко.
Спасибо!!!

\iusr{Владимир Новицкий}
\textbf{Марина Резник} Спасибо Мариночка

\iusr{Ирина Борисенко}

Господи, Владимир, как мне многое знакомо. Правда я родилась в 56 ом, но
прекрасно помню и колонку во дворе Воровского 31а, и кучу угля, и бельё на
улице, и мою бабуленьку, которая перешивала вещи и моя мама ходила самая
модная, и я. Кстати, моя мама закончила 155 школу, но, видимо раньше Вас,
мама 34 года рождения, но может быть Вы её знали Галя Воинова. Серебряная
медаль по окончании школы.

\iusr{Владимир Новицкий}
\textbf{Ирина Борисенко} Спасибо Ирочка. Я в 1956 году закончил 155 школу.

\iusr{Ирина Титова}

Спасибо очень интересно царство им небесное светлая память я всю жизнь прожила
и живу на Воровскогобабушка тоже из дворянской семьи жила на татарке но я
неумею так как вы рассказывать спасибо вам вернули меня в детство

\iusr{Владимир Новицкий}
\textbf{Ирина Титова} Спасибо Ирочка

\iusr{Anna Rymarenko}

Очень нравятся Ваши рассказы! Я росла уже в то время, когда дворов не было- и
уже соседи так не дружили, как описано Вами, сейчас вообще - шестнадцатиэтажный
дом и едва знаешь соседей на площадке- каждый сам по себе, а жаль( . Спасибо за
Ваши воспоминания и ещё за то, с какой любовью вы храните память о родных и
близких людях- есть чему поучиться.

\iusr{Владимир Новицкий}
\textbf{Anna Rymarenko} Спасибо Анна

\iusr{Геннадий Романов}
!!!!!!!!!!!!!!!

\iusr{Leonid Chavoulski}

Спасибо Вам! За рассказ и за память, и за любовь к родителям! За честь киевлян!

\begin{itemize} % {
\iusr{Владимир Новицкий}
\textbf{Leonid Chavoulski} Спасибо Леонид за такой отзыв

\iusr{Margarita Kaminsky}
\textbf{Vladimir Novitsky}

Спасибо Вам за память, за чудесный рассказ о вашей жизни, за любовь, за то, что
Вы навеяли воспоминания о городе, где я родилась и выросла и где меня нет уже
34 года. Будьте здоровы и счастливы ещё долгие-долгие годы и, пожалуйста,
продолжайте писать.

\iusr{Владимир Новицкий}
\textbf{Margarita Kaminsky} Спасибо!!
\end{itemize} % }

\iusr{Инна Валентиновна}

Искрення и добрая статья. Спасибо за ваши воспоминанья, за фото киевлян, ваших
родных. Многое перекликается с судьбами читателей, я думаю. Особенно про
дружбу дворов. Двор - семья - далекая история.

\iusr{Владимир Новицкий}
\textbf{Инна Валентиновна} Спасибо Инна

\iusr{Елена Сидоренко}
Замечательная мама. Светлая ей память, а вашей семье здоровья! @igg{fbicon.heart.beating} 

\iusr{Владимир Новицкий}
\textbf{Елена Сидоренко} Спасибо Леночка

\iusr{Игорь Мацак}

Спасибо Вам большое за рассказ ... Моя мама родилась в 37- м году на
Тургеневской ... Бабушка и погибший в Сталинграде дед были воспитателями в
училище, в районе площади Победы ... Сейчас там ВПТУ связи. Бабушка и мама
много рассказывали о жизни и быте довоенного и послевоенного Киева, о
знаменитом Евбазе, который был в их квартале. Спасибо Вам большое, что
\enquote{окунули} меня в мир воспоминаний о дорогих мне людях ... О бабушке и маме.

\begin{itemize} % {
\iusr{Владимир Новицкий}
\textbf{Игорь Мацак} Я очень рад, что помог Вам окунуться в воспоминания о Ваших близких людях.

\iusr{Игорь Мацак}
\textbf{Владимир Новицкий} Спасибо Вам большое!!!
\end{itemize} % }

\iusr{Svet Lana}
Спасибо за замечательный рассказ. Прочитала на одном дыхании  @igg{fbicon.face.smiling.hearts} 

\begin{itemize} % {
\iusr{Владимир Новицкий}
\textbf{Svet Lana} Спасибо

\iusr{Svet Lana}

А также очень печально и больно осознавать сколько замечательных,
интеллигентных людей было погублено в те годы @igg{fbicon.cry}
@igg{fbicon.face.pleading} 

\iusr{Владимир Новицкий}
\textbf{Svet Lana} Да, очень жаль.
\end{itemize} % }

\iusr{Анна Васичко-Покровская}

Во Франции вы могли и не найти родственников. Если они были живы, когда
началась война, то судьба могла их забросить очень далеко. Как правило
выехавшие, после войны оказывались в Америке, Канаде, Австралии, подальше от
СССР. Таковы были реалии((. Мы искали своих в Белграде, а нашли в Штатах.

\iusr{Елена Полякова}

Вы просто писатель! Вы счастливый человек, что знаете такие подробности о своей
семье. Вы были очень хорошим сыном и пишите о маме с большой любовью. Пусть эта
любовь передастся вашим близким. Будте здоровы и счастливы

\iusr{Владимир Новицкий}
\textbf{Елена Полякова} Спасибо Леночка

\iusr{Gregory Axelrud}
До слез...

\iusr{Ada Goldenko}

Спасибо большое за ваши воспоминания! Как всегда, очень интересно. Я
послевоенный \enquote{ребёнок}, живший в том районе и очень многое помню из рассказов
отца. Ещё раз спасибо!


\iusr{Лариса Артемчук}
Спасибо большое. Очень греет душу

\iusr{Лариса Павловская}
Прекрасный трогательный рассказ. Спасибо

\iusr{Владимир Новицкий}
\textbf{Лариса Павловская} Спасибо Лариса

\iusr{Ирина Одарченко}
Спасибо@igg{fbicon.heart.red}

\iusr{Татьяна Петрова}

Мои бабушка, дедушка, мама и ее сестра жили в оккупированном Киеве, где их
переселяли с ул. Франка, Лысенко, за город, и наконец, на Б. Подвальную
14(Ярославов Вал). Но, к сожалению, я мало знаю подробностей их жизни в
оккупации. А именно их быт. Было голодно и тяжело. Почему-то мало рассказывали.

Дедушка работал где прийдется, хотя в мирное время работал в оперном театре и
русской драме портным, шил костюмы и считался очень хорошим мастером. А как
хорошо, что у вас есть такие тёплые воспоминания о своей маме. Всего вам
доброго и спасибо.


\iusr{Владимир Новицкий}
\textbf{Татьяна Петрова} Спасибо Татьяна

\iusr{Tetyana Zerova-Lyubimova}
Дякую!!!!!!!!!!

\iusr{Владимир Новицкий}
\textbf{Tetyana Zerova-Lyubimova} Дякую i Вам

\iusr{Ольга Дубенко}
Читала не отрываясь, спасибо за ваши воспоминания

\iusr{Елена Мороз}

Благодарю вас, за такой интересный рассказ! Очень люблю читать, или слушать
подобные рассказы, особенно когда описываются места, улицы где провел часть
своей жизни! Будьте здоровы и делитесь своими историями!

\iusr{Владимир Новицкий}
\textbf{Елена Мороз} Спасибо Леночка

\iusr{Valentina Urban}

Спасибо за чудесные неизгладимые в Вашей памяти воспоминания. Каждый Ваш
рассказ частично напоминает мне жизненные истории услышанные мной в детстве от
моих бабушек рожденных и в Киеве в 1905 и 1907 гг., а вот с дедушками мне не
посчастливилось- оба были убиты на войне еще до моего рождения.

Продолжайте писать, Вашим воспоминаниям нет цены. И главное оставайтесь
здоровым!

\iusr{Владимир Новицкий}
\textbf{Valentina Urban} Спасибо Валентина

\iusr{Лена Шклярська}

Замечательные у вас рассказы @igg{fbicon.heart.suit} всегда читаю с большим
удовольствием @igg{fbicon.heart.suit}

Вы свято храните Светлую память о своих близких, за что вам поклон
@igg{fbicon.heart.suit} @igg{fbicon.hands.pray}  @igg{fbicon.face.eyes.star} 

\iusr{Владимир Новицкий}
\textbf{Лена Шклярська} Спасибо Леночка

\iusr{Ольга Кириченко}

Прекрасные рассказы. Читаю с превеликим удовольствием и все пытаюсь представить...

Много, очень много тепла и любви к своей семье, двору, к Киеву положила ваша
мамочка в вашу колыбель. И это все вы смогли пронести через всю жизнь, не
расплескав ни капельки. Спасибо!

\ifcmt
  ig https://scontent-frt3-1.xx.fbcdn.net/v/t39.30808-6/271966280_678208536952267_9128563991347714243_n.jpg?_nc_cat=106&ccb=1-5&_nc_sid=dbeb18&_nc_ohc=_kxmG1PTCwgAX_zNCyb&_nc_ht=scontent-frt3-1.xx&oh=00_AT_F32r9qBmwskF-YKCW_khCJwknh5ZSgqBJO7OjiQPt5w&oe=61EE829B
  @width 0.3
\fi

\iusr{Владимир Новицкий}
\textbf{Ольга Кириченко} Спасибо Ольга

\iusr{Татьяна Морозова}
Спасибо большое @igg{fbicon.heart.red} за прекрасные воспоминания  @igg{fbicon.hands.pray} светлая память Вашей Маме!

\iusr{Владимир Новицкий}
\textbf{Татьяна Морозова} Спасибо Татьяна. Она того стоит.

\iusr{Светлана Недайбида-Бучко}
Прочитала на одном дыхании! Спасибо!!!

\iusr{Yuriy Bubnov}
Спасибо, Владимир! Как это всё знакомо!!!

\iusr{Владимир Новицкий}
\textbf{Yuriy Bubnov} 

Мне очень приятно Юрий, что Вам понравилось, ведь Вы всё это видели своими
глазами. Сколько раз я был в Вашем дворе, который Вы так красочно описали.
Жаль, что тогда мы были незнакомы. Хотя у нас были общие знакомые по Училищу,
которое было рядом с Вашим домом.


\iusr{Мария Бутковская}

Благодарю за ваши такие теплые воспоминания о маме и своей семье, как они
выживали в трудное время. Из ваших воспоминаний, можно написать книгу под
названием \enquote{Жизнь замечательных людей}. Пишите, вспоминайте, а главное -
напечатайте книгу, не большим тиражом, а для своих потомков и друзей !

\begin{itemize} % {
\iusr{Владимир Новицкий}
\textbf{Мария Бутковская} 

Да я написал её для своих потомков. Она из трёх глав. Первая о нашей семье.
Чтобы потомки знали откуда мы, кто были их прадедушки и прабабушки и как они
жили в течении 19 - 20 веков. Вторая глава называется "Мои коммунистические
Университеты, где я описываю как мне приходилось жить. Говорить одно а думать
другое. и третяя глава называется" Америка - которую я открыл, ровно через 500
лет после Колумба. В 1992 году, когда мы приехали в Америку, как раз
праздновали 500 летие ей окрытия Колумбом.

\ifcmt
  ig https://scontent-frt3-1.xx.fbcdn.net/v/t39.30808-6/271663452_4735750869805495_1760313007646598973_n.jpg?_nc_cat=104&ccb=1-5&_nc_sid=dbeb18&_nc_ohc=ap8-IfwjYdAAX9raK32&_nc_ht=scontent-frt3-1.xx&oh=00_AT_KqpCihXgBSD9gd96vyxkhwzZPt8kSJG9WzsdmOsfTEQ&oe=61ED61AB
  @width 0.2
\fi

\end{itemize} % }

\iusr{Нина Косилко}
Спасибо!

\iusr{Мария Воят}
Благодарю за такую чудесную, немного трагическую, но теплую историю @igg{fbicon.face.smiling.eyes.smiling} 

\iusr{Владимир Новицкий}
\textbf{Мария Воят} Спасибо,Мария!!

\iusr{Наталья Сниткина}
Чин и плачу мужественные герои жили в то тяжёлое время, спосибо за рассказ

\iusr{Владимир Новицкий}
\textbf{Наталья Сниткина} Спасибо, но не плачьте, радуйтесь, что сейчас другое время, другая жизнь!!

\iusr{Ирина Хмара}

Вы так интересно всё описали, передали атмосферу. У вас прекрасная семья,
царство небесное и светлая память вашей маме.


\iusr{Владимир Новицкий}
\textbf{Ирина Хмара} Спасибо,Ирина!!

\iusr{Нина Арабская}
Владимир, как всегда - шедевр. Спасибо за рассказ, здоровья Вам и благополучия.

\iusr{Владимир Новицкий}
\textbf{Нина Арабская} Спасибо,Нина!!

\iusr{Неля Архипова}

Интереснейшая и прекрасным языком написанная история вашей семьи и нашего
Киева. Ваша мама- необыкновенная, сильная, смелая предприимчивая женщина,
настоящая мать. Она удачно впитала в себя купеческую жилку своего отца- вашего
деда и сумела не только спасти свою большую семью, взвалив на себя всю
ответственность, но и обеспечить вам вполне достойную жизнь в то тяжелое для
всех время. Отец ваш тоже сумел добиться положения и успеха в жизни. А уж вы-
точно сын своих родителей и внук своего дедушки. Вы очень похожи внешне, а по
деловым качествам всех перещеголяли, добившись высокого положения и
обеспеченной жизни в хорошей, любящей, гармоничной семье, с любимой женой и
детками в советский период нашей жизни. А что гораздо труднее, в постсоветский,
сумев вовремя сориентироваться, эмигрировать и, главное, умело и благополучно
интегрироваться в новую жизнь, в чужой стране, не испугавшись и удачно
преодолев все трудности новой жизни, нового языка, других условий и принципов
жизни. И это уже не в молодом возрасте. Вообщем, я преклоняюсь перед вами и
вашей семьей, обладающей лучшими чертами настоящих людей и заслуживающих самого
большого уважения.

\begin{itemize} % {
\iusr{Владимир Новицкий}
\textbf{Неля Архипова} 

Спасибо Неля за такую высокую оценку. Мне очень приятно, что Вы всё правильно
поняли, видимо Вы читали и предыдущие мои рассказы.
\end{itemize} % }

\iusr{Ирина Фроленко}

Спасибо, что поделились. Это очень личное. И всё-таки самые дорогие воспоминания о Маме

\iusr{Татьяна Зубко Маркина}

Добавлю только большое спасибо за интересный рассказ. Прилетело детство,
родители, родственники. Дворы, их забыть нел зя


\iusr{Татьяна Ховрич}

Супер, спасибо за такой душевный, трогательный рассказ! Мои родители и брат,
также, покоятся на Байковом кладбище, напротив церкви, за склепом. Кстати,
возле могилки моей семьи, несколько могилок с фамилией Новицкая, но они -
\enquote{забронированы}

\begin{itemize} % {
\iusr{Владимир Новицкий}
\textbf{Татьяна Ховрич} Не понимаю,как забронированы??

\iusr{Татьяна Ховрич}
\textbf{Владимир Новицкий} , к сожалению, для будущей продажи: на каждой из 3-х могил - одно и то же фото женщины в берете.

\iusr{Владимир Новицкий}
\textbf{Татьяна Ховрич} Нет это не наши могилки. Наши на 8 участке , сразу за церковью налево

\ifcmt
  ig https://scontent-frx5-2.xx.fbcdn.net/v/t39.30808-6/272044158_4736486863065229_3243016967422347219_n.jpg?_nc_cat=109&ccb=1-5&_nc_sid=dbeb18&_nc_ohc=T9G3D89St2YAX897DAA&_nc_ht=scontent-frx5-2.xx&oh=00_AT-n4nRxEUT7rH72F52CvsuJx-M4HYYs6evCHLHtAmGsrw&oe=61EDCF8D
  @width 0.2
\fi

\iusr{Татьяна Ховрич}
\textbf{Владимир Новицкий} , я знаю, что не Ваших родственников. Могилки моих родителей, я писала выше, напротив церкви, за склепом.
\end{itemize} % }

\iusr{Леся Білоус}
Спасибо большое! Я, прямо, как фильм посмотрела! Замечательно написано.

\iusr{Владимир Новицкий}
\textbf{Леся Білоус} Спасибо,Леся!!

\iusr{Николай Шевчук}

Дякую Вам за чудову, сімейну розповідь. Дочитавши до кінця я зрозумів, що мені
хочеться продовження. Прочитавши цю історію я ніби побував у Вашій сім'ї, у Вас
дома. Мені стало легко і затишно на душі. 62 роки разом. Та Вам потрібно давати
Героя України в сімейному житті. Кажете, що такого звання немає, а повинно
бути. Бажаю Вам з дружиною здоров'я і щастя на довгі літа. Нехай Ваш вогник
щастя освітлює молоді життєвий шлях. Нехай вони вчаться у Вас. Я більше ніж
впевнений, що Ви йдучи по вулиці тримаєтеся за руки. Це мега чудово. От так
тримайтеся, як найдовше. Нехай Ваші діти, як найдовше будуть дітьми!

\begin{itemize} % {
\iusr{Владимир Новицкий}
\textbf{Николай Шевчук} 

Дякую Микола за таку оцiнку мого оповiдання про життя моеi родини. Ви дуже
точно пiдмiтили, що ми досi з дружиною ходимо тримаючись за руки. I дiти в нас
шановливi, бо мають гарнмй приклад. Тепер декiлька слiв про Героя Украiни, дуже
побоююсь, що невдовствi така нагода на превеликий жаль зъявиться, бо ворог на
порозi. Коли 30 рокiв тому ми були у Москвi, щоб виправити вiзу до США, то
опинилися у наших гарних знайомих i сидячи за столом я iм сказав, що, прийде
час i ми будемо воювати, на що вони менi сказали, що я з глузду зiхав. А тепер
вони менi пишуть що я таки мав рацiю. Дай Боже, щоб вiйни не було i нiхто бiлше
не загинув.

\iusr{Николай Шевчук}
\textbf{Vladimir Novitsky} 

дай Боже, щоб мир був. Щоб дітям, онукам, правнукам жилося добре. А ті, що
воюють за владу нехай роздявлять очі і побачать, що робиться з нашим загальним
домом-планетою. Чи може вони думають, що втечуть на іншу планету? Я так не
думаю. Нехай Господь боронить усіх нас від такого лиха. Миру Вам і Вашій
родині.

\end{itemize} % }

\iusr{Татьяна Чута}
Отличные публикации, читаю с большим удовольствием, спасибо  @igg{fbicon.rose} 

\iusr{Владимир Новицкий}
\textbf{Татьяна Чута} Спасибо,Татьяна!!

\iusr{Татьяна Червоношапка}

СПАСИБО!!! Читала на одном дыхании, со слезами на глазах... Мои
дедушка,. бабушка, тётя и папа жили на Воровского и тоже был сквозной
двор, соседи дружные. Дедушка и папа работали на Книжно-журнальной фабрике на
Воровского. Дедушку во время войны забрали немцы и ничего больше о нём не
известно. Бабушка пела в хоре Оперного театра. А родители мамы жили на
ул. Тургеневской и там во дворах тоже были сарайчики с живностью. Помню
магазинчик Босяцкий на углу. Светлая память и царство небесное нашим родным,
которых с нами нет @igg{fbicon.hands.pray}{repeat=3} 

\iusr{Владимир Новицкий}
\textbf{Татьяна Червоношапка} 

Спасибо, Татьяна. Очень рад, что Вы храните память о Ваших родных!!

\iusr{Светлана Чеботарёва}
Светлая память вашей маме!  @igg{fbicon.hands.pray}  А Вас благодарю за рассказ!

\iusr{Владимир Новицкий}
\textbf{Светлана Чеботарёва} Спасибо,Светлана!

\iusr{Елена Ткаченко}
Спасибо большое, за прекрасные воспоминания!

\iusr{Наталья Никифорова}
Всех
Вам
Земных
Благ

\iusr{Владимир Новицкий}
\textbf{Наталья Никифорова} 

Спасибо, Наталья. Мы Вам желаем того же!!

\iusr{Tatiana Thoene}

Ваши воспоминания пробирают до слёз. Это так прекрасно, что Вы можете ими
поделиться и так вдохновенно написать.

\iusr{Владимир Новицкий}
\textbf{Tatiana Thoene} Спасибо Таничка

\iusr{Ирина Беднягина}
Спасибо за историю

\iusr{Алла Маруневич}
Дякую вам

\iusr{Polina Feldman}

Великолепный правдивый рассказ Благодарна Вам когда особенно про Киевские
истории это захватывает  @igg{fbicon.hands.pray}  @igg{fbicon.bouquet}
@igg{fbicon.rainbow} 

\iusr{Наталья Гаевая}

Благословенна женщина, воспитавшая такого сына! Спасибо Вам за трепетную историю Вашей семьи!


\iusr{Владимир Новицкий}
\textbf{Наталья Гаевая} Спасибо,Наталья!!

\iusr{Tina Marsagishvili}

Прочитала на одном дихание. Читаеш про чужую детство, вспоминаешь свое. Спасибо
автору за такую интересную екскурсию в прошлом!


\iusr{Тома Храповицкая}

Такой Светлый образ Мамы в Ваших воспоминаниях! С такой сыновьей любовью Вы с
Нами делитесь! Какая Сила, какое признание в Любви... Божественно...

\iusr{Владимир Новицкий}
\textbf{Тома Храповицкая} Спасибо, Томочка!!

\iusr{Мария Каменцова}

Здорово. Мама чему-то учила... А я сама (хотя мама у меня была всю жизнь)
постигала точно те же самые истины, потому что ей даже в голову не приходило
чему-то нас с братом учить... Могу гордиться, что сделала себя сама – и кстати,
шить тоже научилась самостоятельно, и себя обшивала, и зарабатывала на этом с
12 лет! Ни за чьей спиной никогда не стояла – не было этой спины, и слава Богу!

\begin{itemize} % {
\iusr{Владимир Новицкий}
\textbf{Мария Каменцова} 

Вы молодец, что смогли сами всего добиться. Я тоже всего добивался сам!! Любовь
к маме, это другое!!!

\iusr{Мария Каменцова}
\textbf{Владимир Новицкий} . Конечно, я просто хотела выразить свое восхищение умом и мудростью Вашей мамы! Светлая ей память!

\iusr{Владимир Новицкий}
\textbf{Мария Каменцова} Спасибо Мария
\end{itemize} % }

\iusr{Ирина Архипович}

Обожаю Ваши рассказы!! @igg{fbicon.hand.ok}  @igg{fbicon.thumb.up.yellow} И родительские сразу вспоминаются!! Спасибо!!
@igg{fbicon.heart.red}

\begin{itemize} % {
\iusr{Владимир Новицкий}
\textbf{Ирина Архипович} Спасибо,Ирочка !!!

\iusr{Владимир Новицкий}
\textbf{Ирина Архипович}

\ifcmt
  ig https://scontent-frx5-1.xx.fbcdn.net/v/t39.1997-6/s168x128/47270791_937342239796388_4222599360510164992_n.png?_nc_cat=1&ccb=1-5&_nc_sid=ac3552&_nc_ohc=JKmjAJaZKeUAX9jnXLL&_nc_ht=scontent-frx5-1.xx&oh=00_AT8WyGlGzlrgKwLWJ9iqGN3C7FBbBUuoE3I3GT8ufpLXGg&oe=61EE81B6
  @width 0.1
\fi

\end{itemize} % }

\iusr{Elena Shinovska}
Пишите! У Вас талант!

\begin{itemize} % {
\iusr{Владимир Новицкий}
\textbf{Elena Shinovska} Спасибо, Елена!!

\iusr{Elena Shinovska}
\textbf{Владимир Новицкий} здоровья! Долгих лет жизни в творчестве. Спасибо за внимание.
\end{itemize} % }

\iusr{Ирина Нищимная}

Читаю с первых публикаций, такие душевные, проникновенные, написано с любовью о
родных, можно читать как книгу, перечитывать и учиться у Вас,,, Пусть Господь
Вам даст многие лета в здравии вместе с Лилечкой! Пишите мы вас любим и ждём
продолжения,,,

\begin{itemize} % {
\iusr{Владимир Новицкий}
\textbf{Ирина Нищимная} Спасибо, мы тоже тебя любим Ирочка!!!
\end{itemize} % }

\iusr{Isabella Shakhnovsky}

Прочитала ваши воспоминания с болью и благодарностью, очень многое напомнило
нашу большую и дружную семью, их рассказы о том времени @igg{fbicon.hands.applause.yellow}  @igg{fbicon.face.smiling.hearts}  @igg{fbicon.hands.applause.yellow} Ещё раз большое
спасибо @igg{fbicon.heart.red}

\begin{itemize} % {
\iusr{Владимир Новицкий}
\textbf{Isabella Shakhnovsky} Спасибо, очень рад что Вы помните своих родных,что мои рассказы помогают!!
\end{itemize} % }

\iusr{Наталия Хохлова}

Спасибо за рассказ, такая мудрая и умная была у вас мама, просто восхищаюся, с
вашего рассказа, и себе поучительный урок отметила, пишите по чаще, у вас
талант, ждём следующего рассказа с нетерпением

\iusr{Владимир Новицкий}
\textbf{Наталия Хохлова} Спасибо,Наталья!!

\iusr{Олена Константинова}
Спасибо Вам! Такое уж получила от Вашего рассказа, как будто вместе с Вами всё пережила

\iusr{Владимир Новицкий}
\textbf{Олена Константинова} Благодарю!!

\iusr{Елена Сергеева}

Потрясающие рассказы! Как это интересно - читать и узнавать, какой была жизнь в
нашем городе раньше, от человека, который жил в то время! Спасибо огромное!


\iusr{Владимир Новицкий}
\textbf{Елена Сергеева} Спасибо,Елена!!!

\iusr{Людмила Серая}

Вы напомнили мне нашу послевоенную жизнь. Это большой общий двор, дружные
соседи, помогающие лруг другу. Дети разных национальностей тоже очень дружные,
была совершенно другая жизнь. Ваши воспоминания согрели душу. Спасибо. Здоровья
Вам и вашей семье.

\begin{itemize} % {
\iusr{Владимир Новицкий}
\textbf{Людмила Серая} 

Спасибо, Людмила. Очень рад, что мои рассказы напомнили Вам о друзьях, соседей,
дружбе, которая в то время существовала!!
\end{itemize} % }

\iusr{Lidiya Blasi}

Очень интересная семейная история ! Прочитала с удовольствием ! Уже осталось
мало людей, кто помнит и кто хочет честно рассказать о жизни своей семьи.
Прожив жизнь в сложные времена, Вы все-таки счастливый человек и делитесь этим
счастьем с другими, сохранив благодарность и признательность своей семье !


\iusr{Владимир Новицкий}
\textbf{Lidiya Blasi} Спасибо,Людмила за тёплые слова!!

\iusr{Таня Бобер}

Как же я хотела бы хоть глазком увидеть свои корни,,,,,но я к сожалению видела
только одну бабушку Ито в детстве ели помню,,,,и к сожалению нет ни фото ни
возможности найти....

\begin{itemize} % {
\iusr{Владимир Новицкий}
\textbf{Таня Бобер} Очень жаль,сочувствую!!

\iusr{Таня Бобер}
\textbf{Владимир Новицкий} Спасибо....

\iusr{Таня Бобер}
\textbf{Владимир Новицкий} я извиняюсь, а вам случайно фамилия Бобер ничего не говорит?

\iusr{Владимир Новицкий}
\textbf{Таня Бобер} К сожалению нет. Может напомните почему она мне может о чём то говорить. Время прошло много , Возможно я что то забыл.

\iusr{Таня Бобер}
\textbf{Владимир Новицкий} 

нет, нет, мне просто могло показаться,,,, я с папой когда была маленькая мы всё
время ездили к его другу,,,, мне показалось что это Вы,,,, простите за
беспокойство.

\end{itemize} % }

\iusr{Ольга Почивалова}
Спасибо за рассказ!

\iusr{Илона Уткина}
Очень интересно!)) Спасибо))

\iusr{Alex Frisha}

62 года вместе. Это надо уметь. Поздравляю. Наши предки особенно предвоенного
поколения многое пережили но не все передали нам знаний на эту тему или могли
об этом говорить или рассказывать.

\iusr{Mila Pechenaya}

Прочла на одном дыхании. В памяти увидела наш двор на Подоле. Наши окна
выходили в двор школы \#-10 а под нами был спортивный зал боксеров. А наши
соседи всегда обедали в кухне вместе. Золотое время. А говорят что мы были
бедные-нет мы были самые счастливые и богатые дети 50 и 60 годов у нас было
детство @igg{fbicon.hands.pray} @igg{fbicon.heart.red}
@igg{fbicon.face.full.moon} 

\iusr{Владимир Новицкий}
\textbf{Mila Pechenaya} Спасибо Мила.

\iusr{Нинель Кузницкая}
Владимир, спасибо за такой трогательный рассказ. Как много пришлось пережить
Вашей Маме. И, вообще, людям в ту пору.

\iusr{Владимир Новицкий}
\textbf{Нинель Кузницкая} Спасибо Нинель

\iusr{Наталья Кузнецова}
Какие теплые и нежные бывают воспоминания. Особенно воспоминания детства. Спасибо!

\iusr{Алла Харитонова}
Очень ярко пишете! Продолжайте!!!

\iusr{Римма Майская}
Спасибо большое, за чудесный рассказ. Светлая память Вашей Маме!

\iusr{Владимир Новицкий}
\textbf{Римма Майская} Спасибо Риммочка

\iusr{Анна Джил}
Какое богатство Ваши фото!
Спасибо за них и воспоминания.

\iusr{Владимир Новицкий}
\textbf{Анна Джил} Спасибо Анна

\iusr{Люся Киевская}

Благодарю сердечно ! Как всегда очень интересны Ваши воспоминания ! Восхищаюсь
и удивляюсь вашей памяти ! У Вас несомненно литературный дар ! Будьте здоровы и
благополучны ( вместе со всеми - кого Вы любите) !! С наилучшими пожеланиями,
Л. К.

\begin{itemize} % {
\iusr{Владимир Новицкий}
\textbf{Люся Киевская} Спасибо Люсинька, да ещё Киевская. Мне приятно это слышать от Вас

\iusr{Люся Киевская}
\textbf{Владимир Новицкий} это моя подлинная фамилия ( от деда / прадеда) ...
\end{itemize} % }

\iusr{Ольга Смаль}
Спасибо огромное вам за ваш рассказ. Благодаря вам ожили все эти люди и ваша мама. Растрогали до слез@igg{fbicon.heart.red}.

\iusr{Владимир Новицкий}
\textbf{Ольга Смаль} Спасибо Ольга

\iusr{Кузя Музя}

Нестыковка? Какой же, Ваш папа \enquote{дворянин}, если он был \enquote{сиротой и голодранцем}???


\iusr{Александр Андриевский}

Спасибо Мастер, мой Евбаз похож, но весь после войны, я с 47года, Охмадет, а жили
на Саксаганского 106, почти на углу, где Аптека, дворы такие же, только драки
были у нас часто с \enquote{кугутами} - детьми ГПУ-ков..

\begin{itemize} % {
\iusr{Владимир Новицкий}
\textbf{Александр Андриевский} Тогда это было принято. Улица на улицу.

\iusr{Александр Андриевский}
\textbf{Vladimir Novitsky} 

Евбаз был 'чуток' к 'пришлым' после войны, поэтому у нашегоПоколения была 'чуйка'
на 'чужаков', и мы их били, собираясь целым Кварталом - это не мода, а 'социал,'
чутьё 'не своих' - мы ведь играли в войну, а вы её пережили!!

\end{itemize} % }

\iusr{Троценко Татьяна}
Спасибо, Владимир!! Ждем ваших новых публикаций. Светлая память вашей маме!

\iusr{Владимир Новицкий}
\textbf{Троценко Татьяна} Спасибо Таничка

\iusr{Svetlana Naulko}

Я в потрясении ... истинном потрясении !!! Какая память!!! Фотографии... каждое
фото - отдельная история... хочется рассмотреть все детали, по глазам прочесть
истину...! Несказанно благодарю, Владимир@igg{fbicon.heart.red}{repeat=3}

\begin{itemize} % {
\iusr{Владимир Новицкий}
\textbf{Svetlana Naulko} Спасибо Светлана

\iusr{Svetlana Naulko}
\textbf{Владимир Новицкий}  @igg{fbicon.hands.pray}{repeat=3} 
\end{itemize} % }

\iusr{Оксана Бардош}
Спасибо!!!
Очень душевно!

\iusr{Oksana Voitko}
Дякую!

\iusr{Алекс Скрипач Шевченко}
Здорово....

\iusr{Георгий Майоренко}
Прекрасная киевская история.

\iusr{Владимир Новицкий}
\textbf{Георгий Майоренко} Спасибо Георгий

\iusr{Irina Trisyachna}
@igg{fbicon.heart.red}{repeat=3} Огромное спасибо, за ваши рассказы о прошлом, неоценимый опыт, историю жизни и любви людей друг к другу! Это потрясающе!

\iusr{Владимир Новицкий}
\textbf{Irina Trisyachna} Спасибо Ирочка

\iusr{Наталья Писная}

Благодарю Вас за такие теплые воспоминания о своей семье ! А главное о том
ВРЕМЕНИ, КОТОРОЕ ПЕРЕЖИЛИ НАШИ РОДНЫЕ ! НЕ КАЖДОМУ ДАНО ВОСПРОИЗВЕСТИ ТЕ
СОБЫТИЯ, А У ВАС ЭТО ПРЕКРАСНО ПОЛУЧИЛОСЬ. БУДТО И О МОИХ РОДНЫХ... ОНИ ТОЖЕ
БОЛЬШОЙ СЕМЬЕЙ БАБУШКИ ЖИЛИ В КИЕВЕ...

\iusr{Татьяна Барабаш}
Спасибо за ваш рассказ! Всегда читаю с большим интересом! Светлая память вашей мамочке!

\iusr{Владимир Новицкий}
\textbf{Татьяна Барабаш} Спасибо Татьяна

\iusr{Валентина Булавина}
Замечательные воспоминания

\iusr{Chyzhyshyn Oksana}
Браво!
Чудова розповідь і не менш цікаво розглядати фотографії.

\iusr{Алла Радзивилл}
Прекрасная история!

\iusr{Ирина Парасій-Вергуненко}

Спасибо, Володя за историю очень близкой нам семьи. Спасибо за память, которую
ты сохранил о своих близких,, и за то что ты так душевно делился воспоминаниями
о родном Киеве. Царство небесное тете Клаве и дяде Коле! Вас с Лилей поздравляю
с Крещением Господним! Здоровья вам и Божьего благословения!


\iusr{Владимир Новицкий}
\textbf{Ирина Парасій-Вергуненко} Спасибо Ирочка. И Вас поздравляем с Крещением. Будьте счастливы, здоровы и благополучны. Мы очень любим Вас@igg{fbicon.heart.red}

\iusr{Светлана Пермякова}
Замечательно. Спасибо. Так откровенно о самом дорогом. Это много стоит.

\iusr{Владимир Новицкий}
\textbf{Светлана Пермякова} Спасибо Светлана

\iusr{Исаак Липский}

Читая Ваши воспоминания я вспомнил свое детство и юные годы. Они были такие же.
У нас есть, что вспомнить. Спасибо Володя.


\iusr{Владимир Новицкий}
\textbf{Исаак Липский} Спасибо большое.!!

\iusr{Ольга Долинная}

Очень трогательно! Спасибо! Как важно помнить своих родных, родословную, свои
корни. Мы сейчас живём на Дмитриевской. А Ваш дом 42, сохранился?

\iusr{Марина Иващенко}
Круто!!!!

\iusr{Sasha Ty}
Дуже дякую.
Знайомо, знайомо, що здається з одного двору.
Радіти вам роки і роки.
Цікаво було б побачитись
 @igg{fbicon.smile} 

\iusr{Люба Потемкина}

Замечательный рассказ о жизни моего поколения. А я жила до войны на ул. АРТЕМА
26. Во дворе у нас был Горвоенкомат и с улицы Облвоенкомат. А во дворе был
роскошный, так мне кажется, фруктовый сад и мы детвора играли в разбойников.
Залили по деревьям у каждого было свое дерево. У меня яблоня и на нем так росли
ветки, что получилось кресло. Я разменяла сотый десяток. Вот такая древняя, но
активно живущая, пока.

\begin{itemize} % {
\iusr{Владимир Новицкий}
\textbf{Люба Потемкина} Вы молодец, так держать. Здоровья Вам Люба!!!

\iusr{Svetlana Naulko}
\textbf{Люба Потемкина}  @igg{fbicon.face.smiling.hearts}{repeat=3} 

\iusr{Наталья Гераскина}
\textbf{Люба Потемкина} 

а я жила на Полтавской, а мама работала в Шевченковском военкомате, хорошо знаю
Ваш двор, но фруктового сада не помню. Вообще это мой любимый район, хотя 50
лет уже на Борщаговке.

\iusr{Люба Потемкина}
\textbf{Наталья Гераскина} 

сад был до войны. А мы переехали на Печерск в 1939г. Но несколько лет назад я
была там. В глубине двора есть захудалые деревья, отдаленно напоминающие сад.
Мои дети и внуки сделали мне такой подарок, прогулялисъ со мной по местам моего
далёкого детства. Спасибо всем кто так хорошо пишет о Киеве

\iusr{Елена Сабатовская}
\textbf{Люба Потемкина} Будьте здоровеньки и бодры!

\iusr{Irina Trisyachna}
\textbf{Люба Потемкина} какое счастье! Живите долго! @igg{fbicon.thumb.up.yellow} @igg{fbicon.heart.red} @igg{fbicon.hands.pray} 

\end{itemize} % }

\iusr{Танюша Полино}
Всегда с удовольствием читаю ваши воспоминания

\iusr{Владимир Новицкий}
\textbf{Танюша Полино} Спасибо Танюша

\iusr{Александр Венге}
Спасибо, Владимир. Как всегда. \enquote{От корки до корки}.

\iusr{Владимир Новицкий}
\textbf{Александр Венге} Благодарю Александр

\iusr{Vasyl Parii}
Спасибо!
Очень душевно!

\iusr{Владимир Новицкий}
\textbf{Vasyl Parii} спасибо!!

\iusr{Наталия Баткина}
Благодарю!

\iusr{Владимир Новицкий}
\textbf{Наталия Баткина} Спасибо Вам, Гаталия!!!

\iusr{Лариса Новосёлова}

Спасибо за душевный рассказ, мой папа родился на нижнем валу 7,9 бабушка жила
там с мужем, моим дедушкой ещё до войны, и оккупацию пережила.... Бабушки давно
нет, но кое-что из ее рассказов я немного помню.


\iusr{Валя Билык}

Как хорошо когда человек может так много рассказать о своей жизни, о родителях.
Очень интересно. Спасибо большое. Невольно вспоминаешь и свою жизнь.

\iusr{Наталья Рядковская}
Благодарю за щедрость Владимир! Очень интересно и поучительно читать историю от очевидца.

\iusr{Татьяна Артеменко}
Спасибо

\iusr{Владимир Завялов}
круто!!!спасибо

\iusr{Наталья Гераскина}

Не могу вспомнить, где именно 42 й дом на Дмитриевской хотя я её неплохо знаю.
Училась в 41 школе на Речной, одноклассники жили на Дмитриевской.

\iusr{Наталия Тышкевич}
Жизнь нашей семьи отражается в Ваших рассказах и созвучна Вашим воспоминаниям. Тепло от них.

\iusr{Неонила Вильшановская}

Очень интересно читать ваши воспоминания, которые написаны с такой теплотой.

Мы жили на Дмитриевской 15. Все что вы пишете о жизни двора мне очень знакома
это было именно так, доброта и поддержка друг друга в трудное время.

Мне мама тоже рассказывала как во время войны они с другом на Евбазе у венгров
делали \enquote{гешефты}. Благодарю вас за рассказ. Он вызвал у меня собственные
воспоминания детства.


\iusr{Garik Ukrop}
Наша семья жила на углу Дмитриевской и Викентия Хвойки, соседи. Подол шикарнейший район.

\iusr{Валентина Мірошниченко}

Дякую за розповідь, неначе побувала в дитинстві. Хоч народилась і виросла біля
Центрального стадіона (вул. Куйбишева), але все описане відбувалось і у нашомі
ДВОРІ. Жили дружно, як одна сім'я. Щиро дякую.


\end{itemize} % }
