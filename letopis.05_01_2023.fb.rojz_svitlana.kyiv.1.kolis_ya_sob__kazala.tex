%%beginhead 
 
%%file 05_01_2023.fb.rojz_svitlana.kyiv.1.kolis_ya_sob__kazala
%%parent 05_01_2023
 
%%url https://www.facebook.com/svetlanaroyz/posts/pfbid0BawzvRSM1mQnAwCPGsWYm1eXacc1A4CBwkZTACTUSeiibYLfkffRkBGGMBE9pDmfl
 
%%author_id rojz_svitlana.kyiv
%%date 05_01_2023
 
%%tags 
%%title Колись я собі казала: ось "так" я точно ніколи робити не буду
 
%%endhead 

\subsection{Колись я собі казала: ось \enquote{так} я точно ніколи робити не буду}
\label{sec:05_01_2023.fb.rojz_svitlana.kyiv.1.kolis_ya_sob__kazala}

\Purl{https://www.facebook.com/svetlanaroyz/posts/pfbid0BawzvRSM1mQnAwCPGsWYm1eXacc1A4CBwkZTACTUSeiibYLfkffRkBGGMBE9pDmfl}
\ifcmt
 author_begin
   author_id rojz_svitlana.kyiv
 author_end
\fi

Колись я собі казала: ось \enquote{так} я точно ніколи робити не буду. \enquote{Це} - точно
нижче (важче, тупіше, складніше, зависоке) за мене. Але у життя ж прекрасне
почуття гумору. І через деякий час я себе знаходила в таких самих обставинах,
чи робила теж саме, що я засуджувала в інших, і казала зверхньо: \enquote{Я?! Та
ніколи}. І було цікаво спостерігати, як спочатку я собі казала - ну у мене ж
зовсім інші обставини. Я ж була вимушена, так склалося...  

А потім чесно: так, \enquote{це} я зробила. І знов і знов я ховала міф про свою
ідеальність. І ставала терпимішою до неідеальності інших. Виправляла щось, якщо
це можливо. Брала відповідальність за помилки, якщо це були вони. І жила далі.
І знов помилялася і помиляюся, і буду помилятись. Є вибори та дії - за які мені
досі соромно, яких не виправити. І з якими вже доведеться жити. 

А іноді я дивилась на інших із захватом чи заздрістю і думала: круть! я так
точно не зможу. 

А з часом виявлялось, що я здатна зробити те ж саме, і навіть, більше, ніж
уявляла. 

Ми завжди більше, ніж нам здається. У всіх вимірах. В своїй темряві і своєму
світлі. 

Чи можу я вбити? за ці місяці я знаю, що так. Чи можу я вживати ненормативну
лексику (табу мого мирного часу) - виявилось, що так. 

Чи є у мене цінності, до яких я можу собі дозволити зменшитись, чи навпаки -
розширитись, цінності, які - моя \enquote{несуща стіна}? Сподіваюсь, що є. Я вже писала
на початку війни і часто про це думаю - що в житті поблизу смерті, коли ми
проходимо ініціацію смертю, напевно, ми стаємо більш справжніми, ціннісними. 

Коли щось стається таке, що викличе нашу емоційну реакцію (я зараз не про
загрозу смерті) - між тим, як на ситуацію  зреагує наше тіло, яка буде реакція
на нейромедиатори (саме вони пов'язані з нашими емоціями) і  нашою поведінкою -
буде проміжок часу.

(Ми для цього теж навчаємо дітей "тілесної усвідомленості", знанню де емоції
проявляються в їх тілі, щоб вони швидко могли усвідомити, що з ними
відбувається,  і врегулювати свій стан, скористатись цим проміжком часу).

І саме вибір дій в цей проміжок часу - напевно, і керується нашими цінностями. 

В лекції Роберта Сапольські я зачепилася за слова: \enquote{швидкі рішення  - агресивні}. 

Для виживання агресія саме зараз нормальна та рятівна. Але у нас на цьому етапі
війни - фазі виснаження - може не вистачати сили контролю, щоб відгальмовувати
агресію в відносинах, де немає загрози життю та цілісності, відносинах із
\enquote{своїми}. 

З лекції Сапольські: \enquote{щоб включити емпатію, нам потрібно дати собі
більше часу на рішення}. 

Коли у мене імпульс різко відповісти (родичам, знайомим чи незнайомим) - я
примусово себе відправляю \enquote{в паузу}. Мені, насправді, важливо зберегти
мою чутливість та емпатійність в контакті. 

Ми можемо бути різними і маємо право на різне.  Особливо зараз. 

Але зберегти емпатію та чутливість до своїх - одна із важливих задач
\enquote{із зірочкою}.

А ще знаєте, що мені допомагає? Я питаю себе, з якої ролі я виходжу в контакт
(пишу, реагую, спілкуюсь) 

Це робить і реагує моя \enquote{внутрішня дитина}, і якого віку ця дитина
(малюк чи підліток? І вона вибухає, провокує, дратує в дратується?...)

Чи це реакція моїх внутрішніх батьків  (вони кажуть: \enquote{молодець}, чи
\enquote{айайай, як не соромно}, оцінюють, обезцінюють, чи хвалять та
\enquote{по-батькввські} підтримують) 

Чи це реагує мій внутрішній \enquote{дорослий} - який в контакті з теперішнім,
аналізує, свідомо обирає і свідомо реагує. 

Всі ці ролі важливі і всі мають значення. 

Війна своїми випробуваннями відкидає мене (і багатьох з нас) до реакцій та виборів більш раннього віку. 

Ми зараз можемо не впізнавати себе - в своїй \enquote{темряві} і дивуватись своїй силі
та \enquote{світлу}. Але так хочу, щоб після Перемоги наша Країна була країною
\enquote{Дорослих}. 

Обіймаю, Родино ❤️ як вірю в Перемогу
