% vim: keymap=russian-jcukenwin
%%beginhead 
 
%%file 10_08_2023.stz.news.ua.mrpl.0629.1.ljudmyla_zavalej.qt.2
%%parent 10_08_2023.stz.news.ua.mrpl.0629.1.ljudmyla_zavalej
 
%%url 
 
%%author_id 
%%date 
 
%%tags 
%%title 
 
%%endhead 

\begin{quote}
\em\enquote{Мені дуже пощастило працювати у таких державних підрозділах, які допомагали
дійсно змінювати країну. Коли в 2014-м розпочались всі ці проросійські танці з
бубнами, у податковій був організований новий підрозділ –
інформаційно-довідковий департамент. Ось там я і працювала. Ми займались
практичними консультаціями підприємців – як їм виживати в тих складних і
абсолютно незрозумілих умовах. Реальна допомога реальним людям.

Офіс у нас був у Кальміуській райадміністрації. І я бачила на власні очі, як
над будівлею по декілька разів на день змінюються прапори – то українські, то
\enquote{денеерівські}. Я тоді телефонувала в керуючий  підрозділ, який в той час
знаходився в Донецьку, попереджала про загрозу захоплення, волала про
необхідність вивозити документацію. А мені відповідали там: \enquote{Ну шо ти
кипішуєш. Все нормально}. 

Звісно, ніхто ніякі документи так з Донецька і не вивіз. Хоча можливість була.
Я звільнилась. Після цього мала дуже цікавий досвід роботи в офісі великих
платників податків. Ми працювали з підприємствами, які потрапили в окупацію,
але сплачували зарплати людям у гривні і платили податки в Україні. Цікава була
робота. Поштового зв'язку з окупованими територіями Донецької області вже не
було, а платники податків були, всі документи передавались потайки кур'єрами. 

Після економічної блокади окупованої частини Донеччини, наша робота втратила
сенс}.
\end{quote}
