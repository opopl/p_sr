% vim: keymap=russian-jcukenwin
%%beginhead 
 
%%file 11_08_2022.stz.news.ua.donbas24.1.zlochyny_rosian_proty_kulturn_spadchyny_donbasu.txt
%%parent 11_08_2022.stz.news.ua.donbas24.1.zlochyny_rosian_proty_kulturn_spadchyny_donbasu
 
%%url 
 
%%author_id 
%%date 
 
%%tags 
%%title 
 
%%endhead 

Злочини росіян проти культурної спадщини Донбасу

Знищення і вивіз цінних пам'яток культури та історії є злочином, скоєним
російськими військовими, які спрямовані проти культурної спадщини Донбасу і
загалом України

З початку повномасштабного вторгнення росії в Україну зафіксовано понад 380
випадків пошкоджень та руйнувань російськими військовими українських пам'яток
архітектури, археології, релігії й історії. Зокрема, на Донеччині зруйновано 89
пам'яток культури та архітектури, з них 53 в Маріуполі, понад 60 закладів та
пам'яток культури знищені у Луганській області. Юристи наголошують, що під час
війни, тобто міжнародного збройного конфлікту, між воюючими сторонами діє
міжнародне гуманітарне право, норми якого обмежують сторони конфлікту у засобах
та методах ведення війни. На жаль, ми бачимо жахливі приклади порушення цих
норм російськими військовослужбовцями, військовим та політичним керівництвом.

Якими нормами міжнародного права захищаються культурні цінності під час
воєнного конфлікту?

Важливо, що крім людей (поранені та хворі комбатанти, військовополонені,
цивільне населення), міжнародне гуманітарне право захищає і культурні цінності.
Будь-яке захоплення, знищення чи навмисне пошкодження релігійних, благодійних,
освітніх, мистецьких і наукових установ, історичних пам'яток, творів мистецтва
та науки забороняється та повинно підлягати судовому переслідуванню. Про це
говорить стаття 56 Положення про закони і звичаї війни на суходолі, яке є
додатком до IV Конвенції про закони і звичаї війни на суходолі 1907 року.
Зазначена конвенція є досі чинною та обов'язковою для України та РФ.
Обов'язковими для сторін також є спеціальна Конвенція про захист культурних
цінностей у випадку збройного конфлікту з Протоколом 1954 року (існує ще
протокол 1999 року, до якого приєдналася Україна, але не росія). Більш
загальними, але дуже важливими є окремі положення Додаткового протоколу до
Женевських конвенцій від 12 серпня 1949 року щодо захисту жертв міжнародних
збройних конфліктів від 8 червня 1977 року. Це договори, які стосуються захисту
культурних цінностей в період збройного конфлікту, перш за все міжнародного.
Окрім них, існують звичаєві норми міжнародного гуманітарного права, які також
забезпечують захист культурних цінностей у період збройного конфлікту. Загалом
за нормами міжнародного гуманітарного права культурна спадщина та культурні
цінності у період збройного конфлікту мають підвищений рівень захисту серед
цивільних об'єктів. Сторони конфлікту зобов'язані поважати та захищати їх,
забороняти, попереджати та припиняти будь-які акти крадіжки, грабежу або
незаконного присвоєння культурних цінностей в будь-якій формі, а також будь-які
акти вандалізму. У ході військових операцій сторони повинні вживати особливих
заходів обережності, щоб уникнути заподіяння шкоди культурній спадщині та
культурним цінностям, за винятком випадків, коли вони перетворені на військові
об'єкти. Знищення культурної спадщини та культурних цінностей, яке відбувається
навмисно, кваліфікується міжнародним правом як воєнний злочин. 

Що наразі відомо про воєнні злочини, спрямовані проти культурної спадщини Донбасу?

Очільник Луганської ОВА Сергій Гайдай повідомив, що наразі точно відомо про
знищення на території Луганської області 60 закладів та пам'яток культури.

«Зникає те, що покоління луганчан оберігало і чим всі ми пишалися. Як можна
уявити Сєвєродонецьк без будівлі драматичного театру чи музичного коледжу? Або
Лисичанськ — без бельгійської спадщини? Хоча росіяни, напевно, уявляють, якщо
цілеспрямовано все це руйнують», — наголосив Сергій Гайдай.

За його словами, до початку російського вторгнення у Луганській області було
643 заклади культури. Близько 600 із них зараз перебувають під окупацією. Про
поточний стан багатьох достеменно невідомо. Втім, офіційно зафіксовано повне
руйнування 14 об'єктів. Ще 47 мають серйозні пошкодження. Більшість із них
зруйновані на 70−90%.

У Нью-Йорку на Донеччині пошкоджена обстрілами і пожежею історична пам'ятка,
яка пережила дві світові війни. Це чотириповерховий паровий млин Петера Діка,
який був побудований на Донбасі ще у 1903 році. Третю війну цей німецький
спадок сходу України, на жаль, не пережив.

Втім найболючішою точкою в Донецькій області залишається Маріуполь, де
зруйнована найбільша кількість архітектурних та культурних пам'яток. Зокрема,
зазнала руйнації найвідоміша церква міста — Собор Архистратига Божого Михаїла,
що розташована на вулиці Воїнів-Визволителів у Лівобережному районі міста,
повністю знищений пам'ятник митрополиту Ігнатію, що був розміщений біля собору.
Зруйновано і палаци культури міста — ПК «Український дім» та ПК «Іскра» у
Кальміуському районі. Потрапила під обстріл і Маріупольська мечеть в честь
султана Сулеймана Чудового і Роксолани в Приморському районі.

Також сильно постраждали найстаріші вулиці міста, які ведуть свою історію від
дня заснування міста. Зокрема, на центральній вулиці міста — проспекті Миру —
зруйновано архітектурне серце міста — будівлю Донецького академічного обласного
драматичного театру (м. Маріуполь). Історія будівництва непроста, адже театри
за радянських часів зводили лише в обласних центрах. Будівля театру будувалася
протягом 1956−1960 років. Гарне, добре обладнане приміщення для театру в стилі
радянського монументального класицизму з великою кількістю ліпних декоративних
елементів було прикрасою міста. Під дахом театру зібралось більше 1000 людей,
які не мали більше дому і змушені були шукати інше безпечне місце. Вони спали
на сцені, в гримерках, фойє, а гардероб перетворився на їдальню, а актори, які
раніше розважали глядачів, стали волонтерами та допомагали всім вижити.
Окупанти завдали удару по драмтеатру попри те, що поряд був цей велетенський
напис «ДІТИ». Загинуло близько 300 осіб. Важливо відзначити, що приміщення
драматичного театру було єдиним у Маріуполі, що отримало статус пам'ятки
архітектури державного значення і одним з 6 пам'яток архітектури місцевого
значення. Маріупольський театр став символом гуманітарної катастрофи,
заподіяної росією. Відновити будівлю після закінчення війни та деокупації
пообіцяв уряд Італії.

Ще однією з цінних пам'яток архітектури в Маріуполі, яка була повністю
зруйнована російськими окупантами, є будівля Центру сучасного мистецтва «Готель
Континенталь» (Палац культури «Молодіжний»), розташована на найстарішій вулиці
міста — Харлампіївській. Спочатку будівля, яка по праву вважалася архітектурною
окрасою міста, була збудована як готель «Континенталь». Будівництво завершили
1897 р., на жаль, автор проєкту невідомий, документи втрачені. У першокласному
готелі «Континенталь» проводили свій вільний час маріупольське купецтво та
місцеве міщанство. Господарем особняка був В. Томазо.

Важливо, що, як і в приміщенні драмтеатру, в Центрі сучасного мистецтва «Готель
Континенталь», в якому до війни працювали театральні та танцювальні колективи,
не було військових, це були місця для мирних жителів Маріуполя. Маріупольці
спали у старих підвалах, гримерках, танцювальних залах та холах. Їжу готували
на подвір'ї. А окупанти, незважаючи на написи «Діти» і розуміння, що в будівлях
перебувають сотні мирних мешканців, бомбили ці цінні архітектурні будівлі.

На вулиці Земській — одній з найстаріших вулиць Маріуполя — отримав значні
пошкодження єдиний неоготичний замок Маріуполя — будинок Гампера. Невеликий
будинок у стилі неоготики, побудований у 1907 році із червоної цегли,
розташований в районі Слобідки. Відмінні риси архітектури: стрілчасті вікна,
цегляні візерунки та прибудова у вигляді вежі середньовічного замку. На рубежі
XIX — XX ст. тут жив і працював відомий маріупольський лікар Сергій Федорович
Гампер, який мав велику популярність і серед жителів міста. Будівля схожа на
замок, чим привертає увагу до себе. За своє відносно коротке життя Сергій
Гампер дуже багато зробив для Маріуполя як високоосвічений лікар, і як
громадський діяч. 

Було зруйновано і Маріупольський краєзнавчий музей, заснований 6 лютого 1920
року і розташований на вулиці Георгіївській. У фондах музею зберігалося понад
53 000 музейних предметів, зокрема речові, образотворчі, письмові (рукописні та
друковані), нумізматичні, археологічні, фотодокументальні, природні та інші.

На жаль, було зруйновано і будівлю Художнього музею імені Архипа Куїнджі. Цей
старовинний особняк 1902 року, виконаний у стилі північний модерн був весільним
подарунком голови Маріупольської земської управи Газадінова з нагоди одруження
його дочки Валентини з Гіацинтовим Василем Івановичем, засновником
Маріупольського реального училища. Колекція Художнього музею ім. Куїнджі
налічувала близько 2 тисяч творів живопису, графіки, декоративно-ужиткового
мистецтва та скульптури. Серед них — роботи Івана Айвазовського, Миколи
Глущенка, Тетяни Яблонської, Михайла Дерегуса, Андрія Коцки, Миколи Бендрика,
Леоніда Гаді.

Внаслідок масових обстрілів було зруйновано Музей історії та археології
Маріупольського державного університету, який знаходився на другому поверсі
ВНЗ. У ньому зберігалося близько 500 цінних артефактів. Відомо, що окупанти
повністю вивезли всі цінні експонати з Маріупольського краєзнавчого музею та
Художнього музею ім. Куїнджі. В Маріуполі зберігалось сім оригінальних
шедеврів, що були передані росіянам директоркою Маріупольського краєзнавчого
музею Наталею Капустніковою, яка знала точне місце таємного зберігання шедеврів
та особисто все передала окупантам. Зокрема, вона віддала оригінали трьох
картин Куїнджі «Червоний захід», «Осінь», «Ельбрус»; оригінал картини
Айвазовського «Біля берегів Кавказу»; два оригінали картини Дубовського —
«Море» і «Ніч на Балтійському морі»; оригінал Калмикова «А.І.Куїнджі». А також
вона передала бюст Куїнджі роботи скульптора Беклемішева та три унікальні ікони
— Ісус Христос Вседержитель, Богоматір з Немовлям; Іоан Хреститель. Водночас
російські окупанти вивезли з Маріуполя до Донецька унікальну колекцію
медальєрного мистецтва Юхима Харабета, чоловіка єдиної народної артистки
України в Маріуполі Світлани Отченашенко-Харабет.

Чи можлива конпенсація та відновлення культурної спадщини Донбасу?

Основним засобом компенсації шкоди, заподіяної культурній спадщині та
культурним цінностям внаслідок порушення міжнародних норм у період збройного
конфлікту, є реституція. Реституція — це, перш за все, повернення культурних
цінностей або передача замість них аналогічних за вартістю та цінністю
об'єктів. Зобов'язання щодо реституції — це не тільки відповідні норми мирних
договорів. Це ще й положення Конвенцій ЮНЕСКО, відтворені в резолюціях
Генеральної асамблеї ООН, та звичаєва норма міжнародного права. Завданням
реституції є не просто повернути окремі культурні цінності та компенсувати
заподіяну шкоду, але й відновити спадщину як цілісне культурне надбання нації
та держави. Найчастіше реституція відбувається саме у формі повернення
вивезених у період окупації культурних цінностей, як це мало місце на основі
мирних договорів після Першої та Другої світових воєн.

Друга форма реституції пов'язана із заміною культурних цінностей на аналогічні
об'єкти, втрачені у результаті бойових дій. Зокрема, 21 березня 2022 року
прямим попаданням авіабомби російських окупантів було знищено будівлю
Художнього музею імені Архипа Куїнджі у Маріуполі. Відповідальність за це несе
держава-агресор. Саме тому рф повинна буде передати Україні аналогічні
втраченим шедеври зі своїх колекцій та відшкодувати вартість зруйнованого
приміщення музею або побудувати нове.

Наразі Міністерство культури України веде перемовини з іноземними колегами та
послами щодо фінансової допомоги у відновленні культурних пам'яток. Повідомити
про пошкоджений пам'ятник чи культурну споруду можна на сторінці, адже знищення
і вивіз цінних пам'яток культури та історії є злочином, скоєним російськими
військовими, які спрямовані проти культурної спадщини Донбасу і загалом
України. Відновлення культурної спадщини та повернення культурних цінностей
стане важливим завданням держави та суспільства після перемоги України.

Нагадаємо, раніше Донбас24 розповідав про Євгена Целіка, який пройшов пішки з
Києва до Запоріжжя, щоб зібрати кошти на ЗСУ.

ФОТО: з відкритих джерел
