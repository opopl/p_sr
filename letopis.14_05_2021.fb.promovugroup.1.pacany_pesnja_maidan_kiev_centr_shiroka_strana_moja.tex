% vim: keymap=russian-jcukenwin
%%beginhead 
 
%%file 14_05_2021.fb.promovugroup.1.pacany_pesnja_maidan_kiev_centr_shiroka_strana_moja
%%parent 14_05_2021
 
%%url https://www.facebook.com/groups/promovugroup/permalink/971333576773837/
 
%%author 
%%author_id 
%%author_url 
 
%%tags 
%%title 
 
%%endhead 

\subsection{Майдан - Киев - Пацаны - Гитара - Широка страна моя родная}
\label{sec:14_05_2021.fb.promovugroup.1.pacany_pesnja_maidan_kiev_centr_shiroka_strana_moja}
\Purl{https://www.facebook.com/groups/promovugroup/permalink/971333576773837/}

У переході на Майдані Незалежності троє пацанів років 15 під гітарку  горлають
\enquote{Широка страна моя родная}. Один в штанах для фітнесу, другий із зеленим довгим
волоссям, третій зі стрижкою marine - жоден би не пройшов фейс-контроль в
\enquote{родной стране}.

На словах \enquote{Я другой такой страны не знаю, где так вольно дышит человек} до них
підходить старший чоловік доволі фольклорного вигляду, сивий, з вусами, і каже:

- Ви не знаєте, про що співаєте.  Ви ж юні діти.

Спокійно так каже. 

\enquote{Морпіх} відповідає запально:

- Зачем нам ваш опыт? Вам просто не повезло по жизни. У других был отличный опыт жизни в союзе.  

Так, у багатьох був опыт зашибись, особливо у роки створення цього пісенного
шедевру радянського лицемірства. Це 1936, щоб вам не шукати.

Світлана Кириченко, учасниця українського визвольного руху 60-80-х років у
книжці спогадів \enquote{Люди не зі страху} пише, що їй  довелося працювати під
керівництвом завідувача відділу Інституту літератури,  якого називали
\enquote{флюгером}, і він знав про це прізвисько. 

Не заперечував і видимо не ображався, а говорив: \enquote{Згоден, флюгер. А ви знаєте,
що то таке - місяць за місяцем чекати арешту? Щовечора кладе біля себе клунок з
білизною та сухарями і навіть до ліжка не можеш лягти - легше сидячи біля
дверей. Вслухаєшся: ось скреготнули перед будинком гальма, гупають  чобітьми по
сходах. На якому поверсі, біля чиїх дверей зупиняться? Коли сходить сонце,
падаєш мертвим на постіль}.

Це все доступне, це все можна прочитати. 

Але романтизація СССР  -  \enquote{країни достатку і усього  безкоштовного}, як казав
мені колись один таксист (26 років, уточнила навіть, бо така була впевненість)
- просто таки процвітає. 

І ця фальшива романтика справді  дуже добре сприймається юними, з їхнім
запитом, наприклад, на справедливість і потребою в життєвому шансі.

Але шкільна освіта не встигає  за мас-культурою і російською  пропагандою. І навіть у Києві. 

Батьки і старші члени родин далеко не завжди  здатні критично  проаналізувати
своє життя для самих себе, що вже й казати про те, щоб адекватно і цікаво
розповісти дітям чи онукам.

А життя - це \enquote{опыт}, так. Репресії, дефіцит, блат, хамство і приниження, цензура, \enquote{заповідники} і \enquote{гетто}. 

Цей досвід йшов у пакеті з \enquote{усім безкоштовним} (насправді ні) в соціалістичному
\enquote{раю}. Де найсмачніше у світі ескімо.

Сумно, що історичну амнезію у нас називають аполітичністю.

Vadym Vinytskyi

Нещодавно на вході до м.Контрактова площа в Києві підлітки гугняво, безмисельно
і тупо горлали Цоя, хотілось дати їм грошей, аби замовкли... Явно народжені
значно пізніше піку популярності Айзеншпісового вундеркінда... Але ж їм норм і
зараз!

Василь Оленчук

Мені 63 і в совок не хочу.

Олег Сіренко

Спитати у того типчика, де він працює і на відповідь \enquote{временно не работаю}
розповісти, скільки б йому дали у ті \enquote{щасливі} часи.
