% vim: keymap=russian-jcukenwin
%%beginhead 
 
%%file slova.aktivist
%%parent slova
 
%%url 
 
%%author 
%%author_id 
%%author_url 
 
%%tags 
%%title 
 
%%endhead 
\chapter{Активист}
\label{sec:slova.aktivist}

%%%cit
%%%cit_head
Националисты начали травлю Девушки
%%%cit_pic
%%%cit_text
Отметим, что в Украине нет возможности получать образование на русском языке. В
рамках политики украинизации все ВУЗы переведены на украинский язык. В том
числе и в русскоязычных регионах. В ответ на видео девушки националисты начали
травлю девушки. \enquote{От имени нашего сообщества призываем все компетентные
органы, СМИ, \emph{активистов} и неравнодушных граждан немедленно отреагировать
на эти антиукраинские заявления Маргариты}, - сказано в сообщении одной из
групп \enquote{мовных \emph{активистов}}
%%%cit_comment
%%%cit_title
\citTitle{Студентка негативно высказалась об украинском языке}, , kharkov.strana.ua, 16.06.2021
%%%endcit

%%%cit
%%%cit_head
Активисты не любят, когда Чернобыль называют Чернобыль
%%%cit_pic
%%%cit_text
Недавно украинская студия-разработчик компьютерных игр GSC Game World
представила новый трейлер популярной игры S.T.A.L.K.E.R. 2 с подзаголовком
Heart of Chernobyl (\enquote{Сердце Чернобыля}).  Видеоигра выйдет только в 2022 году.
Но националисты уже пытаются сорвать запущенную рекламную кампанию, обвиняя
разработчиков в недостаточной любви к украинскому языку.  Ряд радикально
настроенных \emph{активистов} в соцсетях возмущаются официальному трейлеру игры на
русском языке и транслитерацией на английский. Они запустили в Twitter свой
флешмоб с хэштегом \verb|#ChornobylNotChernobyl| и предлагают запретить выпуск
франшизы.  \enquote{Страна} разбиралась в тонкостях очередной \enquote{зрады} на языковой
почве
%%%cit_comment
%%%cit_title
\citTitle{Сталкер 2 Сердце Чернобыля - почему хейтят видеоигру украинских разработчиков}, 
Екатерина Терехова, strana.ua, 16.06.2021
%%%endcit


%%%cit
%%%cit_head
%%%cit_pic
\ifcmt
  pic https://strana.ua/img/forall/u/11/2/%D0%BA%D0%BE%D1%80%D0%B0%D0%B1%D0%BB%D1%8C_%D1%81%D0%B0%D0%BF%D0%B0%D1%82%D0%B8%D1%81%D1%82%D0%BE%D0%B2.jpg
	caption Судно, на котором приплыли сапатисты в Европу. Фото Amehd Coca Castillo, eulixe.com
\fi
%%%cit_text
Лозунг \enquote{Вторжение началось} \emph{активисты} опубликовали на 41 языке, включая
русский, украинский и азбуку Морзе.  \enquote{Мы скажем народу Испании две простые
вещи. Первое - они нас не покорили. Мы продолжаем сопротивляться и бунтовать.
Второе - они не должны просить прощения}, - говорится в заявлении сапатистов.
Среди бунтарей, которые пересекли Атлантический океан, семь человек в возрасте
от 19 до 57 лет: четыре женщины, двое мужчин и один трансгендер по имени
Марихосе. Как пошутил один из сапатистов, их девиз: \enquote{Сдавайтесь,
гетеро-патриархальные бледнолицые!} (хотя никто такого по прибытии не
восклицал, как ошибочно написали некоторые СМИ).  Остальные прилетят самолётами
и пробудут в Европе несколько месяцев
%%%cit_comment
%%%cit_title
\citTitle{Зачем в Испанию приплыли революционеры из Мексики}, 
Александра Харченко, strana.ua, 27.06.2021
%%%endcit

%%%cit
%%%cit_head
%%%cit_pic

\ifcmt
  tab_begin cols=2

     pic https://strana.ua/img/forall/u/10/79/%D0%B0%D0%B2%D1%823(1).png

     pic https://strana.ua/img/forall/u/10/79/%D0%B0%D0%B2%D1%82(6).jpg

  tab_end
\fi

%%%cit_text
По его словам, провожать Воробьева пришли также представители посольства
России.  \enquote{А рядом как раз случайно проходили люди всякие, неравнодушные к
судьбе страны. И как раз с собой совершенно случайно у них оказались дымовые
шашки и пару ведер с говном. Поэтому по уже традиции .... какое же
Россотрудничество без говна?! На машинах посольства оказались эти несколько
ведер аутентичного украинского продукта, который четко характеризует ту
деятельность, которую здесь так долго вела эта организация и ее руководитель},
- пишет \emph{активист} и добавляет, что \enquote{автомобилям так стыдно стало, что у них
спустились шины и отпали номера}
%%%cit_comment
%%%cit_title
\citTitle{Активисты вылили нескодько ведер фекалий на автомобили представителей российского посольства}, 
, strana.ua, 27.09.2018
%%%endcit

%%%cit
%%%cit_head
%%%cit_pic
%%%cit_text
Патриотические \emph{«активисты»} находятся постоянном поиске новых врагов
народа.  Украинская полиция завела уголовное дело на пятерых юных девушек, в
возрасте от восемнадцати лет до двадцати одного года. Одна из них отмечала свой
день рождения, и устроила с подружками что-то вроде уличной вечеринки. Молодые
студентки танцевали возле львовского мемориала «Небесной сотни», под звуки
вражеского российского рэпа — и это показалось публике кощунственным глумлением
над святыней.  Бдительные патриоты увидели запись праздника в соцсети, и
донесли об этом в полицию, которая оперативно возбудила дело по статье
«Осквернение памятника».  Подросткам грозит тюремное заключение на срок до пяти
лет, а пока их безнаказанно травят ультраправые \emph{«активисты»}. Девушки удаляют
свои аккаунты, потому что их обзывают самыми оскорбительными словами
%%%cit_comment
%%%cit_title
\citTitle{«Вражеский русский рэп». Львовянок травят за танцы возле мемориала «Небесной сотне»}, 
Андрей Манчук, ukraina.ru, 08.07.2021
%%%endcit

%%%cit
%%%cit_head
%%%cit_pic
\ifcmt
  pic https://odnarodyna.org/sites/default/files/styles/adaptive/public/article/2021/ds07072100.jpg?itok=lM7q--Ip
	width 0.4
\fi
%%%cit_text
Вскормленное войной и безнаказанностью уже почти полумиллионное стадо
«киборгов», «промбергов», «героев АТО» и «ветеранов ООС» в последнее время
необычно напряглось и готово продемонстрировать свою готовность сбросить
криворожского выскочку с гетманской табуретки. Причина – нарушение державой
взятых на себя гарантий по бесконечному обслуживанию и удовлетворению
потребностей тех, кто отправился убивать на Донбасс. Как заявил один из
\emph{активистов} так называемого ветеранского движения Андрей Рева: «Нас, тех
кто выжил в Зеленополье, тех кто мёрз в ДАПе, кто зачищал Авдеевку, нагло
обсасывают, причём во всём!»
%%%cit_comment
%%%cit_title
\citTitle{Бабахнет ли ветеран-майдан? — Одна Родина}, , odnarodyna.org,
07.07.2021
%%%endcit

%%%cit
%%%cit_head
%%%cit_pic
%%%cit_text
При всем при том по итогам двух Майданов у нас возникло деятельное сообщество
«\emph{активистов транснационалов}», которые фактически являются корыстными
посредниками между властью Украины и Западом, преследующим свои интересы на
нашей территории. И эти \emph{активисты} не заморачиваются новыми моделями власти, а
входят в старую, используя одноразовые политические проекты вроде «Самопомочи
или «Голоса» (на подходе еще подобные им).
Почему так происходит? Почему при наличие перспективных концепций, нацеленных в
будущее, их разработчики и носители не становятся победительными
общественно-политическими лидерами? И не получают возможности для претворения
этих идей в жизнь? И это при удивительной жертвенной способности украинцев к
самоорганизации для избавления от обрыдлых правителей или защиты своей земли от
агрессора!
%%%cit_comment
%%%cit_title
\citTitle{В Украине была только одна настоящая революция - в 1991-м}, 
Александр Кочетков, strana.ua, 16.07.2021
%%%endcit

%%%cit
%%%cit_head
%%%cit_pic
%%%cit_text
Во Львове спорят о том, что будет вместо монумента. \emph{"Активисты"} хотели
поставить там мемориал украинским военным, а местные предприниматели хотели
затеять там стройку.  Официально на этом месте запланировали создать сквер
памяти героев Украины, обустроить подземный паркинг, велодорожку. Но хотя
слухи, что на самом деле здесь могут начать строительство
%%%cit_comment
%%%cit_title
\citTitle{Без Родины. Как и зачем во Львове снесли Монумент советским воинам}, 
Александра Харченко, strana.ua, 23.07.2021
%%%endcit


%%%cit
%%%cit_head
%%%cit_pic
%%%cit_text
На львовском Холме Славы окончательно демонтировали монумент в честь воинов
Красной Армии, освободивших Украину от гитлеровцев. Уничтожением мемориального
комплекса совместно занимались местные власти, либеральная общественность и
нацисты. Однако памятник был большим, строился капитально. И его разрушали
долго, расчленяли по частям, постепенно снимая статуи, отбивая бронзовые буквы
и выламывая монументальные барельефы.  Каждый раз это становилось поводом для
глумления и пиара. Ультраправые \emph{активисты} селфились на фоне поверженных
гигантов – солдат, матросов и партизан, женщин и мужчин. Дольше всего
продержалась статуя Родины-матери с протянутыми вперед руками. Но этот
просительный жест не помог. Львовскую Родину-мать снесли с помощью автокрана,
накинув ей на шею веревку. И в этом нельзя не видеть символизм – учитывая, что
возле разрушенного памятника лежат деды многих современных украинских
погромщиков
%%%cit_comment
%%%cit_title
\citTitle{Украинская идеология годится только для концлагеря}, 
Андрей Манчук, vz.ru, 26.07.2021
%%%endcit
