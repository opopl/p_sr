% vim: keymap=russian-jcukenwin
%%beginhead 
 
%%file 15_12_2021.fb.bilchenko_evgenia.1.po_sledam_karmy.cmt
%%parent 15_12_2021.fb.bilchenko_evgenia.1.po_sledam_karmy
 
%%url 
 
%%author_id 
%%date 
 
%%tags 
%%title 
 
%%endhead 
\zzSecCmt

\begin{itemize} % {
\iusr{Светлана Пикта}

Офф мысли после мыслей о прочитанном. Женя, у нас с тобой уникальная
способность вляпываться в истории. То что история нас выбирает, это хорошо, но
я со страхом думаю, что будет, когда мы, наконец-то, встретимся. )

\iusr{Евгения Бильченко}
\textbf{Светлана Пикта} 

Таких встреч, как будет у нас с тобой, в моей жизни было... Сейчас посчитаю...
Саня, Захар, Миша, Гера...Пятая. для сжатия руки.

\iusr{Vera Romanova}

Женя, давят тебя сильные враги и походу по морскому закону-чем быстрее и
радикальнее, тем эффективнее. А и хрен на них. Можно даже перчатку дуэльную туда
забросить и вступить в открытую публичную культурологическую полемику. Мордобой
так мордобой. Помнишь, в личку кидала тебе инфу о военных мемуарах Ивана
Иосифовича Горбатко. Впервые опубликованных в журнале \enquote{Москва} на заре 80-х. Дед
этот был гвардии старшина. Полковника получил позже. Двух маршалов не
опубликовали, а его-да. \enquote{Война глазами солдата}-формат \enquote{букварь войны}. Писал для
курсантов. 1026 гвардейский полк и ни одной неоправданной потери за весь период
войны. Сверх того еще 15 книг. Попалась статья о профессоре Кембриджа Кетрин
Мерридейл. Специалист по истории русской революции. В России акунинские премии
получает и регулярно бывает. Кроме прочего и о войне она написала книжицу под
названием \enquote{Война глазами пехотного Ваньки} (в иных переводах-\enquote{Война глазами
Ивана}). Я ей написала-передала физкультпривет (посмертный) от того самого
пехотного Ваньки, написавшего мемуары очевидца (Курск, Сталинград, западная
Украина, Белоруссия ) и спросила, есть ли у барышни желание хотя бы ознакомиться
с текстом. Ответ оказался быстрым и положительным. Ибо барышня у такой штуке
оказалась не готова. Это я к чему... Женя, ты боец, бери бриттов за жабры и
вызывай их на публичные диспуты по наболевшим темам. В том числе профессуру
Оксфордов-Кембриджей. Чем публичнее-тем лучше. Статус и мозги позволяют тебе
пободаться и отстоять себя. Ну и наши козлы зассут тебя прессовать и травить
после этого.

\begin{itemize} % {
\iusr{Евгения Бильченко}
\textbf{Vera Romanova} 

Нет, ты не совсем поняла, как работает либеральная цензура по всему миру. К ней
вообще неприменимо слово \enquote{давление}: это же не командно-административная
структура государства. Люди просто не хотят упорно понимать, что такое
сентиментальное насилие либерализма. Писала, писала, читала, читала, все равно.
Никакого давления, hard force не используется. Используется символический
капитал - Habitus. Если ты не русофобишь, как истинный интеллектуал, ты - не
истинный интеллектуал. Смарт-сила вытесняет тебя из замкнутой на себя
рукопожатной тусовки статусных учёных, и ты тут же теряешь статус, позволяющий
тебе не то, что пригласить кого-то на полемику, а даже обратиться к нему. Ты
становишься никем, или \enquote{чекистом}. То есть, лишаешься символического капитала.
За тобой тянутся шлейфы: \enquote{она ненавидит Украину - светоч демократии}, \enquote{она
травмат} и так далее. При этом украинофилия приобретает характер протестного
бренда на продажу. Ясное дело, что показать свои убеждения здесь проще, потому
что традиционализм не запрещается, как вна Украине. Но он не приветствуется. Мы
говорили с Секацким. Моя книга может понравиться кому угодно, о судьбе Ассанжа
знают единицы. Интеллектуалы предпочитают Невзорова. Не все, конечно. Есть
диалогоспособные. Потому категории давления, вражды, перчатки, дуэли, дискуссии
в мире, где можно все и нельзя ничего, принимаются как пережиток
консервативного СССР: это так не работает давно. Be net to overcome net, we
will overcome Russia in details, стать сетью, чтобы победить сеть, мы победим
Россию в деталях - ученики Тоффлера знают толк во фрагментации и диффузных
состояниях, невозможно противостоять рою пчел мечом Добрыни или пистолем деда:
он тут же становится предметом оседлывания или превращается в товарный знак
гротескного типа.

\iusr{Евгения Бильченко}
\textbf{Vera Romanova} 

наши открытые дискуссии с барышнями типа Мерридейл кончались тем, в Праге и
Берлине, что Бильченко вдруг оказывалась агентом ФСБ и Кремля (непроницаемая
воображаемая идентичность и демонизация оппонента). А теперь они комфортно
ощущают себя в России, катаясь на инерции пост-ельцинизма, так называемые
\enquote{чистки} - это страшилки для Украины, где реально чистят. Они не любят Россию,
их раздражает Россия, ее народ, ее идентичность, ее религия, ее уменьшительные
суффиксы (Саввушка) - это чистой воды ксенофобия, но в рамках имплицитного
либерализма.

 · Ответить · 7 ч.

\iusr{Евгения Бильченко}
\textbf{Vera Romanova} 

Посему, вывод: я в январе 2021 года вна Украине, прости за юмор, потеряла
статус, который ставил бы меня на уровень, могущий позволить мне выйти на
диалог с Оксфордом. С \enquote{ничем} транснационализм не общается. Россия как традиция
выполняет функцию пустоты, о которую спотыкается символическое шествие
либерального насилия. Такой пустотой было еврейство в годы нацизма. Они этого
боятся, они не знают, что с этим делать. Я смирилась. Я сделала, что могла.
Большего я не могу. Если Ассанжа довели до инсульта, меня просто вытеснили в
небытие, в принципе, это был мой выбор.

\iusr{Vera Romanova}
\textbf{Евгения Бильченко} 

да, Женька, ты сейчас на плахе. Хотели тебя как жертвенного барана загнать на
мокрушную бойню, а ты оттуда сбежала не в схрон, а прямиком на Лобное место.
Только это- не бабская игра в жертву, а... непредсказуемая
разбойничья, блин, удаль. (\enquote{Рони-дочь разбойника} в действии:). Посему по-тихому
замочить тебя им уже слабо. Подорвать здоровье, сделать идиоткой и так далее
попытки, конечно, есть и будут. Но по-тихому они прощелкали момент. Значит,
остается только казнить. Публично. А это уже не бойня, как им хотелось в
хотелках. Это бой.

\end{itemize} % }

\iusr{Анатолий Шевченко}
Культурология притерпевает очень сильные изменения.
Какая-то аномалия .

\begin{itemize} % {
\iusr{Евгения Бильченко}
\textbf{Анатолий Шевченко} 

Предсказание великого ученого Бенно Хюбнера, горжусь знакомством с немецким
мастером: наука, которая заменила на кафедрах марксизма-ленинизма идеологию и
стала рупором партийной пропаганды национализма (Украина) или американского
глобализма (Россия), обречена. Но здесь школа круче, не то, что круче, а в
разы, очень крутые люди со мной общаются, но либеральная цензура - ежовская,
всем плевать на мнение российской государственности, это считается смешным,
интеллектуалы навязывают протест как догму.

\end{itemize} % }

\iusr{Михайло Ященко}

Красивое фото, но выражение лица ему не соответствует ... Все-таки хотелось бы
узнать - какое значение придают культурологии у других вузах Украины, в России,
на трижды проклятом и полностью загнивающем Западе, в Китае, Японии и дальнем
и среднем Востоке или в Африке и Австралии. Вообще в мире. Просветите, Женя
пожалуйста.

\end{itemize} % }
