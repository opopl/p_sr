% vim: keymap=russian-jcukenwin
%%beginhead 
 
%%file articles.de_vy_kobzari
%%parent body
%%url http://www.berdnyk.com.ua/statti4.html
 
%%endhead 

\subsection{Де ви, кобзарі?..}

\url{http://www.berdnyk.com.ua/statti4.html}

Роздуми фантаста

"До чого тут "роздуми фантаста"? - запитає прискіпливий читач. - Кобзарі -
історичний анахронізм, таке собі викопне диво минулих віків. Ясна річ, приємно
послухати думу, пісню прадавньої пори з вуст якогось бандуриста: екзотика! Вона
втішає - і ми ж колись!.. Ого-го! Було, було! Було - та загуло! Не ті часи
тепер, не ті турботи, не ті проблеми! "Де ви, кобзарі"? Чи не смішно гукати ось
так патетично? Це крик у порожнечу..."

Заждіть, не поспішайте проголошувати вирок преславним кобзарям, бо це буде
вирок нашій власній душі, нашому серцю. Скажу одразу: кобзарі існували з
незапам'ятних часів. Скільки є народ - стільки є й співці у його надрах, як
виразники його волі, прагнень, пошуків, поразок і перемог. Вони можуть
називатися по-різному - боянами, гуслярами, піснярами, кобзарями, бардами,
бандуристами, - проте суть їхня не міняється. Це - речники совісті етносу, його
віри в прийдешнє, його історичного оптимізму. А відтак - хіба можна мислити, що
грядущі епохи залишать у минулому цілий духовний інститут творців нашого
сумління й сенсу буття?

Я не мислю такого! Для мене майбуття рідного народу відкриває реальність
гармонійного співжиття, де кобзарі посядуть провідне місце, як деміурги нового
щабля свідомості невмирущої етнодуші.

Якщо так, то звідки ж отой тривожний поклик: де ви, кобзарі?..

Тривога обґрунтована. Проілюструю такими спогадами.

Ще до останньої війни (та й десяток літ після неї) над просторами України
щовечора звучали народні пісні. Кожне село було пісенним полем душі. Кожна хата
співала, раділа, сумувала, плакала. Ті акорди чарівливого мелосу зливалися у
річечки, потоки, річища дивотворення, у хорали, ораторії, симфонії духовного
настрою, роздуму, пошуку, надії, смутку чи радощів.

Пам'ятаю, коли я був ще підлітком: виходжу на вигін біля річища дніпровського.
Зорі у небі, зорі - у водах затоки поміж білих-пребілих лілій. Дівчата у
весняних вінках, парубки у святкових вишиванках збираються в коло, і над
дніпровською долиною лине журлива пісня-сповідь:

Ой Дніпре, мій Дніпре, широкий та дужий,
Чому ти, мій Дніпре, чому ти мовчиш?
Щось маю сказати, дещо й розпитати
І ласки для серця у тебе спитать...

І мені здавалося, що то не просто поетично-пісенний образ, а правдива розмова
засмученої душі з правічним Дніпром-Славутою, і що все довкола - латаття на
водах, лози на піщаних кручах, зірки в дивоколі, заквітчані дівчата та їхня
мелодійна туга - то нерозривне, спільне дійство народного серця, котрому нема
кінця. Бо й справді: прислухаємося, а з Правобережжя, від круч, де розташоване
старовинне село Витачів, долинає інша пісня-діалог:

Ой у полі дві тополі,
Одна одну перехитує,
А в козака дві дівчиноньки,
Одна одну перепитує...

Потоки співу зустрічаються над долиною Дніпра, і не заважають вони один одному,
а сплітаються в калейдоскопічну спіраль такої краси, що перед нею меркнуть
академічні опери та ораторії в театрах і філармоніях.

Уже пізніше, коли я почав розмірковувати над космічним покликанням і прийдешнім
людства, виник такий образ: десь у Всесвіті мають бути мислячі істоти, котрим
не треба радіоприймачів чи потужних пристроїв, щоб чути наше слово, пісню,
думу, вони мають такі тонкі прилади, що отой мелодійний потік, про який ми
згадали, долинає до них у цілості, неушкодженості, в чаклунському консонансі, в
найвищому духосинтезі.

Уявіть таке диво: Земля випромінює не лише в інфраспектрі чи у видимому
промінні, у тепловому, в радіо, нейтрино чи ще там якомусь фізичному
діапазоні,- наша Планетарна Матінка радує далекі світи тисячоголосим хоралом
любові й ніжності, сумніву й пошуку, роздумів і знахідок. А якщо вони -
космічні Брати - уміють розшифрувати наші поняття, слова, образи, то непідробне
захоплення мають викликати навіть у найвищих мудреців космосу глибини ось такої
пісні (простенької на перший погляд):

Од поля до поля
Виросла тополя,
Питав козак дівку:
- Чи будеш ти моя?

- Пощитай, козаче,
Всі зорі на небі,
Як перещитаєш -
Вийду я за тебе...

Щитав козак зорі,
Та й недощитався,
Питав козак дівку -
Та й недопитався...

Ви тільки подумайте, якої любові хоче дівчина? Щоб кохання козака виросло до
масштабів космічних, щоб він, її обранець, обняв розумом і серцем безмірність,
бо лише тоді їхня спілка матиме сенс під оком оцієї зоряної ріки, котра вінчає
їх нічним видноколом, лише тоді вона стане невмирущою. Але безсмертя заслуговує
тільки нетлінне, тільки те, що оволоділо силами стихій, злилося з першосуттю
Природи. Ось чому хлопець відповідає дівчині:

- Пусти, дівко, камінь,
Камінь за водою,
Як попливе камінь -
Звінчаюсь з тобою...

Які титанічні натури! Які богоподібні мрії виникали в надрах нашого народу! Про
який же анахронізм може йти мова? Ми - не хуторяни з діда-прадіда, а суджені
космічні мандрівники, які ще в колисці бачили сни про своє покликання. Хіба
дарма з нашого лона народилися такі генії, як Кибальчич, котрий і в смертну ніч
свою малював контури зоряних кораблів, або Корольов, який здійснив ці
сподівання пращурів, чи Вернадський - творець ноосферного щабля вселюдського
буття?

То що ж? Радіймо? Всесвіт слухає пісню Рідного Краю, милується творенням
тисячолітніх поколінь, жадає ввійти в контакт з унікальним дивом життя й
розуму?

Гай-гай! Все це було! А тепер?..

Років п'ятнадцять тому, побувавши на Тарасовій горі в Каневі, ми з товаришем
вирішили пройти селами й полями України аж до Говерли, щоб безпосередньо
відчути пісенний і мовний дух.

Ми зупинялися вечорами в десятках сіл. Прислухалися, чи не почуємо дівочу або
парубоцьку пісню під зорями? Дарма! Простір німував. Лише інколи в гаю чи в
садку витьохкував свою весільну трель соловейко. І той пташиний спів ще
разючіше підкреслював хворобливість ситуації, що відкривалася враженій душі. Ми
підходили до хат, під вікна світлиць. Може, там співають? Марно! Миготіння
синіх екранів кидало полиски на шибки, і здавалося, що то демонічний
телевізійний автоген безжально й зловтішно випалює з серця народного казкове
коріння, від якого тисячоліття тому відбрунькувалися пагони національного
життя.

Ми наближалися до клубів, до будинків культури. Може, там?.. Гай-гай! Ревище
чужинецької псевдомузики, лжепісні жахало, відштовхувало, оглушувало. А в тому
кабанячому, бугаячому, звіриному верещанні, кувіканні, гарчанні вихилялися,
корчилися, знемагали, паралізовано смикалися тіні наших дітей, наших
спадкоємців, будівників грядущого світу...

Уявіть собі, що ті самі Брати з Далеких Світів, котрі тисячоліттями милувалися
дивозвучним потоком вітчизняної пісні, раптом відчули, зареєстрували,
відзначили: той потік зник, випарувався, пропав. А замість нього покотилася у
безвість лавина безглуздя, дисгармонії. Що вони повинні подумати?

Лише одне: сталася катастрофа. Пісенний етнос знищено якимись напасниками,
викорчувано до тла. Замість нього на планеті розмножилася, стала панівною
цілком інша цивілізація (якщо її можна так назвати!).

Не гратимемося у визначники, не будемо базікати, що, мовляв, це закономірний
процес: настали нові часи, отже, потрібні нові пісні, їх нове покоління співає,
тим більше, що й народні пісні чуємо по радіо, в телепередачах. Браття, риба у
воді - то одне, а риба в консервах - то щось цілком інше. Народ перестав
творити пісню, народ перестав співати - природно, стихійно співати! Той народ,
який навіть у найтяжчі часи співав, сповнював нектаром душі бездонний вулик
зоряного неба. Пам'ятаєте?

Садок вишневий коло хати,
Хрущі над вишнями гудуть,
Плугатарі з плугами йдуть,
Співають ідучи дівчата...

А далі...

Затихло все, тільки дівчата
Та соловейко не затих...

Що ж з нами сталося за сто років? Чи справді з космічної прірви до нас
непомітно вкралися пришельці з Альфа Центавра та втілилися в наших дітей (та й
у нас), а тепер змушують нас співати свої пісні, грати свою музику, формувати
свої руїнницькі пристрої й машини?!

Жарти?

Та ні, далеко не жарти!

Бо чому йде така шалена атака на душі юного покоління, аби змусити його
повністю перейняти демонічну псевдокультуру сьогодення?

Триває дивовижний за масштабністю експеримент по денаціоналізації, дегероїзації, деромантизації народів, а отже - по формуванню космополітичної раси безбатченків, ласих до буржуазних цяцьок, готових за меркантильні псевдовартості на будь-який злочин, байдужих до власного походження, історичних традицій, до всього, що становить сутність людського єства і покликання.

Згадаймо недавнє минуле. Ще вчора всілякі маланчуки розбишакували на ниві
української, білоруської, російської культур, забороняючи кобзарські студії,
припиняючи роботи по створенню хортицького комплексу, винищуючи книги, де
неортодоксально згадувалась історія козацтва, не дозволяючи ставити фільми про
Січ Запорозьку і так далі, і таке інше... Хай у нас В. Добровольському,
народному артистові, котрий палко хотів зіграти Тараса Бульбу, протягом
тридцяти років забороняли підступатися до цієї теми! А чому те самісіньке
відбувалося в Росії, де С. Бондарчуку теж тридцять літ не дозволяли фільм за
цим геніальним твором Гоголя?

Забороняльщики добре знають, що творчий прорив у генетичне минуле народу змете,
як павутину, все їхнє псевдо-культурне сміття, яким вони засипали душі юних
сучасників. Тому й "триває бій на полі мрій, клекоче злом одвічний змій!".

Почався цей бій давно! Спробуємо зазирнути в сутінки історії...

Що ми знаємо про довізантійську пору життя пращурів? Те, що розповів Нестор? Чи
можна беззастережно довіряти йому? Адже навіть тепер події, учасниками й
свідками яких є ми з тобою, шановний читачу, безсоромно перебріхувалися й
містифікувалися на догоду "вождям" і "фюрерам". Хіба не той самий закон для
минулого? Апологети візантійства, одним з яких був Нестор, не лише безжально
винищували найменший знак сонячних вірувань наших прабатьків, а й емоційно
згиджували суть і призначення вікових традицій, звичаїв, ідеалів народного
духу. Дивного тут нічого нема: ще й нині, коли на землі, здається, панує культ
Розуму, Гуманізму, Справедливості, багато вчених мужів люто сваряться за
пріоритети тих чи інших напрямів у пізнанні, світогляді, філософському
осмисленні Буття. А що ж говорити про ту далеку пору, коли мова йшла не лише
про амбітність котрогось із монархів, патріархів чи жерців,- кожен крок у
історичних метаморфозах вирішував долю тисячолітніх циклів розвою Планетарного
Саду. З кожної гілки звисав Змій-Спокусник, який пропонував з'їсти "яблуко
пізнання", щоб стати "такими, як боги". І ми таки з'їли оте сакраментальне
"яблуко" візантійських зміїв, знехтувавши пращурівські традиції. Що практично
означала акція Володимира?

Багато сучасних найавторитетніших учених захоплено твердять, що ми прилучилися
до високої європейської культури, отримали письмо, відкрили вікно у світ
найвищих осягнень людського генія. Чи так же воно?

Уявіть собі таке: росте тисячолітній Дуб, має буйну крону, могутнє гілля,
коріння, занурене в глибини рідної землі; він дає притулок тисячам птахів,
затишок людям, подорожнім, сіячам, він формує структуру довколишніх ґрунтів і
навіть погоду й клімат у своєму краї, він невіддільний від того мікрокосмосу,
де колись проріс пращурівський жолудь, давши такий благословенний паросток...
Потім приходять якісь зайди, зрубують велетня. З гуркотом, стогоном і болем
гримнулося вікове дерево на землю, потрясаючи її до основ. До кореневища
прищеплюється цілком інше дерево, і корінь Дуба, змушений напоювати ту
чужинецьку гілку прадідівськими соками...

Вчені, про яких я згадав, радо повідомляють нам, нібито відбулася асиміляція культур, візантійство органічно злилося з попередніми традиціями, витворило оригінальний синтез нового світогляду й співжиття. Це - обман і містифікація!

Що збереглося, браття, від пращурів?

Християнізовані колядки, щедрівки, огризки примітивних забобонів? А де наші
довізантійські пісні, думи, міфи, казки, філософські концепції, традиції? Все,
все, все викорчувано, з'їдено до тла. Той паросток, що виріс на кореневищі
Праслов'янського Дуба, цілковито змінив характер усього історичного клімату
давньоруського регіону, наклав печать спотворення на душі сучасників тієї
духовної агресії. Прадіди стали слухняними виконавцями чужинецьких релігійних
алгоритмів, а отже, дозволили ліпити себе так, як заманеться будь-якому
експериментаторові, що воліє "прорубувати вікно" на Захід чи на Схід! Воістину,
Див Зоряного Дивокола "обрушився на землю", як справедливо скаржиться автор
"Слова о полку Ігоревім"!

Жорстко й нещадно формувався новий етнос, інше древо буття.

Ми були Синами Світовида, Дажбога.

Ми стали грішниками, недолугими спадкоємцями Адама.

Ми переходили після смерті у країну сонячних прадідів, прилучаючись до потужної
ріки всенародного життя, до Зоряного Віче Предків.

Ми стали підходити до смертної межі, страхаючись грядущого Небесного Трибуналу,
не відаючи - куди потрапимо: до вічного блаженства чи до вічної муки!

Ми формували себе вже не за велінням вікових батьківських традицій, вивірених
генієм попередників, а за імперативом чужинецьких хитрунів, котрі давно
збагнули правило: щоб володіти народом, треба позбавити його власного розуму і
нав'язати чужу програму. І не має значення - яку, з плюсовим знаком чи
мінусовим, - головне, щоб вона не дозволила реалізувати себе!

Апологети візантійства твердять: нам дали культуру, абетку, філософію. Хто дав?
І чим це можна довести? Адже відомо: ще Аскольд і Дір були християнами, отже,
мусили читати Новий Завіт. А пізніше княгиня Ольга та й половина дружинників
Святослава сповідували вчення Учителя з Назарету, а відтак пізнавали його
заповіти з рідних письмових джерел. (Чому ігнорується свідчення самого Кирила,
котрий у Корсуні бачив священні книги, написані руськими письменами?).

В чому суть історичної катастрофи християнства у тих народів, яким воно було
прищеплене?

В насиллі.

Доки вчення Христа було катакомбним, гнаним, доти воно зберігало першочистоту,
пластичність, людяність, гностичність, духовність, суверенність, гідність,
високу романтичність і кличність. Як тільки Константин зробив його державним,
воно потьмяніло і впало. "Не можна служити двом панам",- попереджав Учитель.
Так воно й сталося! Імперські амбіції нової генерації християн спотворили суть
і стратегію Благої Вісті, а потім сформували "мерзоту запустіння" на місці
святому (хто читає - той розуміє): як ще інакше можна характеризувати
діяльність інквізиції (та й не лише цього жорстокого інституту католицизму,
інші церкви були не менш жорстокими в переслідуванні інакше думаючих)?!

Те саме сталося і в Київській Русі. Зникли волхви, щезли відуни, знавці
тисячолітніх традицій та вірувань. Переслідувалися знахарі, народні лікарі,
віщуни, творці сонячних ремесел, пов'язаних з культами сил Природи. Пропадали
бояни, гуслярі, танули в імлі минулих суверенних епох, бо розпадалася вікова
структура праруського краю, совістю й виразником якої були співці.

Ймовірно, що саме зрада громадських, вічових ідеалів прабатьків спричинилася до
страхітливої катастрофи при зіткненні з татаро-монгольською навалою. Адже
християнізовані князі тільки те й знали, що гризлися за столи та наділи,
хапаючи за горло брат брата.

Минули сотні літ вражаючих метаморфоз. У планетарному котлі варилися нові сполуки, нові етнічні компоненти.

Ми тут не маємо наміру оцінювати переваги чи принади, вартості чи перспективи
тих або інших історичних сил. Хочеться відзначити головне: на перехресті сил і
впливів, там, де різні потоки нейтралізувалися, знову почав відроджуватися
Правічний Дуб, уособлений козацьким рухом, а з ним - усе, що йому було
притаманне: пісенна стихія, очолена кобзарями - спадкоємцями боянів та
гуслярів, стихійна мудрість народу, виявлена пізніше в загальній освіті, в
Могилянському Братстві, інститут характерництва, унікальний вияв космогонічного
дійства, ще досі не досліджений, пошук форм народовладдя через ради різного
вияву і масштабу. Недарма класики наукового комунізму назвали Січову Колиску
"козацькою республікою"!

Протягом віків тривав небувалий експеримент відродження, заспівала земля,
запломенів край вогнями мужності, вірності, кохання, мудрості. Забриніли у
просторі струни кобзи, несучи на крилах милозвучних акордів пам'ять і славу
поколінь. Народилася, пустила пагіння в душу народну неповторна, унікальна
генерація кобзарства - цього серця суспільного організму.

Проте все це відбувалося поміж роззявленими пащами монархічних потвор. Чи легко
було оберегти принципи народовладдя й вольності під натиском класових, кланових
і групових сил, котрим поперек горла ставало існування еволюційного центру
свободи?

Димитр Корибут Вишневецький, князь із роду Рюриковичів, який став першим
гетьманом Січі Запорозької, не раз, не два входив у контакт із могутнім
північним сусідом, схиляючи Івана Грозного до спілки, проте тиран категорично
відхиляв такий "симбіоз", страхаючись, що козацька вольниця розмиє мури
деспотичної темниці.

Вишневецький, наречений Байдою, володів мудрістю народною, глибинною, сонячною,
а Грозний ніс імператив мудрості драконячої, великодержавницької, демонічної.
Що могло зв'язати їх воєдино?

До речі, мало хто з сучасників задумується над питанням про прізвисько
славетного героя дум і пісень. Чому Байда? Що за Байда? "Годі тобі, Байдо,
байдувати, сватай мою доньку та йди царювати!"

А тут, навіть в одному цьому прізвиську - незмірна глибина і тайна нашого
минулого. Байдами називалися козаки, які пройшли посвяту на характерника,
нашого козацького архата, мудреця, йога, чарівника. Це поняття (Байда) має
корінь у далекому минулому. Байдья у санскриті означає - лікар, цілитель і
перевізник. Той, хто зумів зцілити себе і може допомогти іншим, той, хто
готовий переправитися на інший берег буття. Байда - то наш український Будда.
Звідси й байдак, човен, і байдужий - тобто спокійний, володіючий внутрішніми
силами. Хіба оспівана кобзарями смерть героїчного Байди-Вишневецького не є
історичним свідченням про унікальних людей, яких породжувала наша земля в пору
гроз і ураганів?!

Повернімося до кобзарів.

Традиція народних співців, започаткована в незапам'яті віків, міцно вросла у
козацький рух, а разом з тим стала кульмінацією його у духовній сфері. Жодна
зміна в суспільному житті України, жодне потрясіння, похід, повстання, етична
проблема, яка поставала перед совістю нації, питання любові, вірності й зради -
ніщо не обходилося без щирого, батьківського, суворого й справедливого слова
кобзарів.

Визвольні походи до Криму чи Туреччини, змагання з лукавою шляхтою, повстання
Хмельницького - всюди кобзарі променем багатющого досвіду висвітлювали минулі
чи грядущі події, схвалювали, застерігали або засуджували. І їхня пісня звучала
як вирок народу!

І коли твердиня волі - Січ Запорозька - була зруйнована, кобзарі понесли її
славу в гущу народну, передаючи вікам і генетичним глибинам нації заповіти
мужності, суверенітету, духовності, свободи. Може, саме тоді народилася
дивовижна легенда, яку я почув од старого діда на берегах Великого Лугу (ще до
того, як там заплюскотіло Каховське море). В тій легенді оповідалося, що не всі
козаки та старшини покинули Хортицю, деякі спалили себе у характерницькому
вогні, аби побудувати Січ Небесну, Небувалу, Непорушну. Натхненний тим
переказом, я колись написав баладу, що закінчувалася так:

Так промайнули віки невимірні,
Хортиця нижня в тумані пропала...
Лицарі Волі у Краї Зазірнім
Січ Полум'яну вже збудували.

Гляньте ночами на ясне склепіння -
То палять вогнища лицарські лави,
То променистих шабель миготіння,
То мерехтіння грядущої слави.

Диво зростає в серці народу,
З неба злітає крилате насіння:
Будьте готові до нових походів -
Довбиші б'ють у литаври сумління

І не зупинять звитягу небесну
Інші Потьомкіни та Текелії!
Браття, виходьте з полонів облесних
В Січ Небувалу великої мрії!..

Пам'ять про козацьку республіку викреслювали із свідомості народу протягом
віків, отже, добре знали, що означає для нових поколінь та героїчна естафета! І
так зуміли перебрехати спогади про подвижництво української нації, так
"по-вченому" містифікували, обставляючи велемудрими та велемовними термінами й
паралелями, що навіть великий "патріот" Панько Куліш, довго не роздумуючи,
хрестив козаків направо й наліво "кровожерами", "гультяями", "руйнаторами", а
Петра й Катерину - культуртрегерами, будівниками цивілізованої держави! І лише
геній Тараса, пірнувши в бездонну криницю народної душі, віднайшов історичну
правду, возвеличивши подвиг козацтва, а разом з ним - кобзарства, назвавши цим
збірним ім'ям увесь свій багатоголосий поетичний доробок, напоєний із
тисячолітньої ріки бунтуючих правічних сил.

Кобзар!

Давши таку назву своїй думі, Тарас окреслив пунктир для прийдешніх співців: не
лише повторювати прадавні думи, оплакувати минулу славу, нагадувати про
втрачене, а й - головним чином - концентрувати в собі та довкола себе найвищі
духовні вартості, цноти, заповіти і дійства. Гуртувати, вести, діяти! Ось що
таке кобзарі!

То де ж ви, сучасні кобзарі? Відгукніться!

Прислухаймося. Де, на яких майданах України бринить віща кобза? Про що квилить,
куди кличе? Кого звеселяє, застерігає, радить, повчає?

В якому вияві шукати, розпізнавати кобзаря?

Історична традиція останніх віків (а може, не традиція, а спотворена легенда,
витворена "правнуками поганими славних прадідів"?) малювала образ нещасного
сліпця, котрий заробляє собі на хліб плачем та стогоном, марно нагадуючи
сучасникам про втрачену славу, загублену на історичних манівцях. Навіть Тарас у
"Перебенді" показав таку реліктову постать, яка то веселить, то засмучує, проте
є лише відлунням минулої грози і не спроможна збудити до дії титанічне серце
народу.

Невже кобзарі завжди були саме такі, як ми їх уявляємо тепер, - нещасні
сіромахи, уособлення повергнутого велетня? О ні! їх породило таке потужне
джерело, що його потоку досить для вгамування спраги безлічі прийдешніх
поколінь.

Погляньмо на численні зображення козака Мамая - таємничої постаті преславного
минулого. Що найбільше вражає в тому образі?

Таємничий воїн-характерник зосереджений і спокійний. Розпряжений кінь, на гіллі
висить шабля, мушкет. Геть відкинуті принади світських веселощів, уособлені в
пляшці та чарці. Обрано позу глибокої медитації-роздуму. Проте ми бачимо не
йогічний транссамадхі, тобто заглиблення у світипозавимірності, ми з радістю й
подивом відзначаємо зосередження на кобзі. Легендарний козак-кобзар заповідає
нащадкам не шлях агресії, розбою чи войовничих походів: він ніби пророкує таку
епоху, коли люди відкинуть небезпечні цяцьки свого недосконалого дитинства і
знайдуть шлях для розвою найзначнішої риси людини мислячої - здатності до
творчості.

Саме в цьому "козацькому роду нема переводу", як про теє натякає у своїй
прегарній книзі Олександр Ільченко. Тоді поклик до сучасних кобзарів набуває
конкретних обрисів і значення. Зрештою, бандуристів у нас досить, їх навчають у
відповідних школах, гуртках ентузіастів, у консерваторіях. Але кобзаря
підготувати в учбовому закладі неможливо, як неможливо вивчити горобця співати
солов'їні рулади.

Хто ж такий кобзар? Де його можна зустріти?

Літ із двадцять тому я прогулювався дніпровською набережною неподалік від
річкового вокзалу. Звернув увагу на знайомих кінодокументалістів, які
метушилися біля апаратури, когось знімаючи. Біля гранітного парапету стояв
непоказний сліпий дідок з бандурою і неквапно перебирав струни. Нас
познайомили. Тоді я вперше почув прізвище - Євген Адамцевич. На жаль, до цього
нічого не чув, не читав, не знав. Доки кіношники щось там готували, когось
чекали, кобзар глянув на мене своїми незрячими стозорими очима, і по спині
побігли мурашки: відчувалося, що співець бачить. Він ледь-ледь щось награвав, а
разом з тим гомонів до Дніпра. Саме так: він розмовляв з водами славутинськими
так, як із живою істотою. Прислухаючись до плюскоту хвилі, запитував (я це
запам'ятав назавжди): "І тебе закували, брате Дніпре, і тебе замучили?"

Так міг запитати лише кобзар.

Коли я познайомився з циклом його пісень, то збагнув суть і призначення народних співців: вони - нерв національного життя. Доки нерв передає біль і радість - організм розвивається і має перспективу для зростання, розкриття, реалізації втаємничених цілей, покликань, призначень. Я запитав Адамцевича - чи багато є в Україні кобзарів, чи не перевелися вони?

- Було до війни ще доста, - скупо одвітив він. - Більше тисячі...
- Де ж поділися? - зацікавився я. - Чому їх не чути?
- А де подівся чистий Дніпро? - відповів кобзар запитанням.
- Так тут... техніка, хімізація, засмічення...
- Із кобзарями... така ж хімізація, - невдоволено буркнув співець. - У тридцяті роки збирали їх... кілька сот під Полтаву... біля тисячі - до Харкова... Кобзарі радо поїхали... у вишиванках, у святкових строях... І жоден не вернувся...

Підійшли кінодокументалісти, почалася зйомка. Мені більше не довелося ні
розмовляти з кобзарем, ні бачити його. Незабаром він помер десь у Криму. Проте
ми знаємо, що збереження ним навіть однієї лише "Запорізької похідної" було
досить для безсмертя Адамцевича. Царство йому небесне і земля пухом!

Я часто думав над тим, що почув із вуст цієї таємничої людини. Чому
бюрократичні мафіозі у всі віки боялися вільних співців? Чому нещадно
переслідували? Очевидно, відчували: навіть оці реліктові небораки несуть у
серцях своїх якусь невмирущу естафету, котра здатна породити бурю, що спроможна
змести всі мури й лабіринти антинародних структур. У кожному разі, будь-яка
спроба відновити інститут кобзарства в часи сталінської тиранії чи в епоху
брежнєвсько-сусловських сутінків зустрічала шалений опір і навіть звинувачення
у "буржуазному ізмі". А разом з тим на сценах вільно демонстрували високу
майстерність та хвацькі веселощі ансамблі й тріо бандуристів. Гай-гай! Усі ті
"дам лиха закаблукам" лише викликали сум і безнадію. І не вірилося, що колись
кобзарі могли піднімати тисячні маси на всенародне рушення!..

Минали роки. Якось я почув у театрі опери (в Києві) виступ кобзарського дуету
братів Литвинів. То був незабутній вечір. Шквал оплесків, віншування, крики
"слава!". І знову - мовчання, забуття. Складалося враження, ніби якась сильна й
нещадна рука скрутила голову дзвінкоголосим солов'ям, що невідомо звідки
з'явилися.

У тривожну пору сімдесятих років доля занесла мене в село Гребені над Дніпром.
Там я збудував хатинку, де можна було б зосередитися й працювати влітку.
Сусідом виявився один з братів Литвинів - Василь. Міцний, з красивими руками і
мужнім обличчям козацького отамана. Та попри свою войовничу зовнішність був
співець спокійний і сумирний. Пропадав на Дніпрі, вудив рибу, мовчав. Мене
цікавили перспективи дальшого розвою його кобзарської майстерності. Він сумно
жартував:

- Натомилася бандура...
- І все ж таки...
- І Земля втомилася, - вперто повторив кобзар. - А вже якщо наша матінка втомилася, то не жди добра...

І він заспівав мені новостворену пісню-сюїту "Спить натомлена Земля". В ній
чувся біль, смуток, глибока тривога за долю не лише рідного краю, людей, а й
усієї планети. Пізніше я збагнув, що зустрів правдивого "реліктового" кобзаря,
котрий чутливо реагує на стан Вітчизни, є ніби її вічнорезонуючим камертоном.

Що відбувалося зі співцем далі, я не знав, бо доля закинула мене в такі краї,
де, як жартували наші мудрі предки, жаба цицьки дає. Кілька літ я мав справу з
тою жабою, а коли повернувся, то застав Василя Литвина цілком в іншому настрої,
вигляді, стані. З бандури ніби якісь вітри здули пилюку, й інструмент у руках
співця дзвенів, гримів, жебонів, променився, усміхався, погрожував, застерігав.
І сам він змінився, перевтілився з дядькуватого козарлюги у правдавнього
бояна-рапсода: довгі сиві кучері до плеча, вуса, борода. Здавалося, що хтось
прадавній і мудрий знайшов його в океані віків і пробудив до нової й термінової
дії. І все довкола змінилося. До співця приходили школярі, прилучалися до гри
на бандурі, його власна дітвора (ціла купа хлоп'ят і дівчаток) співали й грали
на сопілках, палили язичницькі вогнища і стрибали понад ними, складали пісні й
присвячували їх Сонцю, Зіркам, Місяцеві, Землі. Я відчув: кобзар став предтечею
грядущого циклу оновлення.

І ясно стало, що в житті України, в житті усього світу настає вирішальна пора,
котра має змінити долю Планети і кожної людини. Струни нагадували про щось таке
таємниче, забуте, втрачене, яке здавалося навіки загубленим і виглумленим...

У царство духу вогняного
Дорога в сотні тисяч літ...
До того краю осяйного
Йшли безліч, безмір поколінь...

Ішли палкі і серцем юні -
Та в зраді, в мороку зникали,
Безжально рвали серця струни,
І оберталися на камінь...

Та путь тяжку ту подолає,
Хто сам вогненним духом стане...

То воскресали в серці сучасників прадавні веди, що тисячоліття тому були з
коренем вирвані із свідомості русинів. То сама Земля нагадувала сучасникам про
небезпеку виродження і деградації, про загрозу катастрофи для всього сущого,
про потребу берегти найвищу цінність життя - квітку, краплю води, пісню, матір,
вітчизну, планетарну колиску...

Ніщо не обминало камертон серця Василевого: ні гроза, ні вибухи ядерних бомб.
ні засмічення вод, ні, тим більше, замовкання пісенного потоку у просторі
українському. Він бив на сполох!

Розмаїтість і обшир імпровізацій доводить, що кобзар має дотик до втаємничених
глибин природи, як це завжди велося у його преславних попередників. Я був
зворушений і обнадієний: невже воскресає "козак Мамай", невже ворушиться в
розореній могилі похований характерник, який, жартуючи, заховався колись там
від прискіпливих недругів і чекає урочої пори?

Кобзарю, кобзарю, куди ти прямуєш?
На вольную волю...
Кобзарю, кобзарю, хто шлях тобі вкаже?
Вітри в чистім полі...
Кобзарю, кобзарю, що в полі шукаєш?
Прадавню могилу...
Кобзарю, кобзарю, а що в тій могилі?
Незміряна сила...

І я запитав його:

- Василю! То як же бандура - вже відпочила?
- Від грози втома зникає, - сказав кобзар. - Під грозою не сплять. Я почув громовиці і сказав сам собі: "Василю, це вже не жарти! Виходь на поле герцю!"
- І вивірите в перемогу над сном віків, над знемогою душ, над зневірою сердець?
- Кобзар не думає про те, як не думає дощ, чи слід ось тут або там пролитися цілющим потоком. Кобзар співає, сподіваючись, що є серце, спрагле правди й краси...
- Що можуть зробити кобзарі нині, в нашу химерну пору, зіткану з протиріч, сумнівів, заперечень, деградації?
- Дуже багато. Співці мають захистити пісню, бо коли остаточно вмре вона- тоді марні зусилля політиків і вчених: майбуття у Землі не буде! Кобзарі спроможні викликати біль у закрижанілих серцях багатьох сучасників, бо ж без болю не народиться відповідальність, а без відповідальності жодного прийдешнього не побудувати! Нинішні війни у сфері духу буйніші, ніж війни козацького минулого. Тепер крицеві шаблі ні до чого! Ось послухайте...

І Литвин заспівав:

З глибин віків, крізь товщу літ,
В прадавнім болі стогне Мати,
Бо всі покликані у світ
Творить життя, а не вбивати!
Прийшла пора злу круговерть
Вогненним словом розрубати!
Навік здолать пітьму і смерть,
А світ - любов'ю об'єднати!

Прадавнє пророцтво глаголить, що з'являться в "кінці віку" носії Слова-Логосу,
котрі матимуть меч двосічний, який виходить із вуст. Меч Духовний! Меч Пісні!
Хто заперечить супроти такої зброї? Хіба що лише Сини Пітьми.

Б'ють литаври у серця народів,
Чайки вже маячать на воді...
Як не станеш нині до походу -
Пізно буде каятись тоді!

Розкривайте серце, розкривайте
Громовицям буряним навстріч!
Споряджайте душу, посилайте
У останню Полум'яну Січ!..

1988 р.

