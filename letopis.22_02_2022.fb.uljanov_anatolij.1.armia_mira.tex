% vim: keymap=russian-jcukenwin
%%beginhead 
 
%%file 22_02_2022.fb.uljanov_anatolij.1.armia_mira
%%parent 22_02_2022
 
%%url https://www.facebook.com/dadakinder/posts/5279276572091463
 
%%author_id uljanov_anatolij
%%date 
 
%%tags dnr,donbass,lnr,ukraina
%%title В этой истории мы должны выступить армией мира
 
%%endhead 
 
\subsection{В этой истории мы должны выступить армией мира}
\label{sec:22_02_2022.fb.uljanov_anatolij.1.armia_mira}
 
\Purl{https://www.facebook.com/dadakinder/posts/5279276572091463}
\ifcmt
 author_begin
   author_id uljanov_anatolij
 author_end
\fi

В связи с эскалацией украинского кризиса и связанных с ним угроз, хочу
обратиться к левым товарищам: 

1. Несмотря на попытки известных фракций склонить нас на ту или иную сторону в
очередной войне империалистов, наша сторона известна заранее – это рабочий
класс во всех его изводах. У рабочего класса нет национальности. Все наши
действия должны быть субординированы его интересам, а не интересам буржуазных
правительств, национальных элит и неоколониального «гражданского общества». 

\ii{22_02_2022.fb.uljanov_anatolij.1.armia_mira.pic.1}

2. Признание «независимости» мятежных республик в Кремле меняет статус их
ополчения, превращая его в колониальные армии. Аналогичный статус обретают ВСУ
в случае отказа от внеблокового статуса Украины.

3. Войска Российской Федерации, США, Британии и НАТО – это имперские и,
следовательно, оккупационные войска. Любую форму их присутствия на территории
Украины следует понимать как интервенцию.

4. В сложившихся обстоятельствах, когда ни одна из армий конфликта не
представляет интересы рабочего класса, платформой левых является мир и
сохранение жизней. Именно поэтому левые не должны «тактически» сливаться в стаи
буржуазных фракции, и принимать участия в имперских войнах на стороне
государств и «меньших зол». 

5. Насущной задачей левых является организация Международной Антивоенной
Коалиции, пацифистская агитация (пикеты, демонстрации, публичные выступления,
распространение антивоенных мемов, листовок и т.д. под лозунгами мира), и
создание волонтёрской сети взаимопомощи.

В практическом смысле это означает формирование гуманитарных бригад на базе
дома, района, города, функция которых сводится к поддержке нуждающихся жертв
кризиса, безотносительно их взглядов: в первую очередь, женщин, детей,
стариков, и представителей других уязвимых групп – в том числе тех, кого сейчас
ещё активнее начнут преследовать как «агентов», «предателей» и прочих «врагов
народа».

Форма поддержки зависит от возможностей конкретных активистов и потребностей
нуждающихся: от кухонь и медпунктов до оказания психологической поддержки людям
в состоянии стресса, интервентивной деэскалации межличностных конфликтов на
местах, или помощи условной бабушке, которая не может банально выйти в магазин
за продуктами. Что особенно актуально для жителей линий соприкосновения.

Все человеческие и материальные ресурсы такой сети должны быть инвентаризованы
на уровне локальных групп, а участники скоординированы с помощью наиболее
удобных для них средств коммуникации. Управление каждой ячейкой должно быть
демократическим: один человек = один голос. Решения принимаются на основании
голоса большинства.

Как бы и куда бы империалисты не раздували свою авантюру, сеть взаимопомощи
закладывает фундамент демократической организации в условиях милитаристской
истерии, обострения репрессий и конфликтов. Кроме того, такая сеть соединяет
левых с их локальными и региональными сообществами.  

В этой истории мы должны выступить армией мира. Мы не стреляем, не травим, не
принимаем участия в политическом трайбализме. Мы помогаем нуждающимся, и в этом
процессе создаём коллективного политического актора для дальнейшей борьбы за
рабочую демократию и социальную справедливость.
