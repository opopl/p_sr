% vim: keymap=russian-jcukenwin
%%beginhead 
 
%%file 17_01_2022.stz.news.ua.radiosvoboda.1.ukraina_rik_foto.pic.23_25
%%parent 17_01_2022.stz.news.ua.radiosvoboda.1.ukraina_rik_foto
 
%%url 
 
%%author_id 
%%date 
 
%%tags 
%%title 
 
%%endhead 


\ifcmt
  tab_begin cols=3,no_fig,center

     pic https://gdb.rferl.org/084a0000-0aff-0242-c6a5-08d9d90e81b2_w1023_s.jpg
		 @number 23
		 @caption_begin
26 жовтня. Перше бойове застосування безпілотника «Байрактар» сталося у другій
половині дня 26 жовтня у районі населеного пункту Гранітне на південь від
Донецька. Генштаб ЗСУ повідомив, що з території російських гібридних сил
українські позиції обстрілювала батарея гаубиць Д-30, внаслідок чого загинув
один український військовий та один був поранений. За даними Генштабу ЗСУ,
безпілотник, не перетинаючи лінію зіткнення, знищив одну з гармат, після чого
обстріл припинився
		 @caption_end

		 pic https://gdb.rferl.org/084a0000-0aff-0242-f11b-08d9d90e819f_w1023_s.jpeg
		 @number 24
		 @caption_begin
3 листопада. Біля будівлі Верховної Ради України зібралися кілька сотень людей
із вимогою припинити «примусову» вакцинацію та скасувати обмеження «червоної
зони». Більше про акцію антивакцинаторів
\href{https://www.radiosvoboda.org/a/antyvakcynatory-zibralysia-pid-parlamentom-foto/31544278.html}{тут}
		 @caption_end

		 pic https://gdb.rferl.org/8c8e0000-0aff-0242-3d16-08d9d90f84f4_w1023_s.jpg
		 @number 25
		 @caption_begin
18 листопада. Медичні працівники транспортують тіло померлого від коронавірусу
до моргу міської лікарні міста Каховки. Більше про ковід в Україні у 2021 році
у наші статті
		 @caption_end

  tab_end
\fi
