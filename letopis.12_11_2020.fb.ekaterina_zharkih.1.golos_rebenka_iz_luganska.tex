% vim: keymap=russian-jcukenwin
%%beginhead 
 
%%file 12_11_2020.fb.ekaterina_zharkih.1.golos_rebenka_iz_luganska
%%parent 12_11_2020
 
%%url https://www.facebook.com/kate.zharkih.5/posts/5255339027825492
%%author 
%%tags 
%%title 
 
%%endhead 

\subsection{Голос ребенка из Луганска. Интервью с Фаиной Савенковой}
\Pauthor{Жарких, Екатерина}
\Pauthor{Савенкова, Фаина}
\Purl{https://www.facebook.com/kate.zharkih.5/posts/5255339027825492}

\ifcmt
img_begin 
	url https://scontent-waw1-1.xx.fbcdn.net/v/t1.0-9/125308156_5255329664493095_5449755586067392218_n.jpg?_nc_cat=105&ccb=2&_nc_sid=8bfeb9&_nc_ohc=ppqafP5pZkAAX8PhsEw&_nc_ht=scontent-waw1-1.xx&oh=98501ee9b7e06ac3e2aa859f94eac952&oe=5FD4671E
	caption Интервью с Фаиной Савенковой - Екатерина Жарких
	width 0.7
img_end
\fi

\textbf{Фаина Савенкова} – юная писательница из Луганска. Половину жизни девочка живёт
под обстрелами. Но при этом видит прекрасное и человеческое  вокруг, и
потрясающе это показывает в своих сказках, пьесах, эссе. Вроде бы хотелось ее
назвать ребёнком, но она понимает,  по моему мнению, намного больше некоторых
взрослых. Особенно тех взрослых, которые принимают решения в масштабах страны. 

Изучить ее творчество можно тут: \url{https://proza.ru/avtor/faina5}

Советую прочесть «Детский смех победы», «Умри чудовище» и все истории про
ёжика.

Захотелось поделиться с вами этим человечком и взглядом ребёнка, который прошёл
многое. И все равно пишет и развивается. Я отправила Фаине вопросы, и вот ее
ответы:

Здравствуй, Фаина! Расскажи о себе.

Добрый день! Обычная школьница, разве что не побоялась попробовать себя в
творчестве и показать свои работы взрослым. Кроме писательства, занимаюсь
тхэквондо. Но поскольку мне пока только 12 лет и я учусь в
физико-математической школе, то самое главное – конечно учеба.

Где ты была, когда услышала первые звуки войны?

Дома, с семьей. Мне было 5 лет, и я тогда не ходила ни в детский сад, ни в
школу. Наверное, так оказалось даже лучше, потому что близкие поддерживали меня
в те моменты, когда становилось страшно.

За 7 лет, как изменилась твоя жизнь?

Я стала взрослее, пошла в школу и стала профессионально заниматься спортом,
учусь писать хорошие сказки. Наверное, самое главное, что за это время мне
пришлось научиться бороться со своими страхами. Хотя бы с частью из них. Еще
приходится учиться ответственности, потому что понимаю, что мои слова и
поступки уже не принадлежат только мне, а могут повлиять и на других.

Чувствуешь ли ты себя в безопасности сегодня?

Не знаю. Пока я с семьей, пока тихо и не стреляют, можно забыть, что идет
война. Тогда, конечно, даже не задумываешься о безопасности. Наверное, это и
есть главный ее признак. Но как только эхо доносит издалека грохот
разорвавшихся снарядов, сразу становится неспокойно, что тоже не удивительно.
Какого-то однозначного ответа у меня нет. Похоже, что я просто привыкла.

Что тебя вдохновляет для написание пьес, эссе?

По-разному. Иногда природа, иногда сны, иногда люди или события. Так
получилось, что этой зимой мы на пару дней остались без отопления из-за поломки
котла. И пока дожидались мастера, всей семьей усиленно согревали двух котов. На
самом деле не настолько холодно было, но это же котики! Именно тогда придумался
образ замерзших кошачьих слез. Мне стало интересно, почему же эти слезы
замерзли, что вообще заставило плакать такое милое создание? История
выстроилась очень быстро и первая сказка о Мурке появилась за один вечер. Так
что вдохновить может что угодно.

Что ты хочешь сказать своим творчеством? Люди понимают тебя?

Думаю, что какого-то одного универсального ответа быть просто не может, ведь
произведения разные, как и мысли, отображенные в них. Чаще, конечно, пишу о
дружбе и умении найти общий язык с теми, кто находится рядом.

Как относятся к твоему творчеству в ЛДНР?

Как и везде: кому-то нравится, кому-то – нет. Наверное, неплохо, раз приняли в
Союз писателей ЛНР, а затем пригласили для участия в фестивале «Звезды над
Донбассом».

Я знаю ты коммуницируешь с разными известными деятелями искусства. Как вы находите друг друга?
Обычно пока я нахожу, а не меня. Главное, что есть о чем общаться, а не кто кого нашел, ведь так?

От кого ты чувствуешь поддержку в твоей деятельности?

От родителей, учителей, друзей. Каждый помогает по-своему. На самом деле
поддержкой может стать и ценный совет по произведению, и торжественный пинок
под зад, когда это понадобится. Сразу, естественно, можно даже разозлиться на
все это, но быстро понимаешь, что такой вот пинок иногда бывает очень нужен,
потому что он придал ускорение, которого так не хватало.

Сверстники, твои друзья занимаются каким-то творчеством/спортом? Они находят себя, как думаешь?

Да, среди моих друзей и знакомых много тех, кто занимается спортом или
творчеством. Тут и тхэквондо, и шахматы, и художественная гимнастика. По поводу
творчества немного сложнее говорить, потому что для большинства это первые
попытки и им пока неловко рассказывать о своих начинаниях.

Поддерживают ли ваши культурные центры?

Да, конечно. Можно обратиться за советом и получить ответ на свой вопрос.
Помощь старших и более опытных товарищей никогда не помешает ни начинающему
писателю, ни начинающему художнику, мы же еще многого не знаем. Например, не
всегда понятно как теоретические знания можно применить на практике. Да и сама
теория – часто что-то далекое и неизвестное.

Как ты представляешь Украину? Культурные центры в Киеве, Львове Одессе?

Для меня Украина – не другая планета, так что не думаю, что различия между культурными центрами в Луганске или Одессе существенны.
Если бы пригласили, Приехала ли в Украину?

Не задумывалась над этим, честно говоря. Но думаю, что если бы пригласили – приехала, не может же война продолжаться вечно.

Что ты сказала бы жителям Украины? России?

Я много общаюсь с друзьями и в России, и на Украине. Многие говорят, что эта
война не нужна никому, но я поняла, что она все равно может прийти в любой
город и коснуться всех. Поэтому если обращаться к людям, то очень хотелось бы
сказать, что нужно беречь тот хрупкий мир, который у нас есть. Без мира не
будет будущего.
