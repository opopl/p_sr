% vim: keymap=russian-jcukenwin
%%beginhead 
 
%%file 06_12_2020.sites.ru.zen_yandex.russkaja_semerka.1.falsifikacija_rus_istoria
%%parent 06_12_2020
 
%%url https://zen.yandex.ru/media/russian7/kak-falsificirovali-russkuiu-istoriiu-5fcc781d702d845a13d5b02e
 
%%author Русская Семерка
%%author_id russkaja_semerka
%%author_url 
 
%%tags istoria,russia,falsifikacia
%%title Как фальсифицировали русскую историю
 
%%endhead 
 
\subsection{Как фальсифицировали русскую историю}
\label{sec:06_12_2020.sites.ru.zen_yandex.russkaja_semerka.1.falsifikacija_rus_istoria}
\Purl{https://zen.yandex.ru/media/russian7/kak-falsificirovali-russkuiu-istoriiu-5fcc781d702d845a13d5b02e}
\ifcmt
	author_begin
   author_id russkaja_semerka
	author_end
\fi

\ifcmt
pic https://avatars.mds.yandex.net/get-zen_doc/1542122/pub_5fcc781d702d845a13d5b02e_5fcc78468f8c7853edd5638e/scale_2400
\fi

\index[rus]{Русь!История!Как фальсифицировали русскую историю, 06.12.2020}

В истории нашей страны много белых пятен. Отсутствие достаточного количества
надежных источников порождает не только домыслы, но и откровенные
фальсификации. Некоторые из них оказались очень живучими.

\subsubsection{Древнее, чем принято}

По официальной версии государственность пришла на Русь в 862 году, когда
финно-угорские и славянские племена призвали княжить над ними варяга Рюрика. Но
проблема в том, что теория, известная нам со школьной скамьи, взята из «Повести
временных лет», а достоверность содержащихся в ней сведений современной наукой
ставится под сомнение.

Между тем существует множество фактов, подтверждающих, что государство на Руси
было до призвания варягов. Так, в византийских источниках при описании жизни
русов отразились явные признаки их государственного устройства: развитая
письменность, иерархия знати, административное деление земель. Упоминаются
также мелкие князья, над которыми стояли «цари».

Данные многочисленных раскопок, представленные Институтом археологии РАН,
свидетельствуют, что там, где сейчас находится Среднерусская равнина, еще до
наступления новой эры бурлила жизнь. Известный отечественный археолог и
антрополог Татьяна Алексеева нашла достаточное количество доказательств того,
что на территории современной центральной России в период с 6-го по 2
тысячелетие до н. э. был расцвет больших протогородов.

\subsubsection{Украина-Русь}

\index[writers.rus]{Грушевский, Михаил!историк}

Украинский историк Михаил Грушевский создал одну из самых знаменитых
фальсификаций, на которую опирается современная украинская историография. В
своих трудах он отрицает существование единого древнерусского этноса, а говорит
о параллельной истории двух народностей: «украинско-русской» и «великорусской».
По теории Грушевского Киевская держава – государство «русско-украинской»
народности, а Владимиро-Суздальская – «великорусской».

Уже в период Гражданской войны научные взгляды Грушевского подвергались
серьезной критике со стороны коллег. Одним из самых заметных критиков его
концепции «Украина-Русь» был историк и публицист Андрей Стороженко,
рассматривавший такой подход как попытку облечь политические задачи украинского
сепаратизма в историческую форму.

Влиятельный киевский общественный деятель и публицист Борис Юзефович,
ознакомившись с трудами Грушевского, назвал его «ученым-лгуном», намекая на то,
что вся его писательская деятельность связана с желанием занять место
профессора кафедры русской истории Киевского университета.

\subsubsection{«Велесова книга»}

В 1950 году эмигрантами Юрием Миролюбовым и Александром Куром в Сан-Франциско
была впервые опубликована «Велесова книга». Согласно рассказам Миролюбова текст
«Велесовой книги» списан им с утерянных во время войны деревянных дощечек,
созданных примерно в IX веке.

Однако очень скоро была установлена подложность напечатанного документа. Так,
представляемые Миролюбовым и Куром фотографии дощечек на самом деле были
сделаны со специально подготовленной бумаги.

Филолог Наталья Шалыгина говорит: богатый фактический материал убедительно
доказывает, что «Велесова книга» является полной исторической фальшивкой как с
точки зрения лингвистического и филологического анализа, так и с точки зрения
исторической несостоятельности версии о ее обретении.

В частности, стало известно, что в ответ на аргументы научной критики авторы
подделки вносили изменения и дополнения в уже опубликованный материал, чтобы
придать ему большую правдоподобность.

\subsubsection{Завещание Петра Великого}

Эта тенденциозная фальсификация впервые появилась на французском языке в 1812
году. По утверждению составителей документа в его основе лежал стратегический
план действий для преемников Петра Великого на многие века с целью установления
Россией мирового господства; ставилась цель «приблизиться елико возможно ближе
к Константинополю и к Индиям».

Историки пришли к выводу, что основные положения Завещания были сформулированы
в октябре 1797 года близким к Наполеону польским эмигрантом генералом
Сокольницким. Обилие ошибок и нелепостей в тексте заставляют предположить, что
автор документа не был знаком с внешней политикой Петра I. Также установлено,
что первоначально Завещание предназначалось не для пропагандистских целей, а
для внутреннего использования.

\subsubsection{Ненужная Аляска}

Продажа Россией своей заморской территории Соединенным Штатам в учебниках по
истории объясняется просто: содержать Аляску становилось все накладнее, так как
расходы по ее обеспечению намного превышали доходы от ее хозяйственного
использования. Был еще один резон в продаже Аляски – наладить отношения с США.

Историк Иван Миронов говорит, что существует масса документов, опровергающих
официальную версию. Связанная с продажей Аляски история очень напоминает
современные нам события по части коррупционных скандалов, «откатов» и «распила»
бюджетных и народных средств горсткой олигархов и политиков.

Работы по продаже американской колонии начались еще в царствование Николая I. В
планах правительства кроме продажи Аляски было намерение избавиться от
Алеутских и Курильских островов, разумеется, за деньги. Главным лоббистом
сделки 1867 года выступил Великий князь Константин Николаевич, брат императора
Александра II, в числе его подельников был еще ряд влиятельных лиц, в том числе
глава внешнеполитического ведомства Александр Горчаков.

\subsubsection{Личность Распутина}

В воспоминаниях современников Григорий Распутин часто представал одиозной
личностью. Его обвиняли в массе грехов – пьянстве, разврате, сектантстве,
шпионаже в пользу Германии, во вмешательстве во внутреннюю политику. Однако
даже специальные комиссии, расследовавшие дело Распутина, не нашли ничего
компрометирующего.

Вот что любопытно, обвинители Распутина, в частности, проторей Георгий
Шавельский, в своих мемуарах признавались, что сами лично не знали старца или
видели его несколько раз, а все описываемые ими скандальные истории основывали
исключительно на пересказе когда-то и где-то слышанного.

Доктор филологических наук Татьяна Миронова говорит, что анализ свидетельств и
воспоминаний тех дней повествует о методах банального и наглого манипулирования
общественным мнением при помощи фальсификаций и провокаций в средствах массовой
информации.

Причем не обошлось без подмены, продолжает ученый. Бесчинства, приписываемые
Григорию Распутину, зачастую были клоунадой двойников, организованной
прохиндеями в корыстных целях. Так, по словам Мироновой, было и со скандальной
историей, произошедшей в московском ресторане «Яр». Расследование тогда
показало, что Распутина в тот момент в Москве не было.

\subsubsection{Трагедия в Катыни}

Массовое убийство пленных офицеров польской армии, осуществленное весной 1940
года, долгое время приписывалось Германии. После освобождения Смоленска
советскими войсками была создана специальная комиссия, которая, проведя
собственное расследование, заключила, что польские граждане были расстреляны в
Катыни немецкими оккупационными войсками.

Однако, как свидетельствуют опубликованные в 1992 году документы, расстрелы
поляков проводились по решению НКВД СССР в соответствии с постановлением
Политбюро ЦК ВКП(б) от 5 марта 1940 года. Согласно обнародованным данным всего
были расстреляны 21 857 человек, кроме военных там были мобилизованные польские
врачи, инженеры, адвокаты, журналисты.

Владимир Путин в статусе премьер-министра и президента РФ неоднократно
озвучивал мнение, что катынский расстрел - это преступление сталинского режима
и вызвано оно было, в первую очередь, местью Сталина за поражение в
советско-польской войне 1920 года. В 2011 году российские официальные лица
заявили о готовности рассмотреть вопрос о реабилитации жертв расстрела.  

\subsubsection{«Новая хронология»}

В историографии существует множество фальсификаций – событий, документов,
личностей - но одна из них явно стоит особняком. Это знаменитая теория
математика Анатолия Фоменко, по которой вся предшествующая история объявляется
подложной. Исследователь считает, что традиционная история предвзятая,
тенденциозная и призвана служить той или иной политической системе.

Официальная наука, разумеется, называет взгляды Фоменко псевдонаучными и, в
свою очередь, фальсификацией называет его историческую концепцию. В частности,
заявление Фоменко, что вся история античности была сфальсифицирована в эпоху
Возрождения, по их мнению, лишено не только научного, но и здравого смысла.

По мнению ученых, даже при большом желании невозможно переписать такой объемный
пласт истории. Более того, методология, которую использует Фоменко в своей
«Новой хронологии», взята из другой науки – математики - и применение ее для
анализа истории является некорректным. А навязчивое стремление Фоменко всех
древнерусских правителей объединять с именами монгольских ханов у историков и
вовсе вызывает усмешку.

С чем согласны историки, так это с заявлением Фоменко, что его «Новая
хронология» - мощное идеологическое оружие. Кроме этого, многие считают, что
главная цель лжеученого - это коммерческий успех. Историк Сергей Бушуев в
подобной научной беллетристике видит серьезную опасность, так как ее
популярность в скором времени может вытеснить из сознания общества и наших
потомков реальную историю страны.
