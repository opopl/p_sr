% vim: keymap=russian-jcukenwin
%%beginhead 
 
%%file 05_02_2022.stz.news.ua.strana.2.naciz_match_ukraina_rossia
%%parent 05_02_2022
 
%%url https://strana.news/news/375153-match-ukraina-rossija-po-futzalu-i-krichalki-o-moskaljakh.html
 
%%author_id malahova_oksana
%%date 
 
%%tags futbol,nacionalist,nacionalizm,oskorblenie,rossia,sport,ukraina
%%title "Бий москаля, складайте трупи". Как националисты устроили скандал на матче по футзалу Украины с Россией
 
%%endhead 
 
\subsection{\enquote{Бий москаля, складайте трупи}. Как националисты устроили скандал на матче по футзалу Украины с Россией}
\label{sec:05_02_2022.stz.news.ua.strana.2.naciz_match_ukraina_rossia}
 
\Purl{https://strana.news/news/375153-match-ukraina-rossija-po-futzalu-i-krichalki-o-moskaljakh.html}
\ifcmt
 author_begin
   author_id malahova_oksana
 author_end
\fi

В Амстердаме впервые за 14 лет сборные Украины и России сыграли в футбол. В
полуфинале чемпионата Европы по футзалу Украина уступила РФ со счетом 2:3.

От матча ждали не только спортивного, но и политического накала. И проявили
его, как и ожидалось, болельщики с украинской стороны. Они оглашали трибуны
песнями о \enquote{трупах москалей}. 

В какой-то момент диктор на арене даже попросил фанатов прекратить песни.

Разбирались в скандале с кричалками, а также собрали реакцию украинских
пользователей в соцсетях.

\subsubsection{Как прошел матч}

Этот поединок привлек внимание к себе не только составом соперников. Еще раньше
\enquote{зраду} вызвало, что капитан Петр Шотурма и один из футболистов Владимир
Разуванов играли в России, да еще и за клуб коммунистической партии, который
так и называется - КПРФ.

Уже на стадии подготовки матча этот момент начали обсуждать в
националистическом лагере. Но за команду все равно решили болеть. 

Тем не менее, Украина проиграла сборной России со счетом 2:3. Матч проходил
очень напряженно, с большим количеством нереализованных моментов у обеих
сборных.

За минуту до конца матча украинцы не смогли забить пенальти. Кроме того, у них
было два верных момента в самом конце встречи, но сравнять счет наши футзалисты
так и не смогли. Тем не менее, играли они активно и вышли по итогу в \enquote{бронзовый
финал}. То есть, теперь им предстоит сыграть с аутсайдром второй полуфинальной
пары Испания - Португалия.

У украинцев забитыми голами отметились Серый и Абакшин. У российской сборной
забили Соколов, Афанасьев и Ниязов.

\subsubsection{Путин и \enquote{москали}. Кричалки на трибунах}

Если сборная Украины играла достойно, то ее фанаты (среди них преобладали
националисты) вели себя мягко говоря неадекватно.

Во время матча украинские болельщики выкрикивали националистические речевки,
включая матерную о Владимире Путине. Во время трансляции было слышно, как
болельщики на трибунах кричат \enquote{Украина превыше всего}, \enquote{Кто не скачет, тот
москаль}.

\ii{05_02_2022.stz.news.ua.strana.2.naciz_match_ukraina_rossia.pic.1}

Комментатор на \enquote{UA: Першем} даже порадовался такому поведению. Но
диктор на арене на украинском языке попросил фанатов прекратить песни
неспортивного характера. Тем не менее, болельщики продолжили петь лозунги.

Во втором тайме кричалки стали еще суровее. Фанаты спели песню \enquote{Бей
москаля}:

Бий москаля, складайте трупи.

Ще й кулемет беріть у руки.

І нову ленту заряджай.
