% vim: keymap=russian-jcukenwin
%%beginhead 
 
%%file 18_07_2020.fb.lnr.14
%%parent 18_07_2020
 
%%endhead 
\subsection{Украинцы переходят к прямому действию}
\label{sec:18_07_2020.fb.lnr.14}
\url{https://www.facebook.com/groups/LNRGUMO/permalink/2855587521219470/}
  
\vspace{0.5cm}
{\small\LaTeX~section: \verb|18_07_2020.fb.lnr.14| project: \verb|letopis| rootid: \verb|p_saintrussia|}
\vspace{0.5cm}
  
Партизанские меры в условиях западной оккупации

«Цэевропейцы» вынуждены констатировать, что украинский народ не приемлет
внедряемые западные ценности. Причем речь идет не только о пресловутых
содомских «радостях», против которых выступают даже «ландскнехты революции
гидности» - правосеки и прочие неонацисты. Украинцы отвергают даже такие альфа
и омега «национального самосознания», как декоммунизацию.

Об этом, в частности, свидетельствуют результаты соцопроса, проведенного фондом
«Демократические инициативы» и Киевским международным институтом социологии.
Эти данные тем ценнее, что оба этих учреждения работают на западные гранты.

Исследование, которое проходило с 17 по 22 апреля методом телефонного интервью
и охватило 2000 респондентов, показало, что после 6 лет оголтелой пропаганды
одобряют запрет советской символики всего лишь 32\% опрошенных. Остальные
относятся к нему негативно или безразлично.

Что касается переименования улиц и населенных пунктов, то сторонников этих
действий еще меньше --- всего 26,3\%, в то время как 44\% респондентов
высказались против, а еще 19,9\% опрошенных сказали, что им все равно.

Примечательно, что представитель фонда, проводившего опрос, политолог Алексей
Гарань, в комментарии к полученным данным указал, что ориентироваться на мнение
большинства не стоит.

«Не надо ждать, пока будет больше 50\% за «да» или «нет», потому что на самом
деле значительной части людей просто все равно. Если бы ждали таких изменений в
Германии, когда делалась денацификация, то мы можем только представить,
насколько бы она растянулась», --- заявил Гарань.

Впрочем, украинское руководство и так игнорирует мнение большинства населения
страны. Собственно, и фашистско-олигархический мятеж, узурпировавший власть в
стране, был ничем иным, как пренебрежением народной волей.

Об этом свидетельствует и победа на прошедших выборах главы государства
Владимира Зеленского, за которого, несмотря на всю его нелепость, люди отдавали
голоса, лишь бы не дать Порошенко, с которым ассоциировались бандеровские
«идеалы», сохранить кресло.

Собственно, и упомянутый опрос говорит не столько о коммунистических симпатиях
украинцев, сколько о неприятии ими нацистских установок.

Избрание же Зеленского принесло жителям Украины полное разочарование. Они
сейчас прекрасно понимают, что «выборы президента» были ничем иным, как
мероприятием по созданию новой вывески прежнего марионеточного режима.

Сегодня украинцев старается «развести» точно таким же образом Анатолий Шарий,
пытающийся поиграть на поле Зеленского, используя антибандеровские настроения
большинства украинцев. Впрочем, веры ему, похоже, нет.

Люди все меньше склонны доверять демократическим процедурам, и освобождение от
ненавистного режима, скорее, связывают с восточным соседом, или с «Северным
ветром», как говорят в Донбассе.

Но при этом сказать, что все украинцы сидят, сложа руки, будет несправедливо.
Происходящее на Украине говорит о том, что затихшая было в 2016 году городская
герилья вновь оживляется.

Так, на днях в Николаеве сожгли машину главаря местной ячейки неонацистской
группировки «Национальный корпус» Дениса Янтаря.

Сам потерпевший уже сделал заявление, в котором обвинил в происшедшем
«политических оппонентов», которых собирается «достать» сам, не прибегая к
помощи полиции.

Ответственность за поджог автомобиля взяли на себя бойцы некой организации под
названием «Вольный город Одесса».

Соответствующее видео было размещено видеохостинге Ютуб.

Видео снято в заброшенном здании. На нем запечатлены четверо мужчин. У них лица
закрыты масками, в руках они держат оружие.

«Мы, одесские партизаны Ленпоселка, берем на себя ответственность за поджог
машины отъявленных нацистов», - заявляет неизвестный в маске.

Впрочем, можно сказать, что Янтарю повезло --- он расстался с машиной, а не с
жизнью, как, например, известный каратель, причастный к пыткам и внесудебным
казням противников режима и защитников Донбасса. Тело старшего следователя по
особо важным делам 2-го отдела 1-го управления досудебного управления ГСУ СБУ
Романа Закладного нашли ночью в Оболонском районе Киева.

Полковник был обнаружен без обуви, с вывернутыми карманами и травмами головы.
Также отсутствовали два его мобильных телефона. По мнению экспертов, он был
убит в другом месте и позже доставлен в район обнаружения трупа.

Источник в СБУ подтверждает, что Закладный расследовал дела, связанные с
«государственной изменой, сепаратизмом и событиями в Донбассе», работая в
тайных тюрьмах, в том числе и в страшном концлагере на аэродроме Мариуполя.
Согласно данным источника, 15 июля 2020 года следователь должен был уйти в
отставку. Не дотянул двух дней.

Подобные происшествия, по мнению спецпредставителя ЛНР на переговорах по
Донбассу в Минске Родиона Мирошника, говорят о выздоровлении украинского
общества.

«Жечь машины радикалов --- это хороший симптом к выздоровлению общества! Это в
правовом государстве поступают и действуют в соответствии с законом и правами
человека, а где этого нет и государство не способно защитить человека, то
должны поступать адекватно и симметрично, чтобы успокоить зарвавшихся «королей
улицы» и новоявленных штурмовиков», --- написал представитель Луганской Народной
Республики в своем блоге.

И с этим утверждением не поспоришь --- бандеровский режим оставляет украинцам
лишь такую возможность протестовать против насаждаемого беззакония.

Борис Джерелиевский

\url{https://youtu.be/3G0CFJ3KuHA?t=3}
