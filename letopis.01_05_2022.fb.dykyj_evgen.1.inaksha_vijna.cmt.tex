% vim: keymap=russian-jcukenwin
%%beginhead 
 
%%file 01_05_2022.fb.dykyj_evgen.1.inaksha_vijna.cmt
%%parent 01_05_2022.fb.dykyj_evgen.1.inaksha_vijna
 
%%url 
 
%%author_id 
%%date 
 
%%tags 
%%title 
 
%%endhead 
\zzSecCmt

\begin{itemize} % {
\iusr{Vladimir Popov}
Ставлю лайк - потім читаю

\begin{itemize} % {
\iusr{Evgen Dykyj}
\textbf{Володимир Попов} 

дяка, але так не треба  @igg{fbicon.smile}  я ж намагаюсь по реакції, себто лайках та репостах,
зрозуміти чи погоджуються читачі із висловленими думками

\iusr{Vladimir Popov}
\textbf{Evgen Dykyj} після того, як прочитав, лайк не прибрав. Тож, переважно згоден. Звичайно, у мене є свої альтернативні варіанти на ці питання, але у вас є пояснення своєї думки та позиції. Мені це подобається.
\end{itemize} % }

\iusr{Александр Марценюк}

... На жаль розуміння пана Євгена про долю ТрО в Україні таки хибне, бо
виходить зі споглядання за ТрО сьогоднішнього існуючого зразка, який
командування ЗСУ таки наполегливо (впродовж трьох років) ліпило саме під себе
як другий(третій і т.д.) резер підрозділів (дуже)легкої піхоти ...

... Справжнє ТрО це зовсім інше мілітарне утворення чим те що ми маємо назараз
... від слова ЗОВСІМ ! ...

... Замордовані Буча\_Бородянка\_Ірпінь та дуже\_легко\_зданий Херсон тому сумні підтверження ...

\begin{itemize} % {
\iusr{Александр Марценюк}
\url{https://www.facebook.com/groups/921770927864314/permalink/7429469963761012/}

\iusr{Evgen Dykyj}
\textbf{Олександр Марценюк} я не розглядаю у цьому дописі концепції ТрО, її сумну історію в Україні тощо, я тут пишу саме і виключно про реально існуючу ТрО станом на 01.05.2022 на тилових територіях України, і ні про що більше.

\iusr{Александр Марценюк}
\textbf{Evgen Dykyj} 

... так я лайкнув цей Ваш пост, тому що майже з усім погоджуюся ... тільки про
майбутнє ТрО в Україні мені дуже болить ... бо маю власний ТрО\_досвід з двох
етапів ...

1. з 2014-го до 24.02.2022

2. з 24.04.2022 і до вже нашої Перемоги ...
\end{itemize} % }

\iusr{Natalya Starynska}

А в тих, хто виїхав, постійне питання: \enquote{коли вже можна з дітьми повертатись}.
Якщо війна триватиме довго, чоловікам не можна виїжджати, то це або повертатись
під обстріли, або жити в розлуці невідомо скільки часу...

\begin{itemize} % {
\iusr{Hanna Korniyenko}
\textbf{Natalya Starynska} так, і це є відповідальністю кожної родини.

\iusr{Volodymyr Nestorak}
\textbf{Natalya Starynska} на це питання зможете відповісти тільки ви самі

\iusr{Вікторія Кашаюк}
\textbf{Natalya Starynska} так це і є пепежити війну. Але в розлуці не лише ті, хто виїхав і чекає чоловіка, але й усі ті, хто лишився, і чиї чоловіки на фронті...

\end{itemize} % }

\iusr{Sergiy Sakhno}
Як завжди. Логічно, доступно, змістовно!
Дякую, Друже!!!
Слава Naції!!!

\iusr{Hanna Korniyenko}
Дякую. Боляче усвідомлювати, але краще бути без шор на очах.

\iusr{Pavlo Nechitaylo}

З приводу ТРО. Від початку це частина ЗСУ. Черкаські, Хмельницькі, Кам'янецькі,
Дніпропетровські роти трошників вже давно на Сході. Багато хто у важких боях

\iusr{Олексій Боголюбов}
Дочитую до \enquote{мої котики та зайчики}. Далі не цікаво

\iusr{Viacheslav Shramovych}
Дякую. Як завжди, все логічно.

\iusr{Nathalie Ishchenko}
З теробороною все вже вирішено)))

\iusr{Iryna Kozeretska}
Героям слава!

\iusr{Oleg Rubel}
Героям Слава!

\iusr{Ostap Nezruch}
....завершилася «епоха героїчної партизанщини». ..почали. Наши. Там. Успішно.

\iusr{Oleg Safronkov}

З приводу спецур та партизанів подивися оцей пост та його обговорення:
\url{https://www.facebook.com/permalink.php?story_fbid=1053787578869060&id=100027134836740}
- і завваж, що ані автор поста, ані коментатори там не дилетанти і предмет
обговорення добре знають.

\iusr{Oleg Safronkov}

P.S. Локація події описана ось тут, і фото те ж саме:
\url{https://www.facebook.com/LevinYigal/posts/1404297496715021}

\iusr{Криміналіст Дельтапланів}

А шо там за бойових пінгвінів чути? Наче, вже до Гібралтару допливли
@igg{fbicon.face.smirking} ?

\iusr{Оксана Добровольська}
Дякую

\end{itemize} % }
