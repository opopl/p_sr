% vim: keymap=russian-jcukenwin
%%beginhead 
 
%%file 29_12_2015.fb.zharkih_denis.1.vse_delo_v_srede
%%parent 29_12_2015
 
%%url https://www.facebook.com/permalink.php?story_fbid=1711819805698096&id=100006102787780
 
%%author_id zharkih_denis
%%date 
 
%%tags internet,krym,politika,rossia,ukraina
%%title Все дело в среде
 
%%endhead 
 
\subsection{Все дело в среде}
\label{sec:29_12_2015.fb.zharkih_denis.1.vse_delo_v_srede}
 
\Purl{https://www.facebook.com/permalink.php?story_fbid=1711819805698096&id=100006102787780}
\ifcmt
 author_begin
   author_id zharkih_denis
 author_end
\fi

Все дело в среде

Вот развеселил пост френда из Крыма, которой упоротые сообщили, что у них
Интернета нет. И в голову не приходит, что как бы они с крымчанкой по ФБ
разговаривали. Тут почему-то вспомнилась встреча с одним ныне великим и
известным политтехнологом. Тогда, он был еще не в пике славы. Его мне
представили, как перспективного политтехнолога, а меня ему, как автора
избирательных роликов, которые прогремели по всей стране. Так вот это дарование
посмотрело на меня туманным взглядом и изрекло:

- Перестаньте врать. Эти ролики придумал Павловский в подвалах ФСБ. 

Сразу понятно, что это чудо живет в какой-то параллельной реальности. Нет, он
конечно мог сомневаться в авторстве этих работ, хоть я и давал
пресс-конференцию по этому поводу. Но тогда, поскольку это его профессия,
логично было бы ему как-то уличить меня во лжи, установить истину. Но истина
дарованию была совсем не нужна. Это чудо продавало иллюзии. И Павловский,
мутящий что-то там в глубоких подвалах ФСБ намного качественная иллюзия, чем
обыкновенный мужик, который просто умеет читать социологию, разбирается в
анализе фокус-групп, и не лишен творческого подхода. 

И вот интересно, что это дарование, ставшее впоследствии гуру украинских
политтехнологов, никогда и близко ничего подобного не достигло. Но не достигло
в профессиональном плане. В финансовом плане это чудо переплюнуло меня в тысячи
раз.  Я о чем? 

Украинские заказчики быстрее платят за иллюзии, чем за эффективную работу.
Поэтому, украинская политика в таком ужасном состоянии. И эта политика
формирует потребителей иллюзий, не связанных с реальностью. Вот они и выводят
страну в шестое измерение, а  масса людей активно убегает от реальности,
уничтожая этим страну.  Дело не в  этом даровании, а в социальных процессах,
запускающих иллюзии и уничтожающие страну. Это важнейшие социальные процессы,
которые никому в Украине не интересны, а они определяют будущее Украины.
