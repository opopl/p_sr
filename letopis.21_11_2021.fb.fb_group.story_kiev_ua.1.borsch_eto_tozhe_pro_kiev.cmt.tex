% vim: keymap=russian-jcukenwin
%%beginhead 
 
%%file 21_11_2021.fb.fb_group.story_kiev_ua.1.borsch_eto_tozhe_pro_kiev.cmt
%%parent 21_11_2021.fb.fb_group.story_kiev_ua.1.borsch_eto_tozhe_pro_kiev
 
%%url 
 
%%author_id 
%%date 
 
%%tags 
%%title 
 
%%endhead 
\subsubsection{Коментарі}

\begin{itemize} % {
\iusr{Antonina Chepiga}

\ifcmt
  ig https://i2.paste.pics/bf749261e0c456c2efc329bbe7de9beb.png
  @width 0.3
\fi

\iusr{Chyzhyshyn Oksana}
Пару днів не варто настоювати борщ, а от хоч годину - так.
А за ,,сахарной косточкой,, на Житній базар, Поділ.
Там все розкажуть і виберуть найкращу.
Дякую, запахло борщем)
І хоч молодь- це дві години втраченого життя, мені його готовити в задоволення.
І якщо шо - мій борщ найсмачніший!  @igg{fbicon.smile} 

\begin{itemize} % {
\iusr{Lola Madino}
\textbf{Chyzhyshyn Oksana} про годину не погоджусь.
Борщ має бути вчорашній!

\iusr{Chyzhyshyn Oksana}
\textbf{Lola Madino}
Це для тих, хто не стояв за плитою і не готовив його 2 години.
Зразу тарілка гарячого борщу - як винагорода @igg{fbicon.exclamation.mark}  @igg{fbicon.wink} 
\end{itemize} % }

\iusr{Alexander Feldman}
Могу есть на завтрак обед и ужин.

\begin{itemize} % {
\iusr{Chyzhyshyn Oksana}
\textbf{Alexander Feldman}
Повезло вашій дружині )

\begin{itemize} % {
\iusr{Alexander Feldman}
\textbf{Chyzhyshyn Oksana} Я холостой....

\iusr{Alexander Feldman}

\ifcmt
  ig https://scontent-mxp1-1.xx.fbcdn.net/v/t39.1997-6/s168x128/69436487_2620209587991906_8357975979712315392_n.png?_nc_cat=1&ccb=1-5&_nc_sid=ac3552&_nc_ohc=ROeGZ-8K5CgAX8uVXX2&_nc_ht=scontent-mxp1-1.xx&oh=8442fcba21747f86a7fad2016e4b86f7&oe=61B84645
  @width 0.2
\fi

\iusr{Chyzhyshyn Oksana}
\textbf{Alexander Feldman}
Тоді перепрошую)

\iusr{Лариса Чепурна}
\textbf{Alexander Feldman} Повезло вдвойне

\end{itemize} % }

\iusr{Нина Светличная}
\textbf{Alexander Feldman} Мої свекри на Луганщині взагалі не могли без борщу. Свекор міг його їсти тричи на день. Якщо не було другого блюда - це не страшне, а от якщо не було борщу - то \enquote{як це, а їсти що будемо?}))

\iusr{Alexander Feldman}
\textbf{Нина Светличная} занесено в ЮНЕСКО.

\iusr{Виктор Сидоровский}
\textbf{Alexander Feldman} Да его можно просто пить как вино!

\begin{itemize} % {
\iusr{Chyzhyshyn Oksana}
\textbf{Victor Sidorovsky}
\obeycr
Так поляки і роблять)
Вони варять борщ, смачний насичений , а заправляють його в кінці буряковим квасом .
В банку кубиками буряк , вода ,пару сухарів чорного хліба / Бородинський ,на 3-5 дніів в тепле місце.
Так от той квас дає надзвичайний смак і насичений колір.
І ніколи не помішає туди ж відварені сухі білі гриби і відвар з них .
Я це використовую, на жаль ,найбільш на Святий вечір.
А поляки постійно , відціджують борщ і подають рідину в філіжанках / чашках.
А до них печуть з слоєного тіста палички солені з тміном .
Смакота .
\restorecr

\iusr{Alexander Feldman}
\textbf{Виктор Сидоровский} хочу в Польшу...

\end{itemize} % }

\end{itemize} % }

\iusr{Natasha Levitskaya}

\enquote{Пару дней настаивать}...

Мой дедушка, светлая память ему, ел борщ только на второй день. Бабушка варила
и на следующий день давала на обед.

Я, конечно, ем и в первый день, но борщ действительно настаивается и вкусней на
второй! @igg{fbicon.face.smiling.eyes.smiling} 

А про сахарную косточку помню рассказ приятельницы, которая( это было очень
давно) поехала к друзьям в Италию, и они попросили сварить её знаменитый борщ.
Никакой сахарной косточки она не нашла, хотя искала долго - там даже не
понимали, о чем она говорит. Сварила из того, что было, всем очень понравился,
но она сказала, что это совсем не то! @igg{fbicon.grin} 


\iusr{Ирина Тамарова}
Да зачем та сахарная косточка, можно взять хороший кусок свинины.

\begin{itemize} % {
\iusr{Vladimir Petrov}
\textbf{Ирина Тамарова} Можно и из курицы, только название блюда придется менять.

\iusr{Yuriy Shyray}
\textbf{Vladimir Petrov} очень вкусный борщ с утки, просто бомба!

\iusr{Ирина Тамарова}

Зачем менять. Взять курицы и белых грибов. А на хлеб сала копченого. И все
шикарно выйдет. Бррщ может быть на любом бульйоне.. праввда рыбный лично я
считаю ересью. НО это лично я.


\iusr{Виталий Перещ}
Борщ, він і в Америці борщ.

\ifcmt
  ig https://scontent-mxp1-1.xx.fbcdn.net/v/t39.30808-6/259661669_597833558001870_3034863028123993551_n.jpg?_nc_cat=111&ccb=1-5&_nc_sid=dbeb18&_nc_ohc=-GbP8GicyBIAX8PSZ-i&_nc_ht=scontent-mxp1-1.xx&oh=00_AT9MRPKraksZkqM5bSmZ-Cgw3MPOCyMLP0_VBkcCv0er0w&oe=61B966D5
  @width 0.3
\fi

\iusr{Ирина Тамарова}

Ага, а ша начнем сравнивать рецепты и выясним, что на каждой кухне он
разный. Был же флешмоб по борщу не так давно. Там таких вариаций в рецептах
понасбрасывали.. разнообразие, имя тебе бесконечность.

\end{itemize} % }

\iusr{Алексей Ткаченко}
Блюдо \#1 !!!

\iusr{Елена Травкина}
Мммммм! Запахло борщиком!!!
Ви неймовірні і як кулінар!!!
Смачненько!

\iusr{Кретов Андрей}

А помнится, что борщ нужно еще чуть-чуть заправлять старым салом. Наши матери
(40-летним - так бабушки) так делали.

\begin{itemize} % {
\iusr{Chyzhyshyn Oksana}
\textbf{Kretov Andrey}
А ще оте старе сало добавляли, як тушили молоду капусту.
Особливий смак )

\iusr{Татьяна Бригинец}
\textbf{Кретов Андрей} такой борщ как раз бабушкой и пахнет...

\iusr{Yulianna Portnoy}
\textbf{Кретов Андрей} сало старым не бывает, не доживает

\iusr{Кретов Андрей}
\textbf{Yulianna Portnoy} Держали ведь его, старого, немного, для этой цели.
\end{itemize} % }

\iusr{Оксана Марченко}

Когда я слышу про сахарную косточку я вспоминаю Центральный гастроном и мясника
Юру.. Юра знал из какой части туши можна было взять на борщ... на антрекот и на
гуляш. Нет сейчас таких мясников...

\begin{itemize} % {
\iusr{Алла Донская}
\textbf{Оксана Марченко} это правда! По-моему, в каждой семье был свой мясник и всегда нужное на каждое блюдо, мясо))

\iusr{Ростислав Чентемиров}
\textbf{Оксана Марченко}, как говорил мой товарищ: у профессора математики мания величия - он возомнил себя мясником.

\iusr{Олег Сошевский}

В те времена мясники заботились о покупателях и каждому ложили косточку, чтобы
каждый смог сварить суп или борщ, а косточкой наградить свою или соседскую
собачку  @igg{fbicon.smile} 


\iusr{Valeriy Rukhman}
\textbf{Оксана Марченко} я работал на Бесарабском рынке с 1979 года по 1995 год мясников, только не говорите , что я вор и никакого Юру я не знал в Центральном гастрономе

\begin{itemize} % {
\iusr{Оксана Марченко}
\textbf{Valeriy Rukhman} Я работала в ЦГ с 1987 по 1993год. В бакалейном отделе был Юра.

\iusr{Valeriy Rukhman}
\textbf{Оксана Марченко} могит быть, могит, быть  @igg{fbicon.tulip}  @igg{fbicon.sheaf.of.rice}  @igg{fbicon.tulip} 
\end{itemize} % }

\end{itemize} % }

\iusr{Толик Михальченко}
Вещь!!!

\iusr{Татьяна Бригинец}
Борщ - не догма, а руководство к действию!

\iusr{Людмила Глебова}
\textbf{Татьяна Бригинец} , не згодна... Таки ДОГМА.

\iusr{Ирина Берлянд Зельманова}
Я уже забыла вкус мозгов из мозговой косточки!

\begin{itemize} % {
\iusr{Татьяна Литвинова}
\textbf{Ирина Берлянд Зельманова} это точно, я очень любила

\iusr{Chyzhyshyn Oksana}
\textbf{Irina Berlyand Zelmanova}
О, це спогади з дуже далекого дитинства)

\iusr{Oleg Berezhinskiy}
\textbf{Ирина Берлянд Зельманова} А кто мешает вспомнить? Купите набор костей (они даже в супермаркетах продаються) и наслаждайтесь!

\begin{itemize} % {
\iusr{Ирина Берлянд Зельманова}
\textbf{Oleg Berezhinskiy}
Нет, это не те кости.
У нас , в Израиле, не продаются мозговые косточки...

\iusr{Oleg Berezhinskiy}
А у нас - те! Если соберете достаточное количество любителей, то можно наладить импорт... говяжих, естественно  @igg{fbicon.wink} 
\end{itemize} % }

\iusr{Aleksandr Mitryaev}

Мозговые косточки еще недавно продавали в мясном отделе Демеевского
базара. Обрезанные с двух концов. Но базар закрыли для реконструкции, где теперь
покупать мозговые косточки, непонятно ....

И вообще, это женщины все одинаковые, что в супермаркете, что на базаре! А
косточки разные как и мясники!

\iusr{Aleksandr Mitryaev}

Комедия с вами всеми! У вес у всех постсссровская ностальгия, как и у меня по
пирожкам с ливеро по пять копеек и с газированной воде с сиропом. Тетя Циля
продавала её и имела свою копейку с каждого недолитого стакана.

Это время уже не вернется! не вернется потому, что у руля жизни уже другое
поколение, которому не нужны знакомые мясники, им нужен шведский стол на
завтрак с переваренной едой настолько, что она вообще не имеет вкуса. Как на
Мадейре в отеле Four Views в октябре месяце, где я был 9 дней. Из Израиля там
туристов нет. А немцев, поляков, и всей Европе, хоть жопой ешь!

Еда унифицировалась, и никто в цивилизованных странах уже не вари холодец из
ножек, прекрасных свиных с добавлением голяшки - газ дорогой.

\iusr{Анна Кучевская}
\textbf{Ирина Берлянд Зельманова} так в чем проблема? Мозговые кости продаются же.

\begin{itemize} % {
\iusr{Оксана Марченко}
\textbf{Анна Кучевская} 

Так нет сейчас таких костей. Раньше скот держали на фермах, выпасали на
лугах, кормили сеном, силосом и перетертым зерном. А сейчас весь скот на
комбикорме. Где вы видели на прилавке старую корову? У нас в селе у фермера был
пожар. Угорели лошади и коровы. Так пожарники еще сарай не успели потушить а
закупщики уже аукцион устроили кто больше даст.

\iusr{Анна Кучевская}
\textbf{Оксана Марченко} 

ну, старые коровы или молодые и чем их там кормят я не знаю, но кости
продаются. И почему бы не купить? Это азиаты кузнечиков жрут, а у нас ещё пока
хоть нормальное мясо есть.

\end{itemize} % }

\end{itemize} % }

\iusr{Aleksandr Mitryaev}

Спасибо, что напомнили! Пол одиннадцатого дня, не завтракал еще, сижу, крапаю
мемуары! А в холодильнике стоит кастрюля с остатками борща в пятилитровой
емкости. Чесночок, горчичка, сметанки 100 грамм, перчик молотый черный.
отрезать хлеба черного скибку, сало вчера купил на ярмарке на улице
Голосеевской.

До чего ж я люблю рідну Україну!!!!

\begin{itemize} % {
\iusr{Светлана Аникина}
\textbf{Aleksandr Mitryaev} По прочтении у меня на могилке напишут: \enquote{Захлебнулась слюной!}. Ну нельзя же так вкусно писать - теперь придется бежать варить тат самый борщ!

\iusr{Aleksandr Mitryaev}
Сожалею, у меня совсем на дне осталось! Разве что Вы по дороге купите Пшеничную, а сала у меня много!

\iusr{Aleksandr Mitryaev}

Но пельмени я готовлю по рецепту моей Мамы. Плов - по рецепту моего друга из
Ташкента. А вот борщ - по рецепту моей Жены, пальчики оближете! А ещё она
делает пампушки с чесноком - ну это во-още!!!!


\iusr{Надежда Лавренова}
\textbf{Aleksandr Mitryaev} ну як так можна  @igg{fbicon.thumb.up.yellow}{repeat=3}  @igg{fbicon.heart.eyes}{repeat=3} 
Незважаючи на пізній сніданок, аби слиною не вдавитись!!!  @igg{fbicon.heart.eyes}{repeat=3} Боооорщ - то смакотаааа...
\end{itemize} % }

\iusr{Natalya Tarasenko}
 @igg{fbicon.face.savoring.food} 

\iusr{Калиниченко Марина}
Борщ - это целый ритуал и священнодействие.

\iusr{Vadim Vadim}

Борщ варится в чавунке, стоящим в летней кухне, в начале огорода, с него
выкапывается молодая картошка и капуста, тут же рубают петуха и заправляют
помидором из бочки и старым салом.

Всё остальное-пародия)))

\iusr{Aleksandr Mitryaev}

Тема борща вызывает живой интерес! Но надо помнить, что если рецептов плова
около двух тысяч, то рецептов борща - около миллиона! Например я не люблю в
борще много капусты, фасоль и зажарку с салом и луком. и Люблю в борще свиные
косточки, обсосать и погрызть!

\begin{itemize} % {
\iusr{Chyzhyshyn Oksana}
\textbf{Aleksandr Mitryaev}
Немає кращого, як сам собі звариш) Борщ , який тобі найбільш до смаку  @igg{fbicon.wink} 
І коли твій, вже дорослий одружений, син, який дуже добре, між іншим, готує:
- Знаєш ,Мам, а так інколи хочеться тарілку твого борщу з телятиною  @igg{fbicon.hearts.two} 

\iusr{Tatyana Mazerati}
ravvini vas ne poymut
\end{itemize} % }

\iusr{Виктор Сидоровский}
Всё замечательно кроме деревянной ложки! Ну не дано!

\iusr{Liliya Liliya}

Сахарная косточка к борщу, известна многим поколениям. И она действительно дает
вкусный наваристый бульон, особенно вместе с мозговой. Так что сведения из
интернета тут совсем не при чем.

\iusr{Михайло Наместник}

Усіх друзів з днем Гідності та Свободи.
Слава Нації та смерть ворогам.

\iusr{Надежда Лавренова}
\textbf{Михайло Наместник} Слава Україні  @igg{fbicon.heart.yellow}  @igg{fbicon.heart.blue} 

\iusr{Alina Monroz}
І я сьогодні зварила свіженький борщ, не можу діждатись, коли настоїться....

\iusr{Татьяна Сирота}

Читая Ваш пост, чуть слюной не подавилась @igg{fbicon.face.savoring.food}{repeat=3} 

Ну... очень вкусно!

\iusr{Светлана Александренко}

Замечательная инструкция приготовления борща! С юмором, легко. Спасибо за
настроение. Читала с удовольствием!

\ifcmt
  ig https://i2.paste.pics/bf749261e0c456c2efc329bbe7de9beb.png
  @width 0.3
\fi

\iusr{Анатолий Золотушкин}
\textbf{Светлана Александренко} спасибо @igg{fbicon.exclamation.mark}

\iusr{Дима Бузовский}

\ifcmt
  ig https://scontent-mxp1-1.xx.fbcdn.net/v/t39.30808-6/259132995_1493063477732594_2467836348409198486_n.jpg?_nc_cat=111&ccb=1-5&_nc_sid=dbeb18&_nc_ohc=ujWT50OqrD8AX8i3swp&_nc_ht=scontent-mxp1-1.xx&oh=51d344d8e05fe1c32c2726f58ae63d90&oe=61B941D4
  @width 0.4
\fi

\iusr{Ирина Иванченко}
Огласите все ,,меню" на месяц, Анатоль ! Я подготовлю,,иллюстрации" и фотоотчёт ,,о проделанной работе".

\iusr{Анна Горбань}
Похоже, что из ресторана Просвита в Испании, возле малаги.

\iusr{Люда Невзгляд}
Самое вкусное, полезное 1 ое блюдо на планете.

\iusr{Klara Romanova}
Не то я вчера готовила, не то...
Пойду наверстаю.

\iusr{Валерій Килимник}
Ось, із-за таких постів втрачаємо саме дороге- пам'ять про кухню наших дідів-прадідів...Шведський стіл вам в ср...ку!

\iusr{Нинель Кузницкая}
Сижу в поликлинике, ожидая приём у офтальмолога и...вдруг подступило чудовищное чувство голода! Вам удалось!  @igg{fbicon.thumb.up.yellow}{repeat=3} 

\iusr{Serhii Mo}
А где вы берете красную воду для борща?

\begin{itemize} % {
\iusr{Chyzhyshyn Oksana}
\textbf{Serhii Mo}
 @igg{fbicon.smile} 

\iusr{Serhii Mo}
\textbf{Chyzhyshyn Oksana} это шутка  @igg{fbicon.grin} 

\iusr{Chyzhyshyn Oksana}
\textbf{Serhii Mo}
Я догадалась )

\iusr{Alexander Bunin}
\textbf{Serhii Mo} В магазине, где продают красный майёнез для селедки под шубой.

\iusr{Serhii Mo}
\textbf{Alexander Bunin} точно @igg{fbicon.face.grinning.big.eyes}  @igg{fbicon.index.pointing.up}

\end{itemize} % }

\iusr{Alexander Bunin}

После Чернобыля никаких косточек не используем. Ни сахарных, ни мозговых. Из
мяса вкуснее, если жирное, конечно.

\begin{itemize} % {
\iusr{Владимир Таякин}
\textbf{Alexander Bunin} перепробовали все. Остановились на хорошо вываренных в скороварке куриных каркасах с разными добавками, например свинским языком.
\end{itemize} % }

\iusr{Marina Lavrow}

На загал, шановне паньство, існує більш ніж 300 рецептів борщу, так зо не варто
сваритися про те, чий кращий. Я взагалі варю вегетаріанський вже багато років,
і він смачний!! Ми тут в столиці Канади організуємо такий собі благодійний
фестиваль борщу вже понад 10 років. Кошти йдуть на підтримку сиротинцями на
Сході Укараіни. Ще не було двох однакових борщів, і кожний собі вибираю до
вподоби. А у Євгена Клопотенка, київського шеф-повара і ресторатора, можна
знайти цікаві рецепти.


\iusr{Наталия Петрушевская}
Работает картинка...

\iusr{Ирина Чикалова}

Мой дед доставал из готового борща сахарную косточку и выколачивал её в большой
бабушкин половник @igg{fbicon.face.grinning.smiling.eyes} 

\begin{itemize} % {
\iusr{Gennadiy Ab}
\textbf{Ирина Чикалова} Знакомо.

\iusr{Анатолий Золотушкин}
\textbf{Ирина Чикалова} мозговую
\end{itemize} % }

\iusr{Yevheniia Doroshuk}

Конечно, косточки в борще - продукт незаменимый. Но свекла тоже важный
компонент.Я ее тру на крупной тёрке, заправляю сахаром, солюью(по 1 дес.
Ложке), и лимонным соком ( 1 ч. Ложка). Все это маринуется. Добавляю в кастрюлю
вместе с капустой после того, как будет готов бульон с картошкой, морковкой,
луком( не панированные). В моем борще вообще ничего не парируют. Все это на
очень медленном огне. Очень вкусно со свиными, не жирными ребрышками. Все по
вкусу. И, конечно, сметану, зелень и т д. Приятного.... хорошего дня.

\begin{itemize} % {
\iusr{Luda Draganova}
\textbf{Yevheniia Doroshuk} Цікаво, а я починаю з буряку в чистий м’ясний бульйон (якщо не підсмажую)

\iusr{Yevheniia Doroshuk}
\textbf{Luda Draganova} рецептів борщу-сила силенна. І в кожної господині він неперевершений. Гарного вечора.

\iusr{Владимир Таякин}
\textbf{Yevheniia Doroshuk} это рецепт красных щей со свёклой по усть-зажопински? Не называйте это борщем, не наживайте врагов(

\iusr{Лариса Павлюк}
\textbf{Владимир Таякин} Хамам слово не давали.

\iusr{Владимир Таякин}
\textbf{Лариса Павлюк} почему же вы тогда говорите без разрешения??
\end{itemize} % }

\iusr{Ольга Юровских}
Хорошо, что только что борщ с внуками ели.

\iusr{Татьяна Горовенко}
Уже появился аппетит благодаря Вашему смачному повествованию

\iusr{Anneta Kobets}

\ifcmt
  ig https://scontent-mxp1-1.xx.fbcdn.net/v/t39.1997-6/p480x480/106011002_953858235076534_2503726066745003202_n.png?_nc_cat=1&ccb=1-5&_nc_sid=0572db&_nc_ohc=MfZJ55N7ZkQAX_h5FTe&_nc_ht=scontent-mxp1-1.xx&oh=00_AT9mqQls2XC1Y2sihw8-QyftyKVm7tYgGSjYOMnoAI0kMw&oe=61B771D0
  @width 0.2
\fi

\iusr{Оксана Марченко}

Ну вот... Три дня читаю коментарии сидя на работе. И наконец то у меня
выходной))) Варю борщ))) @igg{fbicon.face.smiling.hearts} 

\iusr{Татьяна Зубко Маркина}

\ifcmt
  ig https://scontent-mxp1-1.xx.fbcdn.net/v/t39.1997-6/p480x480/105941685_953860581742966_1572841152382279834_n.png?_nc_cat=1&ccb=1-5&_nc_sid=0572db&_nc_ohc=_iE3XtYBdGYAX_yWoLc&_nc_ht=scontent-mxp1-1.xx&oh=c51abc2b22b58bf5986c988f967a31d1&oe=61B7FD4B
  @width 0.2
\fi


\end{itemize} % }
