% vim: keymap=russian-jcukenwin
%%beginhead 
 
%%file 12_03_2023.fb.kipcharskij_viktor.mariupol.1.r_k_tomu___den_17___.cmt
%%parent 12_03_2023.fb.kipcharskij_viktor.mariupol.1.r_k_tomu___den_17___
 
%%url 
 
%%author_id 
%%date 
 
%%tags 
%%title 
 
%%endhead 

\qqSecCmt

\iusr{Светлана Водзянская-Живогляд}

Вчьорашня розповідь це якраз про сьогоднішній сонячний день. Бомбили багато
приватний сектор від оптового ринку і до заводу Ілліча ...

Хрещений ходив на оптовий так само, але ходив то дуже м'яко сказано.. повз а
вони живуть поруч від нас за 2 вулиці ( прямо навпроти інфекційної лікарні).
Набрав гранатів та мандаринів, набрав 4 пакети на 2 родини, доки доповз до дому
то лишилося 2 пакети, каже що дуже допомогло те що коли ми з ним до цього йшли
і я показувала канави, він вже по пам'яті де вони, по ним і повзав перебігаючи
у проміжках. Літак гудів без перестану. Про молодшого брата що боронить
Маріуполь нема відомостей вже 5 днів, він не приїздив. Крайнього разу привіз
баклагу води їм і 1 буханець хліба, сказав більше нічого не привезе, бо самі
доїдають останні харчі.

Носили бабусі борщ, ми носимо гарячу їжу їй через день, по черзі раз я другий
раз хрещений. Вона живе на перехресті Шевченко і Куїнджі. Бабуся цілує і
тихенько сварить що ми ризикуємо своїм життям заради того щоб її нагодувати,
після того як я потрапила під обстріл 3 березня виходячи від неї вона постійно
просить не приходити. Але впертість в нас від дідуся (він помер у січні, не
дожив до війни ровно 40 днів). Тому я сказала що я буду мене б не було якби не
вона, то я носитиму доки зможу.

\iusr{Светлана Водзянская-Живогляд}

Мій дідусь це окрема історія. Його рівень патріотизму можна оцінити з того що
відразу після проголошення незалежності він відмовився розмовляти російською,
дав нам усім місяць щоб вивчити українську, і розмовляти з ним лише нею!!!
Багато того що я знаю про свою країну все від нього..

з 2014 року він носив хлопцям в окопи смаколики, фрукти, хлопці віталися з ним
на ім'я бо всі знали що оцей манюній дідусь іде через пости на дачу і щось їм
несе смачненьке :)) його сердце зупинилося рівно за 40 днів до початку війни,
він помер в полі недалеко від окопів хлопців, не доповз за допомогою трошечки..

бабусі зателефонували - « доброго дня, тут Вова помер» тихим голосом, вона не
зразу зрозуміла який Вова... так великий шматок мого серця зупинився... помер разом
із ним!!

У війну я увійшла у стані важкої депресї... тому не купувала нічого мабуть, і
харчів не було вдома... але я чула його і відчувала захист його рук як і все
життя. Одна жінка сказала що я більше не ходжу сама, бо він тепер зі мною, тією
яку любив понад усе...
