%%beginhead 
 
%%file 17_01_2023.fb.kushneruk_anna.kyiv.1.pro__ya_ne_spravlyay
%%parent 17_01_2023
 
%%url https://www.facebook.com/kushneruk.anna/posts/pfbid02ziGVrk7mHaKKKfR5eMmEqL6evAjoFBfgTfpqM7XnHhvy6CtVXu6LgwAVpDzb4dKxl
 
%%author_id kushneruk_anna.kyiv
%%date 17_01_2023
 
%%tags dusha,psihologia,chelovek,psihika
%%title Про «я не справляюсь»
 
%%endhead 

\subsection{Про \enquote{я не справляюсь}}
\label{sec:17_01_2023.fb.kushneruk_anna.kyiv.1.pro__ya_ne_spravlyay}

\Purl{https://www.facebook.com/kushneruk.anna/posts/pfbid02ziGVrk7mHaKKKfR5eMmEqL6evAjoFBfgTfpqM7XnHhvy6CtVXu6LgwAVpDzb4dKxl}
\ifcmt
 author_begin
   author_id kushneruk_anna.kyiv
 author_end
\fi

Про «я не справляюсь» 

Можливо, доречно було  б тут перерахувати все, з чим доводиться в ці страшні
часи стикатись, але не буду. 

Не буду  тому, що одна з причин «мене прям геть накрило» і «мене розмотало»

це - ретравматизація. 

Коли біль і переживання болю тисячами повторів проникає в уражене попереднім
болем тіло  ( як от прям буквально фізичне, так і психічне). 

Проникає через усі шпарини( новини, дзвінки, почуті уривки розмов на вулиці,
стрічка / група присвячена ніби то іншій тематиці ( по вирощуванню чайного
гриба якогось). Війна скрізь. Дозволити її собі не бачити - то типу як з 14
року робили вигляд, що війна - то десь інде. Вона скрізь, повторююсь. І вона
вся огидна. 

Але!

Найгостріші стани вмикаються у «витривалих» цивільних на фоні  того, що вони (
долаючи темряву рятівними щоденними  справами) опиняються в колі ( в оточенні )
бомбордовані повторами / перепостами людських страждань. 

Є ті, хто маючи вражаючу кількість контактів і підписників інформують світ,
звітуючи про нашу реальність. Привертають фактичну поміч інформацією . То
робота. Відповідальна і важлива. Вона вимагає наголосів і знаків оклику. 

Є ті, хто колупається ( саме ненависне мені слово після «чьо») в ранах, як у
сфері збудження. І ці перепости  такі ж вже, сука, кількісні і безжальні до ...
ну, до «моєї» Наталі Л, яка не виринає  з того, що паше, як не зна хто, для
наших хлопців. Їй новини відомі, вона в курсі. І той курс на дії. А от кожна
додаткова порція «я теж не можу» «мене прибило» затоплююча, бо вона скрізь.  І
особиста сторінка - то доречне місце ділитися особистим, тут все чесно.
Особистим ! Своїми переживаннями. Не криками і стогіном тих, чиї крики
невпинно в серці у нас всіх. 

Найбільша кількість «зривів» відбувається після світських заходів, які собі в
так людина рідко дозволяє ( типу конкурси - концерти), де у кожній паузі
дублюється втрата і акцентуація трагічних подій. Салюти ж зрозуміло, чому не
можна запускати ? А в «рекламі» і «музичній паузі» залпи як сприймаються ?
Неочікувано і травматично. Там, де я довірився побути в безпеці на 20 хвилин
між своїм до та після, мене яяяяк  ... давайте скажемо «вдарило». 

Слухайте, важливо задаватись питанням НАВІЩО я роблю публікацію чи піднімаю
вразливу ( для всіх психічноздорових) тему.

Не ЧОМУ ( бо болить, бо страшно, бо воно саме, бо боюсь бути окремо).

А саме НАВІЩО. 

Є громадські заяви. Є озвучена позиція. Кредо. Проголошений  вибір з
запропонованим ставленням ( це принципово !) Або з конкретним запитом (чіткім)
про зрозумілу допомогу. 

А є тупий садизм. Тупий, бо навіть же ж неусвідомлений. 

Інформаційна гігієна - то одна із сучасних мов. 

Скидати свої страхи у спільний простір ... не варто.  Працюють фахівці, платні /
безоплатні форми роботи, телефонні лінії підтримки.

Їх треба шукати для СЕБЕ самостійно( це вже половина справи), якщо не в ступорі
фізично. Там допомагають і приносять. 

Бережіть себе, будь ласка, від завалів викидів, бо ними потім і накриває. 

Бережіть тих, кого не знаєте. Бережіть тих, хто  у Вас вбачає можливість
поговорити про той чайний гриб чи що воно... Якщо звертаються за грибом, то про
гриб.  Це про гігієну і інформаційну безпеку в побутовому колі. То важливо.
Побутове коло - то відповідальність.

Ми зараз всі «на роботі» у війні. І хоч іноді, хоч на трохи, за дверима, на
яких написано «безпечно», має бути ... гарантовано безпечно. 

І гарантами є ми для свого оточення.

Є час до останньої краплі сил боротися, не шкодуючи себе. Має бути час, коли
треба поспати. В тиші. Бо її потребують всі. І наші Герої, що сплять під гомін
зброї усіх колібрів, 

і Ви, і я.  І ті, хто просто хотів побути в безпеці.
