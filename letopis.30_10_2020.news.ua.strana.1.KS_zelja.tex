% vim: keymap=russian-jcukenwin
%%beginhead 
 
%%file 30_10_2020.news.ua.strana.1.KS_zelja
%%parent 30_10_2020
 
%%url https://strana.ua/articles/analysis/298098-juristy-nazyvajut-zakonoproekt-zelenskoho-ob-uvolnenii-sudej-konstitutsionnoho-suda-nezakonnym.html
%%author 
%%tags ks,ukraine
%%title 
 
%%endhead 

\subsection{"Посягательство на конституционный строй". Юристы оценили законопроект Зеленского про увольнение судей КС}

\Purl{https://strana.ua/articles/analysis/298098-juristy-nazyvajut-zakonoproekt-zelenskoho-ob-uvolnenii-sudej-konstitutsionnoho-suda-nezakonnym.html}

\Pauthor{Студенникова, Галина}
\Pauthor{Товт, Анастасия}

\ifcmt
img_begin 
	url https://strana.ua/img/article/2980/98_main.jpeg
	caption Сегодня под Конституционным судом собралась акция протеста. Фото: Страна 
img_end
\fi

В Украине продолжается громкий скандал, связанный с последними решениями
Конституционного суда (КС). 

Во вторник Конституционный суд вынес решение о несоответствии Конституции
положений закона "О предотвращении коррупции". Таким образом, становится
практически невозможным привлечение чиновников к какой-либо
ответственности за любые нарушения антикоррупционного законодательства.
 
Сегодня с утра у здания Конституционного суда (КС) в Киеве проходит акция
протеста против решения КС, которое отменяет уголовную ответственность за
недостоверное декларирование.
 
В связи с этим президент Зеленский внёс законопроект в Раду о лишении
полномочий всех судей Конституционного суда в наказание за их решение по
электронным декларациям.
 
У президента считают, что последние решения КСУ были инспирированы
альянсом Коломойского и Порошенко (большинство судей КСУ было назначено во
времена Порошенко и до сих пор, по версии «слуг народа», им
контролируется).
 
"Страна" спросила у авторитетных юристов, противоречит ли законопроект
Зеленского Конституции, и какие последствия он может иметь.

\ifcmt
pic https://strana.ua/img/forall/u/0/25/%D0%9B%D0%B0%D0%B2%D1%80%D0%B8%D0%BD%D0%BE%D0%B2%D0%B8%D1%87-%D0%90%D0%BB%D0%B5%D0%BA%D1%81%D0%B0%D0%BD%D0%B4%D1%80.jpg
\fi

\subsubsection{Александр Лавринович}

{\bfseries 
Александр Лавринович, экс-глава Высшего совета юстиции, экс-министр юстиции Украины:
}

"Конечно, нужно бороться с коррупцией. Но если борьба совершается
неконституционным способом, и на это обращает внимание КС – причем в некоторых
моментах абсолютно правильно указывает на неконституционность отдельных норм, –
то причем тут суд? У нас вообще государство живет в ситуации гибридного
правосудия. С этим нужно заканчивать.

Что касается законопроекта президента, где он предлагает уволить всех судей КС,
ничего нового в этом нет. Есть демонстрация наследственности действий в стране
с 2014 года – полное пренебрежение правом как явлением и Конституцией. В 2014
году принимали закон «О восстановлении доверия к судебной власти в Украине»,
которым прекратили полномочия Высшего совета юстиции, Высшей квалифкомиссии.
Сейчас опять речь идет о восстановлении общественного доверия – просто как
калька. Это не выдерживает никакого сравнения с положениями текста Конституции.

А сегодня президент предлагает переложить с себя ответственность на Верховную
Раду – чтобы она уволила судей КС. Но таких полномочий у парламента нет. По
сути, президент требует, чтобы депутаты ВР шли против Конституции.

В ситуации бесправья говорить о праве – это анекдот.  Это как бороться с
криминалом, поручив наводить порядок уличным бандам.

Последствия этот законопроект может иметь комичные. Когда на следующий день
после принятия закона в КС поступит обращение о его неконституционности, и суд
признает, что он неконституционный. А серьезно, если монобольшинство примет
закон президента, будут большие вопросы к тому, можно ли вообще в Украине
говорить о какой-то законности.

В Конституции четко прописаны права президента. Если он хочет иметь право
назначать и увольнять судей КС, пусть сделает это публично. Он мог обратиться с
предложением внести изменения в Конституцию, чтобы его наделили такими
полномочиями. А сейчас  предложение Зеленского ничего общего с правом не имеет.
Мы продолжаем жить в стране вне права. И это, конечно, будет влиять на
отношение граждан к органам власти и будет способствовать снижению
ответственности коррупционеров.

Борьба с коррупцией и так в Украине имеет перманентно неудачный характер. И
теперь мы получаем еще одно подтверждение, что украинская власть и коррупция –
синонимы, и разделить их невозможно.

Действительно в составе КС есть судьи, назначенные Порошенко. Людей
Коломойского установить там сложно. Как бы там ни было, если рассуждать, кому
выгодно решение КС, безусловно, названным лицам оно выгодно. Но помимо них еще
десяткам лиц. Не думаю, что КС принимал решение о декларациях только ради них.
То, что судьи КС могли получить поощрение за такое решение – допускаю, что у
них может быть личный интерес. Есть информация о том, что у некоторых судей
тоже есть проблемы с декларированием. Они, очевидно, не могут объяснить, откуда
у них взялось то или иное имущество.

Но даже если есть некий альянс в КС, тот способ, которым с ним решил бороться
Зеленский, проблемы не решит. Наоборот - это утвердит верховенство кулуарной
политики и возможности определенных людей влиять на все органы госвласти,
чувствуя себя уверенно и спокойно". 

\subsubsection{Марина Ставнийчук}

\ifcmt
	pic https://strana.ua/img/forall/u/0/25/%D1%81%D1%82%D0%B0%D0%B2%D0%BD%D0%B8%D0%B9%D1%87%D1%83%D0%BA.jpg
\fi

\textbf{Марина Ставнийчук, юрист, член Европейской комиссии "За демократию через право"
(Венецианской комиссии) от Украины 2009-2013 годах:}

"Законопроект президента – это попытка посягательства на конституционный строй государства. 

Зеленский не имеет права требовать уволить судей КС. В Конституции и законах
четко определен порядок прекращения полномочий судьи – подчеркиваю, судьи, –
Конституционного суда. Там вообще нет механизма увольнения всего состава судей.
Это не принято в правовых государствах.

Законопроект Зеленского во всех своих статьях противоречит ряду положений
Конституции: запрещено давление и политическое влияние на суд, судьи КС не
несут ответственности за результат голосования за исключением тех случаев,
когда они совершают преступление или дисциплинарный проступок, но и это должно
быть установлено в соответствии с законом и Конституцией.

Я на протяжении последнего времени наблюдала, как вокруг КС нагнетается
ситуация. Специально создается атмосфера недоверия к судьям КС и к самой
институции. Я утверждаю, что это делает окружение президента, которое
представляет позицию президента и парламента в КС. Эти люди не имеют ни
политического, ни правового опыта. В этой ситуации президента ввели в
заблуждение. Он либо не читал решение КС, либо читал, но с учетом своего уровня
подготовки как гарант Конституции не может дать адекватной оценки этому
решению.

Заседание СНБО могло быть созвано, поскольку вопрос коррупции – правда вопрос
угрозы нацбезопасности. Но предметом рассмотрения СНБО должно было стать
состояние борьбы с коррупцией в Украине. СНБО должно было проанализировать
реальную законодательную базу, механизмы, состояние антикоррупционных органов в
государстве, наличие или отсутствие судебных решений в антикоррупционной сфере.
Проверить практику НАПК не только по проверке деклараций, а и с точки зрения
механизма – чтобы чиновники добровольно правильно могли декларировать свои
доходы, всем были понятны правила. Чтобы в сфере антикоррупционной борьбы
действовали только законы, а не подзаконные нормативные акты, которые часто
противоречат Конституции и законам.

А также СНБО должно было взять во внимание решения КС. Ведь КС за последние
месяцы принял несколько решений по антикоррупционной сфере. Практически ни одно
из них власть не то, что не выполнила, а даже не взяла во внимание. Например,
КС постановил, что назначение директора НАБУ было не конституционным, но
парламент не отреагировал. И ситуация зависла в правовой неопределенности. И
сегодня агенты НАБУ бросают петарды в здание Конституционного суда – что вообще
происходит в стране?!

На основе решений КС можно было внести изменения в законодательство НАБУ, САП,
НАПК, – с учетом мнения не только экспертов из грантовых организаций, а и
высококвалифицированных правовиков Украины. Так должно было бы быть. А все
остальное сейчас выглядит как попытка антиконституционного переворота в
государстве.

Я не адвокатирую КС и судей – я долгое время критикую отдельные юридические
позиции КС. Но в данном случае я забочусь о том, чтобы конституционный строй в
государстве был обеспечен. Хватит делегитимизировать Конституцию ради
сиюминутной политической целесообразности или чьих-то корпоративных интересов".

\subsubsection{Владимир Пилипенко}


\ifcmt
	pic https://strana.ua/img/forall/u/0/25/%D0%BF%D0%B8%D0%BB%D0%B8%D0%BF%D0%B5%D0%BD%D0%BA%D0%BE.jpg
\fi

{\bfseries 
Владимир Пилипенко, юрист, представитель Украины в Венецианской комиссии 2013-2017 годах:
}

"Я думаю, законопроект Зеленского не будет принят. Он может стать основанием
для импичмента президенту. Но скорее он может стать основанием для того, что
завтра кто-то внесет законопроект о досрочном прекращении полномочий
президента. Если уж на то пошло. Это же все абсурд.

Как бы нам не нравилось решение КСУ, увольнять судей предложенным президентом
способом – это пренебрежение нормами Конституции, гарантом соблюдением которых
является сам глава государства. Этот законопроект - попытка госпереворота.

У президента могут считать все что угодно по поводу того, кто мог повлиять на
решения КС. Но мое мнение: надо просто правильно писать законы в соответствии с
Конституцией, и тогда ни у кого не будет возникать мыслей ни о каких альянсах.
И нечего будет бояться решений КС. Ничто не мешает сейчас изменить закон про
НАПК так, как написал КС, и все будет хорошо". 

\subsubsection{Василий Нимченко}

\ifcmt
	pic https://strana.ua/img/forall/u/0/25/%D0%BD%D0%B8%D0%BC%D1%87%D0%B5%D0%BD%D0%BA%D0%BE.jpg
\fi

{\bfseries 
Василий Нимченко, судья Конституционного Суда Украины в отставке, нардеп от ОПЗЖ:
}

"Горит Конституционный суд, который я строил. Мы с фундамента его построили, у
нас много лет были нормальные отношения Конституционного суда и всех остальных
органов власти. Когда мы только его создавали, он задумывался для баланса
власти. Это был прорыв, был создан 33-й Конституционный суд в Европе.

То, что сейчас творится, это политический беспредел. Это часть иностранной
кадровой интервенции, когда пытаются просто ликвидировать органы, которые
действуют не по указанию интервентов. То есть Конституционный суд идет к
ликвидации.

Неконституционное законотворчество провело к тому, что в правовое поле
внедрился правовой нигилизм, Конституция перестала работать. Конституцию не
выполняют государственные органы, включая команду президента и лично Владимира
Александровича. В обход Конституции формировались рудиментные органы. Разве
можно говорить о независимом государстве, если какая-то другая страна требует,
что мы сделали то-то и то-то?

По указанию западных стран создали эти так называемые антикоррупционные органы,
которые работают на кадровых интервентов. Такое антикоррупционное
законодательство никогда не сможет прекратить мародерство, потому что органы,
которые созданы в его рамках, сами занимаются этим мародерством. Почему нормы о
НАПК признаны не конституционными - потому что созданы условия для того, что
НАПК превратились в сыщиков, потому что они получили право на конфискацию
имущества не только декларантов, но членов семьи - детей, тещи, тети, дяди. Их
решения, аналогом которых является приговор суда, по сути своей нарушают
принцип презумпции невиновности в нашем обществе.

Решение Конституционного суда никто не читал, его не обсуждали профессионалы.
Но интервенты дали задание волонтерам жечь Конституционный суд, а сейчас
говорят о ликвидации судей. Мы вернулись в 30-е годы, когда чиновников за людей
не считают, видят субъект контрибуции. Где это видано, что если чиновник не
указал, сколько его теща потратила на лечение, - он коррупционер? Это не права
человека, это беспредел. А идеологи этого беспредела подвешивают чиновников на
крючок, чтобы он не отклонялся от навязанного ими курса. Это делают синдикаты,
включая Сороса и его птенцов, которые заселились у нас на полях и
распространяются, как короновирус. Они ведут страну в пропасть, мы уже одной
ногой там. Общество расколото, идеологии нет, политики суверенной нет.

Что это такое, что президент вносит законопроект и открыто говорит, что для
него "до фени" Конституция, что нужно убрать судей, которые принимают решения,
которые не нравятся ему и его покровителям?

Решение КС обязательно для всех: для чиновников и тем более для главы державы А
невыполнение решения Конституционного суда - это по сути 311 статья Уголовного
кодекса - измена родине. В международной практике подобного рода действия
означают, что президент узурпирует власть, а он не монарх, мы живем в
республике. Такого не бывает, чтобы не выполнять решения суда и за это ничего
не было. Это позорное пятно на правовом поле не на один год.

Я только надеюсь на то, что это детская мальчишечья горячность, отсутствие
опыта и компетенции. Но пройдут выходные и президент одумается, вернется в
правовое поле и не будет устраивать государственный переворот".
