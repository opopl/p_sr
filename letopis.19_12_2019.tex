% vim: keymap=russian-jcukenwin
%%beginhead 
 
%%file 19_12_2019
%%parent dec_2019
 
%%url https://www.facebook.com/permalink.php?story_fbid=593408844754150&id=100022551185600
 
%%author 
%%author_id 
%%author_url 
 
%%tags 
%%title 
 
%%endhead 
\section{19-12-2019}
\Purl{https://www.facebook.com/permalink.php?story_fbid=593408844754150&id=100022551185600}

Правовая основа войны на Донбассе.
Если мы говорим о войне серьезно, тогда наряду с причиной (о которой пока умолчим)  мы должны так же чётко понимать и то, что стало поводом для войны на Донбассе. 
Российская пропаганда говорит нам, что поводом для войны стали «скачущие на майдане» националисты. У нас другое мнение.
Поводом к войне на Донбассе стали события на  выезде из Славянска на лесной опушке возле населенного пункта Семеновка весной 2014 года, где неизвестными была расстреляна группа украинских работников СБУ. Это было нападение на украинское государство.  
Что же такое "нападение на государство", давайте разберемся в этом поподробнее.
После Революции Достоинства на Донбассе (и в других местах)  начали происходить захваты административных зданий, а так же зданий силовых структур.  Наше законодательство устроено таким образом, что для оценки этих действий требует рассмотреть как объективную так и субъективную сторону преступления.
Внимание!!   Захват здания - это всего лишь факт (объективная сторона преступления) , но закон требует исследовать и вторую сторону медали (а именно - необходимо узнать, что конкретно человек думал, производя этот захват здания). Для того чтобы исследовать субъективную сторону преступления, компетентные  органы всегда ведут переговоры с захватчиками - спрашивают зачем они произвели этот захват и чего хотят добиться таким способом. На базе полученных фвктов ведутся дальнейшие переговоры, захватчикам предлагаются альтернативные формы решения проблемы и оценивается, насколько преступными и опасными являются действия активистов.
Так например первый захват СБУ в Донецке произошел 16.03.2014 г. По этому факту велись переговоры и была достигнута договоренность, что митингующие могут частично занять здание ОГА, но при этом покинут здание СБУ.  Эта договоренность была выполнена и на этом основании за первый захват здания СБУ никто не понес никакого наказания.  
Таким образом, переговоры - это очень важный элемент общения народа и государства, особенно в кризисных ситуациях.
12  апреля 2014 года были захвачены здания СБУ и милиции в Славянске и Краматорске.  По этому факту государством была назначена группа переговорщиков из СБУ; все фактически должно было повторить мартовскую ситуацию в Донецке: переговоры и выход на взаимную договоренность. Правда подразделение СБУ прикрывали военные на БТР-ах, поскольку захват зданий производили на этот раз люди в камуфляже и с оружием (а вовсе не так как в Донецке, где это делали штатские и без оружия). Несмотря на это - все разумные люди надеялись на переговоры и разумное разрешение ситуации.
13 апреля 2014 года группа СБУ остановилась у населенного пункта Семеновка, чтобы обсудить с прикрывавшими их военными действия армии в случае возникновения внештатной ситуации. В это время из леса выбежала группа вооруженных людей и открыла огонь по сотрудникам СБУ из автоматического оружия. В результате между сотрудниками СБУ и нападавшими завязался жестокий бой.  В этом бою был убит один сотрудник СБУ и четыре ранены.
На протяжение всего боя аримя стояла без движения и не вмешивалась в события. Почему так произошло??
Дело в том, что нападавшие отлично знали законодательство: подразделение СБУ альфа было уполномочено вести переговоры с лицами захватившими здания в Славянске. Если бы захватчики пошли на переговоры - тогда удалось бы выйти на какие-то договоренности; если бы из захваченных зданий начали стрелять, тогда группа СБУ идентифицировала бы факт нападения на государство. Важный момент - если захватчик идет на переговоры, значит он признает закон страны в которой он находится и тогда его действия квалифицируются по законодательству; если же захватчик открывает огонь по группе переговорщиков, назначенных государством - тогда он выходит за рамки уголовного кодекса страны и фиксируется факт нападения на государство. Если зафиксирован факт нападения на государство, тогда это государство получает правовую возможность задействовать армию.
Аналогичный случай был в США, когда одна религиозная секта "оккупировала" деревушку в Америке. У правохранителей появились подозрения, что секта нарушает закон и американское государство назначило переговорщиками двух полицейских. Когда полицейские прибыли в назначенное место, сектанты неожиданно открыли огонь на поражение и убили обоих полицейских. После этого американское правительство подтянуло к деревушке армию и тяжелую военную технику и произвело полную и безжалостную зачистку. 
Почему? Потому что ни в коем случае нельзя стрелять в назначенных государством переговорщиков, такая стрельба всегда квалифицируется как факт нападения на государство со всеми вытекающими последствиями.
Поэтому когда нам говорят, что захваты административных зданий были и во Львове и в Харькове, почему же дескать армия Украины была задействована только на Донбассе? Ответ прост - потому, что ни во Львове, ни в Харькове никто не посмел стрелять в назначенных государством переговорщиков!  Таким образом, факт нападения на государство был зафиксирован только на Донбассе, со всеми вытекающими отсюда последствиями.
Почему же там, возле Семёновки на лесной опушке,  армия стояла и безучастно смотрела как методично убивают сотрудников СБУ? Увы! Это случилось потому, что не было понятно: нападавшие из леса и захватившие здания - это одни лица или разные. Если бы стрельба началась из захваченных в Славянске зданий по переговорщикам, тогда немедленно был бы идентифицирован факт нападения на государство и армия тот час бы вмешалась. Однако, стреляли неизвестные из леса, и закон здесь был бессилен: армия защищает государство, но не вмешивается в разборки между гражданами. Поэтому командир военного подразделения медлил с приказом, пока - наконец - у одного бойца не выдержали нервы: без получения приказа, он развернул башню своего БТРа и открыл огонь из крупнокалиберного пулемета по нападавшим. После этого толпа нападавших немедленно разбежалась в разные стороны и бой был закончен.
Факт нападения на государство был идентифицирован несколько позже и следующим образом: когда бой закончился и армия уехала с места события, когда были забраны раненые СБУшники -  на место происшествия явились 8-10 человек из числа лиц захвативших здания в Славянске, они осмотрели место события и быстро удалились. Этот факт зафиксировала милиция, а так же в распоряжении СБУ появились записи телефонных переговоров преступников.  Анализ этой доказательной базы позволял сделать вывод, что между захватчиками зданий и стрелявшими из леса есть прямая связь. 
Поскольку после убийства переговорщиков преступники не явились с повинной, был зафиксирован факт нападения на государство и Турчинов объявил о начале АТО.  Армия получила полное право на применение оружия.
Мы видим, что в этой логике к действиям государства не возможно предъявить никаких притензий.
