% vim: keymap=russian-jcukenwin
%%beginhead 
 
%%file 31_03_2021.fb.fb_group.story_kiev_ua.1.ukr_hleb.pic.3.cmt
%%parent 31_03_2021.fb.fb_group.story_kiev_ua.1.ukr_hleb
 
%%url 
 
%%author_id 
%%date 
 
%%tags 
%%title 
 
%%endhead 

\iusr{Semyon Belenkiy}
А это где то на Кожемяцкой?

\iusr{Петр Кузьменко}
\textbf{Semyon Belenkiy} именно так. Про этот магазин в одном из своих постов хорошо написала наша одногруппница \textbf{Любов Огородня}.

\iusr{Любов Огородня}
\textbf{Semyon Belenkiy} це на розі Гончарної і Воздвиженської, урочище Гончари - Кожум'яки

\iusr{Semyon Belenkiy}
\textbf{Любов Огородня} спасибо

\iusr{Раиса Володарская}
\textbf{Любов Огородня} Нет

\iusr{Любов Огородня}
\textbf{Раиса Володарская} що ні???

\iusr{Раиса Володарская}
\textbf{Любов Огородня} Угол Гончарной и Воздвиженский - на этом фото. На Вашем фото и на этом - разные места.

\ifcmt
  ig https://scontent-frx5-1.xx.fbcdn.net/v/t1.6435-9/167832357_1166835503755942_1395545383289029730_n.jpg?_nc_cat=105&ccb=1-5&_nc_sid=dbeb18&_nc_ohc=lh-FEpiRCfAAX969yqc&_nc_ht=scontent-frx5-1.xx&oh=876eb370982adc610fa9098ae0f6b0e2&oe=61DF2081
  @width 0.3
\fi

\iusr{Любов Огородня}
\textbf{Раиса Володарская} 

ви впевнені???? Навпроти цієї піонерської кімнати, яка знаходиться на розі
Гончарної і Воздвиженської,  @igg{fbicon.beaming.face.smiling.eyes}  розташований хлібний магазин. Відмінність лише в
тому, що це фото знизу від хлібного,, а де хлібний - від Піонерської кімнати.
Або - парна сторона і непарна @igg{fbicon.beaming.face.smiling.eyes} 

\iusr{Любов Огородня}
ще таке фото є

\ifcmt
  ig https://scontent-frt3-1.xx.fbcdn.net/v/t1.6435-9/167658758_4031664703567895_1185099551803667325_n.jpg?_nc_cat=107&ccb=1-5&_nc_sid=dbeb18&_nc_ohc=bgZX74C4oG8AX9Gz9KN&_nc_ht=scontent-frt3-1.xx&oh=00_AT8dYkj2pctFa5iPagBcue6zVWADyUDrmh5iRnGtJZxszg&oe=61DBBB56
  @width 0.3
\fi

\iusr{Любов Огородня}
або таке

\ifcmt
  ig https://scontent-frt3-1.xx.fbcdn.net/v/t1.6435-9/167251807_4031674580233574_250746801266590218_n.jpg?_nc_cat=102&ccb=1-5&_nc_sid=dbeb18&_nc_ohc=OgeJXy0kX1YAX9IWms7&_nc_ht=scontent-frt3-1.xx&oh=00_AT9jgGCGFEkuF_a0-YUVpll6_6RAR5ESh_iwMi_pY_FIpQ&oe=61DB592B
  @width 0.3
\fi

\iusr{Алина Маркина Сыченко}
\textbf{Любов Огородня} хлібний магазин знаходиться праворуч вдалині, де людина стоїть. І там будівля закруглена. А біленький - не він.
Я кожен день повз нього ходила до школи і назад, і купляли там трьохкопієчні булочки...

\iusr{Любов Огородня}
\textbf{Алина Маркина Сыченко} біленький, то піонерська кімната була.

\iusr{Раиса Володарская}

Коли я там жила (від народження до 11 років, з 1948 р. по 1959 р.), на
Воздвиженській, 56, на розі цих вулиць було дільниче відділення міліції. І
працював дільничий міліціонер за прізвищем Семілєтка, його знала вся вулиця.
Піонерська кімната, мабуть, була в інший період. Хлібного магазину в той час
там не було. Мій дім N56 був зовсім поруч з Гончарною. А за хлібом ходили в
сторону Дегтярної, там був невеличкий магазинчик.

\iusr{Алина Маркина Сыченко}
\textbf{Раиса Володарская} так, то був наш дуже маленький магазинчик на розі моєї Дегтярної та Ладо Кецховелі (Воздвиженської), у ньому не тільки хліб продавали.

\iusr{Любов Огородня}
\textbf{Алина Маркина Сыченко} на розі Гончарної і Ладо Кецховелі

\iusr{Любов Огородня}
\textbf{Алина Маркина Сыченко} я жила на Гончарній. Не плутайте! Хлібний магазин завжди був там - Ладо Кецховелі- Гончарна

\iusr{Алина Маркина Сыченко}
\textbf{Любов Огородня} я про нього!

\iusr{Любов Огородня}
\textbf{Алина Маркина Сыченко} то він на початку Гончарної, не Дегтярної

\iusr{Алина Маркина Сыченко}
\textbf{Любов Огородня} правильно! Ми жили на Дегтярній, а моя мама деякий час працювала на Гончарній у Київоблгазі. Хлібний магазин у 80-х роках був на правому розі.

\iusr{Любов Огородня}
\textbf{Алина Маркина Сыченко} упс. А ким мама працювала? може я її знаю @igg{fbicon.thinking.face} . А хлібний завжди був на одному місці, незмінний @igg{fbicon.beaming.face.smiling.eyes} . Якщо йти з В. Валу - він з правого боку, якщо з Андріївського Узвозу - з лівого @igg{fbicon.beaming.face.smiling.eyes} 

\iusr{Алина Маркина Сыченко}
\textbf{Любов Огородня} а я мала на увазі, якщо входити з Воздвиженської на Гончарну, то магазин був праворуч. А той біленький тоді (у кінці 70-х - 80 ті) вже був заколочений дошками...
Мама була Маркіна Ірина Василівна. Спочатку вона там працювала у прийомній
начальника (Грандковського Леонарда, я пам'ятаю!), а потім у бухгалтерії. Але
тимчасово, бо мама була архітектором і пішла працювати до КІБІ.

\iusr{Любов Огородня}
\textbf{Алина Маркина Сыченко} 

Грандковський Леонід Іванович! Найсвітліша і порядна людина. Я у нього
секретарем працювала з 1982 року @igg{fbicon.face.smiling.eyes.smiling}  по 1987.

\iusr{Алина Маркина Сыченко}
\textbf{Любов Огородня} 

тоді і я Вас згадала, Ви були дуже молода і схожа на артистку Олену Корєнєву! І
Ви вчили мою маму нормально друкувати!

А мама моя як справжня дружина військового класно варила самогон та ворожила на
картах! І до неї завжди ходили дівчата з бухгалтерії: Люда, Галина Степанівна
Гайдаєнко, жіночка, яка жила у Боярці, заступник главбуха Валентини
Григорівни... Ще Неля Теодорівна... Інше не згадаю! @igg{fbicon.smile} 

\iusr{Любов Огородня}
\textbf{Алина Маркина Сыченко} 

ооо, Неля Теодорівна, шикарна жінка. З приводу ворожіння не пам'ятаю, напевно
дуже молода була і мене не запрошували, а я й не цікавилася @igg{fbicon.beaming.face.smiling.eyes} . А Люда касир,
напевно? Люда Огородня @igg{fbicon.beaming.face.smiling.eyes}  Я знаю, що у нас була Ольга Іванівна, яка також
віщункою була, начебто

\iusr{Петр Кузьменко}
Як у вас тут, га гілці коментарів, затишно та по-родинному тепло, земляки з \enquote{малої Батьківщини}!

\iusr{Yelena Rozengurtel}

Мне не у кого спросить уже, к сожалению. Мне кажется это тот хлебный в котором
работал мой прадед Соломон.

\iusr{Semyon Belenkiy}
\textbf{Yelena Rozengurtel} 

Привет Лена, Ася с уверенностью говорит, что твой прадед работал в хлебном на
Нижнем Валу, 21 и даже был его директором. Уточни у мамы. Фото магазина есть на
этом посту.

\iusr{Yelena Rozengurtel}
\textbf{Semyon Belenkiy}, 

спасибо! Вот по рассказам я знала в какую сторону он шёл с Боричев Тока, и
почему-то всегда думала чтоив том что на углу. Там хлебный был в моём детстве
тоже.

\iusr{Yelena Rozengurtel}

Оказывается Соломон работал в том что на Жданова/Сагайдачного. Папа помнит всё
его рассказы, мама говорит что пешком он на Валы не ходил.

\iusr{Алина Маркина Сыченко}
\textbf{Yelena Rozengurtel} есть у кого спросить! @igg{fbicon.smile} 
Мы уехали оттуда в 1983, когда там уже массово дома опустевали.
Грустные времена, ведь детство на этих улочках было счастливым и почти сказочным! @igg{fbicon.hearts.two} 

\iusr{Yelena Rozengurtel}
\textbf{Алина Маркина Сыченко}, Вы не так поняли. Спросить не у кого в каком именно из хлебных работал мой прадед. Я отдельно коммент написала. А уезали мы с Боричева Тока в 1992 году.

\iusr{Алина Маркина Сыченко}
Теперь понятно! @igg{fbicon.smile} 

\iusr{Марина Резник}
Родной магазинчик!
Он ещё в 80х имел место быть, только уже с водочкой в ассортименте.
Аромат внутри настолько крепкий и трудно описуемый, казалось он впитал запахи Воздвиженской Кожемяцкой за целый век.

\iusr{Марина Бортницкая}

Я в этот магазин бегала за булочками по 3копейки и арнаутом, а ещё там покупали
булочки с повидлом, и конечно хлеб Украинский пахучий теплый, и паляниця с
корочкой, как это было давно


\iusr{Татьяна Панасюк}
Мммм... Арнаут! Скучаю по его вкусу (

\iusr{Нила Арсенькина}
Этот могазин находился на углу Взжвижэнской и Гончарной

\ifcmt
  ig https://i2.paste.pics/bb1dee73e945871e7a282e1565ffcb36.png
  @width 0.2
\fi

\iusr{Adel Friedman}
\textbf{Нила Арсенькина} 

совершенно верно! Угол Гончарной и Воздвиженской! Я сама жила на Гончарной 26
этот дом возле лестницы которая поднималась на Десятинный переулок! Хорошо
помню этот хлебный магазинчик !

\iusr{Петр Кузьменко}

Когда я бегал в этот магазин мне было максимум 9 лет. Хорошо помню где он был
расположен. Спустил название местности. Старею. Память стала подводить...
Благодарю земляков из моих пары кварталов сердца Города за уточнение.
