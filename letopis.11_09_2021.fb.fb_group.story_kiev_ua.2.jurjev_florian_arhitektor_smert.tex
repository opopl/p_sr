% vim: keymap=russian-jcukenwin
%%beginhead 
 
%%file 11_09_2021.fb.fb_group.story_kiev_ua.2.jurjev_florian_arhitektor_smert
%%parent 11_09_2021
 
%%url https://www.facebook.com/groups/story.kiev.ua/posts/1750304981832962/
 
%%author_id fb_group.story_kiev_ua
%%date 
 
%%tags arhitektura,gorod,jurjev_florian.arhitektor.kiev,kiev,smert,ukraina
%%title Сьогодні, 11 вересня, на 93-у році життя пішов у вічність архітектор Флоріан Ілліч Юрьєв
 
%%endhead 
 
\subsection{Сьогодні, 11 вересня, на 93-у році життя пішов у вічність архітектор Флоріан Ілліч Юрьєв}
\label{sec:11_09_2021.fb.fb_group.story_kiev_ua.2.jurjev_florian_arhitektor_smert}
 
\Purl{https://www.facebook.com/groups/story.kiev.ua/posts/1750304981832962/}
\ifcmt
 author_begin
   author_id fb_group.story_kiev_ua
 author_end
\fi

Сьогодні, 11 вересня, на 93-у році життя пішов у вічність архітектор Флоріан
Ілліч Юрьєв. Він відомий багатьом киянам як авто «Літаючої тарілки» на
Либідській площі, яка була зведена як частина будівлі Інституту
науково-технічної інформації. 

\ifcmt
  ig https://scontent-frx5-1.xx.fbcdn.net/v/t1.6435-9/241670918_913542915902078_4553697884809646099_n.jpg?_nc_cat=111&ccb=1-5&_nc_sid=825194&_nc_ohc=UfL45yDEgbQAX8tamxV&_nc_ht=scontent-frx5-1.xx&oh=852548cd2c10535c036b7dcbd9729411&oe=6165334E
  @width 0.4
  %@wrap \parpic[r]
  @wrap \InsertBoxR{0}
\fi

Флоріан Юрьєв причетний до багатьох київських проектів. Він був співавтором
проекту наземного виходу станції метро «Хрещатик» і ресторану «Метро», у
співавторстві з Б. Довганем ним був створений герб Києва 1966-1995рр. (з листом
каштану), Юрьєв виконав проект житлового масиву № 17 в мікрорайоні «Нивки», був
співавтором меморіальної дошки Аллі Горській на вул. Терещенківській і
пам’ятника репресованим художникам на Вознесенському узвозі.
