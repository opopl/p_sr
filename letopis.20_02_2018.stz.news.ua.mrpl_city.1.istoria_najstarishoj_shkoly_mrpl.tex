% vim: keymap=russian-jcukenwin
%%beginhead 
 
%%file 20_02_2018.stz.news.ua.mrpl_city.1.istoria_najstarishoj_shkoly_mrpl
%%parent 20_02_2018
 
%%url https://mrpl.city/blogs/view/istoriya-najstarishoi-shkoli-mariupolya
 
%%author_id demidko_olga.mariupol,news.ua.mrpl_city
%%date 
 
%%tags 
%%title Історія найстарішої школи Маріуполя
 
%%endhead 
 
\subsection{Історія найстарішої школи Маріуполя}
\label{sec:20_02_2018.stz.news.ua.mrpl_city.1.istoria_najstarishoj_shkoly_mrpl}
 
\Purl{https://mrpl.city/blogs/view/istoriya-najstarishoi-shkoli-mariupolya}
\ifcmt
 author_begin
   author_id demidko_olga.mariupol,news.ua.mrpl_city
 author_end
\fi

\ii{20_02_2018.stz.news.ua.mrpl_city.1.istoria_najstarishoj_shkoly_mrpl.pic.1}

Не всім маріупольцям відомо, що колегіум-школа № 1, яка роз\hyp{}ташована в
історичному центрі нашого міста - на тихій вулиці Грецькій, є одним з
найстаріших навчальних закладів Маріуполя. За свою багату історію школа пройшла
декілька етапів у своєму розвитку: від існування і діяльності в її стінах
Маріїнської жіночої гімназії до отримання сучасного статусу \enquote{колегіум-школи}. 

\ii{20_02_2018.stz.news.ua.mrpl_city.1.istoria_najstarishoj_shkoly_mrpl.pic.2}

Завдяки діяльності вчителя історії Галини Матвіївни Штанько у школі в 1975 році
було засновано музей історії, що є унікальним здобутком не тільки школи, але й
всього міста. Масштабна пошукова робота Галини Матвіївни, її учнів та інших
вчителів дозволила зібрати цінні документи, безліч світлин, історичні факти,
присвячені історії школи з XIX ст. Сьогодні музей історії відкритий для
відвідувачів і дотепер вражає цікавими експозиціями та унікальною історією.

\ii{20_02_2018.stz.news.ua.mrpl_city.1.istoria_najstarishoj_shkoly_mrpl.pic.3}

У музеї налічується близько однієї тисячі експонатів, серед яких – документи,
що стосуються створення Маріїнської жіночої гімназії, справжній учнівський
квиток гімназистки, паспорт класної наглядачки, світлини деяких викладачів і
учениць, а також щоденникові записи, похвальні листи, свідоцтва, грамоти,
предмети побуту, відтворена шкільна форма того часу, підручники дореволюційного
періоду і навіть зразок кондуїдського списку (так званої штрафної книги).
Екскурсії проводяться не тільки для учнів колегіум-школи № 1, але й для
школярів інших навчальних закладів, гостей міста вчителем української мови та
літератури Ганною Станіславівною Дежец. Завдяки ознайомленню з матеріалами
музею історії школи та спілкуванню з Ганною Станіславівною нам вдалося
дізнатися про історію найстарішої школи Маріуполя.

\ii{20_02_2018.stz.news.ua.mrpl_city.1.istoria_najstarishoj_shkoly_mrpl.pic.4}

Отже, спочатку в будівлі нинішньої школи, побудованій в 1894 році, на базі
комор купця Момбеллі розташовувалася Маріїнська жіноча гімназія. Виникла вона
завдяки зусиллям видатного діяча Маріуполя, історика, етнографа, першого
дослідника історії Приазов'я Феоктиста Аврамовича Хартахая, який подав
спеціальну заяву міській думі для того, щоб створити чоловічу і жіночу
гімназії. Згідно з його клопотанням була створена Чоловіча Олександрівська
гімназія, а також 20 березня 1876 р. міська дума розглянула і затвердила акт
про те, що необхідно створити й жіночу гімназію.

16 вересня 1876 р. відкрилася Маріупольська жіноча гімназія під начальством
А. А. Генглез. Феоктист  Хартахай очолив педагогічну раду гімназії. Кілька років
гімназія розміщувалася в орендованих приміщеннях, а з 1894 р. – отримала власне
приміщення нинішньої школи. Складалася гімназія з підготовчого і 7 основних
класів. Поступово кількість класів збільшилася до 20: чотири підготовчих, сім
основних, сім паралельних і два додаткових. Незабаром був відкритий додатковий
(восьмий) педагогічний клас, де готувалися домашні вчительки та виховательки,
вчителі міських і сільських початкових училищ.

У 1880 р. було подано клопотання царю Олександру II про те, щоб назвати
Маріупольську жіночу гімназію Маріїнською на честь імператриці Марії
Олександрівни, на що було отримано згоду.

Навчання в гімназії було платним. Для підготовчих класів - 15 рублів, для
основних - 20. Але згодом плата змінювалася і доходила навіть до 100 рублів.
Так, у 1893 р. плата за навчання за підготовчі класи складала 35 рублів, а за
восьмий додатковий клас – 100 рублів. Також додавалося по 5 рублів за кожну
мову, що вивчалася в гімназії. Якщо врахувати, що в той час робітники
отримували платню 60 копійок в день, або 219 рублів на рік, а середня заробітна
плата інженера становила 547,5 рублів в рік, то неважко зробити висновок, що
плата за навчання в гімназії була досить високою.

\ii{20_02_2018.stz.news.ua.mrpl_city.1.istoria_najstarishoj_shkoly_mrpl.pic.5}

Викладання велося на високому рівні. Вивчали дівчата, крім предметів основного
курсу, рукоділля, танці, гігієну. Багато уваги приділялося загальному вихованню
дівчаток. Обов'язковим було щовесни садити дерева. Гімназистки брали участь в
спортивних змаганнях, спектаклях, їздили на екскурсії. Ознайомившись зі
звичаями та статутами гімназії, можна сміливо зробити висновок, що сучасним
дівчатам пощастило набагато більше, ніж гімназисткам, адже за дівчатками
наглядали так звані класні наглядачки навіть у вільний від занять час. Всі
огріхи дівчат записували в так званий кондуїдський список (штрафна книга). У
музеї школи зберігається сторінка з кондуїдського списку, в якій записано, що
учениця гімназії Сангвінеті-Лоретті гуляла в Олександрівському сквері після
заходу сонця, що є несумісним з честю вихованих дітей і принижує честь
навчального закладу.

\ii{20_02_2018.stz.news.ua.mrpl_city.1.istoria_najstarishoj_shkoly_mrpl.pic.6}

Дівчатка ходили обов'язково в однаковій формі та однакових туфельках. За
старанне навчання могли навіть звільнити від плати. Дівчата, які старанно
вчилися, отримували золоті й срібні медалі та могли працювати домашніми
наставницями. Іншим випускницям надавалося звання вчительки народних училищ.
До 300-річчя династії Романових було підготовлено Свідоцтво особливого зразка
за старанне навчання в гімназії, що давало їм право займатися педагогічною
справою як в приватному, так і в державному секторі.

У 1912 році був добудований другий поверх. Свого часу в закутках шкільного
горища був знайдений відреставрований письмовий стіл директора або завуча
гімназії. Сьогодні він займає чільне місце в музеї історії колегіум-школи № 1. 

\ii{20_02_2018.stz.news.ua.mrpl_city.1.istoria_najstarishoj_shkoly_mrpl.pic.7}

Перед Першою світовою війною в початкових училищах Маріуполя успішно працювали
вихованки педагогічного класу Маріїнської жіночої гімназії. Ось тільки кілька
прізвищ: Варвара Леонтіївна Буслова (Гоголівське 1-е училище), Валентина
Леонідівна  Голоцинська (Гоголівське 2-е училище), Марія Іванівна Способіна
(10-е міське училище), Ольга Георгіївна Коваленко (11-е міське училище), Ірина
Йосипівна Давидова (20-е імені імператора Олександра 1 початкове училище),
Олександра Петрівна Солохай (Вознесенське початкове училище в порту), Олена
Костянтинівна Сальченко (Єкатерининське початкове училище) та інші.

Після революції 1917 року гімназія була закрита, а в 1921 році тут же
відкрилася Єдина трудова зразкова школа № 1. Тепер цей навчальний заклад
відвідували не тільки дівчата, а й хлопці. Викладання велося українською мовою.
У музеї школи зберігаються листи випускників трудової школи, унікальні знімки
уроків.

У 1941 році під час Другої світової війни в приміщенні школи розташувався
радянський евакогоспіталь № 1051, а в період окупації – німецький шпиталь.
Окреме місце в експозиціях музею школи займають стенди зі світлинами видатних
учнів школи – віртуозної льотчиці Євдокії Лобко та загиблого в Афганістані
Олександра Кільового.

Восени 1943 року, йдучи з міста, нацисти спалили будівлю майже дощенту. Вона
була відновлена після війни, а в 1951 році тут відбувся перший післявоєнний
випуск.

\ii{20_02_2018.stz.news.ua.mrpl_city.1.istoria_najstarishoj_shkoly_mrpl.pic.8}

У 2005 році школі було присвоєно статус Маріупольського нав\hyp{}чально-виховного
комплексу \enquote{Колегіум-школи} № 1. Сьогодні школа, що веде свій родовід
від Маріїнської жіночої гімназії, живе яскравим, насиченим подіями життям.
Вчителі продовжують поповнювати музей грамотами та світлинами своїх
випускників, які завжди були й є головною гордістю школи.

\ii{20_02_2018.stz.news.ua.mrpl_city.1.istoria_najstarishoj_shkoly_mrpl.pic.9}
\ii{20_02_2018.stz.news.ua.mrpl_city.1.istoria_najstarishoj_shkoly_mrpl.pic.10}
