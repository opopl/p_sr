% vim: keymap=russian-jcukenwin
%%beginhead 
 
%%file slova.knjaz
%%parent slova
 
%%url 
 
%%author 
%%author_id 
%%author_url 
 
%%tags 
%%title 
 
%%endhead 
\chapter{Князь}

%%%cit
%%%cit_head
%%%cit_pic
%%%cit_text
За кілька років після хрещення \emph{князь} Володимир Святий вирішив розмежувати на
Русі небесне і земне. У 995/996 році він видав Десятинній церкві грамоту на
отримання десятини, яка до 1011 року перетворилася на перший церковний статут.
Незважаючи на свою назву, документ стосувався устрою всієї державної влади,
оскільки розмежував юрисдикцію світських і церковних судів, а також віддав
священикам контроль мір і ваг. Всі інші питання залишилися у віданні \emph{князя}.
Пізніше свій церковний статут видав \emph{князь Ярослав}, а потім канонічне право
остаточно відокремилося від державного
%%%cit_comment
%%%cit_title
\citTitle{Сергій Громенко: Українській Конституції – 2300 років}, 
, gazeta.ua, 26.06.2021
%%%endcit

