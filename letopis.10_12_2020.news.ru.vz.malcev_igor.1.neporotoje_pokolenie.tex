% vim: keymap=russian-jcukenwin
%%beginhead 
 
%%file 10_12_2020.news.ru.vz.malcev_igor.1.neporotoje_pokolenie
%%parent 10_12_2020
 
%%url https://vz.ru/opinions/2020/12/10/1074360.html
 
%%author Мальцев, Игорь
%%author_id malcev_igor
%%author_url 
 
%%tags covid,psihologia
%%title Непоротое поколение сходит с ума
 
%%endhead 
 
\subsection{Непоротое поколение сходит с ума}
\label{sec:10_12_2020.news.ru.vz.malcev_igor.1.neporotoje_pokolenie}
\Purl{https://vz.ru/opinions/2020/12/10/1074360.html}
\ifcmt
	author_begin
   author_id malcev_igor
	author_end
\fi

\ifcmt
pic https://img.vz.ru/upimg/experts/4026_26.jpg
width 0.2
\fi
Игорь Мальцев, писатель, журналист, публицист

Галактика Андромеды обязательно столкнется с нашей галактикой через четыре
миллиарда лет – и тем не менее находятся идиоты, которые утверждают, что
прогресса не существует.

Конечно, еще можно вспомнить, что и унитаз-то появился всего лет сто пятьдесят
назад – и тогда мы получим окончательную клиническую картину прогресса
человечества.

Ну действительно, не вспоминать же нам как всего сто лет назад человечество
было немножко занято умиранием от испанского гриппа – миллионов так до ста
двинуло кони в мучениях, и это практически после того, как в Первую мировую
убили взрослых белых мужчин миллионов одиннадцать. И в какой гнусной нищете
оказались выжившие – что сильно подстегнуло такое понятие, как детский труд,
например. Тут не до селфи и «Тик-тока», как вы понимаете. 

А ведь потом была еще и Вторая. И разруха. Если думаете, что она была только
тут, то глубоко ошибаетесь – достаточно было посмотреть на Берлин в 1990 году,
или на Глазго с Будапештом чуть раньше.

Я, в общем, к тому, что более-менее люди стали жить по все стороны границы
реально нормально и без особых провалов в nutrition всего-то лет тридцать как.
(Про совсем отсталые страны мы не говорим). 

Но что-то за эти тридцать лет все такие стали нервические и тонкие, что некого
уже и педерастом назвать – сразу в истерику впадают. Еще месяцок ковида – и
скоро его уже впечатлительные граждане начнут называть самой большой
геополитической катастрофой в истории. И начнут торговать майками «Я пережил
2020».

\ifcmt
pic https://img.vz.ru/upimg/m10/m1074360.jpg
fig_env wrapfigure
width 0.4
\fi

Ну, все уже видели эту идиотскую обложку Time с объявлением 2020-го самым
худшим годом ever. Господи, какая пошлятина. Конечно, здравомыслящие граждане,
которые ныне редки, как амурские тигры, навешали редакции своих впечатлений по
поводу глупости кавер-дизайна. На что редакция стала говорить, что, дескать,
«уже выросло поколение, которое ничего страшней месяца самоизоляции в жизни не
видело» – и, типа, эта обложка про них и для них. Ну что ж, можно их с этим
поздравить, или наоборот – выразить соболезнования. Ведь, как говорил герой
карикатуры в журнале Playboy 70-х годов, «моя следующая песня «Голод» – я
написал ее, когда двери моего «Кадиллака» заклинило, и я остался внутри на
целый час лишь с шампанским в моем мини-баре». 

Страшный 2020-й, конечно, сотворил ужасную вещь – оставил людей в локдауне
наедине с самими собой, что, как выяснилось, не каждый может пережить. 

Основной драйвер протеста против локдауна, например, в Германии – молодые люди,
которым проклятое фашистское государство запретило проводить ночные вечеринки.
«Как они посмели? Мы же молодые и мы не болеем!» – кричали они. И буквально
каждый вечер берлинские менты ездят разгонять нелегальные ночные тусы. Сейчас
стало похолодней слегка – вызовы стали пореже. То есть теперь у нас право на
вечеринку посреди чумки – это, типа, неотъемлемое право человека. Забыли мы
уже, как люди посвящали свою жизнь борьбе, например, за свободу совести, или
там борьбе за то, чтобы геев не кидали в тюрьму за их гейство, или – о боже –
за права пролетариев, про которых теперь забыли уже окончательно. Теперь
выросло непуганое поколение, на которое все надеялись, как на «первое непоротое
поколение», которое по причине своей непоротости одарит мир революционными
открытиями, невероятными прорывами мысли, титаническими стройками века.

А получилась – огромная стая нытиков, для которых все права человека
сконцентрировались в праве нарушать эпидемиологические ограничения. А все
прорывы мысли ограничились рисованием мемчиков на планшетике, построенном с
использованием рабского труда в других странах, на которые всем плевать. До
титанических строек дело так и не дошло. Все стройки – удел худых желтых мужчин
где-то в Китае. 

Но самый пугающий симптом – клиническая невозможность остаться наедине с самим
собой в изоляции. Причем эта изоляция – упакована едой, обеспечена
широкополосной связью, над головой не каплет, под окнами не расстреливают
евреев и джинсы привозит «Амазон».

Но как же это страшно! Это самый страшный год в их жизни. Какой же убогой
должна быть эта жизнь – чтобы вот то, что мы имеем, оказалось самым страшным
испытанием. Если тебе нечего делать наедине с собой – могу представить, как
пусто у тебя в голове. Из мыслительных процессов осталось только составление
комментов в Facebook.  

Со стороны все это выглядит так, словно «непоротое поколение» сходит с ума. Но,
как говорит публицист Ортега – «для этого надо, чтобы было с чего сходить». 
