% vim: keymap=russian-jcukenwin
%%beginhead 
 
%%file 27_10_2021.fb.lesev_igor.1.bajraktar
%%parent 27_10_2021
 
%%url https://www.facebook.com/permalink.php?story_fbid=4721428761221523&id=100000633379839
 
%%author_id lesev_igor
%%date 
 
%%tags bajraktar,donbass,ukraina,vojna,zelenskii_vladimir
%%title Как у Блогера запускали «Байрактар»
 
%%endhead 
 
\subsection{Как у Блогера запускали «Байрактар»}
\label{sec:27_10_2021.fb.lesev_igor.1.bajraktar}
 
\Purl{https://www.facebook.com/permalink.php?story_fbid=4721428761221523&id=100000633379839}
\ifcmt
 author_begin
   author_id lesev_igor
 author_end
\fi

Как у Блогера запускали «Байрактар»

История с боевым применением «Байрактара» на Донбассе вызвала прилив радости в
тех кругах, где обычно эти приливы случаются. Ну вы знаете эти круги в нашей
стране. Брось камень в лужу – и вот уже пошли круги перемоги.

\ifcmt
  ig https://scontent-lhr8-2.xx.fbcdn.net/v/t39.30808-6/249871189_4721428641221535_2215293988197993617_n.jpg?_nc_cat=101&ccb=1-5&_nc_sid=730e14&_nc_ohc=q_Y3qqkCPQYAX_1wtzZ&_nc_ht=scontent-lhr8-2.xx&oh=ce7261e9401d21bcb30eb87f91eb09eb&oe=617FE955
  @width 0.4
  %@wrap \parpic[r]
  @wrap \InsertBoxR{0}
\fi

Как перевернут «Байрактары» и «Джавелины» ход российско-украинской (не путать с
украинско-российской от Данилова и русско-российской от Арестовича) войны на
Донбассе, нам непременно расскажут по центральному ТВ. Ребята и девчата там
успешно избавились от кремлевских нарративов, которые, как оказалось, там все 5
лет цвели при Петре, и теперь лупят в лоб кристальную правду.

Но вот этот «Байрактар» - это больше для внутреннего пользователя. «Президент
мира» Зеленский оказался таким же «президентом мира», как и Порошенко. Правда,
Петру понадобилось на старте чуть меньше года, чтобы сообразить, что войну
безопаснее и корыстней имитировать, нежели вести по-настоящему.

Блогер в общем-то тоже не «воин света», и тырить на «большой стройке», схемах в
«Нефтегазе», атомке и УЗ ему гораздо комфортнее, нежели театрально имитировать
штурм Донецка. Хотя имитация для команды Тупых и недоразвитых – это их главный
программный конек. Кто-то еще помнит, что у нас в августе была «Крымская
платформа»? Год жужжали, надували щеки, и все только для того, чтобы опять
украсть на организации и услышать КВН-историю о бежащем через минные поля Крыма
маленьком мальчике.

Вот в этом весь Зеленский. Бегущий мальчик, покушение на Шефира, новый город на
Черном море, план трансформации Украины на 277 миллиардов долларов… Они даже не
пытаются врать красиво. Они просто несут в массы чушь, прикрывая свое банальное
воровство.

А «Байрактар» - это еще и конец истории о «скором достижении мира». Хомячкам
парить это уже не получится, поэтому пойдут истории о военных победах. А это
уже тема и электоральное поле Петра, где становится все теснее.

Отсюда, к слову, и вся шизофрения и в других направлениях, в том же газовом. У
нас тут российско-украинская война, поэтому мы не можем покупать у
страны-агрессора напрямую газ. Мы его будем покупать у Словакии. Да, это
дороже, но это цена войны и нашей независимости. А еще мы не можем допустить
потери транзита российского газа через нашу территорию, потому что у нас
российско-украинская война и враг ценой транзита оплачивает нам содержание
армии, благодаря чему мы и отражаем российскую агрессию.

Это не только биполярное расстройство, но и схематоз для причастных. Газ
украинской добычи продаешь как «словацкий» с наценкой в десятки раз. И русский
газ продаешь через офшорные прокладки. Т.е., берешь все тот же исключительно
ру-газ, а по докам проводишь его как «словацкий», и впариваешь ЖКХ-хомячкам с
кратной наценкой. Схематозик? Схематозище.

И «вечная война» - еще одна бабловая тема. 320 ярдов согласно бюджету на
следующий год. И цифра не просто космическая. Она космически закрытая. Любое
расследование в эффективности трат – это сразу же статья. Ну потому что там все
тайная тайна. Помните, как с нашими ментальными братьями из Афганистана вышло?
Оказывается, 2/3 боевых подразделений существовали только по зарплатным
ведомствам.

А чтобы хомячки не задавались вопросами, почему 320, а не хотя бы 220
миллиардов – нужны периодические «обострения». Но такие, осторожные. Чтобы без
радикальных последствий. Поэтому, «российско-оккупационные войска» (именно так
пишет Генштаб ВСУ) на Донбассе можно обстреливать, а эти же самые
российско-оккупационные войска в Крыму вдруг нельзя. Схема та же самая, как и с
газом. Покупать у России напрямую газ – нельзя, а качать ру-газ через Украину –
зя. Обстреливать ру-террористов в Горловке – зя, а в Джанкое – нельзя. Логика
где-то в стиле Данилова, где война у нас русско-украинская, но не
украинско-российская.

И вот этот «Байрактар» бьет не по «оккупантам» или «ополченцам» (тут уж кому
что ближе), а уводит инфо-повесточку от Блогера. А там у него все очень
печально. «Севпоток-2» вот-вот будет запущен. (И никто ведь из «зеленых» и 100
баксов не поставит, что этого не случится). Тема офшоров тоже еще не заглохла.
На дворе – ковид-жопа и сотни трупов ежедневно. И плюс отопительный сезон
начался чудесно, и еще более чудесными будут первые платежки.

Это все проблемы, которые надо решать. А ты и твое окружение – решалы только по
схемам. Поэтому, что? Правильно! Имитируем новую перемогу. Запускаем
«Байрактар» и еще пару дней можем тырить без оглядки на новые залеты.

\url{https://t.me/Lesev_Igor}
