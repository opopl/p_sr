% vim: keymap=russian-jcukenwin
%%beginhead 
 
%%file 19_03_2019.stz.news.ua.mrpl_city.1.vasyl_krjachok
%%parent 19_03_2019
 
%%url https://mrpl.city/blogs/view/vasil-kryachok-zhittevij-shlyah-osyayanij-muzikoyu
 
%%author_id demidko_olga.mariupol,news.ua.mrpl_city
%%date 
 
%%tags 
%%title Василь Крячок: життєвий шлях, осяяний музикою
 
%%endhead 
 
\subsection{Василь Крячок: життєвий шлях, осяяний музикою}
\label{sec:19_03_2019.stz.news.ua.mrpl_city.1.vasyl_krjachok}
 
\Purl{https://mrpl.city/blogs/view/vasil-kryachok-zhittevij-shlyah-osyayanij-muzikoyu}
\ifcmt
 author_begin
   author_id demidko_olga.mariupol,news.ua.mrpl_city
 author_end
\fi

\ii{19_03_2019.stz.news.ua.mrpl_city.1.vasyl_krjachok.pic.1}

Тридцять років назад у Маріуполі з'явився неповторний і самобутній камерний
оркестр \textbf{\enquote{Ренесанс}}, який сьогодні є справжньою прикрасою і невіддільною
частиною музичного життя Маріуполя. Його незмінний керівник, заслужений діяч
мистецтв України \textbf{Василь Михайлович Крячок} цього року відсвяткував ювілей – 65
років. Творчий і життєвий шлях Василя Михайловича, немов осяяний музикою,
вражає енергійною діяльністю, наполегливістю, винятковими організаторськими
здібностями і великою кількістю здобутків, завдяки яким маріупольці отримали
унікальну можливість – насолоджуватися у власному місті класичною музикою.
Василю Михайловичу вдалося створити висококваліфікований професійний камерний
оркестр, відомий не тільки в Україні, а й далеко за її межами, який гастролював
у Польщі, Угорщині, Словаччині, Росії. Про нашого героя, який вже давно отримав
визнання в українському мистецькому середовищі, міститься інформація в
\enquote{Енциклопедії сучасної України} та \enquote{Енциклопедії державних нагород України}.
Василю Михайловичу присвятив книгу доктор технічних наук, професор, академік АН
вищої школи України \emph{Ігор Володимирович Жежеленко} – \emph{\enquote{Диригент і його оркестр}}.
Василь Крячок щорічно невтомно дивує новими звершеннями, своїм талантом та
відданістю обраній справі.

\medskip
\begin{minipage}{0.9\textwidth}
\textbf{Читайте також:} \href{https://mrpl.city/blogs/view/mariupol-yak-na-doloni-z-oglyadovogo-majdanchika-vezhi-vidkrivayutsya-charivni-vidi}{Маріуполь як на долоні: з оглядового майданчика \enquote{Вежі} відкриваються чарівні види, VEZHA, mrpl.city, 15.03.2019}%
\footnote{\url{https://mrpl.city/blogs/view/mariupol-yak-na-doloni-z-oglyadovogo-majdanchika-vezhi-vidkrivayutsya-charivni-vidi}}
\end{minipage}
\medskip

Народився Василь Михайлович 1 березня 1954 року в селі Полянецьке, неподалік
від Умані в родині службовця і громадського діяча. Музичну освіту здобув в
Уманському державному музичному училищі і Астраханській державній консерваторії
на оркестровому факультеті, де отримав спеціальність \enquote{Тромбон}. Після
закінчення консерваторії та проходження строкової служби працював викладачем
Маріупольського музичного училища.

З 1982 року разом з дружиною, скрипалькою Ларисою Галлою, був зарахований за
контрактом в групу радянських військ у Німеччині, до військового оркестру, де
до 1988 року працював на різних музичних посадах, зокрема диригентом.

\ii{19_03_2019.stz.news.ua.mrpl_city.1.vasyl_krjachok.pic.2}

Повернувшись до Маріуполя, викладав у дитячій музичній шко\hyp{}лі № 1, пізніше
повернувся до Маріупольського державного музичного училища. Загалом Маріуполь
для музиканта став рідним, одним з найулюбленіших місць вважає Міський сад.
Коли Василь Михайлович працював у нашому місті, його не покидала мрія створити
перший у Маріуполі професійний оркестр. На щастя, цій мрії судилося
здійснитися. 17 квітня 1989 був заснований ансамбль старовинної музики
\enquote{Ренесанс}, до якого увійшли 15 молодих викладачів. З 1994 року оркестр
рішенням Маріупольської міської ради отримав статус муніципального професійного
творчого колективу, став бюджетним закладом і був включений в реєстр культурних
закладів Міністерства культури України. В. М. Крячка було призначено художнім
керівником та диригентом.

\textbf{Читайте також:} \emph{У истоков \enquote{Ренессанса}: 30 лет великой музыки Мариуполя}%
\footnote{\url{mrpl.city.tilda.ws/renessans}}

Нині в складі камерного оркестру 25 артистів. Кожного року \enquote{Ренесанс}
змінюється, набуваючи нового, більш глибокого звучання. Завдяки рідкісному
поєднанню комунікабельності і вмінню організовувати та планувати час, з Василем
Михайловичем легко і, головне, приємно співпрацювати як молодим перспективним,
талановитим музикантам-початківцям, так і всесвітньо відомим майстрам сцени.

У репертуарі оркестру понад 500 творів золотої світової класики різних авторів,
епох і стилів (зокрема, А. Вівальді, Й. С. Баха, В. А. Моцарта, Л. В. Бетховена,
Дм. Шостаковича, С. Прокоф'єва, М. Скорика та багатьох інших класиків
української та світової музики). Водночас оркестр виконує фрагменти опер та
оперет, романси, експериментує, поєднуючи музику з поезією в
літературно-музичних композиціях (\enquote{Євгеній Онєгін}, \enquote{Заметіль}, \enquote{Кармен},
\enquote{Мати} та інші) за участю відомих акторів драматичного театру.

Василь Михайлович завжди велику увагу приділяє роботі з діть\hyp{}ми та молоддю.
Більше ніж десять років організовує і проводить міжнародні та всеукраїнські
дитячі музичні конкурси і фестивалі, на яких конкурсанти мають можливість грати
з оркестром.

\textbf{Читайте також:} \emph{\enquote{Браво!} - посол Франции в Украине Изабель Дюмон музицировала с мариупольским \enquote{Ренессансом}}%
\footnote{\enquote{Браво!} - посол Франции в Украине Изабель Дюмон музицировала с мариупольским \enquote{Ренессансом} (ФОТО+ВИДЕО), Ганна Хіжнікова, 11.07.2018, \par\url{https://mrpl.city/news/view/bravoposol-frantsii-v-ukraine-izabel-dyupon-muzitsirovala-s-mariupolskim-renessansom-foto-plusvideo}}

У 2006 році оркестр \enquote{Ренесанс} випустив перший ліцензійний компакт-диск
(звукорежисер М. Дідик, м. Київ). У 2012 році спільно з Остапом Шутком
(скрипка) було випущено ліцензійний компакт-диск \enquote{Для милих Дам!}.

\ii{19_03_2019.stz.news.ua.mrpl_city.1.vasyl_krjachok.pic.3}

Сьогодні В. М. Крячок виконує обов'язки директора Маріупольської камерної
філармонії, яка офіційно з'явилася в місті в травні 2018 року. На роботу Василь
Михайлович не шкодує ні часу, ні сил. Його дивовижна енергія, наполегливість та
ентузіазм надихають і запалюють учасників колективу. На відпочинок залишається
не так багато часу, наш герой зізнається, що не пам'ятає, коли востаннє читав
книжку чи переглядав фільм. Але, коли працюєш за призначенням, втомлюватися не
встигаєш.

\ii{insert.read_also.demidko.cherepchenko}
