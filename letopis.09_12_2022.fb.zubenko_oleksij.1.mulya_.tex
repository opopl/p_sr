%%beginhead 
 
%%file 09_12_2022.fb.zubenko_oleksij.1.mulya_
%%parent 09_12_2022
 
%%url https://www.facebook.com/aleks.zubenko/posts/pfbid0vU6yq6XcG61G14MdSDGoKm2KHzruvAzcnJQjfamC7cedW8Pkm2vb1idDA5ks8Gsbl
 
%%author_id zubenko_oleksij
%%date 09_12_2022
 
%%tags 
%%title МУЛЯЄ
 
%%endhead 

\subsection{МУЛЯЄ}
\label{sec:09_12_2022.fb.zubenko_oleksij.1.mulya_}

\Purl{https://www.facebook.com/aleks.zubenko/posts/pfbid0vU6yq6XcG61G14MdSDGoKm2KHzruvAzcnJQjfamC7cedW8Pkm2vb1idDA5ks8Gsbl}
\ifcmt
 author_begin
   author_id zubenko_oleksij
 author_end
\fi

(Колись давно побачив допис незрівнянного
\href{https://www.facebook.com/profile.php?id=100001734192803}{Рудольф
Гєльвік}, і тоді задумався. А зараз з огляду на пачку недавніх срачів на ФБ -
остаточно оформив у пост).

Знаєте, котики, є таке явище в нашій кукухє, яке я б назвав "воно мені МУЛЯЄ".

Коли одна людина іншу не займає, живе собі як хоче, нікому не робить шкоди... а
от іншій людині МУЛЯЄ. 

Не вписується воно в її картину світу.

Не так воно, як треба!

Не так одягаються, не ту музику слухають, не з тими і не так їбуться, не тою
мовою розмовляють (або тою, але не так), не той порядок у хаті, не так стосунки
в сім'ї будують... та що хош може бути не так.

Але чесно сказати "мені МУЛЯЄ" - не варіант.

Бо логічних відповідей на цю заявку дві - "ну і що" та "це твої проблеми, а не
мої".

(Ще раз підкреслю: мова про речі, які не несуть шкоди і не заважають іншим).

Тому на це МУЛЯЄ починають, як сову на глобус, натягувати різноманітні
ОБҐРУНТУВАННЯ.

"То завжди так було", "а от в мудрій книзі  написано!", "діди і батьки так
жили, а не так!", "ну це ж всім очевидно!", "це ж огидно/неприродно/аморально",
" і проча єрєсь.

Із повним спектром логічних помилок - починаючи від post hoc та перетворення
часткового на загальне і закінчуючи доведенням твердження через саме твердження
або взагалі безапеляційним "ну це ж хуйня!!!" (з раптовим онімінням при спробі
відповісти, чому саме це хуйня).

Що, втім, муляльників абсолютно не зупиняє. Особливо якщо є скілл зробити морду
кірпічом і тупо не чути жодних контраргументів. Адже ж МУЛЯЄ!!!

Саме звідси ростуть ноги у того кадавра, який у нас прийнято називати
суспільною мораллю.

Але ним справа не обмежується.

І це все наслідки, корінь яких один: бажання залізти в життя іншої людини і
відрегулювати його так, як собі треба.

Шоб не МУЛЯЛО.

Так от, коти і киці.

Наступного разу, коли вас тягне розказати комусь, як треба жити - задайте собі
питання:

Це дійсно шкідливо і має бути виправлено (і чому саме)?

Чи це вам просто МУЛЯЄ?
