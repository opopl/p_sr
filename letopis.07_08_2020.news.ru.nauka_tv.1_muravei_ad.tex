% vim: keymap=russian-jcukenwin
%%beginhead 
 
%%file 07_08_2020.news.ru.nauka_tv.1_muravei_ad
%%parent 07_08_2020
%%url https://naukatv.ru/news/27583?utm_source=smi2
%%tags science,hell,muravei,paleontology
 
%%endhead 

\subsection{В янтаре нашли адского муравья, который терзает свою жертву 99 млн лет}

\url{https://naukatv.ru/news/27583?utm_source=smi2}

\ifcmt
  pic https://naukatv.ru/upload/images/xl/93/93cc62ec58639d288cba34eb529ec4a7ced941b1.jpg
\fi

Палеонтологи нашли уникальный фрагмент янтаря, в котором застыл недавно
идентифицированный учеными адский муравей. Исследование опубликовано в Current
Biology, кратко о нем пишет Science Alert.

Адскими называют всех муравьев мелового периода. Свое неофициальное название
они получили из-за характерного внешнего вида и необычного строения ротового
отверстия.

Исследователь Филип Барден из Института Нью-Джерси считает, что ему и его
коллегам повезло. Найденный кусочек янтаря рассказывает о поведении ископаемых
хищников, что очень ценно для науки.

«Как палеонтологи мы размышляем о функции древних приспособлений, используя
доступные нам данные. Видеть вымершего хищника, пойманного в момент поимки
своей добычи – бесценно», – говорит Барден.

Научное название нового адского муравья – Ceratomyrmex ellenbergeri. Это
насекомое, которое жило как минимум 99 млн лет назад. Находка палеонтологов
примечательна еще и тем, что этот муравей застыл в тот момент, когда он мучал
свою жертву – древнего родственника нынешнего таракана.

\ifcmt
pic https://naukatv.ru/upload/images/xl/71/71dea5a6876d3234c437b18bfa737648b856ca2b.jpg
\fi

На сегодняшний день ученым известны 12 500 различных видов муравьев. Вероятно,
существует еще около 10 000, которые пока не изучены. Однако ни один из
современных видов не похож на недавно найденного адского муравья.

Его ротовое отверстие совершенно не похоже на ротовую полость современного
насекомого. Древний муравей использовал нижнюю челюсть, чтобы двигать ею, а
затем пригвоздить свою жертву к рогоподобной лопасти наверху.

Такой рог существовал и у других древних видов муравьев, он мог служить своего
рода зажимом. Обнаруженная окаменелость стала первым реальным доказательством
этого факта.

Исследователи не знают, почему адский муравей перестал существовать. Барден
подчеркивает, что это явление может говорить о том, что даже самый
распространенный вид может когда-нибудь исчезнуть.   

Ранее ученые выяснили, зачем колониям нужны ленивые муравьи.

Фото: Phillip Barden, Current Biology

