%%beginhead 
 
%%file 02_06_2021.fb.loskutova_natalia.mariupol.1.seminar_formuvannja_kompetencii
%%parent 02_06_2021
 
%%url https://www.facebook.com/1427894275/posts/pfbid02guYCuUdbGGLZykJDJ3bVZXh4sQNAuAJZ7kJMCzrCPHzdMjHPX3yomP6GzfDaPmTLl
 
%%author_id loskutova_natalia.mariupol
%%date 02_06_2021
 
%%tags mariupol,mariupol.pre_war,obrazovanie,language.french
%%title Науково-методичний семінар "Формування іншомовної комунікативної компетенції в процесі викладання іноземної мови"
 
%%endhead 

\subsection{Науково-методичний семінар \enquote{Формування іншомовної комунікативної компетенції в процесі викладання іноземної мови}}
\label{sec:02_06_2021.fb.loskutova_natalia.mariupol.1.seminar_formuvannja_kompetencii}

\Purl{https://www.facebook.com/1427894275/posts/pfbid02guYCuUdbGGLZykJDJ3bVZXh4sQNAuAJZ7kJMCzrCPHzdMjHPX3yomP6GzfDaPmTLl}
\ifcmt
 author_begin
   author_id loskutova_natalia.mariupol
 author_end
\fi

Науково-методичний семінар \enquote{Формування іншомовної комунікативної компетенції в
процесі викладання іноземної мови} на факультеті іноземних мов МДУ

\ii{02_06_2021.fb.loskutova_natalia.mariupol.1.seminar_formuvannja_kompetencii.pic.1}

02 червня 2021 р. на факультеті іноземних мов Маріупольського державного
університету відбувся науково-методичний семінар \enquote{Формування іншомовної
комунікативної компетенції в процесі викладання іноземної мови}. На семінар
було запрошено вчителів м. Маріуполя, які викладають французьку мову у школах,
ліцеях та гімназіях міста. Ведуча семінару – доцентка кафедри німецької та
французької філології Наталія Лоскутова розповіла про найбільш функціональну
модель іншомовної комунікативної компетентності, яка складається з мовної;
мовленнєвої та соціокультурної компетентностей.

При комунікативно орієнтованому навчанні іноземної мови ситуації є основою
навчання спілкуванню, тому на 3 години навчальна аудиторія перетворилася в
ігровий зал, де учасники одночасно працювали в парах, малих групах по 4-5 осіб
та в командах. Кожен з присутніх брав активну участь у творчих видах
діяльності, запропонованих організаторкою: вчителі знайомилися з методикою
brise-glace, виконували ситуативні та дискусійні вправи, проходили квізи та
розігрували рольові ігри. \enquote{Родзинкою} заходу виявилося запрошення француженки
пані Матильди Муліс (Mathilde Moulis), інженерки французької інжинірингової
компанії \enquote{Beten International}. Вчителі мали змогу поспілкуватися з носієм
мови, почути \enquote{живу} французьку мову та зануритися в культуру Франції. По
закінченню семінару учасникам було вручено сертифікати, а також дидактичні
матеріали, які дозволять оптимізувати навчальний процес. Учасники семінару
висловили побажання продовжити ініціативу і провести спільний
науково-методичний семінар у вересні 2021 р.

\ii{02_06_2021.fb.loskutova_natalia.mariupol.1.seminar_formuvannja_kompetencii.pic.2}

