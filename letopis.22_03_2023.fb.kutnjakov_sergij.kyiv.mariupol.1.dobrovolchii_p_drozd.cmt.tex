% vim: keymap=russian-jcukenwin
%%beginhead 
 
%%file 22_03_2023.fb.kutnjakov_sergij.kyiv.mariupol.1.dobrovolchii_p_drozd.cmt
%%parent 22_03_2023.fb.kutnjakov_sergij.kyiv.mariupol.1.dobrovolchii_p_drozd
 
%%url 
 
%%author_id 
%%date 
 
%%tags 
%%title 
 
%%endhead 

\qqSecCmt

\iusr{Anton Simanchiuk}

В ТРО доволі швидко припинили набирати, бо було занадто багато бажаючих. Стали
брати виключно із військовим досвідом. Звідки інформація про «бажаючих
катастрофічно не вистачало»?

\begin{itemize} % {
\iusr{Сергій Іванович}
\textbf{Anton Simanchiuk} мова в тій частині, якщо помітили, про Маріуполь. Мій друг Анатолій 25 лютого прийшов до міського воєнкомату. Той вже не приймав, закривався. З групи новобранців до ТРО відібрали тільки з досвідом військової служби. Толік служив у НГУ, але він тримав пляшку з пивом, тож його послали.
Судячи з суцільно розграбованим на початку березня магазинам, офісам, аптекам, у Маріуполі надійної їх охорони не проводилося. Результат мародерства перевищив всі страхи.

\iusr{Romka Romal}
\textbf{Сергій Іванович} Обов"яззково пишіть всю хроніку подій , це ж наша реальна історія . У мене є ФБ товариш , то він з перших днів почав писати про Харків , і тепер інтернет відновлює пам"ять , як це тепер зручно , а колись же цього не було , тому і історію як хотіли так і перекручували , хоча наша влада і тепер це намагається робити, але ще Ютуб та Гугл не викупили , покищо. А цього мера Бойченка ще часом не нагородили ? Отак , ці горе політики і здали країну . Мене весь час ще Мелітопольський Федоров чомусь напрягає . Не знаю , але можливо помиляюсь ..

\iusr{Сергій Іванович}
\textbf{Romka Romal} дякую. Деякі теми важко даються, але треба. По Федорову мабуть не помиляєтесь

\iusr{Romka Romal}
\textbf{Сергій Іванович} Мені ще тоді було дивно . що в ін попав у полон , а потім його чи відпустили . чи обміняли , темна історія . . До речі і Скіфське золото чомусь не було евакуйовано , а про війну всі ж знали і готувались , як нам сказав Данілов .

\iusr{Romka Romal}
\textbf{Сергій Іванович} Абсолютно вас розумію . Маріуполь це особлива біль, яка не дає спокою нормальним людям в Україні , через те , що людей , яких знищили і закатали в асфальт . Світла їм пам"ять. Пам"ятаємо .

\iusr{Marina Lugina}
\textbf{Сергій Іванович} у мене чоловік в Маріуполі 26 лютого пішов шукати, де його візьмуть. У ТРО, яке базувалось в 42 школі вже його не взяли, сказали, що набрали необхідну кількість чи щось таке. ВІн пішов у військомат, де його направили одразу в 503 бат морпіхів. Вже 26 лютого йому дали зброю і речі, направили в частину, а потім вже на позиціях були з 27 лютого.

\iusr{Сергій Іванович}
\textbf{Маріна Лугина} дякую за інформацію і позицію Вашого чоловіка. Пам'ятаєте, як міськрада до приїзду 16 лютого в місто президента, уряду, 30 послів і Ахметова з Новинським навіть приміщення для ТРО не виділила. Ні разу Бойченко до і після нападу не закликав йти в Тероборону і ЗСУ. Мародерство було страшенне.

\iusr{Marina Lugina}
\textbf{Сергій Іванович} так. Мародерство почалось після того, як світло зникло. ТРО і поліція нічого не робили. Ми бачили, як АТБ на Карпінського обносили натовпом. Підї'їхали хлопці на жигулях розбитих, випустили в повітря кілька черг, натовп розбігся у двори, а вони поїхали, а потім все поновилось. Вже нікому до цього діла не було.

\iusr{Marina Lugina}
\textbf{Сергій Іванович} мені цікаво інше. Ще до вторгнення мова про приміщення для ТРО йшла і планувалось їм виділити приміщення десь на Лівому. Цікаво ще й те, що на Лівому відремонтували і відкрили гуртожиток для переселенців. Мені в 17 році навіть там кімнату пропонували. І все в перші дні вторгнення розбили. Це робилось навмисно?


\iusr{Сергій Іванович}
\textbf{Маріна Лугина} думаю, так

\end{itemize} % }

\iusr{Пані Ольга}

Читаю і згадую той час...

Мирна центральна Україна.

Машини з біженцями... Назустріч- автобуси з чоловіками у камуфляжі....Бабуся,
що зі сльозами на очах молиться і хрестить їх...

\iusr{Світлана Петровська}

Дякую тероборонівцям! Завдяки їм дійсно в Києві був порядок. Ми, жінки,
ховалися в підвалі і особливо вночі часто чули перестрілки на вулиці-автоматні
черги, окремі постріли, так що сиділи тихо і носа не висовували. Одного разу до
нас завітав тероборонівець, підбадьорив. А в той час, на околиці міста ( на
Оболоні) їздили ворожі БТРи. Такої самоорганізації оборони міста, як в Києві,
майбуть більше ніде не було.

\begin{itemize} % {
\iusr{Сергій Іванович}
\textbf{Світлана Петровська} по місту деінде був і "дружній вогонь". Але стрільба чутна була кожну ніч.

\iusr{Світлана Петровська}
\textbf{Сергій Іванович} це був своєрідний самозахист- впадати в глибокий сон, коли нічого не чуєш, тільки тіло відчувало, як здригається земля від ударів ракетами та бомбами.
\end{itemize} % }


\iusr{Radiy Radutny}
Мабуть і я через ваш бп їздив з боячки...

\iusr{Olga Zubenko}
Дякую, Сергію Івановичу!💛💙

\iusr{Діна Надоленко}

\ifcmt
  igc https://i.paste.pics/5d8d7aa35002a489acf44c8939a3a425.png
	@width 0.2
\fi


\iusr{Nataliia Zhyvotnikova}
Дякую 🌿

\iusr{Життєдайна Мантра}

Цікаво. Пишіть ще..

Цей досвід тепер у ваших очах читається

\iusr{Nadia Bohorodetska}

дякую! ...пишіть, добре б ці хроніки книжкою.

\iusr{Лариса Лещук}

Так. Пишіть. Пам'ять нас усіх тримає.

\iusr{Володимир Олівець}
Продовжуй. А теперка де?

\begin{itemize} % {
\iusr{Сергій Іванович}
Гром.сектор. Формуємо Стратегії реінтеграції ТОТ.
\end{itemize} % }


\iusr{Соколюк Сергій}

Це варто книги, друже, безцінний досвід!

\iusr{Liudmyla Chayka}

Мого брата до вас не взяли

Вони з хлопцями-мисливцями взяли свої рушниці і сиділи на в'їзді у Київ самі
кілька днів на самостійно облаштованій "точці", чекали Щоправда не на
Святошино, а на Борщагівці

\iusr{Наталія Вяткіна}

Пишіть, будь ласка. Це корисний для всіх нас досвід. І цікаво ж як, кожна деталь справжня

\iusr{Наталя Ватуля}

Дякую вам. Мій чоловік, в той же період, стояв в районі Одеської площі. До аеродрому у Василькові - 30 км.

\iusr{Bogdan Khmelnitsky}

А це не у вас позивний Дим?

\begin{itemize} % {
\iusr{Сергій Іванович}
Ні

\iusr{Bogdan Khmelnitsky}
\textbf{Сергій Іванович}, сплутав значить, пробачте. А ваш який?

\iusr{Сергій Іванович}

Не повірите – "Сергій")

\iusr{Bogdan Khmelnitsky}
\textbf{Сергій Іванович}, та чого ж не повірю? В мене "Бодя". 🙂
\end{itemize} % }
