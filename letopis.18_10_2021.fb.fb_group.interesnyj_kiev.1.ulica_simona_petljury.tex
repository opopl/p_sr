% vim: keymap=russian-jcukenwin
%%beginhead 
 
%%file 18_10_2021.fb.fb_group.interesnyj_kiev.1.ulica_simona_petljury
%%parent 18_10_2021
 
%%url https://www.facebook.com/groups/449128522277461/posts/1151139362076370
 
%%author_id fb_group.interesnyj_kiev
%%date 
 
%%tags foto,gorod,istoria,kiev,petljura_simon,ulica,ulica.kiev.petljury
%%title Улица Симона Петлюры – привокзальная улица, с которой начинается Киев
 
%%endhead 
 
\subsection{Улица Симона Петлюры – привокзальная улица, с которой начинается Киев}
\label{sec:18_10_2021.fb.fb_group.interesnyj_kiev.1.ulica_simona_petljury}
 
\Purl{https://www.facebook.com/groups/449128522277461/posts/1151139362076370}
\ifcmt
 author_begin
   author_id fb_group.interesnyj_kiev
 author_end
\fi

Улица Симона Петлюры – привокзальная улица, с которой начинается Киев.

В 1855 году улица Коминтерна (ныне Петлюры) впервые упомянута как Игнатьевская,
связавшая железнодорожный вокзал с Бибиковским бульваром.

Называлась она так в честь знаменитого киевского домовладельца. Однако уже в
1869 году была переименована в Безаковскую в честь Киевского, Подольского и
Волынского генерал-губернатора Александра Павловича Безака (1801-1868). В
период его губернаторства и благодаря непосредственно его участию было открыто
движение по Курско-Киевской железной дороге и сооружен железнодорожный вокзал,
к которому и была направлена улица от Бибиковского бульвара (ныне бульвар Т.
Шевченко).

Название улица Коминтерна получила в 1919 году в связи с образованием
Коммунистического Интернационала.

В начале ХХ века близость к вокзалу обуславливала и предназначение основных
зданий улицы. Это были гостиницы "Бристоль" , "Франция", "Либава",
меблированный комнаты "Петроградские" и "Новая Россия". Своеобразная их
архитектура – это стиль модерн и эклектизм. В застройке принимали участие
выдающиеся архитекторы В. Николаев, А. Хойнацкий, А. Краусе.

На пересечении с Бибиковским бульваром был сооружен памятник графу А.
Бобринскому, который руководил строительством первой железной дороги к Киеву. В
1918 году он был снят, и только в 1954 году на его месте установили памятник Н.
Щорсу.

На Безаковской, 25 (угол Жилянской) в 1908-1914 гг. была сооружена Ильинская
церковь в честь 800-летия Михайловского Златоверхого монастыря, которую
разрушили в 1935 году. Интерес представляет и дом №3/25 (архитектор А. Таций)
как образец архитектуры 1930-х годов.

Своеобразен и дом №11/106 на пересечении с улицей Саксаганского. На первом его
этаже находится аптека.  В начале ХХ века дом принадлежал фельдшеру Ю. Коцюбе.
Архитектура его выдержана в стиле неоренесанс.

Уже в послевоенные годы на улице были сооружены здания "Киевгипротранса" и Дом
быта "Столичный". 

В доме №14 раньше печатался журнал "Киевская старина", а вот в доме №12, в
котором с 1890 года находилась водолечебница А. Успенского, в советское время
располагались клуб и библиотека.

Смотрите описание под фото.  @igg{fbicon.pointing.down} 
