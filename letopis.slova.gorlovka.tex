% vim: keymap=russian-jcukenwin
%%beginhead 
 
%%file slova.gorlovka
%%parent slova
 
%%url 
 
%%author 
%%author_id 
%%author_url 
 
%%tags 
%%title 
 
%%endhead 
\chapter{Горловка}
\label{sec:slova.gorlovka}

%%%cit
%%%cit_head
%%%cit_pic
%%%cit_text
И если заезжающие-на-время в город \enquote{\emph{горловчане}} сравнивают его с тем,
что когда-то видели до войны, то у живущих здесь на шкале сравнений отправной
точкой стоит военное время. Первые сетуют на то, что город не столь многолюден,
как при \enquote{закладке Собора}, а мы рады тому, что людей здесь намного
больше, чем во время \enquote{попадания у Собора}. Они сравнивают сегодняшнюю
\emph{Горловку} с той, что была здоровой, а мы с той, которую резали на
операционном столе и она могла вот-вот умереть. Да, сейчас ей тяжело, но она
уже может говорить и с трудом, но подходит к окну. Скажите, как объяснить
никогда не видевшим послеоперационных больных свою радость от того, что близкий
тебе человек выжил? Как объяснить, что при виде порезов и бледного лица не
стоит цокать языком, мол \enquote{пары метров пробежать не можешь? ну и слабак
же ты}
%%%cit_comment
%%%cit_title
\citTitle{Gorlovka.ua: Говорить с Городом - Блоги}, 
Егор Воронов, gorlovka.ua, 06.11.2016
%%%endcit

%%%cit
%%%cit_head
%%%cit_pic
%%%cit_text
Именно поэтому сейчас мне неинтересно, что происходит в Украине. Мне важно
только то, что происходит здесь и сейчас в городе, второй год находящемся на
линии фронта и смеющем жить. Не за что-то (и уж тем более флаги, идеологии и
политические системы), а вопреки. Вопреки ужасу, пережитому два-полтора года
назад. Вопреки страху, что это повторится снова. Вопреки тем, кто хочет видить
его \enquote{лакомым стратегическим куском}. Вопреки тем, кто под шумок распиливает
металлоконструкции. Вопреки свиданиям со Смертью и ее \enquote{засосам} на окраинах.
Вопреки чьей-то жалости, ненависти, разочарованию и пафосу. Что для меня этот
город? Я не понимаю, когда мне говорят - \emph{Горловка ДНР} и \emph{Горловка Украины}. Я
знаю только свой город. Просто \emph{Горловку}. И ее военной судьбы я не желаю ни одну
населенному пункту в мире. Но как писал прекрасный итальянский нобелевский
лауреат Сальваторе Квазимодо
%%%cit_comment
%%%cit_title
\citTitle{Город, рожденный жить}, 
Егор Воронов, gorlovka.ua, 27.09.2016
%%%endcit

