% vim: keymap=russian-jcukenwin
%%beginhead 
 
%%file topics.ukraina.maidan.2014
%%parent topics.ukraina
 
%%url 
 
%%author_id 
%%date 
 
%%tags 
%%title 
 
%%endhead 


% У листопаді відбудуться безкоштовні екскурсії місцями Революції Гідності (ГРАФІК)
% https://bigkyiv.com.ua/u-lystopadi-vidbudutsya-bezkoshtovni-ekskursiyi-misczyamy-revolyucziyi-gidnosti-grafik/

% Революція гідності і цінностей: як Євромайдан змінив Україну.
% https://blogs.pravda.com.ua/authors/viatrovych/6177abfb5efc6/

%\ii{19_02_2021.fb.dmitruk_anastasia.1.rev_gidnosti}
%\ii{20_07_2021.fb.berkut_mvd.1.maidan2}

%\ii{08_09_2015.fb.zagrebelnyj_igor.1.nebesna_sotnja}

%\ii{07_11_2021.fb.pikta_svetlana.1.monahi_maidan}

\href{https://www.youtube.com/watch?v=B8c6haF3jhU}{%
Шуфрич: У Росії була паніка, вони не очікували подій у лютому 2014 року. НАШ 27.09.2021, %
youtube, 27.09.2021%
}

\href{https://censor.net/ru/resonance/3051214/na_frazu_vi_vihovali_geroya_ya_vdpovdayu_scho_robila_tse_ne_dlya_togo_schob_vn_viyishov_na_mayidan_yakiyis}{%
"На фразу "Ви виховали героя" я відповідаю, що робила це не для того, щоб він вийшов на Майдан і якийсь мудило вистрілив йому в голову". Спогади родини Героя Небесної Сотні Романа Гурика, censor.net.ua, 20.02.2018%
}
