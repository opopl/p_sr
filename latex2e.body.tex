% vim: keymap=russian-jcukenwin
%%beginhead 
 
%%file body
%%parent f_main

\section{\\\\@startsection}

Synopsis:

\begin{verbatim}
  \@startsection{name}{level}{indent}{beforeskip}{afterskip}{style}
\end{verbatim}

Used to help redefine the behavior of commands that start sectioning divisions such as \verb|\section| 
or \verb|\subsection|.

Note that the titlesec package makes manipulation of sectioning easier.
Further, while most requirements for sectioning commands can be satisfied with
\verb|\@startsection|, some cannot. For instance, in the standard LaTeX book and
report classes the commands \verb|\chapter| and \verb|\report| are not constructed in this
way. To make such a command you may want to use the \verb|secdef| command.

Technically, \verb|\@startsection| has the form

\begin{verbatim}
  \@startsection{name}
    {level}
    {indent}
    {beforeskip}
    {afterskip}
    {style}*[toctitle]{title}
\end{verbatim}

so that issuing

\begin{verbatim}
  \renewcommand{\section}{\@startsection{name}
    {level}
    {indent}
    {beforeskip}
    {afterskip}
    {style}}
\end{verbatim}

redefines \verb|\section| to have the form \verb|\section*[toctitle]{title}| (here too, the
star * is optional). See Sectioning. This implies that when you write a command
like \verb|\renewcommand{section}{...}|, the \verb|\@startsection{...}| must come last in the
definition. See the examples below.

\paragraph{name}

Name of the counter used to number the sectioning header. This counter must be
defined separately. Most commonly this is either section, subsection, or
paragraph. Although in those cases the counter name is the same as the
sectioning command itself, you don’t have to use the same name.

Then \verb|\thename| displays the title number and \verb|\namemark| is for the
page headers. See the third example below.

\paragraph{level}

An integer giving the depth of the sectioning command. See Sectioning for the
list of standard level numbers.

If level is less than or equal to the value of the counter secnumdepth then
titles for this sectioning command will be numbered (see
Sectioning/secnumdepth). For instance, if secnumdepth is 1 in an article then
the command \verb|\section{Introduction}| will produce output like ``1
Introduction'' while \verb|\subsection{Discussion}| will produce output like
``Discussion'', without the number prefix.

If level is less than or equal to the value of the counter tocdepth then the table of contents will have an entry for this sectioning unit (see Sectioning/tocdepth). For instance, in an article, if tocdepth is 1 then the table of contents will list sections but not subsections.

\paragraph{indent}

A length giving the indentation of all of the title lines with respect to the
left margin. To have the title flush with the margin use \verb|0pt|. A negative
indentation such as \verb|-\parindent| will move the title into the left margin.

\paragraph{beforeskip}

The absolute value of this length is the amount of vertical space that is
inserted before this sectioning unit’s title. This space will be discarded if
the sectioning unit happens to start at the top of a fresh page. If this number
is negative then the first paragraph following the header is not indented, if
it is non-negative then the first paragraph is indented. (Note that the
negative of \verb|1pt plus 2pt minus 3pt| is \verb|-1pt plus -2pt minus -3pt|.)

For example, if beforeskip is \verb|-3.5ex plus -1ex minus -0.2ex| then to start
the new sectioning unit, LaTeX will add about 3.5 times the height of a
letter x in vertical space, and the first paragraph in the section will not
be indented. Using a rubber length, with plus and minus, is good practice
here since it gives LaTeX more flexibility in making up the page (see
Lengths).

The full accounting of the vertical space between the baseline of the line
prior to this sectioning unit’s header and the baseline of the header is
that it is the sum of the \verb|\parskip| of the text font, the \verb|\baselineskip| of
the title font, and the absolute value of the beforeskip. This space is
typically rubber so it may stretch or shrink. (If the sectioning unit
starts on a fresh page so that the vertical space is discarded then the
baseline of the header text will be where LaTeX would put the baseline of
the first text line on that page.)

\paragraph{afterskip}

This is a length. If afterskip is non-negative then this is the vertical space
inserted after the sectioning unit’s title header. If it is negative then the
title header becomes a run-in header, so that it becomes part of the next
paragraph. In this case the absolute value of the length gives the horizontal
space between the end of the title and the beginning of the following
paragraph. (Note that the negative of 1pt \verb|plus 2pt minus 3pt| is
\verb|-1pt plus -2pt minus -3pt|.)

As with beforeskip, using a rubber length, with plus and minus components,
is good practice here since it gives LaTeX more flexibility in putting
together the page.

If afterskip is non-negative then the full accounting of the vertical space
between the baseline of the sectioning unit’s header and the baseline of
the first line of the following paragraph is that it is the sum of the
\verb|\parskip| of the title font, the \verb|\baselineskip| of the text font, and the
value of after. That space is typically rubber so it may stretch or shrink.
(Note that because the sign of afterskip changes the sectioning unit
header’s from standalone to run-in, you cannot use a negative afterskip to
cancel part of the \verb|\parskip|.) 

\paragraph{style}

Controls the styling of the title. See the examples below. Typical commands to
use here are \verb|\centering|, \verb|\raggedright|, \verb|\normalfont|,
\verb|\hrule|, or \verb|\newpage|. The last command in style may be one that
takes one argument, such as \verb|\MakeUppercase| or \verb|\fbox| that takes one argument.
The section title will be supplied as the argument to this command. For
instance, setting style to \verb|\bfseries\MakeUppercase| would produce titles that
are bold and uppercase. 

These are LaTeX’s defaults for the first three sectioning units that are
defined with \verb|\@startsection|, for the article, book, and report classes. For
section, the level is 1, the indent is \verb|0pt|, the beforeskip is 
\verb|-3.5ex plus -1ex minus -0.2ex|, the afterskip is \verb|2.3ex plus 0.2ex|, and the style is
\verb|\normalfont\Large\bfseries|. For subsection, the level is 2, the indent is 0pt,
the beforeskip is \verb|-3.25ex plus -1ex minus -0.2ex|, the afterskip is 
\verb|1.5ex plus 0.2ex|, and the style is \verb|\normalfont\large\bfseries|. For subsubsection, the
level is 3, the indent is \verb|0pt|, the beforeskip is 
\verb|-3.25ex plus -1ex minus -0.2ex|, the afterskip is \verb|1.5ex plus 0.2ex|, and the style is
\verb|\normalfont\normalsize\bfseries|.

\paragraph{examples}

Here are examples. They go either in a package or class file or in the preamble
of a LaTeX document. If you put them in the preamble they must go between a
\verb|\makeatletter| command and a \verb|\makeatother|. (Probably the error message You can't
use \verb|`\spacefactor'| in vertical mode. means that you forgot this.) See
\verb|\makeatletter| and \verb|\makeatother|.

This will put section titles in large boldface type, centered. It says
\verb|\renewcommand| because LaTeX’s standard classes have already defined a \section.
For the same reason it does not define a section counter, or the commands
\verb|\thesection| and \verb|\l@section|.

\begin{verbatim}
  \renewcommand\section{%
    \@startsection{section}% name
      {1}% level
      {0pt}% indent
      {-3.5ex plus -1ex minus -.2ex}% beforeskip
      {2.3ex plus.2ex}% afterskip
      {\centering\normalfont\Large\bfseries}% style
    }
\end{verbatim}

This will put subsection titles in small caps type, inline with the paragraph.

\begin{verbatim}
  \renewcommand\subsection{%
    \@startsection{subsection}%  name
      {2}% level
      {0em}% indent
      {-1ex plus 0.1ex minus -0.05ex}% beforeskip
      {-1em plus 0.2em}% afterskip
      {\scshape}% style
    }
\end{verbatim}

The prior examples redefined existing sectional unit title commands. This
defines a new one, illustrating the needed counter and macros to display that
counter.

\begin{verbatim}
  \setcounter{secnumdepth}{6}% show counters this far down
  \newcounter{subsubparagraph}[subparagraph]% counter for numbering
  \renewcommand{\thesubsubparagraph}%               how to display 
    {\thesubparagraph.\@arabic\c@subsubparagraph}%  numbering
  \newcommand{\subsubparagraph}{\@startsection
                           {subsubparagraph}%
                           {6}%
                           {0em}%
                           {\baselineskip}%
                           {0.5\baselineskip}%
                           {\normalfont\normalsize}}
  \newcommand*\l@subsubparagraph{\@dottedtocline{6}{10em}{5em}}% for toc
  \newcommand{\subsubparagraphmark}[1]{}% for page headers
\end{verbatim}

 
%%endhead 
