% vim: keymap=russian-jcukenwin
%%beginhead 
 
%%file 06_05_2021.fb.maslov_evgenij.1.politruk
%%parent 06_05_2021
 
%%url https://www.facebook.com/maslovevgeniy14/posts/1473923912949312
 
%%author 
%%author_id 
%%author_url 
 
%%tags 
%%title 
 
%%endhead 

\subsection{НИКТО НЕ ЗАБЫТ...  НИЧТО НЕ ЗАБЫТО!}
\label{sec:06_05_2021.fb.maslov_evgenij.1.politruk}
\Purl{https://www.facebook.com/maslovevgeniy14/posts/1473923912949312}

\ifcmt
  pic https://scontent-bos3-1.xx.fbcdn.net/v/t1.6435-9/183295490_1473923889615981_7751444927178024781_n.jpg?_nc_cat=109&ccb=1-3&_nc_sid=730e14&_nc_ohc=zOp_KH0zTu4AX-KS1fk&_nc_ht=scontent-bos3-1.xx&oh=11e20ffa3cad824239fb5106421eec80&oe=60BB8916
\fi

Эту фотографию знает чуть ли не каждый советский человек. Впервые она была
опубликована во фронтовых газетах в 1942 году. Тогда мало кто знал, как зовут
человека, изображённого на снимке. Лишь через 23 года, в канун 20-летия Победы,
его имя стало известно широкой общественности: Алексей Гордеевич Ерёменко,
уроженец села Терсянка Запорожской области.

Он погиб через несколько секунд после того, как был сделан снимок. Это фото
носит название "Комбат". Оно не является постановочным и не является кадром из
фильма.

Легендарный кадр сделан 12 июля 1942 года около села Хорошее (сейчас село
Хорошее Славяносербского района Луганской области) между реками Лугань и
Лозовая. Фотограф Макс Альперт успел сделать снимок советского солдата,
поднявшего роту солдат в атаку, и тут же осколок снаряда разбил камеру.
Фотограф решил, что кадры испорчены и не стал записывать имя человека, которого
сфотографировал. Уже позже, проявляя пленку, он увидел, что кадр получился
превосходным.

Личность человека на фото через некоторое время определили – его звали Алексей
Гордеевич Еременко, 1906 года рождения. Вот что рассказал о тех событиях
очевидец, Александр Матвеевич Макаров:

"Фашисты бросались в атаку за атакой. Было много убитых и раненых. Наш сильно
поредевший полк отбивал уже десятую или одиннадцатую атаку.Гитлеровцы лезли
напролом к Ворошиловграду (Луганску), до которого оставалось около тридцати
километров. К концу дня был ранен командир роты. После ожесточенной бомбёжки,
при поддержке танков и артиллерии, фашисты пошли в очередную атаку. И тогда,
поднявшись во весь рост, со словами: "За мной! За Родину! Вперед!", Еременко
увлек за собой роту навстречу цепям гитлеровцев. Политрук погиб, но атака была
отбита".

Несмотря на то, что Алексей Еременко имел должность младшего политрука, мир
запомнил его как безымянного комбата. Дело в том, что возясь с разбитым
аппаратом в окопе , фотограф некоторое время не следил за обстановкой, но
услышал, как по цепи передали: "Комбата убили". Имя и должность командира
остались автору неизвестны, но услышанное дало впоследствии повод назвать
снимок именно так.

Не бывает безымянных героев – бывают герои, чьи имена были забыты потомками.
