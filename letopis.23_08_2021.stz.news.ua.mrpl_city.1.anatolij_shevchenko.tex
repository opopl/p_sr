% vim: keymap=russian-jcukenwin
%%beginhead 
 
%%file 23_08_2021.stz.news.ua.mrpl_city.1.anatolij_shevchenko
%%parent 23_08_2021
 
%%url https://mrpl.city/blogs/view/anatolij-shevchenko-lyubit-svoyu-krainu-ta-chastishe-hodit-u-teatr
 
%%author_id demidko_olga.mariupol,news.ua.mrpl_city
%%date 
 
%%tags 
%%title Анатолій Шевченко: "Любіть свою країну та частіше ходіть у театр!"
 
%%endhead 
 
\subsection{Анатолій Шевченко: \enquote{Любіть свою країну та частіше ходіть у театр!}}
\label{sec:23_08_2021.stz.news.ua.mrpl_city.1.anatolij_shevchenko}
 
\Purl{https://mrpl.city/blogs/view/anatolij-shevchenko-lyubit-svoyu-krainu-ta-chastishe-hodit-u-teatr}
\ifcmt
 author_begin
   author_id demidko_olga.mariupol,news.ua.mrpl_city
 author_end
\fi

\ii{23_08_2021.stz.news.ua.mrpl_city.1.anatolij_shevchenko.pic.1}

Напередодні \emph{Дня Незалежності України} я недовго думала, про кого напишу
наступний нарис. Звісно, про \emph{\textbf{Шевченка Анатолія Миколайовича}} – заслуженого
артиста України, актора Донецького академічного обласного драматичного театру
(м. Маріуполь), резидента театру \enquote{Terra Incognita Свій театр для Своїх}. Крім
того, що Анатолій Шевченко є однофамільцем Великого Кобзаря, він дуже любить і
шанує свою Батьківщину, свій народ, рідну землю... Актор поділився, що, коли
почалася агресія Росії проти України, \emph{він не думав про своє життя, думав лише
про свою країну, її майбутнє}. І наразі Анатолій Миколайович дуже болісно
переживає те, що відбувається з Україною. Підтримувати свою державу артист
намагається щодня, адже говорить її  мовою як в театрі, так і поза сценою. Для
актора Україна – це його дитинство на Сумщині, це ті невеличкі річки, струмки,
гриби, сильні дощі і довге літо. Україна для Анатолія Миколайовича – це все
його життя з усіма перемогами і негараздами.

Анатолій Шевченко народився у Маріуполі. Театр полюбив ще зі шкільних років.
Згадує, що любов до театру йому прищепив вчитель з літератури. Десятикласником
майбутній актор прийшов в народний театр ПК \enquote{Азовсталь} на чолі з \emph{\textbf{Леонідом
Бессарабовим}}, де назавжди полюбив театральне мистецтво. Батьки, правда,
спочатку не особливо підтримували таке хобі. Але хлопець був наполегливим:
навмисно не склав іспити в ПДТУ – здав на двійку математику.

\ii{23_08_2021.stz.news.ua.mrpl_city.1.anatolij_shevchenko.pic.2}

Фірмовий говір артиста є його головною родзинкою. Цікаво, що свого часу саме
мова з характерною маріупольською вимовою мало не поставила хрест на
акторській освіті Анатолія Шевченка, адже екзаменаторів не влаштовував його
маріупольський говір. Тоді він вирішив вступати до Київського державного
університету імені Карпенка-Карого, де претензій до вимови не було жодних.

Ще зі шкільних років актора дуже надихає гарне слово автора. Від цього, на
думку актора, залежить і якість вистави. Улюбленими авторами є О. Блок та І.
Тургенєв.

\ii{23_08_2021.stz.news.ua.mrpl_city.1.anatolij_shevchenko.pic.3}

Закінчив університет наш герой у 1981 році. Працював в Дніпропетровському
українському музично-драматичному театрі, в театрах Сімферополя і Підмосков'я.
Складно було в той час працювати, адже зарплати не платили, доводилося весь
час змінювати театри. Звісно, хотілося після навчання влаштуватися в рідному
місті, але тодішнє керівництво театру прагнуло брати артистів тільки зі
столичних міст.

І все ж завдяки неабиякій цілеспрямованості та наполегливості Анатолій Шевченко
потрапив до маріупольського драматичного театру. З 1991 року без нього не
обходиться жоден сезон. наразі працювати стало складніше через
військово-політичну ситуацію. Актор з теплом згадує свої знакові ролі. У Дніпрі
найбільш улюблена роль – професор Яворницький, у Маріуполі – князь в \enquote{Ханумі}.
Знаковою вважає роль в \enquote{Поминальній молитві} Тев'є-молочника.

Актор є майстром на всі руки. Раніше і меблі міг зробити, і ціле панно
створити.

\ii{insert.read_also.demidko.hobi_zdibnosti}
\ii{23_08_2021.stz.news.ua.mrpl_city.1.anatolij_shevchenko.pic.4}

Наразі актор грає у виставі \enquote{Маруся} за романом видатної письменниці та
поетеси Ліни Костенко \enquote{Маруся Чурай}. Анатолій Миколайович розповів, що досвід
гри у спектаклі за цим твором у нього вже був майже 30 років тому, тоді він
виконував роль Івана Іскри у спектаклі Дніпропетровського українського
музично-драматичного театру. Зараз він грає полковника Пушкаря, роль йому
подобається. Вважає, що сьогоднішня вистава маріупольського драматичного
театру вийшла кращою, ніж у Дніпрі. вона буде більш зрозумілою глядачу.
Загалом Анатолій Шевченко наголошує, що вистави за творами таких сильних
авторів необхідно дивитися, це велике збагачення для глядачів.

Колеги у захваті від роботи з Анатолієм Шевченком. Провідний майстер сцени
Донецького академічного обласного драматичного театру (м. Маріуполь) \emph{\textbf{Сергій
Мусієнко}} знає Анатолія Миколайовича ще зі своїх студентських років і завжди
захоплювався талантом і характером Шевченка. Він наголосив, що Анатолій
Миколайович – 
\begin{quote}
\em\enquote{справжній професіонал і майстер своєї справи, дуже гарний
партнер, який вміє слухати і допомагати на сцені, грати з ним поруч – суцільне
задоволення. Він віддає всього себе і партнеру по сцені і глядачу}. 
\end{quote}
Сергій Мусієнко сподівається, що незабаром Анатолій Миколайович отримає звання
народного артиста України.

\ii{23_08_2021.stz.news.ua.mrpl_city.1.anatolij_shevchenko.pic.5}

Актриса \textbf{Віра Шевцова} також дуже цінує роботу поруч з Шевченком. Вона любить
спостерігати за роботою актора та прислухається до його порад.

Сьогодні актор працює не тільки у маріупольському драматичному театрі, а й є
резидентом \enquote{Terra Incognita Свій театр для Своїх}. Вважає Анатолія Левченка
своїм режисером. Велике задоволення актор отримав від гри у виставі \enquote{Палата №
6} та співпраці і спілкування з актором Михайлом Загребою.

Артист дуже любить Маріуполь. Найбільше полюбляє відпочивати на морі, особливо,
де немає людей, щоб наодинці подумати про найголовніше.

\emph{\textbf{Прохання до маріупольців:}} 

\begin{quote}
\em\enquote{Любіть свою країну, своє місто, поважайте традиції
України та частіше ходіть у театр, нам дуже необхідна ваша підтримка!}.
\end{quote}

\ii{insert.read_also.demidko.terra_incognita}
\ii{23_08_2021.stz.news.ua.mrpl_city.1.anatolij_shevchenko.pic.6}

\emph{Всі світлини з архіву Донецького академічного обласного драматичного театру (м. Маріуполь).}

Інтерв'ю з актором було проведено в рамках рубрики програми \enquote{Ранок} на
Маріупольському телебаченні \emph{\enquote{Маріуполь театральний}}.
