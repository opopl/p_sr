% vim: keymap=russian-jcukenwin
%%beginhead 
 
%%file 20_10_2019.stz.news.ua.mrpl_city.1.nelli_mykolaivna_iskakova_teatr_moje_zhyttja
%%parent 20_10_2019
 
%%url https://mrpl.city/blogs/view/nelli-mikolaivna-iskakova-teatr-i-e-moe-zhittya
 
%%author_id demidko_olga.mariupol,news.ua.mrpl_city
%%date 
 
%%tags 
%%title Неллі Миколаївна Іскакова: "Театр і є моє життя"
 
%%endhead 
 
\subsection{Неллі Миколаївна Іскакова: \enquote{Театр і є моє життя}}
\label{sec:20_10_2019.stz.news.ua.mrpl_city.1.nelli_mykolaivna_iskakova_teatr_moje_zhyttja}
 
\Purl{https://mrpl.city/blogs/view/nelli-mikolaivna-iskakova-teatr-i-e-moe-zhittya}
\ifcmt
 author_begin
   author_id demidko_olga.mariupol,news.ua.mrpl_city
 author_end
\fi

\ii{20_10_2019.stz.news.ua.mrpl_city.1.nelli_mykolaivna_iskakova_teatr_moje_zhyttja.pic.1}

24 жовтня святкуватиме ювілей людина, яка вже \emph{50 років} віддано служить
маріупольському театру. Вона завжди чуйна, відповідальна і дуже надійна.
Робочий кабінет \textbf{Неллі Миколаївни Іскакової} постійно заповнений людьми: одні
звертаються за рішенням виробничих проблем, інші – за порадою і допомогою в
особистих справах. І вона вирішує, радить, допомагає...

Народилася наша героїня в центрі Росії – в Тулі. Батько був учасником
партизанського руху в Брянських лісах, має низку нагород – орденів і медалей.
Мама була домогосподаркою – мудрою і турботливою. Прожила довге життя (98
років). Неллі з дитинства любила театр. Особливо подобалося слухати передачі
\enquote{Театр біля мікрофона}.

\ii{20_10_2019.stz.news.ua.mrpl_city.1.nelli_mykolaivna_iskakova_teatr_moje_zhyttja.pic.2}

Під час навчання дівчинка брала участь у шкільній самодіяльності, де в журі
була народна артистка Алла Демідова, вона помітила у Неллі талант. Пізніше
Неллі Миколаївна ще зустріне цю видатну артистку в інституті, де вона буде
вести гурток і намагатиметься переконати дівчину пов'язати своє життя з
театром.

\ii{20_10_2019.stz.news.ua.mrpl_city.1.nelli_mykolaivna_iskakova_teatr_moje_zhyttja.pic.3}

Втім, спочатку Неллі закінчила Тульський педагогічний інститут ім. Льва
Миколайовича Толстого (філологічний факультет). До речі, ім'я видатного
письменника червоним рядком проходило крізь життя нашої героїні. Після
університету вона працювала в школі імені Л. Толстого. Її батьки жили в Ясній
Поляні, де Неллі Миколаївна народила свого сина Андрія. Цікаво, що своїх дітей
– \textbf{Наталю та Андрія} – Неллі Миколаївна разом з чоловіком назвали на честь
головних героїв роману-епопеї Л. М. Толстого \enquote{Війна і мир}.

Під час навчання в університеті студентка відвідувала курси радіодиктора,
найбільше їй подобалося читати новини. Також вона закінчила курси медичних
сестер.

До Маріуполя приїхала у 1968 році вже з чоловіком – \textbf{Іскаковим Рудольфом}. Він
був актором, вдвох вони і вирішили спробувати свої сили у міському театрі. Їхня
сім'я стала чудовим прикладом творчого союзу.

\ii{insert.read_also.demidko.otchenashenko}
\ii{20_10_2019.stz.news.ua.mrpl_city.1.nelli_mykolaivna_iskakova_teatr_moje_zhyttja.pic.4}

Неллі Миколаївна була прийнята в театр на посаду помічника головного режисера з
літературної частини в 1969 році. Маючи вищу гуманітарну, але не спеціальну
освіту, їй довелося цю професію швидко освоювати в процесі роботи. І головним
\enquote{двигуном} стали небайдужість, любов до театру, прагнення професійно
вдосконалюватися. Вона освоювала гори спеціальної літератури, отримувала
консультації у завлітів інших театрів, вчилася у Всесоюзному інституті
підвищення кваліфікації в Москві і стала професіоналом. Неллі Миколаївна
забезпечувала театр новими п'єсами: шукала нові твори вітчизняних і зарубіжних
драматургів, пропонувала для постановки найбільш вдалі.

А коли в 1987 році їй запропонували посаду завідувачки трупи, Неллі Миколаївна
вже була справжньою \enquote{людиною театру} та досконало знала театральну справу. У
будь-якому механізмі є гвинтики, на яких, власне, все й тримається. Це про
Неллі Іскакову. \emph{\enquote{Останнім часом трупа нашого театру, як ніколи раніше,
розрослася. Це провідні майстри сцени, артисти вищої, першої та другої
категорій, артисти балету та допоміжного складу. Кожна роль, кожен вихід на
сцену – все враховується}}, – підкреслює завідувачка трупи.

В обов'язки Неллі Миколаївни входить чітке планування всього робочого процесу в
театрі. За цим розкладом живе весь колектив. Це і постановки на великій та
малій сценах, і виїзди, і фестивалі, і творчі зустрічі. Неллі Іскакова
намагається забезпечити зайнятість всіх акторів у виставах, тому бере участь в
розподілі ролей. Незважаючи на те, що цю професію не можна отримати в
університетах, наша героїня працює за покликанням. Їй доводиться співпрацювати
з різними людьми – режисерами, художніми керівниками, акторами, – з усіма вона
здатна знайти потрібний підхід. Обґрунтованість і точність, з якою виконується
ця, здавалося б, не зовсім творча робота, говорять самі за себе: для Неллі
Миколаївни не існує дрібниць. Жінка віддається роботі повністю. Кожною виставою
дуже переймається. Акторам завжди намагається допомогти, десь співчуває, десь
стає на їхній бік. Більше віддає перевагу драмам, які захоплюють і змушують
співпереживати.

\ii{20_10_2019.stz.news.ua.mrpl_city.1.nelli_mykolaivna_iskakova_teatr_moje_zhyttja.pic.5}

Колеги Неллі Миколаївну глибоко цінують і поважають. З такими людьми, як вона,
хочеться працювати - наголошують театральні працівники. Зокрема, народна
артистка України \textbf{Світлана Іванівна Отченашенко} зазначила: \emph{\enquote{Є люди, які своєю
присутністю створюють позитивну атмосферу, заради яких хочеться ставати
кращими. Неллі Миколаївна відноситься саме до таких людей. Вона безмірно
потрібна театру. Інколи виступає психологом, інколи педагогом, а інколи навіть
юристом. Адже вона вирішує безліч проблемних питань. Дуже віддана, надійна і
відповідальна працівниця. На неї можна покластися абсолютно у всьому}}.

Артистка вищої категорії \textbf{Анна Федорівна Фоменко}, яка також багато років працює
разом з Неллі Миколаївною, вважає: \emph{\enquote{Це єдина людина, яка на 100\% на своєму
місці в театрі. Неймовірно успішне попадання. Без неї вже складно уявити
робочий процес}}.

Провідний майстер сцени \textbf{Андрій Луценко} підкреслив: \emph{\enquote{Неллі Миколаївна – справжня
мама для акторів, добра, лагідна, але в разі потреби вимоглива і сувора. Завжди
залишається дуже справедливою. Свою роботу знає досконало. Продовжує
самовдосконалюватися. Зокрема, останнім часом цитує вірші Т. Шевченка, В.
Симоненка, що не може не вражати}}. Сам Андрій Луценко приїхав до Маріуполя з
Івано-Франківська, тому розмовляє з завідувачкою трупи українською, але
вимоглива до себе Неллі Миколаївна вирішила, що і їй слід намагатися розмовляти
частіше державною мовою.

\ii{insert.read_also.demidko.levchenko}
\ifcmt
  tab_begin cols=2,no_fig,center,separate,no_numbering

  pic https://mrpl.city/uploads/posts/redactor/ccwehhv9tifysejj.jpg
  pic https://mrpl.city/uploads/posts/redactor/wriokbxi8msnzfb1.jpg

  tab_end
\fi

В людях наша героїня найбільше цінує доброту, надійність і порядність.
Ненадійні люди дратують. Не терпить несправедливості. Цінує вірність і вміння
дружити. Порядними людьми жінка намагалася виховувати і своїх дітей – сина та
доньку. Обидва вже подарували матусі онуків і навіть правнука. Цікаво, що
донька – Наталя – стала медиком. В дитинстві Неллі Миколаївна хотіла стати саме
лікарем. У вільний час маріупольчанка спілкується з онучкою Дариною та
правнуком Тимуром. До речі, Даринка дуже хоче працювати в театрі. Мабуть, це
зовсім не випадково, адже ще батько Дарини, син Неллі Миколаївни Андрій, в
дитинстві виступав на сцені нашого театру (грав у виставі \enquote{Поспішайте робити
добро}). Сам \textbf{Олександр Кадирович Утеганов} наголошував, що у хлопчика є талант,
хоча Андрійко працювати в театрі не захотів. Можливо, професію дідуся і бабусі
успадкує саме онучка, яка вже досягла чималих успіхів у художній гімнастиці. У
13, 5 років вона отримала звання кандидата в майстри спорту України.

%\ii{20_10_2019.stz.news.ua.mrpl_city.1.nelli_mykolaivna_iskakova_teatr_moje_zhyttja.pic.6}
%\ii{20_10_2019.stz.news.ua.mrpl_city.1.nelli_mykolaivna_iskakova_teatr_moje_zhyttja.pic.7}

\ii{20_10_2019.stz.news.ua.mrpl_city.1.nelli_mykolaivna_iskakova_teatr_moje_zhyttja.pic.8}

Неллі Миколаївна любить, коли є спокій і мир у всьому. Дуже любить тварин.
Домашній улюбленець – кіт Тимофій є справжнім антидепресантом для нашої
героїні. Також Неллі Іскакова полюбляє і вміє жартувати. Найбільше подобається
те, як театр святкує 1 квітня, адже саме цього дня безліч розіграшів і сміху. У
Маріуполі проводить багато часу біля моря. Воно заспокоює, а коли пляжі
безлюдні – допомагає відволіктися від метушні.

\ii{20_10_2019.stz.news.ua.mrpl_city.1.nelli_mykolaivna_iskakova_teatr_moje_zhyttja.pic.9}

\textbf{Улюблений фільм:} \enquote{Тихий Дон} (1958 рік).

\textbf{Улюблена книга:} \enquote{Звіяні вітром} Маргарет Мітчелл. З поетів найбільше любить С.
Єсеніна та О. Пушкіна.

\textbf{Хобі:} Робота і є хобі. Втім, ще любить збирати книги, фотографувати і
вирощувати квіти. Квітів у Неллі Миколаївни багато і вдома, і на роботі.

\emph{\textbf{Неллі Миколаївна бажає всім міцного здоров'я, сил, терпіння та миру у всьому –
в країні, вдома, в головах.}}

\ii{insert.read_also.demidko.kozhevnikov}

\begingroup
\color{blue}
Після розмови з бадьорою і відкритою, щирою і справжньою Неллі Миколаївною,
залишаються дуже приємні враження. Така позитивна і неймовірно працелюбна,
віддана своїй справі людина – це дійсно рідкість і справжня знахідка для
закладу, в якому вона працює. Від усього серця вітаю дорогу Неллі Миколаївну з
ювілеєм і бажаю міцного здоров'я, благополуччя та творчого натхнення! Нехай
невичерпна енергія, оптимізм і надалі будуть Вашими постійними супутниками в
житті, а доля подарує багато світлих років у мирі та добробуті. Впевнена, що
всі маріупольці з радістю приєднаються до привітань.
\endgroup

\ii{20_10_2019.stz.news.ua.mrpl_city.1.nelli_mykolaivna_iskakova_teatr_moje_zhyttja.pic.10}
