% vim: keymap=russian-jcukenwin
%%beginhead 
 
%%file 14_12_2020.news.ru.59.husainov_timur.1.den_47
%%parent 14_12_2020
 
%%url https://59.ru/text/health/2020/10/29/69521499/#post_1607951403
 
%%author Хусаинов, Тимур
%%author_id husainov_timur
%%author_url 
 
%%tags covid,vaccine,sputnik_v,dnevnik
%%title День сорок седьмой
 
%%endhead 
 
\subsection{День сорок седьмой}
\label{sec:14_12_2020.news.ru.59.husainov_timur.1.den_47}
\Purl{https://59.ru/text/health/2020/10/29/69521499}
\ifcmt
	author_begin
   author_id husainov_timur
	author_end
\fi

\index[rus]{Коронавирус!Вакцина!Спутник V}

Всем привет. В пятницу рано утром я сходил в пункт приема анализов и сдал кровь
на антитела к коронавирусу. Результат пришел мне на почту в субботу утром —
уровень антител IgG составил 12,88. Это значит, что в моем организме
сформировался иммунитет к инфекции, чему я, конечно, рад.

\ifcmt
  pic https://static.ngs.ru/news/2020/99/preview/6bbb5a639f080f957b1d23c7d64612f915ca02154_960_728_c.jpg
	caption Результаты анализа, Фото: Тимур Хусаинов
	fig_env wrapfigure
	width 0.5
\fi

Сразу после получения результатов анализа мне стало психологически легче
выходить на улицу и появляться в общественных местах. Это не значит, что я
перестал соблюдать меры предосторожности. Просто понимание того, что
коронавирус мне больше не страшен, делает мою жизнь и жизнь моих близких проще
и безопаснее.

Сейчас я могу с уверенностью сказать, что решение поставить «Спутник V» в моем
случае оказалось верным. Да, я испытал не самые приятные побочные эффекты от
прививки, но конечным результатом я доволен. Мне кажется, что в скором времени
эта вакцина или другие ее аналоги смогут поставить крест на пандемии «ковида».

Если у вас есть вопросы о моем опыте вакцинации, то присылайте их мне на
электронную почту \verb|t.khusainov@iportal.ru|. Ну, и не забывайте, что мы продолжаем
собирать отзывы о вакцине. Их также можно отправлять на мой email.

