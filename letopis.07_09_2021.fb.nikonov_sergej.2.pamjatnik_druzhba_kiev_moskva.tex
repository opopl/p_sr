% vim: keymap=russian-jcukenwin
%%beginhead 
 
%%file 07_09_2021.fb.nikonov_sergej.2.pamjatnik_druzhba_kiev_moskva
%%parent 07_09_2021
 
%%url https://www.facebook.com/alexelsevier/posts/1576122746066368
 
%%author_id nikonov_sergej
%%date 
 
%%tags demontazh,kiev,moskva,pamjatnik,ukraina
%%title Убрали красивый памятник дружбы Киева и Москвы
 
%%endhead 
 
\subsection{Убрали красивый памятник дружбы Киева и Москвы}
\label{sec:07_09_2021.fb.nikonov_sergej.2.pamjatnik_druzhba_kiev_moskva}
 
\Purl{https://www.facebook.com/alexelsevier/posts/1576122746066368}
\ifcmt
 author_begin
   author_id nikonov_sergej
 author_end
\fi

Убрали красивый памятник дружбы Киева и Москвы. Как памятник советской
оккупации. Это старая история. Где-то в 1914 полили желто-голубой краской.
Потом повесили мешок. И он так долго стоял.  Времена дружбы или просто
нормальных отношений людей для них оккупация. Во-вторых, конфликт, не конфликт,
война не война, такие памятники показывают, какие отношения должны быть. Если
власти не хотят хорошего своей стране , искажаете память, то они
недоброжелатели своей страны.  Принимая во внимание только проводимую политику
в целом на управляемой земле, преобладающее отношение жителей непокорных
территорий к РФ В 2014 Г., делаю вывод, что не РФ, а наши власти оккупанты
своей страны. Я не беру во внимание администрации, военное содействие
республикам и др. Формально это  признаки оккупации. Но по смыслу отношений
думаю это совсем не так. Ибо не знаю, что сейчас с отношением к России. Что
касается патриотов, руками которых это сделано. Не нужно давать ненависти
захватить свою душу,  нужно строить государство и защищать его и душой и умом.
Видеть реальность, а не продукт фейкометов.  Карьеристов и продажных патриотов
я отношу к прислужникам власти.
