%%beginhead 
 
%%file 08_03_2023.fb.kipcharskij_viktor.mariupol.1.r_k_tomu_bulo_take__
%%parent 08_03_2023
 
%%url https://www.facebook.com/permalink.php?story_fbid=pfbid0GdTvHiEe8xzJUzDw8XrzvZuzWZoJxjGtsuWQv3arckg2aL1evv6Efskxj2mtpJUdl&id=100006830107904
 
%%author_id kipcharskij_viktor.mariupol
%%date 08_03_2023
 
%%tags mariupol,mariupol.war,08.03.2022,dnevnik
%%title Рік тому було таке:  День 13 - 08.03.22. Вівторок
 
%%endhead 

\subsection{Рік тому було таке:  День 13 - 08.03.22. Вівторок}
\label{sec:08_03_2023.fb.kipcharskij_viktor.mariupol.1.r_k_tomu_bulo_take__}

\Purl{https://www.facebook.com/permalink.php?story_fbid=pfbid0GdTvHiEe8xzJUzDw8XrzvZuzWZoJxjGtsuWQv3arckg2aL1evv6Efskxj2mtpJUdl&id=100006830107904}
\ifcmt
 author_begin
   author_id kipcharskij_viktor.mariupol
 author_end
\fi

Рік тому було таке: 
\ii{08_03_2023.fb.kipcharskij_viktor.mariupol.1.r_k_tomu_bulo_take__.pic.1}

День 13 - 08.03.22. Вівторок. 

4:26. Температура у моїй кімнаті 16,9. Сплю у білизні, а вранці одягаю флісові
штані та гетри.

Негучне гупання. Вулиця вибілена снігом.

5:39. Тиша.

6:06. Незвично тихо аж страшно.

6:10. Нарешті почали стріляти - нас ще захищають. Нас не кинули. (Ми все ще не
вірили, що ми у повному оточенні).

7:00. Теплою водою з термосу залив вівсянку. Напередодні "мародери" з першого
під'їзду роздавали мариновані оселедці - сусідка з п'ятого взяла на нашу долю -
тож я з'їв молоки та запив водою з медом та невісткиним бісквітом.

Поснідавши, зробив "намаз", та пішов на двір. Намалював на снігу, що вкривав
вікна Калини листівки - поздоровлення жінкам з 8-м березня. Кип'ятив воду,
совком збирав з машин сніг та грів його на вогнищі. Максим рахував, скільки
каструль снігу йде на відро води.

Довго стукав до сусіда Едіка - нарешті він дочовгав та відкрив двері. Кажу, щоб
дав свій  чайник чи каструлю, аби я налив йому гарячої води. Він пішов за
чайником, тримаючись рукою за стінку, а коли повертався з чайником, то мало не
впав: в одній руці чайник, а іншою відкривав замок - я його встиг підхопити вже
на сходах  Ми з Олею пропонували Едіку наш суп - відмовився: "Я буду
помирати..." Йому десь 70-т років. Я кажу дай ключі аби ми могли приносити воду
та щось із їжі: "В мене є печиво, мені нічого не потрібно". Ввечері він поклав
у чайник пару яєць які зважилися поки чайник закіпав.


Чорний дим дим - горить "Щирий кум". Кажуть що його підпалили мародери - для
розваги чи помстилися, що не знайшли алкоголю. Їх нібито затримали. І що з ними
робитимуть?

Кажуть, що розстріляли Приморську адміністрацію, Приморський парк,
профілакторій. Переживаємо за Цибульських, Бібікових, Кудінових, Манойлових,
Гагаріних.

Сусід Сергій виніс пачку пакетиків чаю та стіків із цукром - в нього був
магазин. Каже що сам відкрив його і попросив брати що завгодно, тільки не
нищити обладнання: кавомашини та інше...

Син приніс три бутилі води з річки (смердючки) біля гаражів.

Звідкись взявся генератор. Його поставили під балконами між третім та четвертим
під'їздами (от радість людям, під чиїми вікнами таке торохтіння!). Солярку для
нього "добули" з розбитої АЗС  під "шлачкою".

На марафоні сказали, що Герасимов (начальник штаба, командувач округом) став
"хорошим росіянином" - інформація з перехоплених радіопереговорів. (неправда).

12:45. Дуже гучний розрив десь поруч (пізніше дізналися, що то був приліт у
приватний сектор біля Блажевича. Син грів їжу, невістка смажила сирники.

Сашко Щит пригостив Наталчиного Даню цукеркою, а той віддарився трьома стіками
цукру.

Близько 13:00 з боку Азавмашу чули автоматну стрілкотню. 

Кажуть, що гумконвой не пускають у Маріуполь бійці Азову, що полк Донбас
перейшов на бік росіян і воює проти України і подібну маячню, яку слухають з
рашаФМ.

Пізніше по радіо сказали, що Орехів під росіянами, що через нього таким потоком
йдуть війська під Гуляй Поле, що Орехів фактично розділено на дві частини.
(Гуляй Поле й досі наше). Ще сказали, що з Пологів росіяни випустили ракету по
селищу, у якому були евакуйовані з Волновахи. Тобто, Пологи теж захоплено. Це
означає, що розповіді про те, що колону у Маріуполь не пускають Азовці -
брехня: як колона пройшла Орехів та Пологи?

Дочитав онуку "Куму Лису". Завтра почнемо "Как папа был маленьким". Чому читаю
я? Бо він добре читає українською, тому я не хочу, аби звикав до російської.

Показав онуці стік цукру і спитав в неї що це. Вона не знала, а коли я сказав,
що це цукор - не повірила в образилася: дивиться мовчки, а оченята наповнюються
сльозами. Довелося поспіхом розривати стік та висипати цукор.

17:00. Вдовж будинку мародери провезли магазинний візок з вудилищами та іншим
знаряддям для риболовлі - розгромили так і невідкритий магазин навпроти
адміністрації.

По радіо сказали, що Жирік став "хорошим".

Кажуть, що в овочевому на розі за старою ціною продавали перемішані сухофрукти,
але по 2 кг на людину.


Чули гудіння літака, потім вибухи та бачили стовп бурого (пилюка?) та чорного
диму над Азовмашем. Кажуть що там стоїть зенітна батарея С300.

17:30. Літак робить розворот майже над нами (інверсійний слід), кидає бомби на
Азовмаш і відходить на Єйськ чи Таганрог.

Кажуть, що дуже великі руйнування на Східному, Пентагоні-Курчатовому, 24-му,
27-му, спалено Порт-Сіті та Метро...

Діти, Оля та онук  грали у дурня.

Мала прийшла до мене (пробачила за цукор!) з малою колодою (2-5) і ми з нею
грали у дурня, більше-меньше, червоне-чорне - обіграла мене начисто...

Фото:

1 - поздоровлення.

2 - збирання снігу на воду (зелене на капоті - совок). 

3 - горить "Щирий кум".

Навздогін: Едік помер влітку... 

Двері зрізали болгаркою ...

\ii{08_03_2023.fb.kipcharskij_viktor.mariupol.1.r_k_tomu_bulo_take__.pic.2}

%\ii{08_03_2023.fb.kipcharskij_viktor.mariupol.1.r_k_tomu_bulo_take__.cmt}
