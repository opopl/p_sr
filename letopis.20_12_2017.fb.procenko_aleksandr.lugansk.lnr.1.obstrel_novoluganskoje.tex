% vim: keymap=russian-jcukenwin
%%beginhead 
 
%%file 20_12_2017.fb.procenko_aleksandr.lugansk.lnr.1.obstrel_novoluganskoje
%%parent 20_12_2017
 
%%url https://www.facebook.com/alexandr.protsenko.12/posts/1556347747784857
 
%%author_id procenko_aleksandr.lugansk.lnr
%%date 
 
%%tags donbass,lnr,obstrel,ukraina,vojna
%%title Обстрел Новолуганского
 
%%endhead 
 
\subsection{Обстрел Новолуганского}
\label{sec:20_12_2017.fb.procenko_aleksandr.lugansk.lnr.1.obstrel_novoluganskoje}
 
\Purl{https://www.facebook.com/alexandr.protsenko.12/posts/1556347747784857}
\ifcmt
 author_begin
   author_id procenko_aleksandr.lugansk.lnr
 author_end
\fi

Об обстрелах наших территорий в последнее время не пишу, потому что иногда и
слов печатных не могу подобрать. Да и понятно в целом все.

А вот обстрел Новолуганского двухдневной давности не могу обойти вниманием.
Случай, с одной стороны единичный, а с другой - достаточно иллюстративный.

Изначально допускал разные версии и в итоге решил разобраться более детально.
Тема очень сложная и обоюдоострая, но в том, что описываю - уверен. 

Во-первых, Новолуганское расположено в 2 км. от Горловки и до начала января
этого года было нейтральным, "серым". Тогда добробаты в рамках "повзучого
наступу", по их же словам, ночью в кузовах дальномеров скрытно выдвинулись и
внаглую поселок заняли, подняли флаг и т. д. "И т. д." - это значит
обосновались в домах местных, нарыли позиций прямо в на территории поселка
(возле свинофермы, на выезде и т.д.) 

Во-вторых, до этого момента за 2,5 года войны в поселок практически ничего не
прилетало - это подтверждали в т. ч. украинские сми.

И в-третьих. 

Анонимно удалось выяснить то, что думают сами местные.

а) обстрел велся с НЕСКОЛЬКИХ направлений;

б) "Град" летел со стороны Мироновского (подконтрольная Украине территория)

в) до обстрела с украинских позиций в течение 2-х недель регулярно шло тяжелое.

Выводы каждый может сделать сам.
