% vim: keymap=russian-jcukenwin
%%beginhead 
 
%%file 06_09_2021.fb.lohmatova_valentina.1.rossia_vpered
%%parent 06_09_2021
 
%%url https://www.facebook.com/permalink.php?story_fbid=631682318205751&id=100040919651026
 
%%author_id lohmatova_valentina
%%date 
 
%%tags rossia,sevpotok2
%%title РОССИЯ, ВПЕРЁД!!! ПОЛНЫЙ ВПЕРЁД!!!
 
%%endhead 
 
\subsection{РОССИЯ, ВПЕРЁД!!! ПОЛНЫЙ ВПЕРЁД!!!}
\label{sec:06_09_2021.fb.lohmatova_valentina.1.rossia_vpered}
 
\Purl{https://www.facebook.com/permalink.php?story_fbid=631682318205751&id=100040919651026}
\ifcmt
 author_begin
   author_id lohmatova_valentina
 author_end
\fi

РОССИЯ, ВПЕРЁД!!! @igg{fbicon.flag.rossia}{repeat=3} ПОЛНЫЙ ВПЕРЁД!!! @igg{fbicon.flag.rossia}{repeat=3}

Сварена  и уложена последняя труба «Северного потока — 2»

Сегодня, 6 сентября, в Балтийском море сварена и уложена последняя труба
газопровода «Северный поток — 2». Об этом сообщил оператор проекта Nord Stream
2 AG.

\ifcmt
  ig https://scontent-lhr8-1.xx.fbcdn.net/v/t39.30808-6/241045344_631682291539087_3090313933988957596_n.jpg?_nc_cat=108&ccb=1-5&_nc_sid=730e14&_nc_ohc=rubT8n1SlA0AX_IqCqi&_nc_oc=AQkgt_kiqnZuOerVAGuDat36waEidIPaLce4G9bKtXI8ug51YHu1Ttwt02jmYKyxc6c&_nc_ht=scontent-lhr8-1.xx&oh=4398902b640d1228a6fef2524e48963d&oe=613D77F9
  @width 0.5
  @wrap \parpic[r]
\fi

«6 сентября специалисты трубоукладочной баржи „Фортуна“ осуществили сварку
последней трубы второй нитки газопровода „Северный поток — 2“, после чего труба
под номером 200 858 будет погружена на дно Балтийского моря в германских
водах», — сказано в релизе.

В ближайшее время будет проведена стыковка идущей от берега Германии секции
газопровода с тянущейся из вод Дании секцией, и начнутся пусконаладочные работы
по второй нитке для введения трубопровода в эксплуатацию до конца 2021 года.

Как сообщало EADaily, судя по данным навигационных порталов, укладку труб
второй нитки «Северного потока — 2» баржа «Фортуна» завершила сегодня ночью в
немецкой экономической зоне Балтийского моря и ведет работы по сварке
«золотого» стыка. В то же время первую нитку достроили в начале июня и сейчас
на ней идут пусконаладочные работы. Две нитки газопровода смогут поставлять 55
млрд кубометров газа в год от побережья России через Балтийское море до
Германии. В декабре 2019 года работы были приостановлены после того, как
швейцарская Allseas отказалась от укладки труб из-за угрозы санкций США.
Строительство возобновилось в декабре 2020 года.
