% vim: keymap=russian-jcukenwin
%%beginhead 
 
%%file 17_12_2020.news.ru.zapad_moskva.krasnova_elena.1.dogomilov_chudesa_deti
%%parent 17_12_2020
 
%%url https://na-zapade-mos.ru/1031299-verim-v-chudesa-zhiteli-dorogomilova-rasskazyvajut-o-novogodnix-semejnyx-tradicijax
 
%%author Краснова, Елена
%%author_id krasnova_elena
%%author_url 
 
%%tags 
%%title Верим в чудеса! Жители Дорогомилова рассказывают о новогодних семейных традициях
 
%%endhead 
 
\subsection{Верим в чудеса! Жители Дорогомилова рассказывают о новогодних семейных традициях}
\label{sec:17_12_2020.news.ru.zapad_moskva.krasnova_elena.1.dogomilov_chudesa_deti}
\Purl{https://na-zapade-mos.ru/1031299-verim-v-chudesa-zhiteli-dorogomilova-rasskazyvajut-o-novogodnix-semejnyx-tradicijax}
\ifcmt
  author_begin
   author_id krasnova_elena
  author_end
\fi

17.12.2020 Новости районов

Семейный центр «Кутузовский» в преддверии праздников организовал акцию. Ее
участники рассказывали о своих новогодних семейных традициях. Своими историями
они решили поделиться и с нашими читателями.

«Каждый год, в 10-х числах декабря мы все дружно достаем с антресоли огромные
коробки с украшениями. Почему так рано? Для того, чтобы создать себе сказочную
атмосферу и новогоднее настроение. Дети с интересом рассматривают игрушки
гирлянды», - рассказывает Анна Колыбина, мама Амалии и Азалии.

\ifcmt
  pic https://na-zapade-mos.ru/files/data/user/AiF/olga.k/files/2020/2020.12.17-1608171775.7758_2020.12.16-1608115131.9358-079dcec3-8841-4014-b267-de8369b76d2b-1.jpg
  width 0.3
  fig_env wrapfigure
\fi


Семья наряжает елку под рождественские песни. «Всегда это происходит вечером:
так таинственнее и уютнее», - отмечают они. Дети украшают окна снежинками.
«Друзья обязательно нам дарят елочный шарик с символом года, и мы традиционно
тоже вешаем его на нашу зеленую красавицу. Под елкой у нас всегда стоит Дед
Мороз, который передается в нашей семье из поколения в поколение с 1950-ых
годов. Стены тоже украшаем различными декорациями.

\ifcmt
  pic https://upravadorogomilovo.ru/files/data/user/elena/files/now/2020.12.16-1608114683.2064_zzzz.jpg
  width 0.3
  fig_env wrapfigure
\fi


Вместе с детьми мы выбираем скатерть для новогоднего стола, салфетки, решаем,
какие блюда приготовим. Мы стараемся не повторяться, у нас всегда новое меню.
Дети принимают в процессе готовки непосредственное участие».

\ifcmt
  pic https://upravadorogomilovo.ru/files/data/user/elena/files/now/2020.12.16-1608114683.3869_zzzz.jpg
  width 0.3
  fig_env wrapfigure
\fi


Еще одна наша традиция – это письмо Деду Морозу. Дети рассказывают волшебнику о
том, как вели себя весь год, просят исполнить их желания.  Потом мы отправляем
письма и с нетерпением ожидаем праздник. В этом году старшая дочь научилась
писать печатными буквами сама, поэтому для нее это вдвойне интересно: сочинить
письмо самостоятельно и ждать подарки», - говорит Анна.

\ifcmt
  pic https://upravadorogomilovo.ru/files/data/user/elena/files/now/2020.12.16-1608114692.7_kkk.jpg
  width 0.3
  fig_env wrapfigure
\fi


Также мама помогает малышам создавать новогодние поделки для детского сада,
готовить открытки родным и близким. Тем, кто живет далеко, они отправляют
подарки почтой. «Родственники из Мурманской области на новый год, по
возможности, приезжают к нам, либо мы едем туда, и собираемся большой семьей за
праздничным столом: смеемся, радуемся, подводим итоги старого года, мечтаем и
стоим планы на будущее, загадываем желания, которые обязательно сбываются.

\ifcmt
  pic https://upravadorogomilovo.ru/files/data/user/elena/files/now/2020.12.16-1608114701.812_pp.jpg
  width 0.3
  fig_env wrapfigure
\fi


В канун Нового года мы смотрим старые фильмы, наш любимый «Один дома», именно
первая часть. Дети любят советские новогодние мультики. Заранее малыши учат
стихи и песни.

А в новогоднюю ночь к нам всегда «приходит» Дед Мороз. Пока дети спят, мы
делаем следы, как будто он пришел с балкона, и принес подарки, - с улыбкой
отмечает Анна. - Утром, когда ребятишки просыпаются, их уже ждут волшебные
коробочки. И конечно мы делаем очень много фотографий».

А для семьи Росс Новый год - это самый долгожданный и уютный праздник -
снежная, морозная зима и волшебная атмосфера. «У нашей семьи есть свои
традиции, которые делают новогодние дни особенными, создают чудесное настроение
и по-настоящему объединяют нас.

\ifcmt
  pic https://upravadorogomilovo.ru/files/data/user/elena/files/now/2020.12.16-1608115150.3526_b745ab84-3396-4667-ad1d-ec6feabe6974.jpg
  width 0.3
  fig_env wrapfigure
\fi


Ежегодно мы устраиваем фотосессию для всей семьи на фоне елки. Со временем
таких снимков становится все больше. Очень интересно наблюдать, как мы
меняемся, как растут дети. Эти новогодние фотографии-открытки, наполненные
любовью и волшебством, всегда вызывают улыбки у наших бабушек и дедушек!

\ifcmt
  pic https://upravadorogomilovo.ru/files/data/user/elena/files/now/2020.12.16-1608115140.2916_301ff342-911e-4dd8-baed-10bc85fa376a.jpg
  width 0.3
  fig_env wrapfigure
\fi


Наряжаем елку все вместе. Достаем из коробок старые игрушки, бережно завернутых
в бумагу. С каждой из них связаны какие-то воспоминания. А еще деткам нравится
создавать поделки. Мы украшаем елку гирляндами, вешаем конфеты и леденцы.
Конечно же, под елочку кладем подарки», - делится процессом приготовления
Александра, мама Владислава, Никиты и Каролины.

\ifcmt
  pic https://upravadorogomilovo.ru/files/data/user/elena/files/now/2020.12.16-1608115140.4021_301ff342-911e-4dd8-baed-10bc85fa376a.jpg
  width 0.3
  fig_env wrapfigure
\fi


В новогодние праздники семья смотрит любимые фильмы: «Один дома», «Морозко»,
«Ирония судьбы», «Чудо на 34-й улице». Укутавшись в теплый плед, они читают
рождественские сказки.

«Готовим праздничный стол с традиционными блюдами и угощениями нашей семьи, по
рецептам, доставшимся нам от нашей прабабушки. Дарим друг другу подарки и
говорим добрые слова и пожелания, вспоминая все самое лучшее в уходящем году. А
под бой курантов загадываем желания и верим в то, что они непременно сбудутся!

Удачи и успехов в Новом году, а главное – крепкого здоровья Вам и Вашим
близким! Чтите традиции!», - желает всем Александра вместе с детьми.

\ifcmt
  pic https://upravadorogomilovo.ru/files/data/user/elena/files/now/2020.12.16-1608115131.8288_079dcec3-8841-4014-b267-de8369b76d2b.jpg
  width 0.3
  fig_env wrapfigure
\fi


А семья Алабиных рассказала о традиции готовить вареники на Старый новый год.
Дружная семья даже записала видеоролик с рецептом данного блюда. В кулинарном
процессе участвуют все дети. Володя, Андрюша и Надя под руководством мамы
Марины замешивают тесто, делают начинки из творога и картошки. В один из
вареников они кладут монетку, конечно, предварительно ее помыв. Взрослые и дети
считают, что у того, кому он достанется, обязательно исполнится самое заветное
желание. Однако Алабины всех предупреждают, что вареники с сюрпризом нужно есть
с осторожностью, чтобы случайно не проглотить монетку.

Елена Краснова Фото из личных архивов Александры Росс и Анны Колыбиной

%→ Воздушная экскурсия: в ближайшие выходные состоятся авиаперелеты по
%маршруту «Внуково – Внуково»


