% vim: keymap=russian-jcukenwin
%%beginhead 
 
%%file letters.comments
%%parent letters
 
%%url 
 
%%author_id 
%%date 
 
%%tags 
%%title 
 
%%endhead 

%https://www.facebook.com/urtab/posts/pfbid024GtGyWgTauBg4vAk56QR65J5RHCqancaYFYfhHNEaFfpVNeHbsWwVL6uBV77X4Xl

Дуже цікаво пишете, дякую. Знаєте, що тут згадалось. Свого часу був знятий
серіал Геній - американці знімали (National Geographic), так от, перша частина
цього серіалу про Альберта Айнштайна - фізика-теоретика, творця спеціальної та
загальної теорії відносності та також автора листа президенту США Франкліну
Рузвельту, з якого почалось створення атомної бомби в Лос-Аламосі, що в
результаті призвело до атомних бомбардувань Хіросіми і Нагасакі... ну а друга
частина - про Пікассо. Кожна частина присвячена якійсь одній людині, і в кожній
частині багато серій. Короче, суть у чому. Творці цього фільму явно не
намагались зробити із Айнштайна святого, а показали його життя у всіх його
суперечностях, де Айнштайн - крім того, що показані всі його творчі натхненні
пошуки від школяра та студента аж до створення спеціальної та загальної теорії
відносності - його найвищого творчого та наукового досягнення, - виходить як і
дуже добра, неймовірна чуйна людина, так і людина, досить таки неоднозначна з
точки зору моралі та його відносин із власною сім'єю (особливо з точки зору
його відносин з Мілевою Маріч) Короче, вийшов насправді дуже життєвий фільм,
який я із задоволенням передивлявся багато раз. І сподіваюсь, колись подібний
фільм знімуть і про нашого Генія Тараса, і краще буде, якщо це зроблять самі
українці, врешті решт, не чекаючи, поки хтось інший не зробить це за них.


%14:13:17 21-03-23
а, ну добре... LaTeX (TeX) - це просто технологія програмування документів,
де-факто стандарт у науковому світі, яку використовують фізики та математики та
вчені інших спеціальностей для запису та публікації своїх робіт. Насправді це
дуже потужна штука, за допомогою якої можна підготувати публікації або книжки
будь-якого рівня складності... А я її використовую для архівації та збереження
різноманітних публікацій, як про війну, так і не тільки. Останнім часом активно
займаюсь Маріуполем, хоча я сам киянин. Як це виглядає, можете подивитись ось
тут https://t.me/kyiv_fortress_1 або
https://archive.org/details/@kyiv_chronicler

%11:14:24 22-03-23
%unsent
а навпаки. Треба щоби ті, що зараз в тилу, надихались енергією Миру, Світла, і
передавали її потім тим, хто потребує. хто дійсно на війні - воїнам,
постраждалим, тим, хто втратив житло і близьких... Я так вважаю. Бо ж є такий
парадокс, що ті хто на війні і хто кожний день ризикує життям, дуже часто духом
сильніші, ніж ті, хто в тилу, і кому нічого безпосередньо не загрожує
(звичайно, окрім ракети з чергової ракетної атаки), але весь час живе новинами
з війни... 



%11:18:51 04-04-23
% https://www.facebook.com/groups/kyiv.in.photo/permalink/5987520858010339/?comment_id=5992460377516387&notif_id=1680596268113248&notif_t=feedback_reaction_generic&ref=notif
Так, Ваші, та фото та історії і спогади інших маріупольців то є неоціненний,
найдорожчий скарб!! І не тільки для маріупольців, а і для всієї України! І на
основі цього скарбу у свій час постане Новий, Відроджений Маріуполь! Так само,
як наприклад, на основі фото та інших історичних даних був відтворений заново
Михайлівський Золотоверхий Монастир в Києві, знищений в 1937 році радянською
владою, так і буде відтворено все, что було зруйновано, я в цьому навіть не
сумніваюсь! ...Так, в цих фото, є велика Сила, Духовна Сила! А посилання (яке
Ви звісно бачили) я просто додав, щоби інші люди також заходили в группу )

% https://www.facebook.com/photoUkraine/posts/pfbid0W6YZgNAzJmr3ZtJoecjmWMiuFUcpNNgNF3MdPsQtVhDYXRsyfUgpPF3D8FoAjpzXl
а я післявчора в метро знаєте, що.. Історія маленька трапилась... їду в вагоні,
а тут хлопець за метр чи два метри біля мене стоїть, а в телефоні у нього
шахи!! Якась задачка напевне була або може книжку шахову читав, хто його зна. А
я в шахи теж граю, дуже цю гру люблю, і досить непогано граю, до речі... Так
от. Поки їхав від Либідської до Поштової, все намагався очима вгледіти, а що ж
він там розгадує ))) Може це і виглядає трохи негарно, але... Ніяк не міг
втриматись, бо цікаво було... але на жаль, не зміг побачити, бо багато людей і
важко побачити було... Так і залишилось це загадкою для мене, а що ж там за
позиція була. Але нічого страшного, в принципі! )))

%10:38:01 03-05-23

Сложно представить миллионы ебнутых... знаете, я вот недавно картинку
нарисовал, как раз представить очень даже можно! если вот сесть, успокоиться и
подумать, как это все выглядит в общем... то, что они @@бнулись давно и прочно,
мне уже стало ясно наверное полгода назад. Когда сидишь целый день, рубишься с
ними в чатах, и понимаешь, ну реально @@@@@@@@@@@бнутые на всю голову!!! но...
как показывает картинка ниже, во всяком случае для меня, не все так плохо,
потому что Победа будет все равно за Украиной. Все будет Украина. А почему. А
потому что в метафизически-когнитивной сфере, где также вовсю идет война, расея
уже давно заведомо проиграла, потому что их смыслы, их конструкции насквозь
гнилые. Ну а как я вижу все это в целом, вот и  картинка. В общем, такая себе
Коррида получается...

Большой Тяжелый Разъяренный Бык (раша, московия) рычит и готов броситься на
изящно одетого Тореадора (Украина). В руках у Тореадора флаг, он им машет перед
взором быка, вот бык бросился, а тореадор увернулся. Вот бык снова бросился -
снова промахнулся. И так далее. Постепенно бык изматывается, устает, все больше
теряет нить понимания того, что происходит. В общем, то же самое происходит и с
московией сейчас, потому что они все больше и больше запутываются в своем
вранье и безумии. И в конце-концов Тореадор одним точным Ударом убьет быка
насмерть и Коррида завершится... Завершится Победой Украины! Так что все будет
Украина!!! Правда, небольшой комментарий, надо не забывать, что тореадоры
бывает, тоже погибают, потому что удар рогами быка вообще то говоря смертелен
для тореадора...

Добрый день! Тут сложно ответить на комментарий на комментарий, поэтому просто
пишу (текст другого человека выше, на который я здесь хочу ответить).
Касательно рубилова в чатах с орками... Понимаете... Тут вообще то нужно очень
много писать, но я постараюсь уместить... Во-первых. Общение с орками в чатах у
меня имело цель исследовать их психологию, образ мышления и так далее. Поэтому
я не просто с ними рубился в чатах, а попутно записал может быть более чем 1000
скринов. Это были практические полевые исследования безумия орков, и по ним
можно уже целую диссертацию писать. Во-вторых, насчет устроить встречу в тцк и
позиции артиллериста. Вы априри не знаете из моего коммента выше, кто я такой и
где я нахожусь и чем я занимаюсь. Может быть я щас где-то под Бахмутом торчу,
весь в грязи, и меня может быть завтра уже просто не будет, а комменты в телеге
и рисованные картинки это просто способ развлечься и отдохнуть душой, когда
есть время. В-третьих, убить можно скажем 100 000 тысяч@@нутых, и это уже
успешно выполнено, слава ЗСУ!!! а убить 100 миллионов @@тых можно только лишь
если есть большая, очень большая термоядерная бомба. А для этого нужно как
минимум закончить физический факультет какого нибудь универа, чтобы вообще
иметь понимание, как работает атомная бомба. Вы кстати формулу Эйнштейна,
которая связывает энергию и массу, знаете?

%19:05:36 03-05-23
Добрый вечер! Извините, что возможно занимаю тут много места, хотел бы
добавить... Да... меня зовут Иван, я программист из Киева. Моя страничка в
фейсбуке https://www.facebook.com/ivan.ivan.kyiv   К чему я. Выше мне было
предложено, давай запишем тебя в артиллеристы, будешь убивать орков, чем больше
замочишь - тем лучше, чо тут думать-то, все просто ведь... И нафига ты вообще
тратил время на рубилово с орками в чатах... кому это вообще нужно. Но...  дело
в том, что нам нужные разные специалисты в этой войне. Нужны саперы - они
обезвреживают мины. Нужны Воины - они непосредственно воюют. Нужны
артиллеристы. Нужны медики. Нужны хирурги, которые спасают. Нужны снайпера,
которые четко одним движением пальца сшибают мозги уродам. Нужны летчики. И
Верховная Рада, и Президент, и министры тоже нужны, хотя они непосредственно
никого не убивают... И знаете что, программисты... и также представители других
умственных профессий - как психологи, даже историки, аналитики и т.д. тоже
нужны, хоть они тоже непосредственно никого не убивают. Как бы ботаники, и их
легко убить в любом бою, а все таки. А почему, а потому что идет также Война
Информационная, Война Когнитивная... А насчет важности программистов... Ну...
начнем с того, что тут Телеграм. Кто его сделал? Программисты. Фб кто сделал?
Далее... ЗСУ вовсю используют систему DELTA, а ее кто сделал? Пришел
артиллерист и по быстрому что ли на коленке в своем телефоне состряпал, или
как? Старлинки откуда также взялись, их что ли где то в Одессе повара
нахимичили, или как? В общем, я к тому, что сейчас нужны профессионалы в разных
областях, и также хочу подчеркнуть важность того, что мы сейчас живем в
Цифровом Веке. Это раз. А два. Я скажу такое. Насчет рашки. Щас вот глянул в
ютюбе - раша в истерике, на них напали. А год назад, что было, что было!!! А за
этот год... в общем, все уже очевидно, что расея - это дутый мыльный пузырь...
но... она еще сильна... и... такой вот вопрос... который я пока не знаю, что с
ним делать... вот намедни в Музей Чернобыля сходил на Хорива 1... все там
обфоткал... Так вот...  ЗСУ могут освободить Мариуполь, Донецк, Крым, и т.д.
военным путем... и я уверен, это будет сделано и совсем скоро... А вот что
делать с ЗАЭС, а? Здесь дело тонкое... Ведь если рванет, так мало то никому не
покажется...

%https://www.bbc.com/ukrainian/features-64585182

ох! дуже об'ємна тема... я тут скажу таке... щодо жорстокості в фільмах і т.д.
американські фільмі дійсно такі. А чому. А тому що ... ну... я думаю, напевне
трохи якось пізніш розпишу... А зараз... щодо права радіти... звичайно... треба
радіти, маєте повне право радіти і квітам і сонцю і трагедія в Умані -
безперечно - страшна трагедія і страшний воєнний злочин - як і трагедія
Маріуполя, як і трагедія Бучі, як і трагедія тих 5 людей, які заживо згоріли
від ракетного удару по телевишці в Києві в березні 2022 року, - крім того, що
там поруч знаходиться Бабин Яр... - де було в часи нацистскої окупації
розстріляно може більш ніж 100 000 киян, євреїв та інших... і також трагедія
тих людей, що загинули від ракетного удару по ДрамТеатру 16 березня 2022 -
взагалі, за оцей рік з гаком війни сталась просто якась неймовірна кількість
жахливих подій - не забирає у вас права радіти квіточкам і т.д., і якщо Ви
бачите квіточку, але одночасно пам'ятаєте про жертв трагедії - це означає, що у
Вас є Серце і Ви є Людина, тому що Ви здатні і одночасно переживати і сумувати
і співчувати, але також здатні радіти... Розумієте. Кожному свій час, час
радіти і час плакати, час сміятись і час тужити... От знаєте... 

Як Леся Українка писала, Гетьте, думи, ви, хмари осінні!
То ж тепера весна золота!
Чи то так у жалю, в голосінні
Проминуть молодії літа?
Ні, я хочу крізь сльози сміятись,
Серед лиха співати пісні,
Без надії таки сподіватись,
Жити хочу! Геть думи сумні!... 

І щодо Лесі Українки... вона звичайно жила в зовсім інший час, але у неї було
важке життя... хоча б тому що вона з десяти років хворіла на туберкульоз і все
своє життя повинна була кожного божого дня перемагати свій відчай і відчуття
безнадії і йти далі... крім того, що вона була жінка... а як відомо, в ті
консервативні часи жінкам призначалось лише одне місце в суспільстві - тобто,
кухня, сімья, діти, церква і все... і власне, тут я міг би писати багато. Але
хотів би десь тут втулити мою згадку про два фільми, якщо вже зайшла розмова
про фільми. Але... спочатку. Я от як Ви напевне вже бачили, досить таки багато
зараз Маріуполем займаюсь. Але... я киянин, і ніколи в Маріуполі не був, навіть
нікого не знаю персонально з Маріуполя... І трагедію Маріуполя я сприймаю із
своєї колокольні, так скажімо, через призму того, де я живу, що я бачив, що я
читав, в інтернеті, в книжках і т.д. Так от... Фільм-1. Фільм називається
Місто-44 (Город-44). його зняли про трагедію Варшавського повстання 1944.
Поляки зняли, подивіться в інтернеті... Це неймовірно трагічна подія, трагедія
масштабу Маріуполя. Про це знято взагалі багато фільмів, і от недавно зняли
отой фільм. Я його подивився ще до війни, дуже важкий фільм... ніколи не думав,
що таке от колись станеться в Україні і зовсім скоро... Так що... те, що
сталось з Маріуполем... я вже бачив в тому фільмі... я маю на увазі... коли
люди потрапили у справжнє пекло, хоча історичні обставини тут і там,
безперечно, різні. А другий фільм... старий радяньский фільм, Чародеї
називається. От... там є така фраза... пісня в кінці... Кажуть, а ти не вір...
говорять, а ти не слухай! Але... я зараз вже трохи втомився писати... Короче...
все в кінці буде добре! ... як співає Тарас Тополя в пісні ТамДеМиЄ...
https://www.youtube.com/watch?v=2BNoM8eveTU

Верхній Вал починається... тута можна на Воздвиженку... далі піднятись аж до
Музею Історії та того каменю, який Ви вже бачили... хм... але я думаю, Ви тут
вже все облазили і знаєте навіть те, чого я не знаю ))) ееем... там ще навпроти
Житнього Ринку посеред між Верхнім та Нижнім Валом... там, де є трамвайна
колія.. короче, там типу кіоск, чи що... і там гарний мурал намальований... не
знаю, чи помітили...

Іван Іван

а далі по Верхньому Валу ... мурал з Архангелом, який нависає над Києвом... це
десь Верхній Вал 44... далі до того, метро Контрактова Площа. Там є Пузата
Хата... І трохи далі... Хорива 1. Там таке величезне управління Поліції поруч і
власне, на Хорива 1, Музей Чорнобильської Катастрофи... Там я вже був, як видно
по моїх альбомах... Дуже-дуже насичений музей, там я взагалі був в два
заходи... поки все обійшов... я ж в Києві живу, то є тема персональна вже
також... От заплануйте, сходіть, це дійсно важлива тема... Бо ж ЗАЕС-то ніколи
не дівалась і так само дотепер під окупацією...
