% vim: keymap=russian-jcukenwin
%%beginhead 
 
%%file 18_10_2021.fb.bilchenko_evgenia.3.den_otca
%%parent 18_10_2021
 
%%url https://www.facebook.com/yevzhik/posts/4353236094711467
 
%%author_id bilchenko_evgenia
%%date 
 
%%tags bilchenko_evgenia,dedushka,otec,semja
%%title БЖ. День отца
 
%%endhead 
 
\subsection{БЖ. День отца}
\label{sec:18_10_2021.fb.bilchenko_evgenia.3.den_otca}
 
\Purl{https://www.facebook.com/yevzhik/posts/4353236094711467}
\ifcmt
 author_begin
   author_id bilchenko_evgenia
 author_end
\fi

БЖ. День отца

Очень хороший день создали в день моего рождения в России: 4 октября. У меня
никогда не было отца. У меня был Мегаотец - мой дед. Кто не помнит, как я
умирала вместе с ним от рака, как с тех пор не восстановилась, тот поймет
сейчас.

У меня был Мегаотец - мой дед, и меня тошнит от свободы, я хочу маленькой - и
на ручки. Я очень им горжусь, потому что он - самый крутой. Он родился в 1937
году в селе на Хмельниччине, и единственный из трёх детей выбился в город
учиться на врача. В 21 год моя изящная дворянская бабушка решила, что на ее
косолапом синеглазом Ванюшке идеально сидит военная форма. 

\ifcmt
  tab_begin cols=3

     pic https://scontent-lga3-1.xx.fbcdn.net/v/t39.30808-6/246712112_4353236138044796_3671121827861148592_n.jpg?_nc_cat=103&ccb=1-5&_nc_sid=8bfeb9&_nc_ohc=2SMcf3F6kCYAX9AYFKJ&_nc_ht=scontent-lga3-1.xx&oh=0e24220b9c3b0a9ddb6865c6a6cf0232&oe=61723DDF

     pic https://scontent-lga3-1.xx.fbcdn.net/v/t39.30808-6/245789310_4353236438044766_1009633457627270818_n.jpg?_nc_cat=109&ccb=1-5&_nc_sid=8bfeb9&_nc_ohc=UQJvE_PyiWgAX_zXJEE&_nc_ht=scontent-lga3-1.xx&oh=643005b3b44afdc595d9bd7f3717dcf8&oe=6172BFD1

     pic https://scontent-lga3-1.xx.fbcdn.net/v/t39.30808-6/246823944_4353236534711423_1670485169306537327_n.jpg?_nc_cat=106&ccb=1-5&_nc_sid=8bfeb9&_nc_ohc=iuaIuOwumjEAX9Ya6om&_nc_ht=scontent-lga3-1.xx&oh=6bb564f05d531aaea9d6d7304d06457e&oe=61734F5D

  tab_end
\fi

Сказано - сделано. В 22 года Ванечка - главный врач летного полка на Сахалине.
Он отправлял летчиков в небо, и при нём всего один человек на учебке разбился,
и то, соврал врачу, что спал, а сам перед полетом с женой ссорился. Ещё Ванечка
- ухогорлонос. Ещё Ванечка сам пробует всю еду для летчиков и гоняет поваров.
Гонять дед любил своим любимым словом: "Идиот!" На всех действовало. 

\ifcmt
  tab_begin cols=2

     pic https://scontent-lga3-1.xx.fbcdn.net/v/t39.30808-6/246441139_4353236641378079_8210507109787440517_n.jpg?_nc_cat=107&ccb=1-5&_nc_sid=8bfeb9&_nc_ohc=NKGQZiOaIlsAX-co42i&_nc_ht=scontent-lga3-1.xx&oh=90633338a17174f56c2ebe29932f8af8&oe=6171F1BD

     pic https://scontent-lga3-1.xx.fbcdn.net/v/t39.30808-6/246332673_4353236764711400_5160275385042135095_n.jpg?_nc_cat=101&ccb=1-5&_nc_sid=8bfeb9&_nc_ohc=oZt7FixP5_MAX9AAX5I&_nc_ht=scontent-lga3-1.xx&oh=9a8cbe37a7e2ec799e7b09a470fa7301&oe=6171C87F

  tab_end
\fi

Да, он был членом партии. И ещё в его селе в 1933 году все жители (Вербичи)
взяли фамилию "Мантур": все любили хозяина корчмы с этой фамилией, он щедро
наливал. И зерном они делились. И, кто делился, перед ним - шапку долой. 

Ещё дед боксом занимался. Очень любил рассказывать, что четверых раскидывал от
бабушки. Ещё курил. Много. С юности. Умер от рака лёгких в 82 года. Ещё с
парашютом прыгал, чтобы прочувствовать летчика, хотя ему, врачу, можно было не
прыгать. Ещё он часто уезжал, а куда не говорил, а потом хвастал, что в горячие
точки, но я не знаю точно: в удостоверении приравнен к героям Великой
Отечественной и участникам боевых.

Военврач: Сахалин, Стрый, Киев. От Германии отказался. Учился, доучивал ремесло
в Ленинграде, в академии, как и мой прадед.

 К 4 реактору под руководством тов. Телятникова выехал в 02.30 через час после
 взрыва. Вернулся в виде зомби 1 мая, и потом - долго - госпиталь в Москве и
 то, что он как полковник с ревом переживал: демобилизация по здоровью,
 лучевая, 1 группа, минус полщитовидки, операция руки великого Гульчия, Гульчий
 умер, его у нас затравили. 

И дед пошел в горздрав гонять молодых врачей. Те, кого он называл: "Идиот", -
сейчас, плача, ходят на его могилу. Благодарны за вздрючку. Вся гражданская
оборона была на нем, и тогда в Киеве норм было  Потом ему дали заслуженного
врача Украины, орден за ЧАЭС и героя Чернобыля. Один полковник плакал на
похоронах. Это было страшно, Арьевич не плакал на моих глазах. Говорил:
"Прощай, брат". Второй его брат - великий врач Матвиенко - разбился на Пасху.
Их всех знали в городе.

Я привыкла к военным, как к воздуху. Я знаю, как им отвечать. Как подчиняться,
как вертеть носом. Им даже можно ныть по-бабьи про здоровье, но они тебя
строят. Дед писал мне записки: "Жузик". И орал, когда я тратила деньги.

Он зашивал мне носки, делал бусы и коробочки, вырезал из дерева маски, очень
красивые. Он 50 лет жил только с одной бабушкой.

Ещё он возил меня в санаторий, и у нас были самые красивые полковничьи номера.
Ничего особенного: сейчас все круче у людей, но нам было мега. И он приносил
розовые апельсины: инвалидам помощь давали в девяностые. И одежду. И все школы
Киева он заваливал этими апельсинами, а люди алые есть боялись.

Он не верил в мои болезни и серу из ушей тащил пинцетом, а не выдувал. Из всех
лекарств он признавал только фталазол  и левомицитин. Ну, и пеницилиновые. И
все проходило. Кроме рака: он его навязчиво лечил алтейкой, не хотел знать. Он
проклинал коммерческие клиники. 

Он гладил мне блузки и варил супы. Котлеты с пюре делал. И всем показывал мои
дипломы, награды, медали. Сначала он думал, что у нас война. Потом почесал за
ухом и попросил проголосовать за Бойко. А потом плюнул и стал смотреть "В мире
животных".

В шкафу он хранил "Жузику - для познания рода": документы про то, как
партизанила его теща. И бабушкины крестики. И халатики. До конца. И он сделал
около ста коробок, обитых бархатом, на застежках. На кухне до сих пор лежат
записки: "Жузик, форточка!" - с нарисованным от руки шариковой ручкой пугливым
зайчиком, он знал, что я боюсь сквозняков.

Когда он умирал, он вышел в арку покурить и как-то растерянно, по-детски,
показал на свое любимое кафе напротив, до которого дойти уже не мог: "Кофе..."
- и как-то смущённо фыркнул. Без слез. Последнюю ночь он не лежал дома. Сидел и
говорил, на кухне, что военный борется до конца. Я не знаю, как он сидел, он
уже не мог, но он сказал:

- Так надо. Я не кляча. Это моя последняя ночь дома.

Потом он подписал документы там, где очень быстро умирают, чтобы все было
быстро. Без меня. Они сели с братом и по-военному перекинулись парой фраз. От
него осталось 10 пачек его Филиппа Мориса - отдала солдатам, держащим ордена у
гроба.

Тетка запрещала георгиевские ленты. Я впервые увидела юного безусого офицера
ВСУ, который сказал: "Нах, будем держать эти ленты".

Я плачу. Громко. Деда, прости, что я потеряла работу, которой ты так гордился.
Внучка-профессор в 33 года. Но, если бы я сохранила работу, я бы потеряла тебя
и все, чему ты меня учил.

Я храню даже кепку. Твою кепку. Я совсем ничего не могу. Я все могу. Твой
старлей. Твой, как ты говорил, "социальный инвалид Жузик".
