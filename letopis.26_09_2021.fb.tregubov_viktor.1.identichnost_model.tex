% vim: keymap=russian-jcukenwin
%%beginhead 
 
%%file 26_09_2021.fb.tregubov_viktor.1.identichnost_model
%%parent 26_09_2021
 
%%url https://www.facebook.com/victor.tregubov.5/posts/4579020205451478
 
%%author_id tregubov_viktor
%%date 
 
%%tags identichnost',rossia,rusmir,russkie,ukraina,ukraincy
%%title З точки зору ватанів, українці - зіпсовані росіяни
 
%%endhead 
 
\subsection{З точки зору ватанів, українці - зіпсовані росіяни}
\label{sec:26_09_2021.fb.tregubov_viktor.1.identichnost_model}
 
\Purl{https://www.facebook.com/victor.tregubov.5/posts/4579020205451478}
\ifcmt
 author_begin
   author_id tregubov_viktor
 author_end
\fi

Якось сто років тому я в Мережі наштовхнувся на смішний та поганий вірш якогось
ватана, в якому той плакався на злих українців. Весь не пам'ятаю, але
завершувався так.

"...а надо сознанье иуд изменить!

Внушить, что мы вместе - единый народ,

И Русская кровь в наших жилах течет!"

Креатив лайно, автор - ватан, але зафіксуємо поки цю позицію як характерну. На
її фоні видно одне наше слабке місце.

З точки зору ватанів, українці - зіпсовані росіяни. Щоб ця теорія виглядала
трішки менш абсурдно та трішки краще вкладалася в голови, її модифікували до
"більшість українців - нормальні, а от є такі собі галичани, що й не українці
насправді, вони українців, які частина нашого народу, заразили єрессю
українства. І от галичан треба бити, а українців треба повертати".

Ця модель сприйняття світу не має нічого спільного з реальністю. Але хто вивчав
історію науки, той знає, що й помилкові моделі на певному етапі мають
ефективність. Пливеш на захід до Індії, а відкриваєш Америку.

Бо в Україні, на жаль, часто побутує своя помилкова та слабка модель.

Згідно цієї моделі, будь-який українець, що коливається у своїй ідентичності,
вже є зрадником та зомбі. І працювати з ним не треба, а треба стріляти в голову
та не давати себе вкусити, бо все одно зрадить.

Ця модель, насправді, висходить з двох токсичних мотивів.

Перший токсичний мотив - підсвідомий комплекс меншовартості. Переконання, що
українська ідентичність є слабшою за російську, і якщо людина має ознаки обох,
або вагається між вибором, або навіть колись у житті мав якісь ознаки
російської - він насправді росіянин, просто прикидається.

Другий токсичний мотив - це просто бажання здаватися кращим і знецінити
співвітчизника, щоб на його фоні здаватися більш патріотом. Коли диванний
патріот із комсомолом в анамнезі звинувачує ветерана в тому, що он у нього там
серед віршів українською мовою є й російськомовний - це саме воно. А якщо й
україномовний, то був російськомовним, а якщо й не був, то не ходить на
смолоскипні марші та не голосує за націоналістів - коротше, завжди можна
знайти, до чого зачепитися.

Чому я кажу, що така модель слабша?

Бо вона слабша суто математично. Ватники намагаються поширити свій вплив на
тих, що коливаються, знайти до них підхід. Вони їм доводять, що вони - росіяни.
Українські ура-патріоти... теж їм доводять, що вони - росіяни. Єдиним логічним
результатом буде те, що вони з невизначених стануть росіянами. Кому від цього
стане краще? Правильно.

Наука математика та наука логіка нам підказують, що результат буде не дуже.
Вони нам підказують, що окрім батога, потрібний ще й пряник, що окрім
шельмування "недостатніх патріотів" та хорового плювання в цього філіна,
потрібне їх активне залучення до національного будівництва та пошук шляхів
виховання в них патріотів справжніх.

Але, на жаль, наука психологія нам підказує, що висновку науки логики буде
замало, щоб ура-патріоти збавили обороти.

\ii{26_09_2021.fb.tregubov_viktor.1.identichnost_model.cmt}
