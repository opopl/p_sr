% vim: keymap=russian-jcukenwin
%%beginhead 
 
%%file 10_01_2022.vk.doneck_teatr_kukol.1.novgod_kampania
%%parent 10_01_2022
 
%%url https://vk.com/teatrkukoldonetsk?w=wall-108142113_5488
 
%%author_id doneck_teatr_kukol
%%date 
 
%%tags doneck,novyj_god,dnr,teatr,kukla,deti,donbass,kultura
%%title Вот и подошла к концу наша новогодняя кампания 2021-2022
 
%%endhead 
\subsection{Вот и подошла к концу наша новогодняя кампания 2021-2022}
\label{sec:10_01_2022.vk.doneck_teatr_kukol.1.novgod_kampania}

\Purl{https://vk.com/teatrkukoldonetsk?w=wall-108142113_5488}
\ifcmt
 author_begin
   author_id doneck_teatr_kukol
 author_end
\fi

Вот и подошла к концу наша новогодняя кампания 2021-2022, которая проходила с
24 декабря по 9 января  @igg{fbicon.christmas.tree} 

От этого Нового года переполняют эмоции не только у зрителей, но и у нас @igg{fbicon.exclamation.mark}
Почему спросите вы?

Потому что:

 @igg{fbicon.round.pushpin} нам подарили ёлку, которая красуется на площади возле театра и радует не
только наших зрителей, а и всех жителей и гостей Донецка;

 @igg{fbicon.round.pushpin} у нас теперь два зрительных зала - большой и малый;

 @igg{fbicon.round.pushpin} фасад театра украшен настолько ярко, что поговаривают, вечерами его видно
даже из космоса  @igg{fbicon.face.grinning.squinting} ;

 @igg{fbicon.round.pushpin} в фойе театра всех встречает звездное небо;

 @igg{fbicon.round.pushpin} оформлены 5 разных, не похожих друг на друга тематических фотозон;

 @igg{fbicon.round.pushpin} уникальность этого года - не одно новогоднее представление, а целых два:
Новогоднее кукольное шоу «Двенадцать добрых дел» (4+) и Новогодний утренник для
малышей «Бука Новый год» (0+).

А теперь немного статистики:

 @igg{fbicon.chart.increasing}  40 показов Новогоднего кукольного шоу «Двенадцать добрых дел»;

 @igg{fbicon.chart.increasing}  24 показа Новогоднего утренника \enquote{Бука Новый год};

 @igg{fbicon.chart.increasing}  27 выездных показов новогоднего представления «Леди Баг и Супер-кот – самых
храбрый Новый год» по Донецку и городам Донецкой Народной Республики;

 @igg{fbicon.chart.increasing}  более 10000 счастливых зрителей;

 @igg{fbicon.chart.increasing}  около 1000 надутых шариков для маленьких зрителей \enquote{Бука Новый год};

 @igg{fbicon.chart.increasing}  120 сказок показал Кот Баюн;

 @igg{fbicon.chart.increasing}  24 раза Бука начинал жить по-новому;

 @igg{fbicon.chart.increasing}  около 1 т сладостей съели Бош и Крош в сладком королевстве;

 @igg{fbicon.chart.increasing}  более 500 желаний, стихотворений и песен были выслушаны Дедушкой Морозом;

 @igg{fbicon.chart.increasing}  5067 желаний были собраны в сачки для желаний из волшебных соцветий,
собранных на полях Полудницы;

 @igg{fbicon.chart.increasing}  24 раза внучка отправлялась в кругосветное путешествие в поисках весны;

 @igg{fbicon.chart.increasing}  40 раз была укрыта снежным одеялом поляна ярких впечатлений;

 @igg{fbicon.chart.increasing}  самое популярное новое имя для Буки - Константин, Коля и Сашка;

 @igg{fbicon.chart.increasing}  40 раз Хрустальда считала сделанные добрые дела Боша и Кроша;

 @igg{fbicon.chart.increasing}  24 раза звучала фраза \enquote{раз, два, три - Ёлочка, гори!} - и ровно столько же
она загоралась;

 @igg{fbicon.chart.increasing}  1348 выпущенных дымовых колечек из увеселительной пушки Каменными
стражниками, чтобы разогнать дурные мысли;

 @igg{fbicon.chart.increasing}  около 10000 снежков были брошены во время новогоднего утренника, из них 1852
раза попали в Буку, а не в корзину;

 @igg{fbicon.chart.increasing}  100 сотрудников, которые трудились для вас @igg{fbicon.heart.red}

Подробнее  @igg{fbicon.index.pointing.right}  
