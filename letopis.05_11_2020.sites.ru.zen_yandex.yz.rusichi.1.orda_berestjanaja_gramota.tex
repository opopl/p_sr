% vim: keymap=russian-jcukenwin
%%beginhead 
 
%%file 05_11_2020.sites.ru.zen_yandex.yz.rusichi.1.orda_berestjanaja_gramota
%%parent 05_11_2020
 
%%url https://zen.yandex.ru/media/politinteres/ordynskaia-berestianaia-gramota-o-chem-ona-5f9eedb749505f6811f39ea4
 
%%author Русичи (Яндекс Zen)
%%author_id yz.rusichi
%%author_url 
 
%%tags orda,russia,beresta
%%title Ордынская берестяная грамота. О чем она?
 
%%endhead 
 
\subsection{Ордынская берестяная грамота. О чем она?}
\label{sec:05_11_2020.sites.ru.zen_yandex.yz.rusichi.1.orda_berestjanaja_gramota}
\Purl{https://zen.yandex.ru/media/politinteres/ordynskaia-berestianaia-gramota-o-chem-ona-5f9eedb749505f6811f39ea4}
\ifcmt
	author_begin
   author_id yz.rusichi
	author_end
\fi

\index[rus]{Русь!История!Ордынская берестяная грамота, 05.11.2020}

\textbf{Единственная дошедшая до нас}

Больше всего берестяных грамот археологи находят в Новгороде. И объясняется это
в первую очередь особыми природными условиями местной болотистой почвы, на
которой стоит город. Там береста сохраняется в первозданном виде многие века.

\ifcmt
  pic https://avatars.mds.yandex.net/get-zen_doc/1586459/pub_5f9eedb749505f6811f39ea4_5f9fce139ac0705ae4861e8e/scale_1200
\fi

Что касается других регионов, то там большая часть берестяных грамот давно
разложилась. Однако отдельные чудом сохранившиеся экземпляры все-таки
попадаются и в других городах. В Москве, например, найдено четыре единицы. На
Украине одна штука найдена.

И есть еще совершенно уникальная берестяная грамота - золотоордынская! Причем
найдена она была еще задолго до новгородских.

В 1931 году возле села Терновка (ныне это Саратовская область, а тогда была
республика немцев Поволжья) была найдена рукопись, написанная на бересте. Позже
выяснилось, что там находилось золотоордынское поселение. Датируется рукопись
XIV веком.

\ifcmt
  pic https://avatars.mds.yandex.net/get-zen_doc/1708007/pub_5f9eedb749505f6811f39ea4_5f9fccc23910530e0dd7a23a/scale_1200
\fi

Записана она уйгурским вертикальным письмом, но часть текста при этом не
уйгурская. Специалисты, изучив ее, пришли к выводу, что это... монгольские
стихи.

На сегодняшний день вся рукопись расшифрована и переведена. Она состоит из 25
страниц, на шести из которых идет монгольский текст. Остальная часть уйгурская.
Страницы сшиты между собой, так что по факту это целая берестяная книга.

Стихи представлены в обрывочном виде. Вот для примера один из фрагментов:

\begin{leftbar}
				\begingroup
\em\color{orange}
\obeycr
… и зачем
оставаться? К омрачавшемуся
властителю… и
приходя, будешь ты взято, дитя
мое. Достойный и прекрасный кречет!
\restorecr
				\endgroup
\end{leftbar}

\ifcmt
  pic https://avatars.mds.yandex.net/get-zen_doc/3956291/pub_5f9eedb749505f6811f39ea4_5f9fcdaf49505f6811e75aa1/scale_1200
  caption Так выглядит золотоордынская берестяная грамота
\fi

Смысл этого и остальных монгольских фрагментов следующий. Перед нами народная
монгольская песнь матери, провожающей сына на службу к какому-то полководцу.

Перевел текст известный специалист по монгольской литературе и монгольскому
языку Николай Поппе.

\index[names.rus]{Поппе, Николай!специалист по монгольской литературе и монгольскому
языку}
