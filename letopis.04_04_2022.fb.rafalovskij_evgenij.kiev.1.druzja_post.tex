% vim: keymap=russian-jcukenwin
%%beginhead 
 
%%file 04_04_2022.fb.rafalovskij_evgenij.kiev.1.druzja_post
%%parent 04_04_2022
 
%%url https://www.facebook.com/evgen.rafalovsky/posts/10221042393034871
 
%%author_id rafalovskij_evgenij.kiev
%%date 
 
%%tags 
%%title Друзья, этот пост я посвящаю Вам!
 
%%endhead 
 
\subsection{Друзья, этот пост я посвящаю Вам!}
\label{sec:04_04_2022.fb.rafalovskij_evgenij.kiev.1.druzja_post}
 
\Purl{https://www.facebook.com/evgen.rafalovsky/posts/10221042393034871}
\ifcmt
 author_begin
   author_id rafalovskij_evgenij.kiev
 author_end
\fi

Друзья, этот пост я посвящаю Вам! Все  что со мной произошло, куплено и
доставлено благодаря вашей помощи и поддержке. Мы вместе на светлой стороне
Добра и Мира! 

Просыпаюсь, кругом снег, 3 апреля, я с группой Паганелей в Боливии. А, нет, это
так  Польшу снегом весной занесло!   В Польше я первый раз и врядли оказался бы
здесь, если бы не Война в Украине! 

Поэтому мы садимся в машину и едем в Литву,  с
\href{https://www.facebook.com/kostyantyn.storozhenko}{Kostyantyn Storozhenko}
и Юлей, где мы купили две машины для фронта, для нашей Победы!

Вот так украинец, португалец и русская  переганяют тачки. А ведь месяц назад я
был совсем \enquote{зеленым} волонтером. 

\ii{04_04_2022.fb.rafalovskij_evgenij.kiev.1.druzja_post.pic.1}

А Юля, кстати, смоталась из россии. Просто не хочет жить в стране с режимом,
где правит путин. Симон Боливар национальный герой Боливии по этому поводу
говорил: \enquote{Бегите из страны, в которой вся власть сконцентрирована в
одних руках - это страна рабов}. 

Знаете с чего я начал свой путь и кто первым откликнулся на мой призыв?
Хорошая девочка  из россии, живущая в Бразилии. Она мне дала спальник и
палатку. Это те вещи, с которыми я могу очень многое. С ними я улетел спасать
свою семью. Пока я летел через всю Южную Америку, семья двигалась  постепенно
мне на встречу и встретились мы на границе! 

Но, чтобы развернуться на границе Польши и Украины, в Пшемышле,  мне нужен был
крутой напарник, и это Костя! Он  не раздумывая откликнулся на мое предложение!
Наш тандем вместе, значит мы сила. И понеслось..

Все случаи не описать, вот вам несколько историй. 

Эта - самая  первая. С нее  мы по сути хоть и криво, но начали.

Везли мы пожилую бабулю и ее врослую дочку с границы. Они нас ждали, мерзли на
границе. А мы в  польских лесах заблудились. Кругом лесище, думаем, или волков
встретим, или же наших украинских пограничников, в итоге путь преградило
болото. Ни пройти, ни проехать, назад возвращаемся и делаем крюк.

Опаздываем, в  итоге я лихо выскакиваю в поле, где разбит лагерь для беженцев.
Подхватываю бабулю, она на инвалидной коляске, и погнал. В туалет хотите ? -
Нет! Ок, тогда в машину. Мчусь, но почва кругом с примесью песка, коляска
вязнет. Нужно сильно толкать, прямо с разгона. Холодно,  бабуля  укрыта
одеялом. На скорости одеяло попадает под колесо, бабуля вылетает и я чудом
ловлю ее в воздухе. Чуть не выбили страйк! Кое как с поля добрались до машины
и тут  бабуля захотела в туалет. Опять та же история, одеяло под колесо и чуть
бабулю второй раз не катапультировал. Ну все, после этого, бабуля у меня
только задом на коляске ездила.

Едем, в  дороге ночью с 7 го на 8 марта пошел снег.  Много снега, красиво, но и
опасно - гололед, а мы на летней резине!  В  польском Прикарпатье. Вобщем, что
сказать, - олени. Но кто знал, что будет снег. А вот   оленей здесь
действительно полно. Причем настоящих, благородных. Пасутся себе целыми
стадами.  За всю свою жизнь столько в Украине не видел, сколько за три дня в
Польше. Но история не про оленей, а про бабулю. Снежище лупит и мы забуксовали.
В итоге в 4 часа утра не доехали 20 км до поселка Леско, где должны были
разместить бабулю, застряли в сугробах. Что делать? Люди-то из Харькова, уже 3
дня в пути. Вот засада. Останавливаем местную машину. - Подбросите? - Ок. Но в
багажник инвалидная коляска не влазит. Хорошо я тогда еду с бабулей  и ее
дочкой. А Костя остается  ночевать в машине в сугробе посреди дороги. Но это
еще не все.В отеле в Леско ступеньки, и бабульку нужно занести. Хорошо что в
этот раз я не оплошал, но позвоночник немного бабуле выровнял. Зато поздравил
сразу девочек и с приключениями, и с 8 марта.  Дальше - пешком по городу в
кроксах по снегу в свою берлогу.  В итоге: 5:30 утра, "колеса" для бабули
приехали только в обед. А бабуля для дальнейшей транспортировки  в Германию
приходила в себя еще несколько дней! 

Как мы спасали девочку из Беларусии. Всех деталей не могу рассказать. Детей из
Украины вывезли многих без согласия родителей и без каких либо документов!
Схожее сейчас происходит в Мариуполе. Родители ее в шоке, а папа мой
однокурсник Антон в итоге он обратился ко мне. Ехать нужно аж в Брест. Но своих
нужно вытаскивать. Вообще девочек изначально  должно быть две, хотя в общем
застряло их 17,  но большая часть была хоть с документами.  А вот по ее подруге
с документами беда.  Едем в итоге  за  одним ребенком с ее с мамой Верой.  Она
везёт весь пакет документов, всю дорогу переживает, мы успокаиваем.  Я ободряю
ее, рассказываю, что  уже с вами друзья  бывал в разных передрягах и всегда
выходил. Ну что ж, посмотрим...

Под Брестом учения, танки лупят, аж гул стоит. На авто беларусы и россияне Веру
в свои машины не берут, отмазки разные, а вот украинцы в машину взяли. Все лед
тронулся... Дальше 8 часов ожидания, и вот оно, счастье! Вера вытянула свою
дочку.  В ход шли  на границе и  просьбы, и слезы!  Малая все это время что
была в Беларусии по телефону успокаивала родных. - Нет мамочка я не в
интернате. У меня все хорошо! Первый раз когда можно теперь спокойно девочкам
поспать, не переживая что в разлуке.  Счастье есть даже во время войны, это
когда семья вместе! И главное ВЕРА в лучшее. И уже не важно что девченки
возвращаются в Украину и что дома война! Главное в этом, что семья теперь
Вместе!

Лекарства. Многие знают, как сейчас сложно  в Украине с лекарствами. Даже в
Европе многие позиции сложно достать. И вот стоим мы на парковке у центра
беженцев в Пшемышле. Подходит рядом к волонтерам бездомный. Они от него
отмахиваются, мол топай отсюда. В итоге, он подходит к нам, от него несет
перегаром, говорит: Лекарства нужны? Ну, Костя смотрит на него, бомж, да еще и
под шафе, связываться не стал. А я выслушал, красиво рассказывает. Лекарств у
него куча. Перевязочных материалов вобще \enquote{вагон}, он их даже поднять не
может.  Там их три мешка! Ладно, думаю, проверю. - Идем. И точно. Приводит к
точке, где кто-то просто посреди улицы выгрузил гуманитарку. Бомжи местные
прямо как \enquote{король горы}. При этом дают возможность людям и себе взять,
но не все. А тут он мне - Все забирай, ты же это нашим! Я честно говоря 3 мешка
еле до машины донес! Все новое, запечатанное, из Португалиии. Ай да бомж, ай да
красавчик!  Через пару дней его встречаю, он уже важный, где-то форму раздобыл
и польского бомжа напарника! Говорю: Помнишь лекарства нам давал. Он
выпучивает глаза. – Нет, не помню, это не я. Я говорю не переживай! - Все
хорошо!  Спасибо тебе!
Может и хорошо, что не помнишь! Большой привет с Украины!

Это лишь пару историй, которые скоро станут настоящей новой  ИСТОРИЕЙ! А
сколько ещё таких у нас  было с турникетами, гуманитаркой, лекарствами и
передачей очень нужных \enquote{вещей},  которые везли в Украину иногда просто
поездом.

Друзья, если у Вас есть возможность финансово помочь, будем благодарны! Пишите в личку.

А мы  не останавливаемся  и продолжаем! Всех обнял, делаем вместе Мир светлей и
лучше!

\ii{04_04_2022.fb.rafalovskij_evgenij.kiev.1.druzja_post.cmt}
