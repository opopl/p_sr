% vim: keymap=russian-jcukenwin
%%beginhead 
 
%%file slova.zhanr
%%parent slova
 
%%url 
 
%%author 
%%author_id 
%%author_url 
 
%%tags 
%%title 
 
%%endhead 
\chapter{Жанр}
\label{sec:slova.zhanr}

%%%cit
%%%cit_head
%%%cit_pic
\ifcmt
  pic https://avatars.mds.yandex.net/get-zen_doc/3753737/pub_60a51146ec352b33cc616c91_60a601d72125b233e9bec66c/scale_1200
\fi
%%%cit_text
Феерия - очень редкий \emph{жанр} в литературе. Даже редчайший. Многие вообще могут
назвать одно-единственное произведение, которое автор принципиально отнес
именно к такому \emph{жанру}. Но на самом деле в двадцатом веке их было написано целых
три.  Дело в том, что изначально феерия - это чисто театральный термин. Так
называется сложная постановка с обилием сценических эффектов, для которых
используются особые механизмы, с музыкой и танцами, с общей атмосферой
волшебства. В общем, настоящая феерия - слышали такое выражение? Именно из
театра оно и пришло
%%%cit_comment
%%%cit_title
\citTitle{В двадцатом веке было написано только три феерии. Одна бельгийская, две наши}, 
ЛИТИНТЕРЕС, zen.yandex.ru, 03.06.2021
%%%endcit

