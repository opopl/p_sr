% vim: keymap=russian-jcukenwin
%%beginhead 
 
%%file 18_02_2022.stz.news.ua.strana.1.dokumenty_usa_posolstvo_kiev.1.bajdeny_sallivan
%%parent 18_02_2022.stz.news.ua.strana.1.dokumenty_usa_posolstvo_kiev
 
%%url 
 
%%author_id 
%%date 
 
%%tags 
%%title 
 
%%endhead 

\subsubsection{Байдены и \enquote{глашатай вторжения} Салливан}
\label{sec:18_02_2022.stz.news.ua.strana.1.dokumenty_usa_posolstvo_kiev.1.bajdeny_sallivan}

При переезде сотрудники посольства уничтожили часть файлов, среди которых, по
официальной версии, были заявления на грин-карты и необработанные паспортные
документы, объясняя это протоколом по защите конфиденциальной информации (иными
совами, чтоб не попали в руки \enquote{врагам}).

Также, по данным Wall Street Journal, дипломатам приказали уничтожить сетевое
оборудование и компьютерные рабочие станции, а также демонтировать телефонную
систему посольства.

Подробности в Госдепе засекретили. Официальный представитель Госдепартамента
Нед Прайс на брифинге заявил, что не может сообщить подробности об уничтожении
документов в посольстве США в Киеве, но сообщил, что действительные паспорта
уничтожены не были.

\enquote{Существует стандартный набор процедур, которые мы запускаем, когда начинаем
сокращение посольства, - сказал Прайс. - Мы предприняли разумные меры
предосторожности, когда речь заходит о конфиденциальных документах и
чувствительном оборудовании, но просто не можем предоставить
подробностей}.

Есть мнение, что за уничтожением документом на самом деле стоит токсичная
история отношений семьи Байденов с украинской элитой.

Речь идет о компромате, связанном с работой сына Байдена - Хантер, который
получал по 50 тысяч долларов в месяц в украинской нефтяной компании по
протекции отца. А также обвинения в том, что украинские власти по заказу
тогдашней администрации Белого дома вмешивались в американские выборы на
стороне Хиллари Клинтон, помогая ей воевать с Трампом.

\enquote{Скорее рано, чем поздно Команда Трампа и другие республиканцы добьются
расследования. А здесь уже, по счастливому стечению обстоятельств, все архивы
сгорели. Не будем забывать, что в 1917 первым делом сожгли картотеку уголовного
розыска. А здесь Госдеп в рамках переноса посольства США во Львов сам предписал
уничтожить сетевое оборудование и компьютеры в посольстве}, - пишет в своем
Facebook журналист-международник Надежда Сасс.

По мнению Эндрю Статтафорда, редактор журнала National Review, перенос
посольства во Львов, а особенно уничтожение документов и демонтаж оборудования,
выглядело как акт капитуляции.

\enquote{Реакция США на украинский кризис напоминает фиаско в Кабуле}, -
говорит нам Статтафорд.

В уничтожении секретных материалов, по мнению бывшего сотрудника посольства
Украины в США Андрея Телиженко, могут быть заинтересованы помимо Байденов и
высшего звена Демпартии также высокопоставленные американские чиновники
(компромат на которых может ударить по их боссам).

\enquote{Например, господин Джейк Салливан (человек Обамы и Клинтонов, нынче советник
Байдена по национальной безопасности - глашатай вторжения, который громче всех
говорит об \enquote{атаке Путина уже на этой неделе}, - Ред.). Как обнаружило
расследование, он солгал в показаниях по поводу \enquote{Рашагейт} в 2018-2019 годах.
Оказалось, что он заказывал материалы против Трампа во время предвыборной
кампании 2016 года, когда работал в штабе Клинтон. Возможно, в уничтожении
неких документов в Украине заинтересованы не только Байдены, но и чиновники
высшего звена, которые могли организовать координацию с выходом дипмиссии,
чтобы зачистить следы о своих делах. Когда Салливан начал комментировать
информацию о вторжении, появилась информация, что он замешан в кампании против
Трампа. И то, что он говорил до этого, под присягой, было ложью. Еще одни
возможные заинтересованные - Клинтоны, которые нанимали кампанию для слежки за
Трампом, чтобы придумать историю о его связях с российским Альфа-банком. И
многие вещи в рамках этой кампании шли из Украины}, - говорит нам Телиженко.

Действительно, Fox News со ссылкой на источники пишет, что Джейк Салливан
упоминается в расследовании обстоятельств слежки за избирательным штабом Трампа
в 2016 году.

Ранее проводящий проверку спецпрокурор США Джон Дарем предъявил обвинения в
лжесвидетельстве связанному с Демократической партией юристу Майклу Сассману,
который в 2016 году предоставил ФБР сведения о якобы \enquote{секретном канале связи}
между компанией Трампа Trump Organization и российским банком. Юрист скрыл, что
действовал в интересах штаба Клинтон, а также некой компании.

Салливан в обвинительном заключении фигурирует как \enquote{советник по внешней
политике}. По информации телеканала, в документе указывается, что некий юрист
предвыборного штаба Клинтон обсуждал предоставленные Сассманом сведения \enquote{с
менеджером ее кампании, директором по коммуникациям и советником по внешней
политике}.

Fox News отмечает, что упоминание Салливана в обвинительном заключении
демонстрирует \enquote{наиболее короткую к настоящему моменту связь расследования
Дарема с администрацией Джо Байдена}.  

О \enquote{факторе Салливана} и компромате на него говорит и политолог-американист
Дмитрий Дробницкий, который опубликовал карикатуру на скандального советника
Байдена и госсекретаря Блинкена: \enquote{Чем больше компромата на Салливана, тем хуже
с деэскалацией}.

\ii{18_02_2022.stz.news.ua.strana.1.dokumenty_usa_posolstvo_kiev.1.bajdeny_sallivan.pic.1}

\enquote{Когда появились признаки деэскалации и готовилась речь Байдена об Украине,
появились публикации о незаконной прослушке Трампа. Выяснилось, что его слушали
и во время предвыборной кампании, и когда он уже стал президентом. И что к
этому имел непосредственное отношение Салливан. Стали поднимать его
подноготную, показывать ролики, как он говорил, что \enquote{русские совершенно точно
взломали выборную систему и ведут Трампа на президентский пост}. Война
компроматов в США тут же отражается на международной повестке. Связь внутренней
борьбы кланов с тем, что происходит на международной арене, - это, пожалуй,
одно из самых опасных и ранее не встречавшихся явлений в геополитике. Например,
когда был \enquote{Уотергейт} в 1972-1974 годах против Никсона, на внешнем контуре было
спокойно}, - сказал Дробницкий в программе \enquote{Сейчас}.
