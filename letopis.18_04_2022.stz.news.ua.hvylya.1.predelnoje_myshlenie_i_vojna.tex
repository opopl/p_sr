% vim: keymap=russian-jcukenwin
%%beginhead 
 
%%file 18_04_2022.stz.news.ua.hvylya.1.predelnoje_myshlenie_i_vojna
%%parent 18_04_2022
 
%%url https://hvylya.net/analytics/250977-predelnoe-myshlenie-i-voyna
 
%%author_id dacjuk_sergij,news.ua.hvylya
%%date 
 
%%tags 
%%title Предельное мышление и война
 
%%endhead 
 
\subsection{Предельное мышление и война}
\label{sec:18_04_2022.stz.news.ua.hvylya.1.predelnoje_myshlenie_i_vojna}
 
\Purl{https://hvylya.net/analytics/250977-predelnoe-myshlenie-i-voyna}
\ifcmt
 author_begin
   author_id dacjuk_sergij,news.ua.hvylya
 author_end
\fi

\begin{zznagolos}
Война открывает путь не только к Преобразованию, но и к Преображению. Новый текст философа Сергея Дацюка.
\end{zznagolos}

В истории древнегреческой мысли существует курьез, не замечаемый большинством
интерпретаторов. В учении Анаксимандра «апейрон» (ἄπειρον) означает
принципиально неопределенное, бесконечное, беспредельное, неограниченное,
которое является началом всего.

\ifcmt
  ig https://hvylya.net/crops/8cdc34/834x0/1/0/2021/10/18/fOFf8G9cmu4mCBfyVyzAEGrAMhsD8uSsEFC5oHks.png
	@caption Сергей Дацюк. \enquote{Хвиля}
  @wrap center
  @width 0.8
\fi

Однако с чисто языковой точки зрения первичным представлением является вовсе не
«апейрон», а то, из чего происходит и постигается «апейрон». В переводе с
греческого «пейрос» (πειρος) — предел, частица «а» (ἄ) — отрицание, то есть
буквально — «то, что не имеет предела, ограничения». То есть представление о
Пределе является изначальным, потому как именно оно формирует происходящее от
него представление о неограниченности, бесконечности и беспредельности.

Из этого недоразумения возникают две большие традиции.

Анаксимандр Милетский (615-546 гг. до н.э.), ученик Фалеса Милетского
(637/624-547/558), а также учитель Анаксимена Милетского (585/560-525/502 гг.
до н.э.) и возможно Пифагора (570-490 гг. до н.э.). В Древней Греции того
времени говорили о «семи мудрецах» (οἱ ἑπτὰ σοφοί), а не о философах.
Анаксимадр — мыслитель-мудрец, потому как лишь после самоопределения Пифагора
как философа мыслителей-мудрецов начали ретроспективно называть философами.

Представление о бесконечном, наличном и единственном апейроне Анаксимандра уже
у Парменида (540/515-470 гг. до н.э.) оформляется как Бытие, и с тех пор
философия всю мудрость укладывает в это самое Бытие.

Другая традиция лет через 150 после Анаксимандра стараниями Протагора (485-410
гг. до н.э.) пытается вернуться к изначальному Пределу и начинает исследовать
предельное, неопределенное, находящееся между конечностью и бесконечностью,
делимостью и неделимостью, иное и проблемно определенное, множественное и
обнаруживающее пределы всяких представлений и т.д. — это традиция софистики.

Традиция философии вступает в столкновение с софистикой. И вместо того, чтобы
признать или хотя бы легитимировать внутри себя, не говоря уже о том, чтобы
поддержать, софистику, философия на основании критики софистов Сократом и
стараниями Платона и Аристотеля выставляет софистику как обман, глупость,
подтасовку и манипуляцию, а в перспективе лишает ее интеллектуальной
легитимности и предает забвению.

Чтобы вернуть начальность, важность и интерес к представлению о Пределе, нам
нужно последовать по мыслительным стопам софистов. Пределы софистичны, а не
философичны.

Предельное мышление софистики допускает беспредельное, поскольку беспредельное
в своем представлении происходит из предельного. Весьма немудро мыслить
беспредельное, не мысля Предел. Однако не всякое беспредельное мышление
начально допускает Предел, как это оказалось в философии, которая полагает
Предел как объективный, как противоречие, в том числе онтологическое, или же
утверждает, что всякое начало предел, но не всякий предел начало. Признать
предел началом значило бы для философии отказаться от изначальности и
единственности Бытия.

В софистическом залоге — Предел («пейрос») суть начало всего, а не бесконечное
(«апейрон») суть начало всего, как это осталось в философии.

В софистическом залоге: Предел начало, Предел сущность, Предел явлен, а
предельность проявлена, однако Предел — не основание. В этом смысле основание
разрабатывает уже философия.

В предельном мышлении нет ничего правильного или неправильного. В предельном
мышлении нет никаких норм. В предельном мышлении нет никакой мудрости, поэтому
философия здесь совершенно бессильна. Помышление Пределов изначально
софистично.

Оспаривание Пределов в мыслесопоставлении есть явленное утверждение самих
Пределов. Спор пределен. Мыслесопоставление не избегает Пределов, но может
ориентироваться вне Пределов.

Беседа есть попытка скрыть, игнорировать или открыто обозначить Пределы.
Сокрытие, игнорирование или недоговорные разногласия о Пределах превращают
беседу в спор. Договорные разногласия позволяют сохранить беседу.

Предел суть допущение Иного. Без Иного нет Предела. Бесконечность не поглощает
Иное так, что остается Бесконечностью. Поглотившая Иное Бесконечность
обращается в Неопределенность, Пустоту и Хаос.

Как угодно помысленное, волеемое, веруемое и патируемое не может избежать
Предела. Без Предела недопустимы ни Бытие, ни Небытие, ни Инобытие, ни
Внебытие, ибо они допустимы лишь через Предел. Хаос и Пустота предельны в своей
обратимости друг в друга через Бытие.

Все предельно, и Бог пределен. Не в парадоксе всемогущества, который в
авторских «Софизмах» показан как непарадокс, то есть загадка, которая имеет
отгадку. Бог пределен за пределами всего, что ему можно приписать, включая мир,
пространство и время. Пределы Бога — Пустота и Хаос. То есть Пустота и Хаос
внебожественны. Бог может творить из Пустоты, но Бог не может творить в
Пустоте. Бог может упорядочивать Хаос, но Бог не может упорядочивать внутри
Хаоса.

В любых представлениях начало и конец мира есть проявленные Пределы Бытия,
относительно которых можно обсуждать не только их явленную предельность, то
есть их допустимость, но и сущностную предельность, то есть насколько они
предельны в недопустимой инаковости.

Вопреки Пармениду, Платону, Аристотелю, Гегелю и Хайдеггеру — Бытие суть
неизначально: Бытие суть из Предела Иного.

Иное суть небытийно, поскольку Иное Бытие принципиально отлично от инобытия. По
Гегелю инобытие — свое другое, то есть противоположность бытия как понятия в
нем самом, полагающая его развитие и выход за свои пределы. Гегелевские Пределы
внутри Бытия суть Пределы мнимые, полагаемые в противоречии, где затем
наступает снятие. Иное Бытие суть прежде всего Вне-Это-Бытие.

Без преодоления даже Бытию не быть. Преодоление Предела суть встреча с Иным,
допущение-воображение и обналичивание-освоение Иного и опять
отступление-убегание Предела в Иное. Предел, таким образом, абсолютен:
Преодоление Предела суть его отступление-убегание, а не его пересечение, сдвиг
или уничтожение. В каком-то смысле своей псевдоактивности, Предел позволяет
лишь приблизиться к себе, но не пересечь себя.

Абсолют пределен во всяком представлении-воображении-допустимости об Абсолюте,
ибо представление\hyp воображение\hyp допустимость тоже предельны. Как предельны и
мыслимость\hyp волеемость\hyp веруемость\hyp патичность. Если Предел немыслим, неволеем,
неверуем или непатируем, то это значит, что мышление, воля, вера или патия не
были достаточно всеохватны, чтобы наткнуться на Предел.

В любой попытке мыслить «апейрон» не обойтись без «пейроса». Всякая
Беспредельность в своей мнимой абсолютности ограничена абсолютностью Предела.

Предел не вполне адекватно может быть понят в языке. Неопределенность предельна
в Определенности. Однако Предел не связан с Определенностью или
Неопределенностью. Предел не происходит из Определенности и не допускается как
определенный.

Предельность суть инаковости. Две разные неопределенности допустимо сущностно
предельны. Иное сущностно множественно. Столкновение с Пределом всегда
допустимо как столкновение с Пределами в их множественности.

Иное по своему помышлению разнообразнее, нежели неопределенность. Иное суть не
только Неопределенное, но также Немыслимое, Невообразимое и Недопустимое.
Однако даже такие о нем общие представления не исчерпывают его мощности в
помышлении, веровании, волении и патичности.

Полагание пределов — помышление иных допустимостей и возможностей. Это
постановка и ответ на вопрос: что не допустимо (принципиально, ни при каких
условиях) для этой онтологии или для Бытия вообще?

Предел преимущественно неинструментален и неоперабелен. Лимитология суть
оперирование проявленными Пределами. Но не все явленные Пределы проявлены.
Сущностные Пределы не все явлены.

В авторской Лимитологии проявленные отношения предельности рассматриваются как
Допредельность, Определивание, Запредельность, Черезпредельность,
Межпредельность, Обеспределивание. Предел суть неизбывен и непреодолим ни в
Допредельности, ни в Определивании. Предел суть и преодолим в Запредельности,
Черезпредельности, Межпредельности. Предел суть избывен в Обеспределивании.

Сущности, неявленные и непроявленные, беспредельны. Однако сущности, являющиеся
и проявляющиеся, предельны. Явленность сущностей привносит неопределенную
предельность. Проявленность сущностей определяет Пределы.

В соотнесении сущностей, явленных и проявленных, Пределы допустимы как
сущностные, а также мыслимы как явленные и обнаружимы как проявленные.

В явленном допустимы растворения-сгущения, течения, мерцания, однако сущность
не одна — сущностей много. И сущности суть не каша и не бульон. Между
сущностями нет различия, потому что нет ничего иного, кроме сущности, того, что
могло бы установить различие. Сущности разные, и эта разность предельна в
явленном.

Разные сущности соотносятся вне отношений, через Предел. Разные явленные
допустимы в различении Мезоса, где Предел полагается мышлением как сущностный.
Сосущности мышления в явленном соотносятся, а как собственно сущности
полагаются несоотносимыми. Лишь в проявленном явленные сущности вступают в
определенные отношения.

Мезос суть мыслительная конструкция соотнесения сторон, между которыми нет
отношений. Мезос суть не любовь и не патия. Мезос суть лишь допустимость
отношений, но не отношения собственно. Мезос мыслителен. Мезос суть не волевой
и не веровый. И в таком своем качестве мезос суть нефеноменологические
момент-единица-аспект (все слова неточные) мышления.

Мезо-мышление соотнесений отлично от мышления отношениями, где мыслятся
отношения, противоречия, разрешения\hyp снятия и т.д. Предельное мышление отлично
от мышления непредельного, где все пребывает в связях, разрывах, переходах и
т.д.

Всякие отношения предельны, а не бесконечны. Нормативное мышление задано через
отношения, поскольку отношения пронизывают все: и нас, и вокруг нас и даже
далеко вне нас, если это далеко все еще бытийно.

Ненормативное мышление вне отношений исследовалось в софистике как мышление на
пределах, в столкновении с пределами и в попытках помыслить запредельное.
Философия исчезающе редко мыслила предельно. Точно так же в богословии —
апофатическое богословие доминирует в то время, как катафатическое богословие
либо же считается принципиально невозможно, либо выглядит совершенно
беспомощным.

Небытие — не Предел Бытия, а негативная проекция Бытия. Не бывает никаких
отношений к Небытию.

Бессмысленно любить Небытие. Бессмысленно любить тех, кто любит Небытие.

Отношение к Небытию, то есть всякая любовь, всякая патия, приводит к
уничтожению такого отношения. Отношение к Небытию не самозамкнуто, а
самоотрицательно, самоубийственно.

Любовь часто пытается выдать себя за континуальный абсолют в предвосхищении
патии. Однако в мыслительном предвосхищении любовь контингентна, ненеобходима,
случайна. Неконтингентность любви удерживается самостно, то есть мыслительным,
волевым и веровым образом.

Если бы любовь была абсолютной, то в христианстве не нужно было бы призывать к
любви к Богу и к любви к ближнему. Бессмысленно призывать к тому, что и так
пронизывает самость.

Если же призыв к любви есть, то значит для самости любовь не абсолютна. И здесь нет никаких точек зрения или видений. Любовь — это не видение, а ориентация патии, веры, воли. Поскольку сама патия любви мимолетна и удержана может быть верой и волей через помышление ее как сущностной, сообразной самости, осмысленной и перспективной.

Точно так же мимолетны ненависть и вражда. Война и враг допустимы как столкновение с Пределом без допущения Иного.

Отношение к врагу разрушительно и предельно. Враг должен быть уничтожен, чтобы
сохранить себя и возможность иных отношений вне врага.

Любовь к врагу допустима вне состояния войны врага, который пытается нас
уничтожить. Пока враг нас ненавидит, его можно любить. Если враг нас
уничтожает, его нужно уничтожать безжалостно, но без ненависти и где-то даже
любя.

Чистое уничтожение врага допустимо вне патии — мыслительно-волевое и в
ориентации веры на смысл и перспективу вне вражды. Патия войны оказывается лишь
на стороне бытия самости, вне небытия врага, которое разрушительно, особенно,
если оно бессмысленно и бесперспективно.

Известно библейское изречение Матфея (Мф. 5:44) «А Я говорю вам: любите врагов
ваших, благословляйте проклинающих вас, благотворите ненавидящим вас и молитесь
за обижающих вас и гонящих вас.»

Это рефлексия для упреждения ресентимента. Война — уже не ресентимент, а
производство небытия. Поэтому отношение к врагу есть мимолетная антипатия, ведь
врага уничтожают, прекращая и саму антипатию. То есть патия к врагу как
антипатия, направлена на разрушение самой патии-антипатии.

Об отношении к врагу можно многое понять в изречении, приписываемом Александру
Македонскому: «Ищи друзей среди врагов, а врагов среди друзей».

То есть любовь к врагу понятна при его превращении в друга. Это возможно при
осознании бессмысленности и бесперспективности войны обеими сторонами или хотя
бы инициатором-агрессором. Пока враг воюет и не готов меняться, нет к нему
никакой любви.

Пограничная ситуация экзистенциальна: она есть стечение особых событий и
обстоятельств в экзистенциальном переходе. Пограничное состояние переживается
пред лицом смерти, вины, тяжелых жизненных испытаний, сильных стрессов.

Предельное состояние не следует путать с пограничной ситуацией. Предельное
состояние суть столкновение с Пределом. Предельное состояние суть
транзистенциальное — это встреча с Иным и предварение таких изменений, которые
имеют неэкзистенциальный характер, и трасценденция суть лишь одно из пониманий
такого Преодоления. Предельное состояние суть отклик на Зов — Зов Иного,
Сущности, Пустоты, Хаоса, где они все неразличимы.

Зов как транзистенциальный патичен, однако любовь экзистенциальна, а значит транзистенциально предельна. Однако любовь — всегда в иллюзии своей исключительной абсолютности. В этой своей иллюзии абсолютности любовь, как и ненависть, несамостоятельны и беспомощны.

Поэтому у любви есть заблуждение, что она абсолютна. Мышление мыслит себя и
свои Пределы. Любовь и патия несамодостаточны, поскольку не могут мыслить свои
Пределы. Такие же заблуждения есть и у воли, и у веры, и у потенции. Им тоже
кажется, что они абсолютны.

То есть мышление единственное из предвосхищений, мыслящее свою предельность.
Причина этого в том, что мышление рекурсивно. Без рекурсии всякая предельность
немыслима и порождает иллюзию абсолютности.

Непредельность отношений можно построить как шкалу отношенческой дистанции от
полного приятия другой стороны отношений до полного ее неприятия: растворение в
другой стороне, принадлежность, фанатичное преклонение, любовь (эрос, прагма,
платоновская, христианская, виктимная и т.д.), нравящность, почитание,
толерантность (терпимость), соседство или признание наличия, нейтральность,
другость, чуждость, ненависть, отмежевание, отторжение, уничижение, разрушение,
уничтожение, захоронение на огражденном кладбище от своей стороны.

Однако рекурсия мышления радикально соединяет начало и конец этой шкалы в
Мезосе соотнесения и в Пределе несоотнесения.

Свободное движение духа самостей по проявленной в мире этой шкале отношений
суть допущение войны. То есть война как столкновение с Пределом и
принудительное допущение Иного открывает возможность изменений в отнесении к
Иному: как преобразований, так и преображений.

Война суть принудительное столкновение с Пределами — мышления, воли, веры и
патии. Мышление не убивается войной. Мышление суть явлено, проявлено и есть как
война, как изменение, преобразование и даже преображение в столкновении с
Пределами.

Во время войны мышление в социальности проявлено и интерпретируется как
исключительно предельное. Причем только предельность мышления во время войны и
позволяет, собственно, сохранить причастность мышлению.

В проявлениях войны есть стороны, между которыми нет отношений, а есть лишь
мезо, которое предполагает соотношения вне самих сторон, или Предел, который
должен быть осмыслен в контексте изменений до Иного. Мезо-мышление или
предельное мышление — особые. Но они со-сущностны, явлены и проявлены.

Война допустима-вообразима-мыслима. Война сущностна, явлена и проявлена. Война
сущностей допустима, а значит допустимо явлена и может быть проявлена.

Война неизбывна и война есть, как неизбывен сам Предел. Столкновение с Пределом
в мышлении неизбежно, даже если вечно прятаться в сущностной гармонии веры,
равновесии патии мезоса, волевом балансе желаний и интересов социального мира и
дружбы.

Война есть также проявленное столкновение с Иным на Пределе, пронизывающем
Сущности, Явленное и Проявленное. Иное во встрече с ним на Пределе принадлежит
и сущностному, и явленному, и даже проявленному.

Война открывает путь не только к Преобразованию, но и к Преображению.
Преобразование суть непредельные изменения в ориентации на Иное. Преображение
суть предельные изменения в ориентации на Иное. В таком помышлении Преображение
как путь в Иное предельно. Преобразование есть переход. Преображение суть
Преодоление, а не переход.

Преобразование суть допустимые изменения. Преображение суть допустимо
недопустимые изменения.

Дух явлен даже сквозь войну. Дух в войне — это дух, имеющий дело с Пределами.
Дух, имеющий дело только с любовью и отношениями, суть дух незрелый или
самовлюбленный. Дух, не ведающий или отрицающий Пределы, суть дух
саморазрушающийся.

В совершенствовании духа Пределов не нужно избегать. Война полагает Предел
между экзистенцией и неэкзистенцией или квазиэкзистенцией, и важно его
удержать, а не отрицать Пределы в абсолютной любви.

Самоопределение причастных мышлению во время войны становится предельным: не
можешь мыслить предельно во время войны — не мысли, то есть перейди в волю
военного сопротивления, доверься вере или патии, наконец, уйди в социальность,
стань журналистом или пропагандистом, писателем или хроникером, и то и вообще,
потеряй сознание, замри, исчезни, растворись в Пустоте, и тогда мышление
пройдет сквозь Пустоту немыслящего.

Предел торжествует, даже если быть непричастным предельному мышлению.
