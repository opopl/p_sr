% vim: keymap=russian-jcukenwin
%%beginhead 
 
%%file 02_06_2021.stz.edu.lnr.lgaki.1.kukolnyj_multfilm_granpri
%%parent 02_06_2021
 
%%url https://lgaki.info/novosti/pervyj-kukolnyj-multfilm-sozdannyj-v-luganske-vzyal-svoj-pervyj-gran-pri
 
%%author_id 
%%date 
 
%%tags 
%%title Первый кукольный мультфильм, созданный в Луганске, взял свой первый Гран-при
 
%%endhead 
\subsection{Первый кукольный мультфильм, созданный в Луганске, взял свой первый Гран-при}
\label{sec:https://lgaki.info/novosti/pervyj-kukolnyj-multfilm-sozdannyj-v-luganske-vzyal-svoj-pervyj-gran-pri}

Руководитель мультипликационной студии «Юла» Детской академии искусств при
Академии Матусовского, выпускница и преподаватель кафедры художественной
анимации Анна Вегера вернулась из российской столицы, где представляла вуз и
Республику сразу на двух кинофестивалях «Свет миру. Дети» и «Московский
кораблик мечты». Что особенно приятно: на каждом из этих творческих турниров
Анна и наши «юлята» взяли высшие награды!

\ii{02_06_2021.stz.edu.lnr.lgaki.1.kukolnyj_multfilm_granpri.pic.1}

— Фильм «Санитарка», созданный воспитанницами студии Полиной Маловастой и
Екатериной Колгановой совместно с ребятами из Хабаровска и Нового Уренгоя под
руководством режиссера и сценариста Сергея Струсовского, получил диплом 1-й
степени в номинации «Наши герои» на самом светлом и душевном фестивале «Свет
миру. Дети», — рассказала Аня. — Хотелось бы выразить благодарность всему
оргкомитету фестиваля и особенно Олегу Свободину за гостеприимство,
интереснейшие экскурсии по Ярославскому краю, мастер-классы от ведущих
специалистов в области мультипликации и кино, за ценные советы при разборе
фильмов, за теплый прием и дружескую атмосферу, возможность встречи с друзьями
и новые знакомства!

\ii{02_06_2021.stz.edu.lnr.lgaki.1.kukolnyj_multfilm_granpri.pic.2}

А вот первый кукольный мультфильм, созданный в Луганске, придуманный и
воплощенный в нашей «Юле», получил свой первый Гран-При — на фестивале
«Московский кораблик мечты».

— Награду студии вручил художник-мультипликатор, режиссер, сценарист,
заслуженный деятель искусств РФ Леонид Носырев, автор любимых с детства
мультфильмов «Антошка», «Два весёлых гуся», «Волшебное кольцо», — уточнила Анна
Вегера.

Мы от всей души поздравляем и ждем новых мультов от «Юлы»!

Фото из личного архива Анны Вегера.
