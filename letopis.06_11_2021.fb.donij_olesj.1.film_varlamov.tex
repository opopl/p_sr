% vim: keymap=russian-jcukenwin
%%beginhead 
 
%%file 06_11_2021.fb.donij_olesj.1.film_varlamov
%%parent 06_11_2021
 
%%url https://www.facebook.com/oles.doniy/posts/5040018159359809
 
%%author_id donij_olesj
%%date 
 
%%tags donij_oles,film,rossia,ukraina,varlamov_ilja.rossia.blogger
%%title Новий російський інтернет-фільм про Україну
 
%%endhead 
 
\subsection{Новий російський інтернет-фільм про Україну}
\label{sec:06_11_2021.fb.donij_olesj.1.film_varlamov}
 
\Purl{https://www.facebook.com/oles.doniy/posts/5040018159359809}
\ifcmt
 author_begin
   author_id donij_olesj
 author_end
\fi

Відома радіостанція запросила вчора прокомунтувати новий російський інтернет-
фільм про Україну. Якого "відомого російського блогера".

Я ані прізвища цього блогера раніше не чув, ані фільмів не бачив.

Спробував подивитися. Як для російського телеглядача- порівнянно поміркований.
Хоча, звісно, для нас (українців) все одно виглядає неприйнятною інтерпритацією.

Та я про інше. 

Цей фільм за пару днів набрав більше мільйона переглядів. По коментах- майже
половина це українські глядачі. Дивляться і обурюються... Обурюються і
дивляться...

Власний український продукт такою увагою українських глядачів не обдарований.

Залежність від Росії продовжується.

Навіть якщо проявляється у "праведному обуренні".

Комплекс власної меншовартості..

Навіть коли він "навиворіт", коли українці починають стверджувати, що "Україна-
батьківщина мамонтів", чи людства. Так само як стверджують росіяни, до речі.

Уваги і поваги до власного інтелектуального продукту в Україні мало.

І коли я постійно і послідовно повторюю, що країна потребує Світоглядної
Реформації, то це, зокоема означає, і необхідність мислити самостійно. І
поважати тих, хто мислить самостійно, і створює інтелектуальний продукт.
Власний продукт.

А той, хто не помічаж український інтелектуальний продукт, і далі буде
обговорювати російський продукт.

А це теж - допомога Росії. У конкурентному змаганні інтелектів
