% vim: keymap=russian-jcukenwin
%%beginhead 
 
%%file 10_10_2021.fb.fb_group.story_kiev_ua.1.gruppa.cmt
%%parent 10_10_2021.fb.fb_group.story_kiev_ua.1.gruppa
 
%%url 
 
%%author_id 
%%date 
 
%%tags 
%%title 
 
%%endhead 
\subsubsection{Коментарі}

\begin{itemize} % {
\iusr{Вікторія Святненко}
Спасибо Вам и всем участникам чудесной и любимой группы!@igg{fbicon.heart.suit}

\iusr{Natasha Levitskaya}

Спасибо, Олег! Это была лучшая идея и группа получилась лучшая!

Я вступила в группу меньше, чем через месяц после её создания, чему бесконечно
рада, встретила здесь прекрасных людей и приобрела чудесных
друзей, которые теперь друзья в жизни! А каждая встреча с
КИшниками оставляет в душе тепло! @igg{fbicon.heart.red}

УРА!!!  @igg{fbicon.party.popper}{repeat=3} 

\begin{itemize} % {

% -------------------------------------
\ii{fbauth.koval_oleg.kiev.ukraina.helg_smith.kievskie_istorii}
% -------------------------------------

Natasha, ура! @igg{fbicon.heart.red}
\end{itemize} % 

\iusr{Nadiya M Shana}
Спасибо, Олег! И даже расстояния не мешают чувствовать тепло и дух Киева и киевлян.  @igg{fbicon.hands.shake} 

\iusr{Maksym Oleynikov}
Вітаю з досягненням 100 000! Дай Боже, щоб згідно закону діалектики кількість переходила в якість @igg{fbicon.thumb.up.yellow} 

\iusr{Галина Гурьева}
Долгих счастливых лет Киевским историям!

\iusr{Марина Шкляревская}
Велике ДЯКУЮ за радість спілкування!

\iusr{Таня Головина}
Група чудова! @igg{fbicon.heart.sparkling}  Группа замечательная! @igg{fbicon.house.with.garden} 

\iusr{Elena Evgenievna Zelinka}
Вы - лучшие!!!

\iusr{Евгения Ерёменко}
Спасибо, Олег, за нашу группу и Ваше доброжелательное терпение  @igg{fbicon.face.smiling.eyes.smiling} 

\iusr{Майрам Разакова}
Спасибо за группу, чудесные истории !!! Была дважды в Киеве и очень полюбила!
Творческих успехов и ещё больше интересных историй!!!

\iusr{Татьяна Ховрич}
Спасибо!!!

\iusr{Ирина Ещенко}

Спасибо, Олег! В этой замечательной группе, я встретила своих одноклассников,
которые живут сейчас далеко, но по-прежнему любят наш любимый Киев.

\iusr{Dilettante Inlife}

Четыре года всего??? Надо отдать должное или невероятной удаче, или
организаторскому таланту, или тонкому чувству темы, но за четыре года создать
такую притягательную и такую сплоченную группу - это очень сильная работа. Не
пою панегирик, но кто понимает - тот оценит. Я рад, что конечно же случайно,
попал в нашу группу. Спасибо, Олег и все сто тысяч. Вы там отметьте как
положено.

\begin{itemize} % {
\iusr{Олег Коваль}
Благодарю, особенно за нашу, потому что это действительно наша группа.
\end{itemize} % }

\iusr{Sasha Kievskaya}
Группа сейчас интернациональная. Спасибо всем организаторам и contributors!  @igg{fbicon.heart.purple}  @igg{fbicon.heart.blue}  @igg{fbicon.heart.yellow} 

\iusr{Татьяна Сирота}

Отличная возникла идея, которая прекрасно реализовалась.
КИ -самая лучшая группа!!! @igg{fbicon.hearts.two} 

\begin{itemize} % {
\iusr{Ірина Гнатюк}
\textbf{Татьяна Сирота} Дякую Вам за Ваші неповторні дописи, які вже стали візитною карткою моєї улюбленої групи.
\end{itemize} % }

\iusr{Вадим Горбов}
Кто он? Стотысячный участник ( ца) группы?!

\begin{itemize} % {
\iusr{Олег Коваль}
Вадим, оставим в тайне...

\iusr{Вадим Горбов}
\textbf{Олег Коваль} значит, дама. Инкогнито, в секрете от мужа.)

\iusr{Олег Коваль}
Вадим, не факт.  @igg{fbicon.smile}  @igg{fbicon.face.smiling.eyes.smiling} 

\iusr{Maksim Pestun}
Стотысячному участнику полагается приз. Предлагаю книгу своих стихов. (на правах рекламы).
\end{itemize} % }

\iusr{Ирина Данькова}
Очень интересные и познавательные истории!!

\ifcmt
  ig https://scontent-frx5-2.xx.fbcdn.net/v/t39.1997-6/cp0/s110x80/16685601_1341102165951886_4721639878444449792_n.png?_nc_cat=1&ccb=1-5&_nc_sid=ac3552&_nc_ohc=FF2zI1dVoHwAX8IZ1NU&_nc_ht=scontent-frx5-2.xx&oh=5b41aaa4d06252fbee125da86d361caa&oe=6168F1FA
  @width 0.2
\fi

\iusr{Наталия Привалко}
Удачи!

\iusr{Светлана Семенец}
Всего 4 года?? Невероятно!

\begin{itemize} % {
\iusr{Олег Коваль}
Светлана, да, 24 декабря 2017.

\iusr{Светлана Семенец}
\textbf{Олег Коваль} , счастливый оказался день для всех нас.

\iusr{Светлана Манилова}
Светлана, я уже себя без КИ не помню... @igg{fbicon.face.grinning.big.eyes} 

\begin{itemize} % {
\iusr{Олег Коваль}
Светлана, и я.  @igg{fbicon.smile} 

\iusr{Светлана Семенец}
\textbf{Светлана Манилова} , мне кажется, что без КИ нет смысла в соц сети. Ну правда, зачем, если не за приятными эмоциями заходить в фб?
\end{itemize} % }

\end{itemize} % }

\iusr{Tetiana Petrovska}

Олег, благодарю за гениальную идею создания такой группы! Доброта и гармония в
группе зашкаливала! Сколько трогательных, трагичных и даже драматичных историй
нам рассказали участники. Именно в этой группе я поделилась историей своей
семьи и увидела искренний неподдельный интерес. Популярность группы растет
вместе с ее участниками, значит есть социальный запрос. Быть добру!

\begin{itemize} % {
\iusr{Олег Коваль}
\textbf{Tetiana}, благодарю за участие! @igg{fbicon.heart.red}

\iusr{Алексей Добрынин}
очень большой плюс
\end{itemize} % }

\iusr{Мария Елинская}

Олег, велика подяка за цікаві розповіді про наше Місто! Розповіді як ковток
джерельної води. Дякую Добре було б зробити розповідь про наш знищений курорт
Пущу Водицю, бо зараз цегляні мури на місті красивих дерев'яних будинків

\begin{itemize} % {
\iusr{Олег Коваль}
\textbf{Мария}, дякую!@igg{fbicon.heart.red}
\end{itemize} % }

\iusr{Мария Елинская}
Бажаю Вам творчої наснаги!

\iusr{Natalie Zhironkina}
Благодарю вас \textbf{Олег Коваль} за создание и ведение такой прекрасной группы!
Быть добру - 100\%.

\begin{itemize} % {
\iusr{Олег Коваль}
\textbf{Natalie}, благодарю!@igg{fbicon.heart.red}
\end{itemize} % }

\iusr{Людмила Старовойтенко}
Спасибо будем рады пишите нам интересно ваше мнение о историях

\iusr{Людмила Старовойтенко}
Спасибо за рассказ картина супер

\iusr{Светлана Прокопенко}
Благодарю Вас, за интересные истории , атмосферу доброжелательности и любви к городу и людям .
Быть добру!

\iusr{Maksim Pestun}

Спасибо Олег! Многие киевляне, разбросанные по свету, вернулись здесь в город
своего детства. И пусть той атмосферы уже давно нет в реале, в КИ всегда тепло
и душевно!


\iusr{Maksim Pestun}

\obeycr
Как же тянет в детство окунуться,
Все заботы хоть на миг забыть!
В дворик своей юности вернуться,
Чтобы запах детства ощутить...
\restorecr

\begin{itemize} % {
\iusr{Светлана Манилова}
\textbf{Maksim}, это лейтмотив нашей группы!
\end{itemize} % }

\iusr{Ирина Нищимная}
КИ благодарю твоих создателей,ты душа и сердце @igg{fbicon.heart.red}{repeat=2}
нашего родного Киева,,

\iusr{Dina Kozlova} БЫТЬ ДОБРУ !  СПАСИБО !

\iusr{Татьяна Кириленко} Огромное спасибо, Вам, Олег! Без вашей группы ,уже
никак! Вы-Лучшие! Читаю все! И жду новые публикации!

\iusr{Діденко Юлія}

Олег, спасибо  @igg{fbicon.hands.raising}  за возвращения веры в людей,
благодаря Вам и вашей группе  @igg{fbicon.face.happy.two.hands} Душевно, тепло,
интересно  @igg{fbicon.hands.raising}  Благодарю  @igg{fbicon.rose} 

\begin{itemize} % {
\iusr{Олег Коваль}
\textbf{Діденко Юлія}, Благодарю! @igg{fbicon.heart.red}
\end{itemize} % }

\iusr{Арно Швед}
Рад быть вместе!!!@igg{fbicon.heart.red} @igg{fbicon.hands.applause.yellow} 

\begin{itemize} % {
\iusr{Олег Коваль}
Администратор
Арно, взаимно!@igg{fbicon.hand.victory} @igg{fbicon.heart.red}
\end{itemize} % }

\iusr{Леся Лагуна}

Замечательная группа! Благодарю Вас, Олег, за воплощение мечты и создание такой
интересной и очень нужной группы. Благодарю группу «Киевские истории» за
необыкновенные история, рассказы, Воспоминания, фотографии, экскурсии - за
очень нужный ( нам киевлянам), интересный, духовный и душевный контент!

Быть добру!

\iusr{Александр Венге}

Красивая идея. Благородная идея. Очень киевская )) идея.

\iusr{Tatyana Krivtsova}
Запрос на добро - к добру!  @igg{fbicon.face.smiling.eyes.smiling}  @igg{fbicon.heart.beating}  @igg{fbicon.rose} 

\iusr{Светлана Манилова}
Спасибо, Олег, за "душу нашего клуба киевлян". Это КИ навсегда поселились в моей душе!@igg{fbicon.heart.red}

\iusr{Maksim Pestun}

Когда нас будет 500 000 мы достигнем максимального уровня населения города,
который мы все любим и о котором вспоминаем в своих экскурсиях и рассказах (467
591 человек на сентябрь 1917 года). Надеюсь доживем до этого!

\begin{itemize} % {
\iusr{Manz Ura Pochernina}
\textbf{Maksim Pestun} диживем обязательно , щоб все были здоровы  @igg{fbicon.flame}  @igg{fbicon.heart.sparkling} @igg{fbicon.exclamation.mark.double}  @igg{fbicon.100.percent} 
\end{itemize} % }

\iusr{Людмила Ива}

@igg{fbicon.heart.red}

\iusr{Tamara Khmara}

Я до 1961 года жила на Пирогова, нас выселили с подвала в Соломенский район
недалеко от КВО ( Министерство обороны), шестьдесят лет живу сдесь, а постоянно
тянет на старое место жительства. В моем прекрасном высотном историческом доме
сделали гостиницу, не видоизменили, в подвале где была наша квартира сделали
кафешку. Я периодически туда езжу.

\begin{itemize} % {
\iusr{Irena Visochan}
\textbf{Tamara Khmara} Хорошо, что ты бываешь в доме, где прошло твое детство, стены сохраняют воспоминания и тепло ваших рук.@igg{fbicon.heart.red}
\end{itemize} % }

\iusr{Лариса Шелест}
Быть добру)))

\iusr{Жанна Кулишова}
Благодарю всех участников и создателей группы за интересные рассказы  @igg{fbicon.tulip} 

\iusr{Елена Сергеева}
Прекрасная идея и создання группа!

\iusr{Nataliya Kaluzhanina}

\ifcmt
  ig https://scontent-frx5-1.xx.fbcdn.net/v/t1.6435-9/245009389_4460258804051939_2435903391645376483_n.jpg?_nc_cat=105&ccb=1-5&_nc_sid=dbeb18&_nc_ohc=wX2M69zVl4AAX-ydfZc&_nc_oc=AQmmB7n2hZ2WcJ7DMfB-m6RuJBuxCuwmuznN8tBARoplLguXcxt4miVQHDnD4qYkE3Q&_nc_ht=scontent-frx5-1.xx&oh=611b630f2c423e4e4b9c0b77e6265c9d&oe=6189CB83
  @width 0.4
\fi

\iusr{Игорь Мацак}
Спасибо большое ... Группа очень интересная ... С большим удовольствием читаю всех участников.

\iusr{Nataliya Kaluzhanina}

\ifcmt
  ig https://scontent-frt3-1.xx.fbcdn.net/v/t1.6435-9/245092922_4460260750718411_5060135110242940595_n.jpg?_nc_cat=107&ccb=1-5&_nc_sid=dbeb18&_nc_ohc=LOuT7THsjZEAX_0tYNQ&_nc_ht=scontent-frt3-1.xx&oh=0d7204b6e328f7232cd3822d212fa379&oe=618B9D8B
  @width 0.4
\fi

\iusr{Kateryna Masliechkina}
Это лучшая группа, которая мне встречалась на просторах ФБ! @igg{fbicon.heart.red} Спасибо @igg{fbicon.heart.red}

\iusr{Татьяна Соловьева}

Я сделала первую публикацию 1 декабря, почти год тому назад. Она была о
рождественских витринах ЦУМ-а и воспоминаниях детства. Столько комментариев и
новых друзей! Разделение чувств, эмоций - делает их ещё ярче. Спасибо за
замечательную компанию в Киевских историях)

\begin{itemize} % {
\iusr{Светлана Манилова}
Татьяна, 8 декабря. \url{https://www.facebook.com/groups/story.kiev.ua/posts/1534302790099850/}

\begin{itemize} % {
\iusr{Татьяна Соловьева}
\textbf{Светлана Манилова} Спасибо большое, Светочка @igg{fbicon.heart.red}{repeat=3}! Почти год прошёл, не верится)

\iusr{Светлана Манилова}
Татьяна, уже 50+ прошло, и тоже не верится. @igg{fbicon.smile} 

\iusr{Татьяна Соловьева}
\textbf{Светлана Манилова} Это точно!
\end{itemize} % }

\end{itemize} % }

\iusr{Евгения Бочковская}

Поздравляю ВСЮ 100 000-ю аудиторию

"Киевские Истории" с таким знаменательным событием роста, единения и общего
единодушия (единой доброй души Киевлян) в своей любви к родному г.Киеву.

Поздравляю созИдателя и администратора нашей группы "Киевские Истории" Олега
Коваля и активных его помощников - модераторов с таким, уже знаковым, ростом
нашей группы ! Люди стремятся к Добру ! Люди открывают свои сердца и души нам в
своих историях и рассказах и находят новых добрых друзей.

Сколько новых "страничек" мы уже узнали о нашем родном городе, благодаря
историям публикуемым участниками нашей группы !

А значит "Быть добру" и дальше !  @igg{fbicon.heart.beating}  @igg{fbicon.hearts.two}  @igg{fbicon.book.open} 

\iusr{Валентина Макаренко}
Дякую Вам за Київські історії, де є цікаві спогади , чудові історії нашого міста, атмосферу добра і взаємоповаги!

\iusr{Татьяна Гурьева}
Мы очень ценим такое общение в кругу единомышленников. Уверена, самое интересное еще впереди!!! @igg{fbicon.100.percent} 

\iusr{Андрей Надиевец}

Организаторы группы в канун Рождества будут награждены высокими
государственными наградами, указ будет опубликован в газете "Голос Украины".

\begin{itemize} % {
\iusr{Светлана Манилова}
Андрей, шутник, однако. @igg{fbicon.wink} 

\iusr{Андрей Надиевец}
\textbf{Светлана Манилова} Почему, может , наконец то начнут награждать достойных,а то,не всегда так...
\end{itemize} % }

\iusr{Петр Кузьменко}

Благодарю Вас, Олег и Ваших верных друзей-помощников, модераторов группы за
великолепное интеллектуальное и эстетическое наслаждение читать замечательные
публикации авторов нашей лучшей киевской группы, доброй и душевной! Я очень
рад, что включился в КИ, почти с истоков. Даёшь миллион участников - людей
неравнодушных к лучшему Городу Земли! @igg{fbicon.hands.shake} 

\begin{itemize} % {
\iusr{Олег Коваль}
\textbf{Петр}, я благодарен Вам за участие!
\end{itemize} % }

\iusr{Виктория Воронина}
Олег, спасибо! Такая группа, истории, общение - настоящая драгоценность сегодня. Быть добру!

\iusr{Тома Храповицкая}

Добро будет! Обязательно! Группа позитива и вдохновения, тёплых домашних и
исторических воспоминаний! Всё взаимосвязано, всё красиво и щемяще, всё
притягивает и эмоции зашкаливают! Все приложившие свои усилия, свою любовь,
свои знания, Все оставили глубокий отпечаток, в душах и сердцах читающих и
комментирующих! Спасибо Вам Олег, Вы почувствовали...в нужное время, накануне
волшебства...

\begin{itemize} % {
\iusr{Олег Коваль}
\textbf{Тома}, благодарю! @igg{fbicon.heart.red}

\iusr{Светлана Манилова}
\textbf{Тома}, так и мы Олега почувствовали! Вот, в чем еще бОльшая прелесть! @igg{fbicon.heart.red}

\begin{itemize} % {
\iusr{Тома Храповицкая}
\textbf{Светлана Манилова} Светочка! Я пришла намного позже, и написала что чувствовала.

\iusr{Светлана Манилова}
\textbf{Тома}, неважно, когда пришла. Важно, что почувствовала! @igg{fbicon.heart.red}
\end{itemize} % }

\end{itemize} % }

\iusr{Елена Антонова}

Я просто участник группы, не такая активная как хотелось бы..)) но я с большим
удовольствием читаю эти киевские истории и воспоминания! Спасибо вам! Быть
добру!!

\iusr{Светлана Лобанова}
Люблю тебя Киев

\ifcmt
  ig https://scontent-frt3-2.xx.fbcdn.net/v/t1.6435-9/245196677_424271085701387_625888257802884130_n.jpg?_nc_cat=103&ccb=1-5&_nc_sid=dbeb18&_nc_ohc=vYuCvHVXkfgAX_QTeH7&_nc_ht=scontent-frt3-2.xx&oh=e439750b6ab8952eb989de8441154e1b&oe=6189C1CE
  @width 0.4
\fi

\iusr{Людмила СУХАНЮК}

Благодарю всех причастных к рождению увлекательной группы! Всегда с огромным и
неподдельным удовольствием читаю чьи-то воспоминания , расплываюсь в улыбке и
«будоражу» свои, отдыхающие в глубине души , яркие эмоции, однажды пережитые...
но, в таком уже далёком прошлом, где многие ещё были здесь...

% -------------------------------------
\ii{fbauth.bandurin_vadim.kiev.ukraina}
% -------------------------------------

Прекрасно всё сложилось. Не нам решать о том, какие винтики, шурупчики, или
огромные паровые машины задействованы здесь. А может и виртуальный
супер-компьютер киевского Духа.

Как бы то ни было - огромное спасибо, \textbf{Hélg Smith} за нажатие кнопки, в
результате чего объединилось столько людей! По моему скромному мнению,
правильно всё сделано. И пусть мы все - разные. Пусть - несогласные друг с
другом. Но минус на минус не всегда даёт плюс, а разнозаряженные частицы
притягиваются. Иногда приходится отключать комментарии, но это - живые люди,
создающие пространство разных людей в одном Городе. И пусть мы будем РАЗНЫЕ.
Так всегда рождается что-то новое.

Ещё раз - искренняя благодарность за эту группу.

Много пафоса. Да, я знаю. Я его вполне нейтрализую, если надо будет.
@igg{fbicon.smile} 

\iusr{Клим Форманчук}

О картине: Интересная, своеобразная работа мастера ! Трудно не узнать
Андреевский спуск, и знаменитую не только съёмками на лестнице эпизода
неудачного венчания Голохвостого, Андреевскую церковь. А знаменитую ещё и тем,
что её вроде как Варфоломей оттяпал. Правда я в делах церковных не очень, может
подскажет кто ?

\iusr{Вита Вовченко}
Люблю эту группу! Спасибо создателям и участникам! @igg{fbicon.face.eyes.star}
@igg{fbicon.heart.red}

% -------------------------------------
\ii{fbauth.zadvornov_viktor.kiev.ukraina}
% -------------------------------------

Прошу прощения за смелое сравнение. Когда хочется чего-нибудь особенного, идут
к холодильнику. Когда хочется душевного, пусть и виртуального, общения, лично я
иду в ФБ, это с тем, чтобы пообщаться в КИ, прочесть занимательную историю. Во
время последней встречи в Доме писателей не оставляло чувство родства со
многими незнакомыми людьми. Скажете, так не бывает? Отнюдь. Попробую
аргументировать. Пусть и на пару часов, переступив порог Дома писателей, я
окунулся в атмосферу добрых рассказчиков, умелых кондитеров, любителей давней и
новейшей истории... Там - за окном засыпал наш любимый красивый город, а в
светлом и необычном по красоте зале мы говорили о незабываемом и слушали его
пульс. Пульс Киева. Пусть подольше нам слышать его! Желаю это умение продлить
надолго каждому участнику КИ.

\iusr{Тамара Метленко}
С большой благодарностью

\ifcmt
  ig https://scontent-frx5-2.xx.fbcdn.net/v/t39.1997-6/cp0/s110x80/118211260_659965514618313_1559572347261946985_n.png?_nc_cat=1&ccb=1-5&_nc_sid=ac3552&_nc_ohc=2rjnXtdbHRIAX9nfsyl&_nc_ht=scontent-frx5-2.xx&oh=dfe6c19365c9794e78447f8362932355&oe=6169E6AB
  @width 0.2
\fi

\iusr{Оксана Денисова}

Олег, спасибо огромное за идею создания такой группы! Спасибо за ту душевную
атмосферу, которая в группе существует. И за те силы, энергию, время , которое
Вы и все модераторы группы отдаёте на то, чтобы эта необыкновенная аура группы
существовала и объединяла всех нас. Нас, киевлян, которые любят свой
единственный в мире , неповторимый Город! Люблю всех и ценю, что я с вами!!!

\iusr{Виктор Задворнов}

Администратору и модераторам - почет и уважение! От бессонницы лучше всего
принять стакан вина (наливки от Олег Коваль). Сразу не уснешь, но бодрствовать
будет веселее. Но я предпочитаю засыпать под чтение интересных историй от,
простите за тавтологию, КИ. Спокойной ночи!

% -------------------------------------
\ii{fbauth.dubik_natalia.kiev.ukraina}
% -------------------------------------

В группе живёт настоящая энергия Киева, так как все участники являются её
носителями и любят Киев безусловной любовью. Мы Киев защищаем и оберегаем своей
любовью и он, несмотря ни на что, вернеться в состояние каким он должен быть.
Благодарю Вас, за прекрасное место в файсбуке, где можно узнать невероятно
интересные истории киевлян разных поколений @igg{fbicon.hands.applause.yellow}
процветания всем модераторам и участникам группы, и, конечно, нашему любимому
городу Киеву  @igg{fbicon.heart.red}

\iusr{Іллона Зейкан}
Дякую за групу, її сердечність і теплоту! @igg{fbicon.bouquet} 

\iusr{Елена Кравченко}

Це найкраща Київська група. Дякую Вам величезне, що саме Ви і Ваші однодумці
змогли направити в добре, щире, доброзичливе русло учасників. Щиро дякую всіх за
цікаві та пізнавальні історії нашого міста. Саме таким і був Київ мого
дитинства-добрий, затишний, величний, душевний.... @igg{fbicon.heart.red}

% -------------------------------------
\ii{fbauth.zagrebelna_irina.kiev.ukraina}
% -------------------------------------

Шикарная мысль посетила Вас, Олег, четыре года назад. Сколько бы мы все
потеряли, если бы этого не произошло. Теперь для нас эта группа, как
бесконечная книга о нашем любимом городе и его обитателях. Интересная,
познавательная, напоминающая о нашем прошлом и повествующая о настоящем. И как
здорово, что однажды я тоже стала её читать! Искренне благодарю за нашу
замечательную группу!

\iusr{Marina Ganopolska}

Спасибо, что группа есть!

\iusr{Ereena Shvarts}

Спасибо вам, Олег, за рождение прекрасной идеи, создание замечательной и
интересной странички, которую с удовольствием читают киевляне по всему миру!

Всегда рада новой информации, интересным историям и с нетерпением ожидаю новых...
удачи вам и всем групчанам!

\iusr{Леонид Синяк}

Дякую Вам) Не все влаштовує, але - це все про моє місто). І я дуже дякую усій
спільноті за чудові розповіді про моє місто!

\iusr{Олена Медведева- Прицкер}

Спасибо , Олег ! Это удивительная удача - собрать и заочно объединить 100000
людей в группу КИ . Когда я хочу представить себе такое количество людей , я
вспоминаю , что такое количество болельщиков раньше вмещал наш республиканский
стадион ! Будем держаться !

\iusr{Ольга Широкова}
Спасибо Вам, Олег! @igg{fbicon.hands.applause.yellow}  @igg{fbicon.heart.sparkling}  @igg{fbicon.hands.applause.yellow} 

\iusr{Татьяна Новза}
Дякую, дуже приэмна компанія. Картина чудова.

\iusr{Оксана Фурман}
Прекрасная группа!

\iusr{Ольга Еремеева}

Спасибо за прекрасную идею и создание замечательной группы. С удовольствием
читаю очерки из жизни любимого города, вместе с участниками группы прохожу по
его улицам, узнаю много нового. Еще раз спасибо Вам и модераторам группы за
эстетическое удовольствие от общения с Вами.

\iusr{Георгий Майоренко}
 @igg{fbicon.100.percent}  00 это круто! Славная цифра! А интересно, какова ситуация с забаненными? Они удаляются из группы или у них есть шанс исправиться и их разбанят?

\begin{itemize} % {
\iusr{Игорь Моржецкий}
\textbf{Георгий Майоренко} 

Жора, у них есть шанс. Меня, например, забанили за то, что в комменте выложил
ссылку на песню к годовщине Бабьего Яра. И песня была хороша, и в тему очень.

Но ссылки, как оказалось, нельзя. Дура Лекс Сед Лекс, как говорили наши далёкие
предки.

А, к чему я. Потом отключили бан. Через какое-то время.

\begin{itemize} % {
\iusr{Светлана Манилова}

\textbf{Игорь}, хочу уточнить. Проверила по журналу действий администрации группы. Вас не забанили, а перевели в немой режим на неделю. Это не одно и то же. @igg{fbicon.smile} Спасибо, что с пониманием относитесь к правилам группы.

\iusr{Игорь Моржецкий}
\textbf{Светлана Манилова} Спасибо, Светлана. Выходит, что может быть и хуже? Кстати, мне эта неделя показалась вечностью - так много публикаций и авторов, которым хотелось сказать "спасибо"

\iusr{Светлана Манилова}

\textbf{Игорь}, да, может быть и хуже. Злостных нарушителей правил группы и правил поведения в группе исключают и блокируют, что означает невозможность возвращения назад. Бывают исключения, конечно.  @igg{fbicon.smile} 
\end{itemize} % }

\iusr{Олег Коваль}

\textbf{Георгий}, заблокированные удаляются из группы. Раз в год в День
рождения группы мы амнистировали всех заблокированных, но в этом году нам
придётся отказаться от этой практики.

\begin{itemize} % {
\iusr{Георгий Майоренко}
\textbf{Олег Коваль} Спасибо. Просто было интересно.
\end{itemize} % }

\end{itemize} % }

\iusr{Andrew Rudyakov}
Красота...

\iusr{Елена Корниенко}
Спасибо! Всегда читаю с интересом рассказы о родном городе!

\iusr{Елена Нижникова}

Поздравляю!100 тыс - это успех !

Но в Киевских Историях ( даже как- то не хочется ставить кавычки, это... как в
семье ) даже не столько важно количество , сколько качество.

Это не подписчики поп- звезды миллионными тиражами

Это дух, богатство каждого, потрясающая одаренность, очень высокая культура
общения. Создатель группы невероятно правильно и умело ведёт группу и создаёт
атмосферу.

Не пропускаю ни одного поста КИ, хотя в ФБ всегда есть, что почитать.

Но чтобы наполниться духом семьи, иду в КИ.

\iusr{Людмила Бабич}

Поздравляю!!! Много воспоминаний. Можно писать, но люблю участвовать в
комментах! В мелочах нахожу невероятно интересные факты!!! Благодарю сердечно
всех!!!

\iusr{Ксения Паламарчук}

Спасибо за уютную атмосферу группы, за интересные истории, душевность, почти физически ощутимую.

\iusr{Светлана Александренко}

Спасибо большое Вам, Олег, за прекрасную идею, за то,что объединяете всех неравнодушных к нашему родному Киеву!

Я киевлянка от рождения , прожила полжизни в самом центре столицы. И мне тоже есть о чем рассказать.

% -------------------------------------
\ii{fbauth.beregova_ljudmila.kiev.ukraina.prezident.klub.kijanochki}
% -------------------------------------

Сердечно благодарю Вас, Олег!  @igg{fbicon.heart.sparkling} 

Это удивительное место, где можно увидеть, услышать, пообщаться,

поддержать , а иногда и защитить аутентичных киевлян.

Город разростался,, а мы здесь родились выросли и остались. У нас есть в сердце
свой Киев. И если мы делимся, то это очень деликатно!

Благодарю всех тонких неповторимых авторов постов! @igg{fbicon.heart.sparkling} 

Всем здоровья @igg{fbicon.heart.sparkling}, будем мудрыми!

% -------------------------------------
\ii{fbauth.beregova_ljudmila.kiev.ukraina.prezident.klub.kijanochki}
% -------------------------------------

\ifcmt
  ig https://scontent-frx5-2.xx.fbcdn.net/v/t39.1997-6/cp0/s110x80/155456952_176578870689182_7295447585296907464_n.png?_nc_cat=1&ccb=1-5&_nc_sid=ac3552&_nc_ohc=OkwB18PlCWoAX_JGCcE&tn=lCYVFeHcTIAFcAzi&_nc_ht=scontent-frx5-2.xx&oh=69c3fd42552520f1e8afd493bda7e39b&oe=6168487D
  @width 0.2
\fi

\iusr{Александр Ермоленко}

Спасибоооо большое, дружище, то, что сотворили - бесценно и слов благодарности
не хватит, а чувств и эмоций-более чем, поэтому восхищаюсь Вами , обнимаю и
крепко жму руку настоящему мужчине и Киевлянину, искренне всегда Ваш Александр.

\iusr{Наталия Платонова}
Когда брожу по страницам "киевских историй", легче дышать. Я так далеко...

\iusr{Валентина Ненько}
Спасибо большое!  ️@igg{fbicon.heart.red}{repeat=3}  @igg{fbicon.hands.pray}{repeat=3}  @igg{fbicon.rose}{repeat=3}  @igg{fbicon.thumb.up.yellow}  @igg{fbicon.hand.victory}

\iusr{Александр Ермоленко}
До новых встреч!!!!!!

\iusr{Larisa Formagey}

Олег, огромный респект вам и вашим друзьям единомышленникам за грандиозную идею
создания группы и ее осуществление , бесконечно благодарна , что являюсь
участником стотысячного сообщества киевлян, объединённых чувством любви к
лучшему Городу  @igg{fbicon.heart.eyes} 

\iusr{Manz Ura Pochernina}

Спасибо вам большое Олег за группу , очень познавательно
@igg{fbicon.100.percent}  @igg{fbicon.hands.applause.yellow}  лично я не
родилась в Киеве , но люблю город с самого детства , мой отец работал в юности
и служил в этих местах .Он много рассказывал теплых историй о городе , об
Украине .Но по воле судьбы я ввшла замуж за украинца и мы живем в этом городе
более 10 лет .Мой муж оказался из рода графов
@igg{fbicon.hands.applause.yellow} Идем дальше , любя и изучая историю города и
тех кто в нем влюблен и называет своим этот любимый теплый город
@igg{fbicon.heart.red} ️  В этой группе узнаю о многом , утешение одно что
многие женщины приезжая в этот город когда то были невестами , потом стали
мамами и остались киевлянками , если не все  @igg{fbicon.face.grinning.sweat}
@igg{fbicon.heart.red} ️  @igg{fbicon.heart.sparkling} 

Очень благодарна вам за создание такой замечательной группы
@igg{fbicon.100.percent}  @igg{fbicon.hands.applause.yellow}
@igg{fbicon.exclamation.mark.double} ️  Респект .

\iusr{Феликс Ушеренко}

Особенно важна эта группа для киевлян, покинувших любимый город и живущих в
других странах. Ностальгия бродит в жилах, память пока не подводит.

\iusr{Наталия Привалко}

Олег! Спасибо за идею и её осуществление! Благодаря Вам и участникам группы
открываешь для себя неизвестные факты истории любимого города, сопереживаешь,
читая воспоминания и непридуманные жизненные истории. Спасибо и удачи!

\iusr{Елена Аникеева}

Я не коренная киевлянка, хотя живу в Киеве много лет. Очень люблю этот город! Его
просто не возможно не любить! Всегда с большим удовольствием читаю
истории. Спасибо за создание такой группы. Ждем новых историй. Быть добру.

\iusr{Александр Коган}
Спасибо большое что вы есть

\iusr{Лора Пономаренко}
Любимая группа! Спасибо, что Вы есть!


\end{itemize} % }
