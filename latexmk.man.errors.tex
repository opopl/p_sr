% vim: keymap=russian-jcukenwin
%%beginhead 
 
%%file man.errors
%%parent body
 
%%endhead 
\section{DEALING WITH ERRORS, PROBLEMS, ETC}
  
\vspace{0.5cm}
 {\ifDEBUG\small\LaTeX~section: \verb|man.errors| project: \verb|latexmk| rootid: \verb|p_saintrussia| \fi}
\vspace{0.5cm}
  
Some possibilities:

\begin{itemize}
  \item a. If you get a strange error, do look carefully at the output that  is
          on  the  screen  and  in  log  files.   While  there  is  much  that
                is notoriously verbose in the output of latex (and that is
                added to by latexmk),  the verbosity is there for a reason:
                to enable the user to diagnose problems.  Latexmk does repeat
                some messages at the end of a run that  it  thinks would
                otherwise be easy to miss in the middle of other output.

  \item b. Generally, remember that latexmk does its work by running other
          programs.  Your first priority in dealing with errors should be to
                examine what went wrong with the individual programs.  Then you
                need to correct the  causes of errors in the runs of these
                programs.  (Often these come from errors in the source
                document, but they could also be about  missing LaTeX
                packages, etc.)

  \item c. If latexmk doesn't run the programs the way you would like, then you
          need to look in this documentation at the list of command line
                options and  then at the sections on
                configuration/initialization files.  A lot of latexmk's
                behavior is configurable to deal  with  particular  situa-
                tions.  (But there is a lot of reading!)

The  remainder  of  these notes consists of ideas for dealing with more
difficult situations.

\item d. Further tricks can involve replacing the standard commands that
        latexmk runs by other commands or scripts.

\item e.  For possible examples of code for use in an RC file, see the
directory  \verb|example_rcfiles|  in  the  distribution  of  latexmk   (e.g.,
at
\url{http://mirror.ctan.org/support/latexmk/example_rcfiles}).
Even if these examples don't do what you want, they may provide suitable
                inspiration.

\item f. There's a useful trick that can be used when you  use  lualatex
instead  of  pdflatex  (and  in some related situations).  The problem
is that latexmk won't notice a dependency on a file, bar.baz
say, that  is input  by  the  lua code in your document instead
of by the LaTeX part.  (Thus if you change \verb|bar.baz| and
rerun latexmk, then latexmk will  think no  files have changed
and not rerun lualatex, whereas if you had
\verb|'\input{bar.baz}'| in the LaTeX part of the document,
latexmk  would  notice the change.)   One  solution is just to
put the following somewhere in the LaTeX part of the document:

\begin{verbatim}
  \typeout{(bar.baz)}
\end{verbatim}

This puts a line in the log file that latexmk will  treat  as  implying
that  the file bar.baz was read.  (At present I don't know a way of do-
ing this automatically.)  Of course, if the file has a different  name,
change bar.baz to the name of your file.

\item g. See also the section "Advanced Configuration: Some extra resources".

\item h.    Look on tex.stackexchange, i.e., at   
        \url{http://tex.stackexchange.com/questions/tagged/latexmk}  

Someone may  have  already  solved your problem.

\item i. Ask a question at tex.stackexchange.com.
Or ask me (the author of latexmk).  My e-mail is at the end of this
documentation.

\end{itemize}



