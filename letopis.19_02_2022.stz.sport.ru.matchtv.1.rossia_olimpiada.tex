% vim: keymap=russian-jcukenwin
%%beginhead 
 
%%file 19_02_2022.stz.sport.ru.matchtv.1.rossia_olimpiada
%%parent 19_02_2022
 
%%url https://matchtv.ru/olimpijskije_igry/matchtvnews_NI1540234_Nas_ne_sdvinut_Rossija_uznala_okonchatelnoje_mesto_v_obshhem_medalnom_zachete_Olimpiady
 
%%author_id 
%%date 
 
%%tags olimpiada,olimpiada.pekin.2022,rossia,sport
%%title Нас не сдвинуть. Россия узнала окончательное место в общем медальном зачете Олимпиады
 
%%endhead 
 
\subsection{Нас не сдвинуть. Россия узнала окончательное место в общем медальном зачете Олимпиады}
\label{sec:19_02_2022.stz.sport.ru.matchtv.1.rossia_olimpiada}
 
\Purl{https://matchtv.ru/olimpijskije_igry/matchtvnews_NI1540234_Nas_ne_sdvinut_Rossija_uznala_okonchatelnoje_mesto_v_obshhem_medalnom_zachete_Olimpiady}

Медальный зачёт Олимпиады в Пекине постепенно приобретает окончательный вид. Но
«Матч ТВ» все еще надеется, что завоеванное российскими спортсменами в морозном
Пекине совпадет с загаданным экспертами с теплого дивана. О том, что и как мы
загадывали, можно почитать здесь.

\ii{19_02_2022.stz.sport.ru.matchtv.1.rossia_olimpiada.pic.1}

19 февраля Александр Большунов завоевал золото в масс-старте, который внезапно
сократили с 50 до 28 км. Этот успех Александра мы спрогнозировали точно, а вот
серебро Ивана Якимушкина стало приятным сюрпризом.

В фигурном катании, как мы и предполагали, российские пары довольствовались
серебром и бронзой, в то время как победу судьи отдали китайцам. Никакого
международного скандала, которого мы так опасались, хотя и Евгения Тарасова /
Владимир Морозов, и Анастасия Мишина / Александр Галлямов тоже заслуживали
золота. Очень ровные выступления выдали все три ведущих дуэта.

Также мы надеялись на серебро Кристины Резцовой в масс-старте, но при
составлении прогноза не учли капризы китайской погоды. В итоге гонка была
перенесена на 18 февраля и, как мы уже знаем, никакой медали сборной России не
принесла.

Золото Большунова помогло сборной России подняться на восьмое место в медальном
зачете по количеству золотых наград. По сумме медалей мы по-прежнему вторые
(31) и уступаем только Норвегии (35). С этой позиции нас уже не сдвинуть.

20 февраля Олимпиада в Пекине подойдет к концу. Но до церемонии закрытия
спортсмены успеют разыграть четыре золота. В хоккее, где мы ставили на серебро
сборной России, наши парни в финале сыграют с Финляндией. Также шансы на
награды есть в лыжной гонке на 30 км у женщин. Остается надеяться, что героине
Игр Наталье Непряевой и другим нашим девочкам хватит сил на концовку
изнурительного олимпийского марафона.
