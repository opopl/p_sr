% vim: keymap=russian-jcukenwin
%%beginhead 
 
%%file 30_03_2022.stz.news.ua.pravda.2.ohtyrka.5.svolochnost
%%parent 30_03_2022.stz.news.ua.pravda.2.ohtyrka
 
%%url 
 
%%author_id 
%%date 
 
%%tags 
%%title 
 
%%endhead 
\subsubsection{Сволочность наших людей як була, так і залишилась}

Як зараз виглядає мій день?

Щодня я встаю о 5-ій ранку, а о 7-ій вже біля міськради пишу звернення до
містян – розповідаю коротенько те, що можна розповідати. Далі проводжу зі
своїми замами нараду на вулиці – міська рада в нас же розбита. Усі мої зами і
частина депутатів на роботі. 

Щоб усі розуміли – 5 депутатів з ОПЗЖ зі мною кожного дня, з першої хвилини
вони допомагають боротися з ворогом по всіх напрямках. У мене секретар з ОПЗЖ,
усі думали, що вона зрадить, з короваями тут буде... А вона лютою ненавистю
ненавидить цих москалів. Люди і партія – різні речі, як це показало. 

Також зі мною є депутати з партії \enquote{Сила людей} і \enquote{Сила і честь}, я сам з неї,
моя дружина теж.

Кожен депутат працює за своїм напрямом. Один, наприклад, зараз очолює
комунальне підприємство \enquote{Благоустрій} – це вивіз скла й таке інше. 

Ми, на відміну від інших, не прив'язуємо мародерів до дерева. Я їм сказав: або
я вас прив'яжу, і вас люди заб'ють, або я вас харчую, одягаю, а ви мені гребете
місто, відпрацьовуєте.

І от вони зранку до ночі гребуть, місто на сьогодні вже не завалене горами скла
та цегли. Нічого, десь потім скло вставимо, десь нові будинки побудуємо.

Моя дружина зараз займається гуманітарною допомогою. Вона в нас зараз проходить
через комп'ютер, бо це ресурс, хто б що не говорив, обмежений.

За день ми видаємо до 20 тонн гуманітарної допомоги. По машинах це дуже багато
– розгрузити, загрузити, поскладати. Але якщо на 20 тисяч людей, то це по
кілограму. 

Але що хочу сказати – сволочность наших людей як була, так і залишилася, війна
на них взагалі ніяким чином не вплинула. У нас є люди, які покидали своїх
рідних, а потім з-за кордону нам дзвонили і казали: у нас там бабуся в квартирі
лежить 5 днів... 

Ще іноді ворогами для місцевої влади стають самі волонтери. Бо вони вважають,
що повинні роздати все, що отримали. Ніякого контролю, ніякої звітності... 

Так у нас пропало багато бронежилетів, тепловізори, наркотики (йдеться про ліки
– УП). Фірми, з якими ми з перших днів працювали і які надсилали нам допомогу,
перехоплювалися волонтерами. Це все розгружалося і роздавалося.

А потім були ситуації, коли люди казали: у нас у холодильнику лежить 100
коробок інсуліну, з харчами передали. Ми його забирали і віддавали в лікарню.
Були моменти, коли знаходили броніки.

Коли ми вже все систематизували, то були десятки випадків, як людина дзвонила
на один номер – їй привозили допомогу, через годину вона дзвонила на інший – і
знов просила допомогу. Відкриваємо комп'ютер – то вам уже привезли! \enquote{А, я
думала, що буде ще}.

Або починають перебирати: \enquote{Чого нам не італійські макарони?!} Це найбільше
відбиває бажання допомагати.

Повертаючись до свого графіку – протягом дня в мене наради в лікарні, наради на
базах гуманітарної допомоги – у нас їх кілька, щоб у разі бомбардування не було
знищено все. 

Потім зустріч гуманітарних грузів, їхнє розвантаження, зустріч з мешканцями,
емоції, зустріч з військовими. І так з ранку до вечора. Робота, робота, робота.
Плюс 2-3 сотні дзвінків, похорони, було декілька весіль. Всяке було. 
