% vim: keymap=russian-jcukenwin
%%beginhead 
 
%%file slova.vlast
%%parent slova
 
%%url 
 
%%author 
%%author_id 
%%author_url 
 
%%tags 
%%title 
 
%%endhead 
\chapter{Власть}

Что бы ни делала нынешняя \emph{власть}, даже при самом благоприятном раскладе
видим неправильное движение в правильном направлении.  Вроде бы, на знамена
поднимаются адекватны идеи. Но они настолько через одно место реализовываются,
что вызывают когнитивный диссонанс.  Вот и в этом случае – надо уметь настолько
удобно подставиться под критику.  Это как в боксе выйти с опущенными руками и
будто на блюдечке подставить свою челюсть сопернику. Взяли интересную идею — и
разработали идиотскую ее реализацию.  Реструктуризации долгов – правильная
идея, она обсуждается во всём мире. Абсолютно не правы те, кто критикует ее.
Они или защищают интересы финансовых спекулянтов, или просто не читают ведущих
экономистов мира. А те говорят, что реструктуризация долгов для бедных стран в
условии нынешнего кризиса — фактически предопределенная задача. Кредиторы
должны будут согласиться на более мягкие или более жесткие условия
реструктуризации, умеряв свою жадность,
\citTitle{Власть вроде бы озвучивает здравые идеи - но реализовывает их по-идиотски}, Алексей Кущ, strana.ua, 09.06.2021

