% vim: keymap=russian-jcukenwin
%%beginhead 
 
%%file 29_09_2018.stz.news.ua.mrpl_city.1.k_240_letiu_mariupolja_inzhener_vladimir_miheev
%%parent 29_09_2018
 
%%url https://mrpl.city/blogs/view/k-240-letiyu-mariupolya-inzhener-vladimir-miheev
 
%%author_id burov_sergij.mariupol,news.ua.mrpl_city
%%date 
 
%%tags 
%%title К 240-летию Мариуполя: Инженер Владимир Михеев
 
%%endhead 
 
\subsection{К 240-летию Мариуполя: Инженер Владимир Михеев}
\label{sec:29_09_2018.stz.news.ua.mrpl_city.1.k_240_letiu_mariupolja_inzhener_vladimir_miheev}
 
\Purl{https://mrpl.city/blogs/view/k-240-letiyu-mariupolya-inzhener-vladimir-miheev}
\ifcmt
 author_begin
   author_id burov_sergij.mariupol,news.ua.mrpl_city
 author_end
\fi

На Тяжмаше, - так для краткости называли одно из крупнейших предприятий страны,
которое поменяло несколько названий, то Ждановский завод тяжелого
машиностроения, то ПО \enquote{Ждановтяжмаш}, то концерн \enquote{Азовмаш} и так далее,
работало немало ярких личностей – и рабочих, и инженеров, и руководителей. Один
из них – \textbf{Владимир Авдеевич Михеев}.

Он родился 2 марта 1930 года в Мариуполе в поселке Садки. Ему довелось испытать
все невзгоды немецко-фашистской оккупации. После средней школы поступил в
Ждановский металлургический институт. Учился легко, учился охотно, старался как
можно глубже познать каждый изучаемый предмет. По окончании вуза в 1954 году
был направлен на завод имени Ильича в специальное конструкторское бюро, которое
занималось разработкой технической документации для производства
железнодорожных цистерн. Позже на базе этого подразделения было создано бюро по
конструированию заправочных устройств для авиационных и ракетных систем. Стоит
отметить, что работы этого бюро были строго секретными. Знания, трудолюбие и
ответственность молодого специалиста Владимира Михеева были оценены. Он был
назначен заместителем главного конструктора Юрия Николаевича Дзюман-Грека,
выпускника Ростовского машиностроительного института, на пару лет старше своего
нового заместителя.

\ii{29_09_2018.stz.news.ua.mrpl_city.1.k_240_letiu_mariupolja_inzhener_vladimir_miheev.pic.1}

На протяжении нескольких лет Владимиру Михееву приходилось отправляться в
длительные командировки. Пунктом назначения была малоизвестная крохотная
станция Тюра Там. Непосвященные не догадывались, что молодой инженер ездит на
стартовую площадку, где выполняет наладку, регулировку устройств для заправки
космических ракет специальным топливом, где накапливает сведения о неполадках,
там же находит технические решения. Жизнь и события на космодроме Байконур, а
именно туда ездил в командировку, Владимир Авдеевич очень ярко описал в своей
повести \textbf{\enquote{Байконур. Ключ на старт}}. Кстати, в этой повести он проявил
незаурядные писательские способности.

12 апреля 1961 года. На всех радио- и телевизионных каналах мира прозвучало:
\emph{\enquote{Юрий Гагарин в Космосе!}} Через некоторое время в газетах был опубликован
список награжденных государственными наградами, среди них и инженер Ждановского
завода тяжелого машиностроения Михеев В. А. Он был удостоен ордена Ленина. На
заводе перешептывались: \emph{\enquote{За что?}} И только очень узкий круг знал – за активное
участие в подготовке полета человека в Космос. Владимир Авдеевич надеялся, что
разработка новых типов топливозаправщиков будет его делом на всю жизнь. Но, как
говорят: \emph{\enquote{Человек предполагает, а Господь располагает}}. Его инженерная
деятельность сделала крутой вираж. 

\textbf{Читайте также:} 

\href{https://archive.org/details/22_09_2018.sergij_burov.mrpl_city.ko_dnju_mashinostroenia_vladimir_karpov}{%
Ко Дню машиностроителя: Владимир Карпов, Сергей Буров, mrpl.city, 22.09.2018}

Приказом от 1 июля 1969 г. Министерство тяжелого машиностроения СССР передало
производство кранов большой грузоподъемности Ждановскому заводу тяжелого
машиностроения. Этим же приказом было пре\hyp{}дусмотрено создание на предприятии
конструкторского отдела подъемно-транспортного машиностроения. Внутризаводским
приказом главным конструктором нового отдела был назначен инженер Владимир
Авдеевич Михеев.

\underline{Эпизод из воспоминаний Владимира Авдеевича:}

\begin{quote}
\em
Звонит как-то Ефросинья Ильинична, секретарь главного инженера Довженко, и
говорит:

- Зайдите к директору.

Уже это было необычным: в приемной директора работало три секретарши. Почему не
они позвонили?

Прихожу. Карпов за столом в позе египетского сфинкса. Любимая его поза. Глаза
добрые, лицо приятное. Подает мне телеграмму:

- Читай, коллега.

Телеграмма министерская. В ней предлагалось заводу создать два доковых крана
для румынского судостроительного завода в Констанце. Указана общая
характеристика.

Я прочитал телеграмму, Карпов спрашивает:

- Ну что, коллега?

Говорю:

- Сделаем.

Он кивнул головой, мол, хорошо:

- Готовь ответ.

И все, больше ни слова.
\end{quote}

\ii{29_09_2018.stz.news.ua.mrpl_city.1.k_240_letiu_mariupolja_inzhener_vladimir_miheev.pic.2}

\textbf{Читайте также:} 

\href{https://mrpl.city/news/view/mariupolskie-shkolniki-pozdravili-gorod-s-240-letiem-zazhigatelnoj-pesnej-video}{%
Мариупольские школьники поздравили город с 240-летием зажигательной песней, Ігор Романов, mrpl.city, 27.09.2018}

Михеев в решении технических задач стремился найти и находил не\hyp{}стандартные
решения. \textbf{Александр Стефанович Гонтарев, заместитель главного конструктора КО
ГКО} рассказывал:

\begin{quote}
\em
- Значительная часть кранов, изготовленных на нашем заводе, предназначалась для
морских и речных портов. Узлы крана отправили по железной дороге в порт
назначения. И уже там кран собирали, испытывали и т.д. Как-то Владимир Авдеевич
говорит: Лучше краны морским путем отправлять. На \enquote{Азовмаше} мы производим
крупные узлы, их отправляем автомобильным транспортом на берег моря. В
достроечном цехе собирают полностью кран. Приезжает заказчик, мы ему
показываем, какие функции он выполняет. И этот кран грузим на понтонный мост. И
он морем или рекой отправляется заказчику.

И вот на судоремонтном заводе выделили нам площадь, и на этой площади полностью
собрали кран. Ну, конечно, пригласили всех начальников отделов механизации всех
речных портов нашей Украины, из России приехали. Показали функции нашей машины,
и эта машина морем пошла в Одессу. И через две недели кран начали
эксплуатировать. А по старой технологии кран бы собирали три-четыре месяца...
	
\end{quote}

В конце 80-х годов Владимир Авдеевич был назначен директором одного из
институтов при объединении \enquote{Ждановтяжмаш}. На этой должности он поработал
сравнительно недолго. Будучи в Москве в командировке, он перенес приступ
тяжелейшей болезни сердца. Его подлечили. И после этого он перешел на более
спокойную работу в Приазовский государственный технический университет. Там он
занимался наряду с преподавательской деятельностью также и исследовательской и
изобретательской работой. Владимир Авдеевич Михеев был избран академиком
Подъемно-транспортной академии наук Украины, доцентом кафедры
подъемно-тран\hyp{}спортных машин и деталей машин Приазовского государственного
технического университета. Основным направлением его научной работы было
исследование кинематики и прочности портальных и металлургических кранов.

\textbf{Читайте также:} 

\href{https://mrpl.city/news/view/mariupolskij-vuz-primet-uchastie-v-sozdanii-unikalnogo-proekta-quotukrzaliznytsi-quot}{%
Мариупольский вуз примет участие в создании уникального проекта \enquote{Укрзализныци}, Олена Онєгіна, mrpl.city, 04.09.2018}

\textbf{Александр Стефанович Гонтарев, заместитель главного конструктора КО ГКО}:

\begin{quote}
\em	
- Дипломники были у него. Он как-то позвонил мне и говорит: \enquote{Надо дать темы, но
такие, Саша, чтобы они были для вас полезными, чтобы студент ее решил, и она
для вас какое-то полезное зерно оставила}.

Говорят, что если человек талантлив в своем деле, то он одарен и во всем.
Удачным примером этого утверждения является Владимир Авдеевич. Одним из его
увлечений были шахматы. Он мог сесть за пианино и сыграть какую-нибудь пьесу на
заказ. Владимир Авдеевич не порывал связи со своими однокашниками по институту.
Часто был инициатором встреч с ними. Среди других увлечений у него был туризм.
\end{quote}

\textbf{Иван Антонович Петренко, ведущий инженер:} 

\begin{quote}
\em - У Владимира Авдеевича все пережитое, сделанное, обдуманное изложено в его
книгах, в прозе и стихах. Да, он писал стихи, хорошие стихи. А поэтам дан такой
дар Божий - глобальные мысли облекать в краткие формулы в своих строчках
стихов.	
\end{quote}

\textbf{Петр Георгиевич Голубков, зам. главного конструктора:}

\begin{quote}
\em - Непосредственно получилось с ним познакомиться поближе в 2006 году. Когда
возникла идея проводить встречи бывших коллег в День космонавтики. А Владимир
Авдеевич послужил именно тем цементирующим, так сказать, элементом, который
привлек туда к участию как тех, кто еще продолжал работать, так и тех, которые
по воле судьбы ушли либо в другие отделы, либо на другие предприятия. И вот эти
встречи наши стали традиционными. 	
\end{quote}

14 апреля 2016 года азовмашевцы, чья деятельность была связана с
конструированием оборудования для освоения космоса, собрались вместе, чтобы
отметить День космонавтики. Инициатором этой встречи, как всегда, был Михеев. А
20 мая эти же люди пришли попрощаться с Владимиром Авдеевичем и проводить его в
последний путь...
