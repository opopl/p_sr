%%beginhead 
 
%%file 14_06_2023.fb.hazhilenko_ksenia.kyiv.1.donatyty_na_porjatunok_tvaryn_vidbyraty_v_zsu
%%parent 14_06_2023
 
%%url https://www.facebook.com/100006566076129/posts/pfbid0PrYwQTA4V5eEjrV7KYdagBAq9n3WiWcTwnorLfwpFx35vsu5QNTbcTPFtH5AVg5Vl
 
%%author_id hazhilenko_ksenia.kyiv
%%date 14_06_2023
 
%%tags 
%%title Ні, я не поділяю отого поширеного днями ствердження "донатити на порятунок тварин - відбирати в ЗСУ"
 
%%endhead 

\subsection{Ні, я не поділяю отого поширеного днями ствердження \enquote{донатити на порятунок тварин - відбирати в ЗСУ}}
\label{sec:14_06_2023.fb.hazhilenko_ksenia.kyiv.1.donatyty_na_porjatunok_tvaryn_vidbyraty_v_zsu}

\Purl{https://www.facebook.com/100006566076129/posts/pfbid0PrYwQTA4V5eEjrV7KYdagBAq9n3WiWcTwnorLfwpFx35vsu5QNTbcTPFtH5AVg5Vl}
\ifcmt
 author_begin
   author_id hazhilenko_ksenia.kyiv
 author_end
\fi

Ні, я не поділяю отого поширеного днями ствердження «донатити на порятунок
тварин - відбирати в ЗСУ». Розумію його та поважаю погляд автора та його
прибічників, проте - не поділяю. 

Як і більшість людей мого кола, я доначу зараз більшу частку доходів. 

На залізних пташок та коней, тепловізори, такмед, ремонт, бензин, знов такмед і
знов коней.

Але і на Херсон 

І на допомогу тим, хто лишився всього в інших регіонах 

І на тварин також.

Більше того, я їм не лише хліб та воду, як цілком можна було, виходячи з логіки
\enquote{недодавання армії}.

І до роботи дістаюся не ровером, а автівкою, котра любить недешеве пальне (до
речі, частина доходів з останнього також іде на залізних птахів). 

І -ОМГ!- я навіть на манікюр вже регулярно ходжу 

Я більше року боролася з клятим почуттям провини щодо витрати кожної копійки,
яка йшла не на ЗСУ. 

І вилікуватися від нього допомогло розуміння перспектив тривалої війни та
шаленого бажання дожити до Перемоги, не лише фізично, але і психічно
неушкодженою

І щоб досягти цієї мети, дбати потрібно не лише про ЗСУ. 

І не лише про себе. 

На плечі українців зараз, крім іншого звалився тягар латання всіх можливих і
неможливих дір

Все що змарнували, розкрали, пропили, закопали в землю представники влади, яку
30 років обирали ми (в сенсі, всі українці; я ніколи отеє падло не обирала,
проте колективна відповідальність сліпа та безжальна), ми тепер намагаємося
закрити  своїми тілами, душами, гаманцями 

І в цьому латанні паростки нашого майбутнього

В тому, щоб допомогати нужденним

Щоб підтримувати тих, кому важче

Щоб дбати про тварин та рослини

І, звісно, донатити-донатити-донатити на ЗСУ

А ще дорогою додому купити полуниць в бабці, яка половину з проданого
відправить на Армію дронів. А ще копійчину - на херсонських тварин

І можливо я помиляюся, але мені видається: коли по Перемозі наші військові
повернуться додому, їм важливо буде побачити життя, з його садами, будинками,
малечею, квітами та тваринами

Життя, а не пустелю

%\ii{14_06_2023.fb.hazhilenko_ksenia.kyiv.1.donatyty_na_porjatunok_tvaryn_vidbyraty_v_zsu.cmt}
