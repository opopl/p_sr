% vim: keymap=russian-jcukenwin
%%beginhead 
 
%%file 04_02_2022.stz.news.lnr.lug_info.2.tekvondo_sport_argentina
%%parent 04_02_2022
 
%%url https://lug-info.com/news/thekvondisty-iz-lnr-zavoevali-zolotye-medali-na-onlajn-turnire-organizovannom-v-argentine
 
%%author_id news.lnr.lug_info
%%date 
 
%%tags argentina,lnr,sorevnovanie,sport,tekvondo,turnir
%%title Тхэквондисты из ЛНР завоевали золотые медали на онлайн-турнире, организованном в Аргентине
 
%%endhead 
 
\subsection{Тхэквондисты из ЛНР завоевали золотые медали на онлайн-турнире, организованном в Аргентине}
\label{sec:04_02_2022.stz.news.lnr.lug_info.2.tekvondo_sport_argentina}
 
\Purl{https://lug-info.com/news/thekvondisty-iz-lnr-zavoevali-zolotye-medali-na-onlajn-turnire-organizovannom-v-argentine}
\ifcmt
 author_begin
   author_id news.lnr.lug_info
 author_end
\fi

Спортсмены Республиканской федерации тхэквондо ИТФ завоевали золотые медали на
онлайн-турнире \enquote{Кубок мира}, организованном в Аргентине. Об этом сообщила
пресс-служба Министерства культуры, спорта и молодежи ЛНР со ссылкой на
федерацию.

\ii{04_02_2022.stz.news.lnr.lug_info.2.tekvondo_sport_argentina.pic.1}

\enquote{С 1 декабря 2021 года по 30 января 2022 года на онлайн-платформе Международной
федерации тхэквондо ITF прошел первый в истории этого вида спорта онлайн-турнир
\enquote{Кубок мира} по формальным комплексам (туль), организованный в Аргентине. В
соревнованиях приняли участие спортсмены Республиканской федерации тхэквондо
ИТФ: воспитанница инструктора Дмитрия Вакуленко, спортсменка Республиканского
физкультурно-спортивного общества \enquote{Динамо} Милана Колчанова и воспитанник
инструктора Ярослава Миронова Кирилл Бессарабов}, – говорится в сообщении.

\ii{04_02_2022.stz.news.lnr.lug_info.2.tekvondo_sport_argentina.pic.2}

Турнир собрал 688 спортсменов из Перу, Аргентины, Российской Федерации,
Парагвая, Гондураса, Австрии, Мексики, Франции, Венесуэлы, Украины, Эстонии,
США, Колумбии, Боливии, Бразилии, Доминиканской Республики, Канады, Эквадора,
Чили, Испании, Пуэрто-Рико, Никарагуа, Гватемалы, Сингапура, Монголии, Непала,
Ямайки, Великобритании и Казахстана.

\enquote{В связи с масштабностью турнира результаты анализировались в течение двух
месяцев. Спортсмены могли наблюдать свои выступления в онлайн-режиме, где судьи
оценивали формальный комплекс каждого тхэквондиста. Наших ребят оценивали судьи
из Ямайки, Перу, Аргентины и Парагвая, которые обращали внимание на технику,
силу удара, равновесие, контроль дыхания и ритм. По итогам онлайн-состязаний
луганские спортсмены стали чемпионами в своих возрастных категориях}, –
рассказали в федерации.

Президент спортивной организации Анна Евсюкова отметила, что Колчанова и
Бессарабов \enquote{регулярно представляют Луганскую Народную Республику на российских
и международных турнирах по тхэквондо ИТФ}.

\enquote{Мы гордимся своими юными спортсменами, поздравляем их с победой, желаем не
останавливаться на достигнутом, уверенно завоевывать новые высоты и поднимать
авторитет такого боевого вида спорта, как тхэквондо ИТФ}, – сказала она и
поблагодарила руководство Международной федерации тхэквондо и Панамериканский
совет тхэквондо ИТФ \enquote{за организацию соревнований такого уровня}.
