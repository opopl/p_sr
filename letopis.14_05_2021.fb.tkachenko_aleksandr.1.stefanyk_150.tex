% vim: keymap=russian-jcukenwin
%%beginhead 
 
%%file 14_05_2021.fb.tkachenko_aleksandr.1.stefanyk_150
%%parent 14_05_2021
 
%%url https://www.facebook.com/oleksandr.tkachenko.ua/posts/4106495059418065
 
%%author 
%%author_id 
%%author_url 
 
%%tags 
%%title 
 
%%endhead 

\subsection{Василю Семеновичу, з Днем народження! Ваше 150-річчя – подія для всієї країни}
\Purl{https://www.facebook.com/oleksandr.tkachenko.ua/posts/4106495059418065}

Василю Семеновичу, з Днем народження! Ваше 150-річчя – подія для всієї країни

Геній експресіоністичної новели, який своїм потужним і влучним словом зачіпає
за живе кожного читача, – саме так можна охарактеризувати Василя Стефаника.
Його тексти посіли важливе місце серед основоположних в українській літературі,
хоч неодноразово зазнавали спроб редагування за радянського періоду.

Ця дата – нагадування для усіх нас із закликом знову прочитати праці
письменника, переосмисливши їх. Відому та новаторську «Синю книжечку» –
проривну збірку для письменника, або ж новелу «Камінний хрест», яка стала й
натхненням для однойменного кіно. 

А ще глибше зануритись у творчість Василя Семеновича допоможе мистецький
проект, організований \verb|#МКІП| спільно з Державним агентством з питань мистецтв і
мистецької освіти. Презентація відбудеться сьогодні. Відкриття о 20.00 в Києві,
на сцені літнього театру «Мушля». Паралельно, все транслюватиметься онлайн. 

\ifcmt
  pic https://scontent-frt3-1.xx.fbcdn.net/v/t1.6435-9/186482565_4106477496086488_2460373668314581523_n.jpg?_nc_cat=108&ccb=1-3&_nc_sid=8bfeb9&_nc_ohc=-qAHqL5bOGsAX8nsaWo&_nc_ht=scontent-frt3-1.xx&oh=80f51e244c95f228aba9a768d0dc9bdf&oe=60C549C9
  width 0.5
  %fig_env wrapfigure
\fi

Для глядачів і відвідувачів заплановані:

\begin{itemize}
  
\item Виставка під відкритим небом «СтефаникOnline», з хронологічним поглядом на життя та літературний спадок письменника через історичний контекст зламу століть.

\item Театральний перформанс «Найліпше з мойого серця», де новели та листи
        письменника було покладено в основу постановки.
\end{itemize}

Тож раджу не пропустити заново познайомитись з майстром слова. Без радянської цензури і штампів.

Державне агентство України з питань мистецтв та мистецької освіти

Василь Стефаник: мистецький проєкт до 150-річчя
