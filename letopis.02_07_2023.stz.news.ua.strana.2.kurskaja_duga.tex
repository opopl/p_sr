% vim: keymap=russian-jcukenwin
%%beginhead 
 
%%file 02_07_2023.stz.news.ua.strana.2.kurskaja_duga
%%parent 02_07_2023
 
%%url https://strana.news/articles/reconstruction/438554-kurskaja-duha-kohda-i-kak-proizoshel-korennoj-perelom-vo-vtoroj-mirovoj-vojne.html
 
%%author_id news.ua.strana
%%date 
 
%%tags 
%%title Огненная дуга. Как под Курском 80 лет назад произошел перелом во Второй мировой войне
 
%%endhead 
 
\subsection{Огненная дуга. Как под Курском 80 лет назад произошел перелом во Второй мировой войне}
\label{sec:02_07_2023.stz.news.ua.strana.2.kurskaja_duga}
 
\Purl{https://strana.news/articles/reconstruction/438554-kurskaja-duha-kohda-i-kak-proizoshel-korennoj-perelom-vo-vtoroj-mirovoj-vojne.html}
\ifcmt
 author_begin
   author_id news.ua.strana
 author_end
\fi

\ifcmt
  ig https://gcdnb.pbrd.co/images/BCkrBMD0uxJp.png?o=1
  @wrap center
  @width 0.9
\fi

В начале 1970-х в Советском Союзе закончили снимать грандиозную киноэпопею
\enquote{Освобождение}. Вопреки общепринятой версии, согласно которой освобождение СССР
началось под Москвой и Сталинградом, сериал стартовал с фильма \enquote{Огненная дуга},
посвященного Курской битве – сражению, началу которого в эти дни исполняется 80
лет.

Почему режиссер Юрий Озеров начал с него? Это был отголосок длительных споров о
том, что же стало коренным переломом во Второй мировой войне. Десятилетиями
после этой войны военные и историки спорили о том, когда же именно произошел
ключевой момент – на эту номинацию выдвигали и Сталинград, и Эль-Аламейн, и
Курскую дугу, и даже высадку союзников в Нормандии.

\ii{02_07_2023.stz.news.ua.strana.2.kurskaja_duga.pic.1}

Какой из ответов правильный? Чтобы выбрать один вариант, нужно дать определение
– что же такое \enquote{коренной перелом в войне}. В принципе, долго ломать голову не
нужно, потому что оно давно существует – это перелом в ходе войны, в ходе
которого инициатива необратимо переходит к противоположной стороне.

Другими словами, это момент, начиная с которого сторона, которая оборонялась и
контратаковала, переходит к фазе планируемых последовательных наступлений и
вынуждает противника, в свою очередь, перейти к обороне, изредка превращающейся
в контрнаступление.

Исходя из общепринятой формулировки, Эль-Аламейн на победу в номинации явно не
тянет: сражения в Африке во время войны не считали главными ни в Берлине, ни в
Москве, ни в Вашингтоне с Лондоном. Став решающей для событий на Черном
континенте, египетская битва осени 1942 года никак не повлияла на главный,
европейский фронт войны.

Отпадает и высадка союзников в Нормандии – просто потому, что к лету 1944 года
инициатива на всех фронтах принадлежала антигитлеровской коалиции, и вопрос
стоял лишь о сроках завершения войны.

Из двух оставшихся вариантов гораздо чаще коренным переломом считают
Сталинградскую битву, и это можно считать правильным ответом, но лишь
наполовину: сражение на Волге стало моментом политического перелома в войне –
именно после него стало ясно, что Германия победить не сможет. Но между \enquote{не
победит} и \enquote{проиграет} есть существенная разница.

Это и есть разница между последствиями Сталинграда и Курской дуги. Именно
последняя стала в военном плане переломной для всего хода Второй мировой.

\subsubsection{Харьков: освобождение и потеря}

Эффект битвы на Волге был потрясающим. Зимой 1942-43 годов практически
повторилось в зеркальном отражении то, что произошло предшествующим летом,
когда немцы после уничтожения Харьковского котла прорвали фронт Красной армии и
за месяц дошли от границ Украины до Сталинграда. Теперь же советские войска
прорвали германский фронт и в течение одного месяца вернулись от Волги к
Северскому Донцу.

И на Северском Донце все могло не закончиться – начатое возле Сталинграда
контрнаступление имело шанс обернуться для Третьего Рейха глобальной военной
катастрофой, которая действительно означала бы коренной перелом в войне. Но
этого не произошло.

Любое – даже самое успешное – наступление имеет свои пределы, устанавливаемые
как собственными возможностями, так и возможностями противника. Что произошло
после образования Сталинградского котла? Красная Армия вышла на оперативный
простор и на большей части линии фронта погнала немцев перед собой, не давая им
опомниться и организовать сплошную полосу обороны.

И в Москве, и в Берлине многие считали, что волна советского наступления может
докатиться вплоть до Днепра. Но ни советскому, ни немецкому командованию такой
вариант не нравился.

Сталин опасался, что на Днепре немцы организуют такую же непроходимую оборону,
какую его армии в 1942 году организовали на Волге, и потому требовал продолжать
наступление \enquote{на спине} у противника, чтобы попытаться создать плацдарм на
днепровском правобережье.

В свою очередь, Гитлер считал недопустимым потерю Донбасса с его углем, а
потому в середине февраля лично прибыл в Запорожье, чтобы оттуда руководить
восстановлением обороны Вермахта. К тому моменту волна советского наступления
захлестнула Курск, Белгород, Харьков и Ворошиловград (Луганск), причем Красная
армия продолжала идти от Харькова быстрыми темпами к Днепру, и во второй
половине февраля оказалась настолько близко к Запорожью, что создалась даже
угроза пленения самого фюрера.

Впрочем, со стратегической точки зрения для советского командования более
важным был другой город на Днепре – Днепропетровск: именно через него шли все
транспортные артерии, снабжавшие немецкие войска на Донбассе.

Сумей Красная армия в ходе \enquote{послесталинградского} наступления захватить этот
город, и Вермахт во всей восточной Украине ждала бы катастрофа. В Москве это
понимали, и основные силы частей Юго-Западного фронта генерала армии Николая
Ватутина направили именно в сторону Днепропетровска.

Однако, как уже сказано, наступление имеет свои пределы. Со стороны наступающих
оно выражается, во-первых, в том, что тыловые части отстают от рвущихся вперед
\enquote{передовиков}, создавая проблемы со снабжением и ремонтом, во-вторых – в том,
что различные подразделения наступают с разной скоростью: кому-то противник
достался сложнее, у кого-то просто командир талантливее, – и в результате
создаются разрывы в линии наступления, когда у лучших, ушедших вперед, на
флангах не оказывается прикрытия. Именно в такой ситуации оказались войска
советского Юго-Западного фронта, рвавшиеся к Днепропетровску.

У немцев, конечно, поначалу ситуация складывалась не лучше: находясь под
давлением Красной армии, они не успевали организовать оборону левобережной
части Днепропетровска и Запорожья. Однако к концу февраля в этот район все же
пришли резервы – прежде всего танковые дивизии СС, – а во главе Вермахта на
этом участке фронта стал самый способный немецкий военачальник Второй мировой
войны Эрих фон Манштейн.

\ii{02_07_2023.stz.news.ua.strana.2.kurskaja_duga.pic.2}

Склонный к риску вплоть до авантюрного, Манштейн вместо обороны организовал
контрнаступление во фланг советским войскам, наступающим на Днепропетровск, и
неожиданно уже войскам Юго-Западного фронта пришлось фактически выбираться из
котла.

Манштейн тем временем начал контрнаступление на Харьков с целью взять его в
кольцо. Воронежский фронт генерал-полковника Филиппа Голикова, находившийся по
правую руку от Юго-Западного, попытался в спешном порядке организовать оборону,
но теперь уже у них не оказалось времени для этого. Город, освобожденный 16
февраля, спустя месяц пришлось сдать.

К концу марта линия фронта все-таки стабилизировалась по Северскому Донцу. И
из-за потери Харькова (а с ним и Белгорода) севернее него образовался Курский
выступ, который позже вошел в историю как Курская дуга, хотя на дугу он
совершенно не походил.

Куда больше этот выступающий в немецкую сторону участок напоминал
Барвенковско-Лозовской выступ весны 1942 года, существование которого
обернулось Харьковской катастрофой и последующим отступлением до Сталинграда.
Эта аналогия преследовала обе стороны в течение всей первой половины 1943-го,
предопределив поведение и Вермахта, и Красной армии.

\subsubsection{Загадка \enquote{Вертера}}

Еще лет десять назад, когда слово война ассоциировалась для нас исключительно с
медалями дедов, мы имели лишь теоретические знания о прошедшей войне и не могли
представить, что посреди войны вдруг мог оказаться период, когда в течение трех
с половиной месяцев на большей части фронта практически ничего не происходило.
Между тем, именно так и было с конца марта до начала июля 1943 года.

Конечно, для полковника Леонида Брежнева именно этот период остался в памяти
как героическое удержание Малой земли под Новороссийском, но для остальных
фронтов в сводках Совинформбюро все месяцы оперативной паузы звучала одна и та
же фраза: \enquote{На фронтах существенных изменений не произошло}.

Страсти в эти дни кипели в штабных кабинетах, где определяли план боевых
действий на весенне-летнюю кампанию. Первым свое слово уже в середине марта
сказал Манштейн. Однако его план оказался настолько экстравагантным, что его не
принял никто: фельдмаршал предлагал заманить советские войска в ловушку, отдав
им Донбасс, и после этого ударить из района Харькова в направлении Таганрога,
таким образом образовав новый котел.

Интересно, что Манштейн фактически предлагал \enquote{отзеркалить} провалившийся план
советского командования зимы 1942 года, когда маршал Семен Тимошенко хотел
отрезать немецкие войска на Донбассе ударом из Харьковской области в
направлении Мариуполя.

Манштейн обещал фюреру успех, но тот не представлял себе, что может лишиться
Донбасса даже временно, – как по экономическим, так и по внешнеполитическим
причинам: среди союзников Германии уже началось брожение, и потеря восточной
Украины вслед за потерей Сталинграда и Северного Кавказа могла привести их
(особенно Рим) к неприятным для Гитлера решениям.

Дуче Италии Бенито Муссолини как раз весной 1943-го начал уговаривать Гитлера
заключить перемирие с Советским Союзом, чтобы вплотную заняться Средиземным
морем, и для его убеждения необходим был новый захват территорий, а не новые
потери, пусть даже временные.

Поэтому фюрер ухватился за предложение Генштаба, которое бросалось в глаза при
первом же взгляде на карту: срезать Курский выступ сходящимися ударами с юга,
со стороны Харькова и Белгорода, и с севера, со стороны Орла, после чего
развить наступление в сторону Москвы.

\ii{02_07_2023.stz.news.ua.strana.2.kurskaja_duga.pic.3}

Этот план уже в конце марта стал известен Сталину от источника в британской
разведке Джона Кэрнкросса. Также существует легенда, что советский вождь
получил от агента в Берлине уже план операции \enquote{Цитадель} сразу после
утверждения его Генштабом Вермахта. И до сих пор неизвестно, кто же этот агент
\enquote{Вертер}, передавший советской разведке всю информацию о \enquote{Цитадели}, а также
чертежи новейшего немецкого танка \enquote{Пантера}.

Однако дело не только в разведке. Еще 8 апреля (то есть за неделю до
утверждения операции фюрером) замглавкома Георгий Жуков представил Сталину свои
соображения относительно планов Германии на весенне-летнюю кампанию, и там уже
говорилось о том, что Вермахт не готов наступать на широком фронте, а потому
постарается срезать Курский выступ, после чего попробует развить успех либо в
направлении Москвы, либо снова повернет на юго-восток, в сторону Сталинграда
или Кавказа. Такой вывод был вполне очевидным: все видели перед собой карту и
все помнили весну 1942-го.

Фельдмаршал Манштейн тоже видел карту и понимал, что ее видят в Москве, а
потому, когда его предложение отвергли, поддержал проект операции \enquote{Цитадель},
но с одной оговоркой: провести ее нужно как можно быстрее, пусть даже толком не
подготовившись, но и не дав Красной армии организовать серьезную линию обороны
в выступе. Однако и Генштаб, и Гитлер выступили против: военные – потому что не
представляли себе войну в апрельской грязи, фюрер – потому что ждал поступления
в войска новой техники, на которую возлагал большие надежды, – танков \enquote{Тигр} и
\enquote{Пантера}, а также самоходной артиллерийской установки \enquote{Фердинанд}.

Это ожидание сыграло с немецким канцлером роковую шутку. 15 апреля он утвердил
план операции \enquote{Цитадель}, назначив первый срок ее начала – 3 мая. Первый –
потому что Генштаб, которому постоянно чего-то не хватало, постоянно просил о
переносе, а Гитлер, надеясь дождаться как можно больше \enquote{Тигров}, \enquote{Пантер} и
\enquote{Фердинандов}, в этот раз легко соглашался с военными. Так что уже в конце
апреля определилась новая дата – 12 июня, – которая тоже оказалась не
окончательной.

\subsubsection{Мог ли Гитлер не наступать?}

В Москве на самом деле видели карту и помнили о весне 1942 года. Да и едва не
захлопнувшийся уже весной 1943-го новый \enquote{харьковский котел} стоял перед глазами
свежим воспоминанием. Так что советское командование, как говорится, дуло на
воду, действуя по принципу \enquote{не навреди}.

Именно по этой причине, кстати, план \enquote{донбасской ловушки} Манштейна мог не
осуществиться, даже если бы Гитлер его утвердил: велика вероятность, что
Красная армия осталась бы на месте и не пошла бы освобождать Донбасс, несмотря
на отступление немцев.

И именно по этой причине советское командование весной 1943 года приняло
решение отказаться от собственного наступления, что как раз и является
доказательством тезиса о том, что после Сталинградской битвы стратегическая
инициатива не перешла в руки командования РККА, а значит – коренной перелом в
войне еще не произошел.

Даже получив план операции \enquote{Цитадель} и его подтверждение, Сталин отказался от
отдельных предложений нанести упреждающий удар по немецким войскам, находящимся
в районе Орла. Он отдал лишь две команды: вывести основную массу войск,
находящуюся в Курском выступе, и разместить восточнее Курска (чтобы при самом
худшем варианте избежать котла), и начать укрепление основания \enquote{дуги} на юге и
на севере (чтобы этот котел не допустить).

Немцы достаточно быстро заметили оборонительные работы советских войск: уже 27
апреля командующий 9 армией (находилась к северу от Курского выступа) генерал
Вальтер Модель доложил Гитлеру об этом, и через двое суток немецкий
главнокомандующий отдал приказ приостановить подготовку к наступлению.

Еще спустя три дня в сводке разведотдела генштаба сухопутных войск Германии
появилось сообщение: \enquote{Не исключено, что противник разгадал подготовку немецкого
наступления}. Его доложили фюреру, и на мюнхенском совещании 3-4 мая
развернулась дискуссия по глобальной дилемме – нужно вообще наступать или нет.
За отказ от наступления выступал, в частности, Модель, предлагавший дождаться
советского наступления и, измотав противника, перейти к собственному
контрнаступлению. А Гудериан, который был вхож к Гитлеру, просто считал
операцию ненужной: \enquote{Да где это Курск?! Кто в мире догадывается о его
существовании?!}.

По сути, противники наступления предлагали сделать то, что уже запланировало
советское командование для себя. А значит, могла возникнуть патовая ситуация,
когда летом 1943 никто так и не решился бы наступать. Однако к тому времени уже
сложился стереотип относительно того, что немцы могут наступать только летом, а
Красная армия – только зимой, и Гитлер считал, что пауза окажется на руку
Сталину. К тому же, как уже было сказано, ему требовался показательный успех –
иначе Муссолини еще раз предложил бы закончить войну на востоке (там ведь все
равно не воюют!) и заняться Средиземным морем, где над итальянским сапогом уже
нависала англо-американская угроза. 

Тем более, Гитлер верил, что его \enquote{Тигры} прогрызут даже самую крепкую оборону.
Требовалось дождаться одного – чтобы этих \enquote{зверей} оказалось как можно больше.
Поэтому фюрер и отдал приказ перенести начало наступления на 12 июня, но потом
перенес и ее, ожидая поставок танков.

Наконец, 1 июля Гитлер объявил окончательную дату – пятое число, и уже на
следующий день Москва предупредила командующих фронтов: наступления противника
следует ожидать 3-6 июля. В ночь на 5-е советские разведчики захватили саперов,
снимавший минные заграждения, и те рассказали, что время начала операции
\enquote{Цитадель} – три часа ночи.

\subsubsection{Советский сюрприз и южный маневр Манштейна}

Немецкое наступление началось с сюрприза от Красной армии: 5 июля в 2.20 ночи
советская артиллерия нанесла массированный артиллерийский удар по позициям
Вермахта. Он имел огромное психологическое значение, показав немцам, что
советские войска знают об их планах и готовы к обороне. Однако, по словам
Жукова, реальное значение удара преувеличено: в 2.20 германские части только
готовились к выдвижению, а потому более эффективным удар оказался бы, будь он
проведен на 40 минут позже.

Тем не менее немецким командирам пришлось сдвинуть время начала операции: они
провели свою артподготовку в шесть утра, после чего перешли в наступление. Для
участия в операции \enquote{Цитадель} были выделены 34 танковые и моторизированные
дивизии, 16 пехотных дивизий, четыре танковые бригады и ряд более мелких
подразделений – всего, по разным оценкам, 800-900 тысяч человек и 2772 танка и
САУ (включая 295 \enquote{Тигров}, 230 \enquote{Пантер} и 100 САУ \enquote{Фердинанд}). Командовали
войсками: на северной части выступа – командующий группы армий \enquote{Центр}
фельдмаршал фон Клюге, на южной – командующий группы армий \enquote{Юг} фельдмаршал фон
Манштейн.

\ii{02_07_2023.stz.news.ua.strana.2.kurskaja_duga.pic.4}

Им противостояли Центральный фронт генерал-полковника Константина Рокоссовского
(на севере) и Воронежский фронт генерала армии Николая Ватутина (на юге) в
составе 1,27 млн человек и 3444 танков. И, что крайне важно, у Красной армии
имелся резерв – Степной военный округ генерала армии Ивана Конева, уже в ходе
Курской битвы переименованный в Степной фронт, в составе 600 тысяч бойцов и
полутора тысяч танков.

Однако общее численное преимущество Красной армии не полностью отражало
ситуацию, сложившуюся в выступе. Подготовившись к обороне, советское
командование считало, что главный удар будет наноситься с севера, из района
Орла, – поэтому Центральный фронт состоял из 738 тысяч человек, а Воронежский –
из 535 тысяч. Между тем, у немцев как раз более мощной оказалась Южная группа
Манштейна, наносившая удар со стороны Белгорода; особенно это касалось
качественного состава – именно в манштейновскую группировку входил 2 танковый
корпус СС, включавший в себя дивизии \enquote{Адольф Гитлер}, \enquote{Мертвая голова} и
\enquote{Рейх}.

В результате бои на северном и южном фасах Курского выступа сложились
по-разному.

Войска фон Клюге, находившиеся под непосредственным управлением Моделя, за пять
дней наступления не смогли прорвать даже первую линию армий Рокоссовского,
вклинились всего на 12 км и уже 10 июля из-за больших потерь перешли к обороне.

На юге, у Ватутина, все шло хуже. Манштейн наступал сразу двумя ударными
группировками, задачей которых было прорвать две советские линии обороны и
образовать сплошной фронт направлением на северо-восток. Однако план сорвался,
поскольку успех подразделений этой группировки оказался неравномерным: в то
время как 4 танковая армия генерал-полковника Германа Гота (в состав которой
входил 2 танковый корпус СС), имевшая в своем составе больше всего \enquote{Тигров}, за
два дня прорвала две линии советской обороны, шедшая справа от него армейская
группа генерала Вернера Кемпфа к исходу 6 июля прорвала только первую линию. В
результате их пути разошлись, и общего фронта они так и не образовали.

Однако прорыв 4 танковой армии даже сам по себе нес серьезную угрозу. Его
подразделения – единственные из всех частей германской армии – выполняли свою
задачу, продвигаясь к городу Обоянь – прямой дороге на Курск. На этом
направлении, неся тяжелые потери, оборонялась советская 1 танковая армия
генерала Михаила Катукова, и было ясно, что в одиночку она долго сражаться не
сможет.

Именно здесь впервые отчетливо проявилась главная проблема Красной армии в
Курской битве: невозможность для советских танков противостоять \enquote{Тиграм}:
88-миллиметровые орудия немецких танков поражали Т-34-76 на том расстоянии, на
котором были недоступны для \enquote{тридцатьчетверок}, у которых \enquote{76} в названии –
это, собственно, калибр пушек; даже если они всё же подбирались к \enquote{Тиграм} на
нужное расстояние, снаряды пробить броню этих \enquote{зверей} не могли.

Чтобы помочь Катукову, Ватутин выделил дополнительно четыре танковых корпуса
для нанесения удара по германской танковой армии с фланга. Ощутимого урона
немцам он не нанес, однако вынудил их отказаться от движения на Обоянь и
двинуться северо-восточнее – в сторону села Прохоровка.

\ii{02_07_2023.stz.news.ua.strana.2.kurskaja_duga.pic.5}

\subsubsection{Прохоровка: решающий бой}

Уже с первых дней наступления в Берлине поняли, что оно идет не по плану.
Поэтому немецкое радио даже не объявило о начале собственной операции: вместо
этого пропагандисты Йозефа Геббельса сообщили, что в наступление пошла Красная
армия, а вермахт и войска СС \enquote{мужественно отражают} советские атаки.

Таким объявлением Берлин хотел подстраховаться, но на самом деле просто
предвосхитил события, поскольку с каждым днем ситуация для Германии становилась
все хуже. Вслед за прекращением наступления на северном фасе Курского выступа
пришли сообщения из района Средиземного моря, о котором Муссолини беспокоился
совсем не напрасно: 10 июля англо-американские войска высадились на Сицилии.

В такой ситуации успешное наступление Манштейна на Курск с юга оказалось для
фюрера единственной возможностью продемонстрировать успех. И выход 2 танковой
армии в район Прохоровки становился главным элементом этого успеха.

Советское командование все это тоже понимало. Сил Воронежского фронта для
борьбы с противником уже не хватало, поэтому пришлось задействовать резервы
Степного фронта: навстречу танкам Гота оно выдвинуло 5 танковую армию
генерал-лейтенанта Павла Ротмистрова и 5 гвардейскую армию генерал-лейтенанта
Алексея Жадова. Именно им и предстояло столкнуться с элитой немецких танковых
войск – корпусом СС обергруппенфюрера Пауля Хауссера – в грандиозном танковом
сражении, ставшем апогеем Курской битвы.

12 июля стало переломным во всех смыслах. Если до сих пор на Курской дуге все
происходило – удачно или неудачно – по плану \enquote{Цитадель}, то теперь Сталин отдал
приказ на собственное контрнаступление – операцию \enquote{Кутузов}. Утром 12-го числа
посла артподготовки Западный фронт генерал-полковника Василия Соколовского и
Брянский фронт генерал-полковника Маркияна Попова пошли вперед; в этой
синхронной атаке должна была участвовать и группировка, выдвинутая в сторону
Прохоровки, однако тут планы пришлось корректировать, поскольку в то же утро
часть немецких войск – 3 танковая дивизия СС \enquote{Мертвая голова} и 3 танковый
корпус оперативной группы \enquote{Кемпф} – начали собственное наступление: первые – в
обход Прохоровки с запада, вторые – с юга.

Поэтому в течение нескольких часов обе армии с двух разных сторон села
одновременно и атаковали, и оборонялись: западнее него \enquote{Мертвой голове}
противостояла 5 гвардейская армия, южнее 3 танковому корпусу немцев – советская
69 армия, а дивизии СС \enquote{Рейх} и \enquote{Адольф Гитлер} восточнее села сдерживали
наступление 5 гвардейской танковой армии.

Преимущества \enquote{Тигров} никуда не делись, но войска Красной армии сделали все,
чтобы их нейтрализовать. \enquote{Тридцатьчетверки}, пользуясь количественным
преимуществом, навязывали ближний бой, в котором заходили сбоку и оттуда
поражали \enquote{Тигров}. Кроме того, преимущество в воздухе получила советская
авиация, которая и уничтожила авиабомбами большую часть германских \enquote{зверей}.

В цифрах потерь итог Прохоровской битвы оказался не в пользу Красной армии: она
потеряла не менее 328 танков, в то время как немцы – около 70 танков и САУ.
Однако главной цели советское командование достигло: Прохоровку Манштейн не
взял, и последнее наступление противника было остановлено.

Уже на следующий день, 13 июля, Гитлер собрал совещание, на котором предложил
прекратить операцию \enquote{Цитадель} и перебросить часть войск в Южную Италию для
защиты от возможного англо-американского вторжения. Манштейн, однако, настаивал
на возобновлении продвижения его группы \enquote{Юг}, чтобы сорвать советские
наступательные планы.

Фюрер согласился, и 2 танковый корпус СС еще два дня наступал, однако общая
ситуация для вермахта продолжала ухудшаться. 15 июля Центральный фронт
Рокоссовского присоединился в контрнаступлении к Западному и Брянскому, и
германской армии пришлось перейти к обороне по всей линии боевых действий.

В своих мемуарах \enquote{Потерянные победы} Манштейн резко критиковал Гитлера за
прекращение наступления – мол, настоящий успех был близок. Но в военных кругах
эту точку зрения не поддерживают: Манштейн явно недооценивал резервы Красной
армии, чуть позже продемонстрированные в операции \enquote{Полководец Румянцев}
непосредственно против него. Другой вопрос, что к обороне немцы были готовы еще
хуже, чем к наступлению.

\ii{02_07_2023.stz.news.ua.strana.2.kurskaja_duga.pic.6}

На этом Курская битва закончилась, автоматически превратившись в операцию
\enquote{Кутузов}, результатом которой стало освобождение Орла 5 августа, а затем и
Белгорода и Харькова.

\ii{02_07_2023.stz.news.ua.strana.2.kurskaja_duga.pic.7}

Наступательный порыв после Курской битвы привел к целой серии удачных операций,
в ходе которых Красная армия освободила Донбасс, Запорожье и Днепропетровск. А
затем и Киев.

Как и предполагал Гитлер, провал летнего наступления привел к
внешнеполитическим проблемам: уже 25 июля в Риме свергли Муссолини, и
правительство Бадольо почти сразу начало зондировать почву относительно
перемирия с союзниками.

Ситуация на всех фронтах окончательно повернулась в пользу антигитлеровской
коалиции. Это и был тот коренной перелом в войне, который предопределил будущее
крушение Третьего Рейха. После июля 1943 года вермахт проведет всего одну
крупную наступательную операцию – зимой 1944-45 годов в Арденнах на Западном
фронте, – но и ее придется прекратить из-за наступления Красной армии на
Восточном фронте.

В ноябре 1943 года в Тегеране собралась первая конференция лидеров
стран-союзниц – Франклина Рузвельта, Иосифа Сталина и Уинстона Черчилля. На ней
уверенно обсуждали послевоенное устройство мира, включая создание Организации
Объединенный наций. И эту уверенность союзникам дала победа в Курской битве.
