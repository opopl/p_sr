% vim: keymap=russian-jcukenwin
%%beginhead 
 
%%file 16_01_2022.yz.ljubimoje_figurnoje_katanie.1.anna_scherbakova
%%parent 16_01_2022
 
%%url https://zen.yandex.ru/media/lovely_fk/krasnorechivyi-otvet-anny-scerbakovoi-vsem-nedobrojelateliam-silnyi-voin-i-princessa-v-odnom-lice-61e3d6128988a133a2bb59b8
 
%%author_id yz.ljubimoje_figurnoje_katanie
%%date 
 
%%tags sport,rossia,figurist,figurnoje_katanie
%%title Красноречивый ответ Анны Щербаковой всем недоброжелателям. Сильный воин и принцесса в одном лице
 
%%endhead 
\subsection{Красноречивый ответ Анны Щербаковой всем недоброжелателям. Сильный воин и принцесса в одном лице}
\label{sec:16_01_2022.yz.ljubimoje_figurnoje_katanie.1.anna_scherbakova}

\Purl{https://zen.yandex.ru/media/lovely_fk/krasnorechivyi-otvet-anny-scerbakovoi-vsem-nedobrojelateliam-silnyi-voin-i-princessa-v-odnom-lice-61e3d6128988a133a2bb59b8}

\ifcmt
 author_begin
   author_id yz.ljubimoje_figurnoje_katanie
 author_end
\fi

Выступление Анны Щербаковой в произвольной программе чемпионата Европы – это
что-то невероятное. По эмоциональному накалу, наверное, это был самый сильный
прокат субботнего вечера. После неудачной короткой программы казалось, что
ничего хорошего уже не будет.

\ii{16_01_2022.yz.ljubimoje_figurnoje_katanie.1.anna_scherbakova.pic.1}

Тренировки в очередной раз не обнадеживали и не предвещали ничего хорошего.
Фигуристка была словно сама не своя, почти не прыгала и будто бы боялась
заходить на четверной. Состояние всех болельщиков идеально передали
комментаторы Первого Канала Алексей Ягудин и Александр Гришин:

Ягудин: Честно говоря, мы не понимаем, как она это делает. Но каждый раз она
выходит, как в последний раз. То, что происходит сейчас на арене, трудно
передать словами.

Гришин: Такого характера, как у Щербаковой, еще поискать. И я не уверен, что он
найдется. Даже когда, казалось бы, все против нее...

Ягудин: Да, казалось бы, тупиковая ситуация. Но нет! В произвольной программе,
просто я помню, уже много моментов таких было, она выходит и вновь удивляет.
Она нас удивляла, удивляет и, я уверен, будет удивлять.

Да, Аня, конечно, в очередной раз по-хорошему всех удивила и своим безупречным
прокатом закрыла рты всем тем, кто писал про нее плохие вещи и не желал видеть
на Олимпиаде в Пекине.

Что больше всего впечатляет, так это умение фигуристки собираться в нужный
момент. (Непонятно, конечно, зачем нужно было загонять себя в тупик и терять
злополучный каскад в короткой программе, что поделать, бывает...) Но то, как
она умеет выходить победительницей из сложнейших ситуаций, дорогого стоит.

У подавляющего большинства болельщиков чемпионки мира было траурное настроение
после короткой программы. Мало кто верил в очередное чудо, особенно после не
самых радужных тренировок. При этом сама спортсменка, кажется, испытывала
абсолютно схожие эмоции:

\begin{zznagolos}
«До проката мне казалось, что эти два дня одни из худших в моей жизни, что это
невозможно пережить и было много ненужных лишних мыслей.

После падения в короткой было морально тяжело. Я и до соревнований сильно
нервничала, а после короткой вообще было ощущение, что все завалила, ни с чем
не справилась, все потеряно – и теряется уверенность в себе.

Задача каждого спортсмена – собраться в нужный момент и показать самое лучшее,
вопрос – как сильно на тебя волнение влияет. У меня тоже такое было после
короткой – после чего хотелось вернуть себе уверенность» – поделилась
впечатлениями Щербакова после проката.	
\end{zznagolos}

Как, потеряв уверенность в себе, вернуть ее в кратчайшие сроки? Наверное, это
удел по-настоящему сильных людей. Важно не то, сколько раз ты упал, а сколько
раз ты сумел подняться. Анин боевой дух невозможно сломить. И это здорово.
