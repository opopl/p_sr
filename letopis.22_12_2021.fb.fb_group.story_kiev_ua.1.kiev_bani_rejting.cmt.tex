% vim: keymap=russian-jcukenwin
%%beginhead 
 
%%file 22_12_2021.fb.fb_group.story_kiev_ua.1.kiev_bani_rejting.cmt
%%parent 22_12_2021.fb.fb_group.story_kiev_ua.1.kiev_bani_rejting
 
%%url 
 
%%author_id 
%%date 
 
%%tags 
%%title 
 
%%endhead 
\zzSecCmt

\begin{itemize} % {
\iusr{Владимир Бондаренко}
на сковороды напротив церквы Мыколы Доброго. сейчас там БЦ

\iusr{Виктор Задворнов}

С легким паром! Посещал баню на Кременецкой (рядом с военкоматом Святошинского
района), на Брест-Литовском просп. - теперь реконструированное помещение
конторы бывшего \enquote{Брокбизнес банка}. Но легендарной считаю баню в поселке
Коцюбинский. Там отапливали банные \enquote{залы} и грели воду благодаря
функционированию кочегарки. Парадокс: идешь принять душ и попариться, а из
двери котельной выглядывает \enquote{темная} от угольной сажи личность - пьяненький
кочегар. Мыться пожалуйте! Цены в 60-е-70-е были весьма сносными.


\iusr{Sasha Jampolsky}

У меня немного менее исторические заметки про бани. Выросши в общей квартире на
Крещатике, без ванны, то есть ванна была но она не работала наверно с ещё
довоенных времен, ходили мы в Караваевские бани. Раз неделю, я с папой а мама с
бабушкой, ходили иногда в общее , а зачастую в номера. Кажется это стоило тогда
рупь за час. А потом, подросши и распознавши прелести противоположного пола, я
ходил туда с избранницами так как там паспорта у меня не просили потому что
знали с детства и ещё за два рубчика ( один сверху) можно было занять номер на
полные два часа.

А заодно и помыться


\iusr{Елена Мельникова}

В 60-70г. 20 века многие посещали общественные бани. Подол например ходил в баню
по Героев Триполья (ныне Спасская). Общие номера с парилкой, отдельные
номера-душ и ванная. Мылись как правило раз в неделю. Пиво и раков не помню, а
маникюр, парикмахер и замечательная косметичка Галина Григорьевна курившая
только \enquote{Беломор} по фронтовой привычке.

Может кто помнит?

\begin{itemize} % {
\iusr{Vladimir Dotsenko}
\textbf{Елена Мельникова}

С 1954 по 1961 посещали эти бани всей семьей, после в буфете для Отца было
бочковое пиво, а для нас, детишек - сладкий вкуснющая газировка с двойным
сиропом...


\iusr{Ольга Кирьянцева}
\textbf{Елена Мельникова} да, посещали эту баню). Номер с душем и ванной 1 рубль.
\end{itemize} % }

\iusr{Владимир Хворов}

у Нас на Андреевском спуске вода была по среди двора (колонка) Мы всей семьей
по субботам ходили в баню на Малой Житомирской Иногда в номера ст 1 рубль в час
Там были душ и ванная Ванная была просто роскошь Меня тщательно мыли а потом
разрешали плюхаться в ванной Это был кайф


\iusr{Катерина Силина}

Вы забыли о бане на Постышева. Мы всей семьей ходили туда, когда проживали на
Михайловской в коммунальной квартире. Я была ещё маленькая, но помню там был
бассейн в лягушками по углам.

\begin{itemize} % {
\iusr{Олег Чудновский}
\textbf{Катерина Силина} Да это сейчас ул. Малая Житомирская, там о ней есть.

\iusr{Катерина Силина}
\textbf{Олег Чудновский} Спасибо, теперь понятно, что не забыли нашу баню

\iusr{Светлана Николаевна}
\textbf{Олег Чудновский} в детстве жили в коммуналке на Тарасовской и тоже ходили в баню Караваевскую, память осталась на всю жизнь, спасибо напомнили, чудные были времена!!

\iusr{Олег Чудновский}
\textbf{Светлана Николаевна} Приятно, но спасибо надо сказать Юрию Бубнову.

\iusr{Лариса Кушниренко}
\textbf{Катерина Силина} Очень даже помню . Жили на Десятинной и это была наша баня. @igg{fbicon.face.happy.two.hands} 
\end{itemize} % }

\iusr{Петро Гарматюк}
А на вулиці Михайлівській баня була ?
Чи я помиляюся ?

\begin{itemize} % {
\iusr{Irena Visochan}
\textbf{Петро Гарматюк} это на Постышева рядом.

\iusr{Петро Гарматюк}
\textbf{Irena Visochan}
Спасибо.

\iusr{Давид Вайнблат}
\textbf{Петро Гарматюк} Була. Я жив на Прорізаний і туди ходів
\end{itemize} % }

\iusr{Владимир Шевцов}
Печерская Московская угол Суворова. Сейчас там Зоряный

\iusr{Алла Ковтун}
Троицкие до конца 80-х ещё работали.

\iusr{Татьяна Жалнина}
\textbf{Алла Ковтун} точно, я там была ,теперь только воспоминания

\iusr{Микола Ахаладзе}
А ви забули Лазню на проспекті Перемоги (бувший Брест-Литовський проспект).

\iusr{Анатолий Борозенец}

Маю сумнів щодо присутності \enquote{Московських бань} на вул. Ярославів Вал,40.
Будівлю, де нині інститут ім.Карпенка-Карого, звели для економічної школи в
1907-му.

\begin{itemize} % {
\iusr{Елена Гнеденко}
\textbf{Anatoly Borozenets} 

Похоже, что эти бани находились на ул. Бульварно-Кудрявской, приблизительно в
\#6 или8, (точно не помню). Там жили родственники ( и в пятидесятых годах 20
столетия), в чердачных помещениях. Сейчас там административное здание.


\iusr{Анатолий Борозенец}
\textbf{Елена Гнеденко} Ні. \enquote{Московські бані} знаходилися на вул. Ярославській, 17/22.
\end{itemize} % }

\iusr{Pasha Paolo}

Бани Каплера, которые \enquote{Московские} НИКОГДА не располагались на ЯрВалу.
Правильный адрес - Константиновская 22/17. Фактически, это гостиница \enquote{Сион},
сами бани были во дворе и теперь это действительно корпус Театрального
института.

\begin{itemize} % {
\iusr{Анатолий Борозенец}
Корпус Театрального інституту там колись був. Зараз, наскільки мені відомо, там стоматологічна клініка

\ifcmt
  ig https://scontent-frt3-1.xx.fbcdn.net/v/t39.30808-6/269778544_3072023223053391_3238025887301750879_n.jpg?_nc_cat=102&ccb=1-5&_nc_sid=dbeb18&_nc_ohc=V0PlqVYk-XgAX89i0sE&_nc_ht=scontent-frt3-1.xx&oh=00_AT-FdM49YrDS86Fw_dDpru5A7BreVmHPVhzJzqI_JzdLsA&oe=61CACF58
  @width 0.4
\fi

\begin{itemize} % {
\iusr{Александра Майорова}
\textbf{Анатолий Борозенец} 

клиника есть. РЯДОМ В ТОМ ЖЕ ДВОРЕ, КАК ВОЙТИ В АРКУ КОРПУС КГИТИ КАРПЕНКО -
КАРОГО В СТАРЫХ БАНЯХ С 1991 ГОДА. ЭТО ОТДЕЛЬНО СТОЯЩЕЕ 2 - Х ЭТАЖНОЕ ЗДАНИЕ
ВНУТРИ ДВОРА. ВХОД СО СТОРОНЫ СКВЕРА.


\iusr{Анатолий Борозенец}
\textbf{Александра Майорова} Дякую за інформацію!

\iusr{Александра Майорова}
\textbf{Анатолий Борозенец} Нина Апольская сейчас разместила фото с Адой Роговцевой в той самой Арке! Найдете легко в К.И. 52 ФОТОГРАФИИ ОТ НЕЕ....!

\end{itemize} % }

\iusr{Pasha Paolo}
Собственно, фото с вывеской бань.

\ifcmt
  ig https://scontent-frt3-2.xx.fbcdn.net/v/t39.30808-6/269859073_4663856050357469_709307188187473244_n.jpg?_nc_cat=101&ccb=1-5&_nc_sid=dbeb18&_nc_ohc=4urIWXH3oLQAX_0uW_i&_nc_oc=AQkyHhywyUMIYjOhbsiViHly2UBjREkQ6j_cAMX67B9l2o078KmSKIaYIU6gl0EIUvM&_nc_ht=scontent-frt3-2.xx&oh=00_AT-MyS_cxZLhdq4NSKFLk5aYtpahjbqXfmehlvtZL-Gu4w&oe=61CB0E7F
  @width 0.4
\fi

\end{itemize} % }

\iusr{Александр Сторчевой}
А на Ивана Кудри, кто помнит?

\begin{itemize} % {
\iusr{Кира Юкаева}
\textbf{Александр Сторчевой} назывались Боинские

\iusr{Олег Чудновский}
\textbf{Александр Сторчевой} Помним, хорошая баня была, там сейчас новый дом выстроили.

\begin{itemize} % {
\iusr{Алексей Шевнюк}
\textbf{Олег Чудновский} Олег а на Московской возле кинотеатра Авангард?

\iusr{Алексей Шевнюк}
Там в основном ущилище Связи парилось. Ещё с тазиками

\iusr{Elina Modenova}
\textbf{Алексей Шевнюк} нет, не только! Нас мама водила именно в эту баню, возле кинотеатра Авангард на Московской. Тазики- да, парилка- супер. Я старалась залезть на самый верх.

\iusr{Алексей Шевнюк}
\textbf{Elina Modenova}

\ifcmt
  ig https://i2.paste.pics/24f63d14e05ecec1dda838cb22983e41.png
  @width 0.2
\fi

\iusr{Олег Чудновский}
\textbf{Алексей Шевнюк} Помню, Леха и эту и которая на ул.Лейпцигская была!!!
\end{itemize} % }

\iusr{Светлана Власенко}
\textbf{Александр Сторчевой} я ходила с сестрой
\end{itemize} % }

\iusr{Yuriy Bubnov}

Всем, кто пишет: \enquote{а вы забыли упомянуть о бане... такой-то}. Замечу, эта
публикация - не перечень всех бань Киева. К тому же, для вас есть простор для
собственных воспоминаниях о любимых банях - пишите, уверен, все с удовольствием
почитают!


\iusr{Павел Левченко}
А СЕГОДНЯ ХОТЬ ОСТАЛИСЬ ПРИЛИЧНЫЕ БАНИ КИЕВА???

\iusr{Ірина Кравець}
А не Вы наследник бань Бубнова?

\begin{itemize} % {
\iusr{Yuriy Bubnov}
\textbf{Ірина Кравець} Прочтите до конца текст публикации. Там есть ответ на вопрос.

\iusr{Ірина Кравець}
\textbf{Yuriy Bubnov}  Извините, невнимательна
\end{itemize} % }

\iusr{Наталия Дудник}

Спасибо большое  @igg{fbicon.hands.pray} , очень интересно. Я помню в детстве ходила с сестрой в
Демеевские бани в 80-х. Вполне приличные парилки были. С них у меня и началась
большая любовь к этому делу @igg{fbicon.face.smiling.eyes.smiling}. Правда много десятилетий ходим с друзьями в
сауну

\begin{itemize} % {
\iusr{Станіслав Теліцький}
\textbf{Natalia Dudnik}, 

там була розкішна чайна, оформлена під старовину. О 10.00 (до закінчення першої
зміни) вона починала свою роботу. Ми, з друзями, полюбляли ходити туди уранці.
Слід згадати, що там були сауна і класична парна, а також - басейн і суха зала.
А, після процедур - свіжий чай з мармеладом і печивом. Незабутні відчуття!

\iusr{Катерина Силина}
\textbf{Наталия Дудник} Я тоже помню Димеевские бани.

\iusr{Натуся Натуся}
\textbf{Наталия Дудник} У меня детство тоже было связано с Димеевской баней!
С покойной Мамочкой ходили по выходным!
На 2 этаже была парикхматерская и после бани я ждала, когда Мама наведёт красоту!
Для меня, для ребёнка это был интересный мир!!!

\iusr{Наталия Дудник}
\textbf{Натуся Натуся} именно - интересный мир.. это что-то было отличное от текущей жизни

\iusr{Alexey Shahan Snigur}
\textbf{Наталия Дудник} А где именно находились Демеевские бани ?

\begin{itemize} % {
\iusr{Наталия Дудник}
\textbf{Alexey Shahan Snigur} напротив Центрального автовокзала, если для ориентировки...

\iusr{Натуся Натуся}
\textbf{Наталия Дудник} Точнее, напротив стеклозавода!

\iusr{Натуся Натуся}
\textbf{Alexey Shahan Snigur} Точнее, напротив стеклозавода.

\iusr{Наталия Дудник}
\textbf{Натуся Натуся} именно
\end{itemize} % }

\end{itemize} % }

\iusr{Irina Figurskaya}

Дійсно, \enquote{Каравпєвські} були найкращими. Це я знаю по своєму досвіду. Моя бабуся
мене туди водила. Хоча дома була ванна, але бабуся казала, що \enquote{митися у ванній
- не митися}. Дійсно, миття у бані приносило велике задоволення. Після неї -
таке почуття, що от от злітиш! Це цілий ритуал.

P.S. Один раз ми пішли у Троїцьки (не було білетів у Караваєвські). Нам не
сподобалось. Більше не зраджували Караваєвським.

\iusr{Александра Цымбал}

Демиевская баня стоит развалиной почти в самом конце Лобановского
(Краснозвёздного), рядом с бывшей синагогой Боришпольского, в которой сейчас
детские кружки.

\iusr{Катерина Силина}
\textbf{Александра Цымбал} Мы тоже посещали эту баню, а позже пользовались банно-прачечным комплексом самообслуживания

\iusr{Александра Цымбал}
У меня есть книга о киевских банях 19 века - один из томов о разных сферах деятельности в Киеве в то время.

\iusr{Надежда Лабик}
В бани Бубнова я ходила т.к. жила от них недалеко.

\iusr{Ирина Собко}
На малой Житомирской ещё работают?

\iusr{Nick Nesin}
З друзями пиво в цій  @igg{fbicon.face.smiling.hearts} бані пили

\iusr{Polina Feldman}

Мы ходили с бабушкой на Голосаеевсой была баня это был как сщастливый день
железные тазики и отельные кабинки а также кто платил больше отельный номер для
купания в моей памяте осталось это незнаю еть ли там еще это Сталинка это лет 50
тому  ❄ ️ ☃️ @igg{fbicon.bouquet} 

\end{itemize} % }
