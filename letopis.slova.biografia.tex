% vim: keymap=russian-jcukenwin
%%beginhead 
 
%%file slova.biografia
%%parent slova
 
%%url 
 
%%author 
%%author_id 
%%author_url 
 
%%tags 
%%title 
 
%%endhead 
\chapter{Биография}

%%%cit
%%%cit_head
%%%cit_pic
\ifcmt
  tab_begin cols=2
     pic https://strana.ua/img/forall/u/0/36/2021-08-10_09h12_24.png
     pic https://img.strana.ua/img/article/3479/taras-kremin-rabotal-78_main.jpeg
  tab_end
\fi
%%%cit_text
Как Тарас Креминь работал в команде Януковича.  Сейчас Тарас Креминь -
уполномоченный по защите государственного языка. Из-за радикальных высказываний
получил в Сети прозвище "шпрехенфюрер".  \emph{Биография} Креминя начинается
примерно с 2010 года. С тех пор он входил в партию "Фронт перемен" Яценюка, был
депутатом облсовета в родном городе Николаеве.  После Евромайдана недолго, с
февраля по ноябрь 2014 года, занимал пост главы облсовета.  В годы
президентства Порошенко у Тараса Креминя проснулось стремление защищать "мову".
Он был в группе соавторов законопроекта о языковой политике, который еще
называют законом о тотальной украинизации. Документ предусматривает переход на
государственный язык всей публичной сферы - от обслуживания до фильмов на ТВ.
Также на счету у Креминя соавторство скандального закона об образовании. Он
строится на том, что вытесняет языки национальных меньшинств - в особенно
русский - из школ Украины.  Такая бурная деятельность Тарас Креминя, видимо,
была отмечена на высшем уровне. В результате при президенте Владимире Зеленском
он был назначен главным защитником украинского языка в стране. Эту работу
мовный омбудсмен выполняет без учета интересов нацменьшинств - на днях всех,
кто недоволен украинизацией, он призывал покинуть страну, фильмы на русском
называл угрозой национальной безопасности
%%%cit_comment
%%%cit_title
\citTitle{\enquote{Принципиальный парень до жути}. Как украинизатор Креминь помогал Януковичу стать президентом}, 
Анна Копытько; Екатерина Терехова, strana.ua, 10.08.2021
%%%endcit
