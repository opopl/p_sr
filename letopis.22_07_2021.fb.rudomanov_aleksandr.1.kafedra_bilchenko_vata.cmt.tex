% vim: keymap=russian-jcukenwin
%%beginhead 
 
%%file 22_07_2021.fb.rudomanov_aleksandr.1.kafedra_bilchenko_vata.cmt
%%parent 22_07_2021.fb.rudomanov_aleksandr.1.kafedra_bilchenko_vata
 
%%url 
 
%%author_id 
%%date 
 
%%tags 
%%title 
 
%%endhead 
\subsubsection{Коментарі}

\begin{itemize} % {
\iusr{Сергей Дубильер}
Никаких шансов. Борцуны за права "притесняемых и обездоленных" категорий
населения мгновенно найдут, что неэффективный работник относится к категории
страдальцев от феминизма, эйджизма, расизма, гомофобии и др., др., др. И
увольнять его нельзя

\begin{itemize} % {
\iusr{Yurii Lev}
\textbf{Sergey Dubiljer} Роботу всім піда..сам! )

\iusr{Сергей Дубильер}
\textbf{Yurii Lev} "...і кожен з них народ, всі богоносці..." (с)  @igg{fbicon.wink} 
\end{itemize} % }

\iusr{Aleksei Parnowski}
Він не фактично радянський, а просто радянський. Рік ухвалення — 1971.

\iusr{Зоя Звиняцковская}

Чесно - туди цьому всьому і дорога. Бо якого Дерріди в педагогічному
університеті раптом з‘явився осередок вивчення філософії та культурології? Може
вони б зосередились на педагогіці, було б непогано? Зараз у нас в кожному
швацькому коледжі кафедра філософії та культурології - дуже зручно, фіг його
знає, що це таке, складай програму довільно і викладай що хочеш, а грошики зі
студентів капають, а дипломи справжні.


\iusr{Дмитрий Мукойда}

В Германии-Франции-Швеции тоже нельзя увольнять людей. На самом то деле есть
обратная проблема так называемый моббинг-буллинг-боссинг на работе. Одной из
форм является саботаж работы сотрудника тоесть создание условий его
немродуктивности. В Американском праве можно увольнять в один день но если
потом сотрудник обратится в суд то моббинг-боссинг может быть поводом для очень
существенных санкций и компенсаций

\begin{itemize} % {
\iusr{Inna Zvyagintseva}
\textbf{Дмитрий Мукойда} В Германии фиксированные контракты  @igg{fbicon.frown} 

\iusr{Дмитрий Мукойда}
\textbf{Inna Zvyagintseva} право в романо-германской системе построено на НПА как основном источнике права. Так вот законы там защищают сотрудника очень сильно. А контракты всегда под законами а не над ними
\end{itemize} % }

\iusr{Pavlo Moskalenko}
На Заході звільнити держслужбовця ще важче.

\begin{itemize} % {
\iusr{Олександр Рудоманов}
\textbf{Pavlo Moskalenko} але у нас і працівника важко

\iusr{Inna Zvyagintseva}
\textbf{Pavlo Moskalenko} В Австрії й звичайного працівника набагато важче

\iusr{Владимир Раднянко}
\textbf{Олександр Рудоманов} Треба переводити на контракт, як вчителів пенсійного віку, всі нюанси там прописати терміном на 1 рік. Щось не так, не продляємо і все, до побачення....

\iusr{Катерина Стулень}
\textbf{Pavlo Moskalenko} Але там є контракт. І неефективному працівникові просто не пролонгують контракт з ним, і все - його нема.

\iusr{Катя Кондратюк}
\textbf{Pavlo Moskalenko} і що тут хорошого? В Італії саме лівацьке трудове законодавство. І чим закінчилось протистояння , наприклад, фіату? Заводи винесли в східну Європу, італійці залишились з правами, але без роботи(

\iusr{Pavlo Moskalenko}
\textbf{Катя Кондратюк} я не сказав, що це добре.

\iusr{Inna Zvyagintseva}
\textbf{Katya Kondratyuk} Заводи винесли, тому що робоча сила дешевша
\end{itemize} % }

\iusr{Андрій Жидков}

Так у нас якість законів загалом дуже невисока. Ось по цьому прикладу. Закон
про освіту та закон про вищу освіту містять вимоги виховання поваги до держави
і т.інш. Але ніяких механізмів примусити працівника виконувати ці вимоги немає.
Це ж достатнь проста річ. В контакті прописати "дотримуватися закону... при
незгоді - звільнення". Все. Але маємо, що маємо.

\begin{itemize} % {
\iusr{Марфа Соломенцева}
\textbf{Андрій Жидков} спробуй довести, що не дотримується, це ж така суб'єктивна річ ця повага ...

\iusr{Андрій Жидков}
\textbf{Marfa Solomenceva} ну це як з нашими сєпарами. Якщо людина прямо на камеру не скаже "я посягаю на територіальну цілісність держави", то все інше "нещитається"
\end{itemize} % }

\iusr{Andriy Servetnyk}

Саме так. Не буде порядку без можливості в будь-який момент звільнити
працівника. Не потрібен - отримуй компенсацію за угодою сторін і шукай собі
місця деінде. А зараз просто вакханалія, особливо з різноманітними чиновниками,
які поновлюються і поновлюються на роботі через суди...

\begin{itemize} % {
\iusr{Alena Morgan}
\textbf{Andriy Servetnyk} цікаво про компенсацію за згодою сторін))))
\end{itemize} % }

\iusr{Ganna Bezgina}
Але якщо дати можливість звільняти на раз два, то які запобіжники в ситуаціях
коли працівника просто хочуть позбавитись?

\begin{itemize} % {
\iusr{Олександр Рудоманов}
\textbf{Ganna Bezgina} у мене особисто думка, що якщо роботодавець, який дає роботу хоче звільнити працівника, якому він платить зп, то він має право це робити.
Бо в магазині можна вибрати не гнилі продукти, а нормального робітника ні? Але дати тижня за 2 попередження.

\iusr{Михаил Дикун}
\textbf{Ganna Bezgina} роботодавець повинен мати можливість позбавитись незручного працівника, щоб не було такої ситуації, як ця

\iusr{Ganna Bezgina}
\textbf{Олександр Рудоманов} тоді працівник практично позбавляється своїх прав. Роботодавець матиме право без будь якої причини звільнити працівника.

\iusr{Ольга Войтко}
\textbf{Ganna Bezgina} Раніше можна було взяти на випробний термін, щоб зрозуміти якій той працівник насправді. А зараз береш кота в мішку, а звільнити не можеш... Особливо це стосується маленьких ФОП де реорганізацією не пахне, і комісій у складі 3-х людей (без того кого звільнити) скласти не можна, бо тільки сам ФОП і один робітник.

\iusr{Ganna Bezgina}
\textbf{Михайло Дикун} це фактично перетворює працівника в раба.

\iusr{Валентина Шапошник}
Цікаво, якому це працівнику цікаво працювати там, де його не цінують? Як може бути нормальним ККД від таких взаємин?
 · Ответить · 8 нед.
\iusr{Ganna Bezgina}
\textbf{Валентина Шапошник} очевидно ви забуваєте про ситуацію на ринку праці. Притомних робочих місць на всіх не вистачає нажаль.

\iusr{Ganna Bezgina}
\textbf{Ольга Войтко} а зараз норму про випробувальний термін змінено?

\iusr{Дмитро Кучерак}
\textbf{Ganna Bezgina} вам же пояснили, що людина, яка платить гроші вправі вирішувати, хто їй потрібен і ті

\iusr{Ganna Bezgina}
\textbf{Дмитро Кучерак} в жодній країні немає такого не щоб звільнли людину без причини.

\iusr{Гена Дем'янчук}
А чого роботодавець зобов'язаний вас там тримати? В нього є право обирати собі працівників з якими йому вигідно і комфортно працювати

\iusr{Денис Милий}

Право на працю є одним з базових прав людини. Держава створює умови для його
реалізації та захищає. Не подобається це роботодавцю - він завжди може
припинити діяльність зовсім. Це ж добровільна справа, на відміну від дотримання
трудового законодавства.


\iusr{Ольга Войтко}
\textbf{Ganna Bezgina} так, її немає


\iusr{Олександр Рудоманов}
\textbf{Валентина Шапошник} \textbf{Виктор Задворнов}, розкажіть про те, як вас не могли звільнити, хоча ви нічого не робили  @igg{fbicon.smile} 
\iusr{Сергей Дубильер}
\textbf{Олександр Рудоманов} в теории есть практика ежегодного перезаключения договора. Человек дорабатывает до конца законтрактованного периода и не получает новый контракт

\iusr{Ganna Bezgina}
\textbf{Гена Дем'янчук} а тому що існують закони. Ми ж не в феодальному суспільстві живемо

\iusr{Гена Дем'янчук}
\textbf{Денис Милий} За те ніхто в здоровому розумі не хоче офіційно наймати нових працівників. Бо совкова держава зуміла і там все зарегулюватт

\iusr{Денис Милий}
\textbf{Гена Дем'янчук} якщо я зараз вас попрошу перерахувати цих "всіх" - ви цього не зможете зробити. Тому що це м'яко кажучи не так)

\iusr{Гена Дем'янчук}
\textbf{Ganna Bezgina} Закони встановлені ще совковими окупантами такий собі аргумент

\iusr{Валентина Шапошник}
\textbf{Олександр Рудоманов} , а чого це мене не могли звільнити? Як тільки сказали, що я не потрібна, відразу встала і вийшла.

\iusr{Romko Krytko}
\textbf{Ganna Bezgina} якщо працівника хочуть позбавитись то його позбавляться з будь-якими законами...

\iusr{Віктор Револьвенко}
\textbf{Денис Милий} ага розкажіть це медсестрам з 4 тисячами грн зп. Оце право на працю, не хочеш мінімалку на важкій роботі звільняйся, і ти зможеш знайти лише таку ж роботу на мінімалку.

\iusr{Денис Милий}
\textbf{Віктор Махно} а я хіба казав, що держава в нас справляється на відмінно із покладеною на нею функцією створення належних умов для реалізації права на працю? Ні. Але це не означає, що роботодавці не повинні дотримуватися трудового законодавства. Це не виправдання.

\iusr{Romko Krytko}
\textbf{Денис Милий} цікаво, якщо усі роботодавці припинять діяльність зовсім,
як ви будете проводити свою популістично-соціалістичну політику? За чий кошт?
Ви, як депутат оберіть собі помічника, який нічого не робитиме, платіть йому
заралату і ні в якому разі не звільняйте бо у нього є право на халяву. Як у 80\%
усіх хто працює на державних роботах, ось їхній і ваш результати роботи є
показником життя українців.

\iusr{Макс Макс}
\textbf{Ольга Войтко} можна скласти договір

\iusr{Ольга Войтко}
\textbf{Макс Гусейнов} договір не можна, а об'язково. А от після договору фіг звільниш.

\iusr{Олександр Равчев}
\textbf{Олександр Рудоманов} в нас більше 50\% працівників - державна власність)

\iusr{Anatolii Oliynyk}
\textbf{Олександр Рудоманов} ІМХО, з своєї кишені платить приватний роботодавець. І звільняти він має право одноосібно.
В держструктурах інша ситуація. Держава забезпечує право на працю і вона ж платить. Тому звільняти може керівник через рішення трудового колективу. І це рішення працівник має право оскаржити в суді.
По іншому - буде беспредел.

\iusr{Ganna Bezgina}
\textbf{Гена Дем'янчук} німецькі почитайте


\iusr{Олександр Рудоманов}
\textbf{Anatolii Oliynyk} є департамент, який нічого не робить
Чому як платник податків, з моєї кишені йдуть туди гроші?

\iusr{Markijan Hrynovets}
\textbf{Олександр Рудоманов} як викривач корупції, якого вже 2чі за це звільняли - скажу, що бажання роботодавця не може бути причиною для звільнення. Звільняти можна і треба за невідповідність вимогам до місця праці (напрофесіоналізм) або невиконання роботи. Особливо це стосується бюджетної сфери, де кожен начальник - цар зі звичаєвим правом.

\iusr{Андрей Пращарук}
\textbf{Ganna Bezgina} якби особисто мене хотіли б «позбавитися», я б навряд чи хотів лишатися працювати у такому колективі. Єдине, якщо я б розумів, що я виконував роботу добре, а звільнили мене просто так. То я б міг вимагати якусь компенсацію, але б я точно не повернувся б туди працювати назад.
Якщо говорити про скорочення штату, то напевне було б добре виплачувати скороченим працівникам компенсацію у вигляді 1-2 місячної зп, щоб була можливість спокійно знайти нову роботу.

\iusr{Inna Zvyagintseva}
\textbf{Олександр Рудоманов} Ок, а якщо звільнення не пов'язане з поганою роботою? Наприклад, хочуть звільнити вагітну жінку? Які будуть запобіжники в даній ситуації?
 · Ответить · 8 нед.

\iusr{Олександр Рудоманов}
А хіба в декреті працюють, а не беруть типу відпустку?

\iusr{Елена Ена}
\textbf{Romko Krytko} , Ви дуже помиляєтесь. Навіть великі компанії з юристами, я не кажу вже про ФОП, мають зі звільненням працівників складності, якщо виник конфлікт.

\iusr{Romko Krytko}
\textbf{Єна Олена} я знаю, що є складності, але немає неможливого.

\iusr{Елена Ена}
\textbf{Romko Krytko} , Ви оптиміст. А як з такими законами жити реалістам?)

\iusr{ходаковский костя}
\textbf{Олександр Рудоманов} примеров масса... полиция, попробуй там кого то уволить

\iusr{Михаил Дикун}
\textbf{Ganna Bezgina} ні, кому не подобається - написав заяву і вперед на нову роботу. У нас звільнитися за власним бажанням дуже просто і зараз, а от звільнити людину дуже складно. І це треба міняти

\iusr{Bogdan Gerasymchuk}
\textbf{Ganna Bezgina} для того щоб роботодавець не хотів позбавитись
працівника , то останній не має бути мудаком ... як у 90\% випадків . А радянський
трудовий кодекс 1974 по моєму , ліпиться до сьогодення як можливість охороні
праці розірвати будь якого роботодавця , навіть самого правильного і чесного .
І ще , перед тим як щось вимагати у того хто бізнес створив , для початку
потрібно пройти його шлях в створити своє , відпрацювати ефективно хоча б
рочків 5 , а потім вчити роботодавця розуму ....

\iusr{Михаил Дикун}
\textbf{Денис Милий} до чого тут право на працю? Звільнили людину - він пішов на нову роботу, його ніхто права на працю не позбавляє

\iusr{Ganna Bezgina}
\textbf{Олександр Рудоманов} до декрету можуть звільнити жінку.

\iusr{Ganna Bezgina}
\textbf{Bogdan Gerasymchuk} почитайте не радянські закони Европи.

\iusr{Катя Кондратюк}
\textbf{Ganna Bezgina} «тоді працівник практично позбавляється своїх прав. Роботодавець матиме право без будь якої причини звільнити працівника.»
Ви хоч задумуєтесь, що кожна двостороння угода має бути добровільною і рівноправною. Це основа римського права і кодексу Наполеона. Якщо працівник має право легко звільнитись, таке ж право має бути і роботодавця.

\iusr{Катя Кондратюк}
\textbf{Денис Милий} ну шо за... так само, як держава гарантує право на працю, так само гарантує право на підприємництво і найм. Але право - це не гарантія. Неможливість законно звільнити працівника - це якби покупця змушували купувати ті самі продукти щоразу в магазині, бо він колись мав необережність їх купити, адже держава також захищає право на підприємницьку діяльність. Захищати радянський кодекс, який регулював стосунки між держмонополією і фактично рабами і натягувати його на ринкову економіку - це таке нерозуміння природи добровільності договору і рівноправності сторін, що аж страшно(

\iusr{Катя Кондратюк}
\textbf{Ganna Bezgina} для фопа нема.

\iusr{Bogdan Gerasymchuk}
\textbf{Ganna Bezgina} я живу в Україні , і вимушений жити за реаліями що існують , що мені дадуть закони Европи , за мого життя я сумніваюсь що ми зможемо називатись европейцями . Та й я вважаю , що в кожній країні повинні бути свої адекватні правила , навіщо тупо копіювати щось і в когось . То не є панацеєю, доказано Німеччиною з їхнім «утриматись від критики північного потоку 2»

\iusr{Ganna Bezgina}
\textbf{Katya Kondratyuk} всьому свій час. Те що зараз вам здається дуже легким і правильним з часом зміниться. Я не буду з вами сперечатись, я як людина в віці знаю в яких ситуаціях може опинитись людина.

\iusr{Ganna Bezgina}
\textbf{Bogdan Gerasymchuk} при чому тут трудове право до Північного потоку? І захист робітника це не мінус.

\iusr{Катя Кондратюк}
\textbf{Ganna Bezgina} якраз проблема в тому, що час змінився, а кодекс - ні. Те, що писалось для індустріальних трудових відносин в умовах державної монополії, не працює в ринковій постіндустріальній економіці. Не думаю, що ви старші, за кодекс, тому всі випадки несправедливості якраз на совісті утопічного кодексу.
Ок, хочете, щоб людину не міг звільнити роботодавець, тоді ви мали б топити за те, що і людина не може звільнитись коли їй заманеться. Ринкова економіка працює на принципах взаємовигоди і добровільності. Ну не можуть марксистські принципи, закладені в кодексі, працювати в ринковій економіці!

\iusr{Ganna Bezgina}
\textbf{Katya Kondratyuk} назвіть мені будь ласка цивілізовану країну де роботодавець може просто звільнити людину.

\iusr{Катя Кондратюк}
\textbf{Ganna Bezgina} США.

\iusr{Ganna Bezgina}
\textbf{Katya Kondratyuk} ну це не правда. Там теж укладаються трудові угоди. Існує захист профспілковий. І попереджувати про звільнення у більшості випадків потрібно за 60 днів. Луснувши пальцями там теж не звільниш.

\iusr{Катя Кондратюк}
\textbf{Ganna Bezgina} не маніпулюйте, ніхто не казав, що не потрібно попереджати. Чи не потрібно угод.
Профспілковий захист - це де? В Детройті? Захистили? Подобається результат? Якось в силіконовій долині всі ці смузі бари і столи для тенісу працівникам з’явились не завдяки профспілкам, а завдяки ринку.
Була б можливість попередити і звільнити як в США, цієї дискусії б не було. Саме тому в нас і є ця прекрасна практика разом з заявою на прийом брати підписану заяву на звільнення. Це лицемірство, коли навіть в Адміністрації президента є така практика, а лохторату розказують про захист пролетаріату.

\iusr{Катя Кондратюк}
\textbf{Денис Милий} «Право на працю є одним з базових прав людини. Держава створює умови для його реалізації та захищає. Не подобається це роботодавцю - він завжди може припинити діяльність зовсім. Це ж добровільна справа, на відміну від дотримання трудового законодавства.» це тому адміністрація президента сама практикує наперед підписані заяви про звільнення?

\iusr{Inna Zvyagintseva}
\textbf{Олександр Рудоманов} Жінка має право на декрет за певний час до пологів, здається за 75 днів. Є чимало роботодавців, які воліють позбутися жінки на більш ранньому терміні вагітності

\end{itemize} % }

\iusr{Михаил Дикун}
Такі зміни і не ухвалять, як мінімум у найближчій перспективі, бо у нас досі всі граються у соціалізм

\begin{itemize} % {
\iusr{Олександр Рудоманов}
\textbf{Михайло Дикун} та наче пробують, але проблема в тому, що основний виборець - соціаліст по факту, але різні переконання ідеології

\iusr{Михаил Дикун}
\textbf{Олександр Рудоманов} отож, і такі закони - очевидні електоральні втрати

\iusr{Yurii Lev}
\textbf{Олександр Рудоманов} в конституції записано, що ми соціальна країна ...

\iusr{Alena Morgan}
\textbf{Михаил Дикун} весь цивілізований світ загрався в соціалізм!!)))
\end{itemize} % }

\iusr{Алексей Задорожный}

Я вообще не понимаю этого идиотизма. И профсоюзы тоже не перевариваю.
Руководство любого предприятия должно иметь полное право самостоятельно решать,
какие люди на них работают и на каких условиях. А сотрудник либо соглашается на
условия работодателя, либо ищет другого.

\begin{itemize} % {
\iusr{Роман Химич}
\textbf{Алексей Задорожный} сразу видно, что вы настоящий европеец!

\iusr{Катя Кондратюк}
\textbf{Roman Khimich} а справжня Європа будувалась на соціалізмі?

\iusr{Роман Химич}
\textbf{Катя Кондратюк} давайте уточнимо часові проміжки побудови, аби не використовувати поняття соціалізм до часів феодалізму, наприклад

\iusr{Катя Кондратюк}
\textbf{Roman Khimich} та будь який, коли економіка європейських держав була першою в світі. Коли темпи росту були не в межах статистичної похибки.

\iusr{Роман Химич}
\textbf{Катя Кондратюк} після ПСВ Стара Європа навернулася до соціалізму. Після ДСВ він став домінуючим напрямком. Соціальна держава, держава загального благополуччя грунтувалися саме на соціалістичних підходах

\iusr{Катя Кондратюк}
\textbf{Roman Khimich} там у всіх по-різному) соціалістичною стала тільки російська імперія. Найдовше противилась соціалізму Великобританія. І це прямо корелює з добробутом населення цих країн

\iusr{Роман Химич}
\textbf{Катя Кондратюк} мабуть варто визначити, що саме ми розуміємо під соціалізмом.
Про ВБ дуже дивно чути, 50-70 роки це типовий соціалізм, купа держпідприємств, височезні податки і т.і.

\iusr{Катя Кондратюк}
\textbf{Roman Khimich} а потім приходить Маргарет Теччер) там хоч чергувались ліві і консерватори. Тому так, вб найменш ліва. Та ж Італія куди лівіша...

\iusr{Роман Химич}
\textbf{Катя Кондратюк} прийшли Тетчер і що? Ковід продемонстрував, що СОЗ ВБ один з найслабкіших в Європі, на рівні отієї Італії. Про ФРН чи Північну Європу годі й казати

\iusr{Катя Кондратюк}
\textbf{Roman Khimich} до чого тут ковід? Тоді за вашою логікою, Куба справилась, яка лікарів, як рабів, віддає в користування іншим країнам, а оплату забирає собі. Якраз в Італійська медицина найменше постраждала від лівацтва. Лікуватись там дуже дорого, багато приватної медицини... до того ж соз не один ковід характеризує, є ще смертність, заходи профілактики тощо

\iusr{Роман Химич}
\textbf{Катя Кондратюк} я уникаю дискусій, в яких використовуються кліше на кшталт "лівацтво", "фашизм", "вата" і т.і. Не обговорюю чиюсь Віру Істинну і т.і.
Якщо Куба демонструє високі показники суспільного здоровя, значить вона справляється краще за країни, в яких ці показники нижче. Якою ціною - окреме питання.
Наведіть, будь ласка, дані, які характеризують стан СОЗ, подивимось разом
\end{itemize} % }

\iusr{Марфа Соломенцева}

Я таку тітоньку, навіть після реорганізації, мусила відновити на роботі через
12!!! років. Вона виграла всі суди, а тому числі Європейський. І отримала
компенсацію від України.

\begin{itemize} % {
\iusr{Olga Kakaulina}
\textbf{Marfa Solomenceva} в тому то і справа, що завжди в таких випадках порушень процедури вдосталь, для відновлення. Взяти приклад із звільненням поліцейських, які через суд відновились на посадах. А потім ще й платники податків виплачують компенсацію.


\iusr{Олександр Рудоманов}
\textbf{Olga Kakaulina} один з методів в УЗ, це домовленість з кадровиками, щоб вас не правильно звільнили  @igg{fbicon.smile} 

\iusr{Olga Kakaulina}
\textbf{Олександр Рудоманов} так таке не тільки там, на жаль.


\iusr{Олександр Рудоманов}
\textbf{Olga Kakaulina} в приватному бізнесі теж


\iusr{Олександр Рудоманов}
?

\iusr{Olga Kakaulina}
\textbf{Олександр Рудоманов} за приватний не скажу, бо не знаю. А ось держструктури, то стандартна практика((((

\iusr{Юрий Белоусов}
Мабуть погано звільняли. До речі, і цю падругу також звільняють з порушеннями, з неймовірно низької якості наказом, з яким її попереджають. І тут звичайно може бути і сидіння в центрі зайнятості на допомозі до 2 років, і поновлення на роботі, в тому числі через Європейський суд

\iusr{Alexander Kleimenov}
\textbf{Marfa Solomenceva} а якщо б ця людина була не ватою, а викривала б корупцію на факультеті чи у виші, її б роботодавець (якщо виш державний, то держава?) теж мав би право звільнити на раз-два?


\iusr{Олександр Рудоманов}
\textbf{Юрій Білоусов} чому?

\iusr{Марфа Соломенцева}
\textbf{Alexander Kleimenov} не знаю про раз-два, підстава ж має бути. До речі, в державному органі це все і було. А про корупцію, то з моєї кафедри викладача впіймали на хабарі і засудили умовно. Він потім знов працював через декілька років

\iusr{Ірина Єгорченко}
\textbf{Alexander Kleimenov} тоді б точно звільнили б на раз два без розмов. І без порушень щоб жоден суд не відновив

\iusr{Ірина Єгорченко}
Ту ж Пархоменко звільняли з кулька методом ліквідації факультету, суд присудив їй якісь гроші невеликі і все

\iusr{Роман Химич}
\textbf{Alexander Kleimenov} так, звичайно.

\end{itemize} % }

\iusr{Halyna Slotska}

Та так не тільки у Союзі, так у цілому світі. Звільнення з постійної роботи йде
через суди і виплати компенсацій. А як провести різницю між неугодним
працівником і « дурачком». А якщо неефективність буде зумовлена виконанням
заздалегідь важких завдань, потребуючих часу. Наскільки розумію мова йде про
людину, яка вела себе публічно неетично на займаній посаді, порушуючи
конституцію та закони Украіни. От це і є величезною підставою на звільнення.
Питання, чому суд не використовує такі пункти, в чому причина?

\begin{itemize} % {
\iusr{Олександр Рудоманов}
\textbf{Галина Слоцька} політичні погляди або відсутність рішення суду по криміналу не є підставою для звільнення.
Чому, якщо роботодавець дає роботу, то він ще має думати, як не звільняти того, хто йому не приносить гроші?

\iusr{Halyna Slotska}
\textbf{Олександр Рудоманов} Якщо я не помиляюся там мова йде про ту дивну викладачку?


\iusr{Олександр Рудоманов}
\textbf{Галина Слоцька} так, але по ній немає рішення суду.
Тому лише такий мезанізм(

\iusr{Halyna Slotska}
Якщо чесно, то таких людей можна звільнити тільки через реорганізацію. Найоптимальніший вихід. Питання до тих хто її влаштував на цю посаду.

\iusr{Дмитрий Мукойда}
\textbf{Олександр Рудоманов} потому что ты сам можешь создать условия для непродуктивности сотрудника. И что сам подставил сам уволил?
\end{itemize} % }

\iusr{Inna Zvyagintseva}

Для посади зав. кафедри є ряд вимог. Якщо викладачка посідала цю посаду, роблю
висновок, що керівництво вишу вона влаштовувала. Але трапився скандал з ватними
поглядами... Невже в когось є сумніви, що на подібну посаду в пост-радянському
виші не призначають за прозорим конкурсом?

\begin{itemize} % {
\iusr{Julia Zinovjeva}
\textbf{Inna Zvyagintseva} вона ж ніби завкафом не була, я просто багато років вела практику у студентів з тої кафедри, ну то ти, певно, деякі перли читала в ФБ, це було важко, і студентів шкода, але одна викладачка зрозуміла, що практика трохи спасає загальний рівень і дуже просила мене виділяти гьдину на тиждень на гуляння музеєм...Там і погляди ватні, і рівень викладання низький
\end{itemize} % }

\iusr{Дмитрий Бондаренко}
То я колись на шахті працював, там не могли звільнити курв з відділку, який самоспаси з жетонами видають, бо вони відмічали у системі, що людина поїхала у шахту, а по факту воно бухає лежить десь і зп йому тікає.
З останнього що я чув, баба через пів року поновилась через суд з виплатою зп за весь час і ще компенсацію вибила.

\iusr{Sergei Kovalchuk}
Ну да ? 3 письма о нарушениях ,опозданиях и прочее , и увольняетса легко по статье с записью в трудовой. Кадровики просто ленивые или кумовство. Кзот никто не отменял.

\iusr{Андрій Чвалюк}
Ліквідація кафедри також не панацея, якщо було допущено порушення численних норм трудового зак-ва.

\iusr{Юрий Белоусов}
В сенсі? А які підстави звільнення Ви бачите в цьому конкретному випадку? Що вона не так робить як викладач?

\end{itemize} % }
