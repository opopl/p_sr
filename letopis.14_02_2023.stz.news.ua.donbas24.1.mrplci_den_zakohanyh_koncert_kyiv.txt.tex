% vim: keymap=russian-jcukenwin
%%beginhead 
 
%%file 14_02_2023.stz.news.ua.donbas24.1.mrplci_den_zakohanyh_koncert_kyiv.txt
%%parent 14_02_2023.stz.news.ua.donbas24.1.mrplci_den_zakohanyh_koncert_kyiv
 
%%url 
 
%%author_id 
%%date 
 
%%tags 
%%title 
 
%%endhead 

Ольга Демідко (Маріуполь)
14_02_2023.olga_demidko.donbas24.mrplci_den_zakohanyh_koncert_kyiv
Маріуполь,Україна,Мариуполь,Украина,Mariupol,Ukraine,Київ,Киев,Kyiv,Kiev,Concert,Music,Концерт,Музика,Музыка,date.14_02_2023

Для маріупольців до Дня закоханих провели концерт у Києві (ВІДЕО)

Для переселенців з Маріуполя підготували особливий подарунок

13 лютого у Києві переселенці з Маріуполя до Дня Святого Валентина отримали
особливий подарунок — для них підготували святковий концерт, який відбувся у
Будинку вчених. Це був дуже душевний вечір, який дозволив кожному маріупольцю
поринути в затишну та романтичну атмосферу.

Читайте також: У музеї Нью-Йорка визнали Куїнджі українським художником

Організаторкою святкового заходу стала директорка Департаменту
культурно-громадського розвитку Маріупольської міської ради Діана Трима. Вона
наголосила, що вона вважала своїм обов'язком підготувати концерт для
маріупольців з нагоди цього чудового свята.

«Наразі у столиці проживає понад 17 тисяч маріупольців. Ми щороку перед
повномасштабним вторгненням рф проводили у місті святкові заходи і цього року
хотілося також порадувати наших людей. Цей концерт був плановим. Намагалися
підготувати програму з того, що мали. Зокрема, в нас вже є сформований театр,
є свої музиканти. Водночас запросили спеціальних гостей», — розповіла Діана
Трима.

Концерт відкрили юні маріупольці — виконали танок, створений в межах програми
Культури Маріуполя із соціальної реабілітації. Протягом всього заходу
неповторну атмосферу створювала музика Ігоря Трими. Актори Маріупольського
театру авторської п'єси «Conception» прочитали низку віршів українських поетів,
зокрема Ліни Костенко, Василя Симоненка, Сергія Жадана та Юрія Іздрика. Через
рядки віршів відомих українських класиків вони закликали вміти в цей непростий
час віддавати любов, кохати і бути коханими.

Читайте також: Фільм про Маріуполь отримав нагороду на американському кінофестивалі Sundance

Організаторкою та ведучою на концері була Діана Трима. До речі, саме їй вдалося
запросити спеціального гостя на захід — рок-гурт «Кам’яний гість». Артисти
виконали шість власних композицій, зокрема інструментальну, а також заспівали
українською пісню Боба Ділана «Blowin 'in the Wind».

«Я з цим чудовим колективом познайомилася випадково, але дуже рада, що вони
долучилися до нашого концерту. Також у нас з гуртом „Кам'яний гість“
заплановано декілька заходів і в майбутньому. Зокрема, гурт виступить і в
театралізованій виставі, яку незабаром зможуть побачити маріупольці», —
поділилася Діана Трима.

«Ми раді бути на цьому заході, раді підтримати кожного маріупольця. Впевнений,
що згодом Маріуполь стане найбільш сучасним і красивим містом в Україні. Просто
потрібно набратися терпіння і ще трохи почекати на деокупацію», — зазначив
вокаліст гурту Юрій Велес.

Читайте також: Фотографи показали життя українських військових поруч із
позиціями росіян (ФОТО)

На завершення заходу прозвучав вірш-молитва у виконанні відомої української
акторки, телеведучої та громадської активістки Римми Зюбіної. Вона прочитала
рядки полоненої захисниці «Азовсталі» Валерії Карпиленко та розповіла про
трагічну історію кохання мужньої маріупольчанки. Крім того, Римма розповіла про
своє життя під час війни, згадала, як приїздила до Маріуполя до
повномасштабного вторгнення і наголосила, що сьогодні кожне місто України
вважає своїм рідним містом.

«Ми нація сильних, незламних, неймовірно красивих своєю душею людей. І я дуже
рада, що сьогодні тут з вами», — зазначили Римма Зюбіна.

Акторка запросила всіх на свої вистави і пообіцяла всім охочим маріупольцям
подарувати запрошення. Також вона поділилася своїм номером телефона і
наголосила, що в такі складні часи ми тільки разом зможемо все подолати.

Читайте також: «Фортеця Бахмут»: гурт «Антитіла» представив знятий на передовій
кліп (ВІДЕО)


Самих же маріупольців концерт дуже надихнув і подарував надію, що навіть після всього пережитого життя може поступово налагодитися.

«Я дуже вдячна всім організаторам цього чудового концерту за ту атмосферу, яку
вдалося створити. У мене посмішка не сходила з обличчя. Всі виступи були
настільки щирими, що я на якусь мить забула, що в країні війна і я втратила
домівку та стала переселенкою. Дякую і маріупольським акторам, і гурту
"Кам'яний гість", і талановитій Риммі Зюбіній та всім організаторам концерту за
такий незабутній подарунок», — поділилася враженнями маріупольчанка Катерина
Онищенко.

Раніше Донбас24 розповідав, що у Ганновері відбулася прем'єра документальної
вистави «Театроманії 2:0».

Ще більше новин та найактуальніша інформація про Донецьку та Луганську області
в нашому телеграм-каналі Донбас24.

ФОТО: Донбас24
