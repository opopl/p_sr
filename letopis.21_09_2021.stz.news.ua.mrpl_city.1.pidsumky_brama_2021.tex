% vim: keymap=russian-jcukenwin
%%beginhead 
 
%%file 21_09_2021.stz.news.ua.mrpl_city.1.pidsumky_brama_2021
%%parent 21_09_2021
 
%%url https://mrpl.city/blogs/view/pidsumki-festivalyu-teatralna-brama2021
 
%%author_id demidko_olga.mariupol,news.ua.mrpl_city
%%date 
 
%%tags 
%%title Підсумки фестивалю "Театральна брама – 2021"
 
%%endhead 
 
\subsection{Підсумки фестивалю \enquote{Театральна брама – 2021}}
\label{sec:21_09_2021.stz.news.ua.mrpl_city.1.pidsumky_brama_2021}
 
\Purl{https://mrpl.city/blogs/view/pidsumki-festivalyu-teatralna-brama2021}
\ifcmt
 author_begin
   author_id demidko_olga.mariupol,news.ua.mrpl_city
 author_end
\fi

Другий обласний відкритий фестиваль театрального мистецтва Театральна брама –
2021 завершився і час дізнатися його результати. Я зібрала думки та рішення як
експертного журі, так і керівництва маріупольського драматичного театру.

\ii{21_09_2021.stz.news.ua.mrpl_city.1.pidsumky_brama_2021.pic.1}

У цьому фестивалі Донецький академічний обласний драматичний театр (м. Маріуполь)
за свою виставу \enquote{Маруся} за Л. Костенко, режисерки-постановниці, заслуженої
артистки України\par\noindent\textbf{Людмили Колосович} отримав найвищу нагороду – \emph{гран-прі}. \textbf{Віра
Шевцова} за роль Марусі Чурай була відзначена в номінації \emph{\enquote{За високохудожнє
створення образу}}.

\ii{21_09_2021.stz.news.ua.mrpl_city.1.pidsumky_brama_2021.pic.2}

Генеральний директор маріупольського драматичного театру \textbf{Володимир Кожевніков}
наголосив,  що всі члени журі були сильно вражені саме виставою \enquote{Маруся}, яку
показали в останній день фестивалю. Зокрема, \textbf{Валерій Гуйвін}, художній керівник
акторсько-режисерського курсу \enquote{Майстерня 55} Харківського національного
університету мистецтв імені І. П. Котляревського, відзначив цей спектакль і
наголосив, що він є найпотужнішим з усього фестивального репертуару. Приємно
вразила його і гра Віри Шевцової та всієї театральної трупи, яка помітно
виросла. А театрознавиця \textbf{Олена Либо} поділилася, що вона дуже рідко плаче на
виставах, але на цій – стримати себе було важко.

\ii{21_09_2021.stz.news.ua.mrpl_city.1.pidsumky_brama_2021.pic.3}

Також журі відзначили й іншу роботу Донецького академічного обласного
драматичного театру (м. Маріуполь) рок-оперу \enquote{Біла Ворона}
(режисерка-постановниця \textbf{Анжеліка Добрунова}), якою фестиваль будо відкрито.
Насправді багато маріупольців вболівали саме за цей спектакль, адже він давно
для багатьох містян став найулюбленішим. Втім і цю виставу не залишили без
номінацій. Так актриса маріупольського театру \textbf{Дар'я Недавня} за талановите
виконання ролі Жанни д'Арк отримала перемогу в номінації \emph{\enquote{За високохудожнє
виконання ролі}}. Народна артистка України, театрознавиця та членкиня журі
\textbf{Світлана Отченашенко} не могла оцінювати спектаклі свого театру, але зі мною
поділилася, що під час вистави  \enquote{Біла ворона} вона \emph{ловила себе на думці, що
щаслива бути частиною такого талановитого і потужного театру}. Вона наголосила,
що вся трупа \emph{працювала на максимальному видиху}.

\ii{21_09_2021.stz.news.ua.mrpl_city.1.pidsumky_brama_2021.pic.4}

Експертне журі одноголосно вирішило, що цьогорічний фестиваль переріс статус
обласного і наблизився до всеукраїнського. Генеральний директор Донецького
академічного обласного драматичного театру (м. Маріуполь) \textbf{Володимир Кожевніков}
зауважив, що у приїжджих колективів, звісно, є чому повчитися. Водночас він
пишається роботою всіх театральних цехів свого театру, які змогли впоратися з
великою кількістю задач і приємно вразити колективи з інших міст своєю
ефективною роботою.  Представники журі наголосили на актуальності всіх
запропонованих вистав та висловили сподівання, що фестиваль стане щорічним.
Зокрема, член журі, театральний критик та історик театру \textbf{Дмитро
Єрмолович-Дащинський} зауважив, що фестиваль \enquote{Театральна брама} може стати
справжньою перлиною не тільки Донеччини, а й всієї України. А народна артистка
України \textbf{Світлана Отченашенко} висловила вдячність маріупольському глядачу за
його постійну присутність протягом всього фестивалю і зазначила, що має надію,
що цей фестиваль стане  в Маріуполі доброю традицією.

\ii{21_09_2021.stz.news.ua.mrpl_city.1.pidsumky_brama_2021.pic.5}

Виконавці головних ролей у виставах, які сподобалися маріупольським глядачам
найбільше, серед них і \enquote{Маргарита й Абульфаз} за С. Алексієвич (Львівський
академічний обласний музично-драматичний театру імені Юрія Дрогобича), \enquote{Баба
Пріся або на початку і наприкінці часів} П. Ар'є (Херсонський обласний
академічний музично-драматичний театр імені Миколи Куліша), \enquote{Хлібне перемир'я}
С. Жадана (Харківський державний академічний україн\hyp{}ський драматичний театр ім.
Т. Г. Шевченка) та інші були відзначені журі в номінаціях \enquote{За високу
майстерність, високопрофесійний акторський ансамбль, творчий потенціал та
високохудожнє створення образу}. Керівництво маріупольського драматичного
театру дуже сподівається, що цей фестиваль протягом наступних років стане ще
більш масштабним.

\ii{21_09_2021.stz.news.ua.mrpl_city.1.pidsumky_brama_2021.pic.6}

Зокрема, заступниця генерального директора з організації глядачів \textbf{Інна Мащенко}
зауважила, що в Маріуполі цілком можливо проводити фестиваль не тільки
всеукраїнського, але й міжнародного рівня. А членкиня журі та театрознавиця
\textbf{Олена Либо} підкреслила, що наступного року фестиваль повинен стати більш
розгалуженим і включати в себе соціальні, культурно-мистецькі, освітні та
інклюзивні проєкти. Особливо це важливо, на думку театрознавиці, для нашої
території. Важливими є і різноманітні виставки, воркшопи та мистецькі
лабораторії, які допоможуть остаточно вийти за рамки обласного фестивалю.

Загалом другий обласний відкритий фестиваль театрального мистецтва Театральна
брама – 2021 дав можливість маріупольському драматичному театру успішно заявити
про себе. За прогнозами експертів, відтепер маріупольці дуже швидко зможуть
долучатися до всеукраїнських та міжнародних фестивалів, адже такий потужний
колектив захочуть бачити  в своїх стінах й інші театри.

Новий театральний сезон у Донецькому академічному обласному драматичному театрі
(м. Маріуполь) розпочнеться 23 жовтня виставою, що \emph{отримала гран-прі, \enquote{Маруся}}.

\emph{Автор всіх фото з вистав – Лев Сандалов}
