% vim: keymap=russian-jcukenwin
%%beginhead 
 
%%file 20_03_2023.fb.fb_group.mariupol.pre_war.1.fotoprogulyanka_v_d_.cmt
%%parent 20_03_2023.fb.fb_group.mariupol.pre_war.1.fotoprogulyanka_v_d_
 
%%url 
 
%%author_id 
%%date 
 
%%tags 
%%title 
 
%%endhead 

\qqSecCmt

\iusr{Екатерина Костенко}

Як же гарно було,в нашому рідному місті 💔

\iusr{Ludmilla Kastkina}

Будто в свои школьные годы окунулась- 60-е!

Вот уж спасибо, так спасибо!

\iusr{Татьяна Шумейко}

Красивый старый город!

\iusr{Руднєв В'ячеслав}

як хочеться додому...

\iusr{Tatyana Scherban}

Фото старого города так и манят пройтись по его улочкам. Развалины синагоги это
как музейная редкость, всегда восхищали своей тайной. Школьники, познавая
город, любили фотографироваться в этом месте

\iusr{Лариса Кириченко}

Спасибо огромное за подборку фотографий! На одной из них - дом моей бабушки на
улице Семенишина. Здесь прошло все мое детство. А в доме во дворе по соседству
со зданием краеведческого музея родился и вырос мой отец. Все родное и знакомое
до слез.

\iusr{Валентина Пижова}

Добрые воспоминания о старом Мариуполе! Так хочется пройтись по этим улочкам! Сердце рвется на части

\iusr{Svetlana Kudrina}

Самые любимые места в городе. Каждое лето устраивала вечерние прогулки по Старому городу.

\iusr{Гладкова Татьяна}

А ведь ходили столько раз по этим улицам и не замечали их уюта и красоты.(( До
сих не пришло понимание, что всего этого больше нет.

\begin{itemize} % {
\iusr{Алена Браун}
\textbf{Гладкова Татьяна} замечали !!!
\end{itemize} % }

\iusr{Sergei Hazbowski}

Мій рідний дім, все життя пройшло. 😭😭😭😭😭😭😭

\iusr{Алена Браун}

Весна, как в городе побывала, спасибо !!

\iusr{Alena Hriy}

Дякую ❤️

\iusr{Alesya Alesya}

Спасибо 🙏 Так хочется домой.

\iusr{Michael Israel Goldin}

Родные улицы. Уехал из Мариуполя 13.5 лет назад. Как все в памяти свежо.
