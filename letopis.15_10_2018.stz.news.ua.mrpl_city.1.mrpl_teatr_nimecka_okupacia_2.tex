% vim: keymap=russian-jcukenwin
%%beginhead 
 
%%file 15_10_2018.stz.news.ua.mrpl_city.1.mrpl_teatr_nimecka_okupacia_2
%%parent 15_10_2018
 
%%url https://mrpl.city/blogs/view/mariupolskij-teatr-v-roki-nimetskoi-okupatsii-chastina-2
 
%%author_id demidko_olga.mariupol,news.ua.mrpl_city
%%date 
 
%%tags 
%%title Маріупольський театр в роки німецької окупації. Частина 2.
 
%%endhead 
 
\subsection{Маріупольський театр в роки німецької окупації. Частина 2.}
\label{sec:15_10_2018.stz.news.ua.mrpl_city.1.mrpl_teatr_nimecka_okupacia_2}
 
\Purl{https://mrpl.city/blogs/view/mariupolskij-teatr-v-roki-nimetskoi-okupatsii-chastina-2}
\ifcmt
 author_begin
   author_id demidko_olga.mariupol,news.ua.mrpl_city
 author_end
\fi

\emph{Продовжимо розглядати особливості розвитку театральної культури Маріуполя в роки німецької окупації.}

З липня 1942 р. репертуар Маріупольського театру ім. Т. Шевченка поповнився
українськими водевілями: \enquote{На перші гулі}, \enquote{День правди},
\enquote{По ревізії}, \enquote{Кум і мірошник}. Незважаючи на появу нових
вистав, брак модерного театрального репертуару давався взнаки. Саме тому
маріупольські режисери прагнули до різноманіття програми.

\ii{15_10_2018.stz.news.ua.mrpl_city.1.mrpl_teatr_nimecka_okupacia_2.pic.1}

На окупованих територіях почалося \emph{національне українське відродження}, виникали
культурні установи, неможливі за радянської влади. У червні 1942 р. поновлює
свою роботу Маріупольське Товариство \enquote{Просвіта}. \emph{Андрій Ірій (Авраменко)} стає
її керівником, водночас продовжує українізувати Маріупольський театр ім. Т.
Шевченка, вступає до ОУН. З огляду на обставини свого заснування, \enquote{Просвіти}
мусили декларувати лояльність до нового режиму і неприйняття радянської влади.

\textbf{Читайте також:} \emph{Маріупольський театр в роки німецької окупації. Частина 1.}%
\footnote{Маріупольський театр в роки німецької окупації. Частина 1., Ольга Демідко, mrpl.city, 07.10.2018, \url{https://mrpl.city/blogs/view/mariupolskij-teatr-v-roki-nimetskoi-okupatsii-chastina-1}} %
\footnote{Internet Archive: \url{https://archive.org/details/07_10_2018.olga_demidko.mrpl_city.mrpl_teatr_nimecka_okupacia_1}}

Маріупольське Товариство \enquote{Просвіта} мало свою капелу бандуристів та духовий
оркестр. З дозволу та під наглядом німецьких окупаційних властей, які
стверджували репертуар, просвітяни давали концерти в Північному Приазов'ї. 22 і
23 серпня 1942 р. пройшли концерти бандуристів \enquote{Просвіти} в клубах радгоспів
\enquote{Зірка} та ім. Ворошилова. Виконали близько 30 українських пісень. На концертах
були присутні місцеві жителі, офіцери і солдати окупаційної влади. Для артистів
і офіцерів місцева влада після концерту влаштовувала урочисту вечерю.

Драматична секція маріупольської \enquote{Просвіти} мала неабиякий вплив на розвиток
театрального мистецтва регіону. Режисер Маріупольського театру ім. Т. Шевченка
та керівник драматичної секції \enquote{Просвіти} Софія Зав'ялова активно працювала над
питанням утворення професійних драматичних кадрів. У своїй статті вона
вказувала на головні проблеми й недоліки театральної справи. Для їх усунення С.
Зав'ялова вважала за потрібне створити драматичний, режисерський і
художньо-технічний відділи для підготовки артистів, режисерів і художніх
техніків сцени (декораторів, бутафорів, електротехніків тощо). Не менш
важливими завданнями на її думку були підвищення кваліфікації для самодіяльної
художньої роботи аматорів та допомога сільським \enquote{Просвітам} у їх драматичній
роботі.

Незабаром у міській пресі повідомлялось, що завдяки роботі \enquote{Просвіти} в
Маріуполі відроджується українська національна культура. Репертуар драматичних
гуртків \enquote{Просвіти} складався з п'єс української класики. 5 вересня 1942 р. у
Міському саду \enquote{Просвіта} влаштувала великий концерт. Взяли участь артисти,
балет театру ім. Шевченка, хор бандуристів \enquote{Просвіти}. Грали 2 духових
оркестри.

Завдяки діяльності драматичного гуртка Товариства \enquote{Просвіта} у 1943 р. було
поставлено кілька вистав у с. Володарському, зокрема \enquote{Доки сонце зійде, роса
очі виїсть}, \enquote{Безталанна}, \enquote{Лимерівна}, \enquote{Сватання на Гончарівці}, \enquote{Пошились у
дурні}, \enquote{На перші гулі}, а також ряд водевілів. На високому рівні було
проведене Шевченківське свято. Активно проводилася культосвітня робота в селах
Темрюці, Петрівці, Чердакли (де є навіть свій духовий оркестр) та в
Мало-Янісолі.

Вагоме місце в роботі колективу театру посідали виступи акторів у міській
радіомережі. Так, 21 листопада 1942 р. місцевий журналіст зазначав: 

\begin{quote}
\em\enquote{Театр дав
цілу низку тематичних літературно-музичних передач: \enquote{Лірика українського
народу}, \enquote{Тарас Шевченко}, \enquote{Шевченківські вірші та пісні} та інші. У цих
передачах брали активну участь артисти В. Волгрик, М., поет, Харута, Харитонов
та піаністка Л. Тімен}. 
\end{quote}

У квітні 1943 р. міський театр, продовжуючи поповнення свого репертуару,
розпочав роботу над п'єсою І. Карпенко-Карого \enquote{Безталанна}. Режисер постановки
С. Зав'ялова ролі розподілила так, що театр мав змогу водночас з роботою над
\enquote{Безталанною} продовжувати свою працю над \enquote{найгрунтовнішою} п'єсою в репертуарі
цього сезону – драмою Ф. Шилера \enquote{Коварство і любов}, яку ставив головний
режисер театру.

\textbf{Читайте також:} 

\emph{Мариупольцев поздравили с Днем Защитника Украины и Покровом с большого экрана у Драмтеатра}%
\footnote{Мариупольцев поздравили с Днем Защитника Украины и Покровом с большого экрана у Драмтеатра, Анастасія Папуш, mrpl.city, %
14.10.2018, \url{https://mrpl.city/news/view/mariupoltsev-pozdravili-s-dnem-zashhitnika-ukrainy-i-pokrovom-s-bolshogo-e-krana-u-dramteatra-foto-plusvideo}}

Володарський сільський театр, який складався з акторів-амато\hyp{}рів та місцевої
інтелігенції, за 1943 р. підготував багато нових вистав. Найбільшим успіхом
користувалася комедія \enquote{Пошились у дурні}. Зала театру, що мала всього 200
місць, кожного разу була переповнена глядачами. Гра акторів та музичне
оформлення, що здійснювалося під керівництвом Р. Сліпця, не дивлячись на
сільські умови, відрізнялися великою майстерністю.

Влітку 1943 р. на сторінках \enquote{Маріюпільської газети} повідомлялося: \emph{\enquote{Початок
червня відзначився великим злетом у культурному житті нашого міста}}.
Маріупольці побачили на сцені театру ім. Т. Шевченка класичний твір Ф. Шиллера
– трагедію \enquote{Підступність і кохання} російською мовою. Німецька адміністрація
рекомендувала поставити на сцені театру твір видатного німецького драматурга
Фрідріха Шиллера. Тому театральний колектив під керівництвом головного режисера
І. Тяка до підготовки вистави підійшов дуже серйозно. Півтора місяця актори
вивчали ідеї німецької філософії XVIII століття, занурилися в епоху Ф. Шиллера
та в музикальні твори світових класиків. У цілому вистава отримала схвальні
рецензії. Високою театральною майстерністю відзначалася гра акторів. Разом з
тим не всі актори зразково оволоділи зовнішньою технікою сценічної гри:
нюансами міміки, виразністю руху, мистецтвом тонкої обробки фрази й інтонації.
Помітні були недоліки в мові артистів. Деяким акторам, як зазначав
маріупольський рецензент, треба попрацювати над російською мовою. Але костюми
стали прикрасою вистави. Музичне оформлення збільшило художню цінність
спектаклю. Маріупольський журналіст-критик повідомляв, що \emph{\enquote{трагедія
\enquote{Підступність і кохання} є безперечний крок вперед і не лише в естетичному
розумінні, а і як школа для самих артистів, яка дає і досвід, і натхнення для
дальшої роботи}}.

19 червня 1943 р. театр показав глядачам свою нову роботу: одноактну п'єсу
\enquote{Одна кімната}. Велику роботу провів головний режисер театру І. Тяк. Під його
керівництвом театральна молодь добре підготувалась до своїх ролей, текст п'єси
він поновив і доповнив дотепними експромтами на маріупольську тематику, додав
цікаві мізансцени і дбайливо підібрав декорації і реквізит.

Після п'єси \enquote{Одна кімната} 19 й 20 червня відбувся великий дивертисмент, хор,
балет та сольні виступи. Найкраще враження в дивертисменті залишив український
хор, слабкіше лунала російська народна музика на варіації \enquote{Стенька Разін}.
Найбільш вдалими були українські народні пісні з танцями. Глядачі з великим
зацікавленням слідкували за акробатичними виступами чотирьох атлетів, у тому
числі однієї дівчини, що показала силу, спритність і пластику. Разом з тим
постійні театральні глядачі часто просили поновити програму дивертисменту. 

Деякі з вистав ставилися десятки разів. Наприклад, 1 серпня 1943 р. на сцені
Маріупольського театру ім. Т. Шевченка відбулася п'ятдесята ювілейна вистава
комедії О. Островського \enquote{Без вини винні}. Колектив виконавців своєю грою
створив гармонійне ціле. Художньому оформленню вистави сприяли професійний
оркестр, вдалий грим, костюми і декорації художника А. Федорова. \emph{\enquote{Особливо
ефектна декорація й освітлення останнього акту – місячна ніч, панський сад,
тераса поміщицького будинку, мармурові статуї, квіти, рослини}}, - написав
місцевий критик.

На початку серпня відбулася прем'єра Вар'єте. У місцевій газеті повідомлялося:

\begin{quote}
\em\enquote{глядачі з захопленням спостерігали за солісткою А. Фадєєвою. Велике враження
на публіку справила молода артистка, \enquote{дівчина-каучук}, Н. Тітова зі своїм
партнером В. Афендуловим. Професійну майстерність показали О. Макеєв зі своєю
партнершею Н. Дьяковою в ковбойському танку. Надя Дьякова – наймолодша балерина
(16 років), але виконує всі танки найкраще} (Тополя, 1943: 4). 
\end{quote}

Незважаючи на те, що балетному ансамблю доводилося часто виїжджати на фронт для
обслуговування німецьких солдатів, давати виїзні концерти на периферії, – все ж
таки його успіхи в роботі величезні. На думку критика, балетний ансамбль театру
здібний виконувати як звичайні народні танці, так і класичні, завдяки своєму
керівнику балетмейстеру – К. Вольській. Вона приїхала влітку 1943 р. з
фронтового театру м. Юзівки на запрошення дирекції маріупольського театру.
Вар'єтне мистецтво міського театру ім. Т.  Шевченка давало глядачам кілька
годин розваги – зі смаком підібраних мистецьких номерів, у які найбільшою мірою
входили балет, акробатика, співи, гумор.

\textbf{Читайте також:} \emph{Мариупольцам показали военную технику и рассказали как ее использовать}%
\footnote{Мариупольцам показали военную технику и рассказали как ее использовать, Анастасія Папуш, mrpl.city, 14.10.2018, \url{https://mrpl.city/news/view/mariupoltsam-pokazali-voennuyu-tehniku-i-rasskazali-kak-ee-ispolzovat-fotofakt}}

З початком операції зі звільнення Північного Приазов'я від німецьких окупантів
діяльність Маріупольського театру ім. Т. Шевченка поступово припиняється. 4
вересня 1943 р., коли частини 221-ї стрілецької дивізії визволили селище
Павлопіль, вийшов ос\hyp{}танній номер \enquote{Маріюпільської газети}. Праця в
культурницьких установах за роки окупації автоматично перетворювала людину на
\enquote{ворога радянської влади}. Розуміючи це, чимало працівників театру, Товариства
\enquote{Просвіти} обрали для себе шлях на еміграцію, де пізніше зуміли примножити
здобутки вітчизняної культури.

Таким чином, діяльність Маріупольського театру ім. Т.  Шевченка в умовах
окупаційної влади та суворих реаліях прифронтових територій сприяла розвитку
театрального мистецтва в регіоні. Якщо для окупаційного режиму театр мав
пропагандистське значення, то для мешканців регіону він зміг відродити
українську культуру, задовольняв елементарні потреби населення в інформації,
мистецьких враженнях, психологічному розвантаженні.
