% vim: keymap=russian-jcukenwin
%%beginhead 
 
%%file 12_04_2021.fb.midrussia.1.gagarin_japonia
%%parent 12_04_2021
 
%%url https://www.facebook.com/MIDRussia/posts/3324603584305726
 
%%author 
%%author_id 
%%author_url 
 
%%tags 
%%title 
 
%%endhead 
\Purl{https://www.facebook.com/MIDRussia/posts/3324603584305726}

В мае 1962 года Юрий Гагарин покорил своим обаянием ещё одну мировую столицу –
цветущий Токио. Япония стала девятнадцатой по счёту страной, которую космонавт
посетил в рамках своей грандиозной Миссии мира, призванной открыть для народов
Земли не только космос, но и советского человека. 

🙌 Ажиотаж в связи с приездом Гагарина охватил всю страну. Его имя звучало на
всех телеканалах, не сходило с уст журналистов, учёных и государственных
служащих. Десятитысячная толпа японцев встречала космонавта на аэродроме и
выстраивалась в живую стену вдоль дороги в гостиницу. Ликующие лица, море
цветов, транспаранты со старательно выведенными на них русскими приветствиями –
всё указывало на искренний интерес и симпатию японского народа к первому
человеку в космосе. 


\ifcmt
  pic https://scontent-bos3-1.xx.fbcdn.net/v/t1.6435-9/172156567_3324601177639300_2548528086570452924_n.jpg?_nc_cat=102&ccb=1-3&_nc_sid=730e14&_nc_ohc=fMYH1TVfxqgAX_9pPN5&_nc_ht=scontent-bos3-1.xx&oh=fe46f98ce6a444d9f632074478ce18b8&oe=609901C7

	pic https://scontent-bos3-1.xx.fbcdn.net/v/t1.6435-9/171816193_3324601694305915_999817628282478034_n.jpg?_nc_cat=100&ccb=1-3&_nc_sid=730e14&_nc_ohc=hcszuGeQyncAX-DZVUg&_nc_ht=scontent-bos3-1.xx&oh=57451ffcbd9bdc70823eee28b0a35412&oe=609A3F1D

	pic https://scontent-bos3-1.xx.fbcdn.net/v/t1.6435-9/172183410_3324602057639212_8740625466081490215_n.jpg?_nc_cat=106&ccb=1-3&_nc_sid=730e14&_nc_ohc=nyXCGT-8z2EAX8Mr6YG&_nc_ht=scontent-bos3-1.xx&oh=e14457ea53426bbe97bf26955dbe33b7&oe=609B7E61

	pic https://scontent-bos3-1.xx.fbcdn.net/v/t1.6435-9/172307205_3324602264305858_3359161797695745236_n.jpg?_nc_cat=111&ccb=1-3&_nc_sid=730e14&_nc_ohc=PCb07WRPUOkAX-rXWN2&_nc_ht=scontent-bos3-1.xx&oh=78188cecd67c1475da5fe5ae797fbd31&oe=609A2110

	pic https://scontent-bos3-1.xx.fbcdn.net/v/t1.6435-9/172590404_3324602317639186_3268745833579521286_n.jpg?_nc_cat=105&ccb=1-3&_nc_sid=730e14&_nc_ohc=BYlNHWI0uG8AX_Olors&_nc_ht=scontent-bos3-1.xx&oh=0cedd034a2d2ef6031fafbf6fd0f9443&oe=609A3359
\fi


\begin{itemize}
\item  Неподдельные теплота и восторг, с которыми японский народ встречал
советского героя, не могли не беспокоить США. Чтобы отвлечь внимание
обывателей, американцы запланировали на дни визита Гагарина приезд министра
юстиции США Роберта Кеннеди (брата американского президента). Затея оказалась
бессмысленной: интерес к Кеннеди-младшему не шёл ни в какое сравнение с тем
восхищением, которое японцы демонстрировали в отношении Гагарина.

\item  Юрий Алексеевич, в свою очередь, с большим удовольствием знакомился с
особенностями японской культуры. Судя по кадрам хроники, ему не сразу удалось
приноровиться к деревянным палочкам для еды, зато он охотно примерил подаренное
ему кимоно и уверенно поддержал беседу с приглашёнными на торжественный обед
гейшами. 

\item  Чтобы запечатлеть красоты японской архитектуры и флоры, космонавт приобрёл в
лучшем киноателье фирмы «Минольта» 8-миллиметровую кинокамеру. Невзирая на
возражения советского гостя, хозяева магазина завалили его в ответ фирменными
подарками.

\item  Пообщаться с космическим первопроходцем особенно стремились японские
специалисты в сфере ракетостроения. Интерес развивающейся нации к науке и
технике был огромный, к тому же сам Гагарин, будучи по образованию
лётчиком-инженером, умел профессионально и захватывающе рассказать о любимом
деле. 

\item 🤝 Беседуя с японскими учёными, Юрий Алексеевич выражал горячую убеждённость в
том, что однажды советские и японские исследователи вместе поднимутся в небо на
космических кораблях будущего. Эти слова оказались пророческими: в 1990 году
японский космонавт-исследователь Тоёхиро Акияма стартовал в космос на корабле
«Союз ТМ-11» в составе экипажа восьмой основной экспедиции орбитальной станции
«Мир» под командованием советского космонавта Виктора Афанасьева.

\item 🗻 Помимо японской столицы, за время визита Гагарин успел посетить Осаку,
Киото, Нагойю и Саппоро. В поездках его сопровождали японские дипломаты,
руководители Общества дружбы СССР – Япония и первые лица местной
социалистической партии. Со всеми – начиная премьер-министром Икэдой Хаято и
заканчивая маленькими воспитанниками токийской школы – космонавт был весел и
прямодушен. «Хорошо, хорошо, Гагарин!» – одобрительно звучали в ответ слова
народной песни, которую сочинила окрылённая его подвигом японская молодёжь.

\end{itemize}

Документальный фильм о визите Юрия Гагарина в Японию: 

\begin{itemize}
\item \url{www.youtube.com/watch?v=ENvGNdTLcEU} 
\item \url{https://www.mk.ru/science/2016/04/11/poezdka-pervogo-kosmonavta-v-yaponiyu-byla-triumfalnoy.html}
\end{itemize}

\begin{verbatim}
	#Гагарин1961 #МиссияМира #СоветскийСоюз #СССР 
	#Япония #НашаИстория 
	#GagarinPeaceTour #Vamos60
\end{verbatim}
