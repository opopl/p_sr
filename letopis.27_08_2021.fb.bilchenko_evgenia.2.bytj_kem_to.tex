% vim: keymap=russian-jcukenwin
%%beginhead 
 
%%file 27_08_2021.fb.bilchenko_evgenia.2.bytj_kem_to
%%parent 27_08_2021
 
%%url https://www.facebook.com/yevzhik/posts/4192242484144163
 
%%author Бильченко, Евгения
%%author_id bilchenko_evgenia
%%author_url 
 
%%tags bilchenko_evgenia,chelovek,mysli,zhizn
%%title БЖ. Быть кем-то
 
%%endhead 
 
\subsection{БЖ. Быть кем-то}
\label{sec:27_08_2021.fb.bilchenko_evgenia.2.bytj_kem_to}
 
\Purl{https://www.facebook.com/yevzhik/posts/4192242484144163}
\ifcmt
 author_begin
   author_id bilchenko_evgenia
 author_end
\fi

БЖ. Быть кем-то.

Пишу продолжение моей темы "Быть никем". Кто в сорок лет оказался в глубочайшей
заднице, да прочитает это и, дай Бог, найдет что-то полезное для себя. 

Эмм... Меня удивило, что многие люди великое Ничто воспринимают как анархию.
Дескать, девочка выпала из всех систем и счастлива от своего подросткового
нигилизма. Скажу так. Это крайне глупо. Всё ведь сказано до нас: кто в двадцать
лет не революционер, не имеет сердца, кто в сорок лет революционер не имеет
ума. Моя революция и так слишком затянулась и превратилась в нечто уродливое в
образе цветного бунта. Хорошо, что именно православие (причем, с обрядами, со
всеми делами, да) отучило меня от наивного протеста.

\ifcmt
  pic https://scontent-cdg2-1.xx.fbcdn.net/v/t1.6435-9/240702661_4192237174144694_5269823513441918583_n.jpg?_nc_cat=102&ccb=1-5&_nc_sid=8bfeb9&_nc_ohc=GNKA1r7rW0EAX9vScFj&_nc_ht=scontent-cdg2-1.xx&oh=85f9998e73f83eada1fe5885718ed184&oe=614ECF4E
  width 0.4
	fig_env wrapfigure
\fi

Зато теперь, если уже протест, то Такой, за который и грохнуть могут, а не
тупо-поскакать. Вот она, консервативная революция. Подлинная, нелиберальная,
свобода - эта мега-круто и адски больно. Но "свобода от", как говаривал старик
Ницше, не бывает без "свободы для". Я согласна вторично войти в систему, если
система достойна моего маленького Бога внутри меня. 

Вообще, жить без традиции - это манкуртство. Система ценностей нужна, чтобы за
нее ты был готов умереть. Готов умереть, - будешь жить. Не готов, считай, что у
тебя сладкий леденец вместо жизни: обсосал сам, обсосет и другой.

Логос русской поэзии - система, которую создаю я сама и которая создаёт меня.
Но, как же мне становится горько, когда люди наивно сорадуются моему увольнению
и говорят: "Ты теперь просто поэт". Родня, нет такой профессии. Всё это -
барство, утеха для булкохрустов и чай с боярышником. Поэт - это не ремесло, это
дыхательная функция. От того, что я - столяр, пекарь, врач, учитель, - я не
перестаю дышать и наоборот. С Бродским в этом плане не согласна. Может, я так
себе поэт после этого, не знаю.

Меня воспитала семья заслуженных военных врачей СССР и Украины, меня готовили к
преподаванию и науке, потому я раньше всех защитила докторскую с кандидатской и
учила с 22 лет. Мои первые студенты были старше меня, представьте! У меня в
трудовой книге - всего одна запись. 

Я отличница, несмотря на этот свой вид, и православная русская, несмотря на
радикально постмодерные повадки. Потому меня боятся: привыкли русский мир с
беззащитными старомодными бабулями и дедулями ассоциировать. Делезом  их, да по
пятой точке, Сиддхартха Гаутаму им бы Шакьямуни в оба уха для скорейшей мокши.
Но мы отвлеклись... В общем, бабушка с дедушкой гульбень "простапаэт" не
одобрили бы.

Если короче, к чему я? К тому, что Пустота есть Полнота. Негативная свобода
должна нести позитивную. Никакого анархизма и лени - только долг и труд, вот. Я
сейчас вынуждена выживать. Мне стыдно клянчить деньги на удовольствие "быть
поэтом". Я стала бойцом невидимого литературного фронта. Я обладаю разными
скиллами, чтобы не сдохнуть: редактура, корректура, помощь в написании стихов,
романов и научных текстов, переводы. Все могу. Люди довольны. Мои анонимные
строки диссидента летят по миру под разными именами. Это по-своему волнует. 

Но. Есть одно "но". Чтобы выжить, надо круглосуточно фрилансить. Расплата:
полная потеря ВСЕГО времени, имени ученого и личной реализации как писателя. На
конференции времени так точно нет. Бонус: есть, что кушать. Обидно ли мне? О,
да, обидно. Но это и есть Ничто. Я в труде, я не потеряла трудовой чести,
просто я без самореализации. Для меня это - род пострига, монастырское
испытание на вшивость в лице Эго. Хочу пройти это дао до конца. Кстати, мой
любимый - очень высокого мнения о моем мозге, он полагает, что меня хватит:
делать тексты другим и успевать делать себе.

В связи с этим вопрошаю вас. Мне забирать трудовую книжку из Драгоманова? По
закону мне должны предложить три места. И, да, смешно, но меня они так и не
уволили. Вместо этого ушли с работы кое-какие ярые сторонники травли. Теперь -
время гамбита. Итак, каков ход королевы? Уходить СОВСЕМ или цепляться из
вредности? Где больше Победы Духа? Меня сейчас интересует только Победа Духа,
не выгода: вы ж типа смотрите на БЖ, как меня испытывают на прочность и как я
выстаиваю. Так, как круче выстаивать? Вне их или им назло вернуться? Вам же
нужен предмет веры, что русские не сдаются. Вот я тут не знаю, как
маневрировать. 

Я летом решила уходить. Я немного увлеклась новыми занятиями. Мне даже стало,
прости, Господи, интересно на свободном рынке: там тусят важные светские
господа, которые рабам создают псевдорелигиозную  националистическую
реальность, потому ко мне относятся нормально, только иронично просят не
светить имя. А мне, Иванушке, и ладно. Во всяком случае, эти люди - умнее
многих преподов НПУ.

PS. Прошу молитвенной поддержки: грядущая неделя, как мы сегодня пришли к
выводу с моим студентом, будет для меня очень трудной. Очень трудной.
