% vim: keymap=russian-jcukenwin
%%beginhead 
 
%%file 24_10_2019.stz.news.ua.mrpl_city.1.mangush_koneferma
%%parent 24_10_2019
 
%%url https://mrpl.city/blogs/view/mangushska-konefermaunikalne-mistse-dlya-vidpochinku-ta-ztsilennya
 
%%author_id demidko_olga.mariupol,news.ua.mrpl_city
%%date 
 
%%tags 
%%title Мангушська конеферма – унікальне місце для відпочинку та зцілення
 
%%endhead 
 
\subsection{Мангушська конеферма – унікальне місце для відпочинку та зцілення}
\label{sec:24_10_2019.stz.news.ua.mrpl_city.1.mangush_koneferma}
 
\Purl{https://mrpl.city/blogs/view/mangushska-konefermaunikalne-mistse-dlya-vidpochinku-ta-ztsilennya}
\ifcmt
 author_begin
   author_id demidko_olga.mariupol,news.ua.mrpl_city
 author_end
\fi

\ii{24_10_2019.stz.news.ua.mrpl_city.1.mangush_koneferma.pic.1}

На теренах Приазов'я є чимало самобутніх і незвичайних місць, які підкреслюють
унікальність регіону. Та справжньою родзинкою приазовського краю є \textbf{Мангушська
конеферма}, яка сьогодні приваблює своєю неповторною атмосферою та потужною
енергетикою. І це зовсім не дивно, адже коні – це не просто втілення сили,
краси і грації. Ці розумні тварини – неабиякі психологи, спілкуватися з ними і
приємно, і корисно водночас. Вони раді всім відвідувачам і готові не тільки
приймати частування, а й віддавати сторицею. І дійсно, після спілкування з цими
чарівними істотами відновлюються сили, піднімається настрій і навіть
зміцнюється здоров'я.

\ii{24_10_2019.stz.news.ua.mrpl_city.1.mangush_koneferma.pic.2}

Історія єдиної конеферми Приазов'я починається з 1989 року, коли під егідою
голови колгоспу \enquote{Ленінський шлях} \textbf{Могильного Олександра Володимировича} було
вирішено створити конеферму. Головою нового об'єднання став \textbf{Сапах Володимир
Володимирович}. Вже у рік створення ферма в Мангуші придбала власне приміщення
на сорок денників, де затишно розташувалася їхня перша вихованка – чистокровна
скакова кобила. Незабаром у новий будинок увійшли ще три жеребці (Гандбол,
Гороскоп, Менует) і три десятка кобилок. Але далекосяжні плани керівництва на
цьому не зупинялися – того ж 1989 року були закладені ще дві стайні. 

Роком пізніше були завершені друга і третя черги будівництва конеферми - світ
побачили \enquote{маткова стайня} на тридцять сім денників, призначена для кобил та
їхніх лошат. Тоді ж створили і \enquote{тренерську} на двадцять шість секцій з
пристосуваннями для тренованих коней.

\ii{24_10_2019.stz.news.ua.mrpl_city.1.mangush_koneferma.pic.3}

Згодом з'явилися необхідні атрибути добре організованої виробничої бази
племінної конеферми – облаштовані місця для різних типів тренувань, левада,
ренгартен, тренувальне поле, скакове коло. З часів заснування в 1989 році
конеферма стала влаштовувати місцеві кінно-спортивні свята. Як правило,
змагання проводяться в День селища Мангуш, після закриття скакового сезону. До
програми змагань входять скачки на різні призи, рисистий біг коней, а також
конкур і виїздки. На свята запрошують учасників з різних міст, кінних заводів
та кінних клубів.

\textbf{Читайте також:} \emph{Почему лошадям лучше жить с людьми?}%
\footnote{Почему лошадям лучше жить с людьми?, Диана Краевая, mrpl.city, 14.01.2019, \par%
\url{https://mrpl.city/blogs/view/pochemu-loshadyam-luchshe-zhit-s-lyudmi}%
}

З проголошенням Україною незалежності почалися суттєві зміни, що торкнулися і
конеферми. Колгосп \enquote{Ленінський шлях} розформували вже в 1992 році. Йому
на зміну прийшла Агроферма \enquote{Мангуш}, частиною якої і стала конеферма.
Відтепер конеферма стає постійною учасницею різноманітних конно-спортивних
змагань України, на яких отримує низку перемог. Зокрема, першою перемогою
Мангушської конеферми стає присудження кобилі Метрології у 1996 році звання
\enquote{Чемпіон породи}. Хоча це звання і було пізніше оскаржене і відкликано,
проте кращим показником, що було воно присуджено не дарма стало завоювання
кобилою українського Дербі – Великого чотирирічного призу з попоною чемпіона
породи.  Сьогодні на території конеферми є велика простора стайня, розділена на
індивідуальні для кожного коня секції. Опалення немає. Коні гріються за рахунок
власної енергії. Готують їм їсти двічі на день. Витрати на годування досить
великі.

\ii{24_10_2019.stz.news.ua.mrpl_city.1.mangush_koneferma.pic.4}

Початок нового тисячоліття стає початком нової епохи в розвитку племінної
конеферми, коли у 2000 році вона увійшла до складу дочірнього підприємства
\enquote{Ілліч-Агро-Донбас} металургійного комбінату ім. Ілліча (Маріуполь), отримавши
№10 в списку його агроцехів. Завдяки поліпшенню фінансування далі нагороди
слідували одна за однією: 2000 рік – звання \enquote{Чемпіон чистокровної англійської
верхової породи} (кобила Мануелла), 2002 рік – звання \enquote{Чемпіон чистокровної
англійської верхової породи} (кобила Рамфіра), 2009 рік – титул і попона
\enquote{Чемпіон англійської породи} (жеребець Джордано) і незліченна кількість
традиційних призів.

\ii{24_10_2019.stz.news.ua.mrpl_city.1.mangush_koneferma.pic.5}

Середина першого десятиліття нового століття була відзначена для конеферми в
Мангуші проведенням масштабних робіт з благоустрою території: своє місце
зайняла головна брама, у внутрішніх дворах з'явилися клумби і альтанка,
споруджений мініатюрний фонтан, встановлений пам'ятник одному з перших жеребців
– Гороскопу, а на трасі з'явилися відповідні покажчики.

\textbf{Читайте також:} \emph{Дикая фауна Мариуполя}%
\footnote{Дикая фауна Мариуполя, Диана Краевая, mrpl.city, 07.11.2018, \par%
\url{https://mrpl.city/blogs/view/dikaya-fauna-mariupolya}
}

У 2010 році з'явилися нові жеребці: Банкомат (уродженець Росії) та Джордано
(гість з далекої Америки). Брали участь у змаганнях і в Києві, Одесі, Львові.
Загалом на конефермі продовжують розводити англійську чистокровну верхову
породу, яка вважається найбільш жвавою у світі. Їх дуже активно тренують і
випробовують на іподромі. 

\ii{24_10_2019.stz.news.ua.mrpl_city.1.mangush_koneferma.pic.6}

Нині на території Мангушської конеферми крім
виконання основних функцій (племінне розведення і тренінг скакових коней),
проводять іпотерапію, лікувальну верхову їзду та екскурсії для всіх охочих. І
якщо екскурсії можуть стати чудовим відпочинком для всіх охочих, то іпотерапія
має велику користь для здоров'я. Зокрема, важливою іпотерапія є для дітей з
інвалідністю. Сеанси проводить громадська організація \enquote{Країна коней} на чолі з
\textbf{Оксаною Терещенко}.

\ii{24_10_2019.stz.news.ua.mrpl_city.1.mangush_koneferma.pic.7}

Для роботи з маленькими пацієнтами представники громадської організації
використовують трьох спеціально навчених коней. Тварин готували кілька років.
Якщо кінь нервує, коли на ньому дитина - такий кінь не підходить. Також має
значення і ріст тварини. Маленькі коні не підходять під іпотерапію, тому що
вони частіше крокують і дитину сильно хитає. Чому ж іпотерпаія є корисною?!
Справа в тому, що кінь своїми імпульсами рухає тіло всіма чотирма ногами
вгору-вниз, вправо-вліво. В цей час в голові в мозку дитини відбуваються
процеси, які дають досвід прямоходіння. Тобто дитина запам’ятовує як, а потім
його тіло намагається повторити. Дітки, які не ходили роками, після іпотерапії
знову роблять спроби піти. У комплексі багато різних процедур (масажи,
терапії), які мають дуже позитивний ефект. До речі, іпотерапія проводиться і
для ветеранів АТО. Головний інструктор з іпотерапії, щира і відкрита Оксана
Терещенко, навчається в Полтаві за спеціальністю \enquote{Соціальна робота та фізична
реабілітація}. Вона віддається цій справі на всі 100\%. Завдяки її ентузіазму та
невичерпній енергії іпотерапія стала доступною і дуже ефективною.

\ii{24_10_2019.stz.news.ua.mrpl_city.1.mangush_koneferma.pic.8}

До того ж, на конефермі частими є спортивні свята та цікаві змагання. Кінний
спорт досить демократичний, він не містить ніяких обмежень, тому в ньому може
себе проявити як дитина, так і посивілий дідусь чи активна бабуся. Спортивні
змагання, екскурсії та іпотерапія приносять всім відвідувачам радість,
спортивний азарт та найголовніше – бажання жити і насолоджуватися життям!

За роки існування фермі є чим пишатися. Вона вже має численні грамоти і кубки
за перемоги, але найбільше вагоме досягнення – це фізичне і духовне одужання
багатьох українців, як маленьких, так і дорослих, які не з чуток тепер знають,
наскільки корисним може бути приїзд до Мангушської конеферми.

\textbf{Читайте також:} \emph{Истории маленького Аванса}%
\footnote{Истории маленького Аванса, Диана Краевая, mrpl.city, 23.01.2019, \par%
\url{https://mrpl.city/blogs/view/istorii-malenkogo-avansa}
}
