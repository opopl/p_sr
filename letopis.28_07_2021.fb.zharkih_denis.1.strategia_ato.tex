% vim: keymap=russian-jcukenwin
%%beginhead 
 
%%file 28_07_2021.fb.zharkih_denis.1.strategia_ato
%%parent 28_07_2021
 
%%url https://www.facebook.com/permalink.php?story_fbid=3042163602663703&id=100006102787780
 
%%author 
%%author_id zharkih_denis
%%author_url 
 
%%tags ato,donbass,strategia,ukraina,vojna
%%title Ну чем я не пророк? 2014 год, товарищи… - Стратегия АТО
 
%%endhead 
 
\subsection{Ну чем я не пророк? 2014 год, товарищи… - Стратегия АТО}
\label{sec:28_07_2021.fb.zharkih_denis.1.strategia_ato}
 
\Purl{https://www.facebook.com/permalink.php?story_fbid=3042163602663703&id=100006102787780}
\ifcmt
 author_begin
   author_id zharkih_denis
 author_end
\fi

Ну чем я не пророк? 2014 год, товарищи…

Стратегия АТО

Основная стратегия АТО в том, что власть возвращает территории, но не народ
живущий на этих территориях. Да, какая-то часть местных, кто искренно, кто за
деньги и выгоды, приветствуют политику власти. Это и понятно, уничтожаются не
только люди, а экономические связи, а, значит, их можно восстановить по готовым
лекалам и получать выгоду. Поэтому в каждом освобожденном пункте найдутся люди,
встречающие украинских военных и ЧВК с хлебом-солью. Ну, так и фашистов
встречали, например, в моем Киеве достаточно приязненно, и, кстати, далеко не
самые худшие люди в городе (многие потом будут казнены этими самыми фашистами). 

Но в целом, люди, живущие на Донбассе, этой власти совершенно не нужны. С ними
никто не работает, им никто не объясняет, что будет с ними, когда эта власть
добьется победы. Кстати, большинство из них уже ничего и не поймут, поскольку в
них стреляли, бомбили, брали измором и совершенно не спасали.

Хорошо видно, что эта власть совершенно не хочет нести никакой ответственности
за происходящее, во  всем обвиняя Россию, Путина. Она вполне демонстративно
показывает, что на нелояльных граждан ей наплевать, впрочем, как и на лояльных.
Я могу указать причину этого.

В Украине 23 года прививалось СЕЛЬСКОЕ МЫШЛЕНИЕ. Сельский человек в первую
очередь ценит ЗЕМЛЮ, а людей значительно меньше. Земля КОРМИТ, а люди —
НАХЛЕБНИКИ. Поэтому, крики типа: «Чемодан, вокзал, Россия!», это просто из
сельского архетипа «Вон чужаки из нашей улицы!».  А крики типа: «Украинский
хлеб ешь, а Украину не любишь!» тоже оттуда. Ведь основная масса русских и
русскоязычных в Украине чужого хлеба в жизни не едала, они его зарабатывали.
Буржуи, то есть эксплуататоры, начали появляться уже в независимой Украине. А
если говорить о номенклатуре УССР, то русских там было немного, основную массу
составляли колхозообразные динозавры. Вот эти колхозообразные выскочки и
сформировали мышление современной элиты. Именно это мышление уничтожило
практически всю промышленность Украины. Сразу скажу, что именно эту
промышленность и обслуживало русскоязычное население: рабочие ИТР-цы,
руководители всех звеньев. И они давали стране дохода больше, чем сельское
хозяйство в разы. Но для крестьянина ракета не имеет большого значения по
сравнению с куском сала. А если еще и имеются шансы  прихватить чужой земельки,
то смерти врагов значения не имеют. Наоборот — они желательны. Пустая земля —
меньше проблем. 

Остается добавить, что эта стратегия не может сработать даже теоретически. В
этой ситуации киевская власть воспринимается населением Донбасса, как агрессор.
Даже если удасться захватить основные населенные пункты — это не остановит
сопротивление. Да, Россия поддерживает повстанцев, но хорошо видно, что они
слабо организованы, у них свои свары и мало идеологической сплоченности. Но
дело даже не в сопротивлении. Подобная политика уничтожает остатки украинской
промышленности. Более того, она не делает психологических предпосылок создания
новой. Почему? Потому, что сельское мышление антиэкономично, в нем главное
накормить себя и свою семью, а прочие не учитываются. Поэтому любые инвестиции
не окупятся. Ну, разве что добывающая промышленность. Да никто этих инвестиций,
в общем и не предлагает. Поэтому АТО несет не новый мир, а убийства своих
граждан.

2014
