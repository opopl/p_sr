% vim: keymap=russian-jcukenwin
%%beginhead 
 
%%file 17_11_2019.fb.lozovskij_sergej.1.himera.cmt
%%parent 17_11_2019.fb.lozovskij_sergej.1.himera
 
%%url 
 
%%author_id 
%%date 
 
%%tags 
%%title 
 
%%endhead 
\subsubsection{Коментарі}

\begin{itemize} % {
\iusr{Марина Стаматина}
Очень узнаваемо.

\iusr{сергей лозовский}
\textbf{Марина Стаматина} 

Вопрос лишь в том, это только у нас так \enquote{зрение} устроено, и для нас эта
узнаваемость очевидна? Или...

\iusr{Марина Стаматина}

Есть люди, которые играют в Чапаева на шахматной доске и определению не могут
прогнозировать следующий ход ( неизвестно как шашка ляжет). А есть люди,
играющие в шахматы на той же шахматной доске, и предвидеть ситуацию через
несколько ходов для них нормально. Причём, первые, как правило, также считают
себя предвидящими ситуацию, опираясь на то, что доска то шахматная и в первом и
во втором случае. Но, обьяснить им, что правила игры( правила в жизни)
изначально разные - не всегда возможно. Это - ответ на вопрос.


\iusr{сергей лозовский}
\textbf{Марина Стаматина} Ну да... Есть нюанс

\end{itemize} % }
