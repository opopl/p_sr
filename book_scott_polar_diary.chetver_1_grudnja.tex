% vim: keymap=russian-jcukenwin
%%beginhead 
 
%%file chetver_1_grudnja
%%parent body
 
%%url https://www.facebook.com/ScottPolarDiary/posts/102837571673059
 
%%author 
%%author_id 
%%author_url 
 
%%tags 
%%title 
 
%%endhead 
\section{Четвер, 1 грудня}
\Purl{https://www.facebook.com/ScottPolarDiary/posts/102837571673059}

Загалом місяць починається гарно. Вночі вітер став дужчим; ми розігналися до 8,
9 і до 9,5 вузлів. Сильний північно-західний вітер, море буремне. Прокинувся
від сильного хитання.

За таких обставин у судна чудернацький і не надто життєрадісний вигляд.

Внизу весь простір утрамбовано настільки щільно, наскільки це взагалі можливо
для людських рук. А на палубі! Під баком п’ятнадцять поні тісняться
пліч-о-пліч, семеро з одного боку, восьмеро з іншого, головою до голови, а між
ними конюх, гойдаючись, гойдаючись безупинно вслід за мінливими поринальними
рухами корабля.

Якщо зазирнути у діру в перегородці, можна побачити шерегу голів із сумними,
терплячими очима: ті, що з правого борту, схиляються вперед, ті, що з лівого
борту – відхиляються назад, а тоді здіймається лівий борт, а правий
опускається. Мабуть, стояти отак разом із дня у день цілими тижнями – це
справжня каторга для нещасних тварин; і справді, хоча вони й добре годовані,
однак від напруження втрачають вагу й марніють. Менш із тим, не будемо міряти
їх людськими мірилами. Існують коні, що ніколи не лягають, і взагалі всі коні
сплять навстоячки; їхні ноги мають спеціальні зв’язки, які витримують вагу тіла
без жодних зусиль. Навіть наші нещасні тварини примудряються відпочивати і
спати, попри жорстку хитавицю. Решту місця в баку займають 4-5 тонн корму та
пильний невсипущий Антон. Антон страшенно потерпає від морської хвороби, але
минулого вечора викурив сигару. Він покурив трохи, тоді зробив перерву на
блювання, а тоді повернувся до сигари і, погладжуючи живіт, зауважив Отсу:
\enquote{недобре}, – який чемний хлопчина цей Антон! 

Ще чотири поні стоять поза баком з завітряного боку носового люка, завдяки
захисному брезенту, їм, либонь, ведеться краще, ніж їхнім товаришам. Одразу за
льодовнею обабіч від головного люка – дві величезні скрині 16×5×8 з моторовими
саньми; скрині ці визирають на кілька дюймів над палубою і займають страшенно
багато місця. Треті сани стоять впоперек юта, де раніше стояла лебідка. Щоб ці
скрині були в цілковитій безпеці, вони вкриті цупким брезентом та закріплені
важкими ланцюгами й мотузками.

Бензин для цих саней міститься у бляшанках та діжках, захищених міцними
дерев’яними скринями, що стоять безпосередньо перед кормою поблизу моторових
саней. Кількість – 2,5 тонни, і місця він займає пристойно.

Довкруг цих скринь, розтягнувшись від камбуза до штурвала, на палубі стоять
мішки з вугіллям, що становлять наш палубний запас, який стрімко зменшується.

Ми вийшли з Порт-Чальмерза із 462 тоннами вугілля на борту, це навіть більше,
ніж я сподівався, та все одно вантажна марка була на 3 дюйми вище води. На
кормі судно сиділо глибше на два фути, але це незабаром виправиться. 

На мішках з вугіллям, на моторових санях і між ними, а таокж на льодовні
розміщені пси, всього їх тридцять три. Їх доводиться тримати прикутими на
ланцюгу, вони мають прихисток на палубі, але загалом їм не позаздриш. Море
невпинно б’ється об борт і розпорошує хмари тяжких бризок на спину кожному, хто
наважиться спускатися з верхньої палуби на головну. Собаки сидять,
розвернувшись хвостами до всепроникної води, вони наскрізь мокрі. Усім своїм
виглядом вони виказують страждання від холоду й скрути; подеколи хтось із
сердег видає розпачливий стогін. Загалом ця група являє убого гнітючу картину,
нещасним створінням і справді непереливки.

Якимсь чином ми спромоглися знайти кожному місце за столом, хоча нас у
кают-компанії двадцять чотири офіцери. Один чи двоє зазвичай перебувають на
вахті, що трошки полегшує справу, але все одно тіснява. Їжа наша простецька,
але вражає, як наші двоє стюардів Гупер і Нілд дають раду з усіма обов’язками,
миють, прибирають, і при тому незмінно залишаються у найкращому гуморі.

З такою великою командою на борту, що дозволяє ставити одразу дев’ятьох
матросів на кожну вахту, управляти кораблем нескладно, а Мірз та Отс навіть
мають окремих помічників для догляду за собаками та поні; але у такі непогідні
ночі, як вчорашня, ціла \enquote{варта} добровольців пильнує та з чарівливим завзяттям
забезпечує нам безпеку й вигоди: дехто простягає руку помочі, якщо виникають
складнощі із поні та собаками, інші опускають чи здіймають вітрила, треті ж
поповнюють сховище вугіллям із палубного запасу.

Здається, Пріслі найбільше потерпає від морської хвороби. Іншим теж не солодко,
але вони вже мають певний досвід. Понтінг дивитися не може на їжу, але вправно
справляється з роботою; мені розповіли, що дорогою до Порт-Чальмерза він
неодноразово вишикував групи перед кінематографічним апаратом, хоча раз у раз
був змушений повертатися на борт. Вчора він проявляв пластини, тримаючи
ванночку для них в одній руці і звичайнісінький тазик в іншій!

Сьогодні пройшли 190 миль; гарний початок, але дещо утруднений однією
обставиною: ми прямували до острова Кемпбелла, але сьогодні зранку стало
очевидно, що наш швидкий поступ привів би нас на острів посеред ночі, а не
завтра, як я розраховував. Затримуватися в очікуванні дня за таких обставин
було б невигідно, тож цей пункт із розпорядку довелося викинути.

Пізніше вдень вітер перемінився на західний, злегка нас розвертаючи.
Сподіваюсь, надалі він більше не кружлятиме; тепер ми віддалилися на схід від
нашого курсу до криги більше, ніж на румб, і на три румби від завітряного боку
острова Кембпелла, тож до нього ми вже точно ніяк не підійдемо.
