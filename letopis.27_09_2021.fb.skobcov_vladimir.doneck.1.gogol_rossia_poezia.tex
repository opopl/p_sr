% vim: keymap=russian-jcukenwin
%%beginhead 
 
%%file 27_09_2021.fb.skobcov_vladimir.doneck.1.gogol_rossia_poezia
%%parent 27_09_2021
 
%%url https://www.facebook.com/permalink.php?story_fbid=423050395831334&id=100043791301113
 
%%author_id skobcov_vladimir.doneck
%%date 
 
%%tags donbass,gogol_nikolai,malorossia,poezia,rossia,ukraina
%%title Над малоросским фарисейством Горит Адмиралтейства шпиль
 
%%endhead 
 
\subsection{Над малоросским фарисейством Горит Адмиралтейства шпиль}
\label{sec:27_09_2021.fb.skobcov_vladimir.doneck.1.gogol_rossia_poezia}
 
\Purl{https://www.facebook.com/permalink.php?story_fbid=423050395831334&id=100043791301113}
\ifcmt
 author_begin
   author_id skobcov_vladimir.doneck
 author_end
\fi

Прожить за пазухой у Бога
В стране, которой алчет чёрт
Мог Николай Васильич Гоголь
И вряд ли кто-нибудь ещё.
Над малоросским фарисейством
Горит Адмиралтейства шпиль,
Нельзя от Пушкина отречься,
Нельзя Отчизну сдать в утиль.
Надежда шутит шутки злые
И в Назарете под дождём
Мы, как на станции в России
Перекладных в Россию ждём.
Как долго суждено по краю
Идти на свет, как на авось,
Как воздух пьют, так я читаю
То, что спалить не удалось.
Чтоб не обжечь ладонь о книжку,
Проворен, как слуга Семён,
Откроет чёртову задвижку,
Тот, кому тысяча имён.
Горят эпохи, поколенья
Атлантов и кариатид,
Горит бумага на поленьях
И только Слово не горит.
Летит по небу колесница,
Всё птице-тройке нипочём 
И то, что спалено в седмицу
Я в полнолуние прочёл.
