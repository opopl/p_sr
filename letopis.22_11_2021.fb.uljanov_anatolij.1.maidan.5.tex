% vim: keymap=russian-jcukenwin
%%beginhead 
 
%%file 22_11_2021.fb.uljanov_anatolij.1.maidan.5
%%parent 22_11_2021.fb.uljanov_anatolij.1.maidan
 
%%url 
 
%%author_id 
%%date 
 
%%tags 
%%title 
 
%%endhead 
\subsubsection{5}

Борясь с тоталитаризмом в прошлом, легко не заметить его в настоящем. Проще
осудить преследования диссидентов вчера, чем бороться с их преследованиями
сегодня. Тем паче, когда сегодняшних диссидентов преследуют вчерашние. Одно
дело диктатор, тиран, злодей, и другое, когда жертва становится угнетателем.
Вот где открывается простор для террора без тормозов.

Конечно, правду ближнего можно объявить неправдой, а самого его – безумцем;
закрыть глаза, как «украинские националисты» и «сталинские доярки» закрывали
глаза на расчеловечивание ближнего, убеждая свою совесть, что тот – враг. 

Врагов легче сливать, чем товарищей – знаю. И помалкивать легче, когда топчут
врага, а не друга. Несправедливость – это вообще неприятно. Иное дело жучить
\sem{плохого человека}. Пить  кровь по моральному праву.

Как возникает молчание толпы, когда стая бросается, и сильный начинает жрать
слабого? Такое молчание, как и стайный бросок, возникает тогда, когда людям
хочется жить, и страшно умереть. Взгляды вторичны. Да, мы используем слова,
объявляем свои знамёна, но динамика наших социальных отношений пронизана
шкурной логикой зверя. 

Я видел глаза самых близких друзей, которые знают, что друг важнее взглядов, и,
в личном общении, поддерживают друга, но умолкают, или принимают участие в
травле, когда оказываются на людях и среди людей, попадая в своё социальное
окружение. Не потому, что подлые, конечно. А потому, что страшно. Иначе завтра
не пригласят, не позовут, не дадут. 

Человеку страшно быть одному. Страшно идти против всех. Это же биология.
Отвержение группой – это приговор. Поэтому этот помалкивает, тот поддакивает.
Таков их групповой консенсус, основанный на взаимном страхе умереть изгоями.

Сегодня, в очередную годовщину майдана, я не прошу, чтобы мои друзья мне
поддакивали, но мне бы очень хотелось, чтобы они врали. Хотя бы самим себе.

Прятаться вечно не получится, друзья. Я понимаю, что вам страшно. Но это –
полезный страх. Он сообщает вам, что вы – следующие. Потому, что вы –
следующие, когда те, кому вы сегодня поддакиваете, закончат с нами.
