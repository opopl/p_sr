% vim: keymap=russian-jcukenwin
%%beginhead 
 
%%file 13_03_2019.stz.news.ua.mrpl_city.1.maryna_cherepchenko_avtorka_oberegy
%%parent 13_03_2019
 
%%url https://mrpl.city/blogs/view/marina-cherepchenkoavtorka-mariupolskih-oberegiv
 
%%author_id demidko_olga.mariupol,news.ua.mrpl_city
%%date 
 
%%tags 
%%title Марина Черепченко – авторка маріупольських оберегів
 
%%endhead 
 
\subsection{Марина Черепченко – авторка маріупольських оберегів}
\label{sec:13_03_2019.stz.news.ua.mrpl_city.1.maryna_cherepchenko_avtorka_oberegy}
 
\Purl{https://mrpl.city/blogs/view/marina-cherepchenkoavtorka-mariupolskih-oberegiv}
\ifcmt
 author_begin
   author_id demidko_olga.mariupol,news.ua.mrpl_city
 author_end
\fi

% portrait
\ii{13_03_2019.stz.news.ua.mrpl_city.1.maryna_cherepchenko_avtorka_oberegy.pic.1}

Кожне місто має свою символіку – герб, прапор і гімн, вони є втіленням історії
і відображенням теперішнього життя. Однак у кожного міста є і свої неофіційні
символи, що відрізняються своєю історією, унікальністю і важливістю. Маріуполь
не став виключенням! Пропоную познайомитися ближче з \textbf{Мариною Василівною
Черепченко} – авторкою маріупольських оберегів, символами незламності
приазовського краю. Думаю, що ви вже здогадалися, йдеться про \emph{тетраподи}, або
\emph{волноломи}, якими оздобленні українські блокпости, що оберігають кордони
Маріуполя. Саме вони зустрічають кожного маріупольця чи гостей і туристів
міста, які одразу звертають свою увагу на здоровенні бетонні велетні, покриті
різноманітними тематичними малюнками та орнаментами.

\ii{13_03_2019.stz.news.ua.mrpl_city.1.maryna_cherepchenko_avtorka_oberegy.pic.2_5}

\textbf{Читайте також:} \emph{В ООН озвучили число жертв среди мирных жителей на Донбассе за минувший год}%
\footnote{В ООН озвучили число жертв среди мирных жителей на Донбассе за минувший год, mrpl.city, 13.03.2019, \par\url{https://mrpl.city/news/view/v-oon-ozvuchili-chislo-zhertv-sredi-mirnyh-zhitelej-na-donbasse-za-minuvshij-god}}

\ii{13_03_2019.stz.news.ua.mrpl_city.1.maryna_cherepchenko_avtorka_oberegy.pic.6}

Марина народилася та виросла в багатодітній сім'ї у Маріуполі. Батько Марини –
інженер-конструктор, мати – штукатур-мо\hyp{}заїчник. Мати хотіла отримати художню
освіту, але їй це не вдалося, ось чому вона всіляко допомагала і підтримувала
доньку. Тому Марина вступила та закінчила Художню школу ім. А. І. Куїнджі. Її
викладач, \emph{\textbf{Анатолій Борисович Манохін}}, розуміючи, наскільки його учениця здібна,
навіть оплатив їй проїзд до Харкова, сподіваючись, що вона зможе вступити до
Харківської академії дизайну. І незважаючи на те, що Марина потрапила до
академії лише з третьої спроби, були свої життєві труднощі, все ж таки її
викладач не помилився. Від навчального процесу маріупольчанка отримала справжнє
задоволення. Вона була дуже цілеспрямованою студенткою і йшла до своєї мети
впевненими кроками. Марина відноситься до тих людей, кому пощастило працювати
за покликанням, вона відчувала що обрала правильний напрям, залишилося лише
реалізувати власні здібності повною мірою.

\ii{13_03_2019.stz.news.ua.mrpl_city.1.maryna_cherepchenko_avtorka_oberegy.pic.7}

У 1998 році Марина Василівна почала працювати дизайнером проектно-будівельної
компанії \enquote{Азовінтекс}, саме тоді й почалася її професійна діяльність. Марина
пробувала себе у різних напрямах сучасного дизайну: проектуванні інтер'єрів,
графічному дизайні, проектуванні упаковки, створенні реклами. Але поступово
реалізувала себе саме як художник. А сталося це саме у 2013 році, коли виникла
ідея вдихнути нове життя в тетраподи. Відомо, що тетраподи, або волноломи, є
мирними потужними берегозахисними інженерними спорудами, які зберігають
берегову лінію від руйнування водною стихією. Цікаво, що в Японії вже існує
цілий пласт дизайнерських розробок на тетраподну тематику просто тому, що
японцям тетраподи життєво необхідні для збереження їхніх островів. У 2013 році,
коли Маріуполь святкував своє 235-річчя, розпочався арт-проект \emph{\enquote{Твоя стихія}}, в
якому ці величезні бетонні споруди стали арт-об'єктами й почали прикрашати
центральні площі та парки міста. Керівником арт-проекту стала Марина Василівна,
яка зрозуміла, що тетраподи – багатофункціональні, робота з поверхнею цих
велетнів розкриває багато можливостей, можна експериментувати з кольорами та
композиціями. Загалом, працюючи над тетраподами, Марина відкрила для себе
безліч маріупольських талантів. І разом в однодумців виникло розуміння, що
будова і пропорції тетраподів створені за законами \enquote{Золотого перетину}, завдяки
куту міцності їх неможливо порушити з місця без спеціальної техніки, вони немов
символізують водночас і статику, і динаміку у своїй будові. Наша героїня просто
закохалась в цю форму, \emph{\enquote{тому що вона має і Користь, і Міць, і Красу}}. І цій
потужній красі довелося в серпні 2014 року стати на охороні спокою та миру
нашого міста, тому що в серпні Маріуполю почала загрожувати реальна небезпека
військової окупації. Тоді частину тетра-арт-об'єктів довелося перевезти на
українські блокпости, де військові, користуючись особливостями будови
тетраподів (вага кожного понад 25 тонн), зупинили вороже танкове вторгнення.

\textbf{Читайте також:} \emph{Мариупольцев приглашают познать тонкости собственного дела}%
\footnote{Мариупольцев приглашают познать тонкости собственного дела, Олена Онєгіна, mrpl.city, 12.03.2019, \par%
\url{https://mrpl.city/news/view/mariupoltsev-priglashayut-poznat-tonkosti-sobstvennogo-dela-foto}
}

\ii{13_03_2019.stz.news.ua.mrpl_city.1.maryna_cherepchenko_avtorka_oberegy.pic.8}

Розмовляючи з Мариною Василівною, розумієш, наскільки це людина своєї справи.
Вона не просто творча особистість, яка захопилася ідеєю. Марина на 100\% живе
тим, що робить, а головне - вона розуміє, для чого цим займається. Не дивно, що
наша героїня була нагороджена медаллю \emph{\enquote{За жертовність та любов до України}}.
Дизайнер наголошує, що навіть у маленькому масштабі тетрапод може нести велике
естетичне та утилітарне призначення. Використовуючи різні матеріали та техніки,
за п'ять років вдалося створити цілу низку різноманітних речей, які мають у
своїй будові пропорції тетраподу, вони мають особливе значення для всіх, хто
приїжджає до Маріуполя або несе тут військову службу. Це і просто сувеніри,
маленькі копії велетнів-охоронців, і вазочки, що зберігають життя зрізаним
квіточкам, і пляшки для міцних напоїв, і елементи інтер'єру – світильники або
зручні подушки, що надають оселі індивідуальності та затишку, і навіть срібні
ювелірні прикраси, що мають лікувальні властивості для своїх володарок. Також з
тетраподів робили розмальовки-антистреси. Ця ідея належить Дмитру Ступнику,
який в той час був частиною команди Азовінтексу. Важливо, що який би вигляд не
мали тетраподи, всі вони знаходять відгук в душі маріупольців, адже можуть
слугувати як оберегом, так і прикрасою. На багатьох загальноміських заходах
можна побачити Марину Василівну, оточену юними маріупольцями, що є учасниками
майстер-класів від дизайнера.

Сім'я Марину підтримує, особливо її мама, яка часто допомагає перетворювати
сірі конструкції у справжні прикраси міста. Чоловік, викладач політології у
МДУ, ветеран АТО, розуміє свою жінку і підтримує прагнення займатися творчістю,
сам бере на себе частину домашньої роботи. Загалом сім'я Марини переосмислила
життя після війни, стали більше розуміти один одного. Авторка маріупольських
оберегів любить гуляти з рідними людьми узбережжям Азовського моря, яке
заспокоює. Марина давно колекціонує антикварні сімейні реліквії, фарфорові
статуетки.

\textbf{Читайте також:} \emph{\enquote{Пост-мост} продемонстрирует эстетику индустрии Мариуполя}%
\footnote{\enquote{Пост-мост} продемонстрирует эстетику индустрии Мариуполя, Яна Іванова, mrpl.city, 12.03.2019, \par%
\url{https://mrpl.city/news/view/post-most-prodemonstriruet-e-stetiku-industrii-mariupolya}
}

Марину Василівну надихає сам процес творення, мистецтво. Надихаючись творчістю
відомої майстрині петриківського розпису \emph{Вакуленко Тамари Олексіївни}, Марина
вирішила у майбутньому створити на тетраподах приазовську петриківку,
використовуючи пастельну гаму та саме петриківську стилізацію степових трав,
яка присутня в приазовській природі. \emph{\enquote{Петриківка має право розвиватися}}, –
підкреслює її вчителька, і Марина вважає, що це стане унікальним доповненням в
мистецтві Маріуполя. Водночас незабаром маріупольці зможуть спробувати
тетраподики на смак, адже у планах зробити форму для цукерок і тістечок у
вигляді всім знайомої конструкції.

\ii{13_03_2019.stz.news.ua.mrpl_city.1.maryna_cherepchenko_avtorka_oberegy.pic.9}

А загалом планів ще багато, зокрема, Марина
планує створити елементи одягу, сумок у вигляді тетраподів, ще думає про
відкриття творчої майстерні, де можна буде займатися з маріупольською творчою
молоддю. Вона дуже радіє, що незабаром у Маріуполі відкриє свої двері художня
академія, яка посприяє ще більшому розвитку образотворчого мистецтва у місті.

\ii{13_03_2019.stz.news.ua.mrpl_city.1.maryna_cherepchenko_avtorka_oberegy.pic.10}

Маріупольський оберіг-тетраподик, покритий художнім розписом, вже давно став
найкращим подарунком як для маріупольців, так і для гостей міста. Цікаво, що
маріупольські обереги вже зберігаються в колекціях в Латвії, Польщі, Німеччини,
Іспанії, США, Британії та інших. До Марини Василівни часто звертаються
військові, що повертаються додому з Приазов'я, а іноді вона сама робить
подарунки митцям, що відвідають наше місто, людям, що несуть і розвивають
українську культуру в Маріуполі, розповсюджують його місію протистояння
російської агресії по всьому світі. Зокрема, тетраподну подушку повіз додому
гурт-кабаре \emph{\enquote{Dakh Dau\hyp{}ghters}}, маріупольські обереги є в колекції співака
В'ячеслава Купрієнка та письменника Олександра Ірванця, митця Анатолія
Гайдамаки, журналіста \enquote{ZIK} Мирослава Ганущака, режисера фільму \enquote{Кіборги}
Ахтема Сеітаблаєва та багатьох інших українських патріотів. Нещодавно в
Маріуполі проходила конференція скульпторів по металу \enquote{Обереги України}, і всі
учасники отримали на згадку про наше місто свій тематичний тетраподик-оберег.

\ii{insert.read_also.burov.rena_sajenko}

\ii{13_03_2019.stz.news.ua.mrpl_city.1.maryna_cherepchenko_avtorka_oberegy.pic.11}

Марина підкреслює, що тетраподи зустрічають, вітають, оберігають містян та
гостей Маріуполя, поруч з ними вже п'ять років несуть службу військові з усієї
України, тому знайома форма нагадує і про шалений вибух творчості маріупольців
та про буремні часи, коли  промислове місто у моря Маріуполь стало фортецею для
цілої країни.

\textbf{Улюблені книги Марини Василівни:} книги Е. М. Ремарка, Марини та Сергія
Дяченків.

\textbf{Улюблені фільми:} фільми за участю Тома Генкса, зокрема \enquote{Форрест Гамп}.
Любить переглядати й документальні стрічки, особливо вразив фільм виробництва
\enquote{BABYLON'13} \enquote{Десять секунд}.

\textbf{Порада маріупольцям:} \enquote{Любити місто, намагатися робити його кращим, починаючи зі
свого будинку, не падати духом і знаходити резерви для сил, позитивно дивитися
на життя!}

\ii{insert.read_also.demidko.osypenko}
