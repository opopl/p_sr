% vim: keymap=russian-jcukenwin
%%beginhead 
 
%%file 14_10_2020.sites.ru.zen_yandex.yz.chelovek_i_mir.1.habarov
%%parent 14_10_2020
 
%%url https://zen.yandex.ru/media/chelovek_i_mir/kem-byl-habarov-v-chest-kotorogo-nazvan-gorod-habarovsk-5f2fa75604621e7bdf3de31a
 
%%author 
%%author_id yz.chelovek_i_mir
%%author_url 
 
%%tags 
%%title Кем был Хабаров, в честь которого назван город Хабаровск
 
%%endhead 
 
\subsection{Кем был Хабаров, в честь которого назван город Хабаровск}
\label{sec:14_10_2020.sites.ru.zen_yandex.yz.chelovek_i_mir.1.habarov}
\Purl{https://zen.yandex.ru/media/chelovek_i_mir/kem-byl-habarov-v-chest-kotorogo-nazvan-gorod-habarovsk-5f2fa75604621e7bdf3de31a}
\ifcmt
	author_begin
   author_id yz.chelovek_i_mir
	author_end
\fi

\index[names.rus]{Хабаров, Ефорей Павлович!Русский землепроходец и исследователь Сибири и Дальнего Востока, 14.10.2020}

Когда люди не могут внятно объяснить, откуда взялись такие названия, как
"Москва" или "Казань" и что они означают, это хотя бы понятно, прошло много лет
и сведения давно утеряны. Когда не могут достоверно объяснить название города
"Челябинск", основанного не так давно в 1736 году, это уже странно. У
Хабаровска с этим проблем нет, понятно, что город назван по фамилии Хабаров, и
сейчас давайте разберёмся кто это был такой и чем он был известен.

Возможны всякие повороты, как например в случае с Бауманом, в честь которого
назван университет МГТУ имени Баумана,\Furl{https://zen.yandex.ru/media/id/5d6392f81ee34f00add98b5f/5f279e616811bc687075135e?integration=site_desktop&place=export} но в данном случае всё понятно и
закономерно.

Ерофей Павлович Хабаров, в честь которого назвали Хабаровск, был русским
землепроходцем и исследователем Сибири и Дальнего Востока.

\ifcmt
  pic https://avatars.mds.yandex.net/get-zen_doc/3468648/pub_5f2fa75604621e7bdf3de31a_5f2fb9284c263c2627bdcc63/scale_1200
	caption Памятник Хабарову, взято с Яндекс.Картинки
\fi

Родился он предположительно в 1603 году в деревне Дмитриево. Сейчас это
Котласский район Архангельской области.

В начале 20-х годов 17-го столетия вместе с братом Никифором отправился в
Сибирь. Как пишут, причиной этого была попытка уйти от долгов.

С 1625 по 1641 занимался исследованиями в Сибири, на Таймыре и вдоль реки Лены.
С 1641 по 1645 сидел тюрьме в Якутском остроге.

В 1649 году Хабаров снарядил экспедицию и отправился из Якутска вверх по Лене и
Олёкме, а затем в Амур. Поход по Амуру продолжался до 1653 года. Было много
стычек с Маньчжурами и другими местными племенами. Были и раздоры внутри самой
экспедиции. В результате таких конфликтов Хабаров и был в итоге отстранён от
руководства экспедицией, приехавшим из Москвы посланником.

Роль Хабарова его современники оценивали по разному. Кто-то хвалил его и
сравнивал с англичанами, исследовавшими моря и Северную Америку. Кто-то
наоборот обвинял Хабарова в грабежах и ссорах с миролюбивыми племенами,
проживающими на Амуре, из-за чего последние переметнулись к маньчжурам.

Как бы то ни было, Ерофей Павлович Хабаров сделал большой вклад в исследования
новых земель, которые позволили впоследствии освоить всю эту территорию.

Кроме города Хабаровска, есть ещё посёлок с названием "Ерофей Павлович" в
Амурской области на Транссибирской магистрали. Его именем названы транспортные
корабли, арена по хоккею с мячом в Хабаровске, а также улицы во многих городах.


\ifcmt
  pic https://avatars.mds.yandex.net/get-zen_doc/3524532/pub_5f2fa75604621e7bdf3de31a_5f2fb8d403d6871d78669a70/scale_1200
	caption Ерофей Павлович на карте России
\fi

