% vim: keymap=russian-jcukenwin
%%beginhead 
 
%%file 09_02_2022.stz.news.ua.hvylya.1.politychni_hyzhaky
%%parent 09_02_2022
 
%%url https://hvylya.net/analytics/246911-yak-priborkati-politichnih-hizhakiv-v-ukrajini
 
%%author_id druzenko_gennadij,news.ua.hvylya
%%date 
 
%%tags obschestvo,politika,ukraina
%%title Як приборкати політичних хижаків в Україні
 
%%endhead 
 
\subsection{Як приборкати політичних хижаків в Україні}
\label{sec:09_02_2022.stz.news.ua.hvylya.1.politychni_hyzhaky}
 
\Purl{https://hvylya.net/analytics/246911-yak-priborkati-politichnih-hizhakiv-v-ukrajini}
\ifcmt
 author_begin
   author_id druzenko_gennadij,news.ua.hvylya
 author_end
\fi

Генадій Друзенко категорично проти того, аби наступну Конституцію ухвалював
парламент. Не можна політичним хижакам віддавати на відкуп побудову зоопарку.

\begin{zznagolos}
Мне на плечи бросается век-волкодав,

Но не волк я по крови своей...

Осип Мандельштам
\end{zznagolos}

Колись професор Чарльз Тіллі, якого New York Times характеризує як «одного з
найвидатніших істориків та соціологів світу», написав провокаційне есе: «Війна
та розбудова держави як організована злочинність». Ця робота стала класичною.
Гугл Академія стверджує: розвідку цитує більш ніж 5500 наукових робіт, що є
справжнім успіхом в академічному світі. У цьому есе автор демонструє, як
держави (принаймні європейські) виростали з того, що «в меншому масштабі та
менш успішне в реалізації ми називаємо організованою злочинністю». Робота
побачила світ рівно 40 років тому, коли розвал СРСР та «лихі дев’яності» були
ще попереду. Тіллі аналізував саме західний світ, частиною якого ми так
прагнемо стати.

\ii{09_02_2022.stz.news.ua.hvylya.1.politychni_hyzhaky.pic.1}

У своєму чудовому аналізі класик соціології показує хижацьку природу влади як
такої. Травоїдні вкрай рідко потрапляють на політичний олімп. Сама логіка
політичної боротьби служить природному добору «політичних тварин», які скрізь у
світі здебільшого цинічні, підступні та хижі. І в цьому полягає системна
пастка, в яку потрапляють українці, раз-у-раз шукаючи на царство доброго та
порядного царя чи царицю. У політичному тераріумі такі особини зазвичай не
виживають, хоча іноді дійсно трапляються у вирішальні моменти історії своїх
країн. Луцій Квінкцій Цинциннат, Джордж Вашингтон та Нельсон Мандела – можливо,
найбільш відомі приклади таких атипових політиків, які так і не стали залежними
від кокаїну влади.

Саме останній за названих державців надихнув перуанського прозаїка та
Нобелівського лауреата Маріо Варґаса Льосу на такі слова: «Мандела є найкращим
з усіх наявних – і дуже нечисленних в наші дні – прикладів того, що політика є
не лише тією брудною та посередньою справою, як думають багато людей, яка
допомагає спритникам збагатитися, а ледарям вижити, нічого не роблячи, а
діяльністю, яка може також поліпшити життя, замінити фанатизм толерантністю,
ненависть солідарністю, несправедливість справедливістю, егоїзм спільним
благом, і що є політики, як-от південноафриканський державний діяч, які
залишають свою країну, світ значно кращим, аніж він був до їхнього приходу».

Як відомо, винятки лише підкреслюють правила. А правило полягає в тому, що
політична боротьба в усьому світі – це не так боротьба ідей та ідеалів, як
боротьба за право інкасувати конкретну країну (і не тільки). І тому коли
українці сподіваються на доброго царя/царицю чи порядних політиків, які
прийдуть до влади і ощасливлять їх, шанси що це трапиться не більші за
ймовірність зірвати джекпот у казино.

Що ж робити нам, посполитим, усвідомивши, що скрізь у світі політики – це за
своєю природою хижаки, а партії – політичні зграї, що гризуться за владу як
право легітимно інкасувати країну? Побудувати клітку для цього політичного
серпентарію!

Іноді науковці називають цю клітку інституціями, іноді – верховенством права,
іноді – конституційною рамкою, іноді – системою стримувань і противаг. Назви
можуть бути різні. Модні нині автори Джеймс Робінсон та Дарон Аджемоглу
називають це «вузьким коридором» та «скутим Левіафаном». Американські науковці
Стівен Левіцкі та Деніел Зіблат використовують термін guardrail – відбійник,
який не дозволяє політичним шумахерам перетворити державу на суцільний оффроад.

В будь-якому разі, аби держава з найбільш ефективного та масштабного
інструменту організованої злочинності перетворилась на Res publica – спільну
справу громадян, потрібно витворити матрицю, яка змусить руйнівну енергію
егоїзму політичних тварин працювати на загальне благо. А для цього потрібні
лише дві головні передумови: 1) наявність достатньої кількості громадян, тобто
політично активних, економічно самодостатніх та порівняно добре освічених
людей, які почуваються господарями у власній державі; 2) правильна інституційна
конструкція держави, яка каналізує енергію персонального его у загальне благо,
як правильна конструкція автівки перетворює перманентні вибухи палива у двигуні
внутрішнього згоряння у плавний та передбачуваний рух автомобіля.

І тут ми підходимо до головної проблеми інституцій та законів. Юридичні закони,
на відміну від законів хімії чи фізики, не мають об’єктивного характеру. Якщо
яблуко завжди падає на землю, то жодним актом парламенту не змусиш його падати
вгору. Навіть якщо за це проголосує конституційна більшість депутатів. Аби
юридичні закони та інституції запрацювали, потрібна людська участь. Можна
сказати, що право живиться людською волею жити згідно з правилами. Якщо життя
винагороджує порушників законів, якщо грати не за правилами вигідніше за
дотримання правил, якщо «друзям – все, ворогам – закон», тоді жодна
інституційно-юридична рамка не стримуватиме переможця політичних перегонів та
його зграю.

Виконувати закон, жити по правилах має бути стратегічно вигідно. Існує кілька
теорій, як конституційна рамка виникала в державах як альтернатива легітимації
влади найбільш спритного «осілого бандита» (термін зі знаменитої статті Манкура
Ослона «Диктатура, демократія та розвиток»). До речі, друга формула державності
також має свій raison d'être: автократія і навіть диктатура завжди краще за
війну всіх проти всіх. Саме цим пояснюється, чому після чергової руїни українці
шукали хоч якоїсь передбачуваності та стабільності в «русском мире». Але це
окрема тема…

Повернімося до теорії виникнення конституційної рамки, яка обмежує свавілля
влади (навіть отриманої в результаті чесних виборів). Мені здається
перспективною теорія чиказького професора Тома Гінзбурга та його колег, які
стверджують, що конституція – це страховий поліс для еліт у ситуації, коли
жодна політична зграя не певна, що виграє наступні вибори. Саме в такій
ситуації стратегічної невизначеності еліти ладні домовлятися про правила гри,
аби той, хто програє вирішальні вибори, опинився у парламентській опозиції, а
не за ґратами чи в екзилі.

Втім, у цій теорії є свій ґандж: вона не пояснює, чому переможець буде
дотримуватись домовлених було правил. Медісон, головний архітектор
американської конституції, недаремно називав конституційні положення parchment
barriers – пергаментними бар’єрами на шляху агресивного духу влади. Ми,
українці, не раз були свідками, як, отримавши владу, та чи інша політична зграя
плювати хотіла на конституційну рамку. Натомість опозиційні політичні зграї
мали небагато можливостей чинити опір відвертому виходу переможців далеко за
межі своїх конституційних повноважень. І жоден правоохоронний орган, жоден суд
чи парламентська ТСК не могли цьому завадити. Як чудово сформулював
американський суддя і філософ права Біллінгс Льорнд Хенд: «Я часто думаю, чи не
занадто ми покладаємось у наших надіях на конституцію, на закони та на суди. Це
марні надії, повірте мені, це марні надії. Свобода лежить в серцях чоловіків та
жінок; коли вона помирає там, жодна конституція, жоден закон, жоден суд не
здатні врятувати її...». Ми, українці, засвоїли цей урок на наших майданах.
Засвоїли дорогою ціною…

Але майдани – це радше свідчення ненормальності функціонування державного
організму. Це як лихоманка – доказ того, що українському державному організму
вкрай бракує інституційного імунітету, який покликаний розв’язувати природні
для будь-якого суспільства конфлікти у більш цивілізований спосіб. І тут
доречно пригадати, що за одною з гіпотез саме судовий конституційний контроль
(право незалежної від волі більшості судової гілки влади визнавати акти інших
гілок влади, що мають пряму демократичну легітимність, неконституційними) став
тим запобіжним клапаном, який дозволив США уникати народних повстань та
революцій упродовж майже двох з половиною століть. Але як змусити інституції
працювати?

І тут нема ради: будь-якій країні потрібні батьки-засновники, що закладуть
конституційну матрицю ДО того, як у гру вступлять звичайні політичні хижаки.
Перед тим, як дати старт політичним перегонам, потрібно збудувати дороги та
поставити відбійники обабіч них. Перед тим, як починати боротьбу за кращі
помешкання у домі, цей дім потрібно спланувати і побудувати. Перед тим, як
випустити політичних хижаків на сцену, хтось має створити для них клітку, аби
вони не бачили в нас лише здобич, і нам раз-у-раз не доводилося виходити на
вулиці та майдани, аби довести їм зворотне.

Хтось має свідомо закласти традиції дотримання правил та показати особистий
приклад слідування ним. У травні 1782 року Джордж Вашингтон отримав від
офіцерів Континентальної армії послання, яке увійшло в історію як Ньюбурзький
лист. У ньому офіцери закликали свого головнокомандувача стати королем
Сполучених Штатів. Це виглядало більш ніж природно та реалістично, виходячи з
теорії осілого бандита та реалій тогочасного світу, в якому монархії були
нормою, а республіки – винятком. Вашингтон рішуче відмовився. Коли Сполучені
Штати ратифікували свою конституцію та обирали першого президента, ні в кого не
було сумнівів, що ним стане Джордж Вашингтон. На той час американська
конституція не містила жодного обмеження кількості каденцій, які виборний
очільник держави міг залишатися на посаді. Вашингтон міг гарантовано вигравати
вибори за виборами. Тим не менш після другої каденції він свідомо відмовився
балотуватися втретє.

Нельсон Мандела, який після перемоги над апартеїдом був кумиром не тільки
південноафриканців, а й усього вільного світу, також обмежився одним терміном
на президентській посаді. Принаймні двічі суди ПАР визнавали рішення президента
Мандели неконституційними. Одного разу його, 78-річного старця, який провів 27
років у в’язниці, навіть викликали до суду давати свідчення у справі створеної
президентським указом Комісії з розслідування відносин у Південноафриканському
регбійному союзі. Немає жодних сумнівів, що «Мадіба» (як ласкаво називали
лідера АНК) міг собі дозволити ігнорувати судові приписи. Він міг наслідувати
п’ятого президента США Джексона, якому приписують вислів: «Суддя Маршалл
ухвалив рішення. Тепер нехай виконає його!». Радники пропонували Манделі не йти
до суду. Але він пішов і добровільно виконав те, що постановив суддя. Він
зробив це, аби продемонструвати, як належить ставитися до судової гілки влади,
навіть якщо тобі не подобаються її окремі рішення.

В Україні таких прикладів, на превеликий жаль, удень з каганцем не відшукаєш.
Тому незалежність ми здобули і героїчно тримаємо її периметр, а от що має бути
збудовано на «своїй богом даній землі» – досі не тільки не домовились, а й за
гамбурзьким рахунком навіть не почали серйозного обговорення обрисів
майбутнього українського дому. Тому і «маємо те, що маємо», а точніше те, що
великою мірою отримали у спадок від «ненависного совка». Тому, гудячи своє
радянське минуле й цураючись його (часто заслужено засуджуючи його злочини, але
водночас не визнаючи його заслуг), парламент незалежної України ухвалює закони
про декомунізацію в будівлі, яку збудували у сталінські часи для ВУЦВК, уряд
працює в сірому монстрі, який планувався як Народний комісаріат внутрішніх
справ УСРР, а Офіс/Секретаріат/Адміністрація Президента керує Україною з
будівлі ЦК КПУ, з якої жодними освяченнями не виженеш демонів радянської
бюрократії та стилю.

Біда в тому, що проклинаючи комуністів, сучасні лідери України чудово засвоїли
їхній стиль: говорити від імені народу, не підпускаючи посполитих на гарматний
постріл до влади. Навіть сили територіальної оборони стали у нас родом військ
ЗСУ замість бути «добре організованим ополченням», як це прописано у Другій
поправці до Конституції США. Замість обмеження повноважень та розподілу влади
між різними гілками, органами та щаблями, які покликані стримувати та
врівноважувати одне одного, кого б ми не обрали президентом – невдовзі він
перетворюється на осучасненого першого секретаря ЦК КПУ. А його політична зграя
– на інкасаторів країни.

Тому на часі починати засадничу дискусію: який дім ми хочемо звести на своїй
землі, кордони якої ми знову щедро полили українською кров’ю. Для нас уже стала
аксіомою українська державність. Але яка саме держава нам потрібна, ми навіть
не почали дискутувати. А без цієї дискусії переможець отримуватиме все. І,
незалежно від прізвища, інкасуватиме країну із політичною зграєю своїх
поплічників.

Тому, коли будете придивлятись до старих і нових кандидатів у президенти,
спробуйте зрозуміти: перед вами архітектор, який здатний започаткувати та
змодерувати широку суспільну дискусію про те, яким має бути спільний
український дім, а потім організувати його будівництво, або черговий інкасатор,
який веде до влади свою зграю – ділити рештки шкури давно померлого радянського
ведмедя. Шукайте людину, здатну спроектувати та побудувати клітку для
політичного тераріуму, який обов’язково прийде до влади після нього чи неї. Аби
ненаситне его цих політичних тварин почало працювати на загальне благо. А не
навпаки, як це трапляється в Україні досі…

Президентська посада в українській конституційній архітектурі – ідеальний
майданчик для справжнього державця-архітектора. І вкрай небезпечний інструмент,
коли на неї потрапляє політичний хижак. Адже всенародно обраний глава держави
зобов’язаний своєю легітимністю не парламенту, який майже завжди є політичним
серпентарієм, а народу як джерелу влади. Президент має привілей спілкуватися з
народом, який його обрав, напряму. Прикладом такої щирої розмови лідера та
нації можна вважати славнозвісні «Бесіди біля каміну» Франкліна Делано
Рузвельта, які допомогли вивести Сполучені Штати з пастки Великої депресії та
радикально змінити саме розуміння ролі американської держави в житті
американського суспільства. Так 14 червня 1946 року говорив до французького
народу Шарль де Голль, сформулювавши у своїй другій Байонській промові візію
конституційної архітектури П’ятої Республіки.

Якщо ми віримо, що, як записано у 5-й статі Конституції України, «носієм
суверенітету і єдиним джерелом влади в Україні є народ» і саме йому належить
виключне право «визначати і змінювати конституційний лад», тоді надійну
конституційну клітку для політичного серпентарію у майбутньому українському
домі може звести народ і тільки народ. Але хтось має організувати й змодерувати
цей процес. Це і є справжня місія наступного президента. Якщо така синергія
трапиться, є шанс, що народ буде наглядати за цією кліткою та пильнувати, аби
політичні хижаки не вирвалися з неї.

Саме тому я категорично проти того, аби наступну конституцію ухвалював
парламент. Не можна політичним хижакам віддавати на відкуп побудову зоопарку.
Втім, повторюсь: для цього сам президент має бути конструктором-державцем, а не
політичним хижаком. Адже і Гітлер, і Мусоліні, і Перон, і Ленін з Троцьким
також апелювали до народу. Також воліли говорити до нації через голови
політичних еліт. До яких трагедій це призвело, добре відомо. В людській природі
глибоко укорінене те, що мій вчитель Девід Вільямс називає father fantasy –
мрія про президента-батька, який візьме на себе непосильний для багатьох тягар
свободи та відповідальності за їхнє життя. Можливо, найкраще описав цю спокусу
Достоєвський у своєму «Великом инквизиторе». Тому ставки з виборами
національного лідера справді захмарні. Ціна помилки – надзвичайна.

І маленька підказка насамкінець: найкращий кандидат на посаду глави держави –
це кандидат, який публічно пообіцяє, що він чи вона – останній всенародно
обраний президент у цій країні. І якому народ повірить, що він (чи вона)
дотримає свого слова. Посутнісну деконструкцію совка та розставання з «русским
миром», навіть коли вони так старанно пофарбовані у жовто-блакитні кольори,
слід починати з деконструкції віри у доброго царя/гетьмана/генсека/президента,
який здатний привести нас у світле майбутнє. За нас і замість нас…
