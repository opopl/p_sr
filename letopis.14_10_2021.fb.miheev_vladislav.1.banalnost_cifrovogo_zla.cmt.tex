% vim: keymap=russian-jcukenwin
%%beginhead 
 
%%file 14_10_2021.fb.miheev_vladislav.1.banalnost_cifrovogo_zla.cmt
%%parent 14_10_2021.fb.miheev_vladislav.1.banalnost_cifrovogo_zla
 
%%url 
 
%%author_id 
%%date 
 
%%tags 
%%title 
 
%%endhead 
\subsubsection{Коментарі}

\begin{itemize} % {
\iusr{Олег Климов}

Устаревшие данные. Доносов 300 млрд. Ибо "писатели" отправляют их нынче по
несколько штук в день каждый. Так что сталинские времена - это просто сказочная
жизнь. Там, по Довлатову, было 2 млн "писем счастья" - просто мелочь, по
сравнению с нынешним временем дерьмократических ценностей. А по справедливости
и закону отмечу: самое трудное - убедить невиновного виновного в том, что
"закон и справедливость - разные вещи, и вам надо бы лет энцать об этом
подумать, мы вам в этом поможем".

\iusr{Александр Хургин}
\textbf{Олег Климов} По Довлатову было 4 миллиона. Взятых, кстати, с потолка.

\iusr{Олег Климов}
\textbf{Александр Хургин} хочешь об этом поговорить?)

\iusr{Сергей Брандыс}

"Мы не выбирали ни страну, где родимся, ни народ, в котором родимся, ни время,
в котором родимся, но выбираем одно: быть людьми или нелюдями." — Патриарх
Сербский Павел

\end{itemize} % }
