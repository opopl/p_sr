% vim: keymap=russian-jcukenwin
%%beginhead 
 
%%file 08_12_2020.news.ua.strana.terehova_ekaterina.1.vaccine_moskva
%%parent 08_12_2020
 
%%url https://strana.ua/news/305309-vaktsina-sputnik-v-ot-koronavirusa-protivopokazanija-komu-delajut-besplatno-tsena.html
 
%%author Терехова, Екатерина
%%author_id terehova_ekaterina
%%author_url 
 
%%tags 
%%title 42 дня трезвости после прививки. Как в Москве началась массовая вакцинация от коронавируса
 
%%endhead 
 
\subsection{42 дня трезвости после прививки. Как в Москве началась массовая вакцинация от коронавируса}
\label{sec:08_12_2020.news.ua.strana.terehova_ekaterina.1.vaccine_moskva}
\Purl{https://strana.ua/news/305309-vaktsina-sputnik-v-ot-koronavirusa-protivopokazanija-komu-delajut-besplatno-tsena.html}
\ifcmt
	author_begin
   author_id terehova_ekaterina
	author_end
\fi

\ifcmt
pic https://strana.ua/img/article/3053/9_main.jpeg
caption Прививка от коронавируса в России. Фото: Российская газета 
\fi

В минувшую субботу, 5 декабря, в России - прежде всего в Москве и области -
стартовала массовая вакцинация от коронавируса.\Furl{https://strana.ua/news/304926-nachalos-vaktsinirovanie-sputnikom-v-v-moskve-otkryty-70-kabinetov.html}

Третья фаза испытаний завершилась. Де-факто российская столица стала первым
городом мира, где прививают не добровольцев для исследования, а уже население. 

\enquote{Страна} рассказывает, что известно об иммунизации в Москве: как она проходит,
кто в первых рядах, сколько людей хотят вакцинировать, цена инъекций.

Дополнительно мы публикуем финальные данные о противопоказаниях и побочных
эффектах вакцины \enquote{Спутник V}. В том числе - можно ли пить после прививки
алкоголь. Эти сведения были обнародованы в канун общегосударственной
прививочной кампании. 

\subsubsection{Кому делают прививки от ковида в Москве}

Третья фаза клинических испытаний вакцины от коронавируса Спутник V завершилась
18 ноября.
\Furl{https://strana.ua/news/302742-vaktsina-ot-koronavirusa-sravnenie-pfizer-i-biontech-moderna-astrazeneca-i-sputnik-v.html}

Окончательные результаты эффективности направят на публикацию в международные
медицинские журналы в конце декабря. При этом прививочная кампания уже
стартовала в Москве и Подмосковье. На первом этапе прививки будут делать
\enquote{группам риска}. Это:

\begin{itemize}
\item медицинские работники, 
\item педагоги из вузов, техникумов, училищ, школ, воспитатели садиков,
социальные работники. 
\end{itemize}

Врачей и учителей принимают как из государственных, так и частных учреждений.
Для прививки нужно принести справку с места работы. 

Записаться на прививку можно только в электронном виде, на сайте мэра Москвы.
Запись доступна за две недели вперед. Прививки делают ежедневно с 8:00 до
20:00.

Сейчас открылись 70 кабинетов для прививок. Постепенно, обещают власти, их
число будет расширяться. Также будет увеличиваться категория первоочередников
(например, результаты исследований показывают, что в России повально заражаются
вирусом сотрудники транспорта и торговли). 

И только после этого прививать будут всех желающих. 

Власти утверждают, что с начала года мероприятия распространятся на другие
города. При этом без привязки к регионам уже начали вакцинировать сотрудников
вооруженных сил - весь личный состав, у кого нет противопоказаний. 

Кстати, в Украине похожая очередность будущей иммунизации от ковида. Вместо
учителей в первую партию попадут военные. Заключенные получили приоритет перед
профессиональными группами, у которых высокий риск заболеть. Подробнее - об
очередности вакцинации в Украине от ковида.\Furl{https://strana.ua/news/305289-stalo-izvestno-kak-ukraintsev-budut-vaktsinirovat-ot-koronavirusa.html}

Разница, впрочем, в том, что никакой вакцины в Украине пока нет. Но вернемся к
российским реалиям. СМИ сообщают о двух важных элементах вакцинации. 

Во-первых, для медиков столичного региона прививка от ковида не была дефицитной
и ранее. Все желающие врачи уже имели сделать возможность сделать прививку в
том числе в ходе испытаний. Так что эта группа риска во многом уже охвачена
\enquote{Спутником}.

В связи с этим на первый день иммунизации медперсонал Москвы и области был уже
более-менее привит.

Во-вторых, жители столицы сообщают журналистам, что уже вакцинируют не только
категории риска. По словам москвичей, любой человек может прийти и в порядке
живой очереди получить вакцину в своей поликлинике (если она там есть). 

\subsubsection{Противопоказания и побочные эффекты вакцины Спутник V}

В Москве все учителя, врачи и соцработники, которые захотят получить вакцину,
должны соответствовать критериям. Это финальные показания и противопоказания
вакцины Спутник V: 

\begin{itemize}
  \item не болеть ковидом сейчас и ранее,
  \item не болеть ОРВИ в течение последних двух недель и на момент вакцинации, 
  \item не быть беременной, не кормить грудью,
  \item возраст - от 18 до 60 лет, 
  \item не иметь серьезных хронических заболеваний, 
  \item не делать какие-либо прививки в последние 30 дней,
  \item не участвовать в клиническом испытании вакцины в качестве добровольца, 
  \item быть прикрепленным к любой московской поликлинике. 
\end{itemize}

Окончательные возможные последствия прививки могут быть такие в течение первых 1-3 дней:

\begin{itemize}
  \item слабость, недомогание, общая усталость, тошнота (у 10\%), 
  \item озноб, чувство жара, ломота в теле и мышцах, головная боль, повышение температуры выше 37 градусов (у 5,7\%), 
  \item боль, зуд, отек и покраснение в месте вакцины (у 4,7\%), 
  \item насморк, заложенность носа, першение в горле (у 1,5\%), 
  \item повышение давления, учащение пульса (менее чем у 1\%). 
\end{itemize}

Самый серьезный - редкий - побочный эффект, который прогнозируют врачи -
повышение температуры выше 38 градусов. В этом случае врачи рекомендуют
самостоятельно принять ибупрофен или парацетамол. Если же температура выше 39
градусов и не снижается через четыре часа после жаропонижающего, советуют
вызывать скорую. 

\ifcmt
pic https://strana.ua/img/forall/u/10/91/%D0%B2%D0%B0%D0%BA%D1%86%D0%B8%D0%BD%D0%B0(4).jpg
\fi

\subsubsection{42 дня без алкоголя после прививки }

За день до старта массовой вакцинации в Москве вице-премьер Татьяна Голикова
сообщила об одном противопоказании, о котором ранее особо не говорилось. А
именно - после прививки на протяжении 42 дней не пить алкоголь. 

На протяжении 42 дней, поскольку вакцина двухфазная. Второй укол делается через
21 день и еще столько же нужно на полную выработку антител. Не употреблять
алкоголь рекомендовано из-за того, потому что он угнетает иммунную систему и
может замедлить работу вакцины. 

Голикова сказала о спиртом среди прочего: \enquote{...Воздержаться от посещения мест с
большим количеством людей, соблюдать масочный и режим гигиены, минимизировать
контакты, соблюдать социальную дистанцию, ограничить принятие алкоголя и
лекарств, которые угнетают иммунную систему...}

Но россиян, особенно в ключе грядущих новогодних праздников, заинтересовал
именно пункт про алкоголь. Но многие восприняли это не всерьёз, а как шутку.

Так, на \enquote{Эхе Москвы} провели опрос: \enquote{Вы можете прожить 42 дня без алкоголя?}
Ответы, кстати, вполне обнадеживающие. Могут - 77\%, нет - 19\%. 

\ifcmt
pic https://strana.ua/img/forall/u/10/91/%D0%A1%D0%BD%D0%B8%D0%BC%D0%BE%D0%BA_%D1%8D%D0%BA%D1%80%D0%B0%D0%BD%D0%B0_2020-12-07_%D0%B2_22.54_.02_.png
caption Вы можете прожить 42 дня без алкоголя?
\fi

А российский журналист Екатерина Петухова в шутку так прокомментировала
рекомендацию Голиковой: \enquote{Массовая вакцинация с треском провалилась, не
успев начаться}. 

\ifcmt
pic https://strana.ua/img/forall/u/10/91/%D0%A1%D0%BD%D0%B8%D0%BC%D0%BE%D0%BA_%D1%8D%D0%BA%D1%80%D0%B0%D0%BD%D0%B0_2020-12-07_%D0%B2_23.04_.27_.png
\fi

\subsubsection{Сколько человек хотят привить от ковида в России}

В России еще с советских времен вакцинация распространена широко. Многие
прививок входят в график и являются рекомендованными. Среди них - грипп. 

За три месяца - с 1 сентября по 1 декабря 2020 года - прививку от гриппа
сделали 67 миллионов россиян (это 46,2\%). Будет ли достигнут такой же
показатель по коронавирусу - неизвестно. 

По данным опроса правящей партии \enquote{Единая Россия}, 73\% граждан не готовы к
вакцинации от ковида или у них отсутствует такая необходимость. Хотят сделать
прививку 23\%. 

Во многом это объясняется тем, в России во многих регионах высокий
популяционный иммунитет и граждане знают, что уже переболели. Например, по
данным мэра Москвы, естественным образом антитела к ковиду получили около 50\%
москвичей. Им прививки в ближайшее время уже не нужны. Соответственно, в
столице планируют рассчитывать вакцину на 6-7 миллионов (при населении 12,6
миллионов). 

На данный момент в России объявляют, что прививка от ковида будет добровольной.
Но, судя по некоторым примерам, можно предположить иное развитие событий. Как
происходит вакцинация в целом \enquote{Стране} рассказал сотрудник московского
госпредприятия Михаил. 

\enquote{Ежегодно мы проходим медкомиссию. Там в том числе определяют, кому какие
прививки нужны. В прошлом году у меня не нашли антител к кори и отправили на
ревакцинацию. В этом году с перерывом в месяц делали против гепатита и гриппа.
Вероятно, я могу и не хотеть эти прививки. Но без них не допустят к работе.
Полагаю, что в будущем году это же нас ожидает и с ковидом}, - высказал свое
мнение Михаил. 

\subsubsection{Сколько стоит прививка \enquote{Спутником}}

Вакцина от коронавируса тоже внесена в график прививок в России. Это означает,
что для граждан и иностранцев с ВНЖ прививка на постоянной основе будет
делаться бесплатно. 

При этом власти озвучили коммерческую стоимость прививки - для продажи за
границу и реализации внутри страны. Цена - 1942 рубля за два компонента (нужны
для прививки одного человека). В украинской валюте по сегодняшнего курсу это
740 гривен, в американской - 26 долларов. 

Пока, по крайней мере официально, вакцину от коронавируса нельзя сделать в
частной клинике в России. Со временем она будет доступна и там - за деньги, но
в любое удобное время и в том числе иностранным гражданам. 
