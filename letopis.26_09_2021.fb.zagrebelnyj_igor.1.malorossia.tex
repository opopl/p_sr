% vim: keymap=russian-jcukenwin
%%beginhead 
 
%%file 26_09_2021.fb.zagrebelnyj_igor.1.malorossia
%%parent 26_09_2021
 
%%url https://www.facebook.com/permalink.php?story_fbid=3113446362313329&id=100009439885823
 
%%author_id zagrebelnyj_igor
%%date 
 
%%tags __sep_2021.usik.pobeda.dzhoshua,istoria,maloros,malorossia,ukraina,usik_aleksandr
%%title У зв'язку з деякими подіями у світі спорту простір соцмереж знову наповнився терміном "малорос"
 
%%endhead 
 
\subsection{У зв'язку з деякими подіями у світі спорту простір соцмереж знову наповнився терміном \enquote{малорос}}
\label{sec:26_09_2021.fb.zagrebelnyj_igor.1.malorossia}
 
\Purl{https://www.facebook.com/permalink.php?story_fbid=3113446362313329&id=100009439885823}
\ifcmt
 author_begin
   author_id zagrebelnyj_igor
 author_end
\fi

У зв'язку з деякими подіями у світі спорту простір соцмереж знову наповнився
терміном "малорос".

Зрозуміло, що коли "ліцензовані патріоти" вживають слово "малорос", вони
абстрагуються від низки історичних сенсів і просто хочуть позначити цим
терміном українця, який - на їхню думку - є лояльним до Росії. Проте таке
абстрагування виглядає аж занадто грубо. Наприклад - з огляду на те, що якихось
двісті років тому поняття "малорос" могло позначати "хиткість" у ставленні до
імперської столиці, а поняття "українець" - навпаки зразкову вірність
Російській імперії.

\ifcmt
  pic https://scontent-mxp1-1.xx.fbcdn.net/v/t1.6435-9/242863206_3113432132314752_7175361670701108403_n.jpg?_nc_cat=105&ccb=1-5&_nc_sid=730e14&_nc_ohc=xuLrusTr5c4AX-T0Nu3&_nc_ht=scontent-mxp1-1.xx&oh=f0c18cfce0e42ba941ae4b28598684ae&oe=6190451F
  @width 0.8
\fi

На межі XVIII i XIX століть топонім "Малоросія" стосувався передусім земель
історичної Гетьманщини. Поняття "Україна", як відомо, впродовж століть
"мігрувало" з заходу на схід, і в зазначений час воно стосувалося передусім
Слобожанщини. Власне, тоді існувала адміністративно-територіальна одиниця
Слобідсько-Українська губернія.

Між елітами Малоросії-Гетьманщини та України-Слобожанщини була певна напруга.
Еліти Гетьманщини керувалися прагненням приєднати Слобожанщину, але це їм не
вдалося. Водночас вони дивилися на слобожанців як на таких собі "несправжніх
козаків", плебеїв, які занадто легко скористалися соціальним ліфтом. Так,
якийсь невідомий нині автор із Полтавщини на межі століть писав про
слобожанців:

Пасіть собі овець, оріте, чумакуйте,

А лізти в козаки і думать не турбуйтесь.

Селіте слободи, але платіть нам чинш!

А ні, дак із степів погоним вас, киш-киш!

Слобідські еліти теж не вирізнялися любов'ю до своїх сусідів. По-перше, вони
відстоювали своє шляхетство, в ідеалі - наголошуючи, що їхніх предків
нобілітували ще польські королі. По-друге, посилаючись на історичні факти,
ставили під сумнів вірність малоросів царственному дому й імперії,
звинувачували їх у сепаратистських тенденціях. Водночас вони наголошували, що
самі слобожанці-українці завжди були вірними царям та імператорам.

А в цілому - українцям варто навчитися шанувати власну історію. Справжню
історію у її властивих контекстах, а не байки про перманентне "поневолення" та
"визвольну боротьбу".

\ii{26_09_2021.fb.zagrebelnyj_igor.1.malorossia.cmt}
