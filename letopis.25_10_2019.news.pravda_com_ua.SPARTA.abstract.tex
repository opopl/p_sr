% vim: keymap=russian-jcukenwin
%%beginhead 
 
%%file 25_10_2019.news.pravda_com_ua.SPARTA.abstract
%%parent 25_10_2019.news.pravda_com_ua.SPARTA
 
%%url 
%%author 
%%tags 
%%title 
 
%%endhead 

{\em
В селе Караван, в получасе езды от Харькова, находится коммуна "С.П.А.Р.Т.А."
или "Сельскохозяйственная Поэтизированная Ассоциация Развития Трудовой
Активности". 

Участники этого объединения напоминают одержимых персонажей фильмов Эмира
Кустурицы: они доят коров и заседают в "парламенте", штрафуют за курение и
употребление алкоголя, бегают 100 километров, в обязательном порядке сочиняют
стихи, возводят в культ Ленина и высчитывают уровень счастья по специальной
формуле. 

Несмотря на принудительное лечение, уголовные преследования и отсутствие
элементарных удобств, "С.П.А.Р.Т.А." существует уже 23 года, а само движение и
вовсе берет начало в Советском Союзе.

УП побывала в "С.П.А.Р.Т.Е." и поговорила с ее жителями об устоях и идеологии
их организации, коммунизме и демократии, Горбачеве и Ельцине, войне и мире, ЗОЖ
и вредных привычках, многолетнем давлении властей и поиске рецепта счастья.
}
