% vim: keymap=russian-jcukenwin
%%beginhead 
 
%%file 30_11_2021.fb.bilchenko_evgenia.1.novyje_filmy
%%parent 30_11_2021
 
%%url https://www.facebook.com/yevzhik/posts/4490958760939199
 
%%author_id bilchenko_evgenia
%%date 
 
%%tags bilchenko_evgenia,chernobyl.film.kozlovskij_danila,film,film.opolchenochka,kino,kultura,rossia
%%title Еще немного о новых фильмах
 
%%endhead 
 
\subsection{Еще немного о новых фильмах}
\label{sec:30_11_2021.fb.bilchenko_evgenia.1.novyje_filmy}
 
\Purl{https://www.facebook.com/yevzhik/posts/4490958760939199}
\ifcmt
 author_begin
   author_id bilchenko_evgenia
 author_end
\fi

Еще немного о новых фильмах.

Залпом посмотрели "Ополченочку", "Небо", "Я подарю тебе Победу", "Чернобыль"
(наш).

"Ополченочка" - очень любительская, наивная, по сценам насилия терпеть можно,
ничего такого, чтобы охать и ахать (ну, это кому как, я терпеливая, но очень
аматорски снята, зато искренняя и честная, жесткая, как мы любим).

\ifcmt
  ig https://scontent-frx5-1.xx.fbcdn.net/v/t39.30808-6/262837628_4490970730938002_8963046827421276509_n.jpg?_nc_cat=110&ccb=1-5&_nc_sid=730e14&_nc_ohc=SfKrw1xScksAX8lFLck&_nc_ht=scontent-frx5-1.xx&oh=7d5247a65ac1cd0dcc6b7644bd81b404&oe=61AD2D7D
  @width 0.4
\fi

"Небо" - о военных летчиках в Сирии - идеологически довольно пафосно, много
сцен в стиле комиссара Катани, но мне зашло, потому что похоже на похороны
деда. Напомнило "Солнцепек", но больше мещанских историй: возможно, фильм
больше воспитательный для новичков, чем просто дух, чистый дух, в стиле
советских "Офицеров", с которым его сравнивают. Потому моему военному другу не
зашло "Небо". Но свое предназначение фильм выполняет: проверено на новичках.

"Я подарю тебе Победу" - очень понравилось, потому что показаны три хронотопа:
реальный честный военный, воображаемые лихие девяностые и товарно-символический
бездушный московский постмодерн. И как из Победы устраивается сувенир для
олигархов. Очень важно, что культурный код войны показан сквозь призму
современности. История просветлевшего после пережитого олигарха (Башаров) и
честного чекиста (Гармаш) - вроде трюизм, но мне понравилось. Единственное, что
нынче все на фронте крестятся, не уверена, что так было, но для соборности
традиций - оно так лучше.

"Чернобыль" - то ли поздно смотрели, то ли не зашел. Все вроде, как дед и
рассказывал: и про пожарников, и про военврачей под крышей, которые забирали, и
про седьмую отметку, и про реактор. И лучевая хорошо показана, и бэры правильно
выставлены, и тот конфликт насчет гражданских имел место (у деда имел, дед
видел иную схему ликвидации, с большим участием военных), и про добровольцев
верно всё. Но как-то в жизни всё живо, а тут - нет. Хотя тоже фильм ничего так.

Все четыре фильма рекомендую: для разных мнений. Фото из архива деда: очередная
операция по спасению пораженных лучевой болезнью. У деда тоже она была, но ни в
какую Швейцарию его не увезли, а, когда началась рвота и пневмония (догнала
через 6 дней), то в Москву повезли, а в кино всех тяжелых в Швейцарию почему-то
везут....

\ii{30_11_2021.fb.bilchenko_evgenia.1.novyje_filmy.cmt}
