% vim: keymap=russian-jcukenwin
%%beginhead 
 
%%file lysty.olga_roshkevych.20_09_1878
%%parent lysty.olga_roshkevych
 
%%url https://www.i-franko.name/uk/Corresp/1878/1878-09-20.html
 
%%author 
%%author_id 
%%author_url 
 
%%tags 
%%title 
 
%%endhead 

\subsection{20.09.1878}
\Purl{https://www.i-franko.name/uk/Corresp/1878/1878-09-20.html}

20.09.1878 р. До О. М. Рошкевич       

20 вересня 1878 р. Львів                    

Дорога, кохана Олю!                         

Ну, хороші ми, мужчини! Коли о то йде, щоб другому зробити яку-небудь, хоч і
невеличку, приємність, то у нас все тисячні якісь перешкоди, обставини,
чортзна-що, і все воно так складається, що нібито не наша вина, аби тільки не
зроблено було, що треба і в час! А коли о то йде, щоб нам хто зробив щось
доброго, то нетерпеливимося, ходимо, мов приголомшені, ждемо й ще готові
гніватись, коли добро опізнюється. Таке саме й зо мною. Тих дві неділі по
приїзді О[лесько]ва і я ждав ще одного листа від тебе, а сам боявся писати на
Славкове ім’я, щоб хто не перехопив листа.

Але ось Славко приїхав і привіз мені твоє любе, сердечне письмо, котре живо і
ярко відкликало в моїм серці і тебе, і твою щиру любов, – а ось уже більше як
тиждень минув, а ти ще ждеш надарма моєї відповіді. Правда, не моя в тім вина,
і ти можеш собі подумати, який я був лихий на Славка, що на такий довгий час
поїхав до Олеськова, – але все ж таки, коли собі погадаю, як ти там у своїй
пустині нетерпеливо вижидаєш його приїзду і кілько при тім натерпишся, то й сам
не знаю, що з собою зробити. Але що ж, се все нічого не поможе – ссе мусить іти
своєю природною стежкою, будь воно приємне для нас або й ні. Однако, щоби
чим-небудь надгородити тобі таке довге ожидания, хочу написати обширно про своє
життя і загалом про все, що тебе може обходити.

Правда, так дуже нового нічо не можу тобі сказати, бо що ж можуть бути за
новини в такім житті, як моє, де день, до дня подібний, плине незначно, а
швидко серед праці, читання, біганини. Внішні обставини, значить, остаються
більше-менше незмінні, а змінюватися може тільки другий елемент – думки, плани,
погляди.

Власне приходить мені на думку, що мої листи до тебе, якби їх звів докупи і
прочитав одним тягом, представили би дуже чудацьку цілість. З одного боку,
нудна одностайність фактів і обставин, а з другого – аж надто велика
змінчивість планів, намірів, предпринять, котрим, бачиться, завчасу судилось
остатися мертворожденними дітьми фантазії, а не переходити в тіло і кров. Я
знаю, що сеся послідня проява мого темпераменту тобі не подобається (я не раз у
твоїм голосі чув ніжну іронію при споминках о подібних планах), і переконаний
твердо, що в тім зовсім твоя правда, тому ж і не хочу тобі за те робити
найменшого закиду.

Противно, я й сам після довгої і мучачої аналізи власного темпераменту дійшов
до того, що сесі плани та думки дійсно не мають реальної підстави; але дальша
аналіза показала мені, що 1) вони конечний виплив іменно того одностайного на
вні, а більше сидячого спекулятивного життя, в котрім думка мусить сповнювати
ширшу діяльність, ніж тіло, і 2) що ті плани, так змінчиві і несталі, свідчать,
однако, о тім, що я ще зовсім не стратив надії на будучність, що можу ще
оглядатися, розважувати, що, значить, не кидаюся насліпо до незвісної цілі, як
се думає о нас, соціалістах, багато людей з нащої ніби інтелігенції.

Приходжу тепер до найважнішої часті мого письма. Хочу іменно коротенько розібрати і показати тобі саму суть моїх переконань, головну пружину моєї теперішньої і будущої діяльності. До сього склонює мене багато причин, а найважніші ось які дві:

\begin{itemize}
\item 1) Вже звиш півтора року минуло, як ми були довший час разом і мали
спосібність ділитися своїми думками і переконаннями. Від того часу
багато дечого змінилося у нас обоїх, тож час тепер наново розібрати все
докладно, розглянути совісно, наскільки годяться наші переконання, щоб
на будуще уникнути заводів і неприятностей. Правду сказавши, і давніше,
будучи ще разом, ми дуже мало говорили о тих предметах і багато важких
питань закривала нам з-перед очей особиста симпатія: ми любили, а не
філософували, були щасливі самі собою і забували о внішнім світі і
важкій боротьбі, котра нас ждала.

Внішній світ і боротьба аж надто швидко збудили нас із щасливого короткого сну,
пригадали нам, що ми ще не в соціалістичній державі, де свобідно і весело, а
серед теперішніх, жидівсько-конституційних порядків. Чи добре, чи зле се було,
що так швидко збуджено нас із щасливого сну, – сього не буду розбирати. Факт
стався, і факт, думаю, не пройшов дарма, а в твоїм і моїм умі полишав глибокі
сліди. Ми дозріли серед тяжких проб і можемо тепер далеко сміліше і певніше
аналізувати свої переконання, абстрагуючи від них личну любов і симпатію. Се
перша причина.

\item 2) Кілька разів чув я від тебе слова: покинь тоту роботу, віддайся мені,
старайся наперед злучитися зо мною, а тоді вже побачимо, що далі
робити. Ти не знаєш, скільки важкої боротьби причинили мені тоті слова,
і, безперечно, були вони причиною не одного разячого дисонансу в моїх
листах, до тебе писаних. Думка, що я для тебе мав би покинути своє
переконання, видавалася мені такою дикою, негідною тебе і мене, що не
раз бували хвилі (в тюрмі), коли я насилу відганяв від себе твій образ.
Даруй мені, моє серденько, що се так було, – але ти видиш, я хочу тобі
сказати всю правду, будь вона хоч і яка важка. Впрочім, будь певна, що
тепер такий гнів на тебе так далекий від мене, як небо (неіснуюче) від
землі. 
\end{itemize}

Тепер я побачив, що ті слова твої могли бути не випливом егоїстичної
буржуазної любові, котра, крім свого щастя, не хоче нічого бачити, ні знати, –
а що вони радше були випливом надто великої старанності о моє власне добро – я
знаю се і тим сильніше люблю тя за те. Але тим сміліше можу тепер сказати, що
як колись ми будем жити разом (а я вірю твердо, що ми будем жити разом), а ти,
приміром, почнеш знов в’язати моє переконання і здержувати мене від зроблення
того, що ми велить робити совість, то я перестану тебе любити, покину тебе, не
питаючи на жодні побічні згляди. Для того, думаю, тим потрібніше тепер
прояснити нам обопільно свої переконання бодай в найзагальніших чертах, і то у
всіх головних питаннях.

Головні питання, від котрих може в данім разі залежати сякий або такий поворот у нашім житті, се питання 1) економічні, 2) політичні, 3) релігійні, 4) соціологічні, 5) властиво практичні, т. є. тикаючі щоденної нашої діяльності.

Я хочу тобі тут коротенько подати розбір тих питань і, так сказати, моє profession de foi в кожнім із них.

1) Я переконаний, що економічний стан народу – се головна підстава цілого його
життя, розвою, поступу. Коли стан економічний плохий, то говорити про поступ,
науку – пуста балаканка. Я переконаний, що теперішній економічний стан усіх
народів культурних дуже плохий у многих зглядах, а найбільше задля нерівності
маєткової і задля різкого розділення на касту багатих, котрі виключно
користуються добутками науки і культури, і касту бідних, котрих кожний новий
поступ тільки глибше притискає.

Такий стан економічний не є вічний і незмінний. Він повстав з бігом історичного
розвитку і так само мусить упасти, зробити місце другому, досконалішому,
справедливішому, більше людському. Я переконаний, що велика, всесвітня
революція поволі рознесе теперішній порядок, а настановить новий. Під словом
«всесвітня революція» я не розумію всесвітній бунт бідних проти багатих,
всесвітню різанину; се можуть під революцією розуміти тільки всесвітні рутенці,
плосколоби та поліцаї, котрі не знають, що, н[а]пр., винаходка парових машин,
телеграфів, фонографів, мікрофонів, електричних машин і т. д. спроваджує в
світі, хто знає, чи не більшу революцію, ніж ціла кривава французька революція.

Я розумію під революцією іменно цілий великий ряд таких культурних, наукових і
політичних фактів, будь вони криваві або й зовсім ні, котрі змінюють всі
дотогочасні поняття і основу і цілий розвиток якогось народу повертають на
зовсім іншу дорогу. Хто затим говорить о «будущій соціальній революції», той
дає собі свідоцтво бідності, показуючи, що не розуміє поняття революції.
Соціальна революція зачалася вже відтоді, відколи французька революція дала
панування капіталістам, видерши його родовій шляхті.

Від першої хвилі свого панування борються капіталісти з тою новою революцією,
проти котрої бліднуть усі факти і бої, які досі бачила історія. Але вони
борються з нею так, як той чорт з богом у Гете «Фаусті», котрий все злого хоче,
а все добре творить; так само й капіталісти помимо своєї затятості кожним своїм
словом, ділом, поступом допомагають тільки осущенню діла тої великої революції.
Що капіталістична революція 1789 відбулася так криваво – причина тому не в тім,
щоб кожна революція мусила така бути, а в тім, що тиранія пануючих шляхти і
попів не позволяла ши-рення освіти, котра одна могла злагодити страшний вибух,
коли тим часом другі причини: бідність, здирство і т.д. – приспішили той вибух.

Я переконаний, що послідній акт великої революції соціальної буде остільки
лагідніший, а тим самим розумніше і глибше переведений, оскільки освіта і наука
зможе прояснити масам робочого народу ціль і способи цілого діла. Відси
випливає проста задача для кожного, хто пізнає і прийме цілою душею тоту думку:
старатися ширенням здорових і правдивих понять о злагодженпя послідньої
революційної кризи. Як буде виглядати будущий лад суспільний – сього не мож
нині сказати, і воно, впрочім, нічо не становить.

Наука виказує тільки одну головну засаду: капітал продуктивний, т. є. земля,
машини і всі прилади до праці, сирий матеріал і фабрики мають бути спільною,
громадською власністю. Громади стоять до себе в стислім, дружнім (федеративнім)
стосунку, кожна вибирає собі заряд, котрий порядкує її господарські діла. Плоди
праці належаться вповні робітникові. Се головні, основні ідеї, узнані
теперішньою економічною наукою за ціль, до котрої повинна йти поволі економічна
реформа. На тім кінчу начерк економічних понять. Якби тобі що було неясно або
неповне, то напиши, а мені буде дуже приємно пояснити тобі, що треба, – а
особливо дуже мене то втішить, коли побачу, що ти думаєш над тими поняттями,
аналізуєш їх.

Я, правда, повинен би був уже давно післати тобі яку економічну книжку для
докладнішої інформації, але що ж, коли-бо наука економії (господарки народної),
хоч, безперечно, найважніша з усіх наук, досі ще зовсім не оброблена так, як би
того бажалося, а найкращі книжки економічні, які досі вийшли (Marx, «Kapital»,
Schäffle, «Bau und Leben des sozialen Körpers» і др.), писані так тяжко й
незрозуміло, що зовсім неприготованому читати і понімати їх – чиста
неможливість. А найліпше приготування до їх читання і понімання се, по-моєму,
власне думання над економічними питаннями. Впрочім, посилаю тобі тепер Becker’a
історію революції 1789 і № «Zukunft», історію змови Ваbeuf’a, щоби-с побачила
перші початки панування капіталу і перші самосвідомі кроки соціальної
революції.

2) Розбір питань політичних я зроблю якнайкоротшим. Питання політичні мають
тільки при теперішнім ладі велике значення, і то лиш остільки, оскільки тикають
чисто економічних сторін народного життя, напр., податки, рекрутчина, війни і
т. д. При соціалістичнім ладі, де всякі другі питання підпорядковані будуть
питанням економічно-культурним, питання політичні, само собою розуміється,
стратять твелику часть своєї ваги.

Замість податків теперішніх, ішла би в будущім ладі якась часть спільного,
громадського продукту на спільні (межигромадські, крайові, народні) діла,
напр., вищі школи, музеї, наукові предприняття, межинародний заряд. Замість
стоячих військ уорганізувала б ся обивательська міліція, котра в разі нападу
яких диких орд могла б стати в кожній хвилі до оборони свого краю. Народи
культурні вступають з собою в вічну федерацію, а всякі можливі їх звади
залагоджує межинародний виборний суд. Всякі класові привілеї і нерівності
щезають: єдине джерело заслуги становить більша спосібність і більша для
громади корисна праця. Я переконаний, що при такім ладі зможуть далеко скорше і
свобідніше розвиватися поєдинчі народності і що, значить, соціалістичний лад
зовсім не противний національному розвиткові.

Се мої думки про політичний лад на будуще. Але при теперішніх порядках питання
політичні виглядають зовсім інакше і вимагають від соціалістів такого програму
ділання, котрий би міг їм і тепер з’єднати вагу в державі і міг загарантувати
супокійне ширення їх переконань і заразом систематичне вводження в життя таких
реформ, котрі би приготовували оконечний, спокійний перехід до нового ладу.
Способи до того ось які:

a) зав’язання оскільки мож сильної, організованої, явної партії соціалістичної, котра би, користаючи з теперішніх конституційних установ, вводила своїх послів до сейму;

b) переведення загального, безпосереднього голосування при виборах до сейму;

c) державний кредит для нижчих клас робучих і організація робітницьких спілок,
котрі би могли стати предпринимателями державних робіт (будування залізниць,
пароходів і др.) замість теперішніх спілок акційних;

d) повільна експропріація (вивлащення) більших фабрик, посілостей, домів по
містах, т. є. переведення їх з приватних рук у руки громадські або державні і
винаймування робітникам або робітницьким спілкам по низьких, означених цінах.
Се реформи, до котрих повинна йти теперішня внутрішня політика і котрі дійсно
поволі заводять у себе найпередовіші народи: Англія, Америка, Франція.

Думаю, що се вистачить тобі до ясного поняття мого політичного програму. Я не
кажу, щоб такий програм був легкий до сповнення або щоб ми могли побачити
швидко його сповнення, – кажу тільки, що се тота далека ціль, котру все і
всюди, при кожнім поступі мусимо мати на оці.

3) Приходжу до третьої, в нашім часі, може, найдразливішої групи питань, а
іменно релігії. Ти сама легко піймеш, що для нас, соціалістів (думаю, що можу і
тебе обняти тою назвою), питання се іменно найменше дразливе. Соціалізм же
основується на неограниченій свободі личності під зглядом переконань, –
значить, не може ж видавати війни релігійним переконанням, не може на силу
відбирати їх нікому, коли вони кому дорогі або святі. Але, з другого боку, та
сама свобода і толеранція вкладає на соціалістів і другі обов’язки, а іменно:

а) щезає всяке поняття релігії державної; узнаючи одну релігію як пануючу, тим самим ограничується, принижує другі. Те саме веде за собою і другу консеквенцію,

б) релігійні корпорації (церкви, парохії, дієцезії) не можуть мати характеру публічного, урядового; служителі церкви, попи, біскупи, не можуть побирати пенсій з публічного скарбу, а мусять жити або з власної праці, як усі другі обивателі, або з добровільних датків своїх вірних;

ц) з публічного виховання наука релігії виключається або, в крайнім разі, толерується яко надобов’язковий предмет у вищих класах (тоді, коли молодий чоловік спосібний розуміти всяку філософію).

Се мої думки о загалі. А спеціально щодо моєї особи ти знаєш мої задивлювання
на релігію, і про них тепер не буду говорити, щоб не затягати надто завдовжки
сього письма. Додам тільки те, що як у всякім іншім згляді, так і в релігійнім
я не думаю мовчати і таїти своє переконання, обзираючись на побічні обставини,
на крики поліції, духовенства і пр. Правда над усе!

4) Лишаючи питання релігійні, яко найменше спорні, набоці, переходжу до
дальшої, для нас, молодих людей, найважнішої або, властиво, найцікавішої групи
питань – соціологічних, се значить, питань, розбираючих форми спільного життя
людей, розвитку, виховання і т. д. Для легшого перегляду розділю сю групу на
кілька менших і розповім по черзі свої погляди 1) на подружжя, 2) на виховання
дітей, 3) на обопільне становище мужчини і женщини.

Сесі три питання не становлять ще цілості сеї групи, але вони в ній найважніші,
як се дуже легко вирозуміти, бо сяке або таке їх рішення надає сякий або такий
вид цілому суспільному життю: роблять його або безконечною мукою, або
свобідним, веселим і щасливим в собі помимо боротьби і невзгод навні. Подружжя,
після моєї думки, треба все вважати з двох зглядів: яко акт релігійний,
формальний, але при теперішніх обставинах посередньо досить важний, і яко акт
соціологічний, означаючий сполучення двох людей спільною волею і симпатією до
спільної, обоїм любої праці. Розуміється само собою, що другий пункт далеко
важніший від першого, а його маловаження страшно мститься на теперішній
суспільності.

Тепер починається вже реформа на тім полі, хоть дуже слаба. Держава сама поволі
переносить пункт тяжкості з формальної, релігійної сторони на властиву, живу,
соціологічну сторону, позволяючи розводи і цивільні шлюби. Однако розводи і
цивільні шлюби, по-моєму, тільки посередні кроки, ведучі до дальших реформ
подружжя, а мале їх практикування свідчить, що стежка ся не зовсім властива.

Перша річ – подружжя, яко акт соціологічний, повинно бути під публічною,
державною контроллю а зовсім вийти з-під релігійної опіки. Як воно уладиться в
будущім порядку, тяжко знати. Для нас головне діло – розважити добре всі тоті
обставини, при котрих і серед нинішнього стану можна жити щасливо (в моральнім
згляді) і, стаючи мужем і жоною, не переставати бути свобідними людьми.

Вже з вищесказаного мож легко вирозуміти, о що найбільше ходить. Треба, щоб
обоє ті люди, що лучаться з собою, були оскільки мож належито розвинені, щоби
їх темпераменти були згідні і любов настільки сильна, щоб не щезала за
першим-ліпшим прозаїчним, щоденним випадком: треба любові здорової, органічної,
котра, не ідеалізуючи любимий предмет, не строячи го в небувалі прикраси,
розпадається при першім дотику дійсного життя. Любов правдива може повстати у
чоловіка здорово і нормально розвиненого, вона спокійна, чиста, похожа більше
на щиру приязнь, на почуття своєї рівності і солідарності з коханим предметом –
і така любов спосібна перетривати всякі нещастя, – бо вона одно з люблячих
робить для другого конечним, природним складником життя, так як воздух, хліб,
книжку, працю. Се було б одно.

А друге, що потрібно, – се високої, гуманної і чесної цілі, за котру б тото
подружжя боролося спільною силою весь вік; тота боротьба одна може вічно
піддержувати їх любов, бо, люблячи свою спільну ціль, мимоволі любиться і
ціниться кожного, хто йде до тої самої цілі. Ти бачиш, дорога Ольдзю, що по
моїм поняттям до такого щастя соціалістам дорога дуже легка, хоть зато більше
для них трудності і горя в чисто матеріальних зглядах. А третій варунок
щасливого подружжя при теперішніх обставинах – се обопільна свобода ділання:
одно не повинно в’язати другого, повинно бути вирозуміле на всякі похибки
другого, від котрих не свобідний ні один чоловік, – загалом, повинні поступати
приязно і тактовно.

Переходжу до виховання дітей, речі, не менше важної від попередніх, становлячої
одну з найважніших задач усякого подружжя, одну з головних основ родинного
щастя. Не буду тут споминати деяких поглядів новіших і давніших учених, котрих
розбір відкладаю на пізніше, коли буде до того ліпша спосібність, а подам
тільки деякі мислі загальні, котрі, після моєї думки, повинні становити основу
виховання дітей.

Передовсім, як ми се вже говорили усно, діти повинні швидше вчитися читати,
рахувати, мислити, ніж молитись. Але наука мусить іти в парі з розвиненням
тіла, до того служить не безплідна гімнастика, а праця фізична. Розуміється, о
виховуванні в яких-небудь передсудах релігійних, моральних, суспільних не
повинно бути й бесіди. Ті добутки, ті правди і засади, котрі нам приходиться
добувати тяжким трудом, кривавою боротьбою цілого життя, для них повинні
статися чимось нормальним, конечним, таким, що розуміється само собою, о чім
ніщо й розправляти. Тільки таким способом мож виховати правдиво поступове і
сильне покоління. Розуміється само собою, що, виховуючи дітей відмалу в
поняттях розумних, поступових і научних, не треба їм затикати очей і на поняття
і докази противників тих ідей, але розвивати якнайвсесторонніше їх мислення.

Переходжу тепер до третього пункту, а іменно до відносин між мужчиною і
женщиною. Вже з досі сказаного легко вирозуміти, що я признаю для женщини право
і обов’язок зовсім рівного розвою і становища в суспільності, як і для мужчини.
Ти сама мала спосібність не раз переконатися доочне, як я відношуся до всяких
церемоній і конвенієнцій, котрими обставлена зв’язь мужчини з женщиною, – так
само й я переконався, що й ти не менше розумно і ліберально відносишся до
всього того, що заключаемся в пустій фразі «не випадає».

Думаю, впрочім, що сесь пункт у нашім будущім житті буде нам робити найменше
трудності, бо щодо мене, то вже сам мій темперамент, м’який і податливий,
далекий від усякої тиранії. Одного тільки боюся, а іменно того: не раз мені
лучається, живучи ближче з деякими людьми, що коли вони через яке-небудь глупе
поступування стратять у мене поважання, то я стаю для них хвилями дуже прикрий,
несправедливий, їдкий; те поступування пізніше болить мене самого, коли розважу
все ближче, – але в даній хвилі не можу зміркуватися. Се одна з моїх
найпоганіших хиб, така хиба, над котрої усуненням я працюю від кількох літ, а
все надарма. Але я надіюся, що, живучи з тобою, не тільки не буду мав ніколи
причини бути для тебе несправедливим і відмовити тобі свого поважання (що
затруло би і моє життя, і твоє), але, що противно, твоє тактовне поступування
зможе відучити мене від тої поганої хиби.

На тім кінчу своє confession de foi. Правда, я ще навів п’ятий пункт, котрий
може мати вплив на наше спільне життя, ба, котрий по своїй природі мусить на
нього мати дуже великий вплив, але про нього тепер ніщо говорити. Як буде час,
і спосібність, і потреба, то скажу й про нього дещо.

Видиш, любочко, ти в своїм листі питаєш мене, чи не знудив мя твій лист, –
псотнице! – а я отсе насеріо взявся занудити тебе і вибрав до того найліпший
спосіб – наукову розправу! Ось що значить нудити! Але я думаю, що все-таки сей
лист може тобі дещо трохи придатися, бо, може, з одного боку, розсіє деякі
ілюзії, а з другого, скріпить твою віру в мене і в ті принципи, за котрі я хочу
боротися до послідньої хвилі життя.

А знаєш, я собі одно викидаю раз у раз: чому я не можу або не вмію писати таких
ніжних, теплих, сердечних листів, як ти, чому навіть такі слова і звороти не
йдуть мені якось під перо? Не раз думаю, що ти готова з того виносити, що я
холодний, що не люблю тебе або перестаю любити, і тота думка мучить мене ще
гірше. Але ні, – правда, моя кохана Олю, ти так не думаєш? Ти знаєш, що я тебе
люблю так само щиро і гаряче, як ти мене, ти знаєш, що чуття моє, чим гарячіше
і глибше, тим менше може переливатися в слова, ти знаєш, що найщиріше цілується
мовчачи.

Рад би-м написати тобі про моє життя, але й так уже сей лист такий довгий, що
гріх навіть тягнути го довше і заповнювати пустими фразами. Перейду до важніших
речей. Ти пишеш, що Славко, бачиться, не буде сього року служив у війську, а
мені бачиться, що буде, – ба навіть він казав, що за тиждень приїде сюди, і то
він, тато і ти! Чи се мала б бути правда, чи тільки містифікація? Впрочім,
надто дуже твоїм приїздом в товаристві тата я не можу тішитися, бо знаю, що се
додало би нам тільки жалю: видітися, поговорити, побути з собою ми не могли б.
А впрочім, се не має зовсім значити: не приїзди! Противно! Хоть раз тебе
побачити на улиці – все-таки для мене розкіш, щастя… А якби-с могла, любцю,
сповнити свій план, то було би ще краще. Як поїдеш до Іваниківки, напиши ми – я
готов приїхати до Станіслава, щоб бодай здалека часом бачити тебе.

І ще одно діло – щодо переписки. Чи не можна би посилати до тебе листи «poste
restante» до Велдіжа або до Долини? Се мож зробити так, що на кожнім листі
будуть інші цифри – се значить, н[а]пр., я пишу тепер лист і подаю адрес, під
яким напишу слідуючий, в слідуючім подаю адрес дальшого і т. д. Так само й ти,
пишучи до мене. Найліпше класти на конверті які-небудь цифри, літери або знаки,
бо се не звертає уваги.

Вже пізно. Славко збирається додому, пакує, що може. Кінчу й я і посилаю тобі
сердечне добраніч і гарячий поцілуй «в самую точку». Прощай, любочко, до
звидання!

Ага, ще пару слів. Посилаю ти другий том Шекспіра, історію комуни, філософічну
статтю Чернишевського, Beckera і «Zukunft». Крім того, для тебе «Мартовское
движение». Якби ти сама не мала приїхати, то передай Славком згаданий у твоїм
листі шкіц панни Міні, другий том Добролюбова і ще деякі книжки, котрих тобі не
треба. Чи будеш переводити «Темное царство»? Якби-с переводила, то першого
розділу (о критиках Островськ[ого]) не треба. О «Отеч[ественные] зап[иски]»
прошу також ураз із переводом Гонкура.

Твої «Ладканки» я думаю видати осібною книжкою, тільки хочу до них додати свою
розправу про весільний обряд у малорусів, а се робота, котра вимагає досить
часу і студій. Зате, якби-м її скінчив, як треба, – була б гарна штука. Думаю
ще на різдвяні свята поїхати до Нагуєвич і зібрати колядки, в котрих також маса
матеріалу до студій.

Золя другий том приладжую до друку, хоч і дуже поволі. Тепер читаю Brandes’a
«Hauptströmungen der Literatur des XIX. Jahrhunderts» – прехороше діло, котре
треба буде й тобі перечитати.

Прощай. Цілує тя твій навсегда

Іван. 
