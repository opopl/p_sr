% vim: keymap=russian-jcukenwin
%%beginhead 
 
%%file 14_08_2021.fb.fb_group.story_kiev_ua.1.haknuli_arhiv_kpi.cmt
%%parent 14_08_2021.fb.fb_group.story_kiev_ua.1.haknuli_arhiv_kpi
 
%%url 
 
%%author_id 
%%date 
 
%%tags 
%%title 
 
%%endhead 
\zzSecCmt

\begin{itemize} % {
\iusr{Даниэль Шапшович}
Вот кто подал Моссаду идею похищения иранского атомного архива из Тегерана)))

\iusr{Дмитрий Мойсеенко}
огонь @igg{fbicon.flame} 

\iusr{Елена Гродзицкая}

\ifcmt
  ig https://i2.paste.pics/d4dc7c62e6fde714c060ccf1e59b8d76.png
  @width 0.2
\fi

\iusr{Андрей Зражва}

\ifcmt
  ig https://scontent-frx5-2.xx.fbcdn.net/v/t39.1997-6/s168x128/17633073_1652591054767295_6333333619058147328_n.png?_nc_cat=1&ccb=1-5&_nc_sid=ac3552&_nc_ohc=8ZKhRg5cTeUAX_s-srr&_nc_ht=scontent-frx5-2.xx&oh=00_AT9D_HZzt9CW7GhPCaYynf97UTbZWIT6j_RC6yCM1GuJUg&oe=61C13C7C
  @width 0.1
\fi

\iusr{Виталии Рыков}

Блиииин!!!! Я как раз в это время учился в корпусе 20. Правда козы не делал.
Очень любил чертить. Да и получалось у меня это.

\begin{itemize} % {
\iusr{Геннадий Хоролец}
\textbf{Виталии Рыков} 20й Корп, этф 81-87

\begin{itemize} % {
\iusr{Yuri Cherniavsky}
\textbf{Геннадий Хоролец} , ЭТФ 1976-1982, Электрические машины

\iusr{Геннадий Хоролец}
\textbf{Yuri Cherniavsky} 81-87эпа, я-первокурсник, а вы-дипломник, но Попович, Гурин, Пыжов-знакомые фамилии

\iusr{Yuri Cherniavsky}
\textbf{Геннадий Хоролец} , Пыжова не помню, но Давыдова, куратора группы, запомнил на всю жизнь @igg{fbicon.face.symbols.mouth} 

\iusr{Геннадий Хоролец}
\textbf{Yuri Cherniavsky} 

Давыдов-очки и рыжие усы? А у нас \enquote{машины} читал Вакуленко. На потоке в
ЭМ-12 был Чернявский Александр Вильевич, колоритная личность...


\iusr{Геннадий Хоролец}
\textbf{Yuri Cherniavsky} вспомнил: Давыдов- с палочкой

\iusr{Yuri Cherniavsky}
\textbf{Геннадий Хоролец} , Вакуленко и у нас читал. Усы и очки правильные приметы, палочку Д. очевидно приобрёл позднее, неужели в ваше студенческое время он уже ковылял или вы его уже позже в 90е видали?

\iusr{Yuri Cherniavsky}
\textbf{Геннадий Хоролец} , я с 1976 был в группе ЭМ 33А, потом вроде после слияния факультетов группы переименовали, так что, номер той , что я заканчивал не помню, что означает группа ЭМ12? Какой год формирования? Однофамильца такого не знаю, но фамилия не редкая, даже здесь в Австралии встретил бывшего киевлянина-однофамильца.  @igg{fbicon.wink} 

\iusr{Геннадий Хоролец}
\textbf{Yuri Cherniavsky} не, в 80х в20ке ковылял, лабы по машинам вел Тимченко. Общага 18ка в ваши годы запускалась?

\iusr{Yuri Cherniavsky}
\textbf{Геннадий Хоролец} , 

Тимошенко звучит знакомо, но лабы у нас вёл Д. и ещё кто-то , может быть и Т.

18 общага!!!! Да была при нас, более того нас зимой 1980 сняли с занятий на
несколько месяцев на \enquote{ремонт} общедитий, их готовили для туристов к олимпиаде
1980. Как известно большинство туристов так и не приехало. Именно с этих
\enquote{ремонтных} работ на 18 общежитии и начался мой конфликт с Д.

\iusr{Геннадий Хоролец}
\textbf{Yuri Cherniavsky} 

ЭМ-12:электрические машины, год поступл-81, группа-2я. Из ваших годов знаю Ваню
Левченко(с его группы Креп-факультетский комсомол). Мы в 81м учились во вторую
смену, а потом запустили 7й корпус. Куда было распределение?

\iusr{Геннадий Хоролец}
\textbf{Yuri Cherniavsky} 

в 81 вселялись в18ку-конфета! До сих пор существует формулировка \enquote{олимпийские
матрасы}. Да, могли встречаться на перекурах на мостике 20го корпуса


\iusr{Yuri Cherniavsky}
\textbf{Геннадий Хоролец} , 

я защищался осенью 1981, потом военные сборы. Зима-Весна 1982. Красный диплом,
выбрал КБ Института Электросварки. По специальности не работал ни дня.

Впрочем ТОЭ пригодилось ещё на многие годы: делал контрольные работы заочникам
разных киевских техникумов.


\iusr{Геннадий Хоролец}
\textbf{Yuri Cherniavsky} 

О, ТОЭ! Федора(Федоренко)! Первую часть знает студент,
вторую-преподователь,третью- не знает НИКТО!(студенч. фольклор) Должны помнить
Залевского-факультетского \enquote{черчение} и \enquote{начертала}. Баба Вера-матан и точно
была баба Клара-\enquote{тряпки} (изоляц материалы)

\iusr{Yuri Cherniavsky}
\textbf{Геннадий Хоролец} , 

фольклор помню, к сожалению других преподавателей, что вы упомянули нет. Из
всех преподов не относящихся к факультету помню Ситенко ( 1-2 курс) физика.
Помню потому, что она одна носила короткую стрижку разных цветов от зелёного до
фиолетового. Курила импортные сигареты, пила кофе с коньяком на экзаменах.
Согласитесь- редкость в советское время. По-моему её муж был директором кафедры
\enquote{твердого тела}. Возможно Залевский вел начерталку на 1м курсе, этого препода
я помню по факту, но не по имени. Помню еще одну приятную даму по теормеханике,
к сожалению не по имени, единственный предмет, который я пересдавал, чтобы не
получить 3-ку.

\iusr{Геннадий Хоролец}
\textbf{Yuri Cherniavsky} 

теормех-скажу завтра. Неужели не помните Федору-бич
факультета, отсев по армии?! Да, а военной кафедры что, не было?

\iusr{Yuri Cherniavsky}
\textbf{Геннадий Хоролец} , 

Федоренко помню смутно, я вообще плохо помню тех преподавателей, с которыми у
меня не было проблем  @igg{fbicon.wink} . Военная кафедра была, я же писал про сборы зимой
весной 1982. ПВО СВ СССР, ЗРК Круг, помню как сейчас  @igg{fbicon.laugh.rolling.floor} 


\iusr{Виталии Рыков}
\textbf{Геннадий Хоролец} 

ЭТФ, группа ЭМ-11. 81-83(декабрь), ТОЭ, Федоренко. Его \enquote{Москвич} грязно -жёлтого
цвета. Армия. Куратор группы Гурин. Как же это было давно! Осенью прошлого года
заходил в 20й корпус, сложилось впечатление какого-то запустения. Особенно
поразила входная дверная группа с приколхожеными деревянными дверьми.

\iusr{Геннадий Хоролец}
\textbf{Виталии Рыков} Шурик Марченко, Эльденант, Мазур, подольцев, Кармазин, староста-Циля. Ваши?

\iusr{Геннадий Хоролец}
\textbf{Виталии Рыков} 

я-ЭП-12 (Тер-Геворгян-друг Марченко). Привет от Шурика (только что перезвонил). До
сих пор в кошмарных снах вижу Линиченко

\iusr{Геннадий Хоролец}
\textbf{Yuri Cherniavsky} теормех-Ильичишина. Круговская батарея, станция наведения ракет.

\iusr{Виталии Рыков}
\textbf{Геннадий Хоролец} Все!!!

\iusr{Виталии Рыков}
\textbf{Геннадий Хоролец} К сожалению многих фамилий я не помню!!! Но напрягая память вспоминаю одногруппников: Потиха Лена, Грудева Таня, Скрипка Вовка

\iusr{Геннадий Хоролец}
\textbf{Виталии Рыков} а красавица Жанна! Я в общаге жил с Фориным(староста эм-12) иМамедовым. То знаю больше их группу. О вспомнил-Ванька Слободянюк! Из потерь (нет в живых) : Юра Ковалев(12я), Мариуш(наш поляк)-женат на Оле с 12й, Гурин в прошлом году...

\iusr{Виталии Рыков}
\textbf{Геннадий Хоролец} 

Мариуша думаю что помню, хотя возможно путаю с нашим Сашей Матяшем, Жанна была
нашей третьей девушкой в группе, а в ЭМ 12 по-моему было две девушки, одна из них
Оля, даже где-то есть с ней фото из колхоза.


\iusr{Геннадий Хоролец}
\textbf{Виталии Рыков} 

посл инфа о Матче: возглавлял какую-то сельхоз газету, а с Толстым (Марченко) все
эти годы ходим в байдарочных походы (Урал, Карелия), приходил на наши встречи
выпускников (немцы, поляки приезжали), Пыжов был

\iusr{Виталии Рыков}
\textbf{Геннадий Хоролец} 

Пыжов запомнился своей борьбой с джинсами,чесно говоря, не помню чем
спровоцированной. А у меня тогда появились первые \enquote{формовые}, и мне пришлось
доказывать стоя на \enquote{ковре}, что то одежда американских пролетариев. Даже
какие-то вырезки из журналов предоставлял! Лето 83го.

\iusr{Геннадий Хоролец}
\textbf{Виталии Рыков} 

тоже под роздачу попадал: пропуски, отпуск для поездки домой, в общаге что-то. Но
больше всего изощрялся ваш Гурин. Посасывая леденец...

\end{itemize} % }

\iusr{Yuri Cherniavsky}
\textbf{Геннадий Хоролец}, 

да теормех у меня вела она. Спасибо за напоминание. Защиту диплома у меня
принимал Новиков.

\begin{itemize} % {
\iusr{Yuri Cherniavsky}
\textbf{Геннадий Хоролец} 

введение в специальность вёл Постников Иван Матвеевич, совсем уже старый
дедушка. У меня с ним ( его именем) уже в Австралии была любопытная встреча, но
об этом в другой раз. У нас уже ночь.


\iusr{Yuri Cherniavsky}
Статья о моей группе в институтской многотиражке За Радянського Iнженера

\ifcmt
  ig https://scontent-frx5-1.xx.fbcdn.net/v/t39.30808-6/237359218_10158105120966674_7279767062476425378_n.jpg?_nc_cat=105&ccb=1-5&_nc_sid=dbeb18&_nc_ohc=BItwaSASEkkAX9g7lfF&_nc_ht=scontent-frx5-1.xx&oh=00_AT_4chAouQ1TYJGgl88N3j2dFzlp0pB70JAlNF-SxMHpcQ&oe=61C09496
  @width 0.3
\fi

\iusr{Геннадий Хоролец}

По-моему-это зав. кафедрой. О себе: учеба 81-87, распредел - Жулянский машиностр
завод, 1990-1997 зам нач снабжения КПИ ( в 90е это величина! ) - поэтому знаю о КПИ
много...


\iusr{Yuri Cherniavsky}
\textbf{Геннадий Хоролец} , понятно откуда такая память!  @igg{fbicon.beaming.face.smiling.eyes} 

\iusr{Yuri Cherniavsky}
Лето 1981, \enquote{последний звонок}. Прощальный поход нашей группы, сбор у 18 общежития. Я во втором ряду, второй слева.

\ifcmt
  ig https://scontent-frt3-1.xx.fbcdn.net/v/t39.30808-6/235229718_10158105129036674_1558502248398385111_n.jpg?_nc_cat=104&ccb=1-5&_nc_sid=dbeb18&_nc_ohc=knfBi94RM9QAX-bo0aZ&_nc_ht=scontent-frt3-1.xx&oh=00_AT-nw92ob3PyMVYfdPyvSMaatcV7_Ad76vH_FfMPYIhW6A&oe=61C19AB9
  @width 0.6
\fi

\iusr{Геннадий Хоролец}
\textbf{Yuri Cherniavsky} 

Разминулись... Я в конце августа только вселялся в 18ку. Газету переименовали в
\enquote{Политехник}. На фото чудак с рюкзаком-только выселился?

\end{itemize} % }

\end{itemize} % }

\iusr{Александр Прыжиков}

Вечно эти политехники ищут приключения. У меня ПТМ, на 4-м курсе, замдекана
факультета, вел. Станевский Валерий Петрович. Автор всех учебников и методичек
по кранам. На пары даже опаздывать не смели, не говоря о прогулах. Предмет был
профильным, подавался легко и очень познавательно. Потому и слушался с открытым
ртом. Да и сам преподаватель, был очень высокого уровня специалист. И никогда
не \enquote{зверствовал}. Хороший дядька был.

\begin{itemize} % {
\iusr{Irina Zhogina}
\textbf{Aleksandr Pryzhikov} сосед мой площадке)

\iusr{Александр Прыжиков}
\textbf{Irina Zhogina} Да, хорошее было семейство...
\end{itemize} % }

\iusr{Олена Медведева- Прицкер}

Спасибо, напомнили старый студенческий термин \enquote{коза}, передавался от
поколения к поколению во всех Технических Вузах.

\iusr{Vladimir Popoff}
Козлотрон) у нас был термин - дралоскоп.

\begin{itemize} % {
\iusr{Валентина Валентина}
\textbf{Vladimir Popoff} первый раз слышу термин \enquote{коза} в тех. Вузе: драласкоп.

\iusr{Александр Прыжиков}
\textbf{Валентина Валентина} хм... Коза и в Африке - коза. Но прибор назывался \enquote{козлоскопом}. @igg{fbicon.face.hand.over.mouth} 

\iusr{Олександр Сєдих}
\textbf{Vladimir Popoff} так, це термін харківських політехніків, а ось аналогу \enquote{козі} не пригадую.

\iusr{Татьяна Волынец}
\textbf{Vladimir Popoff} окончила КАДИ в 1976г. Козлотрон студентов представлял собой перевернутую табуретку, на которую устанавливалось стекло, а под ним была настольная лампа. И этот термин использовался до компьютеризации

\iusr{Наталия Педос}
\textbf{Vladimir Popoff} И у нас на ТЭФе (1969-1975гг.)!
\end{itemize} % }

\iusr{Валентина Козачук}

Я, когда заканчивала пищевой институт, то на дипломный проект покупала себе
козу, по ней было очень легко писать свою тему.. Чертить любила, все чертежи
делала сама без козы.. Любовь к черчение привил в школе учитель математики и
черчения Нименький Владимир Михайлович, был у нас классным руководителем...

\iusr{Maxim Zuk}
Мой родной корпус... ПТМ сдавали примерно так же...

\begin{itemize} % {
\iusr{Андрей Федорович}
\textbf{Maxim Zuk} таки да)
\end{itemize} % }

\iusr{Виталий Сухинин}
Зворыкин ректор ? Или Гугл сломался или....

\iusr{Виталий Сухинин}
Сорри, Гугл работает правильно...

\iusr{Vadym Kryvobok}

Ну такой себе \enquote{хак}, что было бы если бы бабушка \enquote{божий одуванчик} погнала вас
из архива. Хотя справедливости ради надо сказать, что \enquote{победителей} не судят и
цель была достигнута.

\iusr{Alex Vines}

В 80м делал дипломной проект (КАДИ, заочное отделение). Работал на
Стройиндустрии диспетчером. Туда вывозили макулатуру с Киева и области. Как то
грузчики позвонили и сказали, шо на складе макулатуры огромный ящик с какими-то
дипломами. Оказалось что с КАДИ. Ну а дальше дело техники и казлотрона.

\iusr{Dmitry Tikomirov}
В КАДИ учились также.

\iusr{Данила Сазонов}
Это родной Химмаш на переднем плане. 19-й корпус.

\iusr{Марчук Галина}

Такой титанический труд провели, не проще было бы самим все же спроектировать.
Хотя выпускники КПИ этим и ценятся, решат любую задачу. Браво!

\begin{itemize} % {
\iusr{Ян Медников}
\textbf{Марчук Галина}, не проще. Затраты времени не сопоставимы.
\end{itemize} % }

\iusr{Алексей Чернов}

Мне сделали дипл. проект за 200\$, т. к. я понял, что 4 года просто выпали из
жизни. Мне под 50. Я не использовал знания. Чётко понимаю, что надо было поступать
в КИСИ и меньше слушать умников.

\begin{itemize} % {
\iusr{Елена Смелова}

Можно Вам только посочувствовать. Я с удовольствием и легко училась в КАДИ. И
так же работала 9 лет в проектном институте, а потом 1,5 - в научном. И все
полученные знания пригодились мне в дальнейшем на госслужбе. Спрос на интеллект
и эрудицию будет присутствовать на рынке труда всегда.

\end{itemize} % }

\iusr{Татьяна Юдина}

\ifcmt
  ig https://i2.paste.pics/e0b9fc88efb1fd48ea1daf053be989a9.png
  @width 0.2
\fi

\iusr{Leonid Dukhovny}

Как говорил мой приятель, после защиты дипломного проекта, я нашёл в архиве
своего ПКТИ нужную разработку, изменив в ней только фамилию автора. Кстати,
отдел проектного института, где я проработал 20 с лишним лет, как раз и
занимался разработкой нестандартного внутрицехового транспорта... Вот, когда я
полностью погружался в тему, тут уж реализовывались все твои фантазии ! Честное
слово, это всё равно, что написать приличную песню... К счастью, мне удавалось
делать и то, и другое одновременно. Просто эти ипостаси, как бы дополняли друг
друга... Мои проекты, что приятно, до сих пор работают на многих заводах бывшего
Союза, где нет средств на более современное оборудование.... Может, кто из
нынешних студентов и \enquote{козлит} их... На здоровье!


\iusr{Виталий Шарлай}
Козлотрон КАДИ ...

\iusr{Виталий Шарлай}

\ifcmt
  ig https://i2.paste.pics/a0f4beba43098c6f5bfbbf130037cbb2.png
  @width 0.2
\fi

\iusr{Геннадий Хоролец}

Есть вопросы.

1) Папки дипломных проектов хранятся на кафедрах у зав. лабов, а не в архиве! (в архиве-личные дела).

2) \enquote{тащили рулоны ватмана...} После защиты чертежи СКЛАДЫВАЛИСЬ до А4 формата (до уровня папки).

3) Зворыкина в ректорах КПИ не было НИКОГДА.

4) На фото-корп19 (Химмаш), а мехмаш - в 1м и в 18м. Где-то так...

\begin{itemize} % {
\iusr{Олег Дидиченко}
\textbf{Геннадий Хоролец} Да, рулонов не было, а в 18 был Фивт. Зворыкира не помню

\iusr{Ян Медников}
\textbf{Геннадий Хоролец}, 

для проверки информации про Зворыкина есть Гугл. Зворыкин был ректором в
1904-1905. Рулоны были. Попробуйте сложить 4 листа толстого ватмана и увидите
сколько места займет такая стопка. Хранение в рулонах было более практичным. А
ПТМ изучали не только студенты мех-маша.

\begin{itemize} % {
\iusr{Антон Осьмак}
\textbf{Ян Медников} таки ватмани складались..

\iusr{Ян Медников}

Это мой рассказ про эпизод из моей жизни. Мы с другом тащили чертежи в рулонах.


\iusr{Виктор Клименко}
\textbf{Ян Медников}

Вам не повезло. Обычно чертежи после защиты складывались, что требовало
некоторых усилий, связывались вместе с пояснительной и складировались на
кафедрах. Срок обязательного хранения составлял 5 лет. Кстати, на протяжении
этого срока результаты дипломов могли оспариваться. Правда, с прецедентами я
никогда не сталкивался. Так было всё время с начала 70-х по начало 2000-х,
которое я провёл в КПИ.

\end{itemize} % }

\iusr{Татьяна Недашковская}
\textbf{Геннадий Хоролец} пункт 4, поступала на химмаш, проучилась все годы в 19 корпусе
(он родной любимый) - в дипломе закончила факультет мехмаш @igg{fbicon.face.tears.of.joy} , так что все может
быть

\begin{itemize} % {
\iusr{Геннадий Хоролец}
\textbf{Татьяна Недашковская} какие годы?

\iusr{Татьяна Недашковская}
\textbf{Геннадий Хоролец} 

гораздо позже описанных событий, но при получении диплома и нашем удивлении
почему при написании мы оказались на мехмаше нам ответили что наша кафедра
изначально была там.... в общем все перипетии знает опять таки архив КПИ,
который отвечает за приказы, рестуктуризацию и т. д.

\iusr{Геннадий Хоролец}
\textbf{Татьяна Недашковская} из ваших знал только Сашу Тимонина
\end{itemize} % }

\iusr{Нина Бондаренко}
\textbf{Геннадий Хоролец} 

когда я училась (1975-1981) эти корпуса строились, и нас привлекали на стройку.
Рядом с корпусом Химмаша строили \enquote{силикатный}.

\iusr{Татьяна Комащенко}
\textbf{Геннадий Хоролец} , 

Зворыкин был, но очень давно 1904-05 гг. Мехмаш тут есть и поныне, лазерщики,
ТКМ. Если - что, я в этом корпусе работаю)) А все остальное - правильно.

\end{itemize} % }

\iusr{Андрей Козачук}
Мой корпус 20 рядом.

\iusr{Андрей Федорович}
Сто пудово Химмаш)

\iusr{Светлана Блумен}
фу, учились в технических вузах, а чертить не научились, позорище

\begin{itemize} % {
\iusr{Александр Прыжиков}
\textbf{Светлана Блумен} ой, да ладно Вам! За 5 лет все, кто хотел, научились! Я почти 30 лет назад КИСИ закончил, а подчерк выдает.  @igg{fbicon.wink} 

\begin{itemize} % {
\iusr{Светлана Блумен}
\textbf{Александр Прыжиков} сейчас проще, можно в компе делать и специальными программами, но именно чертежники и проектировщики должны уметь это делать вручную

\iusr{Александр Прыжиков}
\textbf{Светлана Блумен} не знаю, мы карандашами чертили до диплома, некоторые, пробовали даже тушью. Но... Зачем рыть лопатой, если есть экскаватор?

\iusr{Светлана Блумен}
\textbf{Александр Прыжиков} а разницу в цене понимаете проектировать саму лопату и экскаватор? что-то плохо вообще учились

\iusr{Александр Прыжиков}
\textbf{Светлана Блумен} вот сразу в Вас чувствуется гуманитарий. Не проектировать лопату, а эксплуатировать экскаватор. Эх... @igg{fbicon.laugh.rolling.floor} 
\end{itemize} % }

\end{itemize} % }

\iusr{Игорь Иванов}
Сам таким был!

\iusr{Sergej Strakhov}

Я учился в КАДИ в 73-78. А мой отец - там же, но в 51-55. Когда он впервые
услышал от меня термин \enquote{козотрон}, он сказал, что в его время такое устройство
называли \enquote{дралоскоп}, Вот как меняется студенческий сленг.

\begin{itemize} % {
\iusr{Billy Bones}
\textbf{Sergej Strakhov} есть еще вариант сдироскоп

\iusr{Ирина Овсиенко}
\textbf{Sergej Strakhov} Я училась не в Киеве, а в Харькове в ХИИТе (1972г .- 1977г.). У нас тоже использовалось такое приспособление под названием дралоскоп.
\end{itemize} % }

\iusr{Киевские истории}

Друзья, пожалуйста, помогайте администрации группы выявлять ботов и троллей.
Увидев негативный комментарий, жмите в диалоговом окне кнопку пожаловаться на
комментарий.

\ifcmt
  ig https://scontent-frx5-1.xx.fbcdn.net/v/t39.30808-6/238900138_535563604446105_9009127249032702227_n.jpg?_nc_cat=105&ccb=1-5&_nc_sid=dbeb18&_nc_ohc=Se4ZSDT0Bd8AX939qT1&_nc_ht=scontent-frx5-1.xx&oh=00_AT_iJqVH_Wpo5Ln7p4mGI2uOk_DlEytFO5byowu76slOgw&oe=61C090C0
  @width 0.4
\fi

\iusr{Валентин Марченко}

Диплом - 10 листов А1. Даже просто откозлить - полжизни угробил. Самой
интересной, молодой, зажигательной. Ну, не пол. Но до хринища. КИИГА88


\iusr{Yevgeniy Goldshtein}

Я уехал в 1974 в Белоруссию. В общаге Новополоцкого Политеха для дироскопа
(козлотрона) вынималась одна из оконных секций. Один из наших однокурсников,
которого Боженька не обидесл ни силой, ни весом слегка перестарался. Стараясь
получить ну очень жирную линию, он нажал сильнее... Стекло треснуло, разбилось.
Оригинал и копия были смяты и порезаны. За окном мела февральская метель. В 70х
годах прошлого века ещё не слышали о глобальном потеплении...

Мой диплом ТГСВ включал 12 листов.

А многие предметы из НПИ мне пригодились. Геодезия, вентиляция и
кондиционирование, архитектура. В разных странах, на разных континентах.

А особенно 4 года вольной борьбы и самбо. Но это отдельная тема.

\iusr{Лариса Лобановська}

Была очень симпатичная старая, французская комедия о студентах-оболтусах,
которые придумывали миллион способов как обдурить преподов на экзамене. Очень
смешно, но мне всегда было непонятно - зачем?! Не проще ли хоть немного
выучить?!

\iusr{Tatiana V Nskaia}

Я козлила турбину; по—моему 30 лопаток: уже не помню. ТЭФ 1990.

\begin{itemize} % {
\iusr{Геннадий Хоролец}
\textbf{Tatiana V Nskaia} Босый, Ткачук, Варламов и общ1

\begin{itemize} % {
\iusr{Tatiana V Nskaia}
Да!!! Общ не знаю ; жила дома; БосыйКудэляТкачук Попель по парогенераторам  @igg{fbicon.laugh.rolling.floor} 

\iusr{Наталия Педос}
\textbf{Геннадий Хоролец} 

Владимир Васильевич Босый читал у нас термодинамику и теплопередачу. На
практические занятия без логарифмической линейки (единственное счетное
устройство на то время, кроме одного на все 5-е общежитие арифмометра \enquote{Феликс})
не допускал.

Юрий Федосеевич Ткачук был зам. декана при декане Юрии Александровиче Бабенко.
Общежитие N5 теплофака находилось в конце тихой в то время улицы Шолуденко. Мы,
студенты 2-го курса (1970 г.), были задействованы на подготовке места под
строительство нашего 5-го корпуса по улице Политехнической, в котором потом и
мы доучивались.

Мои годы учебы 1969-1975, кафедра \enquote{Автоматизация теплоэнергетических
процессов}.

\iusr{Татьяна Комащенко}
\textbf{Наталия Педос} , Термодинамику читал Куделя, Босый - теплопередачу. Это были разные предметы)

\iusr{Наталия Педос}
\textbf{Татьяна Комащенко} Это Вы к чему мне рассказываете? Термодинамику вам мог читать Петр Петрович, а Владимир Васильевич вел у нас эти оба предмета.

\iusr{Татьяна Комащенко}
\textbf{Наталия Педос} , извините, если обидела, но в те годы Термодинамику читал Чеботарев, по крайней мере у ТС и ТП, а Куделя и Босый ещё не были доцентами. И что б ассистент читал две профилирующие дисциплины - такого не припомню. лекции вели только доценты и профессора.

\iusr{Наталия Педос}
\textbf{Татьяна Комащенко} 

Вовсе не обидели. Не знаю, какие были \enquote{те годы} и что такое \enquote{ТС и ТП}, но в
пору моего студенчества лекции по термодинамике и теплопередаче читали
В. В. Босый (он был доцентом) и В. А. Чеботарев, а практику вели В. В. Босый,
П. П. Куделя, Н. К. Пилипко.

В конце 90-х Владимир Васильевич был деканом нашего ТЭФа.
\end{itemize} % }

\iusr{Zoya Kavutskaya}
\textbf{Tatiana V Nskaia} и я козлила турбину. ТЭФ 1986 год выпуска

\iusr{Татьяна Комащенко}
\textbf{Tatiana V Nskaia} , а какая группа? 1990 - год поступлени или выпуска?

\end{itemize} % }

\iusr{Константин Ковальчук}
Получил удовольствие @igg{fbicon.exclamation.mark} @igg{fbicon.hands.applause.yellow}{repeat=3} 

\iusr{Виктор Клименко}

Был другой способ \enquote{козлить} чертежи - накалывание тонкой иголкой. Т.к. не жил в
общежитии и не имел \enquote{козлотрона}, то таким образом когда-то скопировал чертежи
шестипозиционной кокильной машины, кажется, французской фирмы \enquote{Voisines},
\enquote{синьки} которых удалось заполучить на ВАЗе в Тольятти для курсовой работы.

Когда уже пришёл на производство, то нескончаемо был восхищён, столкнувшись с
\enquote{козлотроном} формата А0 промышленного изготовления.

P.S. Не знаю как сейчас, а раньше выпускников КПИ, действительно, всегда
выделял чётко отработанный за годы учёбы чертёжный почерк.

\begin{itemize} % {
\iusr{Олег Московко}
\textbf{Виктор Клименко} ЛВ,с ИФФ?

\iusr{Виктор Клименко}
\textbf{Олег Московко}
Yes!  @igg{fbicon.face.tears.of.joy} 
Жаль только, что ИФФ уже нет.  @igg{fbicon.face.rolling.eyes} 

\iusr{Олег Московко}
\textbf{Виктор Клименко} 

хорошо когда коза есть, а если надо установку ЭШП положить на чертёж, а в
литейке только лабораторный образец... и \enquote{живых} две на весь Киев.

\iusr{Виктор Клименко}
\textbf{Олег Московко}

К Медовару нужно было обратиться. В своё время мог составить протекцию. Что,
ныне покойный Дмитрий Фёдорович не подсказал?  @igg{fbicon.wink} 

А у меня вся документация на кокильную машину была, правда, вся на французском
языке, коим не владею. И по ихним же стандартам. Тут ещё посмотреть, кому
легче?..  @igg{fbicon.smile} 

\iusr{Олег Московко}
\textbf{Виктор Клименко} 

Д. Ф. был настолько рад, что кафедра наконец обзавелась студентами, что первых
семнадцать человек обучали по программе \enquote{из головы} @igg{fbicon.smile}.

\end{itemize} % }

\iusr{Ольга Васильевна}

Сестра училась не в КПИ, а в филиале Института водного транспорта на
супер-жесть-факультете, выпускающем инженеров+экономистов! Помню это толстое
стекло с письменного стола!, настольную лампу и сестру на девятом месяце на
коленках, упорно \enquote{козлящую} портовые краны!)))  @igg{fbicon.face.tears.of.joy}  та ещё картинка была, что в
комнате, что на чертеже!

\iusr{Ольга Васильевна}

Здесь, вероятно, много читателей с техническим образованием, есть всем вопрос:
что как называется эта линейка и для чего она (находка на старой даче)


\ifcmt
  tab_begin cols=2,no_fig,center

     pic https://scontent-frt3-1.xx.fbcdn.net/v/t39.30808-6/237377981_1178816242594308_8319285921651388253_n.jpg?_nc_cat=104&ccb=1-5&_nc_sid=dbeb18&_nc_ohc=afeuSUQRVd8AX_YVU7g&_nc_ht=scontent-frt3-1.xx&oh=00_AT_zkzVuZoDkkFy2LQtxOApIpHQeBvBKSctDZpYzrVZKcQ&oe=61C16FF9

		 pic https://scontent-frx5-2.xx.fbcdn.net/v/t39.30808-6/237022982_1178816359260963_8404546776883107745_n.jpg?_nc_cat=109&ccb=1-5&_nc_sid=dbeb18&_nc_ohc=ocF0mLAgJrgAX_LhbKw&_nc_ht=scontent-frx5-2.xx&oh=00_AT98jpUlAQxbYtArxWPhFoL-AmSWgpME00dSp76ki5dzXQ&oe=61C06F74

  tab_end
\fi

\begin{itemize} % {
\iusr{Владимир Ильенко}
\textbf{Ольга Васильевна} 

Ольга, это линейки для более точного измерения величин отрезков линий. На
первом снимке справа имеется сетка пронумерованных горизонтальных ,
вертикальных и наклонных линий. При помощи измерителя (похож на циркуль только
имеются заострённые концы на обеих ножках) откладывается , в определенном
масштабе, размер в мм. Если число мм целое, то просто на шкалу ориентируемся.
Если доли мм надо отложить то двигаемся вдоль косой линии. Например если надо
отложить 1.3мм, то после нахождения 1 мм ищем пересечение косой линии с 3-ей по
счету горизонтальной линией. На другом фото и слева на верхнем в торцах те же
функции с различными коэффициентами..


\iusr{Ольга Васильевна}
\textbf{Владимир Ильенко} , немедленно покажу мужу, это его вопрос и очень ждал ответа, спасибо!)))
\end{itemize} % }

\iusr{Ирина Чуприна}
Знаете, что я говорю своим : cчитайте хоть на пальцах, но чтобы ответ был правильным. На войне все средства хороши.

\iusr{Владимир Рукавишников}

Я лично окончил МГТУ Баумана ф-т Спецмашиностроение. Проходил в 90-тые годы 18
месячную стажировку в Германии, и когда участники программы из СНГ расказывали
немецким коллегам, как они выполняли и сдавали зачеты, курсовые и экзамены, то
мы всегда встречали реакцию непонимания и удивления со стороны немецких коллег.

Они нам отвечали, как можно такой профессиональной подготовкой бравировать? Со
временем мы поняли, что подготовка в ВУЗах Германии с ориентированна строго на
запросы компаний и государства. Этим никогда не славился КПИ, а загружать
студента нужно индивидуальными заданиями, и оценивать подготовленность студента
должен преподаватель использующий теоретические и практические знания
необходимые определенной отрасли, которые у нас разрушены до основания и
отстали в своем развитии на 20 - 30 лет. 

Студенты продолжают десятилетиями использовать Козу и Козлотрон для Козления
курсовых и дипломов, а отрасли промышленности живут своей иной жизнью. Очень
многое нужно радикально менять в нашем образовании для инженеров.

\begin{itemize} % {
\iusr{Виктор Клименко}
\textbf{Владимир Рукавишников}

Всё верно. Однако, нужно учитывать, что отечественные ВУЗы, прежде всего,
давали фундаментальную начальную инженерную подготовку, чем КПИ всегда
славился. Получив такую подготовку, инженер мог легко настроится на любое узкое
направление в своей области или даже переквалифицироваться на смежную
специальность. Для выпускников КПИ это никогда не составляло труда. Ведь в доме
крышу всегда можно поменять, а вот заложенный изначально фундамент поменять или
даже упрочнить трудно.

А инженер - это профессия, которой нужно учиться всю жизнь, а не только годы в ВУЗе.

\begin{itemize} % {
\iusr{Владимир Рукавишников}
\textbf{Виктор Клименко} 

Я после стажировки работал 15 лет в немецкой компании и 4,5 года в Германии и
внимательно сравнивал программы подготовки МГТУ Баумана и программы
технического университета в Берлине, Дрездене, Мюнхене и Гамбурге. КПИ имеет
существенно более слабую программу подготовки, базу и преподавательский
персонал чем в МГТУ Баумана. Все программы немецких вузов перепроверяются,
изменяются и адаптируются к запросам отраслей. Все работы ктн и дтн в Германии
востребованы отраслями, у нас в Украине это 95\% фуфло. Тех. Уни в Германии
привлекают к чтению курсов специалистов из отрасле, а у нас читают курсы и их
содержание, что и как могут, а не так как нужно отрасли. Поэтому за
"фундаментальной начальной инженерной подготовкой" подготовкой КПИ стоит
неспособность готовить инженеров на современном техническом уровне. Многие
идут, чтобы получить диплом, а после этого сами разбираются, что нужно забыть и
что нужно изучить самостоятельно.


\iusr{Виктор Клименко}
\textbf{Владимир Рукавишников}

Я понимаю, что Вам, как и мне, удобней и приятней работать в Германии. Но
надеюсь, Вы понимаете, что условия в Украине отличались и до сих пор
кардинально отличаются от германских. В Украине, где устаревшая промышленность
советского времени разрушена напрочь вместо модернизации, а новая ещё только с
большущим трудом делает первые неуверенные шаги, где предприятия то возникают,
то исчезают, где инженерам из-за этого часто требуется менять и профессию, и
квалификацию в поисках работы, ВУЗам необходимо обеспечить должную подготовку
специалистам в соответствии с имеющимися условиями. И сейчас программы в ВУЗах
меняются постоянно, не реже раза в год. Как Вы думаете, если ВУЗ выпустит
узкого специалиста для конкретного предприятия, как в Германии, он сможет себе
тут найти работу, если это предприятие, не дай бог, прикажет долго жить, что мы
видим на каждом шагу? Или Вы хотите, чтобы все наши специалисты скопом
переехали в Германию? И где ВУЗы могут взять современные технологии, когда
таких предприятий в стране меньше, чем пальцев на руках? Вот и выкручиваемся
все в этих условиях – и предприятия, и ВУЗы. Давайте будем объективными!
Естественно, есть к чему стремиться и нужно это делать, но как говорила
известная всем личность, маємо те, що маємо!


\iusr{Ольга Васильевна}
\textbf{Владимир Рукавишников} , 

Вы знаете, в 80е читали курсы в КПИ «производственники»: главные специалисты и
главные инженеры заводов! Очень интересно и доходчиво, очень профессионально
(со слов мужа на факультете машиностроение)))


\iusr{Владимир Рукавишников}
\textbf{Виктор Клименко} 

Вопрос не в том, что удобне кому-то, а в том что нужно отрасли. Для отрасли
нужны различные программы подготовки для инженеров
конструкторов/исследователей, инженеров механиков и энергетиков и организаторов
производства. Сегодня более 80\% получают дипломы и не хотят и не планируют
работать в отрасли, т.к. экономически эта работа инженеров плохо оплачивается.
Студент должен определять проф.ориентацию, а не руководство ВУЗа, и для этого
нужно сокращать валовые показатели, и численность преподавателей согласно
опросу студентов.


\iusr{Виктор Клименко}
\textbf{Владимир Рукавишников}

Кажется, Вы чего-то не понимаете...

Всё, что Вы сказали, разумеется, правильно и нужно! Действительно, многие
выпускники ВУЗов даже не планируют работать по специальности. Задумайтесь,
почему?

Чтобы создавать специалистов необходимых отрасли, во-первых, нужно чтобы эта
отрасль существовала! Один бывший парторг раздарил своим приближённым всю
промышленность, те пустили её на металлолом, а вырученные деньги опустили в
офшоры и занялись обычной спекуляцией, которую назвали бизнесом. Сейчас в
Украине не производится обычный водопроводный кран - покупается китайская
ерунда и перепродаётся тут втридорога. И так в любой отрасли. О какой из них Вы
рассуждаете?

Соответственно, падает уровень преподавания. И не только в ВУЗах, но и в
средних школах. Причём, очень и очень значительно! Теряется квалификация
преподавателей, теряются целые научные школы. Приходилось встречать доцентов,
которые даже не видели производства, курс которого читают студентам. Какие
сказки они могут им рассказать, начитавшись книжек полувековой давности?

А вот количества преподавателей уменьшать нельзя. Наоборот, нужно создавать тут
конкуренцию, когда студент сам выбирает себе преподавателя той или иной
дисциплины. Т.е. по каждой дисциплине должно быть, как минимум, два
преподавателя на выбор. Вот тогда студент и сможет самостоятельно выбирать свой
профиль.

Но на всё это нужны средства, которых нет. А откуда они могут появиться? -
Только от промышленного производства, когда промышленность начнёт развиваться,
а не находиться в состоянии стагнации. Или Вы знаете другой источник
финансирования?  @igg{fbicon.wink} 

\end{itemize} % }

\iusr{Раиса Собчук}
\textbf{Владимир Рукавишников} Практики нужно больше. Я тоже окончила КПИ.

\end{itemize} % }

\iusr{Александр Сопражинский}

Будучи студентом Сварочного факультета КПИ я ездил со Зворыкиным в колхоз,
нормальный мужик.

\iusr{Марина Ятко}

У меня тоже был мостовой кран. Достали \enquote{козу}, но она мне не
понравилась, в результате все считала и чертила сама.

\iusr{Oleg Lypkyn}
Дралоскоп рулил

\iusr{Олександр Сокуренко}
Я ОБИРАЮ УКРАЇНСЬКУ МОВУ.

\iusr{Олег Сафронов}

Помню на защите дипломного проекта (1974, КПИ, кафедра Приборостроения и
средств автоматики) один мой одногруппник даже \enquote{козлить} поленился, а вывесил
перед комиссией 8 листов чертежей формата А1, из которых только один был
белоснежного цвета, потому, что консультант по технологии всё-таки заставил его
переделать конструкцию приспособления, которое в обязательном порядке должно
было присутствовать в проекте. Остальные листы, как вы догадываетесь, были
жёлтого цвета по причине своей старости. Но комиссия не обратили на это
внимания.


\iusr{Elena Veremenko}

А я ничего не козлила. Чертить не любила, но приходилось чертежи делать самой.
Мне только по ТММ помогли сделать чертежи. Учеба в КПИ далась мне весьма
нелегко. Однако диплом свой защитила на 5 баллов и гордилась этим по праву. ЭЭФ
кафедра машин.

\begin{itemize} % {
\iusr{Геннадий Хоролец}
\textbf{Elena Veremenko} какие годы учебы?

\iusr{Elena Veremenko}
\textbf{Геннадий Хоролец} 1970-1977г.г. Один год акадэмка.
\end{itemize} % }

\iusr{Max Gopencko}

Вот это я понимаю, настоящие комсомольцы! Смелые, находчивые, щедрые! Сами
выход нашли и товарищам помогли  @igg{fbicon.smile} 

\begin{itemize} % {
\iusr{Max Gopencko}
мы тоже чудили в студенческие годы, но размах не тот был... совсем не тот  @igg{fbicon.smile} 
\end{itemize} % }

\iusr{Serhii Mo}

Так вот почему невозможно было найти дипломную работу для восстановления
диплома  @igg{fbicon.anger} . Человеку лишних три года пришлось в штатах учится, вместо того,
чтоб восстановить диплом и сделать его эвалюацию. Вы понимаете, что кому-то 3
года жизни пересрали.

\begin{itemize} % {
\iusr{Елена Великая}
\textbf{Serhii Mo} ужас, как можно учиться!! в штатах!!! три года!! и назвать
это «пересранными годами жизни»  @igg{fbicon.shrug}  @igg{fbicon.man.facepalming} 

\iusr{Ян Медников}

Ну да, типа мы взяли десяток дипломов 70-х годов и из-за этого вам пришлось в штатах учится.

\iusr{Oksana Shukayeva}
\textbf{Serhii Mo} сочуствую...

\end{itemize} % }

\iusr{Dumanov Valentin}

Окончил МВТУ в 1983 г.. Некоторые пользовались \enquote{козлотроном} после 3 курса, а
кто раньше вылетали сразу. Даже на \enquote{дипломе} требовалось определение каждой
линии. Диплом занимал 15-20 листов 24 формата . Мой - разрез двигателя Ми8 - 8
листов + технология 8. Рецензию писали главные спецы Москвы. Но .... на
сегодняшний день - декан факультета \enquote{Энергомашиностроение} МГТУ им Баумана -
завхоз моего обучения в училище - товарищ Суровцев !!! А ВЫДАЮЩИХСЯ УЧЁНЫХ и
ПЕДАГОГОВ НЕТ В ИСТОРИИ КАФЕДРЫ! Но я каждый день пользуюсь знаниями
полученными в институте!

\iusr{Татьяна Комащенко}

Что-то не клеится со Зворыкиным. Он был ректором в 1904-05 гг. , Корпус 19
построен в 1977 г. С внуком что-то не складывается. Правнук может? Оказывается,
интересная династия ученых

\begin{itemize} % {
\iusr{Ян Медников}
\textbf{Татьяна Комащенко}, поинтересовался. В Википедии написано, что сын у него родился в начале 20-х.
\end{itemize} % }

\iusr{Виктория Горовенко}
Терминология родная...

\iusr{Rimma Turovskaya}
Операция \enquote{Ы}. @igg{fbicon.grin} 

\iusr{Александр Герасько}
Навчався в 19-му 1999-2004р



\end{itemize} % }
