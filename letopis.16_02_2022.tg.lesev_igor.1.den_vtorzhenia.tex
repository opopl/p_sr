% vim: keymap=russian-jcukenwin
%%beginhead 
 
%%file 16_02_2022.tg.lesev_igor.1.den_vtorzhenia
%%parent 16_02_2022
 
%%url https://t.me/Lesev_Igor/313
 
%%author_id lesev_igor
%%date 
 
%%tags rossia,ugroza,ukraina,vtorzhenie
%%title День Вторженья, как он был от нас далек Как в костре потухшем таял уголек...
 
%%endhead 
 
\subsection{День Вторженья, как он был от нас далек Как в костре потухшем таял уголек...}
\label{sec:16_02_2022.tg.lesev_igor.1.den_vtorzhenia}
 
\Purl{https://t.me/Lesev_Igor/313}
\ifcmt
 author_begin
   author_id lesev_igor
 author_end
\fi

День Вторженья, как он был от нас далек Как в костре потухшем таял уголек...

Знаете, друзья, когда что-то чешут в эфир официальные лица – политики,
спецслужбы и пр., то это всегда оправдано какими-то умыслами. Даже
дезинформация – это элемент политики, и кому как не политикам ею заниматься?

Но нам тут с вами десятилетиями парили о «стандартах западной журналистики».
Ну, которая всегда типа над политикой, правительством и конъюнктурой, а
работает исключительно ради достоверности, на проверенных/доказанных фактах и
во имя общественной значимости.

Самые ушлые среди нас в этих «стандартах» сомневаться начали в тот же момент,
когда познакомились с этой самой «независимой западной журналистикой». Для меня
особо диким стал эпизод из августа 2008 года, когда в прямом эфире прикрыли
девочку-гостью в студии, которая пошла против ветра и стала говорить, что это
не русские напали на грузин, а грузины на осетин. Хуяк, и тут же попрощались.

С 14 года мы с этими «стандартами» познакомились уже гораздо ближе. Но у меня
есть своя индивидуальная история. В Москве у меня есть хорошая знакомая Ия,
которая длительное время стажировалась в Бостоне и Нью-Йорке. И сумела
поработать даже в «Нью-Йоркере», согласитесь, далеко не самом последнем издании
в Штатах.

И вот из этого «Нью-Йоркера» послали корра на Донбасс. Это как раз был разгар
летних боев 14-го. Американец был все время с украинской стороны, его
определили в какой-то добробат, где его и катали по Донбассу, показывая что и
как. А он смотрел, анализировал, разговаривал с местными, а когда вернулся в
Штаты – написал гигантский репортаж. Только там вышел один нюанс. Украинская
сторона там вышла не в шоколаде. Это при том, что – повторяю – все время
американец был только с украинской стороны.

И что же? Главный редактор «Нью-Йоркера» Дэвид Ремник засунул репортаж в сейф и
сказал, что это мы ставить не можем. У Ремника, кстати, есть в загашнике
Пулитцеровская премия – мечта любого журналиста в мире. Но, блд, ставить ТАКОЙ
репортаж не можем. В итоге, тот корреспондент вообще ушел из журналистики и
вернулся к преподаванию в универе.

Но 16 февраля западная журналистика пробила еще и донное дно. Ладно, они не
дают альтернативную точку зрения, как по войне на Донбассе, Августовской в
Грузии, в Сирии и черт те еще где. Ладно, стали выпиливать оппонентов, как
произошло с Трампом и его соцсетями. Но теперь они начали лепить горбатого,
создавая параллельную реальность. Это уже даже не заангажированная
журналистика, а уровень КНДР, где местная сборная по футболу «побеждает» на
чемпионате мира.

В общем, друзья, западная журналистика – это стандарты глубокой помойки. Да, а
у нас по плану – флажки на балконах и в 10.00 коллективное пение гимна. Можно
еще помочиться у порога. Говорят, Путин не нападает на обоссанные квартиры.
