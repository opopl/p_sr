% vim: keymap=russian-jcukenwin
%%beginhead 
 
%%file 03_09_2021.fb.kulyk_roman.1.ukraina_rusj_bezumstvo
%%parent 03_09_2021
 
%%url https://www.facebook.com/romanvkulyk/posts/4222369337831716
 
%%author_id kulyk_roman
%%date 
 
%%tags arestovich_aleksei,bezumie,identichnost',kritika,obschestvo,pereimenovanija,rusj.ukraina,strana,ukraina
%%title Перейменування України на "Русь-Україну" в наш час - безумство чистої води
 
%%endhead 
 
\subsection{Перейменування України на \enquote{Русь-Україну} в наш час - безумство чистої води}
\label{sec:03_09_2021.fb.kulyk_roman.1.ukraina_rusj_bezumstvo}
 
\Purl{https://www.facebook.com/romanvkulyk/posts/4222369337831716}
\ifcmt
 author_begin
   author_id kulyk_roman
 author_end
\fi

Перейменування України на "Русь-Україну" в наш час - це безумство чистої води.
І невігластво, і безвідповідальність.

Біль, що ми зараз взагалі змушені серйозно це обговорювати. Але що вдієш -
Арестович озвучує чергові наміри і це знаходить свого прихильника. Після
встановлення Дня Української Державності, який з'явився в тч з його подачі, ці
вкиди ідей не можна ігнорувати.

1/ Отже, серйозних аргументів на користь такого перейменування держави я поки
не почув. Лише абстракції на кшталт: "вернуть нам тысячелетнее достоинство,
восстановить историческую справедливость, обьединить страну, победить Кремль
субъектно - раз и навсегда, забрать у него первородство". 

Як дивацька в ХХІ столітті подвійна назва поверне нам "достоїнство"? А як
відсутність "Русі" в назві республіки у нас це достоїнство забрало? Відновити
історичну справедливість - так давайте копати до трипільців, чого уж там.
"Украдена Росією історія" не повернеться через те, що ми вирішимо переназвати
країну. 

О ні, наслідки будуть зовсім в іншій площині.

Об'єднати країну? Тут із поточною назвою є купа громадян, для яких власна
держава - це пустий звук. Із якого дива поява "Русі" має "об'єднати країну"?
Оце "об'єднати країну" - тут взагалі до чого? Країна роз'єднана через агресію
та окупацію РФ, а не через свою назву. Не треба фантазувати, що зміна назви -
унікальна формула консолідації суспільства в наших умовах. Тоді багато інших
країн пішли б аналогічним шляхом і зараз мали б умову Галло-Францію чи Священну
Римську федерацію німців, *підставляйте далі. 

"Победить Кремль субъектно - раз и навсегда, забрать у него первородство", - це
найтупіший з аргументів. 

Яким чином переназва країни буде якоюсь "перемогою"? Перемога може бути у
війні, у боротьбі за уми громадян - шляхом просвітництва, освіти, інформаційної
політики та випілювання рос. агентів із публічного простору. Переназва держави
- це ніяка блт не перемога, це ілюзія перемоги. Перегравати тему спадку варто
через культурну дипломатію, посилення напрямку роботи з національною пам'яттю.
Але це все заскладно. 

Мотивація "назло росіянам" для такого серйозного кроку - просто дитсадок. Цей
аргумент завжди може йти виключно приємним бонусом. Підстави до подібних змін
мають обґрунтовуватись тільки й виключно раціональністю, історичним бекграундом
й доцільністю в сучасності. 

Чомусь батьки-засновники Української Народної Республіки обрали саме цю назву,
а не крутились навколо Русі. А поміж них найбільшим авторитетом користувався
саме Михайло Грушевський, який і надав цьому терміну широкого вжитку. 

2/ Я тут бачу зовсім іншу інтенцію. Арестович та нарід із його оточення
неодноразово заявляли, що Україні треба "повертати Русь". І в комплексі з
правами на російську мову, свій, "український варіант". 

Тому в закидуванні ідеї переназвати України в "Русь-Україну" (причому саме в
такому варіанті, той же Грушевський використовував термін "України-Руси") я
бачу далекоідучі наміри протягнути і російську (можливо навіть як другу
державну. а що таке? у назві ж Русь, це і наша мова, альо!). 

І як наслідок будувати і легітимізувати тут свою "ідеальну Русь-Україну", яка в
уяві Арестовича - демократична і трохи інакша, але культурно й  мовно -
субстрат Росії з вкрапленнями україномовних маргіналів. 

У мене здавна враження, що всі ці чудові люди, філософи, гуру і тд, ніби нічого
в житті не читали з історії ХХ століття. Там, де в нас - УНР та
національно-визвольна боротьба українців останніх 100 років, у них - вакуум. І
море рос. наративів довкола. До речі, уявімо, що міжнародна назва України стане
"Rus-Ukraine" (уххх, а яке скорочення чудове  @igg{fbicon.man.facepalming} ). Тут МЗС веде довготривалу
кампанію, щоби Київ за кордоном почали використовувати транслітерацію «Kyiv»,
можемо уявити, що почнеться зі зміною назви держави. 

3/ Ключовий наслідок зміни назви України - це колосальної сили удар по
українській цілісній ідентичності, яка зробила серйозний ривок у наповненні
сенсами, актуалізації та поширенню після 2014-го року. 

Всі ці інтелектуальні дискусії про правонаступництво й спробу переграти РФ на
ідеологічному полі - зараз цікаві кільком відсоткам громадян. Аналогічна
процент українців і доженуть усі "глибокі смисли" перейменування. Тотальна
більшість остаточно заплутається і загубиться, де тут Русь, де України, ми
взагалі тут і зараз - хто і якою мовою говоримо. Наслідком сум'яття в умах буде
критичне розмиття українства загалом. 

PS врешті, щоб змінити назву держави, слід вносити зміни до Розділу І
Конституції України. Тобто треба 300 голосів та всеукраїнський референдум.
Сподіваюсь цього ніколи не станеться.

\ii{03_09_2021.fb.kulyk_roman.1.ukraina_rusj_bezumstvo.cmt}

