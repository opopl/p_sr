% vim: keymap=russian-jcukenwin
%%beginhead 
 
%%file 17_02_2022.fb.goldarb_maksim.1.anglosaksy
%%parent 17_02_2022
 
%%url https://www.facebook.com/maksim.goldarb/posts/2814199668889218
 
%%author_id goldarb_maksim
%%date 
 
%%tags napadenie,rossia,ugroza,ukraina
%%title Англосаксы раздосадованы - поджечь пока не получилось. Ключевое - пока
 
%%endhead 
 
\subsection{Англосаксы раздосадованы - поджечь пока не получилось. Ключевое - пока}
\label{sec:17_02_2022.fb.goldarb_maksim.1.anglosaksy}
 
\Purl{https://www.facebook.com/maksim.goldarb/posts/2814199668889218}
\ifcmt
 author_begin
   author_id goldarb_maksim
 author_end
\fi

Англосаксы раздосадованы - поджечь пока не получилось. Ключевое - пока.

Поэтому начался второй круг воя: Байден, нато, ройтерс, встречи, звонки, «мы
предупреждаем», санкции, «увеличила количество войск», спецназ в Польшу,
прибалты любят беженцев, партии оружия и тд и тп. 

Всё-таки присущая им целеустремлённость и упорство исторически практически
всегда (только когда совсем сильно не получали по зубам) помогала. Ох как им
нужен здесь пожар! 

Но, уверен, они упустили время, это подтверждает реакция и поведение другой,
много большей части мира: Китай за мир в Украине, в частности, и в Евразии в
целом; классическая Европа - за Минск и Нормандию (к тому же ломает первая
ковидные, рождённые в Штатах, ограничения). Более того, ее лидеры челночат
между Москвой и Киевом, очевидно убеждая последний вспомнить о его
географическом положении. 

После американо-английского бегства из Афганистана у здравомыслящих мировых
политиков с ними связана ассоциация - «трусость и предательство»; после
16.02.22 - «ложь».

Даже американские сателлиты здесь, в Украине, поняли, в какое смертельное для
них танго затягивают их боссы, и нажали, как минимум, на тормоза. 

Всё это даёт Украине небольшой, но все же шанс попробовать не быть окончательно
сожжённой в топке чужих и чуждых ей «трансатлантических» интересов.
