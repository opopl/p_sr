% vim: keymap=russian-jcukenwin
%%beginhead 
 
%%file 27_01_2022.stz.news.lnr.lug_info.3.zaschitnik_respubliki_smert
%%parent 27_01_2022
 
%%url https://lug-info.com/news/zashitnik-respubliki-pogib-pri-obstrele-veselogorovki-so-storony-vsu-narodnaya-miliciya
 
%%author_id news.lnr.lug_info
%%date 
 
%%tags donbass,nm_lnr,smert,ukraina,vojna
%%title Защитник Республики погиб при обстреле Веселогоровки со стороны ВСУ – Народная милиция
 
%%endhead 
 
\subsection{Защитник Республики погиб при обстреле Веселогоровки со стороны ВСУ – Народная милиция}
\label{sec:27_01_2022.stz.news.lnr.lug_info.3.zaschitnik_respubliki_smert}
 
\Purl{https://lug-info.com/news/zashitnik-respubliki-pogib-pri-obstrele-veselogorovki-so-storony-vsu-narodnaya-miliciya}
\ifcmt
 author_begin
   author_id news.lnr.lug_info
 author_end
\fi

Военнослужащий Народной милиции ЛНР погиб в результате обстрела села
Веселогоровка со стороны киевских силовиков. Об этом сообщил офицер
пресс-службы управления Народной милиции ЛНР Александр Мазейкин.

\enquote{Сегодня киевские боевики с позиций 30-й бригады в районе населенного пункта
Троицкое, выполняя преступный приказ комбрига (Александра) Зиневича, открыли
огонь из СПГ (станкового противотанкового гранатомета) и крупнокалиберных
пулеметов в направлении населенного пункта Веселогоровка. С прискорбием
сообщаем, что в результате агрессии украинских боевиков, выполняя свой долг по
защите Республики, погиб наш военнослужащий. Выражаем искренние соболезнования
родным и близким погибшего}, - сказал он.

\ii{27_01_2022.stz.news.lnr.lug_info.3.zaschitnik_respubliki_smert.pic.1}

Офицер Народной милиции отметил, что это произошло \enquote{в то время, когда на
международных площадках идет активное обсуждение ситуации на Донбассе и
украинское руководство уверенно заявляет о своей приверженности курсу мирного
урегулирования конфликта, на линии боевого соприкосновения украинские боевики
демонстрируют истинные намерения киевского режима}.

Он добавил, что \enquote{противник сосредоточился на подготовке и проведении
провокационных действий, направленных на дестабилизацию обстановки и создание
выгодных условий для начала агрессии}.

Напомним, что участники Контактной группы по урегулированию конфликта в
Донбассе с осени 2014 года более 20 раз заявляли о достижении соглашений по
соблюдению \enquote{режима тишины} в регионе. Киевские силовики неоднократно нарушали
условия перемирия, в том числе открывая огонь из крупнокалиберных орудий,
минометов и орудий танков, которые должны были быть отведены в соответствии с
Минскими соглашениями.

Власти Украины начали силовую операцию против Донбасса в апреле 2014 года.
Урегулирование конфликта базируется на Комплексе мер по выполнению Минских
соглашений, подписанном 12 февраля 2015 года в белорусской столице участниками
Контактной группы и согласованном с главами стран - участниц \enquote{нормандской
четверки} (Россия, Германия, Франция и Украина). Документ, в частности,
предусматривает прекращение огня и отвод тяжелых вооружений от линии
соприкосновения.
