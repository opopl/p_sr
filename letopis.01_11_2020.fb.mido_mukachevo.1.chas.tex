% vim: keymap=russian-jcukenwin
%%beginhead 
 
%%file 01_11_2020.fb.mido_mukachevo.1.chas
%%parent 01_11_2020
 
%%url https://www.facebook.com/mido.mukachevo/posts/3575673815832769
%%author 
%%tags 
%%title 
 
%%endhead 

\subsection{Тихий час}
\label{sec:01_11_2020.fb.mido_mukachevo.1.chas}
\Purl{https://www.facebook.com/mido.mukachevo/posts/3575673815832769}
\Pauthor{Дочинець, Мирослав}

Днями брали інтерв’ю і запитали, чи мав карантинний режим якісь плюси для мене.

Ще й які! Вдалося уникнути стількох зібрань, засідань, фестів, форумів,
конгресів, семінарів, конференцій, симпозіумів, толок, ярмарок, виставок,
імпрез, автограф-сесій, презентацій, вшанувань, майстер-класів, круглих столів
та інших товкотнеч із обміну словами, жестами й марнославствами. Скільки
повноЦінного часу вивільнилося для прилюбних піших мандрівок на природі, для
порання в саду, для писання, читання, розмислів, для колекції кераміки, для
спілкування з рідними, просто --- для душі і тиші серця.

З часом починаєш відчувати місткість Часу й цінувати його вартість.

Можна провести час, а можна набутися в ньому. Можна згаяти час, а можна
знайти. Можна заповнити час, а можна наповнити його. Цілком різні речі.

Мирослав Дочинець. "Різнотрав'я".

\subsubsection{Коментарі}

\paragraph{Михайло Амарант}

А ось ще один погляд на Час.

Рядки з книги:

"Покликання, буває, надовго заховане в наших душах.  Чекає, поки прийде його
день. Мій день прийшов. І змінив усі мої подальші роки. Почав копати.  Земля
відкрила мені себе.  Спочатку я копав навмання. Поки не навчився вгадувати
поклади скарбів, як відуни вгадують підземні джерела.  Я копав і знаходив. Так
я "занурився" в пласти інших цивілізацій.  Я бавився епохами, перебирав, пестив
Час."

(М. Дочинець, "Книга надиху")
