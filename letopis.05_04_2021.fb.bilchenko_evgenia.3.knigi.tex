% vim: keymap=russian-jcukenwin
%%beginhead 
 
%%file 05_04_2021.fb.bilchenko_evgenia.3.knigi
%%parent 05_04_2021
 
%%url https://www.facebook.com/yevzhik/posts/3778126718889077
 
%%author 
%%author_id 
%%author_url 
 
%%tags 
%%title 
 
%%endhead 
\subsection{БЖ. Моя книжная ярмарка: день последний}
\label{sec:05_04_2021.fb.bilchenko_evgenia.3.knigi}
\Purl{https://www.facebook.com/yevzhik/posts/3778126718889077}


\ifcmt
  pic https://scontent-bos3-1.xx.fbcdn.net/v/t1.6435-9/170126999_3778126505555765_8504542507677939994_n.jpg?_nc_cat=102&ccb=1-3&_nc_sid=8bfeb9&_nc_ohc=oJjxkrturk4AX_SdbW8&_nc_ht=scontent-bos3-1.xx&oh=8418ddb89044bdcab7c7ccc80650a44c&oe=609237C9

	pic https://scontent-bos3-1.xx.fbcdn.net/v/t1.6435-9/169287025_3778126558889093_3091172896030710584_n.jpg?_nc_cat=110&ccb=1-3&_nc_sid=8bfeb9&_nc_ohc=lDjnf_4IW20AX98B8sS&_nc_ht=scontent-bos3-1.xx&oh=b84e59634fb8a69fc22646d7b77d704e&oe=609223BA

	pic https://scontent-bos3-1.xx.fbcdn.net/v/t1.6435-9/168956922_3778126632222419_5471283992660866749_n.jpg?_nc_cat=105&ccb=1-3&_nc_sid=8bfeb9&_nc_ohc=w3Rr7G7OzN4AX8PuUGQ&_nc_ht=scontent-bos3-1.xx&oh=9e1fc4325fec4f92bc427193a4d7d566&oe=609396C7

	pic https://scontent-bos3-1.xx.fbcdn.net/v/t1.6435-9/169901904_3778126678889081_3239433724420422834_n.jpg?_nc_cat=101&ccb=1-3&_nc_sid=8bfeb9&_nc_ohc=57mQE2sDLtEAX8TlbP-&_nc_ht=scontent-bos3-1.xx&oh=18eff906cb67608dcc5cdc9142837bc8&oe=60928799
\fi


Дорогие друзья! Осталось четыре последних экземпляра моих произведений, дальше будут новые. Кратко о каждом.

1. "Субъект. Символическое. Разрыв" - моя единоличная книга о том, как
разорвать с идеологическим прессингом. Нациков книга бесит.

2. "История и современность" - один из самых популярных научных журналов в
России, где опубликовали везде запрещаемую мою статью, почему для нас 9 мая -
праздник, а 8 мая - так, навязанная реклама жертвы. Либералов статья бесит.

3. "Предания о наивных праведниках" - коллективная и типа престижная
международная антология прозы, где все - таланты, а я с большими комплексами
опубликовала там свой первый и последний художественный  рассказ, как девочка
мужика отравить пыталась, но не смогла. Девочки плачут, я себя талантом в прозе
не считаю. Взяли по ошибке или переоценили, но и в моей жизни были фатальные
страсти.

4. "Человек в современном мире" - антология больших учёных Украины об
актуальном человеке постмодерна, где я одна написала, что имела постмодерного
человека в большом потребительском виду, а нравится мне сумасшедший
герой-убийца фильма А. Хичкока, покончивший с собой на сцене театра, потому что
его совесть замучила. И что моральность круче брендов. Этой главой, грешный я
человек, горжусь.

Я очень хочу делиться самым наболевшим с вами. Пишите в лички, договоримся, ибо
больше книг нет. Дальше я готовлю вам что-то совсем другое. Неожиданное.
Духовное. Вам понравится.

Репост приносит радость автору.
