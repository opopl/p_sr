% vim: keymap=russian-jcukenwin
%%beginhead 
 
%%file 24_12_2020.fb.bilchenko_evgenia.3.peterburzhanka
%%parent 24_12_2020
 
%%url https://www.facebook.com/yevzhik/posts/3501749226526829
 
%%author Бильченко, Евгения
%%author_id bilchenko_evgenia
%%author_url 
 
%%tags bilchenko_evgenia,poezia,st_peterburg
%%title БЖ. Петербуржанка
 
%%endhead 
 
\subsection{БЖ. Петербуржанка}
\label{sec:24_12_2020.fb.bilchenko_evgenia.3.peterburzhanka}
\Purl{https://www.facebook.com/yevzhik/posts/3501749226526829}
\ifcmt
 author_begin
   author_id bilchenko_evgenia
 author_end
\fi

БЖ. Петербуржанка.

Она всё понимает, хотя и я не болтаю ей ничего.
Говорят, она - ведьма. В её глазах - православное торжество.
Такой я её и помню: входящую с хлебом в храм,
Открытую всем ветрам и дворам, мирам.

А было холодно. И Васильевский одевался в изморось сумерек.
Свежая булка пахла столовой: тёплым капустным супиком.
Она садилась напротив моих подчёркнуто дурковатых
Братьев - и говорила лишнее, демонстративно спохватываясь.

Она учила Лакана Жака и на платье чёрные розы
Носила. Она могла надерзить человеку Ивана Грозного.
\enquote{Ты велик и несчастен в тайне своей}, - говорила она бойцу,
И он прощал, придвигая ближе своё несчастье к её лицу.

Я таких никогда не видела раньше. С ней было легко молчать.
Она обрывала информвойну, как дитя - политоту чата.
Она бы хотела выйти замуж за капитана,
Как в советском фильме, где есть любовь без предательства и обмана.

Вот... Всё очень просто. Смешно и просто. Безо всяких условных \enquote{если бы}:
Не бьёт на жалость и не смягчает триггер у Достоевского.
Она всё понимает и оттого вопрос её - никаков,
Ибо последний ответ едва ли вяжется с языком.

23 декабря 2020 г.

\ifcmt
  pic https://scontent-lga3-2.xx.fbcdn.net/v/t1.6435-9/132651088_3501749183193500_6985314201021276982_n.jpg?_nc_cat=102&ccb=1-3&_nc_sid=8bfeb9&_nc_ohc=xuoVYHaFR44AX9VVnnT&_nc_ht=scontent-lga3-2.xx&oh=94d08384811c41889b249373b433cef7&oe=60CBEAFF

	caption БЖ. Петербуржанка, Илл.: Ю. Анненков. Портрет М.М. Соколовской.
\fi

