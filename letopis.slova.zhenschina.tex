% vim: keymap=russian-jcukenwin
%%beginhead 
 
%%file slova.zhenschina
%%parent slova
 
%%url 
 
%%author 
%%author_id 
%%author_url 
 
%%tags 
%%title 
 
%%endhead 
\chapter{Женщина}
\label{sec:slova.zhenschina}

%%%cit
%%%cit_head
%%%cit_pic
%%%cit_text
Есть просто исторически сложившаяся роль мужчин.  Удивляет когда начинают
говорить про гендерную дискриминацию в Украине.  Да, если говорить про Украину,
то мужчин больше на руководящих должностях. Их чаще приглашают на разные ивенты
в качестве спикеров.  Только разве это говорит о дискриминации \emph{женщин}?
Почему никто не пытается разобраться в причино-следственных связях?  Никто же
не запрещает и не мешает \emph{женщине} развиваться профессионально, создавать
карьеру или строить бизнес.  Проблема в том, что большинство \emph{женщин} в
Украине этого не хотят!!! Им это не нужно. \emph{Женщины}, в подавляющем
большинстве, видят для себя совершенно другое предназначение! В основном,
родить и воспитать ребёнка.  И при этом функцию добытчика они возлагают на
мужчин. По мнению большинства \emph{женщин} именно мужчина должен содержать и
обеспечивать семью. Даже если она работает, то она считает, что мужчина должен
зарабатывать больше неё
%%%cit_comment
%%%cit_title
\citTitle{Никакой гендерной дискриминации в Украине не существует}, 
Павел Вернивский, strana.ua, 18.06.2021
%%%endcit

%%%cit
%%%cit_head
%%%cit_pic
\ifcmt
  pic https://img.vz.ru/upimg/m11/m1105803.jpg
\fi
%%%cit_text
На Украине составлен рейтинг «самых влиятельных \emph{женщин}», ведущие позиции
которого заняли такие разные люди, как Юлия Тимошенко, Елена Зеленская и
Екатерина Павленко, представлявшая страну на «Евровидении». Самое смешное, что
первое и пятое место отданы англосаксонским дипломатам, что заставляет
вспомнить о судьбе олигарха Коломойского и о том, что Украина – криптоколония.
Список «самых влиятельных \emph{женщин} Украины», составленный редакцией журнала
«Фокус» и опрошенными ею «экспертами», хотя и носит как будто развлекательный
характер, способен сказать об Украине больше, чем любое академическое
исследование. Причем чистую правду.  Главное, пожалуй, в том, что Украина –
государство не суверенное (при том, что убивает в Донбассе людей под соусом
«защиты суверенитета») и глубоко провинциальное. Чтобы понять это, знакомиться
со всей сотней нет никакого смысла – достаточно первой тройки
%%%cit_comment
%%%cit_title
\citTitle{Список самых влиятельных женщин стал позором Украины}, 
Станислав Борзяков, vz.ru, 26.06.2021
%%%endcit
