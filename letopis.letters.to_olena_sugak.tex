% vim: keymap=russian-jcukenwin
%%beginhead 
 
%%file letters.to_olena_sugak
%%parent letters
 
%%url 
 
%%author_id 
%%date 
 
%%tags 
%%title 
 
%%endhead 

Доброе утро, Елена! 

С огромным удовольствием смотрю Ваши фото в группе Мариуполь Довоенный, больше всего, мне 
наверное, нравится Ваша публикация про Старые Двери, и про Шторм в море...
А Сегодня я неожиданно обнаружил, что Вы также вовсю фотографируете мой любимый Киев!
Особенно, озера, и вообще места, где я сам еще не бывал, хоть родился, вырос и живу в Киеве!

Знаете, уже давно как бы собираюсь написать Вам, но вообще то все откладывал,
потому что есть риск, что я могу здорово рассердить Вас, а почему - станет ясно ниже...
Но фото Киева и озер Киева меня просто вдохновили, я бы даже сказал, вставили как следует! 
Вы открываете Киев и его окрестности, не только для себя,
а также для киевлян! Это так круто! Спасибо Вам!

Ну что ж...

Меня зовут Иван, я киевлянин, программист и живу и работаю в Киеве. Много
сейчас читаю про Мариуполь, у меня есть также несколько книжек про Ваш
замечательный Город. К сожалению, до войны не успел побывать в Мариуполе, так
что меня... с Мариуполем связывает только воспоминание о том, как я когда то
сел на поезд Киев-Мариуполь, и потом сошел на промежуточной станции... Давно
это уже было... Но... я принимаю близко к сердцу, все, что произошло с
Мариуполем и с мариупольцами. Страшная, невообразимая трагедия...  которая
забрала ваших близких, родных, детей, родителей, раскидала вас, мариупольцев,
по всему миру... Тем не менее, я уверен, лучшие времена для Мариуполя еще
впереди, и отчаиваться точно не надо... Да, как я сказал выше, я программист. В
свободное время от работы... я также занимаюсь систематизацией разных
публикаций о войне, у меня есть отдельный проект Летописи Войны, который я веду
- пока что сам - довольно уже давно, собственно говоря, я сохраняю от забвения
посты на фейсбуке, потом раскидываю их по авторам, темам. В общем, это нужно
делать, и прежде всего самим авторам - потому что все что на фейсбуке находится
- все время есть риск, что это все пропадет (как это случилось, например, с
постами Игоря Ярмоленка в вашей группе - он удалил свой аккаунт - и все, его
постов там уже нет).  А технология, которую я использую, называется LaTeX (не
знаю, имеете ли вы отношение к физике или математике, но это та технология,
которая повсеместно используется учеными в научном мире для записи и публикации
своих статей и исследований ). Так вот.  Я сохраняю в печатном виде, то, что я
читаю, считаю важным для сохранения...  Про Мариуполь я в последнее время тоже
сохраняю, и довольно много. И про довоенный мирный Мариуполь, и для ужасы
войны. У меня есть телеграм канал https://t.me/kyiv_fortress_1, там уже
довольно много выложено, так же как и на этом фейсбук - аккаунте, также я
постепенно выкладываю собранные  материалы - с уважением к авторам - я всегда
указываю, кто и когда что создал - на https://archive.org - это интернет сайт,
посвященный сохранению всего в Интернете, ссылка на меня там
https://archive.org/details/@kyiv_chronicler.  Зачем все это делается...
знаете... Мариуполь физически то убили... но Мариуполь не умер. Мариуполь
остался в Духовном Пространстве, и это очень очень важно... Но если не
записывать... если этим не заниматься... есть риск, что Мариуполь умрет
духовно... и это будет уже навсегда. А духовная смерть - это...  еще страшнее,
чем смерть физическая... Поэтому я этим и занимаюсь, хотя я и киевлянин, вообще
то говоря. И личного интереса у меня в этом нет - ну то есть я не собираюсь
присваивать себе, что сделали другие - меня, наоборот, волнует... волнует
сохранность памяти... Чтобы все что пишется, не пропало, не растаяло...  И
хочется, знаете, тоже приехать в наш Украинский Возрожденный Мариуполь, как
гость, в свое время ) Пока что я... заехал в гости в Мариуполь чисто
виртуально, знаете, как Паганель из романа Жюль-Верна Дети Капитана Гранта,
который 20 лет обо всем читал, но нигде не был... А уж потом собрался поехать
исследовать Индию... Вспомнилась эта замечательная книжка, да... Надеюсь, все о
чем я пишу, Вам будет интересно. Я думаю послать Вам запрос на дружбу в
фейсбуке, буду рад подружиться с Вами здесь. 

Да... и насчет рассердить. Как Вы может быть уже поняли, я записываю также
посты из группы Мариуполь Довоенный, где вы один из главных модераторов и
вдохновителей. Не сочтите то, что я делаю, за воровство. Кроме того, во всем
этих постах есть два Автора - один автор - это конкретный человек, а второй
автор - это сам Город Мариуполь... И как я уже сказал, личного интереса у меня
в этом нет, я стараюсь всегда записывать наиболее точно, кому и что принадлежит
в сохраненных постах.  ... И это все нужно делать, и на самом деле это дело
мариупольцев, в первую очередь - сохранить всю вашу память о Мариуполе,  чтобы
ничего не пропало, и чтобы были силы и Дух отстроить заново Мариуполь, когда
придет время.  И также есть большая проблема сейчас в Киеве и Украине,
знаете... с книжками про мирный, счастливый, довоенный Мариуполь. Их очень мало
либо вообще нет в Киеве.  Я об этом написал более детально в моем посте на моей
страничке. 

С уважением, Иван.

PS. Фото дверей в Киеве, недавно... как будто похоже на двери в Мариуполе,
а Вы как думаете?
