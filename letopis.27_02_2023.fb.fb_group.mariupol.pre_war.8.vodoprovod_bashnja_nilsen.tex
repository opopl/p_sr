%%beginhead 
 
%%file 27_02_2023.fb.fb_group.mariupol.pre_war.8.vodoprovod_bashnja_nilsen
%%parent 27_02_2023
 
%%url https://www.facebook.com/groups/1233789547361300/posts/1418542325552687
 
%%author_id fb_group.mariupol.pre_war,elena_mariupolskaja
%%date 27_02_2023
 
%%tags mariupol,mariupol.istoria,mariupol.pre_war,nilsen_viktor.mariupol.arhitektor,arhitektura
%%title В начале XX века "отцы города" пришли к идее строительства в Мариуполе водопровода
 
%%endhead 

\subsection{В начале XX века \enquote{отцы города} пришли к идее строительства в Мариуполе водопровода}
\label{sec:27_02_2023.fb.fb_group.mariupol.pre_war.8.vodoprovod_bashnja_nilsen}
 
\Purl{https://www.facebook.com/groups/1233789547361300/posts/1418542325552687}
\ifcmt
 author_begin
   author_id fb_group.mariupol.pre_war,elena_mariupolskaja
 author_end
\fi

В начале XX века \enquote{отцы города} пришли к идее строительства в Мариуполе
водопровода. А в это самое время недавний выпускник Петербургского института
корпуса гражданских инженеров Виктор Нильсен опубликовал небольшую книжечку. В
ней он изложил свой опыт проектирования и строительства водопровода в городе
Рыбинске. Книжка попалась на глаза кому-то из гласных Мариупольской думы, а
может быть, чиновнику городской управы, и ее содержание стало достоянием
местных властей. Решено было пригласить автора в Мариуполь на место городского
архитектора, так как, нужно думать, в обязанности этого должностного лица
входило, в том числе, и устройство водопровода.

Сказано — сделано. Вскоре новый мариупольский городской архитектор приступил к
своим обязанностям. За сравнительно короткий срок он спроектировал сеть
водоснабжения Мариуполя, а к 1910 году на Константиновской улице (теперь ул.
Нильсена) была выстроена водонапорная башня, а на Малофонтанной — насосная
станция, где был организован водозабор из известного уже читателю фонтана.
Поршневыми насосами воду по трубам подавали в бак водонапорной башни. А уж из
бака она распределялась по колонкам-домикам, где горожане за небольшую плату
могли наполнить свои ведра живительной влагой.

%\ii{27_02_2023.fb.fb_group.mariupol.pre_war.8.vodoprovod_bashnja_nilsen.cmt}
