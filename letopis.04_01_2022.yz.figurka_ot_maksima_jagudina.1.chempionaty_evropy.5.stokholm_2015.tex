% vim: keymap=russian-jcukenwin
%%beginhead 
 
%%file 04_01_2022.yz.figurka_ot_maksima_jagudina.1.chempionaty_evropy.5.stokholm_2015
%%parent 04_01_2022.yz.figurka_ot_maksima_jagudina.1.chempionaty_evropy
 
%%url 
 
%%author_id 
%%date 
 
%%tags 
%%title
 
%%endhead 
\subsubsection{Чемпионат Европы 2015 (Стокгольм)}

Новый сезон и новый триумф \enquote{набирающего обороты} российского женского фигурного
катания. Спустя 13 лет мы снова видим полностью российский пьедестал почета.
Главные героини прошлого сезона ушли в тень, а на их место пришли новые. 

\ii{04_01_2022.yz.figurka_ot_maksima_jagudina.1.chempionaty_evropy.5.stokholm_2015.pic.1}

Лена Радионова, наконец, доросла до главных стартов, перед самым турниром ей
исполнилось 16 лет, она только что выиграла Чемпионат России, но в Стокгольме
при двух чистых прокатах Лизы и Лены все решили какие-то десятые в пользу Лизы
Туктамышевой. На последующем дальше Чемпионате Мира Лена не лучшим образом
откатает произвольную программу, но тем не менее добьется главного успеха в
своей карьере - выиграет бронзовую медаль.

\ii{04_01_2022.yz.figurka_ot_maksima_jagudina.1.chempionaty_evropy.5.stokholm_2015.pic.2}

Лиза Туктамышева после большой неудачи в Олимпийском сезоне абсолютно
триумфально выступила в этом сезоне. Победа на Чемпионате Европы стала в списке
побед самой сложной. Кроме Европы 18-летняя Лиза выиграла финал Гран-при и
Чемпионат Мира, после чего пропала из числа участниц главных стартов на долгие
6 лет.

Анна Погорилая в свои 16 лет проводила второй взрослый сезон и уже успела стать
четвертой на прошлом чемпионате Мира. В Стокгольме она выступила гораздо менее
стабильно, чем Лена с Лизой, допустила несколько ошибок, но опуститься ниже 3
места просто не могла в силу отсутствия конкуренции. На последующем Чемпионате
Мира Анна очень сильно огорчит своих болельщиков и займет лишь 13 место.
