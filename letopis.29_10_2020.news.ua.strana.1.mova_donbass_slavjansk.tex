% vim: keymap=russian-jcukenwin
%%beginhead 
 
%%file 29_10_2020.news.ua.strana.1.mova_donbass_slavjansk
%%parent 29_10_2020
 
%%url https://strana.ua/news/297825-slavjansk-skandal-v-marshrutke-iz-za-russkoho-jazyka-popal-na-video.html
%%author 
%%tags 
%%title 
 
%%endhead 

\subsection{Мовный скандал на Донбассе. Как из-за сторонницы Порошенко хотят посадить пассажиров маршрутки}

\Purl{https://strana.ua/news/297825-slavjansk-skandal-v-marshrutke-iz-za-russkoho-jazyka-popal-na-video.html}

\ifcmt
img_begin 
	url https://strana.ua/img/article/2978/25_main.jpeg
	caption Автовокзал города Славянска Донецкой области. Фото с сайта zruchno.travel 
	width 0.7
img_end
\fi

В Славянске произошел мовно-коронавирусный скандал.

Местная активистка и сторонница Порошенко устраивала рейды по маршруткам и
требовала от людей надевать маски, снимая их на видео.

В одном из автобусов с пассажирами по этому поводу возник конфликт, в ходе
которого активистку начали критиковать за украинский язык и заставили выйти из
маршрутки. При этом ее физически никто не трогал. 

Делом озаботились в центральном аппарате МВД и открыли уголовное производство -
за ущемление граждан по признаку языка. И теперь оппонентам активистки грозит
реальный срок. 

Видимо, дело открыли потому, что в лице общественной деятельницы жители
Славянска высказали свое отношение вообще к политике властей - как на ниве
карантина, так и на почве украинизации. Ведь если бы в аналогичном случае
пассажиры "закошмарили" сторонника, скажем, русского языка, то никаких мер
наверняка бы никто не принимал.

"Страна" разбиралась в этой истории.  

\subsubsection{Что произошло в маршрутке}

Вчера в соцсетях появилось видео конфликта в маршрутном такси №9. Люди в
автобусе поспорили о том, на каком языке говорить в быту. После чего защитницу
украинского языка вытолкали из маршрутки. 

Правда, претензии к ней были не столько по поводу мовы. 

На видео женщина, которая говорит на украинском языке, спорит с остальными
пассажирами и называет себя "гражданской активисткой". Снимает всё тоже она. 

Общение происходит на разных языках, мы цитируем все дословно, без перевода. 

Женщина-пассажир: "Пусть смотрит. Мы уже насмотрелись, довели страну.
Громадська діячка! Приехали сюда, нашему востоку жизни от вас нет"

Активистка: "Шо ви... Я живу у місті Слов’янську"

Женщина: "Тогда и балакай по-слов’янські". 

Общественній деятель: "Я розмовляю тією мовою..."

Женщина: "Заколебали такие діячкі". 

Активистка: "Я зі Слов’янську, я народилася у Слов’янську". 

Женщина: "Так вот и да, оно и видно, что ты у Слов’янську народилася. У
Слов’янську нормальные люди все, адекватные". 

Активистка: "Та ви шо, це ви ненормальна жіночка". 

Женщина: "Давай, снимай-снимай. Со своими мовами заколебали уже". 

Активистка: "Оооо!"

Женщина: "Уже не знаешь, на каком языке говорить"

Активистка: "Серйозно? Так ви не знаєте..."

Женщина: "Серьезно, да, из-за таких, как вы..."

Активистка: "А, ну давайте"

Женщина, спорившая с активисткой, вышла на остановке, от других пассажиров
доносится: "Снимай-снимай" и "Х@ли?"  

Потом с места поднялся мужчина-пассажир, который все это слушал, подошел к
общественному деятелю и сказал: "Документы покажи". 

Активистка: "Які документи?"

Мужчина: "Документы покажи, хто ты такая вообще?"

Активистка: "Я?"

Мужчина: "Хто ты какая вообще? Хто ты какая?"

Активистка: "А шо вас цікавить взагалі?"

Мужчина: "Хто ты такая вообще? На каком основании ты снимаешь?"

Активистка: "Я виходжу" [на этой остановке]. 

Мужчина: "Выходи отсюда вообще, чтобы я тебя не видел". 

Активистка: "Руки прибери". 

Мужчина: "Вышла отсюда н@хуй". 

Активистка: "Ви, будь ласка, фільтруйте свою мову". 

Мужчина: "Иди фильтруй". 

Активистка: "Руки прибери. Це моя зупинка" [мужчина не трогает деятеля,
маршрутка остановилась, дверь открылась, но она продолжала не выходить, а потом
она остановила съемку]

\url{https://youtu.be/ceeAYoJACY4}
Спор о языке в маршруте в Славянке Донецкой области.

То есть очевидно, что украинский язык был не основной темой конфликта - он
произошел по какой-то другой причине. А уже в ходе перепалки жители Славянска
высказали активистке претензии в том числе и по поводу украинизации. 

Тем не менее по факту происшествия было возбуждено уголовное дело. Об этом
сообщил в Facebook советник главы МВД Иван Варченко.

"Лица, которые напали на украиноязычную женщину в Славянске, понесут наказание
за разжигание вражды и оскорбление чувств граждан по языковым признакам.
Уголовный кодекс Украины предусматривает наказание за разжигание вражды и
оскорбление чувств граждан по языковым признакам (ст.161). Санкция за
издевательство над людьми, которые говорят на украинском, - до 5 лет
заключения. Правовым путем им будет доказано, что Славянск - это Украина", -
отметил Варченко. 

В полиции заявили, что нашли видео по этому инциденту в интернете и открыли
дело по факту. Но уровень вмешательства поражает - о деле, в котором нет даже
телесных повреждений и которое тянет максимум на легкую "хулиганку", заговорили
в центральном аппарате МВД.

\ifcmt
pic https://strana.ua/img/forall/u/10/91/%D0%A1%D0%BD%D0%B8%D0%BC%D0%BE%D0%BA_%D1%8D%D0%BA%D1%80%D0%B0%D0%BD%D0%B0_2020-10-29_%D0%B2_11.30_.11_.png
\fi

В общем, назревает серьезное "политическое" расследование. Хотя по факту сам
конфликт, судя по всему, разгорелся вообще не из-за мовы, как это пытаются
показать в МВД. 

\subsubsection{Почему завязалась перепалка?}

Активисткой, которая поспорила с маршрутке, оказалась местная жительница Рита
Мазанкова. Она волонтер АТО и, судя по фотографиям на Facebook, сторонница
Петра Порошенко. "Страна" нашла ее аккаунт в соцсетях. 

\ifcmt
pic https://strana.ua/img/forall/u/0/92/%D0%BC%D0%B0%D0%B7%D0%B0%D0%BD%D0%BA%D0%BE%D0%B2%D0%B0.jpg
\fi

Чуть более полное видео перепалки она опубликовала еще 23 октября у себя на
Facebook.

Из него становится понятно, что Мазанкова потребовала у пассажиров надеть маски
и призывала водителя не трогаться с места, пока все этого не сделают.

Вот начало инцидента. По нему видно, что некоторых пассажиров раздражает, что
активистка их снимает и требует надевать маски. Мужчина, который сидит у окна,
пытается закрыть ей рукой телефон. 

Однако эта попытка, видимо, не удалась, и она продолжила снимать видео. На нем
пассажиры требуют от водителя ехать. 

"Весь автобус просит вас: едьте. У всех есть маски. Я не могу ее одеть, потому
что у меня астма", - заявила одна пассажирка водителю. "А у меня давление,
девушка. Мне плохо. Сразу в больницу везите. И вместе с "громадскістю" будете
довольны", - поддержала ее женщина с задних рядов. 

После этого мужчина, который во второй части видео пытался вытолкать активистку
из автобуса, встал и пошел на нее. Однако другие пассажиры его удержали:
"Мужчина, не опускайтесь до этого". После чего он сел на место. 

Дальше последовала та перепалка, которую мы цитировали выше. Закончилось все
выходом Мазанковой из автобуса. 


AddThis Sharing Buttons
Share to Twitter
Twitter
Share to Facebook
Facebook
Share to Telegram
Telegram
Share to Viber
Viber
Автовокзал города Славянска Донецкой области. Фото с сайта zruchno.travel
Автовокзал города Славянска Донецкой области. Фото с сайта zruchno.travel

В Славянске произошел мовно-коронавирусный скандал.

Местная активистка и сторонница Порошенко устраивала рейды по маршруткам и требовала от людей надевать маски, снимая их на видео.

В одном из автобусов с пассажирами по этому поводу возник конфликт, в ходе которого активистку начали критиковать за украинский язык и заставили выйти из маршрутки. При этом ее физически никто не трогал. 

Делом озаботились в центральном аппарате МВД и открыли уголовное производство - за ущемление граждан по признаку языка. И теперь оппонентам активистки грозит реальный срок. 

Видимо, дело открыли потому, что в лице общественной деятельницы жители Славянска высказали свое отношение вообще к политике властей - как на ниве карантина, так и на почве украинизации. Ведь если бы в аналогичном случае пассажиры "закошмарили" сторонника, скажем, русского языка, то никаких мер наверняка бы никто не принимал.

"Страна" разбиралась в этой истории.  
Что произошло в маршрутке

Вчера в соцсетях появилось видео конфликта в маршрутном такси №9. Люди в автобусе поспорили о том, на каком языке говорить в быту. После чего защитницу украинского языка вытолкали из маршрутки. 

Правда, претензии к ней были не столько по поводу мовы. 

На видео женщина, которая говорит на украинском языке, спорит с остальными пассажирами и называет себя "гражданской активисткой". Снимает всё тоже она. 

Общение происходит на разных языках, мы цитируем все дословно, без перевода. 

Женщина-пассажир: "Пусть смотрит. Мы уже насмотрелись, довели страну. Громадська діячка! Приехали сюда, нашему востоку жизни от вас нет"

Активистка: "Шо ви... Я живу у місті Слов’янську"

Женщина: "Тогда и балакай по-слов’янські". 

Общественній деятель: "Я розмовляю тією мовою..."

Женщина: "Заколебали такие діячкі". 

Активистка: "Я зі Слов’янську, я народилася у Слов’янську". 

Женщина: "Так вот и да, оно и видно, что ты у Слов’янську народилася. У Слов’янську нормальные люди все, адекватные". 

Активистка: "Та ви шо, це ви ненормальна жіночка". 

Женщина: "Давай, снимай-снимай. Со своими мовами заколебали уже". 

Активистка: "Оооо!"

Женщина: "Уже не знаешь, на каком языке говорить"

Активистка: "Серйозно? Так ви не знаєте..."

Женщина: "Серьезно, да, из-за таких, как вы..."

Активистка: "А, ну давайте"

Женщина, спорившая с активисткой, вышла на остановке, от других пассажиров доносится: "Снимай-снимай" и "Х@ли?"  

Потом с места поднялся мужчина-пассажир, который все это слушал, подошел к общественному деятелю и сказал: "Документы покажи". 

Активистка: "Які документи?"

Мужчина: "Документы покажи, хто ты такая вообще?"

Активистка: "Я?"

Мужчина: "Хто ты какая вообще? Хто ты какая?"

Активистка: "А шо вас цікавить взагалі?"

Мужчина: "Хто ты такая вообще? На каком основании ты снимаешь?"

Активистка: "Я виходжу" [на этой остановке]. 

Мужчина: "Выходи отсюда вообще, чтобы я тебя не видел". 

Активистка: "Руки прибери". 

Мужчина: "Вышла отсюда н@хуй". 

Активистка: "Ви, будь ласка, фільтруйте свою мову". 

Мужчина: "Иди фильтруй". 

Активистка: "Руки прибери. Це моя зупинка" [мужчина не трогает деятеля, маршрутка остановилась, дверь открылась, но она продолжала не выходить, а потом она остановила съемку]

Спор о языке в маршруте в Славянке Донецкой области 

То есть очевидно, что украинский язык был не основной темой конфликта - он произошел по какой-то другой причине. А уже в ходе перепалки жители Славянска высказали активистке претензии в том числе и по поводу украинизации. 

Тем не менее по факту происшествия было возбуждено уголовное дело. Об этом сообщил в Facebook советник главы МВД Иван Варченко.

"Лица, которые напали на украиноязычную женщину в Славянске, понесут наказание за разжигание вражды и оскорбление чувств граждан по языковым признакам. Уголовный кодекс Украины предусматривает наказание за разжигание вражды и оскорбление чувств граждан по языковым признакам (ст.161). Санкция за издевательство над людьми, которые говорят на украинском, - до 5 лет заключения. Правовым путем им будет доказано, что Славянск - это Украина", - отметил Варченко. 

В полиции заявили, что нашли видео по этому инциденту в интернете и открыли дело по факту. Но уровень вмешательства поражает - о деле, в котором нет даже телесных повреждений и которое тянет максимум на легкую "хулиганку", заговорили в центральном аппарате МВД.

МВД Украины фейсбук

В общем, назревает серьезное "политическое" расследование. Хотя по факту сам конфликт, судя по всему, разгорелся вообще не из-за мовы, как это пытаются показать в МВД. 
Почему завязалась перепалка?

Активисткой, которая поспорила с маршрутке, оказалась местная жительница Рита Мазанкова. Она волонтер АТО и, судя по фотографиям на Facebook, сторонница Петра Порошенко. "Страна" нашла ее аккаунт в соцсетях. 

Чуть более полное видео перепалки она опубликовала еще 23 октября у себя на Facebook.

Из него становится понятно, что Мазанкова потребовала у пассажиров надеть маски
и призывала водителя не трогаться с места, пока все этого не сделают.

Вот начало инцидента. По нему видно, что некоторых пассажиров раздражает, что активистка их снимает и требует надевать маски. Мужчина, который сидит у окна, пытается закрыть ей рукой телефон. 

Однако эта попытка, видимо, не удалась, и она продолжила снимать видео. На нем
пассажиры требуют от водителя ехать. 

"Весь автобус просит вас: едьте. У всех есть маски. Я не могу ее одеть, потому
что у меня астма", - заявила одна пассажирка водителю. "А у меня давление,
девушка. Мне плохо. Сразу в больницу везите. И вместе с "громадскістю" будете
довольны", - поддержала ее женщина с задних рядов. 

После этого мужчина, который во второй части видео пытался вытолкать активистку
из автобуса, встал и пошел на нее. Однако другие пассажиры его удержали:
"Мужчина, не опускайтесь до этого". После чего он сел на место. 

Дальше последовала та перепалка, которую мы цитировали выше. Закончилось все
выходом Мазанковой из автобуса. 

То есть видно, что основной негатив пассажиров вызвали два фактора.

Первый - это съемка пассажиров, которые сидят без масок. Люди требовали не
снимать их на видео, но активистка это проигнорировала. Второй - остановка
маршрутки по требованию Мазанковой. Которая по сути задерживала людей, присвоив
себе право решать, ехать им или нет. При этом заявление пассажирки, что ей
плохо, активистка игнорировала. 

А вот тема языка к конфликту вообще не имеет отношения - ругань разгорелась еще
до упоминания о "мове". И, как видим, самый агрессивный из пассажиров, пытался
"воздействовать" на активистку еще до того, как вообще заговорили о языке. 

Интересно, что подобные рейды по маршруткам Маргарита осуществляет регулярно. 

"На протяжении всего маршрута просила надевать маски, люди по-разному
реагировали, но одевали. Это случай, когда люди ищут виновных где угодно, когда
угодно. Кого угодно. Виноваты все! Власть ... Язык ... Кроме нас, любимых", -
прокомментировала она инцидент в марштурке.

\ifcmt
pic https://strana.ua/img/forall/u/0/92/%D1%80%D0%B8%D1%82%D0%B01.png 
\fi

Предыдущий рейд произошел за день до инцидента, по поводу которого открыли
уголовное дело. Там ее тоже попросили выйти из автобуса. 

\ifcmt
pic https://strana.ua/img/forall/u/0/92/image_2020-10-29_132116.png
\fi

Через два дня после скандала в маршрутке, активистка начала поиски мужчины,
который бросался на нее. 

\ifcmt
pic https://strana.ua/img/forall/u/0/92/image_2020-10-29_132512.png
\fi

То есть видимо с этим и связано уголовное дело, открытое полицией. При этом
отметим, что человек, пытавшийся напасть на Маргариту (но дальше угроз дело не
пошло), ни слова не говорил по поводу украинского языка. Хотя полиция планирует
"закрыть" обидчиков именно по мовной теме. 

Отношение у комментаторов девушки к тому, что она делает, неоднозначное. 

"Смотрю на стадо и начинаю понимать, что бараны не заслуживают нормальной
жизни. Им достаточно простого существования", - рассуждает в комментариях у
волонтерши боец батальона "Донбасс" Святослав Грушевский.

\ifcmt
pic https://strana.ua/img/forall/u/0/92/image_2020-10-29_133349.png
\fi

Другие пользователи упрекнули Мазанкову, что она "достает" людей, у которых и
так нелегкая жизнь. И ей, видимо, нечем заняться. 

\ifcmt
pic https://strana.ua/img/forall/u/0/92/image_2020-10-29_133612.png
\fi

Волонтершу также упрекают, что она снимала людей без их согласия.

\ifcmt
pic https://strana.ua/img/forall/u/0/92/image_2020-10-29_133839.png
\fi

В целом же этот инцидент крайне показательный с точки зрения отношения жителей
Славянск к так называемым "активистам", которые пытаются качать права по тому
или иному поводу.

Самое мягкое определение этого отношения - сильное раздражение и презрение. А в
некоторых случаях и ненависть. Люди уже прямым текстом посылают их на три
буквы.

И украинский язык в данном случае на самом деле не причем. Просто люди
воспринимают таких "активистов" как олицетворение курса, по которому идет
страна последние шесть лет. И отношение к этому курсу у них вполне однозначное.
