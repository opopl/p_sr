% vim: keymap=russian-jcukenwin
%%beginhead 
 
%%file 22_01_2022.fb.fb_group.story_kiev_ua.1.den_sobornosti_ukrainy
%%parent 22_01_2022
 
%%url https://www.facebook.com/groups/story.kiev.ua/posts/1845605265636266
 
%%author_id fb_group.story_kiev_ua,kirkevich_viktor.kiev
%%date 
 
%%tags den.ukr.sobornosti,edinstvo,kiev,kievljane,obschestvo,prazdnik,sobornist,ukraina
%%title Хочу усіх привітати з Днем Соборності України
 
%%endhead 
 
\subsection{Хочу усіх привітати з Днем Соборності України}
\label{sec:22_01_2022.fb.fb_group.story_kiev_ua.1.den_sobornosti_ukrainy}
 
\Purl{https://www.facebook.com/groups/story.kiev.ua/posts/1845605265636266}
\ifcmt
 author_begin
   author_id fb_group.story_kiev_ua,kirkevich_viktor.kiev
 author_end
\fi

Хочу усіх привітати з Днем Соборності України. Цим надзвичайно важливим
державим святом, що відзначається на неналежному рівні. Шкода! Нам Єдності конче
не вистачає!

У січні 2018 року я демонстрував фрагмент своєї колекції, присвяченої бою під
Крутами та російської агресії на Україну в січні - лютому 1918 року. Експозиція
була представлена в Державному військово-історичному музеї, розташованому в
Кріпосному провулку. Нажаль мало хто знає про нього. Це єдиний в Україні музей,
що охоплює історію створення і розвитку військової справи на теренах України
від доби бронзи до сьогодення.Горджуся, що саме тут зараз експонується частина
моєї колекції, присвяченої славному гетьману Івану Мазепі. Багато років активно
просуваю створення в столиці України музеїв Державотворення та незалежності та
Івана Мазепи.У цьому напрямку справа просувається занадто повільно. Мені вже
76, до того ж, я інвалід – погано пересуваюсь. Не можу вже їздити світом,
збираючи матеріали славної історії України. Хоча, в свій час, назбирав їх
чимало! Вони демонструвалися на виставках:у Національному музеї Берліна в 2004
році,в Швеції, в місті Льончопінг у 2009-му,в нью-йоркському Українському
музеї,2010 рік, а також 40 виставок у Києві. Недаремно отримав прізвисько
«Вічний Експозіціонер». Але я вірю, що цім музеям бути! А поки, як кажуть «Не
було на молоці – не буде й на сироватці!

Чому я начебто звітую вам, шановні Кияни, - дуже важкоо боротися із сірістю та
бездушністю населення. Так, населення! «Тяжелая это работа -  тянуть
интеллигента из болота!» (Корній Чуковський.) Я продаю свої книги на
Андріївському узвозі. Їх у мене вийшло 37, з котрих 30 – історія і описи Києва
та України. Більшість російською мовою, якою мені легше писати. З десяток книжок
видані українською: «Собор Покрови», «Гетьманська столиця», «Церкви та
монастирі Сіверщини», «Київ» тощо. Книга «Київ» була перекладена китайською.
Всі мої книги глибоко патріотичні. Київ - моє рідне місто, Україна - моя
Батьківщина. Я словом і ділом її захищаю! І хочу, щоб мої онук та п'ять онучок
жили в цій чудовій країні, яка має все, окрім порядку та єдиності її мешканців!

Чому я наче сповідуюсь? Помітив, що в своїх публікаціях у фейсбуківській групі
«Киевские истории», я іноді отримую якісь незрозумілі докори та необгрунтовані
підозри. Сьогодні розповів у цій групі про представника четвертого покоління
киян моєї родини – онука Дмитра. Отримав 420 лайків та... три ложки дьогтю.
Якийсь Вадим Сандино, гидко заявив мені, що мій онук брав участь у руйнуванні
пам'ятника Леніну. Дмитро участі в цьому не брав! Це наклеп! На мою думку,
монумент слід було прибрати, але не трощити. Цей же Сандино звинуватив корінних
киян в усіх гріхах, окрім Содома (прізвище співзвучне) и Гомори. Якась особина
з японським прізвищем Kata Taki, звинуватила мене, обізвавши «тупим» та
«ватником». Були випади та підозри, щодо як «мои родственники могли выжить в
годы голодомора, Гулага, Бабьего яра...» від Вячеслава Мельника, який порадив
мені «скорее драпать»... Від кого, чи не від нього?!

Я пишу, и буду в подальшому писати книги, статті, нотатки для мислячих людей,
яких 99 відсотків, і це я відчуваю! Честь їм и хвала! Слава Україні!
