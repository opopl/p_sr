% vim: keymap=russian-jcukenwin
%%beginhead 
 
%%file 03_04_2021.fb.1.bratstvo_propaganda_rossia_ukraina
%%parent 03_04_2021
 
%%url https://www.facebook.com/permalink.php?story_fbid=3554103368033373&id=100003012336887
 
%%author 
%%author_id 
%%author_url 
 
%%tags 
%%title 
 
%%endhead 

\subsection{Тєлєга про братерство, сестринство та наочну агітацію в СРСР. З картинками}
\label{sec:03_04_2021.fb.1.bratstvo_propaganda_rossia_ukraina}
\url{https://www.facebook.com/permalink.php?story_fbid=3554103368033373&id=100003012336887}

\ifcmt
  pic https://scontent-yyz1-1.xx.fbcdn.net/v/t1.6435-9/168819235_3554098761367167_4327497035207723479_n.jpg?_nc_cat=102&ccb=1-3&_nc_sid=730e14&_nc_ohc=P7MEEzIt-Z0AX8fHcGF&_nc_oc=AQl9DYUNBa0JLgszeTwihkB7WkePiXobs-EthZGNOZnWdRbyVpUQbhMMEAx-JQDYHBE&_nc_ht=scontent-yyz1-1.xx&oh=d8773a864cc5e3175ceb52ead7ed852d&oe=6090E4AC

  pic https://scontent-yyz1-1.xx.fbcdn.net/v/t1.6435-9/168594839_3554098944700482_8345354639743135673_n.jpg?_nc_cat=108&ccb=1-3&_nc_sid=730e14&_nc_ohc=oU6c_kZ2a_UAX-VvRrl&_nc_ht=scontent-yyz1-1.xx&oh=f0506e37d8f524f8c4938138994c96ee&oe=6090B4BA

  pic https://scontent-yyz1-1.xx.fbcdn.net/v/t1.6435-9/169002566_3554099208033789_6346749922029650886_n.jpg?_nc_cat=110&ccb=1-3&_nc_sid=730e14&_nc_ohc=AoML-9SkqTEAX8AQzSg&_nc_ht=scontent-yyz1-1.xx&oh=d239971d50a5af8a64523bfc35edb843&oe=60909474

  pic https://scontent-yyz1-1.xx.fbcdn.net/v/t1.6435-9/168819255_3554099354700441_6797194458755631866_n.jpg?_nc_cat=110&ccb=1-3&_nc_sid=730e14&_nc_ohc=oD-MCu6sgJkAX-WGVVL&_nc_ht=scontent-yyz1-1.xx&oh=4f01a4749b0800ec750f922e02fc94b3&oe=60904732

  pic https://scontent-yyz1-1.xx.fbcdn.net/v/t1.6435-9/168925426_3554099581367085_7721554902474177460_n.jpg?_nc_cat=109&ccb=1-3&_nc_sid=730e14&_nc_ohc=YIgoSc7hxUwAX8HdBm1&_nc_ht=scontent-yyz1-1.xx&oh=e349b6b1daf783ccdc6142553ab15d07&oe=60904AC7

  pic https://scontent-yyz1-1.xx.fbcdn.net/v/t1.6435-9/169157589_3554099831367060_713799781320423916_n.jpg?_nc_cat=100&ccb=1-3&_nc_sid=730e14&_nc_ohc=N9F_TVV1kzwAX8snr4c&_nc_ht=scontent-yyz1-1.xx&oh=a87f4a552b45888686131b805013d6bd&oe=609036DB

  pic https://scontent-yyz1-1.xx.fbcdn.net/v/t1.6435-9/168646297_3554099934700383_7648706176476197383_n.jpg?_nc_cat=107&ccb=1-3&_nc_sid=730e14&_nc_ohc=j6W--Hc72_sAX_eWkqk&_nc_ht=scontent-yyz1-1.xx&oh=ce2640456b42d141d36a960edd2519a7&oe=609015B7

  pic https://scontent-yyz1-1.xx.fbcdn.net/v/t1.6435-9/168815219_3554100158033694_1184578166780882883_n.jpg?_nc_cat=101&ccb=1-3&_nc_sid=730e14&_nc_ohc=x9nNShmntvYAX_3Jiax&_nc_ht=scontent-yyz1-1.xx&oh=9afbf3d59b0a93f4a98012e782998258&oe=609115B1

  pic https://scontent-yyz1-1.xx.fbcdn.net/v/t1.6435-9/168733161_3554100338033676_142947332576686811_n.jpg?_nc_cat=103&ccb=1-3&_nc_sid=730e14&_nc_ohc=WZ68kcC5WgQAX-iEjsv&_nc_ht=scontent-yyz1-1.xx&oh=fb7dda89056f50031af95fcc7da1e091&oe=6090068D

  pic https://scontent-yyz1-1.xx.fbcdn.net/v/t1.6435-9/168832360_3554100521366991_9189758167896449437_n.jpg?_nc_cat=103&ccb=1-3&_nc_sid=730e14&_nc_ohc=IxgaZaYZYmIAX9wx5W3&_nc_ht=scontent-yyz1-1.xx&oh=40bbd9c621d09b273c51895e9b9f0668&oe=608FD467

  pic https://scontent-yyz1-1.xx.fbcdn.net/v/t1.6435-9/168726813_3554100664700310_4512940103431216901_n.jpg?_nc_cat=105&ccb=1-3&_nc_sid=730e14&_nc_ohc=AwSyk8MenwsAX9owwUN&_nc_ht=scontent-yyz1-1.xx&oh=96d40be1e8896d7113d12fc0abd3f51a&oe=609124F2

  pic https://scontent-yyz1-1.xx.fbcdn.net/v/t1.6435-9/168306590_3554100878033622_4459065584681853036_n.jpg?_nc_cat=104&ccb=1-3&_nc_sid=730e14&_nc_ohc=3Xf9OTRaCzYAX_Vx5U8&_nc_ht=scontent-yyz1-1.xx&oh=0e60458c1e7fd5c1b17363390446a6c0&oe=6090D813

  pic https://scontent-yyz1-1.xx.fbcdn.net/v/t1.6435-9/168685520_3554101021366941_210683266840770991_n.jpg?_nc_cat=104&ccb=1-3&_nc_sid=730e14&_nc_ohc=pLGfgu5pL_YAX9hqU2m&_nc_ht=scontent-yyz1-1.xx&oh=204a1b573e0cef3f4c723b4e14ad3316&oe=609270A2

  pic https://scontent-yyz1-1.xx.fbcdn.net/v/t1.6435-9/168956922_3554101261366917_5428300647800379011_n.jpg?_nc_cat=106&ccb=1-3&_nc_sid=730e14&_nc_ohc=8fjgmEMni6kAX9GzOxb&_nc_ht=scontent-yyz1-1.xx&oh=2a2b55b72b353c489b435732dc8d731b&oe=6092E5DC
\fi


Тєлєга про братерство, сестринство та наочну агітацію в СРСР. З картинками.  

Трапився мені на очі вельми показовий зразок совєтської малої пластики. Така собі порцелянова статуетка: симпатична україночка у хрестоматійному віночку довірливо поклала голову на плече ставної росіянки, що у протекторальному жесті ледь схилила голову, увінчану не менш хрестоматійним кокошником (фото 1).
От що не кажи, а совєти вміли не лише в репресії та голодомори. Іще краще вони вміли в агітацію і пропаганду. З одного боку ляскав цілком реальний батіг, а з іншого – дурманили голови солодкі пахощі примарних пряників. Зокрема таких, як «дружба народів».
При цьому стосунки російського та українського народів возносилися на ще вищий щабель – тут уже йшлося не просто про «дружбу», а про «братерство». Ще б пак! Аби стерти з суспільної пам’яті те, що Росія століттями коїла з українством, проголошення дружби було замало, тут лише братерство годилося. Справді: напакостиш другу – і він вже ніякий не друг, навпаки – ворог. А брата хоч закатуй на смерть – все одно він твій брат. Хоч і мертвий. До того ж, як не крути, а дружба передбачає рівність. У братерстві ж завжди є старший і є молодший.
Меседж про спорідненість і старшого-молодшого втілювався скрізь. Зокрема – у візуальних образах, сила яких майже нездоланна. Проминаючи розум і логіку, вони впливають безпосередньо на емоції і вкорінюються у підсвідомість. І, як переконуємося вже майже протягом 30 років, видерти їх звідси дуже непросто.
Порцелянова парочка, що підштовхнула мене до написання цієї тєлєги, – аж ніяк не випадковість, це потужний тренд. Подивімося, приміром, на монумент на честь так званого возз’єднання України та Росії, встановлений у Переяславі (фото 2). 
Тут промовляє кожна деталь. І те, що Росія йде на півкроку попереду. І те, яким хазяйським і вказівним порухом руки вона запрошує Україну на спільну путь. І те, як високо Росія тримає голову, Україна ж її дещо, маже непомітно, але схилила. 
На словах це все виглядає надто нарочито і нав’язливо, а в скульптурі втілено надзвичайно майстерно і тонко, на межі усвідомлення глядача. Шедевральний витвір совєтського мистецтва пропаганди!
Якщо ще уважніше вдивитися в обличчя і постави того монументу, не важко помітити, що у Росії плечі рівніші і більш розгорнені, а в України більш покаті, проте в неї акцентована талія. Лице Росії має правильні, чіткі та тверді риси, а риси України більш м’які. Крім того, в неї більші очі, що притаманно жіночим та дитячим образам.  Загалом, постать Росії має більше маскулінних рис, а постать України – фемінних. 
Аналогічні тенденції споглядаємо і на обкладинці журналу «Мистецтво» за 1969 рік (фото 3). До слова, 69-й – це вже час після відлиги, час закручування гайок і посилення утисків українства. 
Отже, на цій обкладинці вища на зріст, поставніша і монументальніша Росія міцно тримає за руку розвернену до неї Україну, що лагідно підносить своїй – без сумніву старшій – товаришці ці сестрі квітку. При цьому в України нахилена голова, жіночно відставлена нога з привабливо тонкою щиколоткою, є натяк на талію та окреслене декольте…
Звісно, тут не йдеться ні про яку «нетрадиційщину», на думки про що ми стали надто швидкими. Просто надання образам маскулінних або фемінних рис є найефективнішим способом примусити ті образи транслювати силу і надійність або ж слабкість, яка потребує захисту.
Ні, мені не здалося. І ні - це не випадковість. Я не знайшла жодного зображення, де б Україну уособлювала чоловіча фігура, а Росію – жіноча. Це можуть бути або дві жінки (як у вже розглянутих прикладах), або два чоловіка (про що поговоримо просто зараз), або чоловік-росіянин і жінка-українка (про це згодом).
То що відбувається, коли ми маємо двох «братів»? Візьмемо для прикладу плакат на честь 300-річчя «возз’єднання», який зображує персонажів тих давніх часів (фото 4).
На першому плані тут ніхто інший як бородатий росіянин у червоному однострої царських стрільців. Саме він тримає перед собою щит, оповитий стрічкою з написом «навеки - вместе». Козак-українець з характерними вусами розташований на другому плані – він стоїть ніби за плечем стрільця і притримує той щит.
Обидва чоловіка справні та кремезні, з суворими обличчями. Однак не важко помітити, що у козака різко вигнуті брови, а обличчя наче в запалі подалося вперед: тут ми безпомильно впізнаємо емоцію. Стрілець має більш стриманий вигляд. Чи варто нагадувати стереотипи про те, якому ґендеру притаманна емоційність, а якому – стриманість? 
А ось плакат з ладними, бездоганно мужніми совєтськими хлопцями (фото 5). От тільки один з них трохи вищий, до того ж, він поданий в такому ракурсі, що його плечі здаються ширшими. А другий, той що нижчий, має більш екзальтовані, емоційні жести. Ці двоє – спортсмени. На перший  погляд, вони фінішують одночасно. Але ж усі знають, що в спорті так не буває. І дійсно, фінішна стрічка, якої начебто торкаються корпусами обидва спортсмени, направду проходить під ліктем у того, що в майці з написом «РСФСР»…
Можливо, комусь здається, що я надаю зайвого значенням деталям і взагалі прискіпуюсь. Втім, моя увага до деталей і поруч не стояла з увагою членів совєтських худрад. Тім більше, з увагою тих рад, що ухвалювали і затверджували не просто мистецький твір, а продукцію прямого агітаційного призначення, що мала розійтись багатомільйонними накладами. 
Втім, повернемося до цієї продукції. Є ще один страшенно цікавий різновид зображень, на яких взаємодіють не просто персоніфіковані Росія та Україна, а СОВЄТИ та Україна.
Що ми бачимо на одному з плакатів часів Другої світової (фото 6)? Худий українець у латаній-перелатаній вишиванці поривчасто обіймає і палко цілує солдата при зброї та амуніції.  По всьому видно: цей солдат - захисник і спаситель українця, який сам захиститися нездатний. Майже аналогічний сюжет бачимо ще на одному плакаті (фото 7). Тільки тут зморена, вдягнена в лахміття сім’я, що більше скидається не на батька з матір’ю та сином, а на немічних діда з бабою та онуком, дякує совєтському  солдату. Знову ж таки – самі українці безсилі та беззахисні. 
На обох зображеннях підкреслена неспроможність українців до протистояння загарбникам. Втім, тотожність совєтського і російського не увиразнено. На відміну від наступного витвору пропагандистів. 
О, це цікавезний плакат (фото 8)! Тут до мужнього совєтського солдата-визволителя з багнетом і червоню зіркою на касці знесилено горнеться жінка-Україна, на змученому обличчі якої проступає усмішка надії. Здавалося б – звичайний собі штамп. Але особливої змістовності зображенню додає те, що ці обійми відбуваються не деінде, а на тлі пам’ятника Богдану Хмельницькому. Тому самому, що уклав Переяславську угоду з Московським царством, яка трактувалася і пропагувалася совєтами виключно як «возз’єднання України з Росією». Все зрозуміло? Не возз’єдналася б свого часу Україна з Росією, то хто б тепер її, нещасну і слабосильну, визволив і захистив? Тільки російський, він же совєтський солдат! Ну, хіба ж не геніально?
Іще один приклад того, як художники вирішували пропагандистські задачі, маємо на фото 9. Це картина про 1939 рік, написана 40 років потому - до круглої дати. Тут митець нам розповідає про ту радість, яку нібито відчували українці на окупованих тієї осені совєтами західноукраїнських землях. Жінки, діти та обірвані босі старі захоплено слухають хвацького русявого гармоніста у совєтському однострої і сяючих чоботях. В руках самодіяльного музиканта недарма саме цей інструмент – винайдена в Німеччині гармоніка у 19 столітті розповсюдилася в Росії і фактично стала там національним інструментом. Тобто, нам прозоро натякають, що сірі та убогі українці радіють приходу саме російського солдата, який просто випромінює життєву силу і творче начало.
Перш ніж звернути увагу на те, як неоковирно і бездарно сучасні російські пропагандони намагаються наслідувати совєтські прийоми пропагування російсько-української спорідненості, переглянемо іще кілька доперестроєчних зображень. Вони менш вигадливі, але все одно показові. 
1960-ті. Фото 10. На новорічній листівці, створеній для українського споживача, серед модно вбраної молоді бачимо уособлення Росії та України. Першу втілює чоловік, другу – жінка. Не навпаки. 
Фото 11. Здавалося б, просто малюночок для дітей. Тут є і хлопчики, і дівчатка. Як завжди, Україна жіночої статі. Росія ж представлена хлопчиком. І цей хлопчик у кумачовій косоворотці не лише представляє росіян, і не просто так стоїть у центрі та привертає до себе увагу гордовитою позою і яскравістю одягу. Він тут єдиний, хто тримає в руках вищий символ совєтських досягнень – космічний корабель з червоною зіркою. А решта підтанцьовують навкруги. Ну, і що тут неясно?
1970-ті. Тут, на фото 12, бачимо представників усіх союзних республік. Україна – традиційно в жіночому образі. Загалом, цей плакат не вартував би уваги, якби не чоловік по центру, що єдиний тут вбраний не в національний стрій, а в цілком сучасний костюм з краваткою. Методом виключення можна виснувати, що саме він – увесь такий сучасний і прогресивний –  уособлює тут РРФСР. 
Але не забували в застійний період і образи «сестер». На фото 13 – надзвичайно розповсюджений у 70-ті та на початку 80-х сюжет, де україночка лине до росіяночки. Цю композицію з незначними варіаціями друкували в дитячих журналах та у брошурках-посібниках для творчості. Пам’ятаю, як мій дядько, що захоплювався в той час чеканкою, майстерно виконав за мотивами цього зображення роботу на металі з кольоровими емалями. Народу, як то кажуть, заходило… 
Ну що, насолодилися? А тепер – до сучасних російських маразмів. 
Почну не з найпотворнішого, а з наймаразматичнішого. На фото 14 – картинка, що мала б символізувати дружбу народів РФ, принаймні нею постійно ілюструють повідомлення про заходи на цю тему. З великим прапором – малий росіянин. Решта дітей, представляють не різноманітні корінні етноси РФ, як того можна було б очікувати, а народи незалежних нині держав, що колись входили до складу СРСР. Вони розмахують букетами та маленькими триколорами, зокрема, це робить і дівчинка-україночка.
Ну а далі – на фото 15-19 представлені різні за співвідношенням потворності, лицемірства, підступності та пошлості виплоди нинішніх шовіністичних імперців. Знову – усі знайомі сюжети:   сестринство, братерство, та припадання нещасної жінки до сильного чоловіка. Як кажуть, без коментарів. Зазначу лише, що совєтські пропагандисти були куди більш професійними, та, якщо можна так висловитися, певною мірою навіть поважали свою аудиторію, бо не опускалися аж до такого жахливого рівня. 
І насамкінець, на фото 20, ми бачимо вже цілком професійну роботу, що містить начебто безневинний, а насправді вкрай небезпечний меседж. Він – про братерство. І це найпідступніша з пропагандистських тез. Бо братерство – це те, чого не позбутися, хоч гопки скачи. Саме тому на ньому і робиться акцент завжди і скрізь, коли РФ обгрунтовє свої зазіхання на Украину. І варто лише піддатися навіюванню і хоч на мить повірити казочкам про «сімейність», і все – ти в пастці. 
Ті роботи, де тим чи іншим чином експлуатується тема братерства, покликані промити мізки і прищепити навіть не думку, а підсвідоме відчуття того, що ми зараз маємо справу не з кровопролитної фазою довготривалої війни Росії, спрямованої  на знищення українства, а лише з помилками і сварками братів, якім буквально НА РОДУ написано примиритися і бути разом.
Впливу візуальних образів протистояти непросто. Однак – можливо. Звісно, якщо аналізувати кожне зображення у навколишньому середовищі, то стріха поїде. Але коли при спогляданні будь-якого візуального об’єкту відчуваєш, що в тобі щось заворушилося і відгукнулося, варто поміркувати – кому і навіщо потрібно в тобі ту емоцію викликати і чи в злагоді ця новонароджена емоція з фактами і твоїм розумом.

\ifcmt
  pic https://scontent-yyz1-1.xx.fbcdn.net/v/t1.6435-9/168668378_3554101394700237_2795169413022940864_n.jpg?_nc_cat=108&ccb=1-3&_nc_sid=730e14&_nc_ohc=5eoWmjJ1GM4AX-3ky0c&_nc_ht=scontent-yyz1-1.xx&oh=2310ddb38600b29f3fdc205842b6db58&oe=6093503F
\fi

Перед нами - типовий зразок роботи сучасних імперсько-рашистських
пропагандистів. Такими солоденькими картиночками в РФ прикривають агресію проти
України - загарбання Криму та окупацію Донбасу - війну, в який вже восьмий рік
гинуть наші захисники. Це здається безглуздям - неможливо ж у такий спосіб
прикрити ці злочини. Але, на жаль, історичний досвід доводить, що можливо.
Совєтам дуже непогано вдалося прикрити репресії проти українства і голодомори
брехливою пропагандою про дружбу і братерство. Звісно, не самі візуальні образи
спрацювали, але вони відіграли у тому значну роль: бо сприймаються миттєво, не
потребуючи від реципієнта ані часу, ані найменших зусиль. Газети і книжки треба
читати, кіно - дивитися, радіо - слухати. А візуальний образ впадає в око наче
сам собою і "розбачити" його вже неможна.

\ifcmt
  pic https://scontent-yyz1-1.xx.fbcdn.net/v/t1.6435-9/168372544_3554102491366794_5477374824289571650_n.jpg?_nc_cat=111&ccb=1-3&_nc_sid=730e14&_nc_ohc=dec8IZucdBsAX-qd4kf&_nc_oc=AQkWQmkBlguf7Ig2cn38CmqVZ0fO8XXyVCQ5eY9jU-PfyAvt1GREDqE4qn0GgWMpBEU&_nc_ht=scontent-yyz1-1.xx&oh=700ec36b02df5d193f170e8fd4412086&oe=60926C4E

  pic https://scontent-yyz1-1.xx.fbcdn.net/v/t1.6435-9/168529872_3554102591366784_5175378621040784415_n.jpg?_nc_cat=100&ccb=1-3&_nc_sid=730e14&_nc_ohc=iy_yqD8V4aUAX_ZiH8w&_nc_ht=scontent-yyz1-1.xx&oh=9d96ec4bf6f1f0b4e1c8d68b64bc0a48&oe=60939697

  pic https://scontent-yyz1-1.xx.fbcdn.net/v/t1.6435-9/168437399_3554103041366739_1912754158825625492_n.jpg?_nc_cat=100&ccb=1-3&_nc_sid=730e14&_nc_ohc=1GE3HGu_SxwAX_47n8x&_nc_ht=scontent-yyz1-1.xx&oh=a2ae358d079dd8ecd801a7fbaa424ce1&oe=60907B8B

  pic https://scontent-yyz1-1.xx.fbcdn.net/v/t1.6435-9/169070980_3554103161366727_1790997779457056198_n.jpg?_nc_cat=107&ccb=1-3&_nc_sid=730e14&_nc_ohc=9bSWkOItyHEAX_YYpgT&_nc_ht=scontent-yyz1-1.xx&oh=477089ed8df519deef11d919fcfec086&oe=6091312F
\fi
