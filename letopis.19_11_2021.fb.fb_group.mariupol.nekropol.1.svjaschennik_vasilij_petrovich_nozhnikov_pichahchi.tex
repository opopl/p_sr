%%beginhead 
 
%%file 19_11_2021.fb.fb_group.mariupol.nekropol.1.svjaschennik_vasilij_petrovich_nozhnikov_pichahchi
%%parent 19_11_2021
 
%%url https://www.facebook.com/groups/278185963354519/posts/640603757112736
 
%%author_id fb_group.mariupol.nekropol,marusov_andrij.mariupol
%%date 19_11_2021
 
%%tags 
%%title Священник Василий Петрович Ножников (Пичахчи) (1849-1908)
 
%%endhead 

\subsection{Священник Василий Петрович Ножников (Пичахчи) (1849-1908)}
\label{sec:19_11_2021.fb.fb_group.mariupol.nekropol.1.svjaschennik_vasilij_petrovich_nozhnikov_pichahchi}
 
\Purl{https://www.facebook.com/groups/278185963354519/posts/640603757112736}
\ifcmt
 author_begin
   author_id fb_group.mariupol.nekropol,marusov_andrij.mariupol
 author_end
\fi

**Священник Василий Петрович Ножников **(Пичахчи) (1849-1908)

... а также Марина Цветаева и Архип Куинджи (дополнено).

\#люди\_мариупольского\_некрополя

На прошлых выходных мы освободили от земли и мусора часть гранитного памятника
возле усыпальницы Пиличева-Оксюзовой, перевернули и – прочитали имя...

Его прадед Юрий Параскевин Пичахчи (1751-1826) жил в севастопольском Инкермане
и в 1780 году стал одним из трех тысяч греков-первопоселенцев Мариуполя.

Его дед Параскева Юрьев родился в 1778 году, во время перехода крымских греков
в Приазовье. Переход был тяжелым и мучительным, сотни, если не тысячи людей
умерли. Но ему повезло – он выжил.

Как это часто бывает у приазовских греков, фамилия \enquote{Пичахчи} – говорящая. Она
означает – изготовитель ножей, точильщик. Поэтому, когда во второй половине XIX
века некоторые представители рода решили сменить греческую фамилию на русскую,
они назвались Ножниковыми.

Кроме того, имя \enquote{Параскева} они изменили на - Борис.

Сам \textbf{Василий Петрович Ножников} родился в 1849 году в семье мариупольского
мещанина Петра Паращевина \textbackslash Борисовича Пичахчи (1819-1898). Мы не знаем,
занимался ли его отец родовым занятием – изготовлением ножей. Но мы знаем, что
в 1891 году он именуется в метрике Марии Магдалиновской церкви купцом.

Петру Борисовичу повезло с детьми – они стали врачевателями душ и тел
человеческих - священник, врачи, учителя..!

Самый старший сын, Василий, стал иереем, священником. Он служил в соборной
Харлампиевской церкви, преподавал в Мариупольском духовном училище и был
настоятелем Георгиевской церкви в Сартане. В 1881 году он женился на девушке из
Большой Каракубы - Анфисе Ставриевне Пономаревой-Левтеренко.

У них был единственный ребенок – дочь Анна, которую отец сам крестил в
Сартанской церкви в ноябре 1891 года. Восприемниками стали коллеги по училищу:
учитель, надворный советник Антоний Семенович Федоров и жена учителя того же
училища коллежского секретаря София Георгиевна Иванова.

Его **брат Борис** (1851-1911) сначала пошел по стопам старшего и закончил
Екатеринославскую духовную семинарию. Но потом отказался от карьеры священника
и за собственные средства закончил медицинский факультет.

Борис связал свою судьбу с родиной прадеда – он стал известным врачом в
крымской Ялте. Он возглавлял правление Ялтинского медицинского общества. В 1910
году он навещал в Ялте смертельно больного Архипа Куинджи. Он лечил от
туберкулеза мать гениальной русской поэтессы Марины Цветаевой. Его могила
сохранилась в Аутском некрополе Ялты.

Дом Бориса Ножникова тоже сохранился и является памятником архитектуры местного
значения (см. фото).

Врачебная карьера его брата **Ильи Петровича Ножникова** (р. 1855) тоже
сложилась успешно. Звание \enquote{врача} он получил в 1882 году и вернулся в родной
город. В 1888 году женился на Пелагее Васильевне Волошиновой из купеческой
семьи. Через десять лет он уже городской врач, коллежский советник.

По данным краеведа Сергея Бурова, Илья Ножников жил и принимал пациентов в доме
на ул. Греческая, 32 (где был \enquote{Трактир на Греческой}).

Наконец, еще один брат – **Абрам Петрович Ножников** (род. 1864 г.) – тоже стал
врачом! Он окончил Мариупольскую Александровскую гимназию и в 1885 поступил на
медицинский факультет в Харькове, по данным сайта azovgreeks. В 1905 году – он
работает ассистентом в университетской клинике, а в 1911 году – полноценный
врач...

У братьев были две сестры – Надежда (род. 1862) и Вера\par\noindent (род. 1862).

**Вера Петровна Ножникова** закончила Харьковскую Мариинскую женскую гимназию,
а потом преподавала географию в Мариупольской женской гимназии! В 1917 году она
стала начальницей Мариинки, после Александры Генглез, которая ею руководила с
момента основания в 1876 году...

... Василий Петрович похоронен в шаге от отца. Он пережил его всего на десять
лет.

К счастью, памятник Петра Борисовича Ножникова не был разрушен вандалами. Вы
его легко найдете - его венчает открытая книга со словами \enquote{Упокой душу раба
твоего Петра...}

**О дополнении: 

**огромное спасибо Ольге Бабаевой (Helga Buzlami), краеведу и
эксперту по генеалогии, – именно она предоставила информацию о Борисе и Абраме
Ножниковых, Куинджи и Цветаевой, Вере Ножниковой...

\#ножников\_василий
