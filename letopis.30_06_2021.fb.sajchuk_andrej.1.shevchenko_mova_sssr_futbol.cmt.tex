% vim: keymap=russian-jcukenwin
%%beginhead 
 
%%file 30_06_2021.fb.sajchuk_andrej.1.shevchenko_mova_sssr_futbol.cmt
%%parent 30_06_2021.fb.sajchuk_andrej.1.shevchenko_mova_sssr_futbol
 
%%url 
 
%%author 
%%author_id 
%%author_url 
 
%%tags 
%%title 
 
%%endhead 
\subsubsection{Коментарі}

\begin{itemize}
\iusr{Leonid Kalchenko}
Він раніше десь обіцяв виправитися і виступати українською.

\iusr{Анна Діденко}
Не знала, що він родом з села під Яготином. Феномен дацького #к@ц@пізму, не інакше.

\iusr{Lesya Vakulyuk}
Від авторки ідеї поста - невеличке уточнення: він з Яготинського району) Це Київщина, яка по селах і містечках українською розмовляє з домішками суржика.

\iusr{Юрій Голубінка}
Їжте, не обляпайтесь\par
{\em У Росії вважають, що Україна обіграла Швецію завдяки російській мові}\par
\url{https://shlyahta.com.ua/u-rosii-vvazhaiut-shcho-ukraina-obihrala-shvetsiiu-zavdiaky-rosiiskii-movi}

\iusr{Alina Bodnar}
\textbf{Юрій Голубінка} чому тоді вилетіла росія?)))

\iusr{Joana Madzestes}

Тут нема ніякої України, тут довбана малорось. Я сюди повернулась на два місяці
і засрягла на 17 міс. Люди, скажіть як мені тут вижити? В цьому тоталітарному
неосовку? Я претерпіваю агресію, напади та дискримінацію. Я не можу отримати
адекватну медичну допомогу, я не можу працювати. Я виходячи на вулицю не знаю
чи вернусь жовою додому. Це неосовок!

\iusr{Joana Madzestes}

Пане Андрію, вам смішно з мого коменту? А розказати вам як мені ніж до горлянки
приставляли, чи як ледь не вкокошили даунбасята бо мою київську адресу росіяни
оприлюднили. Не до сміху.

\iusr{Евгений Дьяконов}
Отцепитесь от Андрея.

\iusr{Maksym Kostetskyi}
А відколи це його діяльність оплачується з державного бюджету?)

\iusr{Василь Самохвалов}
Не з державного бюджету, якшошо

\begin{itemize}

\iusr{Andrii Saichuk}
тут справді цікава річ. я чесно не знав цей нюанс фінансування УАФ (не сильно
цікавлюся футболом - швидше як культурним явищем). А хто в такому випадку
платить 60 тис євро в місяць пану тренеру? І чи є він тоді тренером
\enquote{національної збірної}? В чому її \enquote{національність}?

\iusr{Василь Самохвалов}
\textbf{Andrii Saichuk} Ти будеш здивований, але у всьому світі так. Навіть
більше - міжнародні федерації суворо стежать за тим, щоб держава не могла
втручатися у діяльність національних федерацій. Стосовно зарплат - футбол
прибуткова річ, загалом. Виступ збірної на Євро - зокрема.

\iusr{Andrii Saichuk}
\textbf{Vasyl Samokhvalov} виходить це \enquote{світла} сторона українського олігархату
- наявність сильних клубів і потужної національної збірної ! \Laughey[1.0] Але то все не
знімає питання в чому тоді \enquote{національність} збірної, якщо і її тренер і гравець
проводять брифінг мовою - офіційно - держави-агресора 🤷

\iusr{Василь Самохвалов}
\textbf{Andrii Saichuk} питання до уєфа)

\iusr{Andrii Saichuk}
\textbf{Vasyl Samokhvalov} або до українських вболівальників, яким я такими постами, схоже, псую настрій \Smiley[1.0][yellow]

\iusr{Василь Самохвалов}
\textbf{Andrii Saichuk} ну, футболом ти явно не цікавишся)

\iusr{Andrii Saichuk}
\textbf{Vasyl Samokhvalov} футбол мені цікавий - просто в іншому вимірі.

\iusr{Василь Самохвалов}
\textbf{Andrii Saichuk} почни з хокею, там цікавіше

\iusr{Andrii Saichuk}
\textbf{Vasyl Samokhvalov} не в цій країні. Тут має значення футбол, бокс, трохи біатлон

\iusr{Василь Самохвалов}
просто склад команд нашої хокейної ліги за паспортами тебе здивує ще більше, ніж мова шевченка)

\iusr{Василь Самохвалов}
втім, і тренери з біатлону теж)

\iusr{Andrii Saichuk}
\textbf{Vasyl Samokhvalov} та хай буде хоч з Суринаму, аби не москаль)

\iusr{Василь Самохвалов}
\textbf{Andrii Saichuk} ти здивуєшся)

\iusr{Andrii Saichuk}
\textbf{Vasyl Samokhvalov} слухай, я і так знаю, шо всі найліпші українці грали за збірну Канади! )\par
\url{https://www.youtube.com/watch?v=TflsumHdevU}

\iusr{Василь Самохвалов}
\textbf{Andrii Saichuk} 

\url{https://hcdonbass.com/komanda/donbass/}

\iusr{Andrii Saichuk}
так то ж легіонери? вони за збірну не гратимуть.

\iusr{Василь Самохвалов}
\textbf{Andrii Saichuk} вони грають у чемпіонаті україни. а дехто бере й
українське громадянство, для збірної

\end{itemize}

\iusr{Olena Nedoshytko}
Нічого дивного. Андрій Шевченко - малорос і аж ніяк не українець.

\iusr{Ростислав Гудь}
\textbf{Olena Nedoshytko}, оппа, як багато гнівних смайликів))

\iusr{Vladislav Maistrouk}

А причём тут государственный бюджет к зарплате Шевченко, которому платит УАФ,
которая в свою очередь общественная организация не имеющая к государству
никакого отношения, особенно финансового?

\begin{itemize}
\iusr{Andrii Saichuk}

власне намагаюся тоді зрозуміти яке має відношення ця організація до України
(як держави і нації). Чи може презентувати її на національному рівні просто
тому, що купка впливових і багатих людей скинулися і платять зокрема з/п
шевченку? якщо при цьому їм нецікава українська мова і національне
законодавство 🤷

\iusr{Oleksii Bratushchak}
  · 
какаяразніца!)

\iusr{Andrii Saichuk}
Oleksii Bratushchak вот імєнно!)

\iusr{Vladislav Maistrouk}
\textbf{Andrii Saichuk} общение частных лиц никак не регламентируется законом. Что касается языкового вопроса, то это вопрос времени. Перейдут на украинский когда это будет для них естественно. А так это полемика ради полемики

\iusr{Olena Nedoshytko}
Багато питань назбирюється.
А можливо нема ніякої української збірної. Це лише форма їхня, а зміст зовсім інший.
Для себе вирішила, не вболіватиму за них. Чекаю на появлення нового футболу, справжнього, гідного.
😉

\iusr{Ростислав Гудь}
\textbf{Vladislav Maistrouk}, інтєрєсно, пачєму такая ріторіка васнавном пішеца ілі праізносіца на масковском язикє?

\iusr{Vladislav Maistrouk}
\textbf{Ростислав Гудь} патамушта

\iusr{Ростислав Гудь}
\textbf{Vladislav Maistrouk}, ну да, ну да... какаяразніца на каком язикє, проста главнає штоб етат язик бил русскім...

\iusr{Vladislav Maistrouk}
\textbf{Ростислав Гудь} нет, конечно есть разница. Но зависит от контекста.
Когда я в магазине или ресторане, не говоря про госучреждения, я предпочитаю
украинский. От футболистов, зная исторический контекст их формирования, у меня
нет претензий по языку. Да, было бы круто если бы они говорили на украинском,
но как есть. Не за их умение говорить мы за ними следим, и главное что в
головах там нет колорадских жуков. А тараканы то такое...

на войне же никто не делил своих на русскоязычных и україномовних, потому что в
том контексте, как и во многих других, это не повод для разногласий

\iusr{Ростислав Гудь}
\textbf{Vladislav Maistrouk}, Ви там були, на тій війні? А Шевченко Андрюха
був? Може Зінченко? Шо, теж нє? Тоді не маніпулюйте ок?

\iusr{Vladislav Maistrouk}
\textbf{Ростислав Гудь} а вы можете довести обратное? Я общался со многими
ветеранами, волонтёрами который русскоязычные. Это для них не повод кича, но и
не повод стыдится.

\iusr{Ростислав Гудь}
\textbf{Vladislav Maistrouk}, я Вам ніц доводити не збираюсь. Як мінімум тому,
що не зобов‘язаний. Тим паче, ця ситуація ніяким боком не стосується ні війни,
ані військових чи волонтерів. Тому ще раз - залиште при собі свої маніпуляції.


\end{itemize}

\iusr{Іванна Кобєлєва}
Пропонуємо рішення\par 
\url{https://www.facebook.com/100001976791486/posts/4075913742484479/?d=n}

\iusr{Olga Klymko}

Я думаю, він не мав сили говорити взагалі...буває такий стан. Я знаю тих, хто
важко воював,зараз важко живе і досі говорить російською. І я дуже поважаю цих
людей, хто попри все - серцем обрав Україну. І Шевченко тренує не чиюсь збірну,
а нашу.

\begin{itemize}

\iusr{Lesya Vakulyuk}
\textbf{Olga Klymko} він же життям не ризикує, як ті, хто воюють

\iusr{Ірина Рубанова}
\textbf{Olga Klymko} та ні, говорити він сили все ж мав, але говорити
російською. А от українською говорити йому несила все життя. Італійською,
англійською — навчився, бо ж грав за італійський клуб, а от українською — ніяк,
хоч очолює збірну України. Ваші виправдання недолугі, як і тренер збірної
України, етнічний українець, що народився, виріс і навчився грати у футбол в
Україні, але не говорить українською бодай публічно.

\iusr{Ірина Рубанова}

а що до \enquote{тренує не чиюсь збірну, а нашу}, не факт. Росіяни вважають, що раз
одна мова, то одиннарод, а відтак і перемога їхня також. І об'єктивно мають
підстави так вважати, бо культурна і мовна перемога таки за ними, якщо
українська збірна дає інтерв'ю російською.

\ifcmt
  pic https://scontent-mia3-2.xx.fbcdn.net/v/t1.6435-9/209267916_4066428210099695_4499684572777142017_n.jpg?_nc_cat=102&ccb=1-3&_nc_sid=dbeb18&_nc_ohc=E_43G413RJwAX8ll0ko&_nc_ht=scontent-mia3-2.xx&oh=dcfe3c8db5491085e99a6eba2cf89bac&oe=60E52A82
  width 0.4
\fi

\iusr{Olga Klymko}
я не читаю вкиди політтехнологів росіян, і вам не раджу. Шевченко -
молодець,він прославив Україну грою !!! А зарпопонувати йому допомогу у
вивченні - це ОК !

\iusr{Ірина Рубанова}
\textbf{Olga Klymko} які вкиди. Я даю скріншот під моїм коментарем. Про те, що
йому варто би говорити хоч іноді українською говорили, щойно він став головним
тренером, допомоги валом - від Безкоштовних курсів української до цілком
доступних репетиторів на будь-який смак.  Вам не набридло писати бздури?
 · 3 д.
\emph{Andrii Saichuk}
\textbf{Olga Klymko} людину, якій 45 років на носі і вона досі не вивчила
українську мову (при цьому вившивши три іноземні) - мало просто запропонувати
допомогу. Тут потрібний не пряник, а батіг.

\emph{Olga Klymko}
\textbf{Andrii Saichuk} ви були в такому стані виснаження, як Шевченко ? у мене був такий схожий стан один раз, я досягла одного чудового результату, мені дали слово, а я не можу говорити, як каменюки у голові, а не слова. мені дуже жаль, що ви так сприйняли ц... Ещё
 · 3 д.
\emph{Olga Klymko}
\textbf{Iryna Rubanova} ці слова від росіян, то і є вкиди російських політтехнологів, невже ви цьогоне розумієте ? чи той пан з рашки щиро так думає на вашу оцінку ?
 · 3 д.
\emph{Olga Klymko}
\textbf{Andrii Saichuk} ще трошки відвлліку вашуувагу - на мій погляд саме тут зараз закладається ставлення до мови, але уваги до цього процесу майже ніякої. \par
\url{https://www.facebook.com/sqe.poltava/posts/560642931977572}

\end{itemize}



\end{itemize}

