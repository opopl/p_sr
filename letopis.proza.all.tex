% vim: keymap=russian-jcukenwin
%%beginhead 
 
%%file proza.all
%%parent proza
 
%%url 
 
%%author 
%%author_id 
%%author_url 
 
%%tags 
%%title 
 
%%endhead 
ПРИТЧА ДНЯ

💛 Одна женщина всё время плакала. Её старшая дочь вышла замуж за торговца зонтами, а младшая — за торговца лапшой.
Когда женщина видела, что погода хорошая и день будет солнечным, она плакала и думала: «Ужасно! Погода такая хорошая, у моей дочки в лавке никто не купит зонтик от дождя! Как же быть?»
Если погода была плохая и шёл дождь, она опять плакала, на этот раз из-за младшей дочери: «Если лапша не высохнет на солнце, младшая дочь не продаст ее. Что же делать?»
И так она горевала каждый день при любой погоде: то из-за старшей дочери, то из-за младшей. Однажды ей встретился монах, которому стало жаль женщину и он поинтересовался, почему она так горько плачет. Женщина выложила ему все свои горести, но монах лишь улыбнулся и сказал:
— Ты только измени свой образ мыслей, погоду ты ведь никак не изменишь: когда светит солнце, ты не думай о зонтиках старшей дочери, а думай о лапше младшей: «Светит солнце! У младшей дочки лапша хорошо подсохнет, и торговля будет успешной». Когда идёт дождь, думай о зонтиках старшей дочери: «Вот и дождь пошёл! Зонтики у дочки наверняка продадутся хорошо».
Женщина очень обрадовалась совету, поблагодарила монаха, и стала поступать согласно рекомендации. С той поры она больше не плакала, а все время радовалась за дочерей, из-за чего радости в жизни прибавилось и у нее, и у ее дочерей, которые никак не могли утешить бедную мать.

Мораль: Истина стара как мир - если не можешь изменить обстоятельства, измени свое отношение к ним. Стоит только иначе взглянуть на вещи, и жизнь потечёт в ином направлении.

Никто большей любви не имеет, как тот, который жизнь свою готов отдать для своего ближнего. 
А жизнь может быть долгой и трудной. 

Митрополит Антоний (Сурожский).


ЕХАЛИ МЫ В ПОЕЗДЕ... 
 
...давным-давно, на заре туманной юности. Лет шестнадцати от роду ехали из Ленинграда в Свердловск, с подружкой. А на вокзале мы потеряли кошелечек с деньгами. Какие там деньги, - копейки, но даже за постельное белье заплатить было нечем, так вышло. И вообще - мы потеряли всю сумку, а в сумке - бутерброды и курица, их бабушка Роза дала нам в дорогу. Говоря литературным языком, мы были несколько обескуражены. Да. 
 
Ехать надо было двое с лишним суток. Недолго. И умереть с голоду за это время было невозможно. Даже наоборот - полезно очень не есть двое суток, так ведь? Мы похудеем. Это прекрасно! Так мы решили. Мы даже радовались. И воображали, какими стройными мы выйдем из вагона, вот так! Это потому, что нам было шестнадцать лет, понимаете? Сейчас даже не знаю, что я испытала бы, оставшись в плацкартном вагоне без денег, без телефона, без еды и без постели. 
 
Что-то просить у закусывающих пассажиров нам бы и в голову не пришло. Не те времена и не то воспитание. Мы ехали и худели с каждой минутой. Смотрели в окошко, это очень интересно. И болтали, как могут упоенно болтать девочки-подростки. А потом легли спать. А потом встали и снова стали болтать и смотреть в окошко. Воды попили. Воды сколько хочешь можно пить. 
 
А потом к нам подошел солдат. Лицо у него было со следами ожогов. Он держал в руках большие яблоки, апельсины и булки - булки он в поезде купил, по вагонам ходили тетеньки с корзинками и продавали еду. Хороший такой солдат. Черные волосы и глаза раскосые. А на кисти руки - татуировка. 
 
Он все положил нам на столик - мы на боковых сиденьях ехали. И грубо сказал: мол, жрите, девки. Вы совсем ничего не жрете. Наверное, у вас денег нет, да? Так я дам! У меня есть! 
 
Мы снова были обескуражены, даже не поблагодарили от удивления. А солдат вдруг улыбнулся так широко, ясно, искренне. И сказал: " А я живой домой еду. Живой! Наших всех в Афгане убили. А я живой домой еду! Вот счастье-то выпало!"... 
 
И так он это радостно сказал, что и мы разулыбались. И взяли по яблоку - от них не особо растолстеешь. А от денег отказались, конечно. Мы же не такие. Мы не возьмем! 
 
Солдат ушел на свое место и все глядел на нас, улыбался. И подходил еще несколько раз - чай приносил в подстаканниках. И сахар в бумажных упаковочках. И конфеты. И еще булки. Те-то мы съели, не удержались. 
 
...Он просто остался живой. Это большое счастье - быть живым. От этого улыбаешься и делишься всем, что у тебя есть с другими живыми людьми. Пусть тоже живут! 
 
...А на прощанье он мне подарил платок с люрексом из валютного магазина. Он сам так сказал. Такой щедрый солдатик оказался. Живой. 
 
Про это как-то забываешь. Тревожишься, переживаешь, сетуешь на неудобства. Булок просишь или чаю требуешь. 
 
Быть живым хорошо. Просто жить... 
 
Автор: #АннаКирьянова@

Мы, люди, не можем не любить, не хотеть быть любимыми. Потому что это фундамент нашего существования. Бог есть любовь, и всё, что в этом мире есть, жаждет любви. Бог создал всё по Своему образу и подобию, по Своему образцу, образцу Троичных отношений. Мы были созданы как люди, чтобы участвовать в радости отношений в Боге и с Богом. Чтобы испытывать любовь. Чтобы быть вместе. Поэтому и говорится, что рай – это общение со всеми, а ад – невозможность больше любить.

Иеромонах Пантелеимон (Шушня)

