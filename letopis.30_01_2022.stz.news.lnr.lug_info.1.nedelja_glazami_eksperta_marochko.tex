% vim: keymap=russian-jcukenwin
%%beginhead 
 
%%file 30_01_2022.stz.news.lnr.lug_info.1.nedelja_glazami_eksperta_marochko
%%parent 30_01_2022
 
%%url https://lug-info.com/news/nedel-ya-glazami-eksperta-3
 
%%author_id news.lnr.lug_info
%%date 
 
%%tags donbass,lnr,marochko_andrej.nm_lnr,nm_lnr,rossia,ugroza,ukraina,vojna
%%title НЕДЕЛЯ ГЛАЗАМИ ЭКСПЕРТА: Истеричная возня в Киеве, переговорный дуплет и российский ответ западному оружию
 
%%endhead 
 
\subsection{НЕДЕЛЯ ГЛАЗАМИ ЭКСПЕРТА: Истеричная возня в Киеве, переговорный дуплет и российский ответ западному оружию}
\label{sec:30_01_2022.stz.news.lnr.lug_info.1.nedelja_glazami_eksperta_marochko}
 
\Purl{https://lug-info.com/news/nedel-ya-glazami-eksperta-3}
\ifcmt
 author_begin
   author_id news.lnr.lug_info
 author_end
\fi

\begin{zznagolos}
Своим видением событий прошедшей недели, так или иначе связанных с Луганской
Народной Республикой, с ЛИЦ делится военный эксперт, общественный деятель,
подполковник запаса Народной милиции ЛНР Андрей Марочко.
\end{zznagolos}

\subsubsection{НА ЛИНИИ СОПРИКОСНОВЕНИЯ}

Минувшая неделя, несмотря на Договоренности о дополнительных мерах контроля
действующего режима прекращения огня, для жителей Луганской Народной Республики
прошла неспокойно, обстановка на линии соприкосновения по-прежнему оставалась
стабильно напряженной и имела тенденцию к обострению.

\ii{30_01_2022.stz.news.lnr.lug_info.1.nedelja_glazami_eksperta_marochko.pic.1}

По данным наблюдателей представительства ЛНР в СЦКК, боевики вооруженных
формирований Украины осуществили один обстрел нашей территории, применив
станковый противотанковый гранатомет и крупнокалиберный пулемет

В результате украинской агрессии погиб военнослужащий Народной милиции ЛНР, еще
один был захвачен в плен.

Всего с 00:00 часов 21 июля 2019 года режим всеобъемлющего устойчивого и
бессрочного прекращения огня со стороны вооруженных формирований Украины был
нарушен 1021 раз, из них 687 раз – после вступления в силу допмер; 44 защитника
Республики погибли, 24 получили ранения. Среди мирного населения пять человек
погибли, 38 получили ранения, повреждены 194 объекта гражданской
инфраструктуры.

\subsubsection{ИСТЕРИКА БЕЗ МОЗГОВ}

Нагнетание ситуации вокруг так называемого российского вторжения в начале
недели получило новый виток. Сообщения в западных СМИ приобрели уже
истерический характер, а на фоне публикуемых карт с многочисленными сценариями
вторжения России на Украину посыпались сообщения об эвакуации дипломатов и
членов их семей с Украины. В самой же «нэзалежной» на все это смотрели с
недоумением, растерянностью и выпученными глазами. Пытаясь успокоить своих
граждан, некоторые чиновники даже попытались прокомментировать ситуацию, но
вышло, честно говоря, не очень. Так, например, в офисе преЗЕдента посчитали
эвакуацию обычной практикой и заявили, что не считают это тревожным сигналом.
Украинский МИД привел свои аргументы сказав, что на территории Украины
находятся 129 посольств и консульских учреждений иностранных государств, а
также представительств международных организаций. По состоянию на вечер 24
января только четыре государства (США, Великобритания, Австралия и Германия)
изъявили желание вывезти семьи дипломатов. Такие заявления больше несли
смысловую нагрузку песни Леонида Утёсова «Все хорошо прекрасная маркиза», и к
разряду успокаивающих их ну никак отнести было нельзя. Масло в огонь подлила
новость о проведении экстренного заседания СНБО. В аппарате Совбеза сообщили,
что рассмотрят меры по обеспечению национальной безопасности от внутренних и
внешних угроз, текущего положения дел в экономике, энергетике, противодействии
пандемии коронавирусной болезни, а также ряд других важных тем. По итогам
заседания стало известно, что обсуждали ситуацию на границе и заслушали ряд
докладов: разведки, Минобороны и Генштаба, МВД и СБУ, премьер-министра, главы
Минэнерго, и главы Минздрава. Отдельной темой стала кибератака 14 января.
Секретарь СНБО Данилов на брифинге вновь попытался всех заверить, что ситуация
на границах остается контролируемой. По его мнению, на сегодняшний день нет
поводов для паники, в случае агрессии сработает система оповещения, а ВСУ дадут
решительный отпор, угроз функционирования экономики тоже нет. В такой оптимизм
мало кто поверил, да и, собственно говоря, исходя из тех же итоговых заявлений,
слова министра обороны Алексея Резникова о начале формирования первой очереди
сил территориальной обороны нельзя отнести к положительному сценарию развития
событий.

В общем пытаться найти здравый смысл и логику подводя итоги СНБО, на мой
взгляд, бессмысленно. Есть такое произведение Фрэнка Баума «Удивительный
волшебник из страны Оз», и маленький отрывок из него полностью отображает
картину происходящего:

– А как ты можешь разговаривать, если у тебя нет мозгов? – спросила Дороти.

– Я не знаю. Но многие из тех, у кого нет мозгов, очень любят разговаривать, –
ответило Чучело.

\subsubsection{ДОГОВОРИЛИСЬ ДОГОВАРИВАТЬСЯ}

26 января почти одновременно стартовали первые в этом году переговоры минской
Контактной группы в формате видеоконференции и началась очная встреча
политсоветников лидеров стран нормандского формата в Париже. За переговорным
дуплетом следили представители всех крупных СМИ, поскольку ситуация вокруг
внутриукраинского конфликта стала чуть ли не главной темой во всем мире.
Результаты переговоров ожидаемо сенсаций не принесли и носили больше
ознакомительный характер. 

Касаемо Контактной группы, то украинская сторона предпочла продолжить
придерживаться непродуктивной тактики отказа от сближения позиций и в
конструктивное русло на переговорном треке так и не вернулась. Ни по одному
аспекту договоренностей достичь так и не удалось. Полпред РФ в Контактной
группе по Донбассу Борис Грызлов охарактеризовал действия украинских
представителей как демонстративный нескрываемый саботаж минского переговорного
процесса. В свою очередь представитель Луганской народной республики в
политической подгруппе по урегулированию конфликта в Донбассе Родион Мирошник
отметил, что риторика высказываний украинских представителей становится
кардинально другой, они теперь уходят от лозунгов и всячески пытаются
демонстрировать некую свою договороспособность. Такая имитация, на мой взгляд,
весьма очевидна даже посредникам со стороны ОБСЕ, но, к сожалению, однобокая
позиция не позволяет им говорить об этом вслух.

Переговоры в Париже длились восемь с половиной часов, что уже говорит о том,
насколько они были тяжелыми и сложными. Россию на мероприятии в Елисейском
дворце представляет замглавы администрации президента Дмитрий Козак, Украину —
глава офиса президента Андрей Ермак, Германию — советник канцлера Йенс Плётнер,
Францию — советник президента Эммануэль Бонн. По завершении переговорного
процесса первым к журналистам вышел представитель РФ. Комментируя итоги
встречи, Козак заявил, что пока прогресс в переговорах в нормандском формате
почти нулевой, похвастаться нечем, Россия надеется на более конструктивные
переговоры в Берлине. Его украинский визави Ермак был настроен более
оптимистично. Он посчитал, что само проведение встречи и возобновление
переговорного процесса уже очень позитивный сигнал. Главным достижением
встречи, по мнению главы офиса украинского президента, стала поддержка всеми
участниками режима прекращения огня в Донбассе, \enquote{который должен действовать
безусловно}. Еще одним достижением, по его мнению, стало подписание итогового
коммюнике, ведь это первый документ, который удалось согласовать с декабря 2019
года.

Если ознакомится с содержанием согласованного всеми сторонами итогового
документа, то становится понятно, что он носит декларативный характер и больше
определяет направление дальнейшей работы. В тексте сказано: «Советники
подтверждают, что минские договоренности являются основой работы нормандского
формата, и утвердили, что нужно уменьшить имеющиеся разногласия, чтобы
двигаться вперед». Также стороны выступают за безоговорочное соблюдение режима
прекращения огня и полное соблюдение мер по укреплению режима прекращения огня
от 22 июля 2020 года вне зависимости от разногласий по другим вопросам
выполнения Минских договоренностей. Обсудили важность активизации работы
Контактной группе и ее рабочих подгрупп в целях скорейшего продвижения в
реализации Минских договоренностей, и договорились встретиться снова через две
недели в Берлине. Убрав мишуру, остается лишь пресловутое договорились
договариваться. В общем выстрелить с двустволки не получилось, произошла
двойная осечка, следующая попытка «завалить зверя» будет через две недели.

\subsubsection{ОТВЕТ НА ЗАПАДНЫЕ ПОСТАВКИ ОРУЖИЯ}

Эффект разорвавшейся атомной бомбы вызвало заявление первого заместителя
председателя Совета Федерации и секретаря Генерального совета партии «Единая
Россия» Андрея Турчака. 26 января он объявил: «Единая Россия с огромной
тревогой относится к накачиванию Украины западным летальным оружием. Количество
обстрелов Луганской и Донецкой областей кратно увеличилось. Гибнут мирные люди.
В это время западные кураторы подталкивают украинскую хунту к прямому вторжению
на Донбасс. Вооружают нацбаты, для которых герои Бандера и Гитлер, а свастика
уважаемый знак. Считаю, что в этих условиях России должна оказать Луганской и
Донецкой народным республикам необходимую помощь в виде поставок отдельных
видов вооружений для повышения их обороноспособности и сдерживания явно
готовящейся Киевом военной агрессии. Надо остановить киевский режим».

После его заявления последовала масса комментариев, причем спикеры были отнюдь
не рядовые граждане России, а также обладающие властными полномочиями, и все
они сводились к безоговорочной поддержке данной инициативы. Такой явно
неожиданный сюрприз прокомментировали и главы двух республик Донбасса. Они
прежде всего выразили слова благодарности за поддержку, поскольку если такое
решение будет принято, это существенно усилит наши оборонительные возможности и
позволит эффективнее защищать жителей Луганской и Донецкой Народных Республик.

Хочу присоединится к словам Леонида Пасечника и Дениса Пушилина, поскольку если
здравый смысл у руководства Украины все-таки не проснется и будет принято
решение силового захвата наших республик, то нам придется противостоять
новейшим типам вооружения, которыми сейчас буквально фаршируют Украину так
называемые западные партнеры. Противостоять в таком случае можно будет либо
ценой огромных жертв, либо нивелировать превосходство поставкой НМ современных
типов и видов вооружений, а также вспомогательных средств. Хотелось бы
надеется, что сама возможность поставок, остудит пыл украинских боевиков, но,
исходя из опыта, на такое благоразумие рассчитывать не приходится. Не
прибавляет оптимизма военно-политическая обстановка на Украине и кризис во всех
сферах жизни на ее территории. Это может подтолкнуть Зеленского к развязыванию
кровавой бойни, но, когда боевики будут знать, что последует неминуемый
адекватный ответ, желающих исполнять преступные приказы значительно поубавится,
а это в свою очередь сохранит жизни по обе стороны линии соприкосновения.
