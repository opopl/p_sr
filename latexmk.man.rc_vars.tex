% vim: keymap=russian-jcukenwin
%%beginhead 
 
%%file man.rc_vars
%%parent man.errors
 
%%endhead 

\section{HOW TO SET VARIABLES IN INITIALIZATION FILES}
  
\vspace{0.5cm}
 {\ifDEBUG\small\LaTeX~section: \verb|man.rc_vars| project: \verb|latexmk| rootid: \verb|p_saintrussia| \fi}
\vspace{0.5cm}
  
The important variables that can be configured  are  described  in  the section
"List  of  configuration  variables  usable  in initialization files".  (See
the earlier  section  "Configuration/Initialization  (rc) Files"  for  the
files where the configurations are done.)  Syntax for setting these variables
is of the following forms:

\begin{verbatim}
	$bibtex = 'bibtex %O %B';
\end{verbatim}

for the setting of a string variable,

\begin{verbatim}
	$preview_mode = 1;
\end{verbatim}

for the setting of a numeric variable, and

\begin{verbatim}
	@default_files = ('paper', 'paper1');
\end{verbatim}

for the setting of an array of strings.  It is possible  to  append  an item to
an array variable as follows:

\begin{verbatim}
	push @default_files, 'paper2';
\end{verbatim}

Note  that  simple  "scalar"  variables  have names that begin with a \verb|$|
character and array variables have names that begin with a \verb|@| character.
Each statement ends with a semicolon.

Strings  should  be  enclosed  in single quotes.  (You could use double quotes,
as in many programming languages.  But then the  Perl  programming  language
brings  into  play some special rules for interpolating variables into strings.
People not fluent in Perl will want  to  avoid these complications.)

You  can do much more complicated things, but for this you will need to consult
a manual for the Perl programming language.

