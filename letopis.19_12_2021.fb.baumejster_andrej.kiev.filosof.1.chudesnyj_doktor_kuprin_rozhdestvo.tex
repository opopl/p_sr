% vim: keymap=russian-jcukenwin
%%beginhead 
 
%%file 19_12_2021.fb.baumejster_andrej.kiev.filosof.1.chudesnyj_doktor_kuprin_rozhdestvo
%%parent 19_12_2021
 
%%url https://www.facebook.com/andriibaumeister/posts/4555064001281790
 
%%author_id baumejster_andrej.kiev.filosof
%%date 
 
%%tags 1897,hristianstvo,hristos,kiev,kuprin_aleksandr.pisatel,medicina,pirogov_nikolaj.hirurg,rozhdestvo
%%title "Чудесный доктор" Куприна - одна из моих самых любимых Рождественских историй
 
%%endhead 
 
\subsection{\enquote{Чудесный доктор} Куприна - одна из моих самых любимых Рождественских историй}
\label{sec:19_12_2021.fb.baumejster_andrej.kiev.filosof.1.chudesnyj_doktor_kuprin_rozhdestvo}
 
\Purl{https://www.facebook.com/andriibaumeister/posts/4555064001281790}
\ifcmt
 author_begin
   author_id baumejster_andrej.kiev.filosof
 author_end
\fi

Сегодня четвертое воскресение Адвента! В первую неделю мы читали \enquote{Легенды о
Христе} Сельмы Лагерлеф. На второй - \enquote{Снежную королеву} Ганса Христиана
Андерсена. На третьей неделе Адвента я делился размышлениями над
\enquote{Рождественской песнью в прозе} Чарльза Диккенса. 

Сегодня, после Стокгольма, Копенгагена и Лондона, мы посетим Киев. 

Сегодняшний текст - это рассказ Александра Куприна \enquote{Чудесный доктор}. 

Мое видео-рассуждение об этом рассказе - на моем Patreon-канале:

\url{https://www.patreon.com/andriibaumeister} 

\enquote{Чудесный доктор} Куприна - одна из моих самых любимых Рождественских историй.
Куприну удалось придать реальной истории теологическую и философскую глубину. А
сам \enquote{чудесный доктор}, Николай Иванович Пирогов, воплощает в себе черты
профессора, доктора, православного старца и... чудесного ангела. 

\begin{multicols}{2} % {
\setlength{\parindent}{0pt}

\ii{19_12_2021.fb.baumejster_andrej.kiev.filosof.1.chudesnyj_doktor_kuprin_rozhdestvo.pic.1}
\ii{19_12_2021.fb.baumejster_andrej.kiev.filosof.1.chudesnyj_doktor_kuprin_rozhdestvo.pic.1.cmt}

\ii{19_12_2021.fb.baumejster_andrej.kiev.filosof.1.chudesnyj_doktor_kuprin_rozhdestvo.pic.2}
\ii{19_12_2021.fb.baumejster_andrej.kiev.filosof.1.chudesnyj_doktor_kuprin_rozhdestvo.pic.2.cmt}

\ii{19_12_2021.fb.baumejster_andrej.kiev.filosof.1.chudesnyj_doktor_kuprin_rozhdestvo.pic.4}
\ii{19_12_2021.fb.baumejster_andrej.kiev.filosof.1.chudesnyj_doktor_kuprin_rozhdestvo.pic.4.cmt}

\ii{19_12_2021.fb.baumejster_andrej.kiev.filosof.1.chudesnyj_doktor_kuprin_rozhdestvo.pic.5}
\ii{19_12_2021.fb.baumejster_andrej.kiev.filosof.1.chudesnyj_doktor_kuprin_rozhdestvo.pic.5.cmt}

\ii{19_12_2021.fb.baumejster_andrej.kiev.filosof.1.chudesnyj_doktor_kuprin_rozhdestvo.pic.6}

\end{multicols} % }

Рассказ написан в Киеве, впервые опубликован в газете \enquote{Киевское слово} в 1897
году. А события, описанные в рассказе, произошли, скорее всего, накануне
Рождества в 1860 или 1859 года. Именно в эти годы (с июля 1858 по март 1861)
Пирогов был попечителем Киевского учебного округа. 

\ii{19_12_2021.fb.baumejster_andrej.kiev.filosof.1.chudesnyj_doktor_kuprin_rozhdestvo.pic.3}

Удивительная судьба этого человека. Родился Николай Иванович в Москве (перед
началом Рождественского поста), учился вначале в Московском университете
(отделение медицины), сдавал экзамены в Петербурге, защитил докторскую в
Дерпте, с 1833 по 1835 годы находился в заграничной командировке. Затем -
Париж, Дерпт, Санкт-Петербург, Крымская война, Одесса, Киев, село Вишня (под
Винницей), Гайдельберг, снова село Вишня, театр русско-турецкой войны 1877-1878
годов и, наконец, снова село Вишня. 

С 1879 года Пирогов начинает работать над книгой воспоминаний и размышлений
\enquote{Вопросы жизни. Дневник старого врача} (ссылка на электронный вариант книги - в
первом комментарии под постом). Вот как Николай Иванович описывает мотивы
написания Дневника: \enquote{И вот я, для самого себя и с самим собою, хочу рассмотреть
мою жизнь, подвести итоги моим стремлениям и мировоззрениям (во множественном,
– их было несколько) и разобрать мотивы моих действий}.

А вот запись на Рождество (25.12.1879): \enquote{Рождество Христово. Не писал дневника
несколько дней, но зато на моих утренних прогулках по имению старался привести
в порядок и ясность для себя мои понятия о начале жизни}. \enquote{Главное – сделать
для себя ясным свое мировоззрение. Если я только не слукавлю пред Богом и моей
собственной совестью, излагая мое мировоззрение, то дела нет – буду ли я
материалист или глупец в отношении к другим}.

Более поздние записи: \enquote{Если Верховный Разум Творца заблагорассудил произвести
человеческий род от обезьяны, то, несомненно, вера в человеке развилась
постепенно, в течение веков, из грубых чувственных представлений, взятых им из
окружающей природы} (с.137).

\enquote{Не один, однако же, плач и скрежет зубов приводит нас к утешительному идеалу
Богочеловека; и радость, в двух ее видах, увлекает нас невольно к этому же
самому идеалу. Когда на душе тишь да гладь, да Божья благодать, или когда душа
восторженна и торжествует, она всегда находит в этих двух видах радости причину
сближения с другим, и непременно высшим, как будто ей сочувствующим существом,
началом, – не знаю с чем-то} (с.140).

\enquote{Вот и я, грешный, хотя и поздно, но убедился, наконец, что мне при складе и
емкости моего ума не следовало попадать в колеи крепких духом и односторонних
специалистов. Жизнь-матушка привела, наконец, к тихому пристанищу. Я сделался,
но не вдруг, как многие неофиты, и не без борьбы, верующим. К сожалению, однако
же, еще и до сих пор, на старости, ум разъедает по временам оплоты веры. Но я
благодарю Бога за то, что, по крайней мере, успел понять себя и увидал, что мой
ум может ужиться с искреннею верою. И, я, исповедуя себя весьма часто, не могу
не верить себе, что искренне верую в учение Христа Спасителя} (с.141).

Очень рекомендую прочесть фрагменты удивительного дневника \enquote{чудесного доктора}.

И, конечно, сам рассказ Куприна! Ведь Рождество близко. Очень важно
подготовиться к этому Событию не только внешне, но и внутренне...

\ii{19_12_2021.fb.baumejster_andrej.kiev.filosof.1.chudesnyj_doktor_kuprin_rozhdestvo.cmt}
