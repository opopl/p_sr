% vim: keymap=russian-jcukenwin
%%beginhead 
 
%%file 17_12_2020.news.ua.strana.venk_victoria.1.kernes_reactions_nazis
%%parent 17_12_2020
 
%%url https://strana.ua/news/307228-smert-kernesa-reaktsija-ofisa-prezidenta-i-khejterov.html
 
%%author Венк, Виктория
%%author_id venk_victoria
%%author_url 
 
%%tags kernes_gennadii
%%title "За разные взгляды готовы убивать". Как националисты и Офис президента отреагировали на смерть Кернеса
 
%%endhead 
 
\subsection{\enquote{За разные взгляды готовы убивать}. Как националисты и Офис президента отреагировали на смерть Кернеса}
\label{sec:17_12_2020.news.ua.strana.venk_victoria.1.kernes_reactions_nazis}
\Purl{https://strana.ua/news/307228-smert-kernesa-reaktsija-ofisa-prezidenta-i-khejterov.html}
\ifcmt
	author_begin
   author_id venk_victoria
	author_end
\fi

\ifcmt
  pic https://strana.ua/img/article/3072/smert-kernesa-reaktsija-28_main.jpeg
  caption Геннадий Кернес умер после продолжительной болезни. Фото ТАСС 
  width 0.5
  fig_env wrapfigure
\fi

Сегодня умер мэр Харькова Геннадий
Кернес.\Furl{https://strana.ua/news/307133-umer-mer-kharkova-hennadij-kernes-307133.html}
Очень многие отреагировали на смерть этого яркого политика и безусловно
популярного градоначальника, которого в октябре переизбрали уже на третий срок. 

Реакции были разные - от соболезнований и горя до откровенного злорадства и,
как с недавних пор стало принято в Украине, глумления. 

Самое странное, что в сторону последних качнулась оценка Офиса президента,
который позже всех опубликовал свой некролог Кернесу. У Зеленского назвали
только что умершего мэра "неоднозначным". 

Разобрались, кто не удержался от критики Геннадия Кернеса в день его смерти.

\subsubsection{Как соболезновали в ОП}

Через несколько часов после того, как стало известно о смерти Геннадия Кернеса,
Офис президента обнародовал свои, так сказать, соболезнования.

"Отношение к Геннадию Кернесу в нашем обществе не было, да и не могло быть
однозначным. Однако у него было видение развития своего города, и именно как
городской голова он работал, вкладывая в Харьков все свои эмоции и компетенции.
Люди это ценили.

Если следующий мэр Харькова сможет обеспечить качество управления городским
хозяйством и такое же социально ориентированное содержание городской политики,
это будет минимум из того, что необходимо для памяти о мэре Геннадии Кернесе.

Сочувствуем близким, семье и харьковчанам", - написали в Офисе президента.

\ifcmt
  pic https://strana.ua/img/forall/u/0/34/%D0%A1%D0%BD%D0%B8%D0%BC%D0%BE%D0%BA(461).JPG
  width 0.4
  fig_env wrapfigure
\fi

Пост Офиса президента сам стал весьма неоднозначным. Заявление прозвучало
крайне неэтично - вряд ли именно такие слова хотели бы услышать родные и
близкие Геннадия Кернеса, упомянутые в обращении ОП. 

Видимо, у Зеленского не смогли подняться над политическими антипатиями - и
удержаться от личного раздражения в адрес мэра, который помешал "слугам народа"
укрепить свою власть в Харькове. 

При этом в Офисе видимо не отдавали себе отчет, что своим заявлением еще больше
отталкивают от себя харьковчан, которые избрали Геннадия Адольфовича мэром в
первом туре. 

Для сравнения, спикер Рады Дмитрий Разумков высказался куда адекватнее.

\ifcmt
  pic https://strana.ua/img/forall/u/0/92/%D1%80%D0%B0%D0%B7%D1%83%D0%BC%D0%BA%D0%BE%D0%B2(3).jpg
  width 0.4
  %fig_env wrapfigure
\fi

\subsubsection{\enquote{Неоднозначный офис высказался}}

Заявление Банковой активно комментируют в соцсетях.

"Неоднозначный офис президента высказался о смерти однозначно настоящего
хозяйственника и политика Геннадия Кернеса. Светлая память человеку,
влюбленному в свой город", - написала ведущая блога "Ясно.Понятно" Олеся
Медведева.

\ifcmt
pic https://strana.ua/img/forall/u/0/34/%D0%A1%D0%BD%D0%B8%D0%BC%D0%BE%D0%BA(458).JPG
\fi

"Подоляк, с сайта Офиса президента, зачем-то написал, что отношение к Кернесу в
обществе было неоднозначным.

Немного странная эпитафия в принципе, еще более странная, учитывая две подряд
сокрушительные победы Кернеса в первом туре на Харьковского мэра, совсем
странная, учитывая, что он в принципе никак не пытался кому-то понравиться за
пределами Харькова.

И уж совсем дикая, учитывая неоднозначное отношение к самому Подоляку, еще в
марте бывшему резвым порохоботом", - отреагировал нардеп от "Слуги народа"
Максим Бужанский.

\ifcmt
  pic https://strana.ua/img/forall/u/0/34/%D0%A1%D0%BD%D0%B8%D0%BC%D0%BE%D0%BA(462).JPG
  width 0.3
  %fig_env wrapfigure
\fi

"Лучше бы вы сегодня не просыпались. И по сути: практически однозначное (73\%)
было к вам, еще парочка таких зашкваров, и оно станет абсолютно однозначным от
нуля к бесконечному минусу. Убедительная просьба - уже не стоит
останавливаться, катитесь быстрее" - написала журналист Елена Скидан.

\ifcmt
  pic https://strana.ua/img/forall/u/0/34/%D0%A1%D0%BD%D0%B8%D0%BC%D0%BE%D0%BA(457).JPG
  width 0.3
  %fig_env wrapfigure
\fi

"Прочитал на президентском сайте соболезнование в связи со смертью Геннадия
Кернеса.

Надо же быть настолько трусливыми, чтобы демонстрировать это даже в
искусственно выдавленном из себя соболезновании.

Стыдно смотреть на этот бисер перед свиньями и коленно-локтевое преклонение
перед бродячими активистами в социальных сетях.

Светлая память Геннадию", - написал юрист Андрей Портнов.

\ifcmt
pic https://strana.ua/img/forall/u/0/34/%D0%A1%D0%BD%D0%B8%D0%BC%D0%BE%D0%BA(466).JPG
\fi

\subsubsection{Как радовались смерти Кернеса}

Своим заявлением Офис президента пытался провернуть трюк "ни вашим ни нашим".
Но получилось, что он сыграл скорее в пользу тех, кто ненавидит Кернеса. 

Смерть Геннадия Кернеса вызвала волну злорадства среди националистов. В
комментариях под новостью о кончине харьковского мэра в одном из львовских
пабликов устроили настоящую пляску на костях.

Главная претензия к мэру - его якобы "сепаратизм". Который, видимо, выражался в
том, что Кернес возвращал улицам города советские названия и поддерживал
Антимайдан. 

\ifcmt
pic https://strana.ua/img/forall/u/0/34/%D0%A1%D0%BD%D0%B8%D0%BC%D0%BE%D0%BA(463).JPG
width 0.4
\fi

Защитникам Кернеса и тем, кто ставит грустные реакции под новостью о его
смерти, угрожают физической расправой и желают "градов" на их головы.

Некоторые откровенно радовались. "Увидел приятные новости о Кернесе. Поздравляю
всех харьковчан и украинцев", - написал Данило Яковлев.

\ifcmt
tab_begin cols=3
pic https://strana.ua/img/forall/u/0/34/%D0%A1%D0%BD%D0%B8%D0%BC%D0%BE%D0%BA(464).JPG
pic https://strana.ua/img/forall/u/0/92/%D0%BB%D1%83%D1%86%D0%B5%D0%BD%D0%BA%D0%BE_%D0%B8.png
pic https://strana.ua/img/forall/u/0/34/%D0%A1%D0%BD%D0%B8%D0%BC%D0%BE%D0%BA_%D1%8D%D0%BA%D1%80%D0%B0%D0%BD%D0%B0_2020-12-17_%D0%B2_14.02__.46__.png
tab_end
\fi

Кое-кто из оппонентов Кернеса выражал сожаление, что его не успели посадить в
тюрьму. Экс-нардеп Игорь Луценко винит в этом "предателей" из Генпрокуратуры
времен Порошенко. 

В ответ на волну хейта другие пользователи отметили, что им страшно, когда за
разные политические взгляды готовы убивать.

\obeycr
"Знаете, что страшно?
Страшно, когда люди не имеют никаких границ.
Страшно, когда за разные политические взгляды готовы убивать.
Страшно, когда такие среди нас.
У меня с этим существом 82 общих друга в фб.
К Кернесу можно относится по-разному, но никто не имеет права выражаться так об усопших" - написал руководитель молодежного крыла ОПЗЖ Артем Марчевский.
\restorecr

\ifcmt
pic https://strana.ua/img/forall/u/0/34/%D0%A1%D0%BD%D0%B8%D0%BC%D0%BE%D0%BA(459).JPG
width 0.3
\fi
