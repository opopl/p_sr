% vim: keymap=russian-jcukenwin
%%beginhead 
 
%%file 29_01_2022.fb.molchanov_jurij.1.strana_neudobnyh_voprosov
%%parent 29_01_2022
 
%%url https://www.facebook.com/george.molchanov.9/posts/4637288066384125
 
%%author_id molchanov_jurij
%%date 
 
%%tags infvojna,obschestvo,propaganda,strana,ukraina
%%title Страна неудобных вопросов, дискомфортных мыслей и многозначительных пауз
 
%%endhead 
 
\subsection{Страна неудобных вопросов, дискомфортных мыслей и многозначительных пауз}
\label{sec:29_01_2022.fb.molchanov_jurij.1.strana_neudobnyh_voprosov}
 
\Purl{https://www.facebook.com/george.molchanov.9/posts/4637288066384125}
\ifcmt
 author_begin
   author_id molchanov_jurij
 author_end
\fi

Страна неудобных вопросов, дискомфортных мыслей и многозначительных пауз.

Как и полвека назад, когда «на кухне одно, на партсобрании другое», сейчас на
камеры и в соц.сетях одно, а за столом, многозначительно вздыхая, иногда
другое, а порой вообще ничего.

Пустота непонимания, усталость незнания и уныние бессилия в  реальном мире,
компенсируется выплеском дичайшей энергии неудовлетворенности в виртуале.

Кухарки, с аватарками вместо мозгов, до исступления сотрясают фэйсбуки
рецептами решения глобальных геополитических проблем. Ведь они ж после
просмотра очередного ток-шоу теперь точно лучше всех знают, кто, где, когда и
на кого нападет. На чьих танках прокатится их, единственно верная на земле,
справедливость. Эти аватарки у подъезда тебе быстро расскажут, что в голове у
Путина, у Байдена, что не так в минских соглашениях и чем закончится сериал
«Крепостная».

Они крепко присели на информационную иглу и ежечасно хватают в руки телефоны,
чтобы утолить жажду очередного, леденящего кровь прогноза: нападет – не
нападет?

Ушлые торгаши постправдой оказались молодцами. Развернули невероятно успешную
всемирную сеть информационного фэстфуда. От неполживых телеканалов и гэошек, до
пресс-служб посольств и МИДов. Уже совершенно не заботясь об обертках и наличии
канцерогенов в их «медиапродуктах», отладив международную логистику и
коммерческие связи, «институты смыслов» бомбят мозги простого обывателя, доводя
его буквально до изнеможения.

Когда-то модная в шоу-бизнесе максима: «лох должен цепенеть», сегодня ловко
перекочевала со сцены в корневую систему общества. После восьми лет жизни в
постоянном стрессе, психика начинает воспринимать его, как естественную среду
обитания и постепенно начинает требовать его, как кислород, как основной допинг
для гормональной системы.

Бомбоубежища, тревожные чемоданчики, сверток документов, кубышка с остатками
роскоши. Гречка, лапша, аптечка… Почему это не удивительно и не странно?

Потому, что это естественно. 

Для людей, которые искренне считают, что цветные революции в Северной Африке и
на Ближнем Востоке родились сами по себе, исключительно по волшебной воле
народа в борьбе с тиранией. Все майданы постсоветского пространства
исключительно волшебно самоорганизовывались. А фпг и внешние игроки здесь
совершенно ни при чем. Для людей, которые считают развал СССР исключительно
заслугой злой Америки и мирового империализма, а внутренних причин для этого в
светлом царстве КПСС и «социальной справедливости» просто быть не могло. И т.д.
и т.п.

Бинарность и примитивность мышления. Насаждаемые по обе стороны границы.
Горсткой образованных мерзавцев и армией безмозглых кнопкодавов. В СМИ и в
корридорах власти.

С одной стороны границы, политическая труха, из последних сил надрывающая
глотки на ток-шоу и в кабинетах, все еще думает, что закручивание гаек по
правилам полувековой давности может дать эффект и это восстановит некую
имперскую справедливость, утраченную 30 лет назад.

По другую - безликая молодая случайность, наспех собранная по сусекам
неправительственных организаций, цинично полагающая, что с помощью тренингов в
Трускавце, наглости и поддержке старших иностранных товарищей, удастся и рыбку
съесть и пропетлять как-то. Они пришли на смену такой же пропитой и
позавчерашней трухе, как и у соседей, только с другой полярностью, и стали
прямым следствием нарушения алгоритмов эволюции. 

А посерединке уставший до остервенения народ. По обе стороны границы.
Накачиваемый пропагандой 24/7 и давно помещенный в ирреальность, конструируемую
своими политическими импотентами.

Джармушевский сюрреализм: «Христос, которого я чту, враждебен твоему Христу»
стал нашей постреволюционной реальностью. В которой нам упорно пытаются
навязать роль Герострата. В которой нам заколачивают окна, чтобы мы как у
Шварца, don’t look up и даже не думали усомниться в Драконе.

Ведь иначе может случиться так, что, оказавшись с Донбассом вновь в одной
стране, вне оружейного прицела, мы вместе увидим, что Дракон каждой из сторон
голый? 

И пересидел уже все мыслимые сроки в наших головах?
