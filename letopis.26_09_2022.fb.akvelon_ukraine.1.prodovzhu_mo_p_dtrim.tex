%%beginhead 
 
%%file 26_09_2022.fb.akvelon_ukraine.1.prodovzhu_mo_p_dtrim
%%parent 26_09_2022
 
%%url https://www.facebook.com/AkvelonUkraine/posts/pfbid0agRKA5hG6Zr7Sad4XoEPLUgGTj6CBzyp9RzvjrTsVPTFzqXFAi76ztLweJNimJ1jl
 
%%author_id akvelon_ukraine
%%date 26_09_2022
 
%%tags volonter
%%title Продовжуємо підтримувати наших бійців у боротьбі за свободу та незалежність
 
%%endhead 

\subsection{Продовжуємо підтримувати наших бійців у боротьбі за свободу та незалежність}
\label{sec:26_09_2022.fb.akvelon_ukraine.1.prodovzhu_mo_p_dtrim}

\Purl{https://www.facebook.com/AkvelonUkraine/posts/pfbid0agRKA5hG6Zr7Sad4XoEPLUgGTj6CBzyp9RzvjrTsVPTFzqXFAi76ztLweJNimJ1jl}
\ifcmt
 author_begin
   author_id akvelon_ukraine
 author_end
\fi

Продовжуємо підтримувати наших бійців у боротьбі за свободу та незалежність:
передали на фронт майже 54 000\$ за серпень!
⠀
- Придбали та підготували для фронту дві чудові автівки, за допомогою яких
групи бойових парамедиків вивозитимуть поранених військових з лінії зіткнення
та доправлятимуть до наступних етапів евакуації, де їх зустрічатимуть
професійні лікарі: ⠀

Mitsubishi L200 2012 року подолала довгий шлях з Британії та відтепер
працюватиме у 92 бригаді, яка вже під російським кордоном;
⠀
Пікап ISUZA 2010 року, що має німецьке походження, завдяки кузову понад 180 см
та додатково встановленим кунгом, виконуватиме свої функції в медичному
батальйоні 518 ОБСпП.
⠀
Ротний медик Гуцул записав аквелонівцям відеопривіт та подякував від медичного
батальйону за допомогу та подаровану автівку, яку ми прозвали Святою Ісузою у
процесі підготовки та перегону на фронт. Сподіваємось, що вона прослужить довго
та врятує безліч життів.  ⠀

- Завдяки компанії, ми успішно закрили запит на технічне забезпечення груп
аеророзвідки та артилеристів. Придбали 8 планшетів для 10 гірсько-штурмової
бригади, а також закрили потреби 81 окремої аеромобільної ДШБ у 5 комп'ютерах,
які виділили з наявної офісної техніки! Це значно покращить взаємодію
підрозділів, бо дозволяє використовувати ПО для коригування, дешифрації
отриманих відеоданих розвідки та ін.

Було дуже приємно та неочікувано отримати подяку до Akvelon від десантників 81
бригади.
⠀
- 2 дрони-розвідники DJI Mavic 3 Fly More Combo доставлено 36 окремій бригаді
морської піхоти та розвідці 92 бригади.
⠀
- Також завдяки зусиллям аквелонівців та коштам компанії передали 2 тепловізори
HikMicro THUNDER Pro TE2536 та рації Motorola 4800 / 4400 - 6 од. / 1 од. для
36 окремої бригади морської піхоти.

- Придбали 6 блоків та 8 панелей - електрогенератори Ecoflow + сонячні панелі,
що забезпечують електроенергію, зарядку приладів зв'язку, дронів. Розподілили
за запитами між бригадами 51 та 93, що працюють на Бахмутському напрямку.  ⠀

- Придбали та передали тепловізор Infiray(Iray) Finder 35R для 127 бату ТРО, що
зараз на Ізюмському напрямку.
⠀
- Закупили та передали до частини А0891, радіотехнічної бригади: AOR
широкосмугові антени, SDR-приймачі, ноутбук, подовжувачі на котушці, роутери,
акумулятори, зарядний блок, 2 дизельних генератори на 5.5 кВт, бінокль Bushnell
Marine Blue 7x50 мм з компасом і далекомірною сіткою, 6 плитоносок та 2
конвекторні пічки.  ⠀

Неймовірно приємно було отримувати подяки від військових частин на ім'я
компанії та наших директорів, завдяки яким ця допомога стає можливою!
